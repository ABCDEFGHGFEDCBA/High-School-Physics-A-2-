\chapter{固体和液体的性质}


\section{教学要求}


这一章是介绍性的。关于固体,介绍晶体及其微观结构;关于液体,主要介绍液体的表面现象。这部分内容可以把它看成是第一章“分子运动论基础”的具体应用,本章讲授的知识在生产和生活中有一定的实用意义,对于学习分子物理学的研究方法、培养观察能力和抽象思维能力也是有好处的。

全章共六节,可以划分为两个单元,第一节和第二节为第一单元,讲述固体的性质,第三节到第六节为第二单元,讲述液体的性质。

关于晶体的各向异性,教材只就导热性进行了说明,教材中虽然提到了晶体的力学性质、电学性质、光学性质也是各向异性的,但不要求具体讲解,只要学生知道就可以了。

关于空间点阵,只要求学生知道组成晶体的物质微粒在空间中依照一定的规则排成整齐的行列就可以了,至于用点阵概念来解释晶体的规则外形和各向异性,只要求学生有个大体的了解,不要求做进一步的解释。

为了说明液体表面的收缩趋势,教材提出了表面层的概
念。这里不要求讲解表面层的形成,只要学生知道表面层里的分子比液体内部稀疏即可,教材给出了表面张力系数的概念,是为了说明不同液体表面的收缩趋势并不相同,不要求用表面张力系数进行计算。

为了说明浸润和不浸润,教材提出了附着层的概念,这里也不要求讲解附着层的形成,只要求学生知道附着层里分子比液体内部密或稀是形成浸润或不浸润的原因就可以了。至于为什么固体分子的吸引比较弱(或强),附着层里分子就比较稀(或密),则不要求加以解释。

毛细现象的教学,要求学生能够利用浸润现象和表面张力的知识对毛细现象的产生做出解释,但不要求对液柱上升或下降的高度进行定量的计算。

这一章虽然是介绍性的,但在教学中也不能忽视,讲述这一章,可以扩展知识面,使学生多知道一些物理现象,有利于学好物理,通过这一章的学习,要注意使学生体会到,分子运动论不但能从微观上研究气体,而且能够研究固体和液体;人们的研究从宏观领域进入微观领域,使人类对自然界的认识前进了一大步。

这一章的教学要求是:
\begin{enumerate}
\item 知道固体可分为晶体和非晶体两类,知道晶体有单晶体和多晶体;了解晶体和非晶体在外形上和物理性质上的区别。
\item 知道什么叫空间点阵,了解晶体外形的规则性和各向异性可用晶体物质微粒的规则排列来说明。
\item 了解液体的微观结构情况.
\item 了解液体的表面张力,知道它是怎样产生的;了解浸
润和不浸润,知道它们是怎样产生的;了解毛细现象,知道它是怎样产生的。
\end{enumerate}


\section{教学建议}
\subsection{第一单元}
\subsubsection{晶体的特征}

教材对这部分知识要求较低,只简要讲
述晶体的特征,晶体与非晶体的区别,以及固体的微观结构,而不涉及固体的其他性质,要注意把握教学的深广度,不宜讲得太多、太深。

\subsubsection{固体材料}
 为了扩展学生的眼界,引起他们对研究固体性质的兴趣和求知欲望,在本单元教学开始,可以简述一下有关固体材料和固体物理的发展概况。

 长期以来,在科研、生产和日常生活中都广泛利用固体材料,因此,固体在材料科学技术中占有特殊重要的地位,按指定的性能设计新的固体材料已成为固体物理的重要研究内容,近30年来,固体物理已发展成为一门独立的综合性学科,是物理学的重要分支,由于各种尖端技术对固体材料提出多种多样的要求,因此固体物理同现代尖端技术的发展有非常密切的联系,例如,原子能技术需要耐放射性辐射的固体材料:高速飞行、火箭导弹需要耐高温、耐辐射、强度高、质地轻的合金材料;电子技术需要半导体元件、集成电路、铁氧体元件等新型器件;激光技术需要小巧价廉的半导体激光器等等,在
 科研和生产需要的推动下,新理论和新技术相互促进、相辅相成,使固体物理在近代原子理论的基础上得到巨大的发展。

 \subsubsection{利用实验、挂图、模型讲述晶体、非晶体}
 
 本单元知识
 比较抽象,教学中要加强实验演示,运用挂图、模型,使学生获得鲜明、具体的印象,有条件的学校可以把演示实验让学生自己动手进行实验观察,教学效果会更好些,例如,在讲晶体和非晶体时,可在教师指导下一边让学生分组用放大镜观察食盐、砂糖晶体的外形,按教材介绍的方法比较云母晶体和非晶体玻璃的导热性;一边阅读教材,由学生自己概括晶体和非晶体在外形上和物理性质上的区别,最后,教师作适当补充、归纳,在讲空间点阵一节内容时,最好先让学生观察点阵模型,再对照模型和教材图4-5讲明空间点阵的概念;解释晶体外形的规则性和物理性质各向异性;讲明同一物质的微粒能够形成不同的空间点阵,但应当提醒学生:点阵模型并不代表晶体的真实情况,它只是组成晶体的物质微粒有规则排列的示意图。

\subsubsection{教学中要注意的问题}
\begin{enumerate}
\item 教材中指出晶体具有天然的规则外形,教学时应该强调,这种规则的外形不是人工造成的,是晶体本身具有的.
\item 讲晶体的各向异性时要指出:天然的规则的几何形状和各向异性都是晶体区别于非晶体的性质,但晶体有时不完整、它的外形也容易受到破坏使外部特征不显著,所以各向异性就成为判断晶体和非晶体的一个基本特征.
\item 根据初中学过的知识,可以告诉学生,晶体的另一个基本特征是具有一定的熔点,使他们对晶体的特征有较全面的认识,同时指出,这是从宏观上区分晶体和非晶体的重要依据。当物体已被研成粉末,不能从外形及各向异性来鉴别它是否晶体时,只有根据有无一定熔点才能作出准确判断.
\item 多晶体虽无规则的几何形状,物理性质又各向同性,但是组成多晶体的晶粒却有规则的几何形状,物理性质呈各向异性。这是多晶体和非晶体在内部结构上的区别。多晶体与非晶体的区别还在于多晶体跟单晶体一样具有一定熔点,非晶体则没有.
\item 可按乙种本上册255页的叙述,介绍多形性的概念。有些物质能够生成几种不同的晶体,就是因为它能够生成几种形状不同的空间点阵,这种性质叫多形性。换句话讲,同一化学成分的物质可以结晶成结构不同的晶体的性质叫多形性,还可以简要介绍同一种化学成分的物质,既能以晶体形式存在,又能以非晶体形式存在,例如,天然水晶是晶体,而熔融的水晶(即石英玻璃)就是非晶体,以便扩大学生眼界,避免出现片面的、绝对化的错误认识。
\end{enumerate}



\subsection{第二单元}
这一单元讲述液体的性质。主要是从分子运动论的观点来剖析液体的微观结构,研究液体和气体接触时形成的表面层以及液体和固体接触时形成的附着层发生的现象,然后讨论在表面层与附着层的共同作用下产生的毛细现象。教学中应突出教材的这一思路,使学生能对这部分知识从整体上有所认识,本单元的特点是偏重实验现象的分析,因此,需要加强演示实验,注意培养学生的观察能力和抽象概括能力。

\subsubsection{液体的微观结构}

建议在教学中注意以下问题.

重视新旧知识的联系,先引导学生回顾第一章学过的分子运动论的基本观点和对固、液、气体的分子特点的分析,这对学习新知识是十分有益的。

液体的性质介于固体和气体之间,但更接近固体.因此,应对比固体结构讲解液体微观结构的大致图景,并注意用液体微观结构的特点去认识液体的性质:具有一定体积,不易压缩,各向同性,流动性等。

课本76页上提到“液体中的分子也是密集在一起的”,“液体分子距离很小”。为使学生能比较具体认识这个问题,可举如下事实:一摩尔水有$6.022\x10^{23}$个水分子,在常温下的体积为$18{\rm cm^3}$,每立方厘米中有分子
$\dfrac{6.022\x10^{23}}{18}=3.3\x10^{22}$
个。设两相邻水分子间的距离为$r'$, 可得
\[r'=\sqrt[3]{\frac{1}{3.3\x10^{22}}}=3.1\x 10^{-8}{\rm cm}\]
和分子直径同数量级,所以说液体分子几乎一个挨一个地密集在一起,彼此间距离很小。

要指出非晶体的微观结构跟液体非常类似(见参考资料),非晶体的分子也是处于杂乱无章的结构状态,跟液体中小区域杂乱无章的结构状态非常类同,所以非晶体可以看作是粘滞性极大的液体。可以提示学生,严格说来,只有晶体才是真正的固体。

液晶在近10多年来获得了广泛的应用,因此教材把它安排为阅读材料,如教学时间比较充裕,最好先在课堂上简要介绍一下,再要求学生阅读,激发他们学习的兴趣。

\subsubsection{液体的表面现象}

关于液体表面的收缩趋势,教材是通过实验现象的观察得出结论的,因此,必须做好演示实验(见实验指导),并注意突出每个实验在认识液体表面性质中的作用,回忆荷叶上的小水滴,草叶上的露珠及演示玻璃板上的小水银滴,说明液体与空气接触的表面有收缩的趋势;教材图4.8的实验,一方面说明不仅液体与空气接触的表面有收缩的趋势,而且液体与液体接触的表面也存在收缩的趋势,另一方面对照玻璃板上的大液滴呈扁平形,说明消除重力后这个表面有收缩到最小面积的趋势;教材图4.9和图4.10的实验则是对液面具有收缩趋势的验证.

对教材图4.10所示的实验,要从棉线圈的周长一定所围面积最大的几何图形是圆来说明只有棉线圈被张紧成圆形时肥皂膜面积才会最小。

关于表面层,根据教材的要求,可不讲解它是怎样形成的,只要求学生知道表面层里分子比液体内部稀疏就可以了。这个结论是讲解表面张力产生原因的依据,让学生明确认识表面层分子聚集的特点是进行表面张力教学的关键。

从复习分子间相互作用力的特点,说明表面层内分子间相互作用表现为引力,即表面张力,是不太困难的。在教学中需要注意的是:
\begin{enumerate}
    \item 由于张力概念比较难懂,学生过去又没有学过,在这里只要求学生能接受教材上关于表面张力的讲法即可,不必单独讲解“张力”的概念,以免增加学生负担.
    \item 表面张力不是指个别分子间的相互引力,而是表面层中大量分子间引力的宏观表现,凡液体与气体接触的表面都存在表面张力,例如对一层液膜来说,无论其厚薄程度如何都存在两个表面,每个表面都存在表面张力.
    \item 应当提醒学生,教材图4.11中液面的分界线MN是任意选取的,但不论分界线怎样选,分别作用在分界线两侧液面的表面张力f和f总是一对作用力、反作用力。
\end{enumerate}

至于表面张力的方向,即表面张力跟液面相切、跟液面分界线垂直,只要求学生知道,不要求作进一步解释,但对液面是曲面的情况,可作出表面张力的示意图帮助学生理解。

教材不要求用表面张力系数进行定量计算,不过应当指出,两部分液面间的表面张力大小不仅与液体的表面张力系数有关,还与分界线长度有关,其数值等于表面张力系数与分界线长度的乘积,避免学生把表面张力和表面张力系数混为一谈。关于各种液体的表面张力系数不同,可以安排一些有趣的演示(或布置为课外实验,见实验指导)来加以说明,但不宜从理论上多加解释。

\subsubsection{浸润和不浸润} 

这部分内容研究液体与固体接触时发生的现象,是与液体的表面现象平行的知识,教材是从附着层内分子作用力的不同特征来解释浸润和不浸润现象的。因此,在教学中要抓好两个环节:做好演示实验;让学生懂得怎样以附着层内分子聚集的特殊情况(即附着层内分子比液体内部稀或密)为依据,解释浸润和不浸润现象。

做好演示实验的关键在于提高器材的洁净程度,实验中所用的水银和水要洁净,玻璃和玻璃管应洗涤洁净,锌板需用稀硫酸清洗并擦拭洁净,为增大可见度,可用幻灯投影。

讲解浸润和不浸润现象时,应对什么是浸润和不浸润现象作适当概括,便于学生掌握,应当强调,同一种液体对某些
固体浸润,对另一些固体则不浸润,因此讲某种液体是浸润的或不浸润的液体,一定要指明相应的固体,对此可演示水能浸润玻璃但不能浸润石蜡,水银不能浸润玻璃但能浸润锌的实验,加深学生的认识。

讲解附着层的性质时,可与液体表面层的性质进行对比,要引导学生掌握分析问题的思路。附着层的液体分子除受液体内部分子的吸引外,还受到固体分子的吸引。固体分子吸引作用的强弱决定着附着层内液体分子比液体内部分子稀疏或稠密,也就决定着附着层内分子间表现为引力或斥力,进而决定着附着层是收缩趋势还是扩展趋势。附着层有收缩趋势,就表现为液体不浸润固体,附着层有扩展趋势,就表现为液体浸润固体。

有了前面的基础,讲解弯月面的形成就水到渠成,所以建议把弯月面放在最后讲述。

\subsubsection{毛细现象}

毛细现象的发生是附着层的收缩力或相斥力与表面张力共同作用的结果,由于学生已学过上述两方面知识,建议通过指导学生讨论和阅读的方式进行教学,这样做,有利于培养学生的能力。

首先演示教材图4.14和图4.15的两个实验,可用幻灯投影增大实验的能见度,让学生观察并思考从中能得出什么结论?再由教师概括说明什么是毛细现象,强调毛细管内径越小,毛细现象越明显,即浸润液体在管内上升高度越大,不浸润液体在管内下降越厉害。

为什么会出现上述现象呢?可引导学生思考、讨论以下问题.
\begin{enumerate}
    \item 浸润液体在毛细管内,液面为何成凹弯月面?是什么原因使液体上升?上升到什么时候为止?
\item 为什么不浸润液体在毛细管内会下降?下降到什么时候为止?
\end{enumerate}
然后让学生阅读教材,检验自己的认识。教师再根据具体情况适当归纳,着重分析第二个问题。

教材列举了不少毛细现象的实际应用,但没有作详细的说明,可选择其中一些例子让学生讨论,巩固所学知识。例如,
\begin{enumerate}
    \item 为什么棉灯芯能吸油?是否可用丝线做灯芯?    \item 为什么可以用粉笔来吸干纸上的墨水迹?
\end{enumerate}

本单元教材的题目较少,可以联系实际补充一些解释现象的题目,例如:
\begin{enumerate}
\item 人造卫星中有一个盛液体的容器,如果液体浸润器壁,会发生什么现象:如果液体不浸润器壁,又将出现什么现象?(浸润液体沿器壁上升并沿器壁流散;不浸润液体则呈球形。)
\item 教材练习二第1题的缝衣针如果事先用肥皂水洗得很干净并用清水冲洗擦干,将会出现什么现象?为什么?(针下沉,水能浸润缝衣针,针周围的水分子与针上水分子连成一片,使针处在水面以下。)
\item 为什么带有油脂的抹布不能把湿了的桌面擦干t(水对带有油脂的抹布不浸润,抹布就不能带走桌面上的水。)
\item 车轮在潮湿的土地上滚过以后,车辙中就会渗出水来,这是什么缘故?(车轮压紧地面不仅使土壤里原有的毛细管变得更细,而且还会增添新的毛细管,使毛细现象更加显著,地下水更容易上升到地面上来。)
\end{enumerate}

\section{实验指导}
\subsection{演示实验}
\subsubsection{晶体外形的演示}

演示晶体具有天然规则的几何形状,最好事先培养一些体积大点的晶体,便于观察。其中硫酸铜、重铬酸钾、明矾的大晶体比较容易获得,现以硫酸铜为例,介绍制作方法。

在一个大烧杯中盛大半杯热水,将研成粉末的硫酸铜逐浙倒入水中搅拌,直到饱和,再用细线拴一小粒硫酸铜晶体,悬挂在硫酸铜溶液中作为结晶核,让晶体缓慢生长,几天后可获得一块较大的硫酸铜晶体,若溶液过饱和、室温太高、结晶生长太快,得到晶体的质量就不够好,此方法也适用于培养重铬酸钾和明矾大晶体。

也可用厚一点的纸糊一个一端封底的圆筒,圆筒比小电珠略大、略长一点,在筒外包两层纱布。用线把圆筒开口的一端拴住,悬挂起来,使筒的4/5左右浸入硫酸铜溶液中作为结晶核,从而得到一中空的硫酸铜晶体,然后在晶体内部安上一个小电珠,演示时使小电珠发光,从里面照亮硫酸铜晶体,效果会更加明显。

\subsubsection{晶体传热的各向异性}

比较云母片和玻璃板的传热性,说明晶体导热各向异性而非晶体导热各向同性。
\begin{enumerate}
    \item 云母片要薄,最好撕裂成单层薄片使用.
    \item 用四氯化碳溶液将石蜡溶解,用脱脂棉将溶液均匀涂在云母片上,放在阳光下晒干备用,为使石蜡层成为均匀薄层,也可先在云母片上放一小块石蜡,把云母片放在火焰上方烘一下,使石蜡熔成膜状薄层。
    \item 取一根长约40—50毫米的铁钉,用尖嘴钳夹住放在酒精灯上将钉的尖端烧红,然后用尖端去接触云母片涂蜡层的反面,石蜡熔成椭圆形,应当注意,根据云母晶体的特点,该椭圆的离心率不会太大,要让学生留意观察。
\end{enumerate}

用玻璃板作同样实验,所用玻璃板也要薄。还可以用小号培养皿的底部代替玻璃板。

\subsubsection{液体表面的收缩趋势}

橄榄油呈球形的实验.先将橄榄油滴在酒精里,这时油滴沉底。然后在酒精中缓缓注入清水,直到看见油成球形悬浮在酒精与水的混合液中为止。

也可用苯胺代替橄榄油,滴在食盐水中,逐渐调整食盐水浓度,使苯胺滴成球形悬浮在食盐水中。

还可把内盛有色水的玻璃管的尖嘴浸入重机油中,注意控制上部皮管的开关,使尖嘴处缓缓地形成带色的表面呈球形的小水滴,如图4.1所示.
\begin{figure}[htp]
    \centering
%\includegraphics{fig/4-1.png}
    \caption{}
\end{figure}

液膜收缩实验.通常用肥皂水做教材上图4.9和图4.10所示的实验,虽然肥皂水的表面张力系数($40\x10^{-3}{\rm N/s}$)比纯水的表面张力系数($73\x10^{-3}{\rm N/s}$)小得多,但肥皂水的粘滞性强,在重力作用下比纯水流下较缓,所以较容易在表面层之间保持液膜,为了使实验效果较好,肥皂水浓度要
合适,并可滴入几滴甘油增强其粘滞性,铁丝环要平整、光洁,棉线宜细软,事先应将棉线浸湿。

为便于学生看清液膜收缩的过程,可将金属丝做成图4.2所示的U形框架,两侧框杆长约3厘米,相距约4厘米,框架尖端可弯成小钩,将细棉线的一端系在一个小钩上,手执细线的自由端,在另一小钩附近绕一周后拉紧,把框架浸没在肥皂水内,轻轻提起,使框架上形成矩形液膜,并保持液膜处于竖直面内,且线的固定端在上方。然后,放松被拉紧的自由端,可明显看到液膜向U型框架接手柄的一边收缩,把棉线拉成弧形。

\begin{figure}[htp]
    \centering
%\includegraphics{fig/4-2.png}
%\includegraphics{fig/4-3.png}
    \caption{}
\end{figure}

\subsubsection{验证表面张力的存在}

用金属丝做成一金属环,把它悬挂在倔强系数很小的纤细弹簧下方,记下弹簧指针的示数$F_1$, 把金属环浸没在肥皂水中,再把它缓缓拉出,使环四周与液面间出现液膜,如图4.3所示,记下这时弹簧指针的示数$F_2$. 比较这两个示数可以看出$F_2>F_1$, 证明液膜上存在表面张力.

\subsubsection{毛细现象}

在缺少口径不一的毛细管时,也可用以下器材代替,将两块平板玻璃一边靠拢,另一边用木棍隔开,再用橡皮筋把两边箍牢。放入有颜色的水中后,可看到在玻璃板夹缝间的水升起,缝越窄的地方,水升得越高。调节木棍在玻璃板间的位置,减小板的间距,板间各处的水会上升得更高(图4.4)。

\begin{figure}[htp]
    \centering
%\includegraphics{fig/4-4.png}
    \caption{}
\end{figure}

\subsection{课外实验活动}
\subsubsection{观察各种液体的表面张力系数不同}

在桌上铺一张白纸,在纸上再放一块面积稍大一些的干净平板玻璃,在玻璃板上倒少许红色水形成一层薄薄的水层,再用一团脱脂棉花球沾少量酒精,用手指捏一下棉球,滴几滴酒精到玻璃板上,就可发现玻璃板上的红色水带着酒精向四周移动,出现一块“干”的区域,这是由于滴酒精后,在水和酒精相邻的分界线上有表面张力相互作用,水的表面张力系数比酒精大得多,所以红色水就把酒精拉向四周,留下一块“干”的区域。

让两根火柴相隔一段距离平行地浮在水面上,待稳定后,在火柴间轻轻滴入一两滴肥皂水(或用一小块肥皂轻轻接触火柴之间的水面),便可看到两根火柴立即分别向两边跑开。如果在两火柴之间不滴入肥皂水而改滴入一两滴糖水,则两根火柴立即靠拢,这是因为肥皂水或糖水的表面张力系数分别小于或大于水的表面张力系数。

在水面上放几小块樟脑,就会发现这些樟脑块在水面上进行复杂的、紊乱的运动。这是由于樟脑的形状是不规则的,它们在各方面溶解的程度不同,导致樟脑四周的樟脑水溶液浓度不同,表面张力系数也不相同,结果就引起樟脑块作这种奇怪的运动。

\subsection{练习一}
\begin{enumerate}
    \item 把玻璃管的裂断口放在火焰上烧熔,它的尖端就变圆.这是什么缘故?

\begin{solution}
    玻璃烧熔后,它的表面层在表面张力作用下收缩到最小表面积,从而使玻璃管尖端变圆。    
\end{solution}
\item 在处于失重状态的宇宙飞船中,一大滴水银会呈什么形状?

\begin{solution}
    在处于失重状态的宇宙飞船中,由于消除了重力的影响,一大滴水银的表面将收缩到最小面积——球面,水银滴成为球形。    
\end{solution}
\item 把熔化的铅一滴一滴地滴入水中,凝固后可以得到
球形的小铅弹.为什么?

\begin{solution}
    熔化的铅滴在自由下落中,由于失重,液滴表面将在表面张力作用下收缩为球形,进入水后迅速冷却,就会凝固成球形小铅弹。
\end{solution}
\end{enumerate}

\subsection{练习二}

\begin{enumerate}
   \item 把一根缝衣针小心地放在水面上,针可以把水面压弯而不沉没(试试看).解释这个现象.

   \begin{solution}
 由于针的表面有油脂,不能被水浸润,当针放在水面上把水面压弯时,仍处在水的表面层之上,这是因为水面的表面张力,要使被压弯的水面收缩,使它恢复原来的水平液面,从而对针产生一个向上托的力,这个力与水对针的向上的压力一起跟针所受的重力平衡,使针不致下沉。  
   \end{solution}
   \item 布的雨伞虽然纱线间有可以看得出来的孔隙,却仍然不漏雨水.解释这个现象.

   \begin{solution}
    因为水能浸润纱线,在纱线孔隙中形成向下弯曲的水面,弯曲水面的表面张力,承受住孔隙内水所受的重力,使得雨水不致漏下。
   \end{solution}
\end{enumerate}



\subsection{练习三}
\begin{enumerate}
   \item 要想把凝在衣料上面的蜡或油脂去掉,只要把两层吸墨纸分别放在这部分衣料的上面和下面,然后用熨斗来熨就可以了,为什么这样做可以去掉衣料上的蜡或油脂?

   \begin{solution}
    放在衣料上、下的吸墨纸内有许多细小的孔道起着毛细管的作用。当蜡或油脂受热熔解成液体后,由于毛细现象,它们就会被吸墨纸吸掉。
   \end{solution}
\item 建筑楼房的时候,在砌砖的地基上铺一层油毡防潮层.如果不铺这层油毡,楼房就容易受潮,为什么?

\begin{solution}
    因为土壤和砖块里有许多细小孔道,地下水可以通
    过这些毛细管上升到楼房里,使楼房受潮,铺一层油毡后,由于油毡上涂有煤焦油,堵塞了纸料上的孔隙,不会发生毛细现象,从而阻断了地下水上升到楼房的通道,防止房屋受潮。
\end{solution}
\end{enumerate}

\section{参考资料}
\subsection{固体的微观结构}

晶体和非晶体宏观性质上的不同,是由于它们微观结构上的差别造成的,经过伦琴射线的分析,发现组成晶体的原子(或分子)的排列十分整齐,极有规则,结构是周期性重复的。即在长距离范围内作有秩序排列,称为长程有序,而非晶体内部的原子(或分子)的排列不很整齐或很不整齐,没有明显的规则性,不具有周期性重复的结构,因此不是长程有序的。但在一个原子间距的范围内,原子排列还是有一定规则、一定结构的,这称为短程有序,可见,非晶体是无序结构中存在着有序成分,长程无序而短程有序的固体,正因为这样,所以非晶体在宏观上没有一定天然的规则形状;在较大范围内各方向的原子排列情况平均说来是一样的,非品体在宏观上具有各向同性;另外,由于内部各处原子的结合情况不同,有松有紧,不能在同一温度都达到能挣脱相互束缚的程度,因而非晶体不具有确定的熔点。

从有序性来看,非晶体的固态与液体相似,组成液体的物质微粒在短暂时间内,在一个微小的区域中也可保持有一定规则的排列,也属于短程有序,但是它们也有重要区别,液体的这种短程有序受热运动的影响不断被破坏和改变,而非
晶体中的短程有序性却保持相对的稳定。

\subsection{晶体的类型}

晶体中粒子(分子、原子和离子)之间存在着相互作用力,这种力叫结合力。正是这种力使粒子有规则地聚集在一起形成空间点阵。结合力又称化学键,它是决定晶体基本性质的根本原因。根据化学键的不同,可把晶体分为四类。
\begin{enumerate}
\item 离子晶体,晶体由离子组成,靠正、负离子之间的静电力,即离子键把这些离子结合起来,由离子键的作用组成的晶体,称为离子晶体,由于离子键作用强,因此离子晶体具有高熔点、低挥发性、可压缩性小。食盐晶体就是最典型、最简单的离子晶体,半导体材料中的硫化镉、硫化铝也是重要的离子晶体。
\item 原子晶体.晶体由中性原子组成,靠共有电子产生的结合力,即共价键把这些中性原子结合起来,这种晶体称为原子晶体,共价键的作用很强,所以原子晶体硬度大、熔点高、导电性弱、挥发性低,金刚石就是典型的原子晶体,半导体中的重要材料硅、锗、碲也都是原子晶体。
\item 分子晶体.晶体由分子组成,靠分子间的相互作用产生结合力,即范德瓦尔斯键把这些分子结合起来,由范德瓦尔斯键的作用而组成的晶体称为分子晶体,范德瓦尔斯键的作用很弱,所以分子晶体硬度小、熔点低、易于挥发,碘和低温下的情性气体以及许多有机化合物构成的晶体,都是分子晶体。
\item 金属晶体.晶体内金属正离子排列在点阵的结点上,自由电子为全体离子所具有,自由地在正离子形成的点阵内
运动,自由电子的总体称为电子云。正离子与电子云之间的作用力使各粒子结合在一起,这种结合力称为金属键。由金属键作用所组成的晶体叫金属品体,简称金属,金属键的作用可以很强,因此金属可以具有高熔点、高硬度和低挥发性,金属内由于存在自由电子,因而具有良好的导电性和导热性。
\end{enumerate}

对于大多数晶体来说,晶体的结合往往是几种键共同作用的结果.例如石墨有三种键共同作用,如教材图4.6所示,每一层中每一个碳原子有三个电子以共价键与周围三个碳原子相互作用,另一电子为层中所有碳原子共有,而以金属键与层中所有碳原子相互作用,层与层之间则以范德瓦尔斯键相互作用,因此,共价键、金属键、分子键这三种键在石墨晶体中都起作用,致使石墨的性质与同为碳原子但只由共价键组成的金刚石的性质有很大的不同。

\subsubsection{表面层的收缩趋势}
我们知道,分子间的作用力只有在距离很小时才起作用,这个距离约为$10^{-8}$厘米,由于液体分子在各个方向上都是均匀分布的,可以把分子作用的范围认为是一个半径约为$10^{-8}$厘米的球,这个球叫分子作用球,我们把跟气体交界并且厚度等于分子作用球半径的一薄层液体,叫做液体表面层。
\begin{figure}[htp]
    \centering
%\includegraphics{fig/4-5.png}
    \caption{}
\end{figure}

如图4.5所示,设液体内有一分子$A$, 由于该分子的分子作用球处在液体内部,而球内的分子均匀分布,对这个分子来讲是球对称的,因此球内所有分子对$A$的作用力的合力为
零。设表面层里有一分子$B$, 以$B$为中心的分子作用球一部分在液面外,缺少了这部分分子的引力,就使作用在$B$上的全部分子引力的合力$f$垂直于液面指向液内(因分子间表现为斥力时分子间的距离极小,$B$所受分子斥力仍是球对称的,可不予考虑,只考虑分子间表现为引力的情况即可),如果要把液体分子从内部移到表面层去,就必须克服力$f$的作用做功,使分子的势能增加,故分子在表面层比在液体内具有较大的势能。表面层中全体分子因上述原因所具有的势能的总和,叫做表面能,液体的表面越大,具有较大势能的分子数也越多,表面能也越大。力学原理指出,物体系在稳定平衡时,它的势能必须是一切可能值中的最小值,因此,液体在稳定平衡时,它的表面能应当最小,即液体表面应尽可能收缩,直到表面积最小。

\subsubsection{弯曲液面下的压强}

由于表面张力作用,弯曲液面有一个特征,就是在它的下面产生附加压强。
\begin{figure}[htp]
    \centering
%\includegraphics{fig/4-6.png}
    \caption{}
\end{figure}

我们以液面是半径为$R$的球体的一部分为例,求出附加压强的数值,取球面的一部分$\Delta S$(图4.6),作用在这部分边线上的表面张力处处都是与这球面相切的,设表面张力系数
为$\alpha$, 通过边线上每一微段$\Delta\ell$作用在液块上的表面张力$\Delta f=\alpha\cdot \Delta\ell$. 这个力的一个分力垂直于球面半径$OC$, 整个边线各个微段的表面张力的这个分力对$OC$具有轴对称性,合力为零。$\Delta f$的另一个分力平行于$OC$, 如球面是凸的,则中心$C$在液体之内,力$\Delta f_1$压缩$\Delta S$下面的液体形成一正压力;如球面是凹的,则中心$C$在液体之外,力$\Delta f_1$对液体形成一负压力,由图可见,$\Delta f_1=\Delta f\sin\phi=\alpha\Delta \ell\sin\phi$, 因此平行于半径$OC$施加于整个球面部分$\Delta S$上的力
\[f_1=\sum \Delta f_1=\alpha\sin\phi\cdot \sum \Delta \ell\]
但$\sum \Delta \ell$是包围球面$\Delta S$的边线长度,以$r$表示其半径,则$\sum \Delta \ell=2\pi r$, 由此可得 $f_1=\alpha\cdot 2\pi r\sin\phi$.

又由图可知$\sin\phi=r/R$,则:
\[f_1=\frac{\alpha\cdot 2\pi r^2}{R}\]
所以附加压强
\[p=\frac{f_1}{\pi r^2}=\frac{2\alpha}{R}\]

显然,该附加压强和表面张力系数成正比,和表面的半径$R$成反比。表面弯曲越厉害,其半径$R$越小,因而附加压强$p$就越大。对凸液面,液面内压强大于液面外压强,附加压强$p$向下;对凹液面,液面内压强小于液面外压强,附加压强$p$向上。

对一个球形液膜(如肥皂泡)来说,它的液膜有两个表面,且一凸一凹,由于膜很薄,其半径可认为相等,都是$R$(图4.7). 在液膜内取一点$B$, 用$p_A$、$p_B$、$p_C$分别表示$A$、$B$、$C$三点的压强,故
\[p_B-p_A=\frac{2\alpha}{r},\qquad p_C-p_B=\frac{2\alpha}{r}\]
两式相加得
\[p_C-p_A=\frac{4\alpha}{r}\]
由于这种附加压强的存在,使肥皂泡内的压强比泡外的大气压强大。泡的半径越小,泡内外的压强差越大。如在一玻璃管的两端吹制半径不同的肥皂泡$A$和$B$(图4.8), 让两泡相通。因小泡内的压强大于大泡内的压强,空气就会从小泡不断流向大泡,使小泡不断变小,大泡不断增大。泡内压强的这种关系在吹制玻璃器皿时要用到,开始时吹气的压强要比较大,吹大后就要减小压强。

\begin{figure}[htp]
    \centering
%\includegraphics{fig/4-7.png}
%\includegraphics{fig/4-8.png}
    \caption{}
\end{figure}

\subsubsection{液体在毛细管中上升高度的计算}


如图4.9所示,设毛细管内径为$r$, 液体表面张力系数为$\alpha$, 液体密度为$\rho$, 液柱上升高度为$h$. 由于毛细管内径很小,可以粗略认为沿着管壁的液面是竖直的,那么,竖直向上作用到液柱上的表面张力就是$2\pi r\alpha$, 因竖直向下作用到液柱上的重力为$\pi r^2h\rho g$, 当液体保持平衡时
\[2\pi r\alpha=\pi r^2h\rho g\]
由此可以得到
\[h=\frac{2\alpha}{r\rho g}\]

\begin{figure}[htp]
    \centering
%\includegraphics{fig/4-9.png}
%\includegraphics{fig/4-10.png}
    \caption{}
\end{figure}

所以,在毛细管内浸润液体上升的高度跟表面张力系数成正比,跟毛细管内部的半径和液体的密度成反比。

用类似的方法也可推导出不浸润液体在毛细管中下降的
高度同样是$h=\dfrac{2\alpha}{r\rho g}$.

更为一般的方法是根据弯月面内外压强差来推导,如图4.10所示.当毛细管刚插入液体时,由于液体浸润管壁,所以沿管壁上升,使液面成凹弯月面,这将使液面下方的压强小于液面上方的大气压强,而在管外平液面下与$B$点同高的$C$点的压强仍等于液面上方大气压。根据流体静力学的基本原理,$B$、$C$压强应相等。因此液体不能平衡而要在管内上升,一直升到$B$、$C$两点压强相等为止。

设管截面为圆形,对凹弯月面可近似看作半径为$R$的球面,弯月面下$A$点的压强比大气压$p_0$小$2\alpha/R$, 即
\[p_A=p_0-\frac{2\alpha}{R}\]

因$B$点与$A$点高度差为$h$, 所以
\[p_B=p_A+\rho gh=p_0-\frac{2\alpha}{R}+\rho gh\]
又因$p_B=p_C=p_0$, 所以有
\[p_0-\frac{2\alpha}{R}+\rho gh=p_0\]
则:
\[\frac{2\alpha}{R}=\rho gh,\qquad h=\frac{2\alpha}{\rho gR}\]
由图可知:
\[R=\frac{r}{\cos\theta}\]
其中$\theta$为接触角,即固、液接触处,液体表面的切线与固体表面的切线在液体内部所成的角度。将本式代入上式,可得
\[h=\frac{2\alpha\cos\theta}{\rho gr}\]




