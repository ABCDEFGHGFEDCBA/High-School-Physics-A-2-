\chapter{物态变化}
\section{教学要求}
本章讲解的知识,是对初中学过的物态变化知识的扩展和加深,与生产、科研和日常生活实际有着密切的联系。

教材对熔解、凝固、汽化等现象,和熔解热、汽化热等概念,从分子运动论的观点和能量的观点作了定性的解释,是为了使学生对现象和概念理解得深刻一些;这也有利于培养学生的抽象思维能力。这一章的现象多、概念多,讲述时除了加强实验、增加感性知识外,还要多举生活中的实例,帮助学生理解。还要注意弄清楚某些具有共同特点的概念间的区别(如熔解热与汽化热)。

本章的教学可分为三个单元:第一单元,包括第一、二节,讲熔解和凝固的知识,第二单元,包括第三节到第七节,讲解蒸发、沸腾及饱和汽和气体的液化的知识,第三单元,包括第八、九节,讲解空气的湿度和露点。

下面对这一章的教学内容作一些具体说明。

晶体和非晶体在熔解时的区别初中已学过,只要求作简单的复习,课本对此从分子结构不同所作的解释,以及对熔
解热用分子运动论和能量转化的知识所作的说明,可以使学生从理论上加深对这些现象的认识,同时使他们了解理论的应用,从而巩固理论知识。

饱和汽的知识是理解液体的沸腾、气体的液化、空气的湿度、露点等许多现象的基础,需要重点讲解,它又是学生新接触的理论性较强的内容,也是教学的难点,理解饱和汽的知识,关键在于讲好动态平衡的概念。

液体沸腾的条件和沸点跟压强的关系,课本是通过实验观察得出结论的。至于为什么当饱和汽压等于外界压强时液体才沸腾,这个问题在中学阶段很难讲清楚,教材中也没有进一步分析这个问题,希望在教学中注意掌握,不做过高的要求。

课本对汽化热也用分子运动论和能量转化的知识作了说明,液体汽化时用于克服外界压强所做的功不能忽略,沸点变化时汽化热差别较大,这些与熔解热的不同点,要向学生指明。

空气的湿度和露点,在生产和生活实际中常常用到,课本讲解这些知识,就是为了给学生理解一些实际现象打下一个初步基础。

根据以上分析,本章的教学要求是:
\begin{enumerate}
\item 了解晶体和非晶体熔解和凝固时的不同,掌握熔解热的概念。
\item 知道蒸发和沸腾的区别,了解沸腾的条件及沸点跟压强的关系,掌握汽化热的概念。
\item 知道什么是饱和汽,理解动态平衡的概念.了解饱和汽的压强(或密度)跟温度有关,跟体积无关,知道气体液化的方法,了解临界温度的概念。
\item 理解绝对湿度、相对湿度和露点等概念.
\end{enumerate}

\section{教学建议}
\subsection{第一单元}
\subsubsection{熔解和凝固} 

可以先让同学观察萘的熔解和凝固曲线(图5.1和图5.2),思考以下几个问题:
\begin{enumerate}
\item 两个图各是什么曲线?    \item 曲线的哪一段表示物质的固态、液态或固液共存的状态?    \item 曲线的哪一段是熔解和凝固的过程?    \item 熔点和凝固点各是多少度?它们有何关系?
\end{enumerate}
这样就可以把熔解和凝固的过程以及熔点的概念搞清楚了。
\begin{figure}[htp]\centering
    \begin{minipage}[t]{0.31\textwidth}
    \centering
    %\includegraphics[scale=.8]{fig/5-1.png}
    \caption{}
    \end{minipage}
    \begin{minipage}[t]{0.31\textwidth}
    \centering
    %\includegraphics[scale=.8]{fig/5-2.png}
    \caption{}
    \end{minipage}
    \begin{minipage}[t]{0.31\textwidth}
        \centering
        %\includegraphics[scale=.8]{fig/5-3.png}
        \caption{}
        \end{minipage}
    \end{figure}

同样,通过对比萘的熔解曲线和松香的熔解曲线(图5.1和图5.3),可以使学生认识晶体和非晶体熔解过程的不同特点。

对晶体的熔解和凝固过程,运用分子运动论从晶体和液体的微观结构进行解释时,可以先结合熔解曲线对熔解过程作出示范分析,对凝固过程则可让学生自己进行分析说明。

要在讲清多数物质熔解时体积变大、少数物质熔解时体积变小的前提下,讲解物质的熔点与压强的关系,可以先引导学生看课本上各种物质的熔点表,并使他们懂得这些熔点是在一标准大气压下测得的。进而讨论熔点跟压强的关系,还须指出,熔解时物体的体积无论是增大还是减小,都需要吸收热量,用来克服分子力做功,破坏空间点阵,掺杂对熔点的影响主要通过冰、盐混合物熔点降低等实例说明。

\subsubsection{熔解热}

要从熔解过程中能量转换的角度说明熔解时吸收热量,用于克服分子力做功,破坏晶体的空间点阵,增加物体的分子势能。同时,还要说明,由于不同晶体的空间点阵不同,单位质量不同的物质熔解时吸收的热量也不同,为表征物质的这一性质,引入熔解热的概念。可以把熔解热跟比热、热量等概念从物理意义、定义、单位等方面加以对
比区别,加深对熔解热的认识,还应指出,熔解热与凝固时单位质量的物质放出的热量相等,是能量守恒定律的必然结果。

课本上的测定熔解热的方法,需要用到热量的计算、热平衡方程和量热器的构造等初中已学过的知识,需要适当的复习。为了具体地掌握测量的方法,可以一边进行实验一边讲解,把测量熔解热的方法具体化,结合熔解热的测定,还要讲清分析和解决包含物态变化过程在内的热平衡问题的基本思路:首先要明确研究对象,知道参加热量交换的物体是哪些?它们各自的初状态和应该达到的末状态的物态和状态参量(温度、压强、体积)是怎样的?状态变化的过程经历了哪些阶段?中间状态的物态和状态参量又是怎样的?然后,分别列出所有放热物体放出热量和所有吸热物体吸收热量的数学表达式。最后,由热平衡方程列式求解,在不知道末状态物质处于什么物态时,需要先判断末状态是什么物态。可以熔点温度为界线,结合熔解热进行热量的估算,具体判定。比如水与冰混合,可以把水温降低到0℃放出热量跟冰上升到$0^{\circ}{\rm C}$和完全熔解所需的热量进行比较,来判断末状态是水、是冰还是冰水混合物。

\subsection{第二单元}
\subsubsection{蒸发}

蒸发现象、影响蒸发快慢的因素、蒸发的致冷作用等,在初中学过,可以通过举例复习、巩固这些知识,教材中增加的内容是:用分子运动论解释蒸发现象、介绍蒸发
致冷的应用实例,这些学生不难理解。可以让他们自己阅读,然后组织讨论,把学习搞得生动活泼一些。

\subsubsection{饱和汽与饱和汽压}

需要帮助学生复习一些已有的知识为学习新课做准备.可以通过提问,复习以下问题:
\begin{enumerate}
\item 描述气体状态的三个参量(温度、体积、压强)的意义及微观解释是怎样的?
\item 影响蒸发快慢的因素是什么,并用分子运动论加以说明。
\end{enumerate}

正确理解动态平衡的意义是掌握饱和汽概念的关键。可引导学生通过课本图5.2从微观的角度想象,密闭容器内分子逸出液面和返回液面的运动状况,着重让学生理解这两种运动宏观上达到平衡时(即汽体的密度不变,液体不再减少),并非意味着分子运动的停止,而是单位时间内逸出液面的分子数与回到液面内的分子数相等,即处于动态平衡。在此基础上讲清什么是饱和汽和未饱和汽,以及“在一定温度下,未饱和汽密度小于饱和汽密度”的道理。

动态平衡是有条件的,对此要讲解清楚。由于外界条件变化的影响,原来的动态平衡状态被破坏,经过一段时间才能达到新的平衡。比如,温度升高时,分子平均动能增大,单位时间内逸出液面的分子数增多。于是原有的动态平衡状况被破坏;空间汽分子密度逐渐增大,导致单位时间内返回液面的分子数增多,从而达到新的条件下的动态平衡。由此得出,饱和汽的密度随着温度的升高而增大,掌握了动态平衡状态变化的条件,才能更好地理解饱和汽的性质。

饱和汽压,从微观上讲仍然决定于分子密度和分子的平均速率。讲清这一点,能消除对饱和汽压的陌生感,帮助学生
认识饱和汽压的实质。要做好测定饱和汽压值的实验,这不仅能使学生对饱和汽压获得感性认识,也能使他们了解一种测量饱和汽压的方法。

研究饱和汽压跟温度的关系和跟体积的关系,要先从演示实验得出结果,再从理论上加以说明,使学生对结论理解得更确切、深刻些,课本上的实验也可以用其他实验代替,但对课本上的实验也要讲一讲。

在得出“饱和汽压随温度的升高而增大”的结论后,还要进一步讲解饱和气压随温度怎样变化,其关系如课本图5.4所示,可以跟理想气体的等容线作对比,一定质量的理想气体在体积不变时,其压强与绝对温度成正比,从微观上讲,是质量一定、体积一定,因而分子密度未变;但温度升高,分子平均动能变大,平均速率变大,导致压强增大,对饱和汽来说,温度升高时,不仅分子平均动能变大,分子平均速率变大;同时液面进入空间的分子数增多,分子的密度也增大,决定压强的这两个微观因素都变大,这就使饱和气压的值增大得更多。

在讲解饱和汽压跟体积变化的关系时,要注意强调前提条件:温度不变。再从饱和汽密度与温度的关系和动态平衡状态的条件,讲清体积变化时饱和汽压值不变的道理,由于温度不变,分子平均动能不变,分子平均速率不变,设若汽的体积变大,分子密度变小,小于这一温度下饱和汽应有的密度,成为未饱和汽,破坏了原有的动态平衡,液体就会继续蒸发,使汽恢复到原有的分子密度,成为饱和汽,建立起新的动态平衡。因此,决定压强的两个微观因素都没有变化,因而压强不变。这里还应结合温度一定,饱和汽体积变小的过程,讲
清一部分汽会凝结成液态达到减小体积后的新的动态平衡。

要注意归纳饱和汽的特点,说明它与理想气体不同,不能用理想气体定律来解决饱和汽的问题。

\subsubsection{沸腾}

用课本图5.6的实验现象,讲解什么是沸腾以及沸腾的过程,要把它与蒸发对比,让学生认识这两种汽化现象的区别,由观察现象让学生了解:沸腾前,液体内气泡在上升过程中体积变小;沸腾时,液体内气泡在上升过程中体积变大。对这一现象不要作过细分析,为了利于学生理解沸腾的条件,可以讲到这样的程度:沸腾前,容器内壁吸附的空气在器壁和器底形成小气泡,由于周围的水向气泡内蒸发,气泡中除了空气外还有水的饱和汽。当器底的气泡受到液体的浮力上升时,因为液体底部的温度高于上部,气泡在上升过程中温度下降,气泡内部的饱和汽不断液化,气泡的体积不断变小。当液体内各部分的温度都达到某一温度,气泡内的饱和汽压值等于外部大气压强时,气泡在上升过程中体积不再减小,而且由于周围液体不断向泡内蒸发,体积还会继续变大。到达液面时破裂放出饱和汽,液体就沸腾了,这样就可以使学生理解液体沸腾的条件是它的饱和汽压等于外界的压强。

关于沸点,除了引导学生理解课本对沸点的解释外,要强调跟外界压强相等的饱和汽压对应的温度,就是液体的沸点。这样才能帮助学生理解沸点随外界压强变化的关系,对此,除了道理上要讲清楚,还要用演示实验来验证,再用这一关系去说明一些现象,如离地面越高沸点越低,高压锅、蒸汽锅炉用增大压强的办法来提高沸点等。

\subsubsection{汽化热} 

先以蒸发时致冷、沸腾时液体继续吸热而温
度不变的现象,讲解液体汽化时要从周围吸收热量,还要让学生从能量方面了解,固体熔解时,由于体积变化较小,吸收的热量主要用来克服分子间的引力做功;而液体汽化时,由于体积明显增大,吸收的热量,一部分用来克服分子间的引力做功,另一部分用来克服外界压强做功。这是熔解过程和汽化过程的不同之处。

关于汽化热,可以对照熔解热,讲清它的定义,公式和单位,再要求学生阅读课本第101页的两个表格,思考这两个表格各自说明什么问题,怎样用分子运动论作出解释?然后通过讨论,了解不同物质在同一压强下汽化热不同,同种物质在不同温度下汽化热不同的道理。比如,不同物质分子间的距离和作用力的性质、大小不同,汽化时分子逸出液体所做的功不同,因而汽化热不同,再用分子运动论和能量转化的知识,说明在某一温度下气体凝结为液体放出的热量,与同一温度下汽化过程中吸收的热量是相等的。

关于汽化热的测定,要讲好或做好课本图5.8所示的实验。要使学生了解测定汽化热的装置及各部分的作用,实验的原理和方法,引导学生用所测出的各个物理量写出计算汽化热的表达式。当然,也可以把要讲解的内容以问题的形式提出来,让学生结合实物和实验过程展开讨论,得出结论。这样,既能加深对汽化热的理解,又能帮助学生学会应用汽化热等知识,分析解决包括汽化和液化过程在内的热平衡问题。分析解决这一问题的基本思路,可以参照解决熔解热问题的有关步骤,在分析参加热交换的各个物体的物态和温度时,要把汽化或凝结过程中汽液共存的状态考虑在内,在计算热量
时,不要漏掉计算汽化或凝结时吸收或放出的热量。

\subsubsection{气体的液化}

首先要引导学生回忆课本图5.5所示的实验:管内的饱和汽在温度不变、体积变小时,要凝为液滴,管内积存的液体增多,让学生体会到,气体液化的关键是把未饱和汽变为饱和汽。

关于把未饱和汽变为饱和汽的方法,可以先让学生思考两个问题:
\begin{enumerate}
    \item 相同温度下,饱和汽与未饱和汽的密度有何不同?
    \item 在温度不变时可否采取改变体积的办法来使未饱和汽转变为饱和汽?
\end{enumerate}
组织学生讨论得出:在温度不变时,减小未饱和汽的体积,使它的密度增大到这个温度下饱和汽的密度,未饱和汽就变成饱和汽了,再让学生思考以下问题:
\begin{enumerate}
\item 不同温度的饱和汽的密度相同吗?饱和汽的密度与温度有何关系?    \item 把高温下的未饱和汽的温度降低,能使它变为某个低温下的饱和汽吗?
\end{enumerate}
组织学生讨论得出:降低温度可以使未饱和汽变为饱和汽,在这过程中,容器内汽的密度没有发生变化,只是高温下的未饱和汽密度等于某个低温下的饱和汽密度,概括起来,把未饱和汽变为饱和汽的方法:一是减小体积(增大压强),二是降低温度。

使饱和汽凝结为液体,仍可采取减小体积和降低温度的方法,那么,只采用增加压强(减小体积)的办法能否使所有的气体都液化呢?用历史事实说明是不可能的,由此引入临界温度的概念,要强调在这个温度以上,物质只能处于气态,不能单纯用增大压强的方法来使它液化,临界温度是每种气体都具有的一个特殊温度,它的物理含义是:临界温度是物质以液态存在的最高温度,还可以通过临界管实验观察乙醚、水
等物质的临界状态,再通过物质的临界温度的表,弄懂氢、氧、氮等气体在历史上为什么会被当成“永久气体”的原因。

要重视液态气体和低温技术应用的介绍,讲清楚一、两个典型例子,以使学生眼界开阔、思维活跃。

\subsection{第三单元}
这一单元围绕湿度和湿度的测定,讲解了绝对湿度,相对湿度、露点的概念,介绍了几种湿度计的原理和使用方法,是已学基础知识的应用,与实际有紧密的联系。由于又引入了一些新概念,教师要注意引导启发学生用已学知识去理解新知识。

\subsubsection{空气的湿度}

要联系生活事例,说明空气的干湿程度是经常变化的,再讲解绝对湿度的定义,绝对湿度的初始定义-空气中所含水蒸气的密度,即单位体积中所含水蒸气的质量,学生是可以接受的。过渡到“空气中所含水蒸气的压强,叫空气的绝对湿度”时,须引导学生从决定气体压强的两个微观因素:分子的密度和分子运动的平均速率,得出:温度一定时,气体压强与分子密度成正比。从而理解绝对湿度的定义。

再通过实例和课本列举的数据,说明湿度的影响取决于空气中的水蒸气离饱和状态的远近,引入相对湿度的概念,对相对湿度的计算,学生易于接受,可让学生阅读教材及不同温度下饱和汽压的表格,做点练习题,理解这些知识。

\subsubsection{露点}

复习用降低温度使未饱和汽变为饱和汽的方
法及其道理,引入并讲清露点的概念:设温度为$t_1$时,空气中未饱和汽的密度为$\rho$; 降低它的温度至$t_2$, 若$t_2$温度下的饱和汽密度也是$\rho$, 在$t_2$温度下,原来的未饱和汽就变成饱和汽了.温度$t_2$就是空气的露点.

测定露点的实验,瓶面出现的凝结现象不易观察,温度计的可见度小,在课堂上演示时,可请学生参与操作和观察,把结果告诉大家。

由露点来求相对湿度,可以先讲一讲根据露点与原来气温的差值可以大致判断相对湿度是大还是小,再讲清根据露点计算相对湿度的方法。先要明确露点温度水的饱和汽压值,就是原来温度下水的未饱和汽压值,即原来温度下的绝对湿度,这可在不同温度下水的饱和汽压表中查出,再在这个表中查出原来温度下水的饱和汽压值,即可计算相对湿度。

\subsubsection{湿度计}

教师可以对干湿泡湿度计的使用作演示和示范讲解,再让学生阅读教材,了解它的构造、原理和优缺点。阅读前可以提出一些思考题,如为什么湿泡温度计的示数要低于干泡温度计的示数?使用时要读出哪些数据?怎样才能得出相对湿度?还可以给出一些简单的练习题,帮助学生了解怎样用湿度计测定相对湿度。

\section{实验指导}
\subsection{演示实验}
\subsubsection{研究饱和汽性质的实验}

实验装置如图5.4所示.$A$是一根竖直固定在木板上的
玻璃管,上下口用橡皮塞塞紧。$B$是一根上端有进液口$D$和进液阀门$E$的长直细玻管,直穿上下两橡皮塞的正中央,下端口通过橡皮管与长颈漏斗$C$相连。长颈漏斗$C$夹持在木板上,可以上下移动,木板上画有均匀刻度,圆筒上端橡皮塞装有一进水口$F$和插入一温度计$G$. 圆筒下端橡皮塞上装有一带阀门的出水管。以上装置可用J2257型气体定律演示器代替。
\begin{figure}[htp]
    \centering
      %\includegraphics[scale=.7]{fig/5-4.png}
    \caption{}
\end{figure}

\paragraph{饱和汽压强的测量}
启开阀门$E$, 从长颈漏斗$C$的上端灌入清洁水银,提起漏斗,让水银排出$B$管内空气,使水银面上升至$B$管上端的阀门$E$, 恰有少量水银过阀门后,关闭阀门$E$. 在进液口$D$装入适量的乙醚(或其他被测物质),降低长颈漏斗,使$B$管内出现一段真空,两管水银面的高度差应为当地的大气压强$p_0$. 缓慢启开进液阀门$E$, 滴入$B$管适量的乙醚,再关上阀门$B$. 乙醚在$B$管水银面上方空间中蒸发;至$B$管水银面上剩有少许乙醚液为止,这时管内空间充满乙醚的饱和汽,读出此时$B$、$C$两管水银面的高度差$h$。$p=p_0\pm \rho gh$即为乙醚的饱和汽压值(当$B$管的水银面高于$C$管的水银面时,取“$-$”号,反之,取“$+$”号)。

\paragraph{饱和汽压不随体积变化}
保持室温不变,将长颈漏斗缓慢提升,可以看到饱和汽所在空间的体积变小,$B$管中水银面上乙醚液增加,而$B$、$C$管水银面高度差不变,这说明在温度一定时,饱和汽压与体积变化无关。

\paragraph{饱和汽压跟温度变化的关系}
把长颈漏斗置于适当位置,读出温度$t_1$和$B$、$C$水银面的高度差$h_1$, 这时的饱和汽压$p_1=p_0\pm\rho gh_1$. 从进水口$F$向圆筒中注满热水,在温度升高的过程中,$B$管水银面上方乙醚液将蒸发,若蒸发完了,则可启开进液阀门$E$, 继续滴入乙醚,待蒸发到水银面上留有少许乙醚为止,读出这时的温度$t_2$及$B$、$C$管水银面的高温差$h_2$, 饱和汽压为$p_2=p_0\pm \rho gh_2$, 得出饱和汽压随温度的升高而增大。

\subsubsection{温水在低压下沸腾}
如图5.5所示,在广口瓶中装入大半瓶温度约$95^{\circ}{\rm C}$左右的水,瓶口塞紧装有开口弯玻管的橡皮塞。玻管上口用橡皮管与注射器连接。用注射器抽气,使瓶内温水上方的气压降低,可看到温水沸腾现象。
\begin{figure}[htp]\centering
    \begin{minipage}[t]{0.48\textwidth}
    \centering
%\includegraphics[scale=.7]{fig/5-5.png}
    \caption{}
    \end{minipage}
    \begin{minipage}[t]{0.48\textwidth}
    \centering
%\includegraphics[scale=.7]{fig/5-6.png}
    \caption{}
    \end{minipage}
    \end{figure}

如图5.6所示,在烧瓶塞中插一个温度计$C$, 一个三
通管和一个直角弯管,三通管的一个支管带有阀门$A$的,另一支管用橡管接一大容量的注射器$B$; 直角弯管经橡皮管与
U型液体(水银或水)气压计$D$连接。

演示时,将阀门$A$启开与大气相通,用酒精灯将烧瓶内的水加热至沸腾,撤去酒精灯,待水中气泡消失、水温降至沸点以下时,关闭阀门$A$. 用注射器缓慢地抽气,可以看到U型气压计$D$左管液面上升,说明烧瓶内液面上方的压强减小;到一定时候,水重新沸腾。可以读出这时U型气压计两管液面的高度差$h$及温度计的示数$t$, 另测出当时的大气压强$p_0$, 由$p_0$和$h$可算出沸腾时对应的压强值$p$. 这样,可以定量研究$p$、$t$间的关系。

\subsection{学生实验}
\subsubsection{测定水的熔解热}

由于初中的学习,对这个实验已有一定基础。在明确实验目的、原理、操作步骤的前提下,还要注意以下几点。
\begin{enumerate}
\item 由于记录和计算涉及的物理量较多,学生要作充分的准备,分清哪些是实验中测得的?哪些是用表格查得的?哪些是计算得出的?便于正确处理数据,避免错记、漏重记。
\item $0^{\circ}{\rm C}$的冰块要在已准备好的冰水混合物中取出,尽量少带水分,迅速投入量热筒。
\item 温水的温度和质量、冰块的质量等要参照课本给出的参考数据,不要偏离过大,影响实验效果。
\item 搅动小筒中的水时,要用力适度,估计到温度计在水中的位置,以免损坏器材。实验时,温度计的示数从$t$. 下降到最低温度后,又缓慢回升,最低温度$t$就是热平衡时的温度。所以要对温度计密切注视,连续观察记录温度示数,才能正确确定平衡时的温度。
\end{enumerate}

\subsubsection{测定空气的相对湿度}

要明确这里使用的实验装置是一个简易的露点湿度计,通过测出空气的露点,计算相对湿度,实验的研究对象是金属盒外的空气。盒内的冰在水中熔解吸热,起降低温度的作用。

实验装置中光滑的环形金属片是为了易于观察金属
盒上的露滴,与之对比而设置的。要保证它的表面光亮清洁,跟金属盒绝热。

如果冰块的备取有困难,可以准备适量的铵盐(或尿素)代替,并准备一些浓硫酸。

投入水中的碎冰(不要太大块)或铵盐要适量.过少时,温度尚未降到露点,而溶质已熔解完毕,实验不能成功;过多时,温度降到露点以后的回升时间太长,搅拌的快慢,应视温度变化的具体情况而定。若用铵盐代替做实验时,发现温度回升太慢,可以滴入几滴浓硫酸,使温度回升加快。

为及时观察露滴的出现或消失,可用手指(或棉纱)在金属盒的同一地方来回擦动,将擦过的地方与环形金属片表面对比,以期观察是否有露滴出现或消失。一经发现,及时记下温度的示数。

\subsection{课外实验活动}
\subsubsection{测定水的汽化热}

实验误差的主要来源是设计原理的不够完善。这个实验把铝锅里的水每秒钟吸收的热量看作是不变的;实际上,水在升温和汽化(沸腾)的过程中,单位时间内吸收的热量是不同的。因为热交换物体间的温度差不同,单位时间传递的热量也不同,温差越大,传递的热量越多,温差越小,传递的热量越少,另外,随着水温的升高,水跟周围空气的温差增大,在单位时间内放出的热量也增加了。由于这两方面的原因,随着水温的升高,水在单位时间内获得的净热量就会减少,还有,水在
温度没有达到沸点之前就不断蒸发,达到沸点时水的质量已经比原来的少了,这些因素在本实验中都没有考虑,再则,水的温度在未升到沸点前已开始汽化,即水的质量随着温度的升高不断减少,实验误差的另一个来源是由测量的水的质量、水的初温和沸点,以及加热水至沸腾和全部汽化的时间,这是测量都可能产生误差。

\subsubsection{估计水升高的温度}

估计水升高的温度为摄氏几度。

\begin{enumerate}
    \item 测火柴杆的质量,把火柴杆看成是正四方柱体,测出火柴杆正方形横截面的边长$a$, 测出它的长度$b$, 则火柴的体积$V=a^2b$. 查出一般木材的密度$\rho$的值为$(0.4\sim 0.9)\x10^3{\rm kg/m^3}$, 火柴杆的质量$m=\rho a^2b$.
    \item 计算火柴杆燃烧放出的热量,可以认为,火柴杆完全燃烧放出的热量$Q_{\text{放}}=qm$. $q$是木材的燃烧值$1.26\x10^7{\rm J/kg}$。
    \item 估计热传递的效率.考虑到玻璃是热的不良导体,火柴杆燃烧时间内向周围空间传递的热量较多,估计$\eta=20\%$.
    \item 计算水的温度的升高$\Delta t$. 由于
    \[Q_{\text{吸}}=m_{\text{水}}C_{\text{水}}\Delta t,\qquad Q_{\text{放}}=q\rho a^2b\]
    由热平衡方程 $Q_{\text{吸}}=\eta Q_{\text{放}}$, 得
    \[\Delta t=\frac{\eta q\rho a^2b}{m_{\text{水}}C_{\text{水}}}\]
    代入数据,即可算得$\Delta t$.
\end{enumerate}

\subsection{练习一}

\begin{enumerate}
    \item 把玻璃放在火上加热,观察它的熔解情况,看看玻璃是不是先变软,再流动.玻璃是不是晶体?
    
\begin{solution}
 玻璃是先变软,再流动。由于它没有固液共存温度不变的熔点,所以玻璃不是晶体。
\end{solution}
    \item 解释下面的现象:把一块冰放在支承物上(图5.1),
    将两端各挂一个重物的铁丝搭在冰块上.过一段时间后可以看到,铁丝切进冰块,但是铁丝穿过冰块的地方并没有留下切口,冰仍然是完整的一块,铁丝
    为什么能切进冰块?铁丝穿过后上面的冰为什么又成了完整的一块?如果有条件,自己做这个实验.
\begin{figure}[htp]
\centering
%\includegraphics[scale=.8]{fig/5-1.png}
\caption{细铁丝穿过冰块而不留下切口
}
\end{figure}
    
\begin{solution}
因为铁丝与冰块接触面积小,在重物通过铁丝对冰的接触面的压力作用下,产生的压强很大,使冰的熔点降低,即在$0^\circ$C以下就可以熔解.因此接触处的冰熔解成水,铁丝切进冰块,切口处的水在铁丝切过以后,压强又恢复为1标准大
气压,而水温还在$0^\circ$C以下,切口处的水又结成冰,冰块又成了完整的。
\end{solution}
    \item   冬季在菜窖里放上几桶水,可以使窖内的温度不致降低得很多,防止把莱冻坏.这是什么道理?如果在窖内放入200千克10$^\circ$C的水,试计算这些水结成0$^\circ$C的冰时放出的热量.这相当于燃烧多少千克干木柴所放出的热量?干木柴的
    燃烧值约为$1.26\times 10^4$千焦/千克.
    
    \begin{solution}
 由于水的比热和冰的熔解热都较大,要使水降温并凝固成冰,要向周围放出较多的热量。冬季在菜窖内放几桶水时,就能使菜窖的气温下降得少些。

200千克10$^\circ$C的水,结成0$^\circ$C的冰放出的热量
\[\begin{split}
  Q&=c_{\text{水}}m_{\text{水}}(t_1-t_2)+\lambda m_{\text{冰}}\\
  &=4.2\x 10^3\x 200\x (10-0)+3.35\x 10^5\x 200\\
  &=7.5\x 10^{7}{\rm J}
\end{split}\]
若它与质量为$m$的干木材完全燃烧放出的热量相当。则由$Q=qm$得
\[m=\frac{Q}{q}=\frac{7.5\x 10^{7}}{1.26\times 10^7}=6.0{\rm kg}\]
    \end{solution}
\item  铜制量热器小简的质量是160克,装入200克20$^\circ$C的水.向水里放进30克0$^\circ$C的冰,冰完全熔解后水的温度是多少摄氏度?
    
\begin{solution}
  量热器小筒、水和冰进行热交换,热平衡时温度为$t$.

量热器小筒质量$m_{\text{铜}}$,温度由$t_1$ 降到$t$, 放出热量
\[Q_1=c_{\text{铜}}m_{\text{铜}}(t_1-t)\] 
水的质量$m_{\text{水}}$, 温度由$t_1$ 降到$t$, 放出热量
\[Q_2=c_{\text{水}}m_{\text{水}}(t_1-t)\] 
冰的质量$m_{\text{冰}}$, 熔解后温度由0$^\circ$C升高到$t$, 吸收热量
\[Q_3=\lambda m_{\text{冰}} +c_{\text{水}}m_{\text{冰}}t \] 
由热平衡方程$Q_1+Q_2=Q_3$, 得:
\[\left(c_{\text{铜}}m_{\text{铜}}+c_{\text{水}}m_{\text{水}}\right)(t_1-t)=\lambda m_{\text{冰}} +c_{\text{水}}m_{\text{冰}}t \]
因此:
\[\begin{split}
  t&=\frac{\left(c_{\text{铜}}m_{\text{铜}}+c_{\text{水}}m_{\text{水}}\right)t_1-\lambda m_{\text{冰}}}{c_{\text{铜}}m_{\text{铜}}+c_{\text{水}}\left(m_{\text{水}} +m_{\text{冰}} \right)}\\
  &=\frac{(3.9\x 10^2\x0.16+4.2\x10^3\x0.20)\x20-3.35\x10^5\x0.030}{3.9\x 10^2\x0.16+4.2\x10^3\x(0.20+0.030)}\\
  &=7.8^{\circ}{\rm C}
\end{split}\]
\end{solution}
\item  量热器的铜制小筒里盛有200克15$^\circ$C的水,小筒的质量为160克,向水里放入50克0$^\circ$C的冰,求量热器里的末温度是多少摄氏度?
    
\begin{solution}
量热器小筒和水的温度下降至0$^\circ$C时,放出的热量
\[\begin{split}
  Q_{\text{放}}&=\left(c_{\text{水}}m_{\text{水}}+c_{\text{铜}}m_{\text{铜}}\right)t_1\\
  &=(4.2\x 10^3\x 0.200+3.9\x 10^2\x 0.160)\x 15\\
  &=1.4\x 10^4{\rm J}
\end{split}\]
质量为50克的冰完全熔解为0$^\circ$C的水所需要的热量
\[Q_{\text{吸}}=\lambda m_{\text{冰}}=3.35\x10^5\x0.050=1. 7\x10^4{\rm J}\]
由$Q_{\text{放}}<Q_{\text{吸}}$,水和小筒温度降至0$^\circ$C放出的全部热量不足以使冰完全熔解,所以末状态是冰水共存,末温度应为0$^\circ$C。
\end{solution}
\end{enumerate}


\subsection{练习二}

\begin{enumerate}
    \item 举出几个蒸发致冷的例子来.
    
    \begin{solution}
在注射针药前,护士要在注射部位涂碘酒(或酒精)
消毒,涂上后有一股凉爽的感觉,这是由于酒精易于蒸发,蒸发时从涂抹部位的皮肤吸热,致使温度降低,感觉凉爽。

在夏季高温的地区,常常向地面酒一些水,水在蒸发的过程中要吸收热量,从而达到室内降温的目的。      
    \end{solution}
\item 液面上的汽达到饱和时,还有没有液体分子从液面飞出?为什么这时从宏观上看来液体不再蒸发?
    
\begin{solution}
有分子从液面飞出,只不过这时,单位时间内飞出液面和回到液面的分子数是相等的,所以从宏观上看来液体不再蒸发。
\end{solution}
\item 饱和汽的密度怎样随温度而变化?饱和汽的压强怎祥随温度而变化?为什么这样变化?
    
\begin{solution}
饱和汽的密度随温度的升高而增大。因为温度升高后,液体分子的平均动能增大,单位时间内逸出液面的分子数增多,破坏了原有的动态平衡,液体继续蒸发,汽的密度不断增大,直到建立起新的动态平衡为止。

饱和汽的压强随着温度的升高而增大,气体的压强从微观上决定于分子密度和分子的平均速率。当温度升高时,饱和汽分子平均动能增大,分子平均速率也增大;同时,逸出液面进入空间的分子数增多,分子密度随之增大。决定压强的这两个微观因素都变大,因此饱和汽的压强变大。
\end{solution}
\item 在温度不变的情况下,增大液面上饱和汽的体积时,下面的说法哪些是正确的:
\begin{enumerate}
    \item 饱和汽的质量不变,饱和汽的密度减小;
    \item 饱和汽的密度不变,饱和汽的压强也不变;
    \item 饱和汽的密度不变,饱和汽的压强增大;
    \item 饱和汽的质量增大,饱和汽的压强也增大;
    \item 饱和汽的质量增大,饱和汽的压强不变.
\end{enumerate}

    
\begin{solution}
  (b)(e)是正确的。
\end{solution}
\item 解释下面的现象:密闭容器中装有少量液态乙醚,当容器的温度升高时,液态乙醚逐渐减少;容器升高到一定温度
时,液态乙醚消失;容器冷却时,容器中又出现液态乙醚.
    
\begin{solution}
  密闭容器内有液态乙醚,说明空间的乙醚蒸气处于饱和状态,当温度升高时,饱和汽变为未饱和汽,乙醚液要继续蒸发,因而逐渐减少。升高至一定温度时,乙醚液蒸发完毕,液态乙醚消失,容器冷却时温度降低,饱和汽的密度随之减小,空间中一部分乙醚汽的分子凝聚为乙醚液滴,积存下来又出现液态乙醚。
\end{solution}
\end{enumerate}




\subsection{练习三}
\begin{enumerate}
 \item 在1标准大气压下,乙醚的沸点是35$^\circ$C,这个温度时乙醚的饱和汽压是多大?
    
 \begin{solution}
这个温度时乙醚的饱和汽压是1标准大气压。
 \end{solution}
  \item 锡的熔点是232$^\circ$C,但是用锡焊的水壶盛着水放在1000$^\circ$C以上的火上烧,锡并不熔解.为什么?
    
  \begin{solution}
    由于锡焊水壶内盛有水,水壶把从火上吸收的热量及时地传递给水,因而水壶的温度也只能稍高于水的温度。通常情况下,水的沸点是100$^\circ$C, 即使在水沸腾时,水壶的温度也只能是100$^\circ$C多一点,不会达到锡的熔点232$^\circ$C, 所以,锡并不熔解。
  \end{solution}
  \item 在蒸汽暖室装置的散热器里,每小时有20千克100$^\circ$C的水蒸气液化成水,
    并且水的温度降低到80$^\circ$C.求散热器每小时供给房间的热量.
    
    \begin{solution}
质量为$m$的水蒸气在温度$t_1$时凝结成水,其温度再由$t_1$降低到$t_2$, 放出的热量为
\[Q=Lm+cm (t_1-t_2) =m [L+c (t_1-t_2) ]\]

把已知量代入上式得
\[Q=20\x[2. 26\x10^6+4.2\x10^3\x(100-80)]
=4.7\x10^7{\rm J}\]
这就是每小时散热器供给房间的热量。 
    \end{solution}
    \item 某人在做测定水的汽化热实验时,得到的数据如下:
   钢制量热器小筒的质量为200克,通入水蒸气前筒内水的质量为350克,温度为14$^\circ$C;
    通入100$^\circ$C的水蒸气后水的温度为36$^\circ$C,水的质量为364克,他测得的水的汽化热是多少?
    
    \begin{solution}
量热器小筒的质量$m=200$克,通入水蒸气前水的质量$m_1=350$克,它们的温度由$t_1=14^{\circ}$C升高到$t_2=36^{\circ}$C, 吸收的热量为
\[Q_{\text{吸}}= (c_{\text{铜}}m+c_{\text{水}}m_1) (t_2-t_1)\]

通入水蒸气之后水的质量为$m_2=360$克.凝结为水的水蒸气质量为$(m_2-m_1)$, 在温度$t_3=100^{\circ}$C凝结成水以后,温度降低到$t_2$, 放出的热量为
\[Q_{\text{放}}=L (m_2-m_1) +c_{\text{水}} (m_2-m_1) (t_3-t_2)\]
由热平衡方程$Q_{\text{吸}}=Q_{\text{放}}$得
\[(c_{\text{铜}}m+c_{\text{水}}m_1) (t_2-t_1)=(m_2-m_1)[L+c_{\text{水}}(t_3-t_2)]\]
所以
\[\begin{split}
  L&=\frac{(c_{\text{铜}}m+c_{\text{水}}m_1) (t_2-t_1)}{m_2-m_1}-c_{\text{水}}(t_3-t_2) \\
  &=\frac{(3.9\x 10^2\x 0.200+4.2\x10^3\x 0.350)(36-14)}{0.364-0.350}-4.2\x 10^3\x (100-36)\\
  &=2.2\x 10^3 {\rm kJ/kg}
\end{split}
  \]
    \end{solution}
    \item 容器里装有0$^\circ$C的冰和水各500克,向里面通入100$^\circ$C的水蒸气后,
    容器里的水升高到了30$^\circ$C.假设容器吸收的热量很少,可以忽略不计,并且容器是绝热的,
    计算一下通入的水蒸气有多少?
    
    \begin{solution}
参加热交换的物体有冰、水和水蒸气,热平衡时是液态。

水的质量$m_1$, 温度由$t_1=0^{\circ}$C升到$t_2$, 吸收热量
\[Q_1=c_{\text{水}}m_1 (t_2-t_1)=c_{\text{水}}m_1t_2\]

冰的质量$m_2=m_1$, 在温度$t_1=0^{\circ}$C熔解后升温至$t_2$, 吸收热量
\[Q_2=\lambda m_2+c_{\text{水}}m_2(t_2-t_1)=\lambda m_1+c_{\text{水}}m_1t_2\]

水蒸气质量$m_3$, 在温度$t_3$时凝结成水后,温度降低到$t_2$, 放出的热量为
\[Q_3=Lm_3+c_{\text{水}}m_3 (t_3-t_2) \]

由热平衡方程 $Q_1+Q_2=Q_3$得
\[\lambda m_1+2c_{\text{水}}m_1t_2=Lm_3+c_{\text{水}}m_3 (t_3-t_2)\] 
所以
\[\begin{split}
  m_3&=\frac{\lambda m_1+2c_{\text{水}}m_1t_2}{L+c_{\text{水}}(t_3-t_2)}\\
  &=\frac{35\x10^5\x0.500+2\x4.2\x10^5\x0.500\x30}{2.26\x10^6+4.2\x10^3\x (100-30)}\\
  &=0.12{\rm kg}
\end{split}\]    
    \end{solution}
\end{enumerate}




\subsection{练习四}

\begin{enumerate}
	\item 说明使未饱和汽变为饱和汽的方法和道理.
    
  \begin{solution}
使未饱和汽变为饱和汽的方法有两种。

一是降低温度,体积不变的情况下,温度下降,未饱和汽
的密度不变,由于饱和汽的密度跟温度有关系,温度低时,饱和汽的密度小,当温度下降至某一数值时,原来温度下未饱和汽的密度恰好等于该温度下饱和汽的密度,于是未饱和汽就变成了饱和汽。

二是增加压强,温度不变的情况下增加压强是由减小未饱和汽的体积来获得的。体积减小时未饱和汽的密度增大。当汽的密度增大至这一温度下所对应的饱和汽密度时,原来的未饱和汽就变为饱和汽。
  \end{solution}
\item 潮湿的天气里,湿衣服不容易干,为什么?
    
\begin{solution}
  潮湿的天气里,空气的相对湿度大,空气里的水蒸气压强已接近饱和汽压值,衣服上的水蒸发较难进行,所以湿衣服不易干。
\end{solution}
\item 在绝对湿度相同的情况下,冬天和夏天的相对湿度哪个大?为什么?
    
\begin{solution}
冬天的相对湿度大,由公式$B=\dfrac{p}{P}\x 100\%$可知,在绝对温度$p$相同时,水的饱和汽压$P$大的,相对湿度$B$小;$p$小的,$B$大。从不同温度下水的饱和汽压表中可知,夏天温度高,$P$值大;冬天温度低,$P$值小,所以冬天的相对湿度$B$值大。
\end{solution}
\item 空气的绝对湿度是9毫米汞柱,气温是16$^\circ$C,相对湿度是多少?
    
\begin{solution}
由表中查得,$t=16^{\circ}$C时水的饱和汽压$P=13.63$毫米汞柱。这时的相对湿度
\[B=\frac{p}{P}\x100\%=\frac{9}{13.63}\x100\%=66\%\]
\end{solution}
\item 教室里空气的相对湿度是60\%,温度是18$^\circ$C,绝对温度是多少?
    
\begin{solution}
  由表中查得$t=18^{\circ}$C时,水的饱和汽压$P=15.48$毫米汞柱.由公式$B=p/P$得$p=B\cdot P$。将$B=60\%=0.60$代入此式,
  \[P=0.60\x15.48=9.3{\rm mmHg}\]
\end{solution}
\end{enumerate}





\subsection{练习五}

\begin{enumerate}
	\item 在北方,冬天戴着眼镜从寒冷的室外进入温暖的空内时,镜片上常出现一层细小的露滴.这是为什么?
    
  \begin{solution}
在室外镜片温度较低,进入室内时,温度较低的镜片使周围温度下降至露点,空气中的水蒸气,便在镜片上凝成露滴。
  \end{solution}
\item 白天空气的绝对湿度是13.7毫米汞柱.天气预报夜里的最低温度是14$^\circ$C,如果空气的绝对湿度保持不变,夜里会不会出现露水?
    
\begin{solution}
  白天的绝对湿度,即空气中水蒸气的压强为13.7毫米汞柱,把这个压强值作为水的饱和蒸气压所对应的温度——露点,经查表得知约为16$^\circ$C, 而夜里的温度是14$^\circ$C, 已在露点以下,所以会出现露水。
\end{solution}
\item 如果干湿泡湿度计上两支温度计的指示数字相同,这时空气的相对湿度是多少?
    
\begin{solution}
若两支温度计的示数相同,即干湿泡温度差为零,湿泡不再蒸发水分,说明空气的水汽压强(绝对湿度)等于这一温度下的饱和水汽压,相对湿度为100\%.
\end{solution}
\item 空气的温度是20$^\circ$C,露点是12$^\circ$C,这时的绝对湿度和相对湿度是多少?
    
\begin{solution}
查表得,$t=20^{\circ}$C时水的饱和汽压值$P=17.54$毫米汞柱.露点($12^{\circ}$C)时水的饱和汽压值就是$20^{\circ}$C下空气里水蒸气的压强——绝对湿度查表得$p=10.52$毫米汞柱.

这时的相对湿度
\[B=\frac{p}{P}\x100\%=\frac{10.52}{17.54}\x100\%=60\%\]
\end{solution}
\item 空气的温度是25$^\circ$C,相对湿度是50\%, 气温降低到多少摄氏度时,才会有露出现?
    
\begin{solution}
  查表得$25^{\circ}$C时水的饱和汽压$P=23.76$毫米汞柱.相对湿度$B=50\%=0.50$.

由$B=p/P$, 得绝对湿度
\[p=B\cdot P=0.50\x23.76=12{\rm mmHg}\]

空气的绝对湿度,就是温度降至露点时水的饱和汽压,以12毫米汞柱为饱和汽压查得对应的温度约为$14^{\circ}$C, 即为露点,当气温降低到这个温度时,才会有露出现。
\end{solution}
\end{enumerate}

\section{参考资料}
\subsection{关于现有的物态}

当大量微观粒子在一定温度和压力下,相互聚合为一种稳定的结构状态时,称为“物质的一种状态”,简称“物态”。

在本世纪以前,人们还只能从物体的宏观特征(体积、形状)来区别物质的状态;把它们分为固态、液态和气态,但从物质的内部结构来看就远不止这三态了。

对固态而言,就分为结晶态和非晶固态,从宏观看,一切晶体的基本特征有三点:外观上两对应的晶面夹角恒等;物理性质表现为各向异性;相变时有确定的温度,从微观看,晶体分子规则的排列,组成空间点阵,这些是区分结晶态和非晶固态的标志。

在结晶态和液态之间,有不少有机物质,既具有流动性(无一定形状),又具有类似晶体的光学性质,这种物态被称为液晶态。

等离子体,被称为物质的第四态。它是气体电离成为带正电的离子和带负电的电子所组成的集合体,而且正负电量相等,这两种离子的集聚状态叫等离子态,它是1925年美国的兰米尔在研究气体放电时发现的。通过加热到万度以上的高温、辐射、放电等方式,使气体电离就变为等离子体,等离子体存在的温度极高。例如喷焊工艺常用的等离子弧中,氢
等离子体温度为5400K以上,氩等离子体温度为14700K, 这也只能叫“冷”等离子体,灼热的等离子体可达几百万至几千万度高温,等离子体的研究,是目前研究受控热核反应的重要方面,此外,在天体物理、气体电离、微波和超声速流体力学等方面,都有重要的应用。

另外,还有几种在超低温、超高压、超高温下的物态。

在超低温的条件下,某些金属的直流电阻将超近于零,这叫做超导态,1911年荷兰物理学家昂尼斯,在温度降低至
4.2K时,发现水银直流电阻消失,为此他获得1913年的诺贝尔奖。超导材料的制成,将会引起电工技术的巨大变革,但由于需要超低温,目前技术上限制了它的应用。

在超低温的条件下,有的液体的粘滞性也完全消失,这叫做超流态,1930年荷兰科学家基索姆发现,当温度降至2.1K时,液态氮的性质发生突变,其粘滞系数为零,能够无阻地流动,如果在其中插上一根内径为$10^{-5}$厘米的毛细管,它就会象喷泉一样,溢出管外;而流速与液面的压强差和毛细管的长度无关,这种特性,他把它命名为超流。

在超高压条件下,氢可以转变成具有金属特性的固态,称为金属氢。有人估算,在几百万大气压下,氢可以转化为金属。可以推知,非金属也能在高压下转化为金属;再把这些变为金属的物质,在极低的温度下实现超导态,为提高超导转变温度开辟新路,有人在理论上计算,固态氢在压强为$1. 35\x10^5$帕的条件下可以转变为金属氢,而且金属氢的超导转
变温度在80K。目前,世界上有100多个实验室在研制金属氢。近两年来,苏联、日本相继宣布在实验室中研制成功。

在超高压、超高温条件下,物质原子的所有电子都脱离原子核而成为自由电子,所有的裸原子核高度紧密地堆积,自由电子在其间混乱运动,由于密度很高,被称为超固态,根据光谱分析说明,在一种叫白矮星的恒星上,物态处于超固态,其平均密度为水的几万至一亿倍,中心密度可达$10^{10}{\rm g/cm^3}$, 温度为1千万度.

设想的物态总图如图5.8所示。
\begin{figure}[htp]
  \centering
 % \includegraphics[scale=.8]{fig/5-8.png}
  \caption{}
\end{figure}

\subsubsection{水在结冰时体积增大}

水和冰的物理性质不同,与分子的结构有关,水分子属于非直线型的极性分子。其中的氧原子与两个氢原子通过共价键连接,两个O$-$H键间的夹角为$104^{\circ}40'$, 一个氢原子除了用仅有的一个电子与另一原子以共价键结合外,它的原子核(因没有电子)还会被其他原子的电子层吸引而形成“氢键”。 它是含氢化合物之间的一种相互作用,将在产生“缔合分子”的群体和影响晶体结构形状和稳定性方面起重要作用。
\begin{figure}[htp]
  \centering
 % \includegraphics[scale=.8]{fig/5-9.png}
  \caption{}
\end{figure}

水在液态时,由于水分子间氢键的作用,使得水分子不全以单分子存在,而是三三两两地缔合在一起的,如图5.9所示,如水的温度降低时,分子平均动能减小,缔合分子数增多.到$4^{\circ}{\rm C}$, 缔合分子排列最紧密,此时分子的密度最大,水的体积最小,温度继续降低时,出现更多的缔合分子;到$0^{\circ}{\rm C}$时,全部水分子因氢键结合,排列成规则的六方晶系,这是一
种以氢键作为桥梁架设起的较松散的结构,两个相邻的水分子中心的距离约为$2. 76\x10^{-10}$米,其间有很大的空洞,几乎可以容纳一个水分子,因而冰的外观体积较大,这就是水从液态转变为结晶态时体积增大的原因。


\subsubsection{水沸腾时气泡体积的变化}

水中空气泡的存在,是沸腾的重要条件。它是由附着在器壁上、未溶于水的微量气体形成的。

沸腾前,气泡还附着在器壁上时,随着水温升高,气泡周围的水不断向泡内蒸发,气泡的体积及泡内水的饱和汽压随温度升高而变大。当气泡体积增大到某一程度,水对它的浮力大于它对器壁的附着力时,气泡脱离器壁,在浮力作用下上升(图5.10)。
\begin{figure}[htp]
  \centering
 % \includegraphics[scale=.8]{fig/5-10.png}
  \caption{}
\end{figure}

由于沸腾前水的上、下层温度不同,离液面越近,水温越低,这时,气泡表面内,外压强的力学平衡条件是:
\begin{equation}
  p_0+p_{\text{静}}+p_{\text{表}}=p_{\text{饱}}+p_{\text{气}}
\end{equation}
其中气泡外部的压强:$p_0$是液面上方的大气压强;$p_{\text{静}}=\rho gh$是容器内液体的静压强;$p_{\text{表}}=2\sigma/r$是气泡表面张力引起的附加压强,$\sigma$是表面张力系数。气泡内部的压强有:
\[p_{\text{气}}=\frac{mRT}{\mu V}\]
$p_{\text{气}}$
是气泡内空气的压强,$m/\mu$是空气的摩尔数;$p_{\text{饱}}$是泡内水汽的饱和气压。

把上式变为
\begin{equation}
  p_0+\rho gh+\frac{2\sigma}{r}=p_{\text{饱}}+\frac{mRT}{\mu V}
\end{equation}
考虑到$p_{\text{静}}$约为$p_0$的千分之几,$p_{\text{表}}$在$r=0. 1$毫米时约为$p_0$的百分之一点几,此两项略去不计,得
\[p_0=p_{\text{饱}}+\frac{mRT}{\mu V}\]
于是
\[V=\frac{mRT}{\mu (p_0-p_{\text{饱}})}\]

沸腾前,在气泡上升过程中,温度$T$变小,$p_{\text{饱}}$随着变小,$(p_0-p_{\text{饱}})$变大,故气泡体积$V$变小。

当继续加热,上、下层水温渐趋一致地达到某一温度$T$时,气泡上升过程中温度$T$不变,$p_{\text{饱}}$不变。从式(5.1)看,气泡内部的压强$(p_{\text{饱}}+p_{\text{气}})$不变;而气泡外部的压强中,$p_{\text{静}}$由于$h$变小而变小。所以,外压强小于内部压强,气泡的体积变大;$p_{\text{表}}$随$r$的变大而减小,致使体积$V$迅速变大。

再从力的平衡方程讨论。由(5.2)式可得
\[V=\frac{mRT}{\mu(p_0-p_{\text{饱}}+\rho gh+\dfrac{2\sigma}{r})}\]

这时$\dfrac{2\sigma}{r}$仍然可以略去,即得
\[V=\frac{mRT}{\mu(p_0-p_{\text{饱}}+\rho gh)}\]
整个水的温度上升至某一温度$T$时,所对应的饱和水汽压
$p_{\text{饱}}=p_0+\rho gh$. 当气泡接近水面时,$h\to 0$, $p_{\text{饱}}\to p_0$, 体积$V\to \infty$, 气泡升至液面破裂,放出蒸汽,水就沸腾了。

\subsubsection{干湿泡温度计的干湿泡温度差与相对湿度的关系}

在干湿泡湿度计中,湿球水分汽化(蒸发),要从湿泡温度计吸热,使湿泡温度计的示数低于干泡温度计的示数,湿球水分的汽化,也从它周围的空气中吸热,使它周围空气的温度降低,在达到热平衡时,湿泡的温度跟它周围空气的温度相同。

设在干泡温度计的示数为$t_1$, 湿泡温度计的示数为$t_2$时达到热平衡,我们来求空气的相对湿度。

含有水的未饱和汽的空气与湿球热交换,温度由原来的温度$t_1$(即干泡温度计的示数)降低到湿球的温度$t_2$。根据热量交换理论可知,空气放出的热量$Q_1$跟干湿泡温度差成正比,即$Q_1=h(t_1-t_2)$. 式中$h$为热量交换系数,它的量值决定于风速大小。

设在$t_1$温度下,湿球蒸发的水蒸气的质量为$m$, 在这温度下水蒸气的饱和汽压为$p$, 周围空气中水蒸气的压强(绝对湿度)为$p$, 大气压强为$P_0$, 汽化热为$L$. 根据蒸发的理论研究可知,在湿球温度下蒸发的水蒸气的质量,跟饱和汽压与绝对湿度的差$(P-p)$成正比,跟大气压强成反比,即$m=k(P-p)/p_0$. $k$为比例系数.这部分水蒸发所需的热量
\[Q_2=mL=Lk (P-p) /P_0\]

在湿泡温度计示数恒定时,处于热平衡状态,$Q_1=Q_2$, 即
\[h (t_1-t_2) =Lk (P-p) /p_0\]
由此得
\[p=P-p_0 \frac{h}{Lk} (t_1-t_2) \]

令$A=\dfrac{h}{Lk}$,绝对湿度
\[p=P-Ap_0(t_1-t_2)\]
相对湿度
\[B=\frac{p}{P}\x 100\%=\left[1-\frac{Ap_0}{P}(t_1-t_2)\right]\x 100\%\]
式中$A$称为湿度系数,它跟风速有关:
\begin{center}
  \begin{tabular}{p{.3\textwidth}l}
    对室内静止空气& $A=1.2\x10^{-3}$ 1/度\\
风速$>2{\rm m/s}$& $A=0.66\x10^{-3}$ 1/度\\
在百叶窗中&  $A=0.79\x10^{-3}$ 1/度
  \end{tabular}
\end{center}

若已知干泡温度计的示数$t_1$, 干湿泡温度差$(t_1-t_2)$和当时的大气压强$p_0$, 即可求得绝对湿度$p$和相对湿度$B$。

例如:在室内干湿泡湿度计上读得$t_1=30^{\circ}$C, $t_1-t_2=8.3^{\circ}$C, 又得知$p_0=$750毫米汞柱.
由饱和水汽压表中查得$30^{\circ}$C时,$P=31.82$毫米汞柱.
\[\begin{split}
  p&=P-p_0A(t_1-t_2)\\
  &=31.82-750\x 1.2\x 10^{-3}\x 8.3\\
  &=24.35{\rm mmHg}
\end{split}\]
\[B=\frac{p}{P}\x100\%=\frac{24.35}{31.82}\x100\%=76.5\%\]


