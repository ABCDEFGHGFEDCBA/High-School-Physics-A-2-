\chapter{稳恒电流}
\section{教学要求}
这一章是电学部分的重点章,本章讲授的直流电路基本
知识,有较大的实用价值,是学习电工和电子技术的基础,对
日常生活和生产中使用电器也很有用。

教材注意了联系前章学过的电场知识,注意了联系实际。
在讲电流的产生、电功、电路中的电势降落等知识时,注意用
前章学过的电场、电势能、电势等知识给予说明,以加深学生
的理解。在讲串联和并联电路、电路的分析和计算、闭合电路
的欧姆定律等知识时,注意它们的实际应用,在培养能力方
面,注意教给学生分析解决电路问题的方法。

全章教材共十四节,可以划分为五个单元,第一节到第
三节为第一单元,讲述直流电路的一些基本概念和规律。第
四节和第五节为第二单元,从功和能的观点研究电路中的电
能转化。第六节到第九节为第三单元,讲述串联电路和并联
电路,以及简单的混联电路。第十节到第十二节为第四单元,
讲述闭合电路的欧姆定律及其应用。第十三节和第十四节为
第五单元,讲述测量电阻的方法。

部分电路欧姆定律和闭合电路欧姆定律是电学中最基本
的规律之一,它们不仅是解决直流电路有关计算的依据,而且
对学习交流电路有重要作用。所以,欧姆定律是全章的中心
和贯穿全章教材的主线。电流、电压、电动势和电阻是稳恒电
流的几个基本理量,欧姆定律正反映了这些量之间的相互
关系;电功、电功率、焦耳定律都和欧姆定律有着密切联系,研
究了电能与其他形式能的转化,有重要的实际意义;串联电路
和并联电路是电路连接的基本形式,掌握串、并联电路的基本
特点,是应用欧姆定律进行电路计算的关键,这些知识都很
重要,是研究直流电路的基础和工具,电池组在生活、生产和
实验中都常用。学生应该知道并联电池组和串联电池组的
特点,并能根据实际需要选用和组成电池组。电阻的测量在
电学实验、业余无线电活动以及生活、生产中经常遇到。教材
讲解电阻的测量主要是综合运用本章学过的知识,以提高学
生运用知识的能力。值得说明的是,电动势的概念尽管重要,
但由于这个概念比较复杂,限于学生的知识水平,中学阶段很
难讲清楚,不要求涉及电源内部的情况和电动势是怎样产
生的。

这一章的教学要求是:
\begin{enumerate}
\item 了解电流的产生条件,理解电流强度的概念,理解电
阻和电阻率的概念,掌握电阻定律。巩固掌握欧姆定律。
\item 理解电功和电功率的概念,掌握电功和电功率的公
式,掌握焦耳定律,了解电功和电热的关系
\item 巩固掌握串联电路和并联电路的特点,会分析、解决
直流电路中的混联问题。
\item 了解电动势的概念,巩固掌握闭合电路的欧姆定律。
掌握串联电池组及并联电池组的计算和应用,了解混联电池
组的应用。
\item 理解用伏安法和欧姆表法测电阻的原理。
\end{enumerate}

\section{教学建议}
本章教学的重点是闭合电路欧姆定律,难点是电动势概
念以及电动的几个公式的意义和运用条件。

由于全章教材内容具有鲜明的实践性,教学中应当加强
演示实验和学生实验,培养学生的实验能力。要特别注意教
给学生分析问题的思路和方法,培养他们的思维能力及应用
物理规律解决实际问题的能力。


\subsection{第一单元}
这一单元主要复习初中教材的电流强度、电阻、以及欧姆
定律,并在复习的基础上应用电场理论适当予以深化、提高。
同时补充学习电阻定律、电阻率等新知识。初中受学生知识
水平和思维能力发展的限制,只能从生活经验及实验事实出
发讲得比较简单,考虑到不学好这些概念、规律,很难进一步
学好其他电学知识,而在学过电场知识以后又可能讲得更深
入一些,因此教材仍把它们作为重要内容认真讲述。教学中
切不可因为学生已有一定的基础而掉以轻心,既要重视突出
知识的要点,又应当针对学生容易出错的问题,采用适当的教
学方法予以强调。

\subsubsection{电流}

电流产生的条件可按教材讲述的层次,利用课
本图7.1和图7.2, 逐步引导学生认识:
\begin{enumerate}
\item 形成电流
的条件是既要有能自由移动的电荷-自由电荷,又必须在
导体中建立电场.    \item 由于导体内存在自由电荷,所以导体中
存在持续电流的条件是保持导体两端有电势差。电源的作用
正是为了保持电路两端的电势差,使电路中存在持续电流。
\item 在学完稳恒电流与直流电的区别后应明确:要得到大小、
方向都不随时间变化的稳恒电流,导体两端的电势差必须保
持恒定。
\end{enumerate}

教材指出“自由电子除了做无规则的热运动外,还要在电
场力的作用下做定向移动,”对这段叙述,学生往往缺乏清晰
的认识,应当指出,自由电子作定向运动时,其无规则热运动
并未消失。它们一方面继续作无规则热运动,另一方面又沿
某一方向定向运动,即在无规则热运动速度的基础上叠加了
一个定向移动速度。为了避免学生从生活经验出发认为自由
电子定向移动速率非常大的错误,可以把第八章中关于自由
电子热运动的平均速率,定向移动速率和电场传播速率的数
量级分别为105$\ms$、$10^{-5}\ms$,$3\x10^8\ms$提前介绍.强
调电流传导速率实际上是电场传播速率,而不是自由电子定
向移动速率。

在讲解电流强度的概念及电流方向时,可以提出一些问
题让学生讨论,如:
\begin{enumerate}
\item 当通过不同横截面积的几根导线的电
流强度相等时,在相同时间内通过各导线横截面的电量跟截
面积大小有无关系?为什么?    
\item 为什么说导体中电流的方向总是从电势高的一端流向电势低的一端?
\item 电解液中电流强
度的大小和方向怎样确定?通过讨论加深学生对这部分知识
的理解,澄清一些模糊认识。
\end{enumerate}


\subsubsection{欧姆定律}

电阻是电路中的三个基本物理量之一,正
确理解电阻的物理意义对理解和掌握欧姆定律有重要作用。
因此,教材首先对电阻的概念作了认真的剖析。$R=U/I$
是电
阻的定义式,表明了一种量度电阻的方法。但学生在学习中
往往出现以下错误,如“电阻跟电压成正比,跟电流强度成反
比。”“既然电阻是表示导体对电流的阻碍作用的物理量,所以
导体中没有电流时导体就不存在电阻。”教学中应当有针对性
地将这些问题提出来讨论,以阐明$R=U/I$
的物理意义,纠正学
生的错误认识。要强调对金属和电解液电阻R是一个只与导
体本身有关的量。当导体的材料、粗细、长度、温度等因素一
定后,导体的电阻就确定了。电压和电流变化了,它们的比值$U/I$
即$R$是不会变的。同时应当指出,不管有无电流通过导体,
反映导体对电流阻碍作用大小的电阻$R$是客观存在的,只不
过当有电流通过导体时,这个作用才具体表现出来。

在进行欧姆定律$I=U/R$
的教学时,应通过具体题目的分
析,反复强调公式中的$U$、$I$、$R$是表示同一段电路的三个物
理量,决不能把属于不同电路的$U$、$I$、$R$值代入公式计算。

伏-安特性曲线是说明物质导电特性常用的一种表示方
法,在电工学和无线电电子学中有着广泛应用,要让学生对
此有所了解。应引导学生认识金属导体和电解液的伏-安特
性曲线是通过坐标原点的一条直线,具有这种性质的电阻叫
线性电阻,直线的斜率等于导体电阻的倒数。因此,通过比
较直线斜率的大小可以判断电阻的大小,但是有的导体和元
件,如热敏电阻、晶体二极管、真空二极管等的伏-安特性曲
线都不是直线而是特殊形状的曲线。这些导体和元件的电阻
就不是线性电阻而是非线性电阻了。它们的电阻$R$不是一个
常量,其大小与电压、电流的值有关。所以,它们不遵从欧
姆定律。

从线性电阻和非线性电阻的伏-安特性曲线的分析,我
们可以看出,尽管欧姆定律$I=U/R$
和电阻的定义式$R=U/I$
从数学形式上并无本质不同,但是物理意义却有区别。线性
电阻遵从欧姆定律,非线性电阻不遵从欧姆定律,然而,无论
线性还是非线性电阻都可以用$R=U/I$
来定义。这个问题只要
点到就行,不必展开讲解。

\subsubsection{电阻定律和电阻率}

电阻定律是一个实验定律.由
于在初中只作过定性实验,为了增加学生的感性认识,熟悉运
用实验总结物理规律的研究方法,在教学中有必要通过定量
实验得出电阻定律$R=\rho\ell/S$.
并且说明电阻定律不是对所有
导体都适用的电阻定义式,它适用于粗细均匀的金属导体及
浓度均匀一致的电解液。它定量揭示了这类导体的电阻由自
身的哪些物理条件决定,也反映了这类导体的电阻大小跟加
在导体两端的电压和通过的电流强度无关。

材料的电阻率$\rho$是描述材料导电性好坏的一个物理量,
可按教材安排的顺序分以下几步讲解.
\begin{enumerate}
\item 从分析电阻定律$R=\rho\ell/S$
中比例系数$\rho$的意义入手,讲清电阻率的意义.
\item 
强调指出电阻率与导体的大小、形状无关,仅取决于导体材料
的性质和导体所处的条件(如温度).公式$\rho=RS/\ell$
表示材料
的电阻率在数值上等于这种材料制成的长1m,横截面积
$1{\rm m^2}$的导体的电阻.
\item 说明在国际单位制中电阻率单位为
什么是欧姆·米。
\item 引导学生讨论电阻和电阻率有什么区
别,使他们认识到电阻率反映物质对电流阻碍作用的特性,
电阻则反映物体对电流阻碍作用的特性,因此,前者仅与物
质种类有关,后者却不但与构成物体的物质种类有关还与物
体的长度、横截面积有关。教材练习二第3题有助于学生理
解这个问题,可安排课堂讨论.
\item 指导学生阅读教材180页
的表列数值,使他们对常用金属的导电性能强弱有一个清楚
的认识。
\end{enumerate}

各种物质的电阻率都和温度有关,因此,各种导体的电阻
也都随温度变化。实验表明,所有纯金属的电阻率都随温度
升高而增大。为了加强教学的直观性,可以增加如下一个演
示实验:把一个废白炽灯泡的玻璃外壳小心去掉,取其内部一
段几厘米长的钨丝接在灯芯的灯丝支架上,与一个1.5伏电
源,一个安培表接成串联电路,测出电流强度,然后用酒精
灯把灯丝烧红,将看到电流强度明显减小。从而定性说明温
度越高,金属电阻率越大。精确实验证明,当温度变化范围不
大时,电阻率跟温度近似有如下线性关系$\rho=\rho_0(1+\alpha t)$. 式
中$\rho$和$\rho_0$分别是温度为$t$和$0^{\circ}{\rm C}$时材料的电阻率,$\alpha$为材料的
温度系数.多数纯金属的$\alpha$值近似等于$4\x10^{-3}/$度.对电
阻率跟温度的这个定量关系式,一般不宜在课堂上讲授,如:
果教师认为有必要介绍,也只需让学生知道就行,不要作过多
的阐述。但是,应当向学生指出:今后在涉及电阻值的计算
时,为使问题简化,如果没有特殊要求,我们通常不考虑温度
对电阻率的影响。

关于超导体的讲授,可引用后面参考资料提供的材料,适
当充实教学内容,以开阔学生眼界,激发他们进行科学探索的
志趣。

\subsection{第二单元}
这一单元仍属复习提高性质,目的在于使学生从电场力
做功和电势能变化的角度加深对电流做功的理解,搞清楚电
路中的能量转化关系。

\subsubsection{电功和电功率}

根据教材的逻辑顺序,从复习电场力
移动电荷做功$W=qU$结合电流强度的定义式$I=q/t$
推导出
电流做功的计算式$W=UIt$; 从复习电场力做功与电势能变
化的关系,用能量转化的观点阐明电流做功的实质,学生理解
起来并不太困难,但是他们往往对电流是怎么做功实现能量
转化的具体物理过程提出疑问。限于学生的现有知识水平,
一时不可能讲清。因此,建议最好先以能量转化关系明显,学
生容易接受的电炉、电动机等实际用电器为例,讲清通过电场
力移动电荷做功,电势能减少,电能转变为内能、机械能等,再
概括为一般结论:电流通过用电器做功的过程实际上是电能
转化为其他形式的能的过程。电流做了多少功,就有多少电
能转化为其他形式的能。

讲解电功率的概念时,应强调$P=UI$是电功率的一般表
达式,适用于任何用电器,表示用电器消耗的全部电功率,为
学习下一节内容奠定基础,要针对学生的常见错误,着重讲清
用电器的额定功率和实际功率,一方面通过实例的计算、分
析,具体说明它们的区别.如举这样的例题:把标有“220V
100W”的灯泡接到220伏的电路中,过灯丝的电流多大?
灯泡的功率多大?如果接到110伏电路中电流和功率又分别
是多少?假定灯丝电阻不变。另一方面可通过演示某一灯泡
在额定电压(需用伏特表观察)、低于额定电压、适当高于额定
电压等三种情况下的实际功率(亮度),以鲜明的直观加深学
生的印象,也可采用讨论式进行教学。首先提出以下两个问
题供学生看书讨论.
\begin{enumerate}
\item 公式$P=UI$表示的物理意义是什
么?讲一个用电器的电功率大,表示什么意思?   
 \item 什么叫用电
器的额定功率?什么又叫用电器的实际功率?标有“220V
100W”的灯泡是否一定比标有“220V40W”的灯泡亮?为什
么?在什么情况下,前者才会比后者亮?
\end{enumerate}
然后师生共同归纳这
部分知识的要点:电功率的概念及数学表达式,额定功率及实
际功率,最后通过上述演示实验强化用电器的额定功率和实
际功率的区别。

\subsubsection{焦耳定律}

在讲解焦耳定律时,应指出焦耳定律是反
映电流热应的定量规律,在国际单位制中,热量和功都用
焦耳作单位,故焦耳定律的数学表达式为$Q=I^2Rt$. 但是,如
果热量采用卡作单位,$I$、$R$、$t$用安培、欧姆、秒作单位,上式
两边的单位就不统一,需要通过热功当量$1{\rm Cal}=4.2{\rm J}$改写
为$Q=0.24I^2Rt$. 还应当强调$Q=I^2Rt$是焦耳定律的基本形
式。无论对任何电路,只要有电阻$R$存在,由电流热效应产生
的热量都可以通过这个公式计算,只有在纯电阻电路中,由
于$U=IR$, 焦耳定律才能表示成$Q=U^2t/R$
或$Q=UIt$的形式。

混淆电功与电热的概念,把$W=IUt$和$W=I^2Rt$、
$W=U^2t/R$
等同看待,是学生的一个常见错误,因此,阐明电功
和电热的关系是本节教学的难点。具体教法可采用理论分析
与实验演示相结合的形式。

首先从能量转化和守恒的观点分析:
\begin{enumerate}
\item 对只含纯电阻用
电器(如电灯、电炉等)的电路——纯电阻电路,电流做的功等
于$UIt$, 电流热效应产生的热量等于$I^2Rt$. 由于这时电路两
端的电压$U=IR$, 因此$W=UIt=U^2t/R=I^2Rt=Q$, 即电功等-
于电热,电能全部转化为内能.
\item 对于包含电动机、电解槽
等用电器的非纯电阻电路,尽管电功仍为$UIt$, 电热仍为
$I^2Rt$, 但它们的能量转化关系跟纯电阻电路不同.在电动机.
里,电能$\to $机械能$+$内能;在电解槽里,电能$\to $化学能$+$
内能。说明电能除部分转化为内能外还要转化为其他形式能
量.因此,$UIt>I^2Rt$, 即电功大于电热.这时电路两端电压
$U>IR$.
\item 综合上述分析得出结论:在纯电阻电路中$W=UIt$
和$W=I^2Rt$等效,电功等于电热;在非纯电阻电路中电功大
于电热,只能用$UIt$和$I^2Rt$分别计算电功和电热.并通过
教材最后一段的例子用具体数据说明这个结论.
\item 向学生
指出用电器消耗的全部电功率$P=UI$跟用电器因发热而消
耗的功率$P=I^2R$也存在着上述区别,同样不能混为一谈.
\end{enumerate}

其次,通过实验演示(见实验指导,演示实验1)以加深学
生的印象,帮助他们确信教材最后一段所举例子的真实性,直
观地认识电功与电热的区别。

第一、二单元的教材,大部分属于复习、巩固初中知识的
内容。实践证明采用指导学生自学的教学方式可以取得较好
的效果,如果过去对学生的自学能力培养不够,可在每节课
通过布置自学提纲或思考、讨论题等形式指导学生自学、讨
论,但要注意及时归纳小结。如果学生的自学习惯已基本形
成,不妨分单元按以下四个环节组织教学:
\begin{enumerate}
\item 教师概要介绍
全单元知识要点和知识间联系,提出自学要求.    
\item 学生自学,写出读书笔记,用课本上的练习题自我检查学习效果。教
师则个别指导、答疑,并收集学生中出现的问题.    
\item 教师根据学生的问题和教学重点,有针对性地讲解.    
\item 讲练结合,辅之必要的小组讨论,以达到复习、巩固的目的。
\end{enumerate}


\subsection{第三单元}
这一单元是前两个单元中基本概念、基本规律的具体应
用。学生在初中通过实验认识了串联电路和并联电路的基本
特点,学习了这两种电路总电阻的计算,高中进一步研究它们
的电压分配、电流分配、功率分配等。并且通过例题讲述了混
联电路的实际意义和分析、计算方法,教学中应突出分析的
思路,即首先搞清楚电路各部分间的串、并联关系,然后运用
有关的知识进行计算。应当注意,要求学生解决的混联问题
应该是比较简单和有实际意义的。所谓比较简单,是指电路
各部分间的串、并联关系比较容易辨认出来;所谓有意义,是
指电路应是实际中存在的或者是实际电路的简化或抽象。让
学生去解那些实际意义不大的繁难题目,教学中应该避免,至
于那些不能最终化为串、并联的电路,需要用基尔霍夫定律才
能求解的复杂电路(网路),更是不必涉及。

\subsubsection{串联电路}

在初中是通过实验得出串联电路的电流、
电压特点的。现在联系前一章学过的电场知识和稳恒电流的
特点作进一步说明,以加深学生的理解。

在讲解串联电路两端的总电压等于各部分电路两端的电
压之和时,可从电场力移动电荷做功跟电势能变化的关系出
发,分析为什么电流通过串联电路各电阻时,沿电流方向每通
过一个电阻,电势要降低一个数值,并结合教材图7.6说明
\[U=U_1+U_2+U_3\]

在上述的基础上,运用欧姆定律不难推出串联电路的总
电阻,电压分配、功率分配等三个具体规律。教学中要着重讲
清并让学生掌握推导的思路。只有真正理解了这些规律的意
义及相互关系,才不会孤立地死记硬背,也才能在解决实际问
题时灵活运用。

进行这部分知识的教学,应注意以下三个问题。

\begin{enumerate}
    \item 讲清“等效”的意义.所谓等效,就是指作用效果相
同。例如在力学中,某一个力对物体的作用效果与另外几个
力对物体的共同作用效果相同,我们就可以认为这个力是那
几个力的等效力,又叫那几个力的合力,用等效力去代替原
来那几个力,在处理某些问题时会更简便。在串联电路中,等
效电阻就是指串联电路的总电阻,它在电路中起作用效果
跟原来几个电阻的共同作用效果相当,在电路计算中,我们
常用总电阻去代替原来的几个电阻,使问题简化。还应指出,
等效方法十分有用,在今后的学习中还会遇到,如电路分析中
的等效电路。
\item 讲解滑动变阻器作分压器使用时,最好配合实验演
示,以加强教学的直观性,可按教材图7.7接线,并在$cd$间
接上一个伏特表,尤其要对照电路图和实物突出滑动变阻器
的连接方法。当改变滑动端在两个固定端之间的位置时,让
学生观察伏特表示数的变化,再分析发生这种变化的原因,阐
明分压器原理。建议在讲完并联电路后,安排一次学生分组
实验“研究串联电路和并联电路”通过实验巩固串联、并联
电路的规律,进一步训练学生正确地使用安培表、伏特表、滑
动变阻器、电键等基本器材,为后面的学生实验作准备,实验
应把区分滑动变阻器的两种接法(制流电路和分压电路)作为
重要内容,具体电路如图7.1所示,比较这两种电路中滑动变
阻器的连接有什么不同?灯泡两端电压变化的范围有何不同?

\begin{figure}[htp]
    \centering
  %  \includegraphics[scale=.7]{fig/7-1.png}
    \caption{}
\end{figure}
\item 对串联电路的电压分配和功率分配关系可通过下面
的实验定性验证,把电阻值不同的灯泡(如“220V100W”和
“220V25W”)串联起来接入照明电路,每个灯泡两端各并联
一只交流伏特表,将使学生观察到阻值大的灯泡两端电压
大,消耗的功率多(灯泡亮);阻值小的灯泡两端电压小,消耗
的功率少(灯泡暗)。教学中还应通过实例介绍运用串联电路
电压分配公式和功率分配公式的比例解题法的优越性。
\end{enumerate}

\subsubsection{并联电路}

推导出并联电路的总电阻计算公式以后,
教材根据
\[\frac{1}{R}=\frac{1}{R_1}+\frac{1}{R_2}+\cdots+\frac{1}{R_n}\]
直接得出结论:并联电路的
总电阻比每一个电阻都小。可引导学生从不同角度去认识这
个问题,以培养他们的发散性思维。一方面可根据电阻定律
$R=\rho\ell/S$
定性说明:几个电阻并联相当于增大了导体的横截
面积,总电阻当然减小。另一方面直接用总电阻计算公式讨
论.如果两个电阻$R_1$和$R_2$并联,由
$\dfrac{1}{R}=\dfrac{1}{R_1}+\dfrac{1}{R_2}+\cdots+\dfrac{1}{R_n}$
得总电阻
\[R=\frac{R_1R_2}{R_1+R_2}\]
再改写为
\[R=\frac{R_1}{R_1+R_2}R_2\quad \text{或}\quad R=\frac{R_2}{R_1+R_2}R_1\]
因为$R_1+R_2>R_1$, $R_1+R_2>R_2$, 所以$R<R_1$, $R<R_2$. 由此可类推多个
电阻并联后它们的总电阻必小于其中任何一支路的电阻。但
应注意防止学生把
\[\frac{1}{R}=\frac{1}{R_1}+\frac{1}{R_2}+\frac{1}{R_3}\]
误写为
\[R=\frac{R_1R_2R_3}{R_1+R_2+R_3}\]
通过以上分析再次强调无论串联电路还是并联电路,总电阻
实际上是指这个电路的等效电阻。

并联电路的电流分配关系和功率分配关系,可以用下面
的实验定性地验证。把阻值不同的灯泡(如“220V100W”
和“220V25W”)各串一个交流安培表后并联在照明电路
里,可观察到电阻值小的灯泡亮(功率大),通过的电流强;阻
值大的灯泡暗(功率小),通过的电流弱,同时应通过实例让
学生掌握运用并联电路的电流分配式和功率分配式的比例
解题法。

最后,应要求学生把串联电路和并联电路的特点作一比
较,使他们对串、并联电路能有一个全面的认识。

\subsubsection{分压和分流在伏特表和安培表中的应用}

这部分内
容是串联电阻分压和并联电阻分流的具体应用。因为学生已
具有一定的知识基础,为了使他们学习更主动,课堂教学的形
式最好活跃一点。例如采用教师引导下的讨论式教法。

首先指导学生观察大型演示用电流表,让他们对电流表
的大体结构以及测量电流、电压的基本原理有一个初步了解。
要注意强调电流表的内阻$R_g$和满度电流$I_g$仅决定于表的
内部结构。因此,对给定的一个电流表,它的这两个参量及最
大允许电压(满度电压)$V_g=I_gR_g$就是确定不变的。

其次,启发学生对下列问题展开讨论.
\begin{enumerate}
    \item 为什么电流表不能测量较大电压和较强电流?
    \item 如果要用电流表去测量较
大电压,并且能从刻度盘上直接读出待测电压值应当怎么办?假设有一个电流表,内阻$R_g=1000$欧,满度电流$I_g=100$微
安,要把它改装为量程是3伏的伏特表,试具体说明改装的方
案.
\item 如果要用电流表去测量较大电流,并且能从刻度盘上
直接读出待测电流值,应采取哪些措施?假如用上一问题中的电流表改装为量程是1安的安培表,试具体说明改装的方案.
\item 试分别概括电流表改装为伏特表和改装为安培表的原理及具体方法。
\end{enumerate}
通过学生讨论得出结论后,指导他们阅读教材
196页到198页的有关部分,检验,完善自己的看法,加深对问
题的理解。

然后,教师归纳小结,着重讲明以下几点.
\begin{enumerate}
\item 将电流表
改装成伏特表或改装为安培表,需分别串联一只分压电阻或
并联一只分流电阻,串联或并联的实际电阻值应根据分压原
理或分流原理,根据欧姆定律及需要扩大的量程进行计算,要
改装的伏特表量程越大,需串联的分压电阻值就应越大;要改
装的安培表量程越大,需并联的分流电阻值就要越小。要求
学生掌握的是改装的原理,分析、计算的方法,不必死记具体
计算公式,包括练习六第(2)题的关系式.
\item 电流表和串联
的分压电阻所构成的整体才叫伏特表,因此,伏特表刻度盘
上标出的伏特值,不表示加在电流表上的电压,而是直接表示
加在伏特表上的电压,在用伏特表测量一段电路的电压时,
伏特表指针的示数表示伏特表两端的电压,也表示待测电路
两端的电压。电流表和并联的分流电阻所构成的整体才叫安
培表,因此,安培表刻度盘上标出的安培值,不表示通过电流
表的电流,而是直接表示通过安培表的电流.
\item 电流表改装
为伏特表和改装为安培表的原理及计算方法同样适用于扩大
伏特表的量程和扩大安培表的量程。
\item 由于伏特表的内阻
很大,安培表的内阻很小,把它们接入待测电路时,一般对
原电路的电流、电压的分配影响不大。因此,题目如果没有明
确要求考虑伏特表的内阻和安培表的内阻时,通常可以认为
伏特表的内阻无穷大,安培表内阻为零。
\end{enumerate}

最后,还可以提出一些有启发性的问题供学生思考、讨
论。例如:用安培表和伏特表测量一段电路的电流强度和电压
时,若不慎将安培表并入电路或将伏特表串入电路,将会产生
什么后果?又如:试定性说明由于安培表、伏特表有内阻,将对
实际测量值带来什么影响?

\subsubsection{电路的分析和计算}

本节教材运用前面已学过的串
联电路和并联电路的知识,通过例题分析,介绍简单混联电路
的分析计算方法。对中学生来说,主要是教会他运用知识
分析问题的基本思路及基本方法,以利于巩固知识和培养能
力,不宜把过多力量用来教给学生某些具体的解题方法和
技巧。

掌握电路分析的方法是进行电路计算的基础。可以通过
举例,介绍识别简单混联电路中各部分间串、并联关系的基本
方法。如根据电流的分支与汇合,判断各电阻的串、并联关
系。若电流通过各电阻时没有分支,则这些电阻为串联;若电
流有分支,则从分流点到汇流点之间的各支路为并联。如果
电路中没有电源,可假设一个电流方向去判断。不过,应控制
教学深度,不宜要求学生掌握串、并联关系较难识别的电路
图,以免脱离教学要求,加重学生负担。

讲解教材上的两个例题,要重视启迪学生的思维。例如,
可采用教师启发讲解和学生讨论、练习相结合的教学方式。

在讲例题1时,教师首先应引导学生分析题意,画出电路
图,弄清电路中所有灯泡、输电线之间的串、并联关系。接着,
可依次提出以下问题启发学生积极思考,学习分析问题的方
法:怎样求每盏灯的实际功率1怎样求每盏灯的电阻?怎样求
每盏灯的实际电压?怎样求输电线的电压?怎样求输电线上的
电流强度?怎样求整个电路的总电阻?如何才能求出各灯泡并
联的电阻?分析问题的思路清楚了,可让学生自己进行计算。
然后,教师根据学生的计算结果提出问题供他们讨论。例如:
在电压不变的条件下,为什么多开灯比少开灯时每盏灯消耗
的功率小?在这两种情况中电源输出的功率是否相等?为什
么?使学生通过例题明确当部分电路的电阻发生变化时,必
然会引起整个电路的电流、电压、电功率发生变化的道理,懂
得这时应对电流、电压、电功率的分配重新计算。

讲解例题2时,先应引导学生认识接入伏特表后电路发
生了什么变化,认识伏特表的读数表示伏特表与10k$\Omega$电阻组
成的并联电路两端的电压.启发学生从例题1的结论-部
分电路的电阻值发生变化将引起整个电路的电压重新分配出
发,理解接入伏特表后,$a$、$b$间电压发生变化的原因,接着
指导学生自己根据串、并联电路的特点和欧姆定律进行具体
计算。然后再提出问题让学生讨论。伏特表接入电路时,测量
值比真实值偏大还是偏小?产生这种现象的原因是什么?为什
么选用内阻大的伏特表进行测量比较准确?教师应对学生讨
论的结果作出小结,使学生有一个完整的认识。伏特表接入
电路时,由于其内阻与被测电路并联的等效电阻小于被测电
路的电阻,若电源电压和另一个串联电阻不变,整个电路的电
压分配将发生变化,使并联电路分得的电压减小。
因此,伏特表的测量值将小于被测电路电压的真实值。伏特表的内阻比
被测电路的电阻大得越多,其并联等效电阻就越接近被测电
路电阻。使整段电路上电压分配的比例关系变化越小,伏特
表测量的误差就会越小。

综合两道例题的分析,最后应归纳进行电路计算的基本
思路:
\begin{enumerate}
\item 弄清电路中各部分之间的串、并联关系.
\item 根据题
目的已知条件对电路的局部和整体进行分析,从串、并联电路
的特点出发,找出该段电路和相邻电路的电流、电压关系。特
别要注意分析当电路连结方式改变或某个外电阻发生变化
时,电流、电压、功率分配的变化情况.
\item 正确选用基本公式
列出方程求解。
\end{enumerate}

\subsection{第四单元}
这一单元在引入电动势概念的基础上,把欧姆定律扩展
到包括电源在内的整个闭合电路,着重讲述闭合电路的欧姆
定律及其应用。这是本章的新知识,也是重点。对学生的分
析能力,推理能力要求较高。教学中应加强实验演示,强调物
理思考,注意训练分析问题的方法。

\subsubsection{电动势、闭合电路的欧姆定律} 

电动势是电学中的
一个重要概念,但它比较抽象难懂。为了减少学生学习的困
难,教材只要求学生知道电势反映了电源的一种特性,它的
大小等于外电路断开时两极间的电压,也等于外电路接通时
内、外电路上的电压之和,还从能量转化的角度讲解了电动势
的物理意义。教学中要掌握好分寸,不要在理论上补充、加深。

教材图7.25所示的实验有承上启下的作用,一是介绍
内电路、内电阻的概念;二是分析电源跟外电路接通后,两极
间电压小于电源电动势的原因;三是提出内、外电路上电势降
落之和应等于电源电动势的设想。

教材图7.26所示实验有助于学生进一步理解电动势的
意义,为讲述闭合电路欧姆定律奠定基础,因此,做好这个实
验(详见后面演示实验部分)是本节教学成功的关键,学生的
一个常见错误是认为接在电源两极间的伏特表的示数表示电
源内电压,在介绍实验装置时,可针对这个问题强调电源内
电压的测量方法。

在从能量转化角度来阐明电动势的物理意义时,应当进
一步指出:电源电动势是表示电源本身属性的物理量,即反映
电源把其他形式能量转化为电能的本领的大小,电动势越
大,表示电源把其他形式能量转化为电能的本领越大。因此,
对一个给定的电源,电动势有确定的数值,其大小只由电源的
性质和结构决定,而与外电性质以及是否接通没有关系。可
提出这样的问题启发学生思考、讨论.如:电动势为1伏,电
势降落为1伏,电势为1伏,各表示什么意思?这三个物理量
有什么区别?使他们弄清为什么$U+U'$不叫电动势,而只是
数值上等于电动势的道理。

闭合电路欧姆定律是关于电路的一条重要定律,要引导
学生理解它的物理意义,要指出闭合电路欧姆定律跟部分电
路欧姆定律的适用条件不同,前者适用于包括电源的整个闭
合电路,后者适用于不含电源的某一部分电路。应当通过教
材208页例题的分析引导学生认识:当外电路发生变化时,将
引起电路各部分的电流、电压重新分配。但是,电源电动势和
内电阻是保持不变的。这个特点对解电路计算问题十分重要。

在讲过闭合电路欧姆定律之后,应引导学生对整个电路
中能的转化情况作全面认识,学会从能的转化观点分析有关
电路的问题,如理解$I\mathcal{E}=IU+IU'$的物理意义,了解电源的
总功率、输出功率和内电路消耗的功率的概念和它们间的
关系。

\subsubsection{路端电压}

路端电压随外电阻而变化是一个重要问
题,讨论这个问题对巩固和运用闭合电路的欧姆定律很有好
处,所以教材把它单列一节讲述。

应用演示实验,给学生以鲜明的直观印象是十分重要的。
演示实验可用教材7.26的装置,在外电路上再串联一个安培
表,也可用教材图7.30所示电路演示.如果由于干电池内阻
小,实验效果不明显;不妨用一段阻值约十几欧的电炉丝把两
节干电池串联起来作为电源,并且使用的滑动变阻器的阻值
不要大(以0—50欧为宜).

在教学的具体安排上,先研究路端电压变化的一般规律,
再从一般到特殊,讨论断路、短路两种情况,讲解一般规律
时,一种办法是先通过实验演示,从实验现象的观察中总结出
路端电压变化的规律。再从理论上分析原因;另一种办法是
先从理论上用闭合电路欧姆定律分析路端电压的变化规律,
然后用实验来验证,并把理论分析的整个物理过程完整地表
现出来进行归纳小结。两种讲法各有其特点,但无论采用哪
种方法都应当引导学生认识以下几个问题.
\begin{enumerate}
\item 电源有内电
阻是路端电压变化的根本原因.
\item 不能用$U=IR$讨论路端电
压的变化。这是因为在闭合电路中决定电流强度的不是$U$和
$R$, 而是$\mathcal{E}$、$R$和$r$. 在电源确定的情况下,$\mathcal{E}$、$r$不变,由
\[I=\frac{\mathcal{E}}{R+r}\]
可看出外电阻$R$的变化直接改变电路中的电流,引
起内电压$Ir$的变化,从而影响路端电压,用$U=\mathcal{E}-Ir$来讨
论,容易看出路端电压的变化情况,若用$U=IR$来讨论,则
由于$R$增大,$I$就减小,到底$U$如何变化,无法判断。

\item 电源的外特性曲线($U$-$I$图象)是一条直线,它反映了整个闭合电
路中路端电压随电流的增大而线性减小,直线的斜率表示电
源的内电阻,直线在$U$轴上的截距表示电源电动势的大小。因
此,$U$-$I$图象反映了电源的特性。它与欧姆定律一节讲述的
反映导体导电性能的伏安特性曲线($I$-$U$图象)是有区别的。
\item 外电路某部分电阻发生变化时,判断路端电压及整个电路
的电流、电压分配的变化情况,可遵循以下步骤:弄清外电路
的串、并联关系,分析外电路总电阻怎样变化;由$I=\dfrac{\mathcal{E}}{R+r}$
确定闭合电路的电流强度如何变;由$U=\mathcal{E}-Ir$确定路端电压的
变化情况;用欧姆定律$U=IR$及分流、分压原理讨论各部分
电阻的电流、电压变化情况。让学生掌握从部分同整体的关
系上来分析电路的方法。
\end{enumerate}



断路和短路是路端电压随外电阻变化的两种特殊情况。
可提出以下问题引导学生通过讨论,自己得出结论.
\begin{enumerate}
\item 断路
和短路有什么不同?
\item 为什么用伏特表测出断路时的路端电
压并不准确等于电动势?怎样才能提高测量的精确程度?
\item 
有人说:断路时$I=0$, 由$U=IR$得出路端电压$U=0$, 这种说
法是否正确?为什么?
\item 短路时电源有什么危害?这时电源电
动势是否为零?在$U$-$I$图中怎样确定短路电流的大小?
\end{enumerate}


\subsubsection{电池组}

讲解这个问题可通过实例从使用电池组的
必要性出发引入新课。应当指出,电池存在允许通过的最大
电流是由于电源存在内电阻,电流流经内电阻会发热,若电
流太强,温升过高,将损坏电池。

讲解串联电池组的总电动势$\mathcal{E}_{\text{总}}=n\mathcal{E}$时,可把理论分析与
实验演示结合起来,边演示边分析.按教材7.33所示,用伏
特表测定每个电池的电动势,示数均为$\mathcal{E}$。说明断路时路端
电压等于电源电动势,每一电池正极的电势比它的负极高$\mathcal{E}$。
再用导线把伏特表正接线柱与第一个正极相连,将与伏特表
负接线柱的导线分别与第一个电池的负极和第二个电池的正
极相连,可观察到伏特表示数相同,说明前一个电池的负极
与后一个电池的正极电势相同。接着依次把伏特表负极与第
二个电池负极,第三个电池正极……等相连,逐一观察分析,
得出结论。

对串联电池组的使用,要引导学生认识以下几点,第一,
使用串联电池组的目的是为了获得较大的电动势,使用的条
件是:用电器额定电流小于单个电池允许通过的最大电流。第
二,不要把某些电池接反.可按教材图7.34演示、分析.弄
清这个问题将有助于今后理解矩形线框在垂直于线框平面
的匀强磁场中平动的总电动势为零的道理。第三,不要把新
旧电池混合起来串联使用.因为一节干电池电动势为1.5伏
时,内阻约0.5欧,在使用一段时间后,随电动势减小,内阻将
增大,当电动势降到1.1伏左右,内阻甚至可增大到几百欧。
这样一来,在内电阻上损耗的功率过大,是很不合算的。

要启发学生从用导线连接起来的所有极板电势都相等这
一事实去理解为什么并联电池组的总电动势等于单个电池的
电动势,应当注意,在中学阶段,我们仅研究相同电池的并
联,不涉及电动势和内电阻不同的电池的并联问题。使学
生理解,使用并联电池组的目的是为了给用电器提供较强的
电流。使用的条件是用电器的额定电压低于单个电池电
动势。

如果用电器的额定电压和额定电流都大于单个电池的电
动势和允许通过的最大电流应当怎么办呢?在讲解混联电池
组时,可先提出这样的问题组织学生利用刚学过的串联和并
联电池组的知识思考、讨论。要启发他们认识,需根据用电器
的额定电压和单个电池的电动势来确定串联电池的个数;根
据用电器的额定电流和单个电池允许通过的最大电流来确定
并联的组数,从而连成混联电池组。可通过具体例子加深学
生对混联电池组的理解。

\subsection{第五单元}

这一单元运用稳恒电流的基本理论讨论电阻的测量问
题。测量电阻的方法很多,这里主要介绍伏安法、欧姆表和惠
斯通电桥,着重研究它们的测量原理,分析测量误差产生的
原因,并提出减小误差的方法。教学中要特别注意启发、诱导
学生应用已学知识主动地分析、解决遇到的新问题,借以提高
他们运用知识的能力。

\subsubsection{伏安法}

伏安法测电阻的原理是欧姆定律,学生在初
中已学过,本节教材着重于运用串、并联电路的知识去分析
伏安法测电阻的误差。应提醒学生,这里所讲的误差不是指伏
特表、安培表的精度及读数引起的误差,而是由于伏特表、安
培表存在内阻,把它们连入电路中不可避免地要改变电路本
身,给测量结果带来的误差。

首先应引导学生了解伏安法测电阻有安培表外接(如教
材图7.37甲)和安培表内接(如图7.37乙)两种连接方式.其
次,引导学生讨论:这两种连接方法测电阻产生误差的原因是
什么?在什么条件下采用哪种接法,测量误差较小?并通过归
纳小结,使学生有一个较完整的认识。

采用安培表外接方式测量电阻时,伏特表的示数反映了
待测电阻两端电压的真实值,但由于伏特表有内阻,安培表的
示数$I$并不表示通过待测电阻的电流的真实值$I_R$, 而是表示
通过待测电阻和伏特表的总电流,即$I=I_R+I_V$. 这时待测
电阻的真实值$R=U/I_R$
和测量值$R'=U/I$就存在差异。因为$I>
I_R$, 所以$R'<R$. 可见,这种接法必然使测量值小于待测电阻
的真实值。引起误差的根本原因在于伏特表内阻的分流作
用,导致安培表示数大于待测电阻的电流。如果$R_V$越大,其
分流作用就会越小,安培表示数将越接近待测电阻的电流,
测量误差就会越小。当$R_V\gg R$时,$I_V\ll I_R$, 这时可认为
$I=I_R+I_V\approx I_R$, 则$R\approx R'$. 因此当$R_V\gg R$, 即测量小电阻
时,宜用安培表外接法(详见参考资料),采用安培表内接法测
电阻时,安培表的示数反映了待测电阻的电流真实值。但安
培表内阻的存在使伏特表示数$U$表示待测电阻两端电压与安
培表上电压降之和$U=U_R+U_A$. 这时待测电阻真实值$R=U_R/I_R$
就不等于测量值 $R'=U/I_R$。
因$U>U_R$, 故$R'>R$. 可见这
种接法必然使测量值大于待测电阻真实值。引起误差的根本
原因是安培表内阻的分压作用,导致伏特表示数大于待测电
阻两端电压,若$R_A$越小,其分压作用越小,伏特表示数就接
近待测电阻两端电压,测量误差就会越小,当$R_A\ll R$时,
$U_A\ll U_R$, 这时可认为$U=U_R+U_V\approx U_R$, 则$R\approx R'$. 因此,当
$R_A\ll R$, 即测量大电阻时,宜用安培表内接法。

\subsubsection{欧姆表}

讲解欧姆表的原理和构造时,应当注意以下
几个问题.
\begin{enumerate}
\item 讲解中需要有演示实验配合,以加深学生印
象,并对欧姆表的使用方法有一个初步了解。    
\item 应引导学生
认识欧姆表的设计思路。用伏安法测电阻比较麻烦,不仅需用
伏特表和安培表,而且要对测量结果进行计算才能得到待测
电阻值。如果在测量时将所用电压固定,直接用电流表的示
数来表示待测电阻的阻值大小,就能大大简化实验过程。欧
姆表就是基于这个思想设计的.   
 \item 欧姆表的工作原理是闭
合电路欧姆定律$$I=\dfrac{\mathcal{E}}{R_g+r+R+R_x}$$
式中$R_g+r+R$为欧姆
表内电阻。要强调欧姆表应有内电源,要装上干电池才能使
用。在万用表欧姆挡,红、黑表笔尽管分别插入标有“$+$”、“$-$”
号的插孔中,但这个“$+$”、“$-$”号并不表示内部电源的正、负
极。这两个符号表示万用表无论在欧姆挡、直流电流挡还是直
流电压挡,电流都应该从“$+$”插孔流入,从“$-$”插孔流出,以
保证表头指针向顺时针方向偏转。
\item 关于欧姆表的刻度,教
材中没有涉及。教师可根据学生实际酌情处理。一般说来,可
讲到这个程度:第一,红、黑表笔短路$R_x=0$, $I=\dfrac{\mathcal{E}}{R_g+r+R}$
此时电流最大。可调整$R$使指针满度,即$I=\dfrac{\mathcal{E}}{R_g+r+R}=I_g$,
指针所指的表盘上的满度位置可定为0欧,红、黑表笔断开
$R_x=\infty$, $I=0$, 指针不动,这时指针在表盘上所指的位置可定
为“$\infty$”,表示电阻无穷大。因此,欧姆表的刻度与其他表的刻
度相反。第二,由$I=\dfrac{\mathcal{E}}{R_g+r+R+R_x}$
可以看出电流强度$I$与
待测电阻阻值$R_x$之间不是线性变化关系.所以,欧姆表表盘
刻度是不均匀的(见参考资料)。
\end{enumerate}

\subsubsection{惠斯通电桥}

惠斯通电桥的教学可以充分发挥学生
学习的主动性,在教师启发、引导下,多让学生活动。

首先引导学生回顾伏安法和欧姆表测电阻的误差原因、
然后提出设计任务,组织学生讨论。要求设计一个电路既能
避免伏特表分流、安培表分压的影响,又能消除电源电动势和
内电阻的变化对测量的影响,提高测量的精确程度,教师要
适时启发他们,在设计的新电路中应当没有伏特表和安培表,
也不能用电流的强弱来反映电阻的大小。逐步把学生的思路
引导到采用将待测电阻跟已知电阻相比较的方法上来,并指
出惠斯通电桥就是根据这一指导思想设计的。

在介绍惠斯通电桥电路图及电桥平衡条件的基础上,可
让学生自己利用电桥平衡条件和欧姆定律推导出确定待测电
阻的公式
\[R_x=\frac{R_2R_3}{R_1}\]

讲解惠斯通电桥测电阻的精确程度时,可组织学生议论:
影响惠斯通电桥测量精确程度的因素有哪些?理由是什么?然
后指出,要使测量结果精确,应选用准确程度高的电阻,如电
阻箱作为已知电阻;应选用表头灵敏度高的电流表作检流计。
在讲解滑线式电桥时应当把实物演示与讲述有机结合,

既分析构造和工作原理,又介绍操作程序和注意事项。要告
诉学生,教材图7.40所示电路图中的滑动变阻器有两个作
用:一是在开始实验时起限流作用,保护电流表;二是在实验
中通过减小它的电阻以提高$A$、$C$间电压,检验在电流表灵敏
度范围内电桥是否真正平衡,保证测量的精确度。但也要注
意,不能使流过各臂的电流太大以致各臂电阻发热,影响其阻
值的准确性,甚至烧坏桥臂。尤其要提醒学生,按下滑动触头
与电阻线$AC$只能瞬时接触,以免当$D$的位置跟恰使电桥平
衡的位置相距较大时,一开始就可能因电流过大使电流表损
坏。更不允许按下$D$后在$AC$上移动$D$的位置,这样才不致
破坏电阻丝的均匀性。

练习十一第5题和习题10、11题对培养学生解决实
际问题的能力很有训练价值,可视具体情况选择其中一、二题
作为课堂讨论,并用实验验证讨论结果。


\section{实验指导}
\subsection{演示实验}
\subsubsection{电功和电热}
将玩具电动机(1.5V—6V)、J2355型(0—50$\Omega$)滑动变阻
器、电键、大型演示用伏特表和安培表、干电池组或蓄电池组
(电源电动势视电动机规格确定).按图7.2所示电路连接.

先用手捏住电动机转轴使其不动,调节变阻器阻值让安
培表示数不要过大,记下此时安培表及伏特表示数$I_1$、$U_1$, 计
算出电动机电枢线圈的电阻$R=U_1/I_1$
松开手指让电动机转动,
可以看到安培表示数减小、伏特表示数增大,记下它们的值
$I_2$、$U_2$. 不难看出电动机两端电压$U_2>I_2R$。 通过实际计算还
可以看出,在相等时间$t$内,电功$U_2I_2t$大于电热$I_2^2Rt$.
\begin{figure}[htp]\centering
    \begin{minipage}[t]{0.48\textwidth}
    \centering
 %\includegraphics[scale=.8]{fig/7-2.png}
    \caption{}
    \end{minipage}
    \begin{minipage}[t]{0.48\textwidth}
    \centering
 %\includegraphics[scale=.8]{fig/7-3.png}
    \caption{}
    \end{minipage}
    \end{figure}

\subsubsection{闭合电路欧姆定律}
这个实验是要验证$\mathcal{E}=U+U'$. 由于普通蓄电池和伏打
电池的内阻很小,内电阻不易测出,因此做好这个实验的关键
在于增大电源内电阻。具体作法如下。

\paragraph{电源装置(图7.3所示)}

蓄电池式.在两个500毫升的烧杯内装入适量的浓
度为20\%的稀硫酸,再将盛满稀硫酸的U型玻璃管倒插入两
烧杯中(也可用过滤纸或浸透酸液的塑料海绵条,把两端分别
浸入两烧杯的酸液中,改变过滤纸或海绵条的数目,就能改变
电源内电阻),使两烧杯连通。取旧蓄电池极板或纯铅板两块
放在同一稀硫酸槽中组成内阻小的电源进行充电(这样可节
省充电时间)。待充电结束,把两极板分别放入两烧杯酸液内
作为电源正、负极,组成电源.它的电动势约2伏.

取两支废圆珠笔芯去掉笔尖,在其塑料管上每隔2—3毫
米用针扎一个孔,把它们分别捆在正,负极板上,作为插入探
针的针管。用能插入针管的铜丝作探针,要在测量内电压时
才插进针管。

伏打电池式。电源装置和上述蓄电池式相仿,不同的
是用铜板和锌板作电源的正、负板。为了减小极板的极化,可
在稀硫酸中加入5\%左右的重铬酸钾或高锰酸钾作为去极剂.
这个装置不用充电,使用起来也比较方便。但由于伏打
电池电动势约为1.02伏,为使实验效果显著,应把普通大型
演示用电表作适当改装,即在灵敏电流计上串一只适当的分
压电阻,使伏特表的满刻度电压为1伏左右.

\paragraph{注意事项}
\begin{enumerate}
    \item 所使用的两个示教用大型伏特表应有相同的准确度
和合适量程。
\item 连接电路应注意两个伏特表的正、负极不要
接反。特别要注意内电路,靠近负极板的探针电势高于靠近
正极板的探针电势,所以前根探针应与伏特表$V_2$的正极相
连,后根探针则应与$V_2$的负极相连.
\item 滑动变阻器(或电阻
箱)的最大阻值应大于电源内阻,否则外电压不易测出。
\item 实
验前应将铅板(或铜板、锌板)、铜丝擦去氧化物。
\item 实验后要
将全部零件从酸液中取出,(对伏打电池,更需立即取出),用
清水冲洗干净,晾干备用。
\end{enumerate}

\subsection{学生实验}
\subsubsection{测定金属的电阻率}
这个实验涉及电阻定律、欧姆定律等基本规律,是具有长
度测量和电学测量综合性质的实验。上述知识都是学生已学
过的,因此教材要求学生自选器材、自行设计来进行实验。教
学中可以通过一系列问题启发学生思考、讨论,形成完整的设
计方案,再实际操作,方案设计建议分以下几步:
\begin{enumerate}
\item 弄懂实验原理。可提出下列问题,让学生思考:要
测出一段金属丝的电阻率,应当测量哪些物理量?怎样求出
电阻率?

\item 初步拟订实验方案.考虑到学校使用电表的量程,
教材提出,要测出一段长约0.5米,直径约0.3毫米.阻值约
3欧姆的金属导线(这是指锰铜丝,如选用其他材料或其他规
格的金属丝作待测对象,教师应先初测一下,不可照搬教材上
的数据)的电阻率。要求学生自己提出一个设计方案,包括选
用哪些实验器材、画出实验电路图、简要说明实验步骤,启发
学生相互交流、讨论。对电阻的测量,学生一般会根据初中的
知识,提出用伏特表和安培表来测量。如学生还提出其他方
法,可让他们进行讨论,判断是否可行。

\item 完善设计方案.从提高实验精确程度出发,引导学
生统一认识。用米尺(最小刻度为毫米)量金属丝的长度,用
千分尺(或游标卡尺)测金属丝直径(应及时复习一下有关测
量的读数方法),用伏安法测电阻。应指出用安培表外接形式
测量金属丝电阻精确程度较高(其道理可视学生情况作简要
分析,或完全回避,指出今后再学),可引导学生讨论:根据金
属丝电阻的大概数值结合教材提出的注意事项-电路中电
流不宜过大(要启发学生弄懂为什么),应选电动势为多大
的电源?量程为多大的伏特表和安培表?采用什么办法可以
控制电流不致过大?最后归纳学生意见并画出实验线路如图7.4所示.再简要概括出实验
步骤及实验注意事项。要提醒
学生,在测量各物理量时均应
读出估计数字来。
\end{enumerate}

\begin{figure}[htp]
    \centering
     %\includegraphics[scale=.8]{fig/7-4.png}
    \caption{}
\end{figure}

\subsubsection{把电流表改装为伏特表}
整个实验分三步进行,第一步,测定电流表的内电阻$r_g$.
第二步,根据测出的$r_g$, 计算分压电阻。并实际组装成伏特
表,第三步,把改装的伏特表跟标准伏特表校对,并计算相对
于满刻度的百分误差。

在指导学生实验时,需要注意以下问题:

应介绍电位器的使用方法.电位器有三个接线片,
可引导学生把电位器跟滑动变阻器比较:中间接线片相当于
滑动变阻器金属棒的接线柱,两边的接线片相当于滑动变阻
器中线圈的两接线柱。让学生知道电位器作分压器或可变电
阻使用应如何接线。

要引导学生弄懂用半偏法测电流表内阻的原理和条
件.在教材图9.7所示电路中,当断开$K_2$、闭合$K_1$调整
$R$使电流表指针恰偏转满度$I_g$时,电路中总电流强度$I=I_g$.
闭合$K_2$后,必然有一部分电流要流经$R'$, 使通过电流表的电
流减小,如能保持电路中总电流$I$不变,调节$R'$使通过电流
表的电流恰为$\frac{1}{2}I_g$, 由分流原理可知,这时$R'$的阻值一定等
于电流表内阻$r_g$, 只要读出电阻箱的电阻值就能得出电流表
内阻。

怎样才能保持总电流强度$I$不变呢?这就必须保持外电
路的总电阻不变.因为$K_2$闭合前外电路总电阻为$R+r_g$,
$K_2$闭合后调$R'$至$R'=r_g$时,外电路总电阻为$R+\dfrac{R'}{2}$,
所以
只有$R\gg R'$(即$R\gg r_g$) 时,才能认为外电路的总电阻基本没
有改变,也才能使电路的总电流强度在$K_2$闭合前后变化不
大,保证测量$r_g$的精确程度。因此,用半偏法测电流表内阻对
器材的要求就是$R\gg R'$.

由于一般电流表的内阻约100欧左右(如杭州电表厂的
J-DB$_2$XA型电流表,量限为$\pm300\mu {\rm A}$, 动圈内阻$92{\Omega}\pm 10$),
教材上说$R$可用470千欧的电位器,完全满足$R\gg R'$这个
条件。在准备学生实验时,根据学校现有器材怎样选用,才
能满足$R\gg R'$呢?事实上,只要$R\ge 100R'$就可以用$R'$代替
$r_g$; 它产生的相对误差不大于1\%,若受实验条件限制,取$R\ge 
50R'$, 测量的相对误差也不会超过2\%(见参考资料).这对
中学生分组实验的误差要求已基本可以了。

在讲解把改装成的伏特表跟标准伏特表校对时,应
启发学生认识教材图9.9所示电路中,滑动变阻器作
分压器使用是为了用一个阻值较小的滑动变阻器就能保证伏
特表两端的电压变化可以从零增加到2伏特,满足校对表的
要求。要提醒学生注意搞清改装后电流表上刻度的每一小格
表示的电压数,注意相对于满刻度的百分误差的计算方法。

在学生动手实验前,应告诫他们:闭合$K_1$前$R$应
调到最大值,以免闭合$K_1$后通过电流表的电流过大而损坏
表头;$K_2$闭合后,在调节$R'$时不能再改变$R$值,否则将改变
电路的总电流强度,影响实验的精确程度。

\subsubsection{用安培表和伏特表测定电池的电动势和内阻}
这个实验应注意以下几个问题。

实验器材的选择.要引导他们分析教材276页图
9.9所示电路,了解实验时应选用较大内阻的伏特表和较小
阻值的变阻器。伏特表内阻越大,变阻器阻值越小,安培表的
读数就越接近通过电池的真实电流,实验误差就会越小。但变
阻器阻值又不能太小,以免电流大于电池或变阻器允许通过
的最大电流,教师在准备实验器材时,干电池需选用已使用
过一段时间的干电池。以使测得的内阻适当大一些。由于现
在一般学校配备的安培表量程为0—0.6—3A,伏特表量程为
0—3—15V,为使读数比较准确,可用两节干电池串联作电
源,安培表量程选用0—0.6A挡,伏特表量程选0—3V挡,这
时伏特表内阻约1千欧,所以滑动变阻器宜选用J2355型
(电阻值约为0—50$\pm 10\%\Omega$, 额定电流1.5A)。若变阻器阻值
选得过小(如用0—10$\Omega$的J2354型)则阻值变化范围太小,作
图表时示数值的点会过于密集,影响实验的准确性。

如何作图。应引导学生认识用作图法求电池电动势
和内电阻的优越性在于:可以省去解方程、求平均值的运算,
比较简捷地求出答案。要引导学生了解,在数据准确的情况
下怎样作图才能使实验结果更准确。第一,实验时滑动变阻
器的阻值应在几欧到几十欧的较大范围内变化,使各组数据
差别大一点,作图时才能使数据点适当拉开一些。第二,实验
数据要多取几组(至少5组).由于读数的偶然误差,描出的
点不在一条直线上,在作图时应通过尽可能多的点画一条直
线,并使不在直然上的各点分布在直线两侧的数目大致相等,
个别偏离直线太远的点,则舍去,不予考虑。这样才能使各次
实验的偶然误差得到部分抵消,以提高实验的精确程度,第
三,要适当选取横坐标$I$和纵坐标$U$的标尺比例和坐标起点,
使实验数据点大致布满整个图纸,不要集中在一边或一角。由
于这个实验的$U$值不宜过小,因此纵坐标$U$的起点可根据实
测数据从不为零的某一数值开始,但因为要用$I=0$时图线
在$U$轴上的截距来求电动势$\mathcal{E}$, 横坐标$I$必须以零为起点。

实验的注意事项.第一,接通电源前应将变阻器的
阻值调到最大值,避免$K$闭合时通过安培表的电流过大;在实
验中决不允许将滑动触头滑在使变阻器阻值为零的位置,造
成外电路短路。第二,利用$U$-$I$图象计算电池的内电阻时,
不能简单地在实验数据中任选两组$U$、$I$值分别相减再相除。
这样达不到取平均值,减小偶然误差的目的。也不能用量出
图线与横轴的夹角,通过查正切函数表求$r$的值。而应当在
图线上任选两个相距较远的点,计算图线的斜率$\Delta U/\Delta I$
以求得内阻$r$.


\subsubsection{练习使用万用表}
教材是以供中学实验用的J0411型万用表为例来介绍万
用表的使用方法,如果学校没有这种型号的万用表,可根据
教材的要求结合自己的具体情况做此实验。

本实验应该注意以下几个问题。

在介绍万用表的外形及测量前的准备工作时,可按
教材图9.10制作一张幻灯片或挂图,配合讲解.

应特别强调并严格要求学生遵守教材提出的测量
前,测量时的有关注意事项和操作程序。除此之外,还要提醒
学生:第一,测量电流和电压时应在用表笔接触测量点的同
时,注视电表指针的偏转情况,并随时准备在出现不正常现象
时,使表笔离开测量点。第二,用欧姆挡测电阻时,不得测额
定电流极小的电阻(如灵敏电流计的内阻);不得测带电的电
阻;如果待测电阻跟别的电路元件相连,应当把待测电阻同它
们断开,否则测出的阻值将包括待测电阻和其他元件在内的
等效电阻,第三,使用完毕,务必将万用表选择开关拨离欧姆
挡,应拨到空挡或最大交流电压量程处。

具体练习使用万用表时,建议通过测量直流电压,直
流电流和电阻,结合验证串、并联电路的特点。下面给出参考
电路图,实验器材的规格可根据各校实际自行选择,但注意不
要超过万用表各挡的量程。

测量电流时,按图7.5甲所示电路实验,分别测出干路的
总电流强度$I$和各支路电流强度$I_1$、$I_2$, 验证$I=I_1+I_2$.

\begin{figure}[htp]
    \centering
% \includegraphics[scale=.7]{fig/7-5.png}
    \caption{}
\end{figure}

测量电压时,先按图7.5乙所示电路分别测出$R_1$、$R_2$两
端的电压$U_1$和$U_2$以及总电压$U$, 验证$U=U_1+U_2$. 再按图
7.5甲所示电路分别测出$R_1$、$R_2$两端电压,验证并联电路两
端电压相等,$U_1=U_2$.

测电阻时,先分别测出$R_1$、$R_2$的阻值,接着把$R_1$跟$R_2$串
联测出总电阻$R$, 验证$R=R_1+R_2$. 再把$R_1$跟$R_2$并联,测出
总电阻$R'$, 验证$R'=\dfrac{R_1R_2}{R_1+R_2}$。
为了提高测量电阻的精确程度,
选取的欧姆挡量程应使指针的偏转角度尽量靠近表盘中间
(见参考资料).例如,测量1千欧的电阻时,可供选用的量程
有“$\x$1k”、“$\x$100”、和“$\x$10”,但以选用“$\x$100”的量程最为
适宜。因为这时指针的位置相对而言最靠近表盘中心。

\subsubsection{用惠斯通电桥测电阻}
对于没有滑线式电桥供学生实验的学校,可以选用以下
器材自行组装.滑线采用直径0.3毫米,有效长度0.5米(阻
值约3.2$\Omega$)的锰铜丝,张挂在0.5米长的刻度尺上,刻度尺的
最小分度为1毫米.滑键采用铜片自制.滑动变阻器要选用
阻值比锰铜丝电阻大得多的规格(如0—50$\Omega$). 已知电阻$R$
用0—9999.9$\Omega$的电阻箱.待测电阻$R_2$选用阻值约几千欧
的定值电阻,电源用3—4.5伏干电池组。把它们和灵敏电
流表、电键K按教材图9.11接线,组成电桥.

应结合实物采用边讲边演示的方式,引导学生弄懂教材
280页到281页提出的问题;强调实验操作程序和要求,并严
格要求学生按这些规定进行实验。

实验中应注意的问题.
\begin{enumerate}
\item 通过锰铜丝的电流不宜太大,
连续使用的时间不能过久,否则,锰铜丝会明显发热变长,影
响测量效果.因此,滑动变阻器的阻值不要减小太多.    
\item 滑
动触头要能与锰铜丝接触良好,以减小因接触电阻引起的误
差.    
\item 锰铜丝要保持粗细均匀,如表面生锈或有碰伤,应更换新的锰铜丝。
\end{enumerate}

\subsection{课外实验活动}
\subsubsection{自制电池}
不同金属具有不同的标准电极电位,只要把不同的金属
放在电解质溶液中,由于化学作用,就会产生电动势,成为一
个化学电源——电池。从原则上讲,任何两种不同金属都可
以作为电池的两个电极,一般说来,活动性较大的金属是负
极,活动性较小的金属是正极。但是,要获得好的实验效果,
应该选择金属活动性相差较大的两种作为电池的两极。这是
因为它们的标准电极电位相差较大,产生的电动势较大。教
材介绍用锌片、铜片及吸满盐水的吸水纸自制电池,实际上就
是一个最原始的伏打电堆。这种电池的作用机理超出了高中
化学的内容,不必从化学角度对这个问题作过多的阐述。

\subsubsection{研究电灯泡的电阻}

这个实验是要使学生进一步认识,金属电阻率随温度升
高要增大。用电灯泡的额定功率和额定电压计算出的电阻,
是灯泡在正常发光的温度下灯丝具有的电阻值。如果灯泡两
端的电压不等于额定电压,灯丝的电阻也会相应发生变化。


\section{习题解答}

\subsection{练习一}







