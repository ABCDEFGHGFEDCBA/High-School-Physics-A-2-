\chapter{内能、能的转化和守恒定律}
\section{教学要求}
这一章主要讲述内能的概念、热力学第一定律以及普遍
的能的转化和守恒定律,这些内容贯穿整个热学教材,是热
学的基础知识。能的转化和守恒定律是自然界的一个普遍定
律,学生很好地掌握这个定律,逐渐学会从能的转化的观点
来分析物理现象和解决物理问题,不仅对学好热学而且对学
好整个物理学都很重要,因这一章在整个中学物理教学中
占有重要地位。

这一章也是在初中物理的基础上讲解的。跟初中比较,
虽然重述了某些内容,但对概念的讲解和对问题的分析都有
所扩展和加深。为了便于初高中的衔接,在教学中应重新做
一下初中做过的演示。

这一章也是力学部分机械能一章的继续和延伸,那里讲
了机械能守恒定律以及功和能的关系,这里进一步扩展为热
力学第一定律以及能的转化和守恒定律。因此在教学中适当
联系或者复习机械能一章所讲的内容,特别是功和能的关系
以及功是能的转化的量度,对这一章的教学将会有所帮助。

这一章可分为两个单元:

第一单元包括第一节至第三节,讲述物体的内能、改变内
能的两种方式以及热功当量。

第二单元包括第四节至第六节,讲述热力学第一定律、能
的转化和守恒定律以及能源的利用和开发。

这一章的重点是物体的内能以及热力学第一定律和能的
转化和守恒定律。

讲述内能时,教材在前一章知识的基础上,先说明了温度
的微观含义,这里首先要使学生了解,在热现象中,人们所关
心的不是个别分子的情况,而是大量分子所表现出来的集体
行为。使学生一开始就了解这个观点,对于他们今后学习用
分子运动论解释热现象,将会有所帮助,考虑到在中学阶段
对分子的动能不展开讲解可能便于教学,因而没有明确区分
分子的平动能、转动能和振能,关于温度,只是笼统地指出
“温度是物体分子运动的平均动能的标志”。这里,实际上是把
分子当作质点来处理的。这一点,教师要有所掌握,对学生则
不必做进一步解释,同样,教材对内能的概念也不过于追求
严格,只给出内能的狭义定义,即所有分子无规则运动的动
能和分子势能的总和(内能的广义定义,不仅包括分子的动
能和势能,还包括原子内的能量、原子核内的能量等)。

跟中学阶段内能的狭义定义相适应,教材指出内能是由
温度和体积决定的。这里,为了讲解方便,教材是把温度和体
积这两个参量作为自变量来看待的。其实,对某些简单系统
(如气体、液体和各向同性固体)来说,由于温度、体积和压强
三者由物态方程联系起来,所以上述三个量中的任何两个都
可以作为独立参量,而内能作为状态函数由两个独立参量所
决定。这一点,教师要有所掌握。

讲过内能之后,要求学生能够区别开内能和机械能。除
了课文明确指出这一点,还安排了区分机械能和内能的练习
题(练习一第5题).

通过学习物体的内能这一节教材,要求学生从能的观点
对热的本质有明确的认识。教材安排了阅读材料,使学生了
解人类对热的本质有过曲折的认识过程。教师也可以在课堂
上简单介绍一下热质说,讲一下人类对热的本质的认识过程。
第二节讲述改变内能的两种方式。这方面的知识,在初
中讲过,为了使学生回忆起初中讲过的事实,重做一下初中
做过的演示还是必要的。教材讲的做功可以改变内能,实际
上仍限于机械功,不涉及其他形式的功(如电流做功等)。为
了使学生具体认识机械功,教材举出了几种不同的做功情况。
对气体膨胀做功,教材不要求定量地推导公式$W=p\Delta V$。讲
述这一节教材,还应着重使学生认识到,在只有做功或热传递
的情况下,可以分别用功或热量来量度内能的改变,以便为讲
解热力学第一定律打好基础。

热功当量的测定,在历史上有重要作用,为能的转化和守
恒定律奠定了坚实基础(既然在多种不同的实验中,消耗同样
多的机械能总会产生相同数量的热,这就意味着热是能的一
种形式,并且在机械能转化为内能的过程中能量是守恒的)。
同时定量地证明了功和热量是等效的。热功当量的意义并不
仅是确定了两个单位的换算关系,通过讲述热功当量,应当使
学生对这些有所理解。

第四节讲述热力学第一定律与能的转化和守恒定律。热
力学第一定律也就是宏观过程中能的转化和守恒定律。在热
力学第一定律的表达式$W+Q=\Delta E$中,如果功$W$包括各种功
(机械功、电功等),$E$是指研究对象的总能量(包括各种形式
的能,也包括机械能在内),那么,热力学第一定律就是宏观过
程中的普遍的能的转化和守恒定律,在微观世界中,对个别
粒子间的相互作用,能量守恒定律仍是成立的,但热力学第一
定律在此没有意义,因此不能把这两个定律完全等同起来。

为了从热力学第一定律扩展为普遍的能的转化和守恒定
律,教材在给出热力学第一定律的表达式后,着重从能的转化
的观点进行分析。一方面,从能的转化的观点指出功和热量
的不同,同时也从能的转化的观点明确了热力学第一定律的
意义,在此基础上引出能的转化和守恒定律 并举出一些能
的转化的实例,要求学生学习从能的转化的观点来分析物理
现象。

永动机(指第一类永动机)不可能制成,在历史上对建立
能的转化和守恒定律起了重要作用;人类是通过正反两方面
的经验总结出能的转化和守恒定律的。为了使学生理解这一
点,教材简单说明了永动机不可能制成,讲解这个问题,应使
学生确信:人类利用自然必须遵从自然规律,违反自然规律,
必将一事无成。

能的转化和守恒定律是自然界的一条普遍的、重要的规
律。为了使学生具体认识这一点,教材第五节讲述这个定律
的重要意义,在这方面,教师可以做些补充,使学生对这个问
题的认识更丰满一些。

第六节讲述能源的开发和利用。讲述这类实际知识,应
该要求学生对科学技术与社会生活的联系有一个初步的了
解,不要求涉及更多的技术问题,还要注意联系已有的物理
知识来讲解,使学生知道人类怎样利用物理知识来寻求解决
能源问题的道路。

归纳以上所述,这一章的教学要求是:
\begin{enumerate}
\item 理解温度的微观含义,理解内能的概念,了解物体的
内能跟物体的温度和体积有关。
\item 了解改变内能的两种方式,了解内能的变化可以分别
由功和热量来量度,理解热功当量的意义。
\item 掌握热力学第一定律,能够从能的转化的观点理解这
个定律,并会用表达式$W+Q=\Delta E$来分析计算有关问题。
\item 掌握能的转化和守恒定律,理解这个定律的重要意
义,学会从能的转化和守恒的观点来分析物理现象解决物理
问题,培养综合运用力学知识和热学知识的能力。
\end{enumerate}

\section{教学建议}
\subsection{第一单元}
在本单元中,教材讲述了什么是内能,指出物体的内能
跟温度和体积有关,从而使学生认识与热运动相联系的一种
新的能的形式。又讲述了怎样可以改变物体的内能,说明做
功和热传递对改变内能是等效的。这样,学生对改变能量的
方式的认识也将有所扩展。教师要引导学生在正确理解内能
概念的基础上,注意领会怎样从能量的观点来考虑和认识
问题。

\subsubsection{温度的微观含义} 教学时,可向学生强调指出:
\begin{enumerate}
\item 
物体里各个分子的动能并不相同。但物体里的分子数目极其
巨大,所有分子又都处在永不停息的无规则运动之中,要想求
出每个分子的动能是不可能的,也是没有必要的。在热现象
的研究中,人们关心的不是物体里每个分子的动能,而是分子
热运动的平均动能。或者说,人们研究的是大量分子所表现
出的集体行为,而不是个别分子的运动情况,为此,建立了统
计方法。这是用分子运动论解释热现象的基本思想方法。在
一章讨论分子的速率分布和气体压强的微观解释等问题
时,将初步介绍这种方法。    \item 物体的温度升高,分子热运动
的平均动能增加;温度降低,分子热运动的平均动能减小。物
体每一确定的温度,都跟一定的分子热运动的平均动能相对
应。所以说温度是分子热运动的平均动能的标志。但不能
说“温度等于分子热运动的平均动能。”
\end{enumerate}


温度是热学中一个重要的基本概念。学生从初中到高中
学习这个概念经历了由表及里、由现象到本质的认识过程。在
初中,只知道温度表示物体的冷热程度,至于物体为什么冷,
为什么热,它的内在因素是什么,那时是不了解的。经过前
一章分子运动论基础的学习,了解到温度与分子的无规则运
动有着紧密联系。温度越高,分子无规则运动越激烈。现在
又知道温度是分子热运动的平均动能的标志,这就懂得了温
度这个宏观物理量的微观含义了,教师可启发学生通过小结
对上述认识过程有所体会。

\subsubsection{分子势能与分子间距离的关系} 讲述内能与物体的
体积有关,需要说明分子势能跟分子间距离的关系、教材定
性地讲述了这种关系,但学生对分子势能随分子间距离而改
变常常感到抽象、难懂.为此,建议教学时注意以下各点:
\begin{enumerate}
\item 
先引导学生复习一下力学中学过的有关知识,即外力克服弹
力(重力)做功,弹性(重力)势能增加,外力做了多少功,弹性
(重力)势能就增加多少。弹力(重力)做功,弹性(重力)势能
减小,弹力(重力)做了多少功,弹性(重力)势能就减小多少。
学生明确地回忆起这些知识,将有利于他们在学习知识时
引起联想和进行类比.    
\item 尽量采用比喻说明的方法,启发学
生通过思考理解分子势能随距离改变的情形。例如,教师可
以利用教材上的比喻,结合教材13页图1.7, 引导学生分段
讨论,逐步认识:分子间存在着相互作用力,因此分子间具有
由它们的相对位置所决定的分子势能,这跟弹簧具有弹性势
能相似。要改变分子间的相对位置,就必须克服分子间的作
用力而做功,这跟拉伸或压缩弹簧时,必须克服弹力做功相
似。弹簧没有形变时,弹簧的弹性势能为最小值。这跟分子
间的距离$r=r_0$时,引力和斥力互相平衡,合力为零,分子势
能为最小值相似。当分子间的距离$r>r_0$时,分子间的作用
力表现为引力,把分子间的距离增大,就必须克服分子间的引
力做功;距离增大越多,克服分子间引力做的功就越多,分子
势能也就越大。所以说,在分子间的作用力表现为引力时,分
子势能随着分子间距离的增大而增大。这跟弹簧被拉长时,
弹性势能增大的情形类似。当分子间的距离$r<r_0$时,分子
间的作用力表现为斥力,把分子间的距离减小,就必须克服分
子间的斥力做功,距离减小越多,克服分子间的斥力做的功就
越多,分子势能也就越大,所以说,在分子间的作用力表现为
斥时,分子势能随着分子间的距离减小而增大。这跟弹簧
被压缩时,弹性势能增大的情形类似。对于气体来说,分子间
的相互作用力是引力,当分子间的距离缩小时,分子间的引力
做功,所以分子势能减少。这跟弹力(重力)做功,弹性(重
力)势能减少相似。学生搞清楚上述的情况,就不难理解分子
势能与分子间的距离,因而与物体的体积有关了。
\end{enumerate}

\subsubsection{内能}

物体的内能是本章的一个重点.为了让学生
清楚地理解内能这个概念,建议引导学生认识:
\begin{enumerate}
\item 从微观角
度看,物体的内能是指物体里所有分子的动能与势能的总和,
不是指单个分子的动能与势能之和.   
 \item 内能是由物体的温
度和体积决定的(体积跟物体所含分子数目的多少有关),在
热现象的研究中,人们感兴趣的是内能的变化,而不是内能的
绝对数值,所以在练习一1—4题中,要求回答的都是内
能怎样改变,即内能是增加还是减少.   
 \item 内能是不同于机械
能的另一种形式的能量。机械能是就宏观物体整体来说的,
内能则是指物体里所有分子的动能与势能的总和,物体在具
有一定内能的同时,也可以具有一定的机械能,例如练习一
第5题中在高空飞行的炮弹就是这样。可以让学生讨论这
个题目,认清机械能与内能的区别,不把二者混为一谈,还可
以举出其他实例帮助学生思考,例如一个静止在地面的物
体,如果以地面作为势能的参考零点,那么这个物体的总机械
能为零,但是这个物体里的分子却始终处在永不停息的热运
动之中,所以它的内能绝不为零。 
\item 在工程上和日常生活中,
一般都只说热能,很少用到内能。实际上说热能的地方往往
就是指内能,所以说,热能是内能的一种通俗说法。
\end{enumerate}

\subsubsection{改变内能的两种方式} 这部分知识,学生在初中已经
学过。这里,可引导他们在复习已有知识(包括重新观察初中
做过的演示实验)的基础上进一步认识:
\begin{enumerate}
\item 热量是单纯由于
热传递使物体的内能发生变化时,用来量度物体内能变化的
物理量。物体吸收(或者说外界传递给物体)多少热量,物体的
内能就增加多少;物体放出(或说物体传递给外界)多少热量,
物体的内能就减少多少。这就是说,热量只能在热传递过程中
用来衡量物体内能增减的多少。除此以外,热量就失去了意
义,因此,说“物体具有或含有多少热量”,是没有意义的,因
而也是不科学的.    \item 在单纯由于做功或单纯由于热传递使
物体的内能发生变化时,可以分别用功或热量来量度物体内
能的变化。如果物体内能的变化是同时通过做功和热传递来
实现的,这时就既不能单独用功,也不能单独用热量来量度物
体内能的变化了.    \item 做功和热传递对改变物体的内能是等
效的。但两者还是有区别的。做功改变内能是物体有规则运
动的能量转化为分子无规则运动的能量;热传递则是分子的
无规则运动的能量在物体间的转移。
\end{enumerate}

\subsubsection{热功当量} 初中已讲过热功当量,知道热量的单位
“卡”和功的单位“焦”之间的换算关系。现在,要引导学生从
做功和热传递对改变物体内能等效的角度,认识热功当量是
指相当于单位热量的功的数值。并要向学生明确指出:$J=
4.2$焦/卡,不仅确定了“卡”和“焦”这两种单位的换算关系,更
重要的是,它表示实验测得传递给物体1卡热量使物体增加
的内能,相当于对物体做4.2焦的功所增加的内能,认识到
这一点,才算是懂得了热功当量的含义。

介绍焦耳测定热功当量的实验(教材22页图2.2),要
引导学生认识这个实验是利用做功使水的内能增加,与假定
不是由于做功而是由于热传递使水增加同样多的内能所需热
量之间的关系,来测出热功当量的数值的,还可以向学生介
绍:焦耳为了精确地测定热功当量的数值,不断改进实验方
法,先后采用各种方法做了400多次实验。现在教材上介绍
的这个实验,是从1845年开始,到1849年测定才告一段落.
在实验中,最初用的是水,后来改用鲸鱼油,实验结果表明,水
和鲸鱼油的热功当量数值是相当接近的。但焦耳并不满足,
又换用比热较小的水银来进行测定,以提高实验的可靠性,此
后,焦耳仍不停步,继续采取其他实验方法,他最后得到的实
验结果,其数值与现在公认的热功当量值相差不到5\%. 100
多年以前,焦耳就能做出这样准确的实验,真是难能可贵。焦
耳在长期实验工作中表现出的严肃认真、精益求精的科学态
度与不断创新的精神,在物理学史上也是相当突出的,值得我
们很好学习。

对测定热功当量在历史上的重要作用,可引导学生从以
下两点来认识。一点是定量地证明了做功和热传递对改变物
体内能的等效关系.这一点,可组织学生讨论练习二第4
题,让他们通过从反面设想,得出正确答案,获得较深刻的认
识。另一点是,用各种不同的方法来测定热功当量,都得到了
相同的结果。这就意味着机械能并未消失,它转化成了另一
种形式的能量,而且在转化过程中,能量是守恒的。比如,放
开使叶片在盛水的量热器中转动的重物,水的温度就升高一
定的度数.这个实验事实在19世纪中叶的科学界产生了重
大的影响,为能的转化和守恒定律的建立打下了坚实的基础。

\subsection{第二单元}
这一单元先讲热力学第一定律,然后再扩展为普遍的能
的转化和守恒定律,接着讲述能的转化和守恒定律的重要意
义,介绍能源的利用和开发。引导学生正确理解内能和其他
形式能量的转化,启发学生从能的转化的观点来分析物理现
象、解决物理问题,是本单元教学的中心问题,也是学生学好
这一章的关键。

\subsubsection{热力学第一定律}

这个定律阐明了当物体跟外界同
时发生做功和热传递的过程因而使物体的内能发生变化时,
功、热量和内能的变化三者间的定量关系。建议教学时注意
引导学生理解热力学第一定律数学表达式的意义。按照教材
的讲法,$W$表示外界对物体所做的功,$Q$表示物体吸收的热
量,$\Delta E$表示物体内能的增加,公式$W+Q=\Delta E$的意义是:如
果物体跟外界同时发生做功和热传递的过程,那么,外界对物
体所做的功$W$加上物体从外界吸收的热量$Q$, 等于物体内能
的增加$\Delta E$. 应向学生强调指出,公式中各量的正负号就是针
对上述情况规定的,如果规定正负号的法则有了改变,热力学
第一定律的表达式也就随之改变。这方面的知识,最好结合
讨论练习三第5题进行,以利于加深学生的印象,在应用公
式解题时,要提醒学生把公式中各量的单位都统一成焦,并特
别注意公式中各量的符号所反映的物理意义。

如教师认为有必要,还可以告诉学生:热力学第一定律提
供了定量地测定物体内能变化的方法。

\subsubsection{能的转化和守恒定律}

建议按教材的线索,引导学生
逐步深入地来认识这个自然界的普遍规律。即先从能的转化
的观点认清做功和热传递的区别,以及热力学第一定律的意
义。接着认识内能不但可以和机械能相互转化,还可以和其
他形式的能(如电能、化学能、光能等)相互转化,并且在转化
过程中能量守恒,然后认识各种形式的能都可以相互转化,
并且在转化中守恒,在此基础上,向学生强调指出:能的转化
和守恒定律是大量经验的总结,是实验定律。

为了让学生对能的转化的认识更加丰满,建议教师多举
一些常见的、类似练习四第1题那样的实例(如分析打夯时,
从人抬起石夯,到石夯落到地面后静止,其中能的转化过程;
转动着的空竹,发出嗡嗡的响声,转动不断变慢,其中能的转
化过程等)。还可以启发学生举出自己了解的实例,以利于培
养他们学习从能的转化的观点来分析物理现象。

对热力学第一定律与能的转化和守恒定律的联系、区别,
可由教师给以适当说明。

永动机不可能制成,是人们实践经验的总结,它从反面
证实了能的转化和守恒定律是不可违反的客观规律,远在
17—18世纪的时候,人们为了满足生产对于动力的日益增多
的要求,幻想制造一种不消耗任何能量,却能永久工作的机
器,在这种幻想指引下,曾经有许多人提出过各式各样的永动
机的设计,但是所有这些设计在实践中都失败了.因此,1775
年法国科学院宣布不再接受审查关于永动机的发明。这说明
在能的转化和守恒定律建立之前,科学界已经从长期积累的
经验中,认识到制造永动机的企图是没有成功的希望的 建
议在教学中适当补充一点这方面的历史背景材料,帮助学生
了解:制造永动机的失败,导致了能的转化和守恒定律的发
现。而这一定律的建立,对于永动机不可能制成,则给予了科
学的最后判决。它告诉人们必须丢掉违反自然规律的幻想,
遵循能的转化和守恒定律,去研究各种不同形式的能相互转
化的具体条件,以求得最有效地利用自然界所能提供的多种
能源。这个是人类利用自然的正确道路,也是值得后人认真
吸取的历史经验。

如果教师认为有必要,也可以举出永动机的实例(如高中
物理乙种本上册242页的永动机设计方案)来具体说明.但
这种说明不宜过细,应着眼于引导学生认识违反自然规律,必
将一事无成,甚至受到自然的惩罚。

\subsubsection{能的转化和守恒定律的重要意义}

建议第五节教材
由学生自己阅读,教师着重从以下几点引导他们加深认识。
\begin{enumerate}
\item 能的转化和守恒定律的建立,是许多科学家长期探索,辛
勤劳动的结晶。体现了理论与实验合,理论家与实际工作
者结合(如卡诺、科耳定等人对这一定律的建立也作了重要贡
献,他们既是工程师,又搞物理学研究),物理学与其他学科结
合(如生理学)的特点.
\item 这个定律是自然界的基本规律之
一。它能把理、化、生、天、地等学科和各种工程技术联系起来,
对于人们认识自然,改造自然,从事科学实践,具有巨大的预见性和指导作用,中微子的发现就是一个例证,所以有的科学
家说:“与其拒绝能量守恒定律,还不如想出一种新的能量形
式为好。”(彭加勒语)
\end{enumerate}

教师还可向学生指出:能的转化和守恒定律贯串在全部
物理学中,可以把各种物理现象联系起来。因此,应用能的
转化的观点来分析物理现象,解决物理问题,是很重要的物理
思维方法。

\subsubsection{能源的利用和开发}
 这个问题是大家很关心的世
界性问题.建议引导学生注意了解:
\begin{enumerate}
\item 在合理开发和有效
利用常规能源的同时,不断探索能的转化的新途径,大力开发
和利用新能源,是解决能源问题的正确道路。    
\item 促使燃料完
全燃烧,减少热量损失;充分利用“余热”,提高燃料的总利用
率,是节约能源的有效途径.    
\item 当前,原子能的开发和利用,
既可使我们获得巨大的能量,在经济上也比较合算.   
 \item 解决
能源问题,最根本的还是要靠科学研究和技术进步。
\end{enumerate}

在教学中,还可以结合我国在能源生产上取得的成就和
存在的问题,对学生进行爱国主义教育。激发他们为振兴中
华而努力学习和勇于探索的精神。

\section{实验指导}
\subsection{演示实验}
\subsubsection{压缩气体做功,气体的内能增加}
这个实验常用压缩气体引火仪来做。压缩气体引火仪如
图2.1所示,它主要由压缩手柄、活塞、圆筒、底座等组成.

若用硝化棉作燃烧剂,则应选用优质硝化棉(可取极少量
硝化棉在空气中燃烧,优质硝化棉的火焰明亮,燃烧过程极
快,无残留灰分),实验时,要先把绿豆大的片状硝化棉放在
圆筒底或活塞下面的钩上,然后将活塞垂直放进圆筒中,先用
不太大的力气下压活塞一次,进行预热,再用力快速将活塞压
下,硝化棉就会燃烧,发出明亮火光。

如果实验效果不好,应检查实验装置的密闭性能是否良
好。如果是底座漏气,可用凡士林密封;如果是活塞跟圆筒内
、壁接触不够紧密,就应更换活塞上的橡皮胀圈,同时在活塞上
薄薄涂一层缝纫机油。

\begin{figure}[htp]\centering
    \begin{minipage}[t]{0.48\textwidth}
    \centering
\includegraphics[scale=.6]{fig/2-1.png}
    \caption{}
    \end{minipage}
    \begin{minipage}[t]{0.48\textwidth}
    \centering
    \includegraphics[scale=.6]{fig/2-2.png}
    \caption{}
    \end{minipage}
    \end{figure}

\subsubsection{克服摩擦做功,物体的内能增加}
如图2.2所示,把演示仪器夹紧在桌边,往黄铜管里滴入
乙醚约占全管容积的1/4—1/5, 把塞子塞好,防止漏气,但也
不宜过紧.用麻绳或涂有松香的纱带(约1厘米宽)绕黄铜管
一、二周,手执绳子两端迅速往复拉动,使绳和管壁摩擦,经过
几分钟,管里乙醚便沸腾起来,产生乙醚蒸汽,把塞子冲开。

如果摩擦很久还不能把塞子冲开,可将塞子稍为拔松,但
要注意不能把头部正对管口,以防管内液体或蒸汽喷入眼里,
造成伤害。

这个实验也可用酒精来做,但酒精的沸点较高,演示时间
较长。


\section{习题解答}

\subsection{练习一}

\begin{enumerate}
	\item 壶里的水被加热而温度升高,水的内能怎样改变?液体的热膨胀很小,可不予考虑.
	
\begin{solution}
    水分子的平均动能增大,水的内能增加。
\end{solution}    
\item 一根烧红了的铁棍逐渐冷却下来,铁棍的内能怎样改变?固体的热膨胀很小,可不予考虑.
	
\begin{solution}
    铁分子的平均动能减小,铁棍的内能减少。
\end{solution}    
\item 容器里装着一定质量的气体,在保持体积不变的条件下使它的温度升高,气体的内能怎样改变?在保持温度不变的条件下把气体压缩,气体的内能怎样改变?
	
\begin{solution}
    容器里一定质量的气体,在保持体积不变的条件下使
它的温度升高时,气体的分子平均动能增大,内能增加;在保
持温度不变的条件下把气体压缩时,气体的分子势能减小,内
能减少。
\end{solution}    
\item 设想我们对固体进行压缩.当分子间的距离小于$r_0$时,随着固体被压缩分子势能怎样改变?
	
\begin{solution}
    当分子间的距离小于$r_0$时,分子间的相互作用力表
现为斥力。压缩固体使分子间的距离减小,必须克服斥力做
功,因此分子势能随着固体被压缩而增大。
\end{solution}    
\item 一颗炮弹在高空中以某一速度$v$飞行,由于炮弹中所有分子都具有这一速度,所以分子具有动能,又由于所有分子都在高处,所以分子具有势能.所有分子的上述动能和势能的总和就是炮弹的内能.上述说法正确不正确?为什么?
	
\begin{solution}
    不正确。因为炮弹的内能应是炮弹中所有分子的热
运动的动能和分子势能的总和;而题中所说的动能和势能则
是炮弹整体的动能和重力势能,它们的总和是炮弹的总机械
能,不是炮弹的内能。
\end{solution}    
\end{enumerate}

\subsection{练习二}
\begin{enumerate}
\item 举出几个实例来说明:做功可以改变物体的内能.

\begin{solution}
    例如,克服摩擦做功,物体的温度升高,内能增加;压
缩气体做功,气体的温度升高,内能增加;气体膨胀时做功,气
体的温度降低,内能减少。
\end{solution}
\item 锅炉中盛有150千克的水,由20$^\circ$C加热到100$^\circ$C,水的内能增加多少?

\begin{solution}
    水的内能增加等于水吸收的热量,即$\Delta E=Q$. 而
$Q=cm(t_2-t_1)$, 将已知数据代入此式,得
\[\begin{split}
    \Delta E&=Q=c_{\text{水}}m_{\text{水}}(t_2-t_1)\\
&=4.2\x10^3\x150\x(100-20){\rm J}=5.0\x 10^7{\rm J}
\end{split}\]
\end{solution}
\item 一个物体的内能增加了20焦.如果物体跟周围环境不发生热交换,周围环境需要对物体做多少焦的功?如果周围环境对物体没有做功,需要传给物体多少焦的热量?

\begin{solution}
    如果物体跟周围环境不发生热交换,需要对物体做
20焦的功;如果周围环境没有做功,需要传给物体20焦的
热量。
\end{solution}
\item 设想在测定热功当量的不同实验中得到的结果并不相同,还能不能得到结论说:做功和热传递对改变物体的内能是等效的?讨论一下这个问题.

\begin{solution}
    如果测定热功当量的不同实验得到的结果并不相同,
这就表明做功和热传递对改变物体的内能没有确定的数量关
系。所以不能得出做功和热传递对改变物体的内能是等效的
结论。
\end{solution}
\item 在图2.2所示的实验中,已知重物$P$和$P'$的质量都是14千克,水的质量是7千克,重物连续从2米高处落下20次后,水的温度升高0. 37$^\circ$C.不考虑传给量热器和外界的热量,试根据这些数据计算热功当量.

\begin{solution}
    重物$P$和$P'$从2米高处落下20次所做的功为
\[\begin{split}
    20\x 2mgh&=20\x2\x14\x9.8\x2.0{\rm J}\\
&=1.1\x10^4{\rm J}
\end{split}\]
质量是7千克的水,温度上升0.37$^\circ$C所需热量为
\[\begin{split}
    Q=cm\Delta t&=1.0\x7.0\x10^3\x0.37{\rm cal}\\
&=2.6\x10^3{\rm cal}
\end{split}\]
所以,热功当量
\[J=\frac{1.1\x10^4{\rm J}}{2.6\x10^3{\rm cal}}=4.2{\rm J/cal}\]
\end{solution}
\end{enumerate}


\subsection{练习三}
\begin{enumerate}
		\item 做功和热传递对改变物体的内能虽然等效,但从能的转化的观点来看是有区别的,这种区别是什么?
		
\begin{solution}
    做功使内能发生变化时,是其他形式的能和内能的转
化。热传递使内能发生变化时,只是内能在物体之间的转移,
没有能量形式的转化。
\end{solution}
	\item 在图2.2所示的焦耳测定热功当量的实验中,什么其他形式的能转化成了水的内能?在历史上,热功当量的确定为建立能的转化和守恒定律提供了坚实的实验基础.你怎样理解这句话?讨论一下这个问题.
		
    \begin{solution}
    是机械能转化成了水的内能。热功当量的确定,表明
    消耗同样多的机械能总会产生相同数量的热,这就意味着热
    是能的一种形式,并且在机械能转化为内能的过程中能量是
    守恒的,所以说热功当量的确定为建立能的转化和守恒定律
    提供了坚实的实验基础。
    \end{solution}
	\item 用活塞压缩气缸里的空气,对空气做了900焦的功,同时气缸向外散热210焦.空气的内能改变了多少?
	
\begin{solution}
由热力学第一定律$W+Q=\Delta E$, 根据题意,活塞压缩
空气,对空气所做的功$W$为正,气缸向外放出的热量$Q$为
负值。代入已知数据,则空气内能的改变
\[\Delta E=W+Q=900-210=690{\rm J}\]
\end{solution}
	\item 空气压缩机在一次压缩中,活塞对空气做了$2\times 10^5$焦的功,同时空气的内能增加$1.5\times 10^5$焦.这时空气向外界传递的热量是多少?
		
    \begin{solution}
因为$W+Q=\Delta E$, 所以
\[Q=\Delta E-W=1.5\x10^5-2\x10^5 =-5\x10^4{\rm J}\]
$Q$为负值,表示空气向外界放出的热量是$5\x10^4$焦.
    \end{solution}
	\item 如果用$Q$表示物体吸收的热量,用$W$表示物体对外界所做的功,热力学第一定律也可以表达为下式:
	\[Q=\Delta E+W\]
	怎样解释这个表达式的物理意义?试根据课文中的表达式推
导出这个表达式.
		
\begin{solution}
    在表达式$W+Q=\Delta E$中,按题意外界对物体所做的
功为$-W$, 代入上式可得$-W+Q=\Delta E$.所以
\[Q=\Delta E+W\]
这个表达式的物理意义是:物体从外界吸收的热量$Q$, 等
于物体内能的增加$\Delta E$, 加上物体对外所做的功$W$.
\end{solution}
\end{enumerate}

\subsection{练习四}

\begin{enumerate}
	\item 试说明下列现象中能量是怎样转化的:
	\begin{enumerate}
		\item 在水平公路上行驶的汽车,发动机熄火之后,速度越来越小,最后停止.
		\item 在阻尼振动中,单摆的振幅越来越小,最后停下来.
		\item 火药爆炸产生燃气,子弹在燃气的推动下从枪膛发射出去,射穿一块钢板,速度减小.
		\item 用柴油机带动发电机发电,供给电动水泵抽水,把水从低处抽到高处.
	\end{enumerate}

\begin{solution}
\begin{enumerate}[(a)]
    \item 机械能转化为内能。\item 机械能转化为内能。\item 化
学能转化为内能,内能再转化为机械能,机械能又转化为内
能。\item 化学能转化为内能,内能转化为机械能,机械能再转化
为电能,电能又转化为机械能。
\end{enumerate}
\end{solution}


\item 取一个不高的横截面积是30${\rm cm^2}$的圆筒,筒内装水0.6千克,用来测量射到地面的太阳能,在太阳光垂直照射2分钟后,水的温度升高了1$^\circ$C.计算在阳光直射下地球表面每平方厘米每分钟获得的能量.

\begin{solution}
圆筒内0.6千克水的内能的增加为
\[\begin{split}
   \Delta E=Q=mc\Delta t&=0.6\x4.2\x10^3\x1\\
&=2.5\x10^3{\rm J} 
\end{split}\]

每分钟0.6千克水增加的能量为$2.5\x 10^3/2=1.25\x
10^3$焦.

因圆筒的横截面积为$3{\rm dm}^2=3\x10^2{\rm cm}^2$. 所以地球
表面每平方厘米每分钟获得的能量为
\[\frac{1.25\x10^3{\rm J}}{3\x10^2}=4.2{\rm J}\]
\end{solution}
\item 从20米高处落下的水,如果水的势能的20\%用来使水的温度升高,水落下后的温度升高多少摄氏度?

\begin{solution}
    依题意可得$0.20mgh=mc\Delta t$, 所以
\[\Delta t =\frac{0.20mqh}{mc}\]
已知$h=20{\rm m}$,取$g=9.8{\rm m/s^2}$, $c=4.2\x10^3{\rm J/(kg\cdot^{\circ}C)}$
代入上式,得
\[\Delta t=\frac{0.20\x 9.8\x 20}{4.2\x 10^3}{\rm ^{\circ}C}=9.3\x 10^{-3}{\rm ^{\circ}C}\]
\end{solution}
\item 用铁锤打击铁钉,设打击时有80\%的机械能转化为内能,其中50\%用来使铁钉的温度升高.打击20次后,铁钉的温度升高多少摄氏度?已知铁锤的质量为1.2千克,铁锤打击铁钉时的速度为10${\rm m/s}$,铁钉的质量为40克,铁的比热为$5.0\times 10^2 {\rm J}/({\rm kg\cdot— ^\circ C})$.

\begin{solution}
铁锤打击铁钉,用来使铁钉温度升高的机械能为
\[0.50\x0.80\x20\x\frac{1}{2}m_1v^2\]
已知铁锤质量$m_1=1.2$kg,铁锤
打击铁钉时的速度$v=10{\rm m/s}$.代入已知数据得
\[0.50\x0.80\x20\x\frac{1}{2}\x1.2\x10^2=4.8\x10^2{\rm J}\]
则铁钉增加的热量$Q=m_2c\Delta t=4.8\x10^2$焦.所以铁钉
升高的温度\[\Delta t=\frac{4.8\x10^2{\rm J}}{m_2c}\]
已知铁钉的质量$m_2=0.040$kg,铁的比热$c=5.0\times 10^2 {\rm J}/({\rm kg\cdot— ^\circ C})$, 代入已知数据,铁
钉升高的温度
\[\Delta t=\frac{4.8\x10^2 }{0.040\x 5.0\x 10^2}—^{\circ}{\rm C}=24^{\circ}{\rm C}\]
\end{solution}
\item 在光滑的桌面上放着一个木块,铅弹从水平方向射中木块,把木块打落在地面上,落地点与桌边的水平距离为0.4米.铅弹射中木块后留在木块中.设增加的内能有60\%使铅弹的温度升高,铅弹的温度升高多少摄氏度?已知桌面高为0.8米,木块的质量为2千克,铅弹的质量为10克,比热为$1.3\times 10^2 {\rm J}/({\rm kg\cdot —^\circ C})$.取$g=10{\rm m}/{\rm s^2}$.

\begin{solution}
    铅弹打中木块后与木块一起做平抛运动的初速度
\[v=\frac{x}{\sqrt{\frac{2y}{g}}}=\frac{0.4}{\sqrt{\frac{2\x 0.8}{10}}}{\rm m/s}=1{\rm m/s}\]

用$M$、$m$分别表示木块和铅弹的质量,根据动量守恒定
律可以求出铅弹射入木块时的速度$v'$
\[v'=\frac{(M+m)}{m}v=\frac{(2+0.010)\x 1}{0.01}{\rm m/s}=2\x 10^2{\rm m/s}\]

系统机械能的损失为
\[\begin{split}
   & \frac{1}{2}{mv'}^2-\frac{1}{2}(M+m)v^2\\
&=\left[\frac{1}{2}\x 0.010\x(2\x10^2)^2-\frac{1}{2}\x(2+0.01)\x 1^2\right]\\
&=2\x 10^2 {\rm J}
\end{split}\]

故增加的内能$\Delta E=2\x10^2$焦.

由$cm\Delta t=0.60\Delta E$, 所以铅弹升高的温度
\[\Delta t=\frac{0.60\Delta E}{mc}=\frac{1.2\x10^2}{0.010\x 1.3\x 10^2}—^{\circ}{\rm C}={\rm 92^{\circ}C}\]
\end{solution}
\end{enumerate}


\section{参考资料}
\subsection{温度的微观含义}
将实验得到的理想气体状态方程跟气体分子运动论的压
强公式相比较,可以找出气体温度与分子的平均平动动能之
间的关系,从而揭示出温度这一宏观物理量的微观含义。

如果用$N$表$M$克气体的分子数目,用$N_0$表示1摩尔气
体的分子数目,$m$表示一个分子的质量,那么$M=mN$, $\mu=mN_0$,
把这些关系代入克拉珀龙方程$pV=\frac{M}{\mu}RT$, 得
\begin{equation}
    p=\frac{mN}{V}\cdot \frac{R}{mN_0}\cdot T=\frac{N}{V}\cdot \frac{R}{N_0}\cdot T
\end{equation}

由于1摩尔任何气体的分子数$N_0$都相同,$R$是普适恒
量,所以$R/N_0$
也是个常数,用$k$表示,叫做玻尔兹曼常数。

$N/V=n$,表示单位体积内的气体分子数,因此(2.1)式可写为
\begin{equation}
    p=nkT
\end{equation}

气体分子运动论指出,气体对于器壁的压强由下式决定
\begin{equation}
    p=\frac{2}{3}n\left(\frac{1}{2}m\bar{v}^2\right)
\end{equation}
式中,$\frac{1}{2}m\bar{v}^2$表示分子的平均平动动能。

比较(2.1)、(2.3)两式可得
\[\frac{1}{2}m\bar{v}^2=\frac{3}{2}kT\]

上式表明,宏观量温度只与气体分子的平均平动动能有
关。它指出气体分子的平均平动动能与热力学温度成正比。
这就是说,上式反映了气体温度的统计意义,即气体的温度
是大量分子平均平动动能的量度,是大量分子热运动的集体
表现,说个别分子有多高的温度,是没有意义的。

从分析温度的微观含义,我们可以得到启示:在热现象的
研究中,宏观量与微观量是描述同一物理现象的两种不同方
法中所用的量,因而它们之间必然是有联系的。

\subsection{热质说简介}

18世纪经典力学和天文学取得的成就,向人们指出了定
量地表示现象,用数学解析来进行处理,是物理学所必须遵循
的方法。在这种思想的影响下,为了开辟定量地处理电、磁、
热等现象的道路,当时的物理学界普遍采用了电流体、磁流体
和热流体(热质)的设想。

最早阐明热质说的是菏兰医生、化学家波哈夫(1668—
1738),而把热质说作为热学理论根据的是英国化学家布莱克
(1728—1799).他于1760年左右确定了热容量的概念,明确
叙述了热平衡的概念,首次区分了温度与热量,从而打下了定
量地研究热现象的基础。可以说,热现象研究的真正发展就
是从这时开始的.1789年,法国著名化学家拉瓦锡(1743—
1794)在他所著《化学纲要》一书中第一次引入了热质一词,到
十八世纪末,热质说已在热学理论中占了统治地位。

从总体上看,热质说自然是一种不正确的理论,但其中仍
包含有一些合理的东西。以热质说为基础的研究,在某些范
围内还是取得了很大的成果,这是因为在某些现象中热量守
恒是成立的。比如,在热量守恒的前提下,热传导的理论研
究和气体比热的测定都有了明显的进展;卡诺(1769—1832)
提出自己的理论时,最初也是站在热质说的立场上展开讨论
的。总之,历史上热质说的出现并在一段时间内得到大多数
科学家的承认,都不是偶然的,它反映了人们对热的本质的认
识经历了一个曲折的过程,19世纪中叶,能的转化和守恒定律
的建立,否定了热质说,为分子运动论的发展开辟了道路。分
子运动论取代热质说的史实表明,人们正是在不断地进行科
学实践中,抛弃原有的不正确的理论,保留其中的合理因素,
才建立起了更接近事实、更加完善的科学理论。

\subsection{能的转化和守恒定律的建立}
在18世纪40—50年代,焦耳(1818—1889)、迈耶(1814
—1878)、亥姆霍兹(1821—1894)等一批科学家,几乎在同
一时期内各自独立地提出了能的转化和守恒的观点,尽管他
们各自研究的范围和深广度并不一样,但在核心问题上的主
张却是相同的。这就意味着能的转化和守恒定律的发现,确
有某种时代因素在起作用。这些因素中,重要的是:
\begin{enumerate}
    \item 经过许多科学家的努力,经典力学在17—18世纪取
得了巨大成就。在经典力学中,已经蕴含着机械能的转化和
守恒的初步思想.18世纪90年代,伦福德(1753—1814)等
人对摩擦生热的实验进行了研究,否定了热质说,揭示了机
械能与物体内能变化的联系,从18世纪末到19世纪40年
代,随着物理学研究范围的扩大,陆续发现了许多现象相互
联系、相互转化的事实.例如,1800年发明了电池,不久就发
现了电流的热效应、磁效应、化学效应以及电磁感应现象等。
跟发明电池同年,还发现了红外线,经过对红外线的研究,人
们又了解到光能转化为内能的情况,在其他方面,如生物学
界,也发现了动物的体温和进行机械活动的能量跟它摄取的
食物的化学能有关。这一切,都为能的转化和守恒定的发
现作了必要的准备。
\item 18世纪中叶以来,在产业革命的推动下,生产技术
有了很大进步,社会对动力的需求也日益增多,尤其是蒸汽机的发展和迅速普及,促使人们普遍关心能的转化问题和提高
动力机的效率问题。正是这样的时代要求,把一些科学家和
工程技术人员引向了能的转化和守恒的研究。例如,焦耳开
始研究能量守恒的目的就是为了提高发动机的效率;而伦福
德是从事火药和武器研究的军官。
\end{enumerate}

以上表明,能的转化和守恒定律的建立,既是时代的要
求,也是自然科学本身发展到一定阶段的必然结果。

据一些历史资料介绍,首先了解和提倡能的转化和守恒
思想的,大都是些年轻人,而且他们的主要职业兴趣都不在物
理学方面.这些人中,迈耶是医生,28岁;亥姆霍兹是生理学
家,32岁;焦耳是酒厂主,25岁;卡诺也是工程师,34岁;伦福
德的年龄最大,45岁。这些情况,如教师认为有必要,可以向
学生介绍,并引导他们从中获得有益的启示。

\subsection{中微子与能量守恒定律}
自1914年直到30年代初,在$\beta$衰变的研究中,最令人困
惑不解的就是$\beta$粒子的连续能谱问题。

实验研究的结果表明,在$\alpha$放射性元素衰变过程中放出
的$\alpha$粒子能量值都是分立的.例如,镭($^{226}_{88}{\rm Ra}$)放出的$\alpha$粒
子有两种不同的能量值,铋($^{263}_{83}{\rm Bi}$)放出的$\alpha$粒子有六种不同
的能量值,钋($^{212}_{84}{\rm Po}$)放出的$\alpha$粒子有四种不同的能量值。这
就是$\alpha$粒子的不连续能谱。这种情况跟原子发出的光谱很相
似,人们由此认识到原子核内部也有能级存在,$\alpha$粒子的能谱
是跟核的能级分布相联系的。

我们以$^{226}_{88}{\rm Ra}$的衰变为例稍微详细地说明一下这个问
题。在
$^{226}_{88}{\rm Ra}\to ^{222}_{86}{\rm Rn}+^{4}_{2}{\rm He}$的衰变反应中,氡核有两个不
同的能级,一个是基态,一个是激发态,当镭核衰变为基态的
氡核时,放出的$\alpha$粒子能量较高:$E_{\alpha_1}=4793$兆电子伏;当镭
核衰变为激发态的氡核时,放出的$\alpha$粒子能量较低:$E_{\alpha_2}=
4.612$兆电子伏.由放出的$\alpha$粒子的能量$E_{\alpha}$, 应用公式$E_d=\left(1+\frac{m}{M}\right)E_{\alpha}$(式中的$m$和$M$分别为$\alpha$粒子和剩余原子核的质
量),可以求出与之对应的衰变能$E_d$(即发出的$\alpha$粒子的能量
与剩余原子核的反冲能量之和):$E_{d_1}=4.879$兆电子伏,$E_{d_2}=
4.695$兆电子伏.这两个衰变能的差值$\Delta E_{d}=E_{d_1}-E_{d_2}=0.184$
兆电子伏就是氡核两个能级间的能量差。当氡核由激发态跃
迁到基态时应该放出能量为0.184兆电子伏的$\gamma$射线
(图2.3),在实验中果然观察到Ra的$\alpha$衰变过程中伴随有
这种$\gamma$射线。所以从$\alpha$放射性的能谱可以获得原子核能级
的数据。
\begin{figure}[htp]
    \centering
\begin{tikzpicture}[>=latex]
    \node at (0,3){能量(MeV)};
\draw[very thick] (0,2)node[left]{4.879}--node[above]{Ra}(2,2);
\foreach \y/\ytext in {-0.5/0.184,-1/0}
{
    \draw[very thick] (3,\y)--(5,\y);
    \draw[dashed] (0,\y)node[left]{\ytext}--(3,\y);
}
\draw[->](4,-.5)--node[right]{$\gamma$,\; 0.189{MeV}}(4,-1)node[below]{Rn};

\draw[->,thick](.8,2)--node[left]{$\alpha$,\; 4.793MeV}(3.1,-1);
\draw[->,thick](1.5,2)--node[right]{$\alpha$,\; 4.612MeV}(3.5,-.5);
\end{tikzpicture}
    \caption{镭的$\alpha$射线与氡核的能级}
\end{figure}

$\beta$衰变中的电子($\beta$粒子),也是由原子核放出来的。根
据原子核内部的能级分布,有理由认为衰变中放出的$\beta$粒子
的能量也应该是不连续的,然而事实与人们的期望相反,$\beta$粒
子具有连续能谱.图2.4是实验测得的$\beta$能谱的一般情形。
\begin{figure}[htp]
    \centering
\includegraphics[scale=.4]{fig/2-4.png}
    \caption{}
\end{figure}

从这个能量分布曲线可以看到:衰变中放出的$\beta$粒子能量有
一个最大值$E_m$; 分布曲线的极大值对应的能量(表明放出的
该能量的$\beta$粒子数最多)约等于$E_m$的$1/3$。
可见,$\beta$粒子能量
的平均值要比$E_m$小。另一方面,实验测量的结果证明了$E_m$
正好等于理论上计算出的衰变能$E_d=(M_2-M_{2+1})c^2$. 式中
的$M_2$和$M_{2+1}$分别为衰变前后原子的质量.

$\beta$粒子能量的连续分布,使当时的物理学面临了严重的
困难:既然衰变过程中放出的粒子能量小于初态和终态之间
的能量差,那么失去的那部分能量到底哪里去了呢?经过多年
的研究,仍然不能揭开失去的能量之谜,致使一些著名的物理
学家(如玻尔)也准备在核领域中放弃能量守恒定律。在看来
毫无希望之时,泡利于1930年提出了一个大胆的设想:如果认
为在$\beta$衰变过程中,同时产生了一种未被察觉粒子,它带走
了一部分衰变能,上述矛盾就可以解决了。为什么这种粒子
没有被记录下来呢?是因为它不带电荷、没有静止质量,很难
跟其他粒子发生作用,因而观察不到它的出现所引起的效应。
泡利把这种粒子称为“中子”。1934年,费米为了把这种粒子
跟存在于原子核内的中子相区别,把它命名为中微子。$\beta$衰
变实际上是核内中子放出一个电子和一个中微子(后来知道
是反中微子)衰变为质子的过程:${\rm n}\to {\rm P}+^0_{-1}{\rm e}+\bar{\nu}$. 中微子的
假设,既保持了能量守恒定律又能圆满地解释$\beta$衰变的连续
能谱,使诞生不久的核物理从困境中摆脱出来。

尽管中微子的假设很成功,因为它跟所有粒子之间的作
用都非常弱(它可以毫无困难地穿过地球),所以要截获它证
明它的存在,是非常困难的,只有利用核反应堆中裂变产物
放出的强大反中微子流和高效率的探测装置,才能看到中微
子被原子核吸收时产生的效应,1956年人们终于在精心设
计的实验中证实了中微子的存在。实验的安排大体如下:

在反应堆附近,用一罐含有镉的水做质子靶,水罐外用一
层液态闪烁物做探测器。一旦反应堆中放出的反中微子被水
中的氢原子核吸收,就会产生反$\beta$衰变:${\rm P}+\bar{\nu}\to {\rm n}+ ^0_{-1}{\rm e}$, 放
出中子和正电子。正电子几乎立即会碰到电子湮成两个光
子。中子在水中穿行几微秒后,也会被镉原子核所俘获,产生
一个较重的处于激发态的镉原子核.这个核会发出3—4个
$\gamma$光子,释放出8兆电子伏的能量.探测器依次探测到电
子-正电子湮灭和镉激发核衰变时放出的光子,就证明了反$\beta$
衰变的发生和反中微子的存在。

\subsection{我国能源建设的成就简介}
1985年,我国煤产量已达85000万吨,由过去居世界第
三位上升为第二位;石油产量已达12500万吨,比1980年增
加1900万吨;发电量已达4073亿千瓦时,比1980年增加
1073亿千瓦时,我国在开发和利用新能源方面也有明显的
进展。例如,“六五”期间核电厂(站)的建设已经起步。目前
浙江秦山核电厂和广东大亚湾核电站的建设工程正在顺利
进行。




