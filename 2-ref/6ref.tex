\chapter{电场}
\section{教学要求}
这一章讲授静电学,从电菏在电场中受力和电荷在电场中具有能量两个角度出发来研究电场的基本性质。本章内容是电学的基础知识,也是学习后面各章的准备知识。

基本概念多而且抽象,是这一章的突出特点,针对这个特点,教材注意从具体情况出发引入概念,而不过于强调抽象的论证;注意加强演示实验,力求使学生获得感性知识;注意讲清楚概念的物理意义。这一章的另一个特点是许多知识要在力学知识基础上学习,教材在内容选择、确定讲述方法时注意了这个特点,希望教学中也给予注意,把新旧知识联系起来。

这一章的教材内容,是从两种电荷出发,学习电荷守恒定律和库仑定律,以此为基础,认识表征电场的力的性质和能的性质的物理量-电场强度和电势,以及它们之间的相互联系,构成较完整的静电场基础知识,作为上述知识的应用,学习了电场中的导体、带电粒子在电场中的加速和偏转、密立根实验。作为基础理论的引伸,学习了电容的概念及电容器的连接。

这一章教材可分为四个单元:第一单元包括第一节到第五节,讲述库仑定律和电场强度,第二单元包括第六节到第十节,讲述电势能、电势和电势差跟场强的关系。第三单元包括第十一节和第十二节,讲述带电粒子在电场中的运动,介绍密立根实验,第四单元包括第十三节到第十五节,讲述电容器及电容器的连接。

下面对这一章的教学内容作些具体说明。库仑定律是这一章的基础。教材详细介绍了库仑定律的实验,目的是使学生了解一些重要的物理实验方法,活跃学生的思维。教材介绍了电介质中的库仑定律,只要求学生了解什么是电介质,以及电介质中电荷间的作用力比真空中小,而不给予微观上的解释。

电场的概念,对于学生是个新概念,开始只要求大体有所了解,不要求深入地解释电场为什么是物质的一种特殊形式。电场强度和电势就是从电荷在电场中受力并具有能量这一客观事实的基础上,建立的两个重要的表征电场性质的物理概念,通过学习这两个概念以及它们的相互联系,使学生对电场的物质性有进一步的认识,由于这两个概念较抽象,又是这一章的两个教学难点,因此,在讲解电场强度之前,教材讲述了检验电荷在电场中不同位置所受的电场力的大小和方向不同的实验,使学生先对电场的强弱有所认识,便于引入电场强度的概念,对于电势的讲述,教材结合具体的电场定性地说明电势能跟电量的比值为一恒量,从而引入电势的概念,不要求作定量的论证,其目的是为了使学生容易接受一些,为了把场强和电势这两个难点分散开,在它们之间,安排讲授电场中的导体。

教材经过分析推导得出带电粒子在电场中的侧移距离和偏角公式,但不要求学生记忆这些公式,应该注意培养学生运用物理规律对具体问题进行具体分析的能力。讲述这一内容时,还要综合运用力学和电学知识,有利于发展学生综合运用知识的能力,教学中要予以重视。

密立根实验是物理学的经典实验之一,教材介绍了这个实验的基本思路,限于学生知识基础,不要求进一步讲解密立根是如何测定油滴半径的,不要求讲解密立根实际所做的实验。这个实验只讲给学生,不要求实际去做。

电容的概念比较抽象。教材讲述了平行板电容器的电容,并给出了公式,其目的在于让学生领会电容是由电容器本身的因素决定的,不要求用公式进行定量计算。

教材介绍了静电的应用及其危害,是为了加强理论知识与实际的联系,以利于扩展学生眼界。

这一章的教学要求是:
\begin{enumerate}
\item 了解电荷守恒定律,掌握库仑定律,能够计算点电荷间的相互作用。
\item 了解电场的概念,理解电场强度和电力线,掌握电场强度的公式和单位,了解匀强电场的特点。
\item 理解导体处于静电平衡时的特点。
\item 理解电势能、电势、电势差的物理意义,了解等势面.掌握匀强电场中场强和电势差的关系。
\item 掌握带电粒子在电场中的运动规律,能够分析解决加速和偏转方面的问题。
\item 了解密立根实验的简单原理和测定基本电荷的意义.
\item 理解电容器的电容概念、掌握电容器串联和并联的公式。
\item 了解静电的应用.
\end{enumerate}

\section{教学建议}
\subsection{第一单元}
\subsubsection{两种电荷}

教学中首先要利用实验演示电荷的种类、电荷的相互作用、电荷的相互增强和减弱、抵消等现象。比如,用一绝缘线悬挂一导体球,使其带上某种电荷。用毛皮摩擦后的橡胶棒接近它时,相互排斥;用丝绸摩擦后的玻璃棒接近它时,相互吸引。这说明橡胶棒和玻璃棒上带两种不同的电荷.再通过讨论问题:
\begin{enumerate}
    \item 什么叫物体带电?电荷有哪两种?它们之间有怎样的相互作用?
    \item 什么叫电荷的相互增强和中和?
    \item 摩擦起电的过程是怎样的?
\end{enumerate}
使学生回顾和复习初中的静电知识。

然后,结合演示实验讲解什么是静电感应和感应起电,这里只说清楚现象,理论解释留在电场中的导体一节讲述。至此,可以归纳已学过的起电的各种方法(接触起电、摩擦起电和感应起电)的特点,使之对起电过程,即使物体中电荷分离的过程有较深的理解。

\subsubsection{电荷守恒定律}

充分运用前面演示实验的事例:课本
图6.1所示的电荷相互增强和中和的现象,摩擦起电的过程,课本图6.2所示的静电感应现象的分析;说明以下问题:电荷的中和是否正、负电荷都消失了?物体呈中性,是否说明物体中没有电荷?感应起电中,是否导体两端新产生了正、负电荷?从而得出荷守恒定律的内容,强调在电荷的分离和转移的过程中总量保持不变。

\subsubsection{库仑定律}

从初中对电荷间相互作用的认识过渡到对相互作用力的定量研究,首先就要弄清楚电荷间相互作用力的大小和方向。库仑定律就是反映这种关系的物理规律,是电学的基础规律。

点电荷的概念,可以从质点的概念出发来理解。指出这是一个理想化模型,明确在哪种实际情况下,可以把带电体科学地抽象为点电荷,强调研究点电荷间的相互作用,是库仑定律成立的前提条件。

得出库仑定律,是以库仑扭秤实验为基础的,通过教学,不仅可以了解库仑定律的建立过程,而且能学到一些物理实验的方法,这个实验不要求实际去做,但应运用模型或挂图,以增强教学的直观性。同时,要引导学生联想卡文迪许实验验证万有引力定律的过程,以帮助学生了解库仑扭秤实验装置和原理。教学中可采用讲解或自学讨论的方法,如果用后一种方法,可提出以下问题:库仑是用什么办法测算出点电荷间的作用力的?当时还没有电量的单位,库仑是怎样解决作用力与电量的关系的?让学生带着问题阅读教材,展开讨论,明确库仑扭秤实验的过程。当然,要指出两个金属球必须是大小相同的,但不必说明理由。

对库仑定律的内容,要把文字表述和数学表达式结合起来理解,还要把库仑定律和万有引力定律作对比,以利于记忆和应用,要再次强调定律成立的条件是点电荷,指出它的适用范围可以推广到静止的电荷与运动电荷之间的相互作用(例如原子核对运动电子的作用);运动电荷间的相互作用则不适用了。

电介质中的库仑定律,可以用两个相互作用的带电小球间插入一电介质(如塑料板)后,其作用力减小的现象推理得出:如果这电介质充满空间,两个带电小球的相互作用力比在真空中的要小,这样对公式$F=kQ_1Q_2/\varepsilon r^2$的理解就会容易一些,对电介质的意义和介电常数,只作介绍,不作深入探讨。

在应用定律进行运算时,电量要用绝对值代入,力的方向由是引力或斥力具体确定,公式中各物理量的单位都统一使用国际单位制的单位。

在解决多个点电荷间作用问题时,要注意每两个点电荷间就有一对库仑力,它们遵从牛顿第三定律。

在解决综合力学问题时,带电体不仅受到库仑力,还可能受到万有引力(重力)、弹力、摩擦力的作用。教材的例题计算结果说明,在研究微观粒子的相互作用时,库仑力比万有引力大得多,因而万有引力可以忽略;但在其他带电体的平衡或运动的问题中,是否可以忽略万有引力(重力),应视具体情况而定。

以上各点,应结合例题、习题的教学,尽可能启发学生自行得出,使之易于理解、记忆和应用。

\subsubsection{电场和电场强度}

电场是学生新接触的抽象概念,教
学时要加强直观性,做好课本图6.4甲的演示实验,可以采用讲解法,与磁极间的相互作用比较,由学生已有的磁场概念形成对电场的认识,再结合对电磁场、电磁波及其应用于实际的广播、电视的介绍,帮助学生理解电场是一种特殊的物质。并指出电场的物质性的表现之一,就是它对其中的电荷具有力的作用。

关于电场强度的引入,首先要讲清检验电荷的意义。为了感知电场的存在及其性质,要用检验电荷进行探测,只有点电荷,由它来确定电场中某点的位置才有确切的意义;只有带电量很小,它自身的电场对源电荷电场的影响才能忽略。然后,做好课本图6.4甲所示的实验,定性地说明检验电荷在电场中的不同位置,所受电场力大小不同,进而说明电场各点的强弱不同,为此引入电场强度这个物理量。

讲解电场强度的定义时,可提出“能否用电荷所受电场力的大小表示电场的强弱”这一问题来思考,引导学生回忆密度的概念:对某种物质,体积大的质量大,用单位体积的质量、即密度来表示物质的这种特性;在电场中某点,电荷的电量大受到的电场力就大,用单位电量的电荷受到的电场力、即电场力与电量的比值,表示电场的力的性质——电场的强弱,再利用课本图6.4乙,以点电荷电场中的$A$点为例,分析电场力随电量的变大而增大,概括出电场力与电量的比值相同;对不同的点则比值不同,由此抽象出这个比值与检验电荷的电量无关,表示电场本身的性质:在比值大的点电场强,在比值小的点电场弱。并把点电荷电场的定义推广到任何电场。

场强的方向,可以在课本图6.4乙的基础上来说明,以$+Q$为圆心、以$r_1$为半径作圆,在圆上取任一点$D$, 说明$A$、$D$两点场强的大小相等,同一电荷分别在这两点所受的电场力力大小相等,但力的方向是不同的。从而说明场强是有方向的,是矢量,然后再进一步指出,场强的方向是怎样规定的。由于场强是矢量,对几个电场的叠加,其合场强要用平行四边形法则进行矢量运算;对此,只要求简单介绍,不要求定量计算。

可以把电场强度与电荷所受的电场力,从意义、公式、方向、单位等方面用列表的形式,加以比较,加深对场强的理解。还应该通过解答习题,帮助学生认识公式$E=F/q$与公式$E=kQ/r^2$和$E=kQ/\varepsilon r^2$的区别和联系。

\subsubsection{电力线}

先通过讲解,明确引入电力线可以形象地表示电场分布。再演示悬浮在蓖麻油中的木屑在各种电场(正、负点电荷的电场,两个等量异种电荷和等量同种电荷的电场)中排成一系列直线或曲线的情况。再说明电力线是人为设想的,不是真实存在的。最后讲述电力线的定义,说明线上各点的切线方向,与该点的场强方向一致。

匀强电场是最简单而又很重要的电场,要在演示木屑微粒在带正,负异种电荷的平行板间的分布形状的基础上,讲清什么是匀强电场?它的场强的特点和电力线分布的特点是什么?实际的匀强电场有哪些?

\subsubsection{电场中的导体}

教材先用分析推理的方法,讲述导体在电场中的一些基本性质,然后用实验验证,希望能发展学生的推理能力。为了激发兴趣、设置悬念,加强直观,可以先演示带电小球在电场中受力、小球罩上金属空腔后又不受电场力的现象。再提出问题:电场中的金属空腔为什么能起这样的作用:把研究金属空腔,转变为研究电场中的导体。

引导学生复习金属导体的组成和导电的微观机制,有层次地讲解:
\begin{enumerate}
\item 导体在电场$E$中,内部的自由电子受电场力作用作逆电场方向的定向移动,导致垂直于场强方向的两端面出现等量异种电荷;
\item 同时,这种电荷将在导体内形成与外电场$E$方向相反的附加电场$E'$;
\item 在$E'<E$时,自由电子的定向移动按原方向继续进行,两端面积累的电荷增多,附加电场$E'$增强;
\item 当$E'=E$时,导体内部自由电子的宏观定向移动停止,处于动态平衡状态,两端面积累的电荷稳定;由于$E$与$E'$叠加的结果,导体内的合场强为零,因而没有电力线分布。
\end{enumerate}
在以上的讲解的基础上,说明静电平衡状态的意义,以及在这种状态下场强的特点。

导体在静电平衡时其表面上任一点的场强方向与导体表面垂直和电荷只能分布在导体表面,教材都是用反证法来说明的,学生不易理解;教师既要讲清知识,又要讲清思路。之后,再用演示实验验证。















