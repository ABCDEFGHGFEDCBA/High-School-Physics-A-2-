
\chapter{分子运动论基础}
\minitoc[n]

\section{教学要求}
这一章介绍分子运动论的基本观点,它贯穿在整个热学
教材中,是基础性的一章。为了帮助学生顺利地进入热学学
习,教材一开始就指出热现象的含义,以及热学的研究对象、
研究方法。教材中讲到了研究热现象的两种不同方法,是为
了便于学生今后的学习,只要学生有个初步印象就可以了,不
宜过多讲解。

这一章讲述的内容,在初中物理中大都做过初步介绍,这
里,更加强调了分子运动论的实验基础。目的是使学生认识
分子运动论是在实验基础上建立起来的,而不是人们凭空想
象出来的。讲述这些实验,要注意引导学生了解人们是怎样
经过实验进入分子世界的;知道人们进入分子世界的线索,这
对学习物理知识特别重要。为了强调实验基础,有些实验初
中虽然做过,这里仍做了叙述。建议在教学中也重做一下这
些实验。

第一节讲述分子运动论的建立,是为了让学生了解这一
学说的建立决非轻而易举,是人类经过长期探索研究的结果,
教材中提到的历史事实,不要求作过细的讲解。

在第一节中教材指出:“按照分子运动论,热现象是大量
分子无规则运动的表现,温度表示分子无规则运动的激烈程
度,热能是大量做无规则运动的分子具有的能”。这里并不要
求展开讲,只要求学生先有个概括的了解,以便把分子运动与
热现象联系起来。在未讲内能之前,教材仍沿用初中用过的
热能的提法。关于温度的微观解释以及内能的概念,将在下
一章讲述。

第二节讲述分子的大小和阿伏伽德罗常。能够定量地
测出分子的大小,这就为分子运动论打下了坚实的基础,测
定分子的大小要根据分子运动论所推导出来的某些关系,这
些关系把宏观量与分子的大小直接联系起来。但在中学不好
讲述这些关系,所以教材选择了早期粗略测定分子大小的一
种方法-油膜法。这种方法容易为学生理解。利用宏观实
验测定微观量的大小,这是第一次。可向学生指出,微观量的
大小都是通过宏观测定得出的,明确这一点,可以使学生了解
人类进入微观世界的线索,打开学生探索微观世界的思路,这
对他们今后学习物理是很重要的。学生在化学课中已经学过
阿伏伽德罗常数,这里要求他们进一步体会这个常数的重要,
即它是联系微观世界与宏观世界的桥梁。

第三节讲解布朗运动,要使学生明确:布朗运动是微粒
的运动,而不是分子本身的运动;布朗运动的无规则性,反
映了液体内部分子运动的无规则性。布朗运动是由“涨落”现
象造成的。没有“涨落”,就没有液体分子对微粒撞击作用的
不平衡性;而微粒越小,由“涨落”所造成的不平衡性越明
显。这方面的道理,对中学生很难讲清楚,因此在教学中不要
求深入讲解,可通过一定的比喻,使学生初步了解微粒越小,
撞击作用的不平衡性越明显就可以了。

第四节讲述分子间的相互作用力。要明确指出:分子间
是同时存在着引力和斥力的。对于引力、斥力、合力怎样随距
离而变化,教材采取课本13页图1.7甲、乙两图对照的办法
来说明,以期学生能够清楚地了解甲图中曲线所表示的意
义,要使学生注意了解在分子间的距离大于和小于$r_0$时,引
力和斥力随距离变化的不同特点,这是认识合力怎样随距离
而变化的关键。至于分子间相互作用力的起源,在中学阶段不
可能讲解这个问题,因而教材只是指出分子力是由分子、原
子的带电粒子之间相互作用引起的,而不再进一步说明。

在本章最后,教材根据分子运动论简要说明了气、液、固
三种物态的情况,目的是使学生对此先有个一般了解,便于今
后学习。这里并不要求更多地讲解。

这一章的教学要求是:

\begin{enumerate}
\item 掌握分子运动论的基本内容,了解它的实验基础,了
解人们进入分子世界的线索。
\item 了解测定分子大小和阿伏伽德罗常数的方法,了解阿
伏伽德罗常数是联系微观世界和宏观世界的桥梁,会用这个
常数进行计算。
\item 了解什么是布朗运动以及布朗运动是怎样产生的,理
解布朗运动的无规则性反映了液体内部分子运动的无规
则性。
\item 了解分子间作用力的特点,了解分子间的引力、斥力以及它们的合力随分子间距离而变化的情形。
\end{enumerate}

\section{教学建议}
分子运动论的基本要点,学生虽然在初中学过,但本章对
这一理论在实验基础、定量分析和研究方法上的阐述都比初
中内容丰富、深入,要求提高了,这一点,教学时应提醒学生
注意。

\subsection{全章引言}
在引言教学中可以指出:
\begin{enumerate}
    \item 热现象与人
们生活、生产的关系非常密切,从远古时代起,人类就利用热
能来为自己服务。我国古代就有燧人氏“钻木取火”的历史传
说。正是火的发现和使用,使得古代人有可能利用热能改变
周围环境和生产生活的条件,促进了人类自身的发展.
\item 力
学现象只涉及物体的机械运动,不涉及温度和物态变化;热现
象则与温度和物态变化有关。热现象虽与力学现象不同,但
在实际中二者往往同时出现,而且同一物体,随着温度的变
化,它的某些力学性质如弹性、硬度等也要改变。如果只从力
学的角度考察现象,就不能解释这些力学性质的改变跟什么
因素有关.  
\item 热现象的理论,是各种热机、致冷设备以及喷
气式发动机和火箭工作原理的基础,在工农业生产、科学
术、医药卫生、日常生活等许多方面,都有广泛的应用,所以学
好热学是很必要的.  
\item 研究热现象的两种不同方法,是互为
补充,相辅相成,互相促进的。根据经验事实总结出的热现象
的宏观规律,需用分子运动论阐明其微观机理,而按照分子运
动论对热现象所作的解释,又需用热现象的宏观规律加以检
验,判断其是否正确。正因为人们从宏观和微观两个方面对热
现象进行探索,从而推动了热学的研究日益深入发展.
\item 这
一章和下一章将初步介绍热现象的微观理论和宏观理论的基
本内容,是整个热学的基础。
\end{enumerate}

\subsection{分子运动论的建立} 原子理论的萌芽产生于2000多
年前的古希腊时期。此后虽然经过了许多年,但因中世纪的
封建统治,生产和科学发展缓慢,物质结构的学说也就长期
没有得到发展,直到17—18世纪,由于产业革命的推动,蒸
汽机得到改进和普遍使用,使得提高热机效率成为社会的迫
切要求,从而捉进了热学的发展,促使人们开始探讨热现象的
本质,于是出现了定性的分子运动论学说。然而这个学说在
当时并未得到公认,人们普遍相信的是热质说.18世纪末
已有一些实验事实(例如,下一章阅读材料中讲的伦福德实
验),动摇了热质说的基础,特别是19世纪中叶,建立了能的
守恒和转化定律彻底否定了热质说,为分子运动论的发展开
辟了道路。以后,定量而系统的分子运动论迅速发展起来,到
本世纪初期达到了比较完善的地步.在200多年的漫长岁月
中,许多科学家为分子运动论的建立作了不懈的努力。历史
事实表明,人类对物质结构的认识和科学的发展,要受到包括
社会条件在内的多种因素的制约,是在曲折的道路上发展
的。一种科学理论的建立,必须经过长期的积累和客观事实
的检验,绝非轻而易举的事。通过本节的教学,应该使学生对
此有所体会。

在本节中,教材还对分子运动论的要点以及分子运动跟
热现象的联系,作了概括性的叙述。这些都是初中讲过了的。
建议让学生自己先作一番回忆,然后再阅读课文进行对照 这
样做,既可使学生对已学过的内容印象更深些,又有利于培养
学生的归纳、概括能力。

\subsection{分子的大小} 分子的大小是由实验测出的。测定分子
大小的实验,是建立分子运动论的重要基础,也是人类定量研
究分子世界的开端。因此,本节教学的首要问题是做好利用
油膜法粗略地测定分子大小的演示实验(参看本章的实验指
导),有条件的学校可让学生亲自做一做这个实验,使他们对
分子大小的数量获得深刻印象,并对通过宏观测定求微观
量的大小有点实在的感受。为使学生在探索微观世界的思路
方面受到启示和便于今后学习,可以指出,在本册和第三册教
材中还会遇到类似的测定。这表明,微观世界的信息都是通
过宏观测定得来的。

在考虑分子的大小时,“把分子看作小球,是分子运动论
中对分子的简化模型”。这一点要引起学生足够注意。要让
学生知道研究物理问题既要依靠实验,又要善于思考,利用
理想化模型来分析研究对象,就是进行物理思考的一种重要
方法。自然界中各种现象总是包含有多种因素,涉及许多方
面的关系,其实际状况往往十分复杂。如果不是根据问题的
性质和需要把研究对象加以简化,建立起与实际情况近似的
理想化模型,我们对物理现象就很难作定量分析与深入研究,
也就不能形成科学概念,找出规律,至于同一对象随着研究
范围和条件的变化,需要建立不同的模型(如本册教材第三章
讨论气体分子运动时,又有理想气体模型),则是灵活运用理
想化方法处理问题的表现,所以习惯于利用模型来思考分析
问题,对学好物理,培养思维能力和灵活运用知识的能力,都
是很重要的。

\subsection{数量级}
教学时,可向学生说明:用10的乘方来表示
很大或很小的物理量、物理常数的测量结果,是物理学中一种
习惯的科学记数方法、通常叫数量级,例如,地球的质量是
$5.98\x10^{24}$千克,我们就说地球质量的数级是$10^{24}$千克,真
空中光的速度是$3\x10^8$米/秒,光速的数量级就是$10^8$米/秒.
对于不少的物理量和物理常数,由于测量原理不够完善或测
量技术的限制,只能量得它的大致范围,或者只需要知道它的
数量级来进行估算,在这种情况下,就可只说出它的数量级。
比如,说分子直径有多大,就应强调它的数量级是$10^{-10}$米,这
是因为分子直径大小的具体数值,是在采用简化模型(把固
体、液体的分子看作一个挨一个排列的小球)的情况下求得
的。换句话说,这是对分子大小的粗略估算 所以重要的是知
道分子大小的数量级,形成一个数量观念,而不必花过多的精
力去讨论各种分子大小的具体数值。

学生对分子大小的数量级往往印象不深,为此,在作练习
一第(3)题之前,可让他们先猜一猜排满一米长所需的分子个
数,再看猜想的数目跟$10^{10}$个相差多少.这样,学生对分子微
小程度的印象会更深刻一些。还可以让学生把分子的直径同
一根头发的直径(数量级为$10^{-5}$米)进行比较,也有助于对
分子的微小程度获得具体的认识。

\subsection{阿伏伽德罗常数}
 学生在化学中已学过阿伏伽德罗
常数,教学时可引导他们:
\begin{enumerate}
\item 复习摩尔、摩尔质量、摩尔体积
等的含义。
\item 认识阿伏伽德罗常数有多种测定方法,本章教材就讲了两种测定方法,即测出分子的大小或分子的质量都
可求得阿氏常数,到本册第八章讲过法拉第常数后,还将介绍
根据$N=F/e$测阿伏伽德罗常数的方法。\item 从阿伏伽德罗常
数把分子大小,分子质量这些无法直接测量的微观量跟摩尔
体积、摩尔质量等宏观量联系起来的事实,认识这个常数是联。
系微观世界和宏观世界的桥梁,它为人们从微观角度定量地
研究热现象提供了重要条件。
\end{enumerate}

为使学生对分子的轻、小和阿伏伽德罗常数的巨大程度
有更深刻的印象,可以举出日常生活中认为很小或很少的东
西(如1克食盐,1${\rm cm}^3$水等),让学生估计其中所含的分子
数,然后再根据计算结果来检验自己的估计、经验表明,这种
办法常常会使学生对阿伏伽德罗常数的巨大程度感到吃惊,
从而留下难忘的印象。

教学中还可引导学生联系已学过的万有引力恒量初步认
识,物理常数的存在乃是物理世界客观规律性的反映。因此
科学家们总是不断努力采用多种方法来更精确地测定这些重
要物理常数。

分子虽然看不见,摸不到,但通过实验和科学分析却能测
定分子的大小和阿伏伽德罗常数,而且用很不相同的各种方
法测出的结果彼此相符(或数量级相符),这就为分子的客观
存在和物体是由大量分子组成的提供了有力证据,为分子
运动论的进一步发展奠定了坚实基础,同时也告诉我们物理
理论彼此和谐一致,正确地反映了自然,因而它对指导人们从
事生产和科学研究具有重大作用,建议在讲过分子的大小和
阿伏伽德罗常数之后,启发学生对上述问题有所体会。

\subsection{布朗运动}
 在讲布朗运动之前,最好引导学生复习一
下初中学过的扩散现象,然后指出:扩散现象虽能说明分子
在不停地运动,但布朗运动现象则是分子永不停息地做无规
则运动最早而又最明显的实验证据。在布朗运动这样的实验
事实面前,历史上一些曾经持有不同观点的科学家,也不得不
相信分子运动论确实是正确的理论了。

本节教学能否取得较好的效果,关键在于是否做好布朗
运动的演示实验(参看本章实验指导),及启发学生通过观察、
思考、分析对实验现象获得正确的定性理解。

学生对布朗运动并不是分子本身的运动比较容易接受,
但对布朗运动是液体分子运动所造成的,是分子永不停息地
做无规则运动的反映,就常常感到费解。为此,要引导他们从
多方面思考、分析,充分认识产生布朗运动的原因不是来自外
界影响,而在液体内部.如:
\begin{enumerate}
\item 让学生认真阅读教材第10页
第2段,从中了解认为布朗运动来自外界影响的种种看法,
经过实验检验,都是不合理的臆测.    \item 依据教材的插图(一个
微粒受到液体分子撞击的情景图)和分析,引导学生认识布
朗运动是悬浮在液体中的微粒不断地受到液体分子撞击的结
果。在液体中悬浮的微粒越小,同时撞击微粒的分子就越少,
这种撞击作用的不平衡性就表现得越明显;在液体中悬浮的
微粒越大,同时撞击微粒的分子就越多,撞击作用的不平衡性
就表现得越不明显,可见,微粒的布朗运动不是液体以外的
作用所引起的.    \item 组织学生讨论练习二第(3)题,从反面设
想,证实布朗运动是液体分子无规则运动的反映.    \item 提醒学
生在实验中注意观察不同小微粒的布朗运动情况是否相同。 结合讨论分析练习二第(4)题,进一步认识外界因素不可能
造成不同小微粒的布朗运动情况不同的现象。
\end{enumerate}

总之,通过实验
和分析,要使学生知道布朗运动现象只跟微粒的大小和温度
有关,布朗运动虽不是液体分子本身的运动,但它的运动状
况完全是由液体分子的运动状况决定的,正是液体分子对微
粒的无规则撞击,迫使微粒也做无规则运动,所以说,布朗运
动是间接地、但又更明显地证实了分子的无规则运动。

在指导学生阅读教材第9页图1.5时,应当指出这个图
并不表示做布朗运动的微粒的运动轨迹,它只是记录下了每
隔30秒微粒所在位置的变化,并用直线依次把这一系列的位
置变化连接起来,就成了图中所示的“做布朗运动的微粒的运
动路线”.实际上,即使在这短短的30秒内,微粒的运动也是
无定向的、极不则的。如果把记录的时间间隔取得更短,则
连接各个位置之间的直线就会成为更复杂的折线。

布朗运动随温度升高而愈加激烈的现象,一般不容易看
清楚。建议用不同温度下扩散现象快慢程度不同的演示实验,
来说明温度越高,分子的无规则运动越激烈。

做布朗运动的微粒是由成千上万个分子组成的宏观物
体,温度也是反映大量分子集体行为的宏观物理量。但布
朗运动现象证实了分子的无规则运动,揭示了温度与分子无
规则运动的联系。从这里可以再一次启发学生,认识在宏观
实验的基础上,经过思考、分析、推理形成一定的认识,又不断
地接受客观事实和新的实验检验,使认识逐步深入,这就是人
们探索微观世界的途径和线索。

\subsection{分子间的相互作用力}
在讲分子间有空隙时可首先提出问题,如提出为什么说布朗运动和扩散现象说明了分子
间有空隙?让学生经过思考,认识到如果真是象为估算分子直
径大小而作的设想那样,固体和液体中的分子是一个挨一个
地紧密堆积起来的,那么分子就不可能从一个地方移动到另
一个地方,因而也就不会出现布朗运动和扩散现象。所以这
两个现象说明了分子间必然有空隙。

本节教材从布朗运动和扩散现象出发,经过分析、推理,
得出分子间有空隙;又根据分子间虽有空隙,大量分子却能聚
集在一起形成固体或液体,推出分子间有引力;再由分子间既
有引力,又有空隙,推出分子间还有斥力。对所推出的结论,都
用有关实验和事实给以证明,这种以实验和事实为依据,运
用逻辑推理方法,从已知探求未知的思路,值得学生认真领
会。建议按照教材安排做好证明分子间有空隙和分子间存在
引力等演示实验,并指导学生仔细阅读教材第12页1、2、3
段,理出其中论证分子间存在引力和斥力的线索,这是本节
教学应当着重突出的地方。

要向学生强调指出:分子间的引力和斥力是同时存在的。
只是引力和斥力的大小要随着分子间距离的变化而变化。至
于实际表现出来的分子力究竟是引力还是斥力,则取决于分
子间的引力与斥力的合力。分子间的相互作用比较复杂,可
指导学生阅读教材第13页图1.7甲、乙,通过对照弄清图
1.7甲的曲线表示的物理意义,以利于学生对引力、斥力、合
力随分子间距离变化而变化的情形,有较形象具体的解。在
图线教学中,要提醒学生注意认识$r$大于和小于$r_0$时,引力
和斥力随距离变化的不同特点。即当$r<r_0$时,引力和斥力
都随距离减小而增大,但斥力比引力增大得更快,所以合力表
现为斥力;当$r>r_0$时,引力和斥力都随距离增大而减小,但
斥力比引力减小得更快,所以合力表现为引力,它随着距离的
增大迅速减小;当分子间距离的数量级大于$10^{-9}$米($10r_0$)时,
引力和斥力都变得十分微弱,这时分子间的作用力就可以忽
略不计了。

在此基础上,可组织学生讨论练习三第(4)题,使他们通
过讨论熟悉题中举出的粗略估算分子直径大小的方法,并在
灵活运用知识方面受到启示。还可以让学生根据分子间的相
互作用力与距离的关系解释一些现象(如在演示分子间存在
引力的实验中,为什么两块铅的端面必须是平滑的才能粘合
在一起?打碎的玻璃,为什么不能利用分子力拼接起来,使它
恢复原状?),这对培养学生分析解决实际问题的能力,是有
好处的。

\section{实验指导}
\subsection{演示实验}
\subsubsection{用油膜法估测分子的大小}

取一滴油酸酒精溶液滴到水面上,其中的油酸就会
在水面上形成单分子层油膜.油酸的分子($\rm C_{17}H_{33}COOH$)呈
长形,它的一端是$\rm COOH$, 对水有很强的亲合力,被水吸引在
水中;另一端是$\rm C_1H_{33}$, 对水没有亲合力,便冒出水面.科学家
发现油酸分子都是直立在水中的。因此,单分子油膜的厚度,
可以认为等于油酸分子的长度。

取极少量的油酸,并准确地测定它的体积,观测表
明,一小滴未经稀释的油酸在水面上散开后,形成的单分子层
油膜要占很大面积。为了在实验中能用普通面盆大小的盛水
容器来观测油酸的单分子层油膜,就需用无水酒精稀释油酸
(油酸是有机酸,不溶于水,而溶于酒精),使一滴油酸酒精溶
液中只含极少量的油酸,通常可按1:200的体积比配制油酸
溶液.即在装油酸的量筒中用移液管准确地取出1毫升油酸
注入一只200毫升的瓶内,再加酒精到刻度线,摇动瓶子,使
油酸在酒精中充分溶解。那么,每取一滴这样的油酸酒精溶
液,其中纯油酸的体积应为每滴溶液体积的$1/200$. 又用移液
管将油酸酒精溶液滴人另一只有刻度的小量筒内,数出滴满1
毫升的滴数,例如为125滴,则用同样的滴管滴一滴这种溶液
时,其中所含油酸的体积为
\[\frac{1}{200}\x\frac{1}{125}{\rm cm}^3=4\x 10^{-5}{\rm cm}^3\]

准确地测定油膜的面积,如图1.1所示,在盛水盘内
装蒸馏水约1厘米深.为便于观测油膜的面积,可在水面上轻
轻地撒一层石松粉,待粉末均匀分布后,往水盘中央滴一滴油
酸酒精溶液,于是油酸在水面上迅速散开,到油膜面积不再扩
大时,用一块玻璃盖在盘缘上描出油膜的轮廓图。然后把这
块玻璃放在方格纸上(图1.2),数出油膜面积所占的格数,算
出油膜的面积。若算得油酸的体积为$V$, 面积为$S$, 则油膜的
厚度$D=V/S$.
\begin{figure}[htp]\centering
    \begin{minipage}[t]{0.48\textwidth}
    \centering
\includegraphics[scale=.6]{fig/1-1.png}
    \caption{}
    \end{minipage}
    \begin{minipage}[t]{0.48\textwidth}
    \centering
    \includegraphics[scale=.6]{fig/1-2.png}
    \caption{}
    \end{minipage}
    \end{figure}

实验前,必须把所有的实验用具擦洗干净,实验中
吸取油酸、酒精和溶液的移液管要分别专用,不能混用,否则
会增大误差,影响实验效果。

这个实验的构思相当巧妙.实验后,可提出问题(如:
在水面上滴下油酸溶液后,若油膜布满了整个容器的水面,能
否根据这时的油膜面积求油酸分子直径?为什么?)启发学生
思考。

\subsubsection{观察布朗运动}
在布朗运动实验中,是否制备好检验液是影响实验效果
的主要因素。检验液中的微粒既要大小适度,又要均匀。若微
粒过大,布朗运动就不明显;若微粒大小不均匀,则只能看见
较小的微粒在动,而且容易跑到视野以外去。制备检验液可
拿书写用的上等松烟墨(劣质墨颗粒太大,效果不好)磨成墨
汁,用30—40倍蒸馏水冲淡,搅拌后静置几小时,大颗粒会沉
淀在下层,但上层仍含有悬浮物,需取其中间部分的液体,才
能获得大小适度、颗粒均匀的悬浊液。取一块薄载玻片,在它
上面浇一层厚度约0.5—1毫米的石蜡,再在石蜡中间挖一直
径约4—5毫米的凹坑.向坑内滴一、二滴蒸馏水,然后用铁
丝蘸一滴制备好的悬浊液放入蒸馏水中,搅拌后,检验液以
略带一点颜色为适宜(颜色太深或几乎没有颜色,均表明检验
液的浓度不合要求)。盖上盖玻片,即可放到显微镜的载物台
上去观察。

显微镜用600倍的较好.调节时应先用粗调将镜筒调下,
直到物镜下端几乎跟盖玻片接触为止,然后边观察边转动微
调旋钮使镜筒上升,直到能看见清晰的像。在调焦成像时,不
准将镜筒向下调,以免压碎盖玻片和损伤物镜。

实验时还应注意:
\begin{enumerate}
\item 显微镜头,载、盖玻片等必须干净。
任何一点污迹都会影响实验效果.
\item 盖上盖玻片时,要注意
不使凹坑内残留气泡,溢出多余的液体可用滤纸吸干.
\item 
显微镜的反光镜反射的光线应强弱合适,过强反而不易看清
楚,太弱则微粒颜色灰暗。
\end{enumerate}

指导学生观察时,可让他们先注意看一个微粒的布朗运
动,然后看不同微粒的布朗运动情况。

实验后可向学生指出,布朗运动现象虽是1827年发现
的,但当时人们并不能正确地解释它,经过30多年的研究,才
认识到布朗运动是由液体分子撞击悬浮微粒所引起的,并发
现静止气体中的悬浮微粒也在作布朗运动。

\subsubsection{扩散现象随温度升高而加快}
取两只试管,一只内装热水,另一只内装凉水,热水与凉
水的温差在$60^{\circ}{\rm C}$左右.用吸有少量红墨水的移液管分别往
两试管底部注入一些红墨水,然后将移液管慢慢从试管中取
出。可以看见装热水的试管中的红墨水比装凉水的试管中的
红墨水扩散得快些,从而说明扩散现象随温度升高而加快。

\subsubsection{酒精与水混合后体积减小}
在长1米左右一端开口的玻璃管(可用托里拆利管)内装
一半水,再将染色酒精沿管壁慢慢注入管内,直到注满玻管。
因酒精的密度比水小,这时可清楚地看到它们的分界面。然
后封住管口,把玻璃管上下颠倒几次,使水和酒精充分混合,
可以发现混合后液体的体积比混合前两种液体的总体积缩小
了。这个现象表明,酒精跟水混合后,分子重新排列,混合液
体的分子间的空隙减小了,所以总体积也就缩小了。

\subsubsection{把两个铅柱压紧,由于分子间的引力,它们就合在
一起}

做这个演示前,必须把铅柱端面的油污和氧化层刮掉,使
端面清洁、平滑。演示时应使两端面的刮纹一致,再从边缘开
始,使两个铅柱沿着相互接触
的端面相对平移,使端面吻合,
并把两个铅柱压紧(图1.3)。
\begin{figure}[htp]\centering
    \centering
    \includegraphics[scale=.6]{fig/1-3.png}
    \caption{}
    \end{figure}

为了不使学生误认为两铅
柱合在一起是大气压的作用,
可在铅柱上挂2千克左右的物
体,然后拉开端面让学生观察
压痕.铅圆柱体的截面积一般为3厘米2左右,从观察压痕
可知,两个铅柱的实际接触面积只有几个平方毫米,如果是大
气压作用的话,只要挂0.1千克左右的物体就把两个铅柱拉
开了。可见,两个铅柱合在一起并不是大气压的作用,是分
子间存在引力的结果。

\subsection{课外实验活动}
\subsubsection{观察扩散现象}

实验时,可将水和浓度较大的硫酸铜溶液先后装入
高玻璃杯中,要把玻璃杯放在不受震动的地方,使它完全处
于静止状态,以表明扩散现象是在不受外界影响的情况下发
生的。

实验的结果表明:密度较大的液体的分子会向上移
动,进入密度较小的液体;密度较小的液体的分子会向下移
动,进入密度较大的液体。从而有力地证实了分子在运动。

我们观察到的溶液的蓝颜色并不是硫酸铜分子的颜
色,而是水合铜离子的颜色。因此,硫酸铜分子的扩散,是由
水合铜离子的扩散推知的。

\section{习题解答}
\subsection{练习一}
\begin{enumerate}
\item  一般分子的直径,以厘米作单位时数量级是多大?

\begin{solution}
    是$10^{-8}$厘米.
\end{solution}

\item  把体积为1$\rm mm^3$的石油滴在水面上,石油在水面上
形成面积为3$\rm m^2$的单分子油膜.试估算石油分子的直径.

\begin{solution}
\[\begin{split}
    \text{石油分子的直径}&=\frac{\text{石油滴的体积}}{\text{单分子油膜面积}}=\frac{1{\rm mm}^3}{3{\rm m}^2}\\
&=\frac{1\x 10^{-9}{\rm m}^3}{3{\rm m}^2}=
3\x10^{-10}{\rm m}
\end{split}\]
\end{solution}
\item  设想把分子一个挨一个地排起来,要多少个分子才
能排满1米的长度?

\begin{solution}
\[    \text{需要的分子个数}=\frac{1{\rm m}}{10^{-10}{\rm m}}=
    10^{10}\]
\end{solution}
\item  1$\rm cm^3$水中含有多少个水分子?10克氧中含有多少
个氧分子?

\begin{solution}
    水分子的体积大约是$3\x10^{-29}{\rm m}^3$, 所以1厘米
水中含有的水分子个数约为
\[\frac{1{\rm cm}^3}{3\x 10^{-29}{\rm m^3}}=\frac{1\x 10^{-6}{\rm m^3}}{3\x 10^{-29}{\rm m^3}}=3\x 10^{22}\]
\[
\text{10克氧的摩尔数}=\frac{1.0\x 10^{-2}{\rm kg}}{3.2\x10^{-2}{\rm kg/mol}}=0.31{\rm mol}\]
所以10克氧中含有的氧分子个数约为
\[0.31{\rm mol}\x 6.02\x10^{23}{\rm mol}^{-1}=1.9\x10^{23}\]
\end{solution}
\item  一个氧分子、一个氮分子的质量各是多少千克?

\begin{solution}
    一个氧分子的质量
\[m_{\rm O_2}=\frac{3.2\x 10^{-2}{\rm kg/mol}}{6.02\x 10^{23}{\rm mol}^{-1}}=5.3\x 10^{-26}{\rm kg}\]
    一个氢分子的质量
    \[m_{\rm H_2}=\frac{2\x 10^{-3}{\rm kg/mol}}{6.02\x 10^{23}{\rm mol}^{-1}}=3\x 10^{-27}{\rm kg}\]
\end{solution}
\item  已经测得一个碳原子的质量是$1.995\times 10^{-26}$千克,
求阿伏伽德罗常数.

\begin{solution}
    已知一个碳原子的质量是$1.995\x10^{-26}$千克,碳的摩
尔质量取$1.200\x10^{-2}{\rm kg/mol}$,所以阿伏伽德罗常数
\[N=\frac{1.200\x10^{-2}{\rm kg/mol}}{1.995\x10^{-26}{\rm kg}}=6.015\x 10^{23}{\rm mol}^{-1}\]
\end{solution}
\item  已知金刚石的密度是$3500{\rm kg/m^3}$,有一小块金刚
石,体积是$5.7\times 10^{-8}{\rm m^3}$.这小块金刚石中含有多少个碳原
子?设想金刚石中碳原子是紧密地堆在一起的,估算碳原子的
直径.

\begin{solution}
    这一小块金刚石的质量
\[m=\rho V=3500{\rm kg/m^3}\x 5.7\x 10^{-8}{\rm m^3}=2.0\x 10^{-4}{\rm kg}\]

$2.0\x 10^{-4}{\rm kg}$碳的摩尔数为
\[\frac{2.0\x 10^{-4}{\rm kg}}{1.2\x 10^{-2}{\rm kg/mol}}=1.7\x 10^{-2}{\rm mol}\]
这一小块金刚石所含的碳原子个数为
\[1.7\x10^{-2}{\rm mol}\x6.02\x10^{23}{\rm mol^{-1}}=1.0\x10^{22}\]
一个碳原子的体积为
\[\frac{5.7\x10^{-8}{\rm m^3}}{1.0\x10^{22}}=5.7\x10^{-30}{\rm m^3}\]
把金刚石中的碳原子看成球体,则由公式$V=\dfrac{\pi}{6}d^3$,\quad $d=\sqrt[3]{\dfrac{6V}{\pi}}$
可得碳原子的直径约为
\[\sqrt[3]{\frac{6\x5.7\x10^{-30}}{3.14}}{\rm m}=2.2\x10^{-10}{\rm m}\]
\end{solution}
\end{enumerate}

\subsection{练习二}
\begin{enumerate}
\item 有人说布朗运动就是分子的运动,这种说法对吗?为
什么?

\begin{solution}
这种说法不对,因为做布朗运动的微粒是由千千万
万个分子组成的,而分子的运动我们是看不见的,但是微粒的
布朗运动的无规则性,却反映液体内部分子运动的无规则性。
\end{solution}

\item 为什么悬浮在液体中的颗粒越小,它的布朗运动越
明显?为什么悬浮在液体中的颗粒越大,它的布朗运动越不
明显以至观察不到?

\begin{solution}
悬浮在液体中的颗粒越小,在某一瞬间跟它相撞的分
子数越少,在不同方向上颗粒受到的撞击作用也就越不平
衡。同时颗粒越小,它的质量也越小,在受到液体分子撞击
时也容易改变其运动状态 所以颗粒越小,它的布朗运动越
明显。

悬浮在液体中的颗粒越大,在某一瞬间跟它相撞的分子
数越多,颗粒在各个方向受到的撞击作用越接近平衡状态。所
以颗粒越大,它的布朗运动越不明显以至观察不到。
\end{solution}

\item 为什么说布朗运动的无规则性反映了液体内部分子运动的无规则性?设想液体内部分子的运动是有规则的,比如在任何时刻所有分子都向着某个方向运动,还能不能产生布朗运动?

\begin{solution}
    因为布朗运动是由液体分子不断地撞击悬浮在液体
中的微粒而引起的,所以说布朗运动的无规则性反映了液体
内部分子运动的无规则性。

如在任何时刻所有液体分子都是做有规则的定向运
动,那么微粒的运动就会有一定的规则,因而也就不能产生布
朗运动现象了。
\end{solution}

\item  图1.5中所示的不同小颗粒的布朗运动的情况并不相同,人们由此考虑到布朗运动不可能是由外界影响引起的.为什么?找几位同学一起讨论一下,并说明你的理由.

\begin{solution}
    假若小颗粒的布朗运动是由某一确定的外界因素所
引起的,那么在同一影响下,不同小颗粒的运动情况就会相
同.但图1.5所示的情况正好跟上述假设相反,所以产生布
朗运动的原因不可能在外部,只能在液体内部。
\end{solution}

\end{enumerate}

\subsection{练习三}
\begin{enumerate}
	\item 什么事例说明分子间有引力?什么事例说明分子间有斥力?
	
\begin{solution}
    用力拉伸物体,物体内要产生反抗拉伸的弹力。把两
    块铅压紧,两块铅就合在一起。这说明分子间有引力。

    固体和液体很难被压缩,即使气体,压缩到一定程度后再
    继续压缩也很困难,这说明分子间有斥力。

    (答案中所举事例,可不限于教材上已提到过的。)
\end{solution}
	\item 当分子间的距离大于$r_0$时,随着距离的增大,引力和斥力哪个减小得快?当分子间的距离小于$r_0$时,随着距离的减小,引力和斥力哪个增加得快?
		
\begin{solution}
在前一种情况下斥力减小得快;在后一种情况下斥力
增加得快。
\end{solution}
	\item 物体为什么能够被压缩,但又不能无限地被压缩?
		
\begin{solution}
    组成物体的分子之间是有空隙的。用力压物体时,物
    体内分子间的距离缩小,就表现出物体的可压缩性。但随着
    分子间距离的缩小,斥力会迅速增大,压缩到一定程度时,斥
    力与外力平衡,物体就不能再被压缩了。
\end{solution}
\item	从图1.7看出,当分子中心间的距离小于$r_0$时,分
子间的作用力表现为斥力,它随着距离的减小而很快地增大.分子间作用力的这一特点,可以借助于下述模型想象出来.设想分子为弹性钢球,当两个钢球相撞时,它们都发生微小的形变,因而在它们之间产生相互推斥的弹力,如同分子间的作用力表现为斥力一样.钢球发生微小形变就可以产生很大的弹力,所以这个弹力随着钢球中心间距离的减小而很快地增大.利用这一模型可以粗略地估计出分子直径的数量级为$10^{-10}$米.这是怎样估计的?
	
\begin{solution}
    设想分子为弹性钢球,当两个钢球恰好接触,但并未
    相互挤压时,它们之间不发生力的作用;当两个钢球相撞时,
    它们都发生微小的形变,因而在它们之间产生相互推斥的弹
    力.从课本图1.7看出,当分子间的距离为$r_0$时,它们之间的
    作用力为零;当分子间的距离小于$r_0$时,它们之间的作用力表
    现为斥力。于是可以认为,两分子间的距离为$r_0$时,它们恰
    好接触,但并未相互挤压。已经知道,$r_0$的数量级约为$10^{-10}$
    米,因此可以粗略地认为分子的直径约为$10^{-10}$米.
\end{solution}
\end{enumerate}

\section{参考资料}

\subsection{分子运动论发展简史}

根据记载和传说,远在2500多年前,古希腊就有物质
是由某些元素所组成的假说,其中以物质的原子论最为深
刻.约在公元前462年左右—370年左右,古希腊的著名思
想家德谟克利特归纳了古代的原子论。他认为万物皆由大量
不可分割的微小物质粒子组成,这种粒子叫做原子(希腊文为
atomos, 即不可分割的意思)。按照德谟克利特的观点,各种
原子没有质的差别,只有大小、形状和位置的差异;原子不断
地在空虚的空间中运动着;世界是由运动的原子及其组合物
构成的,任何自然现象都可以用这些原子的各种组合给以解
释。德谟克利特还举出实例来说明他的学说。比如他曾经这
样解释过花的香味:从花中飞出来的原子冲进人们的鼻孔里,
于是引起了有香味的感觉。德谟克利特的原子论在古希腊后
期和古罗马时期曾经有所发展。这种古代的原子论,虽是对
物质结构的一种朴素猜测,但它的基本思想,即无限的虚空和
在其中运动的粒子,则是近代原子理论的先驱。

此后经过了许多年,物结构的学说长期没有得到发
展。在中世纪,原子论受到了宗教的非难和压制,基督教会就
曾禁止传播“世界一切是由原子构成的”这种无神论思想。

直到17—18世纪,由于产业革命的推动,蒸汽机得到改
进和普遍使用,使得提高热机效率成为社会的迫切要求,因而
促进了热学的发展,促使人们开始探索热现象的本质,于是出
现了分子运动论学说,1658年,伽森第以分子运动论的观点
解释了物质的固、液、气三态的区别 接着,胡克和伯努利等人
发展了分子运动论.1738年,伯努利在《流体动力学》一书
中,根据气体是由激烈地运动着的大量粒子所组成这一假说,
解释了气体的压力是由粒子对器壁的碰撞而产生,并通过考
察气体体积变化时碰撞次数的变化,得出压力与体积成反
比。在伯努利的论述中,包含有分子动量的变化产生压力这种
卓越的思想。罗蒙诺索夫继续发展了分子运动论,他在《关于
冷和热的原因的探讨》一文中,提出了如下的设想:构成物
体的微粒极小,因此肉眼不可能看见微粒本身的运动,但是
它的运动表现在无数的现象之中。热无非是微粒的运动而
已。尽管这些先驱者的学说中有许多正确的观点和锐敏的构
思,但因缺少定量的实验基础,没有从数学上进行理论推导,
这时的分子运动论还是处在定性的探讨阶段.而18世纪的
热学,以热质守恒为基本原理,已积累了不少实验数据,认为
热是一种运动的表现,在当时自然难于得到公认,甚至到了
19世纪30年代,在热学中占统治地位的理论仍然是热质
说。这表明,要使分子运动论为大家普遍接受,还有待于能从
更广泛的角度阐明热与其他运动形式相互转化的能的转化和
守恒定律的确立。

19世纪中叶,建立了能的转化和守恒定律,否定了热质
说,为分子运动论的发展开辟了道路。此后,定量而系统的分
子运动论迅速发展起来,经过大约半个世纪的时间,克劳修
斯、麦克斯韦、玻尔兹曼等就在前人工作的基础上建立起较为
完善的分子运动论.1857年—1858年,克劳修斯根据气体分子
对器壁产生的冲量算出了气体的压强,解释了有关的气体实
验定律,又在气体分子运动论中引入了平均自由程的概念。
1860年,麦克斯韦导出了分子运动的麦克斯韦函数分布律.
至此,气体分子运动论的思想方法已大体完备.1865年,洛喜
密特算出了分子的大小,为分子运动论的发展打了坚实的基
础。玻尔兹曼进一步研究分子运动论,与麦克斯韦共同建立
了能量均分原理。从19世纪后期到本世纪初,在玻尔兹曼、
麦克斯韦和吉布斯等的努力下,建立了把分子运动论作为一
个分支包括在内的经典统计力学,这是一门用统计方法研究
由大量微观粒子所组成的系统的科学,它能对热力学已经获
得的结果从微观角度给以深刻解释。

就在分子运动论迅速发展的时期,以奥斯特瓦尔德为首
的一些科学家曾强烈反对过原子、分子论。他们认为原子、分
子是无用的、多余的假说,一切自然现象只要看作是能量的转
换,仅从热力学的观点出发,就能得到解释,本世纪初,由于
爱因斯坦和其他科学家从理论上和实验上对布朗运动进行了
深入研究,提出了布朗运动的定量公式,这就充分证实了分子
热运动的真实性,表明分子运动论不但不是多余的、无用的假
说,而且是无可置疑的事实。从此分子运动论就成为科学家
们公认的理论了。

\subsection{离子显微镜简介}

离子显微镜,又叫场离子显微镜,是本世纪50年代后期
出现的仪器.由于它的放大倍数在$10^6$以上,而且分辨能力高
达0.2—0.3纳米,因此,利用离子显微镜可以直接观察固体
表面原子实际排列的状况,或者说可以“看到原子”。

离子显微镜的“成像”机理,课本上已经讲了,它是靠镜
内空间的氮离子在5千至3万伏的高压电场作用下离开针尖,
沿着电力线运动打到荧光屏上使之发光。这样,以氦离子作媒
介,就把针尖上的点与荧光屏上的点一一对应起来了。因为
电力线是从针尖向外辐射的,所以荧光屏上一个个分散的光
点所占的范围要比针尖的表面积大许多倍,从而起到了放大
针尖上钨原子分布图样的作用。

从微观角度看,针尖表面是一个半径很大的球面,由钨原
子紧密排列而成的球面还很不光滑,由于带电体附近的电场
强弱分布跟带电体的形状有关,在针尖表面突出的原子附近,
电场最强,氦原子也就在这里被电离,因此,荧光屏上打出的
光点是与针尖表面上处于突出位置的原子相对应的。

在离子显微镜中,氦原子被电离而离开针尖时还在做热
运动(振动),在与半径垂直的方向上(横向)还有一定的速
度。这个速度的存在,会使氨离子打到荧光屏上的光点偏离
不考虑热运动时的理论位置,研究表明,由于横向速度的影
响,氦离子到达荧光屏时要落在以理论位置为圆心的一个圆
面上。邻近的圆面可能互相重叠,使得离子像的分辨能力大
为降低,为了提高离子显微镜的分辨能力,既需要设法减小
氦离子热运动的速度,又需要提高针尖与荧光屏之间的电压
以缩短氦离子从针尖飞到荧光屏的时间。现在人们已用液氨
或液氮、液氢来冷却针状电极,使氦离子的热运动速度大大减
小;选用很难电离的氮气作成像气体,也是为了尽可能地提高
针尖与荧光屏之间的电压。

离子显微镜内要抽成真空,然后充入氦气,通常应使镜
内的真空度达到$1.33\x10^{-4}$—$1.33\x10^{-7}$帕.离子像的亮度
与氦气的压强有关,压强高,离子像相对地明亮些,但压强太
高,氦原子的密度就大,这会增加氦离子与其他氦原子碰撞而
偏离原来运动方向的机会,导致荧光屏的离子像变得模糊。
一般应将氦气压强控制在$1.33\x10^{-4}$—$1.33$帕之间.此外,
所用氮气必须有较高纯度,否则也会严重影响氦离子所成
的像。

如需进一步了解离子显微镜,可看“场离子显微镜简介”
(《物理》82年10期,作者陆华;《物理教师》1985年6卷,作
者杜敏)、“场离子显微镜技术介绍”(《物理》1983年1期,作
者陆华)等文章。

\subsection{布朗运动}
按照分子运动论学说,分子的无规则运动服从统计规律,
即分子向各个方向运动的几率相等,同时,涨落现象是无规则
运动不能避免的,据此可以说明在一定的悬浊液中布朗运动
是否明显,取决于颗粒的大小 如果颗粒的线度大于$10^{-6}$米,
其周围的液体分子对它碰撞的次数极多,在各个方向上它受
到碰撞的次数就相差不大,涨落现象就不明显,我们也就难以
观察到它的布朗运动,但是当颗粒的线度足够小(不超过$10^{-6}$
米或$10^{-7}$米)时,相对来说颗粒受到液体分子的碰撞次数较
少,碰撞作用出现不平衡的机会就随之增大,即涨落现象变得
明显,使颗粒产生移动的倾向增大。这样,颗粒将向所受冲力
的合力方向运动,但因分子运动的无规则性,颗粒在某一时
刻所受合力是偶然的,作用力在各个方向上是机会均等的,因
此颗粒的运动看起来杂乱无章而内部又蕴藏着一定的规律,
显然,颗粒的运动不是分子运动,但它和液体分子的运动规律
类同。所以说布朗运动揭示了分子的无规则运动。

顺便指出:通常,在光的照射下,我们看见空气中灭尘的
运动乃是气流所引起的,并不是悬浮在气体中的颗粒的布朗
运动。

\subsection{分子间的相互作用力}
现在,人们公认自然界有四种力,即万有引力、电磁力、弱
力和强力。在微观领域里,万有引力同其他三种力相比就显
得微不足道,它的强度只有强力的$10^{39}$分之一,电磁力的$10^{37}$
分之一,弱力的$10^{25}$分之一。因此,对于分子、原子来说,万有
引力可以忽略不计。

弱力和强力的有效作用距离约在$10^{-15}$米以内,在原子
和原子核中这两种力才起显著作用。可见,分子与分子之间
的作用力属于电磁力。

分子、原子都是复杂的带电系统,深入讨论分子间的相互
作用力,要涉及量子力学理论。这里只做最简单的定性说明。

分子间的引力作用来源于分子中的原子核对相邻分子的
电子云的静电引力,具体说来有三种情况:第一,有些物质的
分子是有极性的,即其正电荷的中心不跟负电荷的中心重合,
相当于电偶极子,一个“电偶极子”的正电端与另一个“电偶极
子”的负电端互相吸引就形成了分子之间的引力。第二,极性
分子使邻近的非极性分子极化成为感生偶极子,它跟原来的
极性分子之间的作用形成分子之间的引力。第三,即使在上
面讲的两种力都不存在的非极性分子构成的物质中(如惰性
气体),由于电子的运动,原子中也存在着瞬时电偶极矩,它在
邻近空间中激发出瞬变电场,从而在邻近分子中感生出电偶
极矩来,分子之间就产生了引力。F·伦敦于1930年证明,对
于大多数分子,三种作用中的后一种作用引起的分子引力是
最大的。

分子间的引力是很弱的,只有在分子彼此接近到几乎相
接触时,才起作用。分子之间还同时存在着斥力,它来源于相
邻分子的电子云间的排斥力和相邻分子的原子核之间的排斥
力,不过引力要比排斥力强一些,但是当分子间彼此接近到
它们的电子云发生重叠时,情况就改变了。这时重叠区域中
电子云的密度增大,电子的能也随之增大(服从泡利不相容
原理),因而发生强烈的排斥作用。另外,由于原子核相互间更
接近了,静电斥力也会增大。这时,分子间的斥力就超过引
力了。































