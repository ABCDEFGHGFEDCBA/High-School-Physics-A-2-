\chapter{分子运动论基础}
\section{教学要求}
这一章介绍分子运动论的基本观点,它贯穿在整个热学
教材中,是基础性的一章。为了帮助学生顺利地进入热学学
习,教材一开始就指出热现象的含义,以及热学的研究对象、
研究方法。教材中讲到了研究热现象的两种不同方法,是为
了便于学生今后的学习,只要学生有个初步印象就可以了,不
宜过多讲解。

这一章讲述的内容,在初中物理中大都做过初步介绍,这
里,更加强调了分子运动论的实验基础。目的是使学生认识
分子运动论是在实验基础上建立起来的,而不是人们凭空想
象出来的。讲述这些实验,要注意引导学生了解人们是怎样
经过实验进入分子世界的;知道人们进入分子世界的线索,这
对学习物理知识特别重要。为了强调实验基础,有些实验初
中虽然做过,这里仍做了叙述。建议在教学中也重做一下这
些实验。

第一节讲述分子运动论的建立,是为了让学生了解这一
学说的建立决非轻而易举,是人类经过长期探索研究的结果,
教材中提到的历史事实,不要求作过细的讲解。

在第一节中教材指出:“按照分子运动论,热现象是大量
分子无规则运动的表现,温度表示分子无规则运动的激烈程
度,热能是大量做无规则运动的分子具有的能”。这里并不要
求展开讲,只要求学生先有个概括的了解,以便把分子运动与
热现象联系起来。在未讲内能之前,教材仍沿用初中用过的
热能的提法。关于温度的微观解释以及内能的概念,将在下
一章讲述。

第二节讲述分子的大小和阿伏伽德罗常。能够定量地
测出分子的大小,这就为分子运动论打下了坚实的基础,测
定分子的大小要根据分子运动论所推导出来的某些关系,这
些关系把宏观量与分子的大小直接联系起来。但在中学不好
讲述这些关系,所以教材选择了早期粗略测定分子大小的一
种方法-油膜法。这种方法容易为学生理解。利用宏观实
验测定微观量的大小,这是第一次。可向学生指出,微观量的
大小都是通过宏观测定得出的,明确这一点,可以使学生了解
人类进入微观世界的线索,打开学生探索微观世界的思路,这
对他们今后学习物理是很重要的。学生在化学课中已经学过
阿伏伽德罗常数,这里要求他们进一步体会这个常数的重要,
即它是联系微观世界与宏观世界的桥梁。

第三节讲解布朗运动,要使学生明确:布朗运动是微粒
的运动,而不是分子本身的运动;布朗运动的无规则性,反
映了液体内部分子运动的无规则性。布朗运动是由“涨落”现
象造成的。没有“涨落”,就没有液体分子对微粒撞击作用的
不平衡性;而微粒越小,由“涨落”所造成的不平衡性越明
显。这方面的道理,对中学生很难讲清楚,因此在教学中不要
求深入讲解,可通过一定的比喻,使学生初步了解微粒越小,
撞击作用的不平衡性越明显就可以了。

第四节讲述分子间的相互作用力。要明确指出:分子间
是同时存在着引力和斥力的。对于引力、斥力、合力怎样随距
离而变化,教材采取课本13页图1.7甲、乙两图对照的办法
来说明,以期学生能够清楚地了解甲图中曲线所表示的意
义,要使学生注意了解在分子间的距离大于和小于ro时,引
力和斥力随距离变化的不同特点,这是认识合力怎样随距离
而变化的关键。至于分子间相互作用力的起源,在中学阶段不
可能讲解这个问题,因而教材只是指出分子力是由分子、原
子的带电粒子之间相互作用引起的,而不再进一步说明。

在本章最后,教材根据分子运动论简要说明了气、液、固
三种物态的情况,目的是使学生对此先有个一般了解,便于今
后学习。这里并不要求更多地讲解。

这一章的教学要求是:

\begin{enumerate}
\item 掌握分子运动论的基本内容,了解它的实验基础,了
解人们进入分子世界的线索。
\item 了解测定分子大小和阿伏伽德罗常数的方法,了解阿
伏伽德罗常数是联系微观世界和宏观世界的桥梁,会用这个
常数进行计算。
\item 了解什么是布朗运动以及布朗运动是怎样产生的,理
解布朗运动的无规则性反映了液体内部分子运动的无规
则性。
\item 了解分子间作用力的特点,了解分子间的引力、斥力以及它们的合力随分子间距离而变化的情形。
\end{enumerate}

\section{教学建议}
分子运动论的基本要点,学生虽然在初中学过,但本章对
这一理论在实验基础、定量分析和研究方法上的阐述都比初
中内容丰富、深入,要求提高了,这一点,教学时应提醒学生
注意。

\textbf{1.全章引言}\quad 在引言教学中可以指出:
\begin{enumerate}
    \item 热现象与人
们生活、生产的关系非常密切,从远古时代起,人类就利用热
能来为自己服务。我国古代就有燧人氏“钻木取火”的历史传
说。正是火的发现和使用,使得古代人有可能利用热能改变
周围环境和生产生活的条件,促进了人类自身的发展.
\item 力
学现象只涉及物体的机械运动,不涉及温度和物态变化;热现
象则与温度和物态变化有关。热现象虽与力学现象不同,但
在实际中二者往往同时出现,而且同一物体,随着温度的变
化,它的某些力学性质如弹性、硬度等也要改变。如果只从力
学的角度考察现象,就不能解释这些力学性质的改变跟什么
因素有关.  
\item 热现象的理论,是各种热机、致冷设备以及喷
气式发动机和火箭工作原理的基础,在工农业生产、科学
术、医药卫生、日常生活等许多方面,都有广泛的应用,所以学
好热学是很必要的.  
\item 研究热现象的两种不同方法,是互为
补充,相辅相成,互相促进的。根据经验事实总结出的热现象
的宏观规律,需用分子运动论阐明其微观机理,而按照分子运
动论对热现象所作的解释,又需用热现象的宏观规律加以检
验,判断其是否正确。正因为人们从宏观和微观两个方面对热
现象进行探索,从而推动了热学的研究日益深入发展.
\item 这
一章和下一章将初步介绍热现象的微观理论和宏观理论的基
本内容,是整个热学的基础。
\end{enumerate}

\textbf{2.分子运动论的建立}\quad 原子理论的萌芽产生于2000多
年前的古希腊时期。此后虽然经过了许多年,但因中世纪的
封建统治,生产和科学发展缓慢,物质结构的学说也就长期
没有得到发展,直到17—18世纪,由于产业革命的推动,蒸
汽机得到改进和普遍使用,使得提高热机效率成为社会的迫
切要求,从而捉进了热学的发展,促使人们开始探讨热现象的
本质,于是出现了定性的分子运动论学说。然而这个学说在
当时并未得到公认,人们普遍相信的是热质说.18世纪末
已有一些实验事实(例如,下一章阅读材料中讲的伦福德实
验),动摇了热质说的基础,特别是19世纪中叶,建立了能的
守恒和转化定律彻底否定了热质说,为分子运动论的发展开
辟了道路。以后,定量而系统的分子运动论迅速发展起来,到
本世纪初期达到了比较完善的地步.在200多年的漫长岁月
中,许多科学家为分子运动论的建立作了不懈的努力。历史
事实表明,人类对物质结构的认识和科学的发展,要受到包括
社会条件在内的多种因素的制约,是在曲折的道路上发展
的。一种科学理论的建立,必须经过长期的积累和客观事实
的检验,绝非轻而易举的事。通过本节的教学,应该使学生对
此有所体会。

在本节中,教材还对分子运动论的要点以及分子运动跟
热现象的联系,作了概括性的叙述。这些都是初中讲过了的。
建议让学生自己先作一番回忆,然后再阅读课文进行对照 这
样做,既可使学生对已学过的内容印象更深些,又有利于培养
学生的归纳、概括能力。

\textbf{3.分子的大小}\quad 分子的大小是由实验测出的。测定分子
大小的实验,是建立分子运动论的重要基础,也是人类定量研
究分子世界的开端。因此,本节教学的首要问题是做好利用
油膜法粗略地测定分子大小的演示实验(参看本章的实验指
导),有条件的学校可让学生亲自做一做这个实验,使他们对
分子大小的数量获得深刻印象,并对通过宏观测定求微观
量的大小有点实在的感受。为使学生在探索微观世界的思路
方面受到启示和便于今后学习,可以指出,在本册和第三册教
材中还会遇到类似的测定。这表明,微观世界的信息都是通
过宏观测定得来的。

在考虑分子的大小时,“把分子看作小球,是分子运动论
中对分子的简化模型”。这一点要引起学生足够注意。要让
学生知道研究物理问题既要依靠实验,又要善于思考,利用
理想化模型来分析研究对象,就是进行物理思考的一种重要
方法。自然界中各种现象总是包含有多种因素,涉及许多方
面的关系,其实际状况往往十分复杂。如果不是根据问题的
性质和需要把研究对象加以简化,建立起与实际情况近似的
理想化模型,我们对物理现象就很难作定量分析与深入研究,
也就不能形成科学概念,找出规律,至于同一对象随着研究
范围和条件的变化,需要建立不同的模型(如本册教材第三章
讨论气体分子运动时,又有理想气体模型),则是灵活运用理
想化方法处理问题的表现,所以习惯于利用模型来思考分析
问题,对学好物理,培养思维能力和灵活运用知识的能力,都
是很重要的。

\textbf{4.数量级}\quad 教学时,可向学生说明:用10的乘方来表示
很大或很小的物理量、物理常数的测量结果,是物理学中一种
习惯的科学记数方法、通常叫数量级,例如,地球的质量是
$5.98\x10^{24}$千克,我们就说地球质量的数级是$10^{24}$千克,真
空中光的速度是$3\x10^8$米/秒,光速的数量级就是$10^8$米/秒.
对于不少的物理量和物理常数,由于测量原理不够完善或测
量技术的限制,只能量得它的大致范围,或者只需要知道它的
数量级来进行估算,在这种情况下,就可只说出它的数量级。
比如,说分子直径有多大,就应强调它的数量级是$10^{-10}$米,这
是因为分子直径大小的具体数值,是在采用简化模型(把固
体、液体的分子看作一个挨一个排列的小球)的情况下求得
的。换句话说,这是对分子大小的粗略估算 所以重要的是知
道分子大小的数量级,形成一个数量观念,而不必花过多的精
力去讨论各种分子大小的具体数值。

学生对分子大小的数量级往往印象不深,为此,在作练习
一第(3)题之前,可让他们先猜一猜排满一米长所需的分子个
数,再看猜想的数目跟$10^{10}$个相差多少.这样,学生对分子微
小程度的印象会更深刻一些。还可以让学生把分子的直径同
一根头发的直径(数量级为$10^{-5}$米)进行比较,也有助于对
分子的微小程度获得具体的认识。

\textbf{5.阿伏伽德罗常数}\quad 学生在化学中已学过阿伏伽德罗
常数,教学时可引导他们:
\begin{enumerate}
\item 复习摩尔、摩尔质量、摩尔体积
等的含义。
\item 认识阿伏伽德罗常数有多种测定方法,本章教材就讲了两种测定方法,即测出分子的大小或分子的质量都
可求得阿氏常数,到本册第八章讲过法拉第常数后,还将介绍
根据$N=F/e$测阿伏伽德罗常数的方法。\item 从阿伏伽德罗常
数把分子大小,分子质量这些无法直接测量的微观量跟摩尔
体积、摩尔质量等宏观量联系起来的事实,认识这个常数是联。
系微观世界和宏观世界的桥梁,它为人们从微观角度定量地
研究热现象提供了重要条件。
\end{enumerate}

为使学生对分子的轻、小和阿伏伽德罗常数的巨大程度
有更深刻的印象,可以举出日常生活中认为很小或很少的东
西(如1克食盐,1${\rm cm}^3$水等),让学生估计其中所含的分子
数,然后再根据计算结果来检验自己的估计、经验表明,这种
办法常常会使学生对阿伏伽德罗常数的巨大程度感到吃惊,
从而留下难忘的印象。

教学中还可引导学生联系已学过的万有引力恒量初步认
识,物理常数的存在乃是物理世界客观规律性的反映。因此
科学家们总是不断努力采用多种方法来更精确地测定这些重
要物理常数。

分子虽然看不见,摸不到,但通过实验和科学分析却能测
定分子的大小和阿伏伽德罗常数,而且用很不相同的各种方
法测出的结果彼此相符(或数量级相符),这就为分子的客观
存在和物体是由大量分子组成的提供了有力证据,为分子
运动论的进一步发展奠定了坚实基础,同时也告诉我们物理
理论彼此和谐一致,正确地反映了自然,因而它对指导人们从
事生产和科学研究具有重大作用,建议在讲过分子的大小和
阿伏伽德罗常数之后,启发学生对上述问题有所体会。

\textbf{6.布朗运动}\quad 在讲布朗运动之前,最好引导学生复习一
下初中学过的扩散现象,然后指出:扩散现象虽能说明分子
在不停地运动,但布朗运动现象则是分子永不停息地做无规
则运动最早而又最明显的实验证据。在布朗运动这样的实验
事实面前,历史上一些曾经持有不同观点的科学家,也不得不
相信分子运动论确实是正确的理论了。

本节教学能否取得较好的效果,关键在于是否做好布朗
运动的演示实验(参看本章实验指导),及启发学生通过观察、
思考、分析对实验现象获得正确的定性理解。

学生对布朗运动并不是分子本身的运动比较容易接受,
但对布朗运动是液体分子运动所造成的,是分子永不停息地
做无规则运动的反映,就常常感到费解。为此,要引导他们从
多方面思考、分析,充分认识产生布朗运动的原因不是来自外
界影响,而在液体内部.如:
\begin{enumerate}
\item 让学生认真阅读教材第10页
第2段,从中了解认为布朗运动来自外界影响的种种看法,
经过实验检验,都是不合理的臆测.    \item 依据教材的插图(一个
微粒受到液体分子撞击的情景图)和分析,引导学生认识布
朗运动是悬浮在液体中的微粒不断地受到液体分子撞击的结
果。在液体中悬浮的微粒越小,同时撞击微粒的分子就越少,
这种撞击作用的不平衡性就表现得越明显;在液体中悬浮的
微粒越大,同时撞击微粒的分子就越多,撞击作用的不平衡性
就表现得越不明显,可见,微粒的布朗运动不是液体以外的
作用所引起的.    \item 组织学生讨论练习二第(3)题,从反面设
想,证实布朗运动是液体分子无规则运动的反映.    \item 提醒学
生在实验中注意观察不同小微粒的布朗运动情况是否相同。 结合讨论分析练习二第(4)题,进一步认识外界因素不可能
造成不同小微粒的布朗运动情况不同的现象。
\end{enumerate}

总之,通过实验
和分析,要使学生知道布朗运动现象只跟微粒的大小和温度
有关,布朗运动虽不是液体分子本身的运动,但它的运动状
况完全是由液体分子的运动状况决定的,正是液体分子对微
粒的无规则撞击,迫使微粒也做无规则运动,所以说,布朗运
动是间接地、但又更明显地证实了分子的无规则运动。

在指导学生阅读教材第9页图1.5时,应当指出这个图
并不表示做布朗运动的微粒的运动轨迹,它只是记录下了每
隔30秒微粒所在位置的变化,并用直线依次把这一系列的位
置变化连接起来,就成了图中所示的“做布朗运动的微粒的运
动路线”.实际上,即使在这短短的30秒内,微粒的运动也是
无定向的、极不则的。如果把记录的时间间隔取得更短,则
连接各个位置之间的直线就会成为更复杂的折线。

布朗运动随温度升高而愈加激烈的现象,一般不容易看
清楚。建议用不同温度下扩散现象快慢程度不同的演示实验,
来说明温度越高,分子的无规则运动越激烈。

做布朗运动的微粒是由成千上万个分子组成的宏观物
体,温度也是反映大量分子集体行为的宏观物理量。但布
朗运动现象证实了分子的无规则运动,揭示了温度与分子无
规则运动的联系。从这里可以再一次启发学生,认识在宏观
实验的基础上,经过思考、分析、推理形成一定的认识,又不断
地接受客观事实和新的实验检验,使认识逐步深入,这就是人
们探索微观世界的途径和线索。

\textbf{7.分子间的相互作用力}\quad
在讲分子间有空隙时可首先提出问题,如提出为什么说布朗运动和扩散现象说明了分子
间有空隙?让学生经过思考,认识到如果真是象为估算分子直
径大小而作的设想那样,固体和液体中的分子是一个挨一个
地紧密堆积起来的,那么分子就不可能从一个地方移动到另
一个地方,因而也就不会出现布朗运动和扩散现象。所以这
两个现象说明了分子间必然有空隙。

本节教材从布朗运动和扩散现象出发,经过分析、推理,
得出分子间有空隙;又根据分子间虽有空隙,大量分子却能聚
集在一起形成固体或液体,推出分子间有引力;再由分子间既
有引力,又有空隙,推出分子间还有斥力。对所推出的结论,都
用有关实验和事实给以证明,这种以实验和事实为依据,运
用逻辑推理方法,从已知探求未知的思路,值得学生认真领
会。建议按照教材安排做好证明分子间有空隙和分子间存在
引力等演示实验,并指导学生仔细阅读教材第12页1、2、3
段,理出其中论证分子间存在引力和斥力的线索,这是本节
教学应当着重突出的地方。

要向学生强调指出:分子间的引力和斥力是同时存在的。
只是引力和斥力的大小要随着分子间距离的变化而变化。至
于实际表现出来的分子力究竟是引力还是斥力,则取决于分
子间的引力与斥力的合力。分子间的相互作用比较复杂,可
指导学生阅读教材第13页图1.7甲、乙,通过对照弄清图
1.7甲的曲线表示的物理意义,以利于学生对引力、斥力、合
力随分子间距离变化而变化的情形,有较形象具体的解。在
图线教学中,要提醒学生注意认识$r$大于和小于$r_0$时,引力
和斥力随距离变化的不同特点。即当$r<r_0$时,引力和斥力
都随距离减小而增大,但斥力比引力增大得更快,所以合力表
现为斥力;当$r>r_0$时,引力和斥力都随距离增大而减小,但
斥力比引力减小得更快,所以合力表现为引力,它随着距离的
增大迅速减小;当分子间距离的数量级大于$10^{-9}$米($10r_0$)时,
引力和斥力都变得十分微弱,这时分子间的作用力就可以忽
略不计了。

在此基础上,可组织学生讨论练习三第(4)题,使他们通
过讨论熟悉题中举出的粗略估算分子直径大小的方法,并在
灵活运用知识方面受到启示。还可以让学生根据分子间的相
互作用力与距离的关系解释一些现象(如在演示分子间存在
引力的实验中,为什么两块铅的端面必须是平滑的才能粘合
在一起?打碎的玻璃,为什么不能利用分子力拼接起来,使它
恢复原状?),这对培养学生分析解决实际问题的能力,是有
好处的。





