\documentclass[a4paper, openany]{ctexbook}

%\usepackage{anysize}
%\papersize{26cm}{18.5cm}
%\marginsize{2.25cm}{2.25cm}{2cm}{2cm}
\usepackage[margin=3cm]{geometry}


% 修改脚注的编号为加圈样式,并且各页单独编号

\usepackage{pifont}
\usepackage[perpage,symbol*]{footmisc}
\DefineFNsymbols{circled}{{\ding{192}}{\ding{193}}{\ding{194}}
{\ding{195}}{\ding{196}}{\ding{197}}{\ding{198}}{\ding{199}}{\ding{200}}{\ding{201}}}
\setfnsymbol{circled}



\usepackage{amsmath,amsfonts,mathrsfs,amssymb}
\usepackage{graphicx}

\usepackage[font=bf,labelfont=bf,labelsep=quad]{caption}

\usepackage{tikz}



\usepackage{physics}
%\usepackage{amsthm}
\usepackage{ntheorem}
\theoremseparator{\;}



\usepackage{blkarray}
\usepackage{bm}
\usepackage[colorlinks=true, linkcolor=black]{hyperref}

%\usepackage{enumerate}


\theoremstyle{plain}
\theoremheaderfont{\normalfont\bfseries} 
\theorembodyfont{\normalfont}


% \newtheorem{defn}{定义}[chapter]
% \newtheorem{thm}{定理}[chapter]
% \newtheorem{yl}{引理}[chapter]
% \newtheorem{tl}{推论}[chapter]

%\usepackage[framemethod=tikz]{mdframed}
%
%\newmdtheoremenv[
%  linecolor=black,
%  leftmargin=20pt,
%  innerleftmargin=5pt,
%  innerrightmargin=2pt, roundcorner=5pt
%  ]{defn}{定义}[chapter]
%  \newmdtheoremenv[
%	linecolor=black,
%	leftmargin=20pt,
%	innerleftmargin=5pt, 
%	innerrightmargin=2pt, roundcorner=5pt
%	]{thm}{定理}[chapter]
%
%	\newmdtheoremenv[
%  linecolor=black,
%  leftmargin=20pt,
%  innerleftmargin=5pt,
%  innerrightmargin=2pt, roundcorner=5pt
%  ]{yl}{引理}[chapter]
%
%  \newmdtheoremenv[
%  linecolor=black,
%  leftmargin=20pt,
%  innerleftmargin=5pt,
%  innerrightmargin=2pt, roundcorner=5pt
%  ]{tl}{推论}[chapter]
%
%
\newtheorem{example}{\bf 例题}[chapter]
\newenvironment{solution}{\noindent {\bf 解答:}}{}%{\hfill $\blacksquare$\par}


\renewcommand{\proofname}{\bf 证明:}
\newenvironment{proof}{{\noindent \bf 证明:}}{}%{\hfill $\square$\par}

\newcommand{\E}{\mathbb{E}}
\renewcommand{\Pr}{\mathbb{P}}
\newcommand{\EP}{\mathbb{E}^{\mathbb{P}}}
\newcommand{\EQ}{\mathbb{E}^{\mathbb{Q}}}
\newcommand{\dif}{\,{\rm d}}
\newcommand{\Var}{{\rm Var}}
\newcommand{\Cov}{{\rm Cov}}



 \usepackage{tcolorbox}
 \tcbuselibrary{breakable}
 \tcbuselibrary{most}

% \tcolorboxenvironment{tl}{colback = cyan!5!white, colframe = cyan!75!black,
%     colbacktitle = cyan!85!black, enhanced,  breakable}

% \tcolorboxenvironment{thm}{colback = magenta!5!white, colframe = magenta!75!black, 
%     colbacktitle = magenta!85!black, enhanced,     breakable}

% \tcolorboxenvironment{yl}{colback = red!5!white, colframe = red!75!black, 
%     colbacktitle = red!85!black, enhanced,     breakable}

% \tcolorboxenvironment{defn}{colback = blue!5!white, colframe = blue!75!black,
%     colbacktitle = blue!85!black, enhanced,    breakable
% }


%\tcolorboxenvironment{tl}{colback = cyan!5!white, colframe = cyan!75!black, fonttitle = \bfseries,
%    colbacktitle = cyan!85!black, enhanced,
%    attach boxed title to top center={yshift=-2mm}, breakable}
\setcounter{tocdepth}{1}

\setcounter{secnumdepth}{3}



\ctexset {
section = {
	name = {第,节},
	number = \chinese{section}},
subsection = {
	name = {\hspace{2em},、\hspace{-1em}},
	number = \chinese{subsection}
},
subsubsection = {
	name = {\hspace{2em}(,)\hspace{-1em}},
	number = \chinese{subsubsection},
}
}


%\usepackage{fancyhdr}
%
%% \usepackage{enumitem}   [itemindent=1em]
%
%
\renewcommand{\contentsname}{目~~录}

\usepackage{paralist}
\let\itemize\compactitem
%\let\enditemize\endcompactitem
\let\enumerate\compactenum
%\let\endenumerate\endcompactenum
\let\description\compactdesc
%\let\enddescription\endcompactdesc


\usepackage{titlesec}
\titlespacing{\chapter}{0pt}{*1}{*1}
\titlespacing{\section}{0pt}{*1}{*1}
\titlespacing{\subsection}{0pt}{*1}{*1}

\titlespacing{\subsubsection}{0pt}{*1}{*1}

\renewcommand{\le}{\leqslant}
\renewcommand{\ge}{\geqslant}
\usepackage{mathtools}

\setlength{\abovecaptionskip}{0.cm}
\setlength{\belowcaptionskip}{-0.cm}

%\setmainfont{Times New Roman}
\usetikzlibrary{decorations.pathmorphing, patterns}
\usetikzlibrary{calc, patterns, decorations.markings}
\usetikzlibrary{positioning, snakes}

%\cover{cover.pdf}

\usepackage{yhmath}
\newcommand{\ms}{\text{m}/\text{s}}
\newcommand{\cms}{\text{cm}/\text{s}}
\newcommand{\msq}{\text{m}/\text{s}^2}
\newcommand{\cmsq}{\text{cm}/\text{s}^2}
\newcommand{\kmh}{\text{km}/\text{h}}
\newcommand{\NC}{\text{N}/\text{C}}

\newcommand{\x}{\times}


\usepackage{circuitikz}
\usepackage{tkz-euclide}



\begin{document}
\fontsize{11}{14}\selectfont














\title{高级中学\\物理(甲种本)第二册\\教学参考书}



\author{人民教育出版社物理室~~编}
\date{1987年1月}

\maketitle

\tableofcontents


\frontmatter

\chapter{高中物理甲种本第二册的说明}

(1)高中物理课本(甲种本)第二册讲述的有热学和电学
两部分内容,热学共分五章。第一章讲述分子运动论的基本
内容。第二章讲述内能、热力学第一定律以及能的转化和守
恒定律,研究热现象有两种方法,即微观方法和宏观方法,它
们相辅相成使热学的研究越来越深入。与此相适应,教材头
两章分别介绍这两种方法,给整个热学知识的讲解打好基础,
因此这两章是基础性的两章。这两章所讲的知识要贯穿在整
个热学知识中。其中,能的转化和守恒定律是自然界的普遍
规律,不仅贯穿热学部分,而且要贯穿以后所学的知识。

气体比较简单,热学性质研究得比较透彻,中学阶段有可
能讲得扎实深入一些,因此教材把第三章“气体的性质”作为
热学部分的重点知识来讲解。着重讲解了理想气体状态方
程,用气体分子运动论解释了气体的三个实验定律。最后介
绍理想气体的内能,并综合运用热力学第一定律和气体性质
的知识分析了理想气体内能的变化。第四章介绍液体和固体
的性质。这一章是介绍性的,从宏观上介绍了晶体的性质和
液体的表面现象,并从微观上作适当解释,把宏观现象微观
解释联系起来,第五章讲述物态变化,这部分知识在许多方
面都要用到,初中虽然学过一些,但讲得比较简单,高中要予
以扩展,并给予必要的微观解释,以便使学生对知识的理解深
化一些。

这一册教材的后一部分讲述三章电学知识。第六章讲静
电学,讲述电学中的几个基本物理量(场强、电势、电容等),并
初步接触到物理学中一个基本概念一场。这些知识很重
要,是学习电学知识的基础,因此这一章是电学部分基础性的
重要一章,第七章讲述直流电路方面的知识,既是重要的基
础知识,以后学习中常常用到,而且具有实际意义。第八章讲
述物质的导电性,包括金属的导电性,液体和气体的导电性,
真空中的电流以及半导体的导电性。学习电学知识,了解各
种物质的导电机理,有利于加深学生对导电现象和导电规律
的理解,有利于知识的系统化。

(2)这册教材的内容同第一册力学比较起来,定量讲解
的内容相对减少,定性分析物理现象的内容相对增加,这样
做是必要的。中学物理教学不但要有一定的深度,而且要有
一定的广度,不但要按部就班系统地学习知识,而且要灵活
渗透一些知识。学知识面窄,眼界不宽,兴趣不广,思路不
灵活,这对在中学学好物理不利,对将来参加工作和进一步学
习也不利。物理教学不能仅仅是物理计算,而且要重视引导
学生灵活地运用概念和规律分析解释物理现象。即使物理计
算,也不能刻板地去搞所谓类型,而应着重于物理思考,学习
分析问题的方法,培养分析物理问题的能力,不能因为某部
分教材的定量计算少,就忽视这部分教材。各部分教材都有
它本身的作用,不能忽视。

这册教材涉及到微观领域的知识。教材在讲热学知识时
注意用微观知识解释热现象及其规律,在讲解物质的导电性
时注意说明导电机理 虽然有些知识,学生在初中物理课中
学过,或者在化学课中学过,但是人类是怎样获取这些知识
的,学生并不很了解。微观世界看不见,摸不到,而人们却能
掌握微观世界的规律。人类进入微观世界并不是凭空想象出
来的,而是经过实验、经过分析,建立各种假设和模型,并进而
经过实验的检验,逐步认识微观世界的,作为物理课,讲解微
观领域的知识,要着重讲述这些知识的实验基础,注意使学生
知道人们进行微观世界的线索和思路,人们认识了微观世
界,反过来又对宏观现象进行微观解释,从而对宏观现象的认
识更深入一步,因此,培养学生用微观世界的知识对宏观现
象进行微观解释,在物理教学中是十分重要的。

这册教材较多的涉及到一些物理恒量,如摩尔气体恒量、
阿伏伽德罗常数、基本电荷、法拉第恒量等。物理恒量及其测
定,在物理学的发展中起重要作用。物理恒量的存在是物理
世界的规律性的反映,用不同的方法测得的物理恒量的数值
相符,表明物理学理论本身和谐一致,也证明理论正确地反映
了客观规律 因此,在高中物理教学中,适当地讲解一些恒量
及其测定是很有意义的,它可以开阔学生的思路,培养他们
灵活地运用知识,对于微观恒量,要使学生知道它们的大小
都是通过宏观量测定而得来的,这有助于使学生了解人类进
入微观世界的线索。

这册教材进一步扩展和加强了能的转化和守恒的观点。
在静电学中对电势能的讲解比较仔细,使学生对功和能之间
的关系的认识得到扩展,通过分析电荷在电场中的运动,动
能定理扩展到电场力做功的情形;同时,能的转化扩展到电势
能与运动电荷的动能之间的转化,在稳恒电流一章中,继续注
意贯穿能的转化和守恒这条线索 讲解电功时,明确指出电流
做功实际上是电场力移动电荷做功,因而电流做功的过实
质上是电能转化为其他形式的能的过程。为了使学生搞清电
路中能的转化关系,教材指出:在纯电阻电路中,电完全转
化为内能;在非纯电阻电路中,电能除转化为内能外,还转化
为其他形式的能,如在电动机中转化为机械能,在这种转化
中能量守恒。关于电动势的讲法,虽然没有讲非静电力做功,
但仍注意从能的转化角度进行分析 这样,讲过闭合电路的
欧姆定律之后,学生对整个电路中能的转化情况将会有一个
全面了解,并能够从能的转化的观点分析有关电路的问题,建
议在这册教材的总复习时,对力学中讲过的功能关系予以扩
展,对能的转化和守恒作一个小结,有意识地加强培养学生用
能的观点分析问题的能力。

(3)这册教材在讲解知识时,注意了渗透研究物理的方
法,培养分析问题的能力。

在物理学中常常用各种模型来反映客观事物。通过建立
模型来进行研究,这是研究物理学的重要方法。在高中物理
中要讲到一些模型。力学中的质点,其实就是一种简单的模
型,这册教材在讲解分子运动论时,涉及分子的模型。要使
学生理解,因问题的需要,对分子可以采取不同的模型。模型
常常是对现实的一种简化或理想化,至于哪些因素被忽略,要
看问题的性质而定,这册教材还提到了理想气体的微观模型。
要使学生知道建立模型的重要,并且逐渐习惯于运用模型来
分析和考虑问题。同时,也要适当指出模型的近似性,不要把
模型当成是绝对完整地反映实际事物的 这可以发展他们的
想象能力和思维能力,培养他们灵活地运用知识,在教学中
值得重视。
在物理学中常常要利用两个量的比值定义一个新的物理
量。从第一册开始,速度就是用比值来定义的。这册教材在
讲电场强度、电势等概念时也是用比值来定义的,为了使学
生对用比值定义物理量这种方法进一步熟悉起来,教材安排
了一个阅读材料,对这个问题加以说明,物理量的定义方法
是各种各样的,除了基本的物理量而外,总是要通过已知的物
理量来定义新的物理量。用比值定义物理量只是一种形式,当
然还有其他形式。功、动量、冲量就是用两个量的乘积来定义
的,建议指导学生学习这个阅读材料,并对如何定义物理量,
作些指导与说明,这将有助于学生今后学好物理知识。

“等效”的概念在分析问题时常常用到,这册教材在讲述
串并联电路时提出了等效电阻的概念,目的是要使学生初步
接触一下等效的概念。教师如果感到必要,也可以提前,在讲
述电容的串并联时提出。对某些问题可以介绍学生用等效的
方法来处理。比如,一个复杂的电路,只要知道这个电路两端
的电压和通入(或流出)的电流,根据等效电阻的概念,就可以
直接得出这个电路的总电阻,学会用这种方法来处理问题,对
于学生灵活运用知识是有好处的。

利用图线可以形象地表达物理规律。学生熟悉图线的表
示方法,将有助于他们掌握物理规律,对规律所表达的内容有
一个形象具体的理解。这册教材对图线的利用有所加强。教
材在讲述分子的互相作用、气体的实验定律、欧姆定律、串联
电路的电压、路端电压等问题时,都利用了图线。对图线的教
学,最基本的是使学生清楚地理解图线所表达的物理意义,看
到图线就能在头脑中对物理的变化规律有具体的认识。为了
减轻学生的负担,教材并不要求利用图线进行定量的计算。

(4)这册教材一共安排了十三个学生实验.这些实验,
从实验训练角度上看,所涉及的方面还是比较全的。就实验
的性质来说,有基本定律的验证,有物理量的测定,也有新的
仪器的使用,就使用的仪器来说,中学阶段要求掌握的仪器,
大部分使用到了、因此这册教材的学生实验,在整个中学的
实验教学中具有重要的地位和作用。

在少实验中,要用到初中和高一已经学过的仪器,学生
对这些仪器的掌握情况,直接关系到这册中的实验效果,因
此,教学中要特别注意提高学生掌握和使用这些仪器的
技能。

学生实验同第一册一样,要求学生在理解实验原理的基
础上知道怎样操作以及为什么这样做,避免依照书本上的步
骤盲目操作。“测定金属的电阻率”这个实验,原理比较简单,
要求学生自己选择实验仪器和确定实验步骤,这样做,要求提
高了,教学中可先让学生讨论,或给以必要的提示,“练习使用
万用表”这个实验,没有提出具体测量项目。这是为了便于教
师在教学中自己掌握,根据学校器材情况来确定测量项目。
(5)高中物理(甲种本)第二册的教学内容可按每周3课
时,全学年共96课时讲授完。各章所用的课时数是:第一章
分子运动论基础5课时,第二章内能、能的转化和守恒定律
6课时,第三章气体的性质13(2)课时(括号内的数字是学生
实验的课时数,下同),第四章固体和液体的性质5课时,第五
章物态变化10课时,第六章电场17(2)课时,第七章稳恒电
流18(7)课时,第八章物质的导电性14(4)课时,平时复习
和机动时间8课时。






































\mainmatter


\chapter{分子运动论基础}
\minitoc[n]

\section{教学要求}
这一章介绍分子运动论的基本观点,它贯穿在整个热学
教材中,是基础性的一章。为了帮助学生顺利地进入热学学
习,教材一开始就指出热现象的含义,以及热学的研究对象、
研究方法。教材中讲到了研究热现象的两种不同方法,是为
了便于学生今后的学习,只要学生有个初步印象就可以了,不
宜过多讲解。

这一章讲述的内容,在初中物理中大都做过初步介绍,这
里,更加强调了分子运动论的实验基础。目的是使学生认识
分子运动论是在实验基础上建立起来的,而不是人们凭空想
象出来的。讲述这些实验,要注意引导学生了解人们是怎样
经过实验进入分子世界的;知道人们进入分子世界的线索,这
对学习物理知识特别重要。为了强调实验基础,有些实验初
中虽然做过,这里仍做了叙述。建议在教学中也重做一下这
些实验。

第一节讲述分子运动论的建立,是为了让学生了解这一
学说的建立决非轻而易举,是人类经过长期探索研究的结果,
教材中提到的历史事实,不要求作过细的讲解。

在第一节中教材指出:“按照分子运动论,热现象是大量
分子无规则运动的表现,温度表示分子无规则运动的激烈程
度,热能是大量做无规则运动的分子具有的能”。这里并不要
求展开讲,只要求学生先有个概括的了解,以便把分子运动与
热现象联系起来。在未讲内能之前,教材仍沿用初中用过的
热能的提法。关于温度的微观解释以及内能的概念,将在下
一章讲述。

第二节讲述分子的大小和阿伏伽德罗常。能够定量地
测出分子的大小,这就为分子运动论打下了坚实的基础,测
定分子的大小要根据分子运动论所推导出来的某些关系,这
些关系把宏观量与分子的大小直接联系起来。但在中学不好
讲述这些关系,所以教材选择了早期粗略测定分子大小的一
种方法-油膜法。这种方法容易为学生理解。利用宏观实
验测定微观量的大小,这是第一次。可向学生指出,微观量的
大小都是通过宏观测定得出的,明确这一点,可以使学生了解
人类进入微观世界的线索,打开学生探索微观世界的思路,这
对他们今后学习物理是很重要的。学生在化学课中已经学过
阿伏伽德罗常数,这里要求他们进一步体会这个常数的重要,
即它是联系微观世界与宏观世界的桥梁。

第三节讲解布朗运动,要使学生明确:布朗运动是微粒
的运动,而不是分子本身的运动;布朗运动的无规则性,反
映了液体内部分子运动的无规则性。布朗运动是由“涨落”现
象造成的。没有“涨落”,就没有液体分子对微粒撞击作用的
不平衡性;而微粒越小,由“涨落”所造成的不平衡性越明
显。这方面的道理,对中学生很难讲清楚,因此在教学中不要
求深入讲解,可通过一定的比喻,使学生初步了解微粒越小,
撞击作用的不平衡性越明显就可以了。

第四节讲述分子间的相互作用力。要明确指出:分子间
是同时存在着引力和斥力的。对于引力、斥力、合力怎样随距
离而变化,教材采取课本13页图1.7甲、乙两图对照的办法
来说明,以期学生能够清楚地了解甲图中曲线所表示的意
义,要使学生注意了解在分子间的距离大于和小于$r_0$时,引
力和斥力随距离变化的不同特点,这是认识合力怎样随距离
而变化的关键。至于分子间相互作用力的起源,在中学阶段不
可能讲解这个问题,因而教材只是指出分子力是由分子、原
子的带电粒子之间相互作用引起的,而不再进一步说明。

在本章最后,教材根据分子运动论简要说明了气、液、固
三种物态的情况,目的是使学生对此先有个一般了解,便于今
后学习。这里并不要求更多地讲解。

这一章的教学要求是:

\begin{enumerate}
\item 掌握分子运动论的基本内容,了解它的实验基础,了
解人们进入分子世界的线索。
\item 了解测定分子大小和阿伏伽德罗常数的方法,了解阿
伏伽德罗常数是联系微观世界和宏观世界的桥梁,会用这个
常数进行计算。
\item 了解什么是布朗运动以及布朗运动是怎样产生的,理
解布朗运动的无规则性反映了液体内部分子运动的无规
则性。
\item 了解分子间作用力的特点,了解分子间的引力、斥力以及它们的合力随分子间距离而变化的情形。
\end{enumerate}

\section{教学建议}
分子运动论的基本要点,学生虽然在初中学过,但本章对
这一理论在实验基础、定量分析和研究方法上的阐述都比初
中内容丰富、深入,要求提高了,这一点,教学时应提醒学生
注意。

\subsection{全章引言}
在引言教学中可以指出:
\begin{enumerate}
    \item 热现象与人
们生活、生产的关系非常密切,从远古时代起,人类就利用热
能来为自己服务。我国古代就有燧人氏“钻木取火”的历史传
说。正是火的发现和使用,使得古代人有可能利用热能改变
周围环境和生产生活的条件,促进了人类自身的发展.
\item 力
学现象只涉及物体的机械运动,不涉及温度和物态变化;热现
象则与温度和物态变化有关。热现象虽与力学现象不同,但
在实际中二者往往同时出现,而且同一物体,随着温度的变
化,它的某些力学性质如弹性、硬度等也要改变。如果只从力
学的角度考察现象,就不能解释这些力学性质的改变跟什么
因素有关.  
\item 热现象的理论,是各种热机、致冷设备以及喷
气式发动机和火箭工作原理的基础,在工农业生产、科学
术、医药卫生、日常生活等许多方面,都有广泛的应用,所以学
好热学是很必要的.  
\item 研究热现象的两种不同方法,是互为
补充,相辅相成,互相促进的。根据经验事实总结出的热现象
的宏观规律,需用分子运动论阐明其微观机理,而按照分子运
动论对热现象所作的解释,又需用热现象的宏观规律加以检
验,判断其是否正确。正因为人们从宏观和微观两个方面对热
现象进行探索,从而推动了热学的研究日益深入发展.
\item 这
一章和下一章将初步介绍热现象的微观理论和宏观理论的基
本内容,是整个热学的基础。
\end{enumerate}

\subsection{分子运动论的建立} 原子理论的萌芽产生于2000多
年前的古希腊时期。此后虽然经过了许多年,但因中世纪的
封建统治,生产和科学发展缓慢,物质结构的学说也就长期
没有得到发展,直到17—18世纪,由于产业革命的推动,蒸
汽机得到改进和普遍使用,使得提高热机效率成为社会的迫
切要求,从而捉进了热学的发展,促使人们开始探讨热现象的
本质,于是出现了定性的分子运动论学说。然而这个学说在
当时并未得到公认,人们普遍相信的是热质说.18世纪末
已有一些实验事实(例如,下一章阅读材料中讲的伦福德实
验),动摇了热质说的基础,特别是19世纪中叶,建立了能的
守恒和转化定律彻底否定了热质说,为分子运动论的发展开
辟了道路。以后,定量而系统的分子运动论迅速发展起来,到
本世纪初期达到了比较完善的地步.在200多年的漫长岁月
中,许多科学家为分子运动论的建立作了不懈的努力。历史
事实表明,人类对物质结构的认识和科学的发展,要受到包括
社会条件在内的多种因素的制约,是在曲折的道路上发展
的。一种科学理论的建立,必须经过长期的积累和客观事实
的检验,绝非轻而易举的事。通过本节的教学,应该使学生对
此有所体会。

在本节中,教材还对分子运动论的要点以及分子运动跟
热现象的联系,作了概括性的叙述。这些都是初中讲过了的。
建议让学生自己先作一番回忆,然后再阅读课文进行对照 这
样做,既可使学生对已学过的内容印象更深些,又有利于培养
学生的归纳、概括能力。

\subsection{分子的大小} 分子的大小是由实验测出的。测定分子
大小的实验,是建立分子运动论的重要基础,也是人类定量研
究分子世界的开端。因此,本节教学的首要问题是做好利用
油膜法粗略地测定分子大小的演示实验(参看本章的实验指
导),有条件的学校可让学生亲自做一做这个实验,使他们对
分子大小的数量获得深刻印象,并对通过宏观测定求微观
量的大小有点实在的感受。为使学生在探索微观世界的思路
方面受到启示和便于今后学习,可以指出,在本册和第三册教
材中还会遇到类似的测定。这表明,微观世界的信息都是通
过宏观测定得来的。

在考虑分子的大小时,“把分子看作小球,是分子运动论
中对分子的简化模型”。这一点要引起学生足够注意。要让
学生知道研究物理问题既要依靠实验,又要善于思考,利用
理想化模型来分析研究对象,就是进行物理思考的一种重要
方法。自然界中各种现象总是包含有多种因素,涉及许多方
面的关系,其实际状况往往十分复杂。如果不是根据问题的
性质和需要把研究对象加以简化,建立起与实际情况近似的
理想化模型,我们对物理现象就很难作定量分析与深入研究,
也就不能形成科学概念,找出规律,至于同一对象随着研究
范围和条件的变化,需要建立不同的模型(如本册教材第三章
讨论气体分子运动时,又有理想气体模型),则是灵活运用理
想化方法处理问题的表现,所以习惯于利用模型来思考分析
问题,对学好物理,培养思维能力和灵活运用知识的能力,都
是很重要的。

\subsection{数量级}
教学时,可向学生说明:用10的乘方来表示
很大或很小的物理量、物理常数的测量结果,是物理学中一种
习惯的科学记数方法、通常叫数量级,例如,地球的质量是
$5.98\x10^{24}$千克,我们就说地球质量的数级是$10^{24}$千克,真
空中光的速度是$3\x10^8$米/秒,光速的数量级就是$10^8$米/秒.
对于不少的物理量和物理常数,由于测量原理不够完善或测
量技术的限制,只能量得它的大致范围,或者只需要知道它的
数量级来进行估算,在这种情况下,就可只说出它的数量级。
比如,说分子直径有多大,就应强调它的数量级是$10^{-10}$米,这
是因为分子直径大小的具体数值,是在采用简化模型(把固
体、液体的分子看作一个挨一个排列的小球)的情况下求得
的。换句话说,这是对分子大小的粗略估算 所以重要的是知
道分子大小的数量级,形成一个数量观念,而不必花过多的精
力去讨论各种分子大小的具体数值。

学生对分子大小的数量级往往印象不深,为此,在作练习
一第(3)题之前,可让他们先猜一猜排满一米长所需的分子个
数,再看猜想的数目跟$10^{10}$个相差多少.这样,学生对分子微
小程度的印象会更深刻一些。还可以让学生把分子的直径同
一根头发的直径(数量级为$10^{-5}$米)进行比较,也有助于对
分子的微小程度获得具体的认识。

\subsection{阿伏伽德罗常数}
 学生在化学中已学过阿伏伽德罗
常数,教学时可引导他们:
\begin{enumerate}
\item 复习摩尔、摩尔质量、摩尔体积
等的含义。
\item 认识阿伏伽德罗常数有多种测定方法,本章教材就讲了两种测定方法,即测出分子的大小或分子的质量都
可求得阿氏常数,到本册第八章讲过法拉第常数后,还将介绍
根据$N=F/e$测阿伏伽德罗常数的方法。\item 从阿伏伽德罗常
数把分子大小,分子质量这些无法直接测量的微观量跟摩尔
体积、摩尔质量等宏观量联系起来的事实,认识这个常数是联。
系微观世界和宏观世界的桥梁,它为人们从微观角度定量地
研究热现象提供了重要条件。
\end{enumerate}

为使学生对分子的轻、小和阿伏伽德罗常数的巨大程度
有更深刻的印象,可以举出日常生活中认为很小或很少的东
西(如1克食盐,1${\rm cm}^3$水等),让学生估计其中所含的分子
数,然后再根据计算结果来检验自己的估计、经验表明,这种
办法常常会使学生对阿伏伽德罗常数的巨大程度感到吃惊,
从而留下难忘的印象。

教学中还可引导学生联系已学过的万有引力恒量初步认
识,物理常数的存在乃是物理世界客观规律性的反映。因此
科学家们总是不断努力采用多种方法来更精确地测定这些重
要物理常数。

分子虽然看不见,摸不到,但通过实验和科学分析却能测
定分子的大小和阿伏伽德罗常数,而且用很不相同的各种方
法测出的结果彼此相符(或数量级相符),这就为分子的客观
存在和物体是由大量分子组成的提供了有力证据,为分子
运动论的进一步发展奠定了坚实基础,同时也告诉我们物理
理论彼此和谐一致,正确地反映了自然,因而它对指导人们从
事生产和科学研究具有重大作用,建议在讲过分子的大小和
阿伏伽德罗常数之后,启发学生对上述问题有所体会。

\subsection{布朗运动}
 在讲布朗运动之前,最好引导学生复习一
下初中学过的扩散现象,然后指出:扩散现象虽能说明分子
在不停地运动,但布朗运动现象则是分子永不停息地做无规
则运动最早而又最明显的实验证据。在布朗运动这样的实验
事实面前,历史上一些曾经持有不同观点的科学家,也不得不
相信分子运动论确实是正确的理论了。

本节教学能否取得较好的效果,关键在于是否做好布朗
运动的演示实验(参看本章实验指导),及启发学生通过观察、
思考、分析对实验现象获得正确的定性理解。

学生对布朗运动并不是分子本身的运动比较容易接受,
但对布朗运动是液体分子运动所造成的,是分子永不停息地
做无规则运动的反映,就常常感到费解。为此,要引导他们从
多方面思考、分析,充分认识产生布朗运动的原因不是来自外
界影响,而在液体内部.如:
\begin{enumerate}
\item 让学生认真阅读教材第10页
第2段,从中了解认为布朗运动来自外界影响的种种看法,
经过实验检验,都是不合理的臆测.    \item 依据教材的插图(一个
微粒受到液体分子撞击的情景图)和分析,引导学生认识布
朗运动是悬浮在液体中的微粒不断地受到液体分子撞击的结
果。在液体中悬浮的微粒越小,同时撞击微粒的分子就越少,
这种撞击作用的不平衡性就表现得越明显;在液体中悬浮的
微粒越大,同时撞击微粒的分子就越多,撞击作用的不平衡性
就表现得越不明显,可见,微粒的布朗运动不是液体以外的
作用所引起的.    \item 组织学生讨论练习二第(3)题,从反面设
想,证实布朗运动是液体分子无规则运动的反映.    \item 提醒学
生在实验中注意观察不同小微粒的布朗运动情况是否相同。 结合讨论分析练习二第(4)题,进一步认识外界因素不可能
造成不同小微粒的布朗运动情况不同的现象。
\end{enumerate}

总之,通过实验
和分析,要使学生知道布朗运动现象只跟微粒的大小和温度
有关,布朗运动虽不是液体分子本身的运动,但它的运动状
况完全是由液体分子的运动状况决定的,正是液体分子对微
粒的无规则撞击,迫使微粒也做无规则运动,所以说,布朗运
动是间接地、但又更明显地证实了分子的无规则运动。

在指导学生阅读教材第9页图1.5时,应当指出这个图
并不表示做布朗运动的微粒的运动轨迹,它只是记录下了每
隔30秒微粒所在位置的变化,并用直线依次把这一系列的位
置变化连接起来,就成了图中所示的“做布朗运动的微粒的运
动路线”.实际上,即使在这短短的30秒内,微粒的运动也是
无定向的、极不则的。如果把记录的时间间隔取得更短,则
连接各个位置之间的直线就会成为更复杂的折线。

布朗运动随温度升高而愈加激烈的现象,一般不容易看
清楚。建议用不同温度下扩散现象快慢程度不同的演示实验,
来说明温度越高,分子的无规则运动越激烈。

做布朗运动的微粒是由成千上万个分子组成的宏观物
体,温度也是反映大量分子集体行为的宏观物理量。但布
朗运动现象证实了分子的无规则运动,揭示了温度与分子无
规则运动的联系。从这里可以再一次启发学生,认识在宏观
实验的基础上,经过思考、分析、推理形成一定的认识,又不断
地接受客观事实和新的实验检验,使认识逐步深入,这就是人
们探索微观世界的途径和线索。

\subsection{分子间的相互作用力}
在讲分子间有空隙时可首先提出问题,如提出为什么说布朗运动和扩散现象说明了分子
间有空隙?让学生经过思考,认识到如果真是象为估算分子直
径大小而作的设想那样,固体和液体中的分子是一个挨一个
地紧密堆积起来的,那么分子就不可能从一个地方移动到另
一个地方,因而也就不会出现布朗运动和扩散现象。所以这
两个现象说明了分子间必然有空隙。

本节教材从布朗运动和扩散现象出发,经过分析、推理,
得出分子间有空隙;又根据分子间虽有空隙,大量分子却能聚
集在一起形成固体或液体,推出分子间有引力;再由分子间既
有引力,又有空隙,推出分子间还有斥力。对所推出的结论,都
用有关实验和事实给以证明,这种以实验和事实为依据,运
用逻辑推理方法,从已知探求未知的思路,值得学生认真领
会。建议按照教材安排做好证明分子间有空隙和分子间存在
引力等演示实验,并指导学生仔细阅读教材第12页1、2、3
段,理出其中论证分子间存在引力和斥力的线索,这是本节
教学应当着重突出的地方。

要向学生强调指出:分子间的引力和斥力是同时存在的。
只是引力和斥力的大小要随着分子间距离的变化而变化。至
于实际表现出来的分子力究竟是引力还是斥力,则取决于分
子间的引力与斥力的合力。分子间的相互作用比较复杂,可
指导学生阅读教材第13页图1.7甲、乙,通过对照弄清图
1.7甲的曲线表示的物理意义,以利于学生对引力、斥力、合
力随分子间距离变化而变化的情形,有较形象具体的解。在
图线教学中,要提醒学生注意认识$r$大于和小于$r_0$时,引力
和斥力随距离变化的不同特点。即当$r<r_0$时,引力和斥力
都随距离减小而增大,但斥力比引力增大得更快,所以合力表
现为斥力;当$r>r_0$时,引力和斥力都随距离增大而减小,但
斥力比引力减小得更快,所以合力表现为引力,它随着距离的
增大迅速减小;当分子间距离的数量级大于$10^{-9}$米($10r_0$)时,
引力和斥力都变得十分微弱,这时分子间的作用力就可以忽
略不计了。

在此基础上,可组织学生讨论练习三第(4)题,使他们通
过讨论熟悉题中举出的粗略估算分子直径大小的方法,并在
灵活运用知识方面受到启示。还可以让学生根据分子间的相
互作用力与距离的关系解释一些现象(如在演示分子间存在
引力的实验中,为什么两块铅的端面必须是平滑的才能粘合
在一起?打碎的玻璃,为什么不能利用分子力拼接起来,使它
恢复原状?),这对培养学生分析解决实际问题的能力,是有
好处的。

\section{实验指导}
\subsection{演示实验}
\subsubsection{用油膜法估测分子的大小}

取一滴油酸酒精溶液滴到水面上,其中的油酸就会
在水面上形成单分子层油膜.油酸的分子($\rm C_{17}H_{33}COOH$)呈
长形,它的一端是$\rm COOH$, 对水有很强的亲合力,被水吸引在
水中;另一端是$\rm C_1H_{33}$, 对水没有亲合力,便冒出水面.科学家
发现油酸分子都是直立在水中的。因此,单分子油膜的厚度,
可以认为等于油酸分子的长度。

取极少量的油酸,并准确地测定它的体积,观测表
明,一小滴未经稀释的油酸在水面上散开后,形成的单分子层
油膜要占很大面积。为了在实验中能用普通面盆大小的盛水
容器来观测油酸的单分子层油膜,就需用无水酒精稀释油酸
(油酸是有机酸,不溶于水,而溶于酒精),使一滴油酸酒精溶
液中只含极少量的油酸,通常可按1:200的体积比配制油酸
溶液.即在装油酸的量筒中用移液管准确地取出1毫升油酸
注入一只200毫升的瓶内,再加酒精到刻度线,摇动瓶子,使
油酸在酒精中充分溶解。那么,每取一滴这样的油酸酒精溶
液,其中纯油酸的体积应为每滴溶液体积的$1/200$. 又用移液
管将油酸酒精溶液滴人另一只有刻度的小量筒内,数出滴满1
毫升的滴数,例如为125滴,则用同样的滴管滴一滴这种溶液
时,其中所含油酸的体积为
\[\frac{1}{200}\x\frac{1}{125}{\rm cm}^3=4\x 10^{-5}{\rm cm}^3\]

准确地测定油膜的面积,如图1.1所示,在盛水盘内
装蒸馏水约1厘米深.为便于观测油膜的面积,可在水面上轻
轻地撒一层石松粉,待粉末均匀分布后,往水盘中央滴一滴油
酸酒精溶液,于是油酸在水面上迅速散开,到油膜面积不再扩
大时,用一块玻璃盖在盘缘上描出油膜的轮廓图。然后把这
块玻璃放在方格纸上(图1.2),数出油膜面积所占的格数,算
出油膜的面积。若算得油酸的体积为$V$, 面积为$S$, 则油膜的
厚度$D=V/S$.
\begin{figure}[htp]\centering
    \begin{minipage}[t]{0.48\textwidth}
    \centering
\includegraphics[scale=.6]{fig/1-1.png}
    \caption{}
    \end{minipage}
    \begin{minipage}[t]{0.48\textwidth}
    \centering
    \includegraphics[scale=.6]{fig/1-2.png}
    \caption{}
    \end{minipage}
    \end{figure}

实验前,必须把所有的实验用具擦洗干净,实验中
吸取油酸、酒精和溶液的移液管要分别专用,不能混用,否则
会增大误差,影响实验效果。

这个实验的构思相当巧妙.实验后,可提出问题(如:
在水面上滴下油酸溶液后,若油膜布满了整个容器的水面,能
否根据这时的油膜面积求油酸分子直径?为什么?)启发学生
思考。

\subsubsection{观察布朗运动}
在布朗运动实验中,是否制备好检验液是影响实验效果
的主要因素。检验液中的微粒既要大小适度,又要均匀。若微
粒过大,布朗运动就不明显;若微粒大小不均匀,则只能看见
较小的微粒在动,而且容易跑到视野以外去。制备检验液可
拿书写用的上等松烟墨(劣质墨颗粒太大,效果不好)磨成墨
汁,用30—40倍蒸馏水冲淡,搅拌后静置几小时,大颗粒会沉
淀在下层,但上层仍含有悬浮物,需取其中间部分的液体,才
能获得大小适度、颗粒均匀的悬浊液。取一块薄载玻片,在它
上面浇一层厚度约0.5—1毫米的石蜡,再在石蜡中间挖一直
径约4—5毫米的凹坑.向坑内滴一、二滴蒸馏水,然后用铁
丝蘸一滴制备好的悬浊液放入蒸馏水中,搅拌后,检验液以
略带一点颜色为适宜(颜色太深或几乎没有颜色,均表明检验
液的浓度不合要求)。盖上盖玻片,即可放到显微镜的载物台
上去观察。

显微镜用600倍的较好.调节时应先用粗调将镜筒调下,
直到物镜下端几乎跟盖玻片接触为止,然后边观察边转动微
调旋钮使镜筒上升,直到能看见清晰的像。在调焦成像时,不
准将镜筒向下调,以免压碎盖玻片和损伤物镜。

实验时还应注意:
\begin{enumerate}
\item 显微镜头,载、盖玻片等必须干净。
任何一点污迹都会影响实验效果.
\item 盖上盖玻片时,要注意
不使凹坑内残留气泡,溢出多余的液体可用滤纸吸干.
\item 
显微镜的反光镜反射的光线应强弱合适,过强反而不易看清
楚,太弱则微粒颜色灰暗。
\end{enumerate}

指导学生观察时,可让他们先注意看一个微粒的布朗运
动,然后看不同微粒的布朗运动情况。

实验后可向学生指出,布朗运动现象虽是1827年发现
的,但当时人们并不能正确地解释它,经过30多年的研究,才
认识到布朗运动是由液体分子撞击悬浮微粒所引起的,并发
现静止气体中的悬浮微粒也在作布朗运动。

\subsubsection{扩散现象随温度升高而加快}
取两只试管,一只内装热水,另一只内装凉水,热水与凉
水的温差在$60^{\circ}{\rm C}$左右.用吸有少量红墨水的移液管分别往
两试管底部注入一些红墨水,然后将移液管慢慢从试管中取
出。可以看见装热水的试管中的红墨水比装凉水的试管中的
红墨水扩散得快些,从而说明扩散现象随温度升高而加快。

\subsubsection{酒精与水混合后体积减小}
在长1米左右一端开口的玻璃管(可用托里拆利管)内装
一半水,再将染色酒精沿管壁慢慢注入管内,直到注满玻管。
因酒精的密度比水小,这时可清楚地看到它们的分界面。然
后封住管口,把玻璃管上下颠倒几次,使水和酒精充分混合,
可以发现混合后液体的体积比混合前两种液体的总体积缩小
了。这个现象表明,酒精跟水混合后,分子重新排列,混合液
体的分子间的空隙减小了,所以总体积也就缩小了。

\subsubsection{把两个铅柱压紧,由于分子间的引力,它们就合在
一起}

做这个演示前,必须把铅柱端面的油污和氧化层刮掉,使
端面清洁、平滑。演示时应使两端面的刮纹一致,再从边缘开
始,使两个铅柱沿着相互接触
的端面相对平移,使端面吻合,
并把两个铅柱压紧(图1.3)。
\begin{figure}[htp]\centering
    \centering
    \includegraphics[scale=.6]{fig/1-3.png}
    \caption{}
    \end{figure}

为了不使学生误认为两铅
柱合在一起是大气压的作用,
可在铅柱上挂2千克左右的物
体,然后拉开端面让学生观察
压痕.铅圆柱体的截面积一般为3厘米2左右,从观察压痕
可知,两个铅柱的实际接触面积只有几个平方毫米,如果是大
气压作用的话,只要挂0.1千克左右的物体就把两个铅柱拉
开了。可见,两个铅柱合在一起并不是大气压的作用,是分
子间存在引力的结果。

\subsection{课外实验活动}
\subsubsection{观察扩散现象}

实验时,可将水和浓度较大的硫酸铜溶液先后装入
高玻璃杯中,要把玻璃杯放在不受震动的地方,使它完全处
于静止状态,以表明扩散现象是在不受外界影响的情况下发
生的。

实验的结果表明:密度较大的液体的分子会向上移
动,进入密度较小的液体;密度较小的液体的分子会向下移
动,进入密度较大的液体。从而有力地证实了分子在运动。

我们观察到的溶液的蓝颜色并不是硫酸铜分子的颜
色,而是水合铜离子的颜色。因此,硫酸铜分子的扩散,是由
水合铜离子的扩散推知的。

\section{习题解答}
\subsection{练习一}
\begin{enumerate}
\item  一般分子的直径,以厘米作单位时数量级是多大?

\begin{solution}
    是$10^{-8}$厘米.
\end{solution}

\item  把体积为1$\rm mm^3$的石油滴在水面上,石油在水面上
形成面积为3$\rm m^2$的单分子油膜.试估算石油分子的直径.

\begin{solution}
\[\begin{split}
    \text{石油分子的直径}&=\frac{\text{石油滴的体积}}{\text{单分子油膜面积}}=\frac{1{\rm mm}^3}{3{\rm m}^2}\\
&=\frac{1\x 10^{-9}{\rm m}^3}{3{\rm m}^2}=
3\x10^{-10}{\rm m}
\end{split}\]
\end{solution}
\item  设想把分子一个挨一个地排起来,要多少个分子才
能排满1米的长度?

\begin{solution}
\[    \text{需要的分子个数}=\frac{1{\rm m}}{10^{-10}{\rm m}}=
    10^{10}\]
\end{solution}
\item  1$\rm cm^3$水中含有多少个水分子?10克氧中含有多少
个氧分子?

\begin{solution}
    水分子的体积大约是$3\x10^{-29}{\rm m}^3$, 所以1厘米
水中含有的水分子个数约为
\[\frac{1{\rm cm}^3}{3\x 10^{-29}{\rm m^3}}=\frac{1\x 10^{-6}{\rm m^3}}{3\x 10^{-29}{\rm m^3}}=3\x 10^{22}\]
\[
\text{10克氧的摩尔数}=\frac{1.0\x 10^{-2}{\rm kg}}{3.2\x10^{-2}{\rm kg/mol}}=0.31{\rm mol}\]
所以10克氧中含有的氧分子个数约为
\[0.31{\rm mol}\x 6.02\x10^{23}{\rm mol}^{-1}=1.9\x10^{23}\]
\end{solution}
\item  一个氧分子、一个氮分子的质量各是多少千克?

\begin{solution}
    一个氧分子的质量
\[m_{\rm O_2}=\frac{3.2\x 10^{-2}{\rm kg/mol}}{6.02\x 10^{23}{\rm mol}^{-1}}=5.3\x 10^{-26}{\rm kg}\]
    一个氢分子的质量
    \[m_{\rm H_2}=\frac{2\x 10^{-3}{\rm kg/mol}}{6.02\x 10^{23}{\rm mol}^{-1}}=3\x 10^{-27}{\rm kg}\]
\end{solution}
\item  已经测得一个碳原子的质量是$1.995\times 10^{-26}$千克,
求阿伏伽德罗常数.

\begin{solution}
    已知一个碳原子的质量是$1.995\x10^{-26}$千克,碳的摩
尔质量取$1.200\x10^{-2}{\rm kg/mol}$,所以阿伏伽德罗常数
\[N=\frac{1.200\x10^{-2}{\rm kg/mol}}{1.995\x10^{-26}{\rm kg}}=6.015\x 10^{23}{\rm mol}^{-1}\]
\end{solution}
\item  已知金刚石的密度是$3500{\rm kg/m^3}$,有一小块金刚
石,体积是$5.7\times 10^{-8}{\rm m^3}$.这小块金刚石中含有多少个碳原
子?设想金刚石中碳原子是紧密地堆在一起的,估算碳原子的
直径.

\begin{solution}
    这一小块金刚石的质量
\[m=\rho V=3500{\rm kg/m^3}\x 5.7\x 10^{-8}{\rm m^3}=2.0\x 10^{-4}{\rm kg}\]

$2.0\x 10^{-4}{\rm kg}$碳的摩尔数为
\[\frac{2.0\x 10^{-4}{\rm kg}}{1.2\x 10^{-2}{\rm kg/mol}}=1.7\x 10^{-2}{\rm mol}\]
这一小块金刚石所含的碳原子个数为
\[1.7\x10^{-2}{\rm mol}\x6.02\x10^{23}{\rm mol^{-1}}=1.0\x10^{22}\]
一个碳原子的体积为
\[\frac{5.7\x10^{-8}{\rm m^3}}{1.0\x10^{22}}=5.7\x10^{-30}{\rm m^3}\]
把金刚石中的碳原子看成球体,则由公式$V=\dfrac{\pi}{6}d^3$,\quad $d=\sqrt[3]{\dfrac{6V}{\pi}}$
可得碳原子的直径约为
\[\sqrt[3]{\frac{6\x5.7\x10^{-30}}{3.14}}{\rm m}=2.2\x10^{-10}{\rm m}\]
\end{solution}
\end{enumerate}

\subsection{练习二}
\begin{enumerate}
\item 有人说布朗运动就是分子的运动,这种说法对吗?为
什么?

\begin{solution}
这种说法不对,因为做布朗运动的微粒是由千千万
万个分子组成的,而分子的运动我们是看不见的,但是微粒的
布朗运动的无规则性,却反映液体内部分子运动的无规则性。
\end{solution}

\item 为什么悬浮在液体中的颗粒越小,它的布朗运动越
明显?为什么悬浮在液体中的颗粒越大,它的布朗运动越不
明显以至观察不到?

\begin{solution}
悬浮在液体中的颗粒越小,在某一瞬间跟它相撞的分
子数越少,在不同方向上颗粒受到的撞击作用也就越不平
衡。同时颗粒越小,它的质量也越小,在受到液体分子撞击
时也容易改变其运动状态 所以颗粒越小,它的布朗运动越
明显。

悬浮在液体中的颗粒越大,在某一瞬间跟它相撞的分子
数越多,颗粒在各个方向受到的撞击作用越接近平衡状态。所
以颗粒越大,它的布朗运动越不明显以至观察不到。
\end{solution}

\item 为什么说布朗运动的无规则性反映了液体内部分子运动的无规则性?设想液体内部分子的运动是有规则的,比如在任何时刻所有分子都向着某个方向运动,还能不能产生布朗运动?

\begin{solution}
    因为布朗运动是由液体分子不断地撞击悬浮在液体
中的微粒而引起的,所以说布朗运动的无规则性反映了液体
内部分子运动的无规则性。

如在任何时刻所有液体分子都是做有规则的定向运
动,那么微粒的运动就会有一定的规则,因而也就不能产生布
朗运动现象了。
\end{solution}

\item  图1.5中所示的不同小颗粒的布朗运动的情况并不相同,人们由此考虑到布朗运动不可能是由外界影响引起的.为什么?找几位同学一起讨论一下,并说明你的理由.

\begin{solution}
    假若小颗粒的布朗运动是由某一确定的外界因素所
引起的,那么在同一影响下,不同小颗粒的运动情况就会相
同.但图1.5所示的情况正好跟上述假设相反,所以产生布
朗运动的原因不可能在外部,只能在液体内部。
\end{solution}

\end{enumerate}

\subsection{练习三}
\begin{enumerate}
	\item 什么事例说明分子间有引力?什么事例说明分子间有斥力?
	
\begin{solution}
    用力拉伸物体,物体内要产生反抗拉伸的弹力。把两
    块铅压紧,两块铅就合在一起。这说明分子间有引力。

    固体和液体很难被压缩,即使气体,压缩到一定程度后再
    继续压缩也很困难,这说明分子间有斥力。

    (答案中所举事例,可不限于教材上已提到过的。)
\end{solution}
	\item 当分子间的距离大于$r_0$时,随着距离的增大,引力和斥力哪个减小得快?当分子间的距离小于$r_0$时,随着距离的减小,引力和斥力哪个增加得快?
		
\begin{solution}
在前一种情况下斥力减小得快;在后一种情况下斥力
增加得快。
\end{solution}
	\item 物体为什么能够被压缩,但又不能无限地被压缩?
		
\begin{solution}
    组成物体的分子之间是有空隙的。用力压物体时,物
    体内分子间的距离缩小,就表现出物体的可压缩性。但随着
    分子间距离的缩小,斥力会迅速增大,压缩到一定程度时,斥
    力与外力平衡,物体就不能再被压缩了。
\end{solution}
\item	从图1.7看出,当分子中心间的距离小于$r_0$时,分
子间的作用力表现为斥力,它随着距离的减小而很快地增大.分子间作用力的这一特点,可以借助于下述模型想象出来.设想分子为弹性钢球,当两个钢球相撞时,它们都发生微小的形变,因而在它们之间产生相互推斥的弹力,如同分子间的作用力表现为斥力一样.钢球发生微小形变就可以产生很大的弹力,所以这个弹力随着钢球中心间距离的减小而很快地增大.利用这一模型可以粗略地估计出分子直径的数量级为$10^{-10}$米.这是怎样估计的?
	
\begin{solution}
    设想分子为弹性钢球,当两个钢球恰好接触,但并未
    相互挤压时,它们之间不发生力的作用;当两个钢球相撞时,
    它们都发生微小的形变,因而在它们之间产生相互推斥的弹
    力.从课本图1.7看出,当分子间的距离为$r_0$时,它们之间的
    作用力为零;当分子间的距离小于$r_0$时,它们之间的作用力表
    现为斥力。于是可以认为,两分子间的距离为$r_0$时,它们恰
    好接触,但并未相互挤压。已经知道,$r_0$的数量级约为$10^{-10}$
    米,因此可以粗略地认为分子的直径约为$10^{-10}$米.
\end{solution}
\end{enumerate}

\section{参考资料}

\subsection{分子运动论发展简史}

根据记载和传说,远在2500多年前,古希腊就有物质
是由某些元素所组成的假说,其中以物质的原子论最为深
刻.约在公元前462年左右—370年左右,古希腊的著名思
想家德谟克利特归纳了古代的原子论。他认为万物皆由大量
不可分割的微小物质粒子组成,这种粒子叫做原子(希腊文为
atomos, 即不可分割的意思)。按照德谟克利特的观点,各种
原子没有质的差别,只有大小、形状和位置的差异;原子不断
地在空虚的空间中运动着;世界是由运动的原子及其组合物
构成的,任何自然现象都可以用这些原子的各种组合给以解
释。德谟克利特还举出实例来说明他的学说。比如他曾经这
样解释过花的香味:从花中飞出来的原子冲进人们的鼻孔里,
于是引起了有香味的感觉。德谟克利特的原子论在古希腊后
期和古罗马时期曾经有所发展。这种古代的原子论,虽是对
物质结构的一种朴素猜测,但它的基本思想,即无限的虚空和
在其中运动的粒子,则是近代原子理论的先驱。

此后经过了许多年,物结构的学说长期没有得到发
展。在中世纪,原子论受到了宗教的非难和压制,基督教会就
曾禁止传播“世界一切是由原子构成的”这种无神论思想。

直到17—18世纪,由于产业革命的推动,蒸汽机得到改
进和普遍使用,使得提高热机效率成为社会的迫切要求,因而
促进了热学的发展,促使人们开始探索热现象的本质,于是出
现了分子运动论学说,1658年,伽森第以分子运动论的观点
解释了物质的固、液、气三态的区别 接着,胡克和伯努利等人
发展了分子运动论.1738年,伯努利在《流体动力学》一书
中,根据气体是由激烈地运动着的大量粒子所组成这一假说,
解释了气体的压力是由粒子对器壁的碰撞而产生,并通过考
察气体体积变化时碰撞次数的变化,得出压力与体积成反
比。在伯努利的论述中,包含有分子动量的变化产生压力这种
卓越的思想。罗蒙诺索夫继续发展了分子运动论,他在《关于
冷和热的原因的探讨》一文中,提出了如下的设想:构成物
体的微粒极小,因此肉眼不可能看见微粒本身的运动,但是
它的运动表现在无数的现象之中。热无非是微粒的运动而
已。尽管这些先驱者的学说中有许多正确的观点和锐敏的构
思,但因缺少定量的实验基础,没有从数学上进行理论推导,
这时的分子运动论还是处在定性的探讨阶段.而18世纪的
热学,以热质守恒为基本原理,已积累了不少实验数据,认为
热是一种运动的表现,在当时自然难于得到公认,甚至到了
19世纪30年代,在热学中占统治地位的理论仍然是热质
说。这表明,要使分子运动论为大家普遍接受,还有待于能从
更广泛的角度阐明热与其他运动形式相互转化的能的转化和
守恒定律的确立。

19世纪中叶,建立了能的转化和守恒定律,否定了热质
说,为分子运动论的发展开辟了道路。此后,定量而系统的分
子运动论迅速发展起来,经过大约半个世纪的时间,克劳修
斯、麦克斯韦、玻尔兹曼等就在前人工作的基础上建立起较为
完善的分子运动论.1857年—1858年,克劳修斯根据气体分子
对器壁产生的冲量算出了气体的压强,解释了有关的气体实
验定律,又在气体分子运动论中引入了平均自由程的概念。
1860年,麦克斯韦导出了分子运动的麦克斯韦函数分布律.
至此,气体分子运动论的思想方法已大体完备.1865年,洛喜
密特算出了分子的大小,为分子运动论的发展打了坚实的基
础。玻尔兹曼进一步研究分子运动论,与麦克斯韦共同建立
了能量均分原理。从19世纪后期到本世纪初,在玻尔兹曼、
麦克斯韦和吉布斯等的努力下,建立了把分子运动论作为一
个分支包括在内的经典统计力学,这是一门用统计方法研究
由大量微观粒子所组成的系统的科学,它能对热力学已经获
得的结果从微观角度给以深刻解释。

就在分子运动论迅速发展的时期,以奥斯特瓦尔德为首
的一些科学家曾强烈反对过原子、分子论。他们认为原子、分
子是无用的、多余的假说,一切自然现象只要看作是能量的转
换,仅从热力学的观点出发,就能得到解释,本世纪初,由于
爱因斯坦和其他科学家从理论上和实验上对布朗运动进行了
深入研究,提出了布朗运动的定量公式,这就充分证实了分子
热运动的真实性,表明分子运动论不但不是多余的、无用的假
说,而且是无可置疑的事实。从此分子运动论就成为科学家
们公认的理论了。

\subsection{离子显微镜简介}

离子显微镜,又叫场离子显微镜,是本世纪50年代后期
出现的仪器.由于它的放大倍数在$10^6$以上,而且分辨能力高
达0.2—0.3纳米,因此,利用离子显微镜可以直接观察固体
表面原子实际排列的状况,或者说可以“看到原子”。

离子显微镜的“成像”机理,课本上已经讲了,它是靠镜
内空间的氮离子在5千至3万伏的高压电场作用下离开针尖,
沿着电力线运动打到荧光屏上使之发光。这样,以氦离子作媒
介,就把针尖上的点与荧光屏上的点一一对应起来了。因为
电力线是从针尖向外辐射的,所以荧光屏上一个个分散的光
点所占的范围要比针尖的表面积大许多倍,从而起到了放大
针尖上钨原子分布图样的作用。

从微观角度看,针尖表面是一个半径很大的球面,由钨原
子紧密排列而成的球面还很不光滑,由于带电体附近的电场
强弱分布跟带电体的形状有关,在针尖表面突出的原子附近,
电场最强,氦原子也就在这里被电离,因此,荧光屏上打出的
光点是与针尖表面上处于突出位置的原子相对应的。

在离子显微镜中,氦原子被电离而离开针尖时还在做热
运动(振动),在与半径垂直的方向上(横向)还有一定的速
度。这个速度的存在,会使氨离子打到荧光屏上的光点偏离
不考虑热运动时的理论位置,研究表明,由于横向速度的影
响,氦离子到达荧光屏时要落在以理论位置为圆心的一个圆
面上。邻近的圆面可能互相重叠,使得离子像的分辨能力大
为降低,为了提高离子显微镜的分辨能力,既需要设法减小
氦离子热运动的速度,又需要提高针尖与荧光屏之间的电压
以缩短氦离子从针尖飞到荧光屏的时间。现在人们已用液氨
或液氮、液氢来冷却针状电极,使氦离子的热运动速度大大减
小;选用很难电离的氮气作成像气体,也是为了尽可能地提高
针尖与荧光屏之间的电压。

离子显微镜内要抽成真空,然后充入氦气,通常应使镜
内的真空度达到$1.33\x10^{-4}$—$1.33\x10^{-7}$帕.离子像的亮度
与氦气的压强有关,压强高,离子像相对地明亮些,但压强太
高,氦原子的密度就大,这会增加氦离子与其他氦原子碰撞而
偏离原来运动方向的机会,导致荧光屏的离子像变得模糊。
一般应将氦气压强控制在$1.33\x10^{-4}$—$1.33$帕之间.此外,
所用氮气必须有较高纯度,否则也会严重影响氦离子所成
的像。

如需进一步了解离子显微镜,可看“场离子显微镜简介”
(《物理》82年10期,作者陆华;《物理教师》1985年6卷,作
者杜敏)、“场离子显微镜技术介绍”(《物理》1983年1期,作
者陆华)等文章。

\subsection{布朗运动}
按照分子运动论学说,分子的无规则运动服从统计规律,
即分子向各个方向运动的几率相等,同时,涨落现象是无规则
运动不能避免的,据此可以说明在一定的悬浊液中布朗运动
是否明显,取决于颗粒的大小 如果颗粒的线度大于$10^{-6}$米,
其周围的液体分子对它碰撞的次数极多,在各个方向上它受
到碰撞的次数就相差不大,涨落现象就不明显,我们也就难以
观察到它的布朗运动,但是当颗粒的线度足够小(不超过$10^{-6}$
米或$10^{-7}$米)时,相对来说颗粒受到液体分子的碰撞次数较
少,碰撞作用出现不平衡的机会就随之增大,即涨落现象变得
明显,使颗粒产生移动的倾向增大。这样,颗粒将向所受冲力
的合力方向运动,但因分子运动的无规则性,颗粒在某一时
刻所受合力是偶然的,作用力在各个方向上是机会均等的,因
此颗粒的运动看起来杂乱无章而内部又蕴藏着一定的规律,
显然,颗粒的运动不是分子运动,但它和液体分子的运动规律
类同。所以说布朗运动揭示了分子的无规则运动。

顺便指出:通常,在光的照射下,我们看见空气中灭尘的
运动乃是气流所引起的,并不是悬浮在气体中的颗粒的布朗
运动。

\subsection{分子间的相互作用力}
现在,人们公认自然界有四种力,即万有引力、电磁力、弱
力和强力。在微观领域里,万有引力同其他三种力相比就显
得微不足道,它的强度只有强力的$10^{39}$分之一,电磁力的$10^{37}$
分之一,弱力的$10^{25}$分之一。因此,对于分子、原子来说,万有
引力可以忽略不计。

弱力和强力的有效作用距离约在$10^{-15}$米以内,在原子
和原子核中这两种力才起显著作用。可见,分子与分子之间
的作用力属于电磁力。

分子、原子都是复杂的带电系统,深入讨论分子间的相互
作用力,要涉及量子力学理论。这里只做最简单的定性说明。

分子间的引力作用来源于分子中的原子核对相邻分子的
电子云的静电引力,具体说来有三种情况:第一,有些物质的
分子是有极性的,即其正电荷的中心不跟负电荷的中心重合,
相当于电偶极子,一个“电偶极子”的正电端与另一个“电偶极
子”的负电端互相吸引就形成了分子之间的引力。第二,极性
分子使邻近的非极性分子极化成为感生偶极子,它跟原来的
极性分子之间的作用形成分子之间的引力。第三,即使在上
面讲的两种力都不存在的非极性分子构成的物质中(如惰性
气体),由于电子的运动,原子中也存在着瞬时电偶极矩,它在
邻近空间中激发出瞬变电场,从而在邻近分子中感生出电偶
极矩来,分子之间就产生了引力。F·伦敦于1930年证明,对
于大多数分子,三种作用中的后一种作用引起的分子引力是
最大的。

分子间的引力是很弱的,只有在分子彼此接近到几乎相
接触时,才起作用。分子之间还同时存在着斥力,它来源于相
邻分子的电子云间的排斥力和相邻分子的原子核之间的排斥
力,不过引力要比排斥力强一些,但是当分子间彼此接近到
它们的电子云发生重叠时,情况就改变了。这时重叠区域中
电子云的密度增大,电子的能也随之增大(服从泡利不相容
原理),因而发生强烈的排斥作用。另外,由于原子核相互间更
接近了,静电斥力也会增大。这时,分子间的斥力就超过引
力了。

































\chapter{内能、能的转化和守恒定律}
\minitoc[n]
\section{教学要求}
这一章主要讲述内能的概念、热力学第一定律以及普遍
的能的转化和守恒定律,这些内容贯穿整个热学教材,是热
学的基础知识。能的转化和守恒定律是自然界的一个普遍定
律,学生很好地掌握这个定律,逐渐学会从能的转化的观点
来分析物理现象和解决物理问题,不仅对学好热学而且对学
好整个物理学都很重要,因这一章在整个中学物理教学中
占有重要地位。

这一章也是在初中物理的基础上讲解的。跟初中比较,
虽然重述了某些内容,但对概念的讲解和对问题的分析都有
所扩展和加深。为了便于初高中的衔接,在教学中应重新做
一下初中做过的演示。

这一章也是力学部分机械能一章的继续和延伸,那里讲
了机械能守恒定律以及功和能的关系,这里进一步扩展为热
力学第一定律以及能的转化和守恒定律。因此在教学中适当
联系或者复习机械能一章所讲的内容,特别是功和能的关系
以及功是能的转化的量度,对这一章的教学将会有所帮助。

这一章可分为两个单元:

第一单元包括第一节至第三节,讲述物体的内能、改变内
能的两种方式以及热功当量。

第二单元包括第四节至第六节,讲述热力学第一定律、能
的转化和守恒定律以及能源的利用和开发。

这一章的重点是物体的内能以及热力学第一定律和能的
转化和守恒定律。

讲述内能时,教材在前一章知识的基础上,先说明了温度
的微观含义,这里首先要使学生了解,在热现象中,人们所关
心的不是个别分子的情况,而是大量分子所表现出来的集体
行为。使学生一开始就了解这个观点,对于他们今后学习用
分子运动论解释热现象,将会有所帮助,考虑到在中学阶段
对分子的动能不展开讲解可能便于教学,因而没有明确区分
分子的平动能、转动能和振能,关于温度,只是笼统地指出
“温度是物体分子运动的平均动能的标志”。这里,实际上是把
分子当作质点来处理的。这一点,教师要有所掌握,对学生则
不必做进一步解释,同样,教材对内能的概念也不过于追求
严格,只给出内能的狭义定义,即所有分子无规则运动的动
能和分子势能的总和(内能的广义定义,不仅包括分子的动
能和势能,还包括原子内的能量、原子核内的能量等)。

跟中学阶段内能的狭义定义相适应,教材指出内能是由
温度和体积决定的。这里,为了讲解方便,教材是把温度和体
积这两个参量作为自变量来看待的。其实,对某些简单系统
(如气体、液体和各向同性固体)来说,由于温度、体积和压强
三者由物态方程联系起来,所以上述三个量中的任何两个都
可以作为独立参量,而内能作为状态函数由两个独立参量所
决定。这一点,教师要有所掌握。

讲过内能之后,要求学生能够区别开内能和机械能。除
了课文明确指出这一点,还安排了区分机械能和内能的练习
题(练习一第5题).

通过学习物体的内能这一节教材,要求学生从能的观点
对热的本质有明确的认识。教材安排了阅读材料,使学生了
解人类对热的本质有过曲折的认识过程。教师也可以在课堂
上简单介绍一下热质说,讲一下人类对热的本质的认识过程。
第二节讲述改变内能的两种方式。这方面的知识,在初
中讲过,为了使学生回忆起初中讲过的事实,重做一下初中
做过的演示还是必要的。教材讲的做功可以改变内能,实际
上仍限于机械功,不涉及其他形式的功(如电流做功等)。为
了使学生具体认识机械功,教材举出了几种不同的做功情况。
对气体膨胀做功,教材不要求定量地推导公式$W=p\Delta V$。讲
述这一节教材,还应着重使学生认识到,在只有做功或热传递
的情况下,可以分别用功或热量来量度内能的改变,以便为讲
解热力学第一定律打好基础。

热功当量的测定,在历史上有重要作用,为能的转化和守
恒定律奠定了坚实基础(既然在多种不同的实验中,消耗同样
多的机械能总会产生相同数量的热,这就意味着热是能的一
种形式,并且在机械能转化为内能的过程中能量是守恒的)。
同时定量地证明了功和热量是等效的。热功当量的意义并不
仅是确定了两个单位的换算关系,通过讲述热功当量,应当使
学生对这些有所理解。

第四节讲述热力学第一定律与能的转化和守恒定律。热
力学第一定律也就是宏观过程中能的转化和守恒定律。在热
力学第一定律的表达式$W+Q=\Delta E$中,如果功$W$包括各种功
(机械功、电功等),$E$是指研究对象的总能量(包括各种形式
的能,也包括机械能在内),那么,热力学第一定律就是宏观过
程中的普遍的能的转化和守恒定律,在微观世界中,对个别
粒子间的相互作用,能量守恒定律仍是成立的,但热力学第一
定律在此没有意义,因此不能把这两个定律完全等同起来。

为了从热力学第一定律扩展为普遍的能的转化和守恒定
律,教材在给出热力学第一定律的表达式后,着重从能的转化
的观点进行分析。一方面,从能的转化的观点指出功和热量
的不同,同时也从能的转化的观点明确了热力学第一定律的
意义,在此基础上引出能的转化和守恒定律 并举出一些能
的转化的实例,要求学生学习从能的转化的观点来分析物理
现象。

永动机(指第一类永动机)不可能制成,在历史上对建立
能的转化和守恒定律起了重要作用;人类是通过正反两方面
的经验总结出能的转化和守恒定律的。为了使学生理解这一
点,教材简单说明了永动机不可能制成,讲解这个问题,应使
学生确信:人类利用自然必须遵从自然规律,违反自然规律,
必将一事无成。

能的转化和守恒定律是自然界的一条普遍的、重要的规
律。为了使学生具体认识这一点,教材第五节讲述这个定律
的重要意义,在这方面,教师可以做些补充,使学生对这个问
题的认识更丰满一些。

第六节讲述能源的开发和利用。讲述这类实际知识,应
该要求学生对科学技术与社会生活的联系有一个初步的了
解,不要求涉及更多的技术问题,还要注意联系已有的物理
知识来讲解,使学生知道人类怎样利用物理知识来寻求解决
能源问题的道路。

归纳以上所述,这一章的教学要求是:
\begin{enumerate}
\item 理解温度的微观含义,理解内能的概念,了解物体的
内能跟物体的温度和体积有关。
\item 了解改变内能的两种方式,了解内能的变化可以分别
由功和热量来量度,理解热功当量的意义。
\item 掌握热力学第一定律,能够从能的转化的观点理解这
个定律,并会用表达式$W+Q=\Delta E$来分析计算有关问题。
\item 掌握能的转化和守恒定律,理解这个定律的重要意
义,学会从能的转化和守恒的观点来分析物理现象解决物理
问题,培养综合运用力学知识和热学知识的能力。
\end{enumerate}

\section{教学建议}
\subsection{第一单元}
在本单元中,教材讲述了什么是内能,指出物体的内能
跟温度和体积有关,从而使学生认识与热运动相联系的一种
新的能的形式。又讲述了怎样可以改变物体的内能,说明做
功和热传递对改变内能是等效的。这样,学生对改变能量的
方式的认识也将有所扩展。教师要引导学生在正确理解内能
概念的基础上,注意领会怎样从能量的观点来考虑和认识
问题。

\subsubsection{温度的微观含义} 教学时,可向学生强调指出:
\begin{enumerate}
\item 
物体里各个分子的动能并不相同。但物体里的分子数目极其
巨大,所有分子又都处在永不停息的无规则运动之中,要想求
出每个分子的动能是不可能的,也是没有必要的。在热现象
的研究中,人们关心的不是物体里每个分子的动能,而是分子
热运动的平均动能。或者说,人们研究的是大量分子所表现
出的集体行为,而不是个别分子的运动情况,为此,建立了统
计方法。这是用分子运动论解释热现象的基本思想方法。在
一章讨论分子的速率分布和气体压强的微观解释等问题
时,将初步介绍这种方法。    \item 物体的温度升高,分子热运动
的平均动能增加;温度降低,分子热运动的平均动能减小。物
体每一确定的温度,都跟一定的分子热运动的平均动能相对
应。所以说温度是分子热运动的平均动能的标志。但不能
说“温度等于分子热运动的平均动能。”
\end{enumerate}


温度是热学中一个重要的基本概念。学生从初中到高中
学习这个概念经历了由表及里、由现象到本质的认识过程。在
初中,只知道温度表示物体的冷热程度,至于物体为什么冷,
为什么热,它的内在因素是什么,那时是不了解的。经过前
一章分子运动论基础的学习,了解到温度与分子的无规则运
动有着紧密联系。温度越高,分子无规则运动越激烈。现在
又知道温度是分子热运动的平均动能的标志,这就懂得了温
度这个宏观物理量的微观含义了,教师可启发学生通过小结
对上述认识过程有所体会。

\subsubsection{分子势能与分子间距离的关系} 讲述内能与物体的
体积有关,需要说明分子势能跟分子间距离的关系、教材定
性地讲述了这种关系,但学生对分子势能随分子间距离而改
变常常感到抽象、难懂.为此,建议教学时注意以下各点:
\begin{enumerate}
\item 
先引导学生复习一下力学中学过的有关知识,即外力克服弹
力(重力)做功,弹性(重力)势能增加,外力做了多少功,弹性
(重力)势能就增加多少。弹力(重力)做功,弹性(重力)势能
减小,弹力(重力)做了多少功,弹性(重力)势能就减小多少。
学生明确地回忆起这些知识,将有利于他们在学习知识时
引起联想和进行类比.    
\item 尽量采用比喻说明的方法,启发学
生通过思考理解分子势能随距离改变的情形。例如,教师可
以利用教材上的比喻,结合教材13页图1.7, 引导学生分段
讨论,逐步认识:分子间存在着相互作用力,因此分子间具有
由它们的相对位置所决定的分子势能,这跟弹簧具有弹性势
能相似。要改变分子间的相对位置,就必须克服分子间的作
用力而做功,这跟拉伸或压缩弹簧时,必须克服弹力做功相
似。弹簧没有形变时,弹簧的弹性势能为最小值。这跟分子
间的距离$r=r_0$时,引力和斥力互相平衡,合力为零,分子势
能为最小值相似。当分子间的距离$r>r_0$时,分子间的作用
力表现为引力,把分子间的距离增大,就必须克服分子间的引
力做功;距离增大越多,克服分子间引力做的功就越多,分子
势能也就越大。所以说,在分子间的作用力表现为引力时,分
子势能随着分子间距离的增大而增大。这跟弹簧被拉长时,
弹性势能增大的情形类似。当分子间的距离$r<r_0$时,分子
间的作用力表现为斥力,把分子间的距离减小,就必须克服分
子间的斥力做功,距离减小越多,克服分子间的斥力做的功就
越多,分子势能也就越大,所以说,在分子间的作用力表现为
斥时,分子势能随着分子间的距离减小而增大。这跟弹簧
被压缩时,弹性势能增大的情形类似。对于气体来说,分子间
的相互作用力是引力,当分子间的距离缩小时,分子间的引力
做功,所以分子势能减少。这跟弹力(重力)做功,弹性(重
力)势能减少相似。学生搞清楚上述的情况,就不难理解分子
势能与分子间的距离,因而与物体的体积有关了。
\end{enumerate}

\subsubsection{内能}

物体的内能是本章的一个重点.为了让学生
清楚地理解内能这个概念,建议引导学生认识:
\begin{enumerate}
\item 从微观角
度看,物体的内能是指物体里所有分子的动能与势能的总和,
不是指单个分子的动能与势能之和.   
 \item 内能是由物体的温
度和体积决定的(体积跟物体所含分子数目的多少有关),在
热现象的研究中,人们感兴趣的是内能的变化,而不是内能的
绝对数值,所以在练习一1—4题中,要求回答的都是内
能怎样改变,即内能是增加还是减少.   
 \item 内能是不同于机械
能的另一种形式的能量。机械能是就宏观物体整体来说的,
内能则是指物体里所有分子的动能与势能的总和,物体在具
有一定内能的同时,也可以具有一定的机械能,例如练习一
第5题中在高空飞行的炮弹就是这样。可以让学生讨论这
个题目,认清机械能与内能的区别,不把二者混为一谈,还可
以举出其他实例帮助学生思考,例如一个静止在地面的物
体,如果以地面作为势能的参考零点,那么这个物体的总机械
能为零,但是这个物体里的分子却始终处在永不停息的热运
动之中,所以它的内能绝不为零。 
\item 在工程上和日常生活中,
一般都只说热能,很少用到内能。实际上说热能的地方往往
就是指内能,所以说,热能是内能的一种通俗说法。
\end{enumerate}

\subsubsection{改变内能的两种方式} 这部分知识,学生在初中已经
学过。这里,可引导他们在复习已有知识(包括重新观察初中
做过的演示实验)的基础上进一步认识:
\begin{enumerate}
\item 热量是单纯由于
热传递使物体的内能发生变化时,用来量度物体内能变化的
物理量。物体吸收(或者说外界传递给物体)多少热量,物体的
内能就增加多少;物体放出(或说物体传递给外界)多少热量,
物体的内能就减少多少。这就是说,热量只能在热传递过程中
用来衡量物体内能增减的多少。除此以外,热量就失去了意
义,因此,说“物体具有或含有多少热量”,是没有意义的,因
而也是不科学的.    \item 在单纯由于做功或单纯由于热传递使
物体的内能发生变化时,可以分别用功或热量来量度物体内
能的变化。如果物体内能的变化是同时通过做功和热传递来
实现的,这时就既不能单独用功,也不能单独用热量来量度物
体内能的变化了.    \item 做功和热传递对改变物体的内能是等
效的。但两者还是有区别的。做功改变内能是物体有规则运
动的能量转化为分子无规则运动的能量;热传递则是分子的
无规则运动的能量在物体间的转移。
\end{enumerate}

\subsubsection{热功当量} 初中已讲过热功当量,知道热量的单位
“卡”和功的单位“焦”之间的换算关系。现在,要引导学生从
做功和热传递对改变物体内能等效的角度,认识热功当量是
指相当于单位热量的功的数值。并要向学生明确指出:$J=
4.2$焦/卡,不仅确定了“卡”和“焦”这两种单位的换算关系,更
重要的是,它表示实验测得传递给物体1卡热量使物体增加
的内能,相当于对物体做4.2焦的功所增加的内能,认识到
这一点,才算是懂得了热功当量的含义。

介绍焦耳测定热功当量的实验(教材22页图2.2),要
引导学生认识这个实验是利用做功使水的内能增加,与假定
不是由于做功而是由于热传递使水增加同样多的内能所需热
量之间的关系,来测出热功当量的数值的,还可以向学生介
绍:焦耳为了精确地测定热功当量的数值,不断改进实验方
法,先后采用各种方法做了400多次实验。现在教材上介绍
的这个实验,是从1845年开始,到1849年测定才告一段落.
在实验中,最初用的是水,后来改用鲸鱼油,实验结果表明,水
和鲸鱼油的热功当量数值是相当接近的。但焦耳并不满足,
又换用比热较小的水银来进行测定,以提高实验的可靠性,此
后,焦耳仍不停步,继续采取其他实验方法,他最后得到的实
验结果,其数值与现在公认的热功当量值相差不到5\%. 100
多年以前,焦耳就能做出这样准确的实验,真是难能可贵。焦
耳在长期实验工作中表现出的严肃认真、精益求精的科学态
度与不断创新的精神,在物理学史上也是相当突出的,值得我
们很好学习。

对测定热功当量在历史上的重要作用,可引导学生从以
下两点来认识。一点是定量地证明了做功和热传递对改变物
体内能的等效关系.这一点,可组织学生讨论练习二第4
题,让他们通过从反面设想,得出正确答案,获得较深刻的认
识。另一点是,用各种不同的方法来测定热功当量,都得到了
相同的结果。这就意味着机械能并未消失,它转化成了另一
种形式的能量,而且在转化过程中,能量是守恒的。比如,放
开使叶片在盛水的量热器中转动的重物,水的温度就升高一
定的度数.这个实验事实在19世纪中叶的科学界产生了重
大的影响,为能的转化和守恒定律的建立打下了坚实的基础。

\subsection{第二单元}
这一单元先讲热力学第一定律,然后再扩展为普遍的能
的转化和守恒定律,接着讲述能的转化和守恒定律的重要意
义,介绍能源的利用和开发。引导学生正确理解内能和其他
形式能量的转化,启发学生从能的转化的观点来分析物理现
象、解决物理问题,是本单元教学的中心问题,也是学生学好
这一章的关键。

\subsubsection{热力学第一定律}

这个定律阐明了当物体跟外界同
时发生做功和热传递的过程因而使物体的内能发生变化时,
功、热量和内能的变化三者间的定量关系。建议教学时注意
引导学生理解热力学第一定律数学表达式的意义。按照教材
的讲法,$W$表示外界对物体所做的功,$Q$表示物体吸收的热
量,$\Delta E$表示物体内能的增加,公式$W+Q=\Delta E$的意义是:如
果物体跟外界同时发生做功和热传递的过程,那么,外界对物
体所做的功$W$加上物体从外界吸收的热量$Q$, 等于物体内能
的增加$\Delta E$. 应向学生强调指出,公式中各量的正负号就是针
对上述情况规定的,如果规定正负号的法则有了改变,热力学
第一定律的表达式也就随之改变。这方面的知识,最好结合
讨论练习三第5题进行,以利于加深学生的印象,在应用公
式解题时,要提醒学生把公式中各量的单位都统一成焦,并特
别注意公式中各量的符号所反映的物理意义。

如教师认为有必要,还可以告诉学生:热力学第一定律提
供了定量地测定物体内能变化的方法。

\subsubsection{能的转化和守恒定律}

建议按教材的线索,引导学生
逐步深入地来认识这个自然界的普遍规律。即先从能的转化
的观点认清做功和热传递的区别,以及热力学第一定律的意
义。接着认识内能不但可以和机械能相互转化,还可以和其
他形式的能(如电能、化学能、光能等)相互转化,并且在转化
过程中能量守恒,然后认识各种形式的能都可以相互转化,
并且在转化中守恒,在此基础上,向学生强调指出:能的转化
和守恒定律是大量经验的总结,是实验定律。

为了让学生对能的转化的认识更加丰满,建议教师多举
一些常见的、类似练习四第1题那样的实例(如分析打夯时,
从人抬起石夯,到石夯落到地面后静止,其中能的转化过程;
转动着的空竹,发出嗡嗡的响声,转动不断变慢,其中能的转
化过程等)。还可以启发学生举出自己了解的实例,以利于培
养他们学习从能的转化的观点来分析物理现象。

对热力学第一定律与能的转化和守恒定律的联系、区别,
可由教师给以适当说明。

永动机不可能制成,是人们实践经验的总结,它从反面
证实了能的转化和守恒定律是不可违反的客观规律,远在
17—18世纪的时候,人们为了满足生产对于动力的日益增多
的要求,幻想制造一种不消耗任何能量,却能永久工作的机
器,在这种幻想指引下,曾经有许多人提出过各式各样的永动
机的设计,但是所有这些设计在实践中都失败了.因此,1775
年法国科学院宣布不再接受审查关于永动机的发明。这说明
在能的转化和守恒定律建立之前,科学界已经从长期积累的
经验中,认识到制造永动机的企图是没有成功的希望的 建
议在教学中适当补充一点这方面的历史背景材料,帮助学生
了解:制造永动机的失败,导致了能的转化和守恒定律的发
现。而这一定律的建立,对于永动机不可能制成,则给予了科
学的最后判决。它告诉人们必须丢掉违反自然规律的幻想,
遵循能的转化和守恒定律,去研究各种不同形式的能相互转
化的具体条件,以求得最有效地利用自然界所能提供的多种
能源。这个是人类利用自然的正确道路,也是值得后人认真
吸取的历史经验。

如果教师认为有必要,也可以举出永动机的实例(如高中
物理乙种本上册242页的永动机设计方案)来具体说明.但
这种说明不宜过细,应着眼于引导学生认识违反自然规律,必
将一事无成,甚至受到自然的惩罚。

\subsubsection{能的转化和守恒定律的重要意义}

建议第五节教材
由学生自己阅读,教师着重从以下几点引导他们加深认识。
\begin{enumerate}
\item 能的转化和守恒定律的建立,是许多科学家长期探索,辛
勤劳动的结晶。体现了理论与实验合,理论家与实际工作
者结合(如卡诺、科耳定等人对这一定律的建立也作了重要贡
献,他们既是工程师,又搞物理学研究),物理学与其他学科结
合(如生理学)的特点.
\item 这个定律是自然界的基本规律之
一。它能把理、化、生、天、地等学科和各种工程技术联系起来,
对于人们认识自然,改造自然,从事科学实践,具有巨大的预见性和指导作用,中微子的发现就是一个例证,所以有的科学
家说:“与其拒绝能量守恒定律,还不如想出一种新的能量形
式为好。”(彭加勒语)
\end{enumerate}

教师还可向学生指出:能的转化和守恒定律贯串在全部
物理学中,可以把各种物理现象联系起来。因此,应用能的
转化的观点来分析物理现象,解决物理问题,是很重要的物理
思维方法。

\subsubsection{能源的利用和开发}
 这个问题是大家很关心的世
界性问题.建议引导学生注意了解:
\begin{enumerate}
\item 在合理开发和有效
利用常规能源的同时,不断探索能的转化的新途径,大力开发
和利用新能源,是解决能源问题的正确道路。    
\item 促使燃料完
全燃烧,减少热量损失;充分利用“余热”,提高燃料的总利用
率,是节约能源的有效途径.    
\item 当前,原子能的开发和利用,
既可使我们获得巨大的能量,在经济上也比较合算.   
 \item 解决
能源问题,最根本的还是要靠科学研究和技术进步。
\end{enumerate}

在教学中,还可以结合我国在能源生产上取得的成就和
存在的问题,对学生进行爱国主义教育。激发他们为振兴中
华而努力学习和勇于探索的精神。

\section{实验指导}
\subsection{演示实验}
\subsubsection{压缩气体做功,气体的内能增加}
这个实验常用压缩气体引火仪来做。压缩气体引火仪如
图2.1所示,它主要由压缩手柄、活塞、圆筒、底座等组成.

若用硝化棉作燃烧剂,则应选用优质硝化棉(可取极少量
硝化棉在空气中燃烧,优质硝化棉的火焰明亮,燃烧过程极
快,无残留灰分),实验时,要先把绿豆大的片状硝化棉放在
圆筒底或活塞下面的钩上,然后将活塞垂直放进圆筒中,先用
不太大的力气下压活塞一次,进行预热,再用力快速将活塞压
下,硝化棉就会燃烧,发出明亮火光。

如果实验效果不好,应检查实验装置的密闭性能是否良
好。如果是底座漏气,可用凡士林密封;如果是活塞跟圆筒内
、壁接触不够紧密,就应更换活塞上的橡皮胀圈,同时在活塞上
薄薄涂一层缝纫机油。

\begin{figure}[htp]\centering
    \begin{minipage}[t]{0.48\textwidth}
    \centering
\includegraphics[scale=.6]{fig/2-1.png}
    \caption{}
    \end{minipage}
    \begin{minipage}[t]{0.48\textwidth}
    \centering
    \includegraphics[scale=.6]{fig/2-2.png}
    \caption{}
    \end{minipage}
    \end{figure}

\subsubsection{克服摩擦做功,物体的内能增加}
如图2.2所示,把演示仪器夹紧在桌边,往黄铜管里滴入
乙醚约占全管容积的1/4—1/5, 把塞子塞好,防止漏气,但也
不宜过紧.用麻绳或涂有松香的纱带(约1厘米宽)绕黄铜管
一、二周,手执绳子两端迅速往复拉动,使绳和管壁摩擦,经过
几分钟,管里乙醚便沸腾起来,产生乙醚蒸汽,把塞子冲开。

如果摩擦很久还不能把塞子冲开,可将塞子稍为拔松,但
要注意不能把头部正对管口,以防管内液体或蒸汽喷入眼里,
造成伤害。

这个实验也可用酒精来做,但酒精的沸点较高,演示时间
较长。


\section{习题解答}

\subsection{练习一}

\begin{enumerate}
	\item 壶里的水被加热而温度升高,水的内能怎样改变?液体的热膨胀很小,可不予考虑.
	
\begin{solution}
    水分子的平均动能增大,水的内能增加。
\end{solution}    
\item 一根烧红了的铁棍逐渐冷却下来,铁棍的内能怎样改变?固体的热膨胀很小,可不予考虑.
	
\begin{solution}
    铁分子的平均动能减小,铁棍的内能减少。
\end{solution}    
\item 容器里装着一定质量的气体,在保持体积不变的条件下使它的温度升高,气体的内能怎样改变?在保持温度不变的条件下把气体压缩,气体的内能怎样改变?
	
\begin{solution}
    容器里一定质量的气体,在保持体积不变的条件下使
它的温度升高时,气体的分子平均动能增大,内能增加;在保
持温度不变的条件下把气体压缩时,气体的分子势能减小,内
能减少。
\end{solution}    
\item 设想我们对固体进行压缩.当分子间的距离小于$r_0$时,随着固体被压缩分子势能怎样改变?
	
\begin{solution}
    当分子间的距离小于$r_0$时,分子间的相互作用力表
现为斥力。压缩固体使分子间的距离减小,必须克服斥力做
功,因此分子势能随着固体被压缩而增大。
\end{solution}    
\item 一颗炮弹在高空中以某一速度$v$飞行,由于炮弹中所有分子都具有这一速度,所以分子具有动能,又由于所有分子都在高处,所以分子具有势能.所有分子的上述动能和势能的总和就是炮弹的内能.上述说法正确不正确?为什么?
	
\begin{solution}
    不正确。因为炮弹的内能应是炮弹中所有分子的热
运动的动能和分子势能的总和;而题中所说的动能和势能则
是炮弹整体的动能和重力势能,它们的总和是炮弹的总机械
能,不是炮弹的内能。
\end{solution}    
\end{enumerate}

\subsection{练习二}
\begin{enumerate}
\item 举出几个实例来说明:做功可以改变物体的内能.

\begin{solution}
    例如,克服摩擦做功,物体的温度升高,内能增加;压
缩气体做功,气体的温度升高,内能增加;气体膨胀时做功,气
体的温度降低,内能减少。
\end{solution}
\item 锅炉中盛有150千克的水,由20$^\circ$C加热到100$^\circ$C,水的内能增加多少?

\begin{solution}
    水的内能增加等于水吸收的热量,即$\Delta E=Q$. 而
$Q=cm(t_2-t_1)$, 将已知数据代入此式,得
\[\begin{split}
    \Delta E&=Q=c_{\text{水}}m_{\text{水}}(t_2-t_1)\\
&=4.2\x10^3\x150\x(100-20){\rm J}=5.0\x 10^7{\rm J}
\end{split}\]
\end{solution}
\item 一个物体的内能增加了20焦.如果物体跟周围环境不发生热交换,周围环境需要对物体做多少焦的功?如果周围环境对物体没有做功,需要传给物体多少焦的热量?

\begin{solution}
    如果物体跟周围环境不发生热交换,需要对物体做
20焦的功;如果周围环境没有做功,需要传给物体20焦的
热量。
\end{solution}
\item 设想在测定热功当量的不同实验中得到的结果并不相同,还能不能得到结论说:做功和热传递对改变物体的内能是等效的?讨论一下这个问题.

\begin{solution}
    如果测定热功当量的不同实验得到的结果并不相同,
这就表明做功和热传递对改变物体的内能没有确定的数量关
系。所以不能得出做功和热传递对改变物体的内能是等效的
结论。
\end{solution}
\item 在图2.2所示的实验中,已知重物$P$和$P'$的质量都是14千克,水的质量是7千克,重物连续从2米高处落下20次后,水的温度升高0. 37$^\circ$C.不考虑传给量热器和外界的热量,试根据这些数据计算热功当量.

\begin{solution}
    重物$P$和$P'$从2米高处落下20次所做的功为
\[\begin{split}
    20\x 2mgh&=20\x2\x14\x9.8\x2.0{\rm J}\\
&=1.1\x10^4{\rm J}
\end{split}\]
质量是7千克的水,温度上升0.37$^\circ$C所需热量为
\[\begin{split}
    Q=cm\Delta t&=1.0\x7.0\x10^3\x0.37{\rm cal}\\
&=2.6\x10^3{\rm cal}
\end{split}\]
所以,热功当量
\[J=\frac{1.1\x10^4{\rm J}}{2.6\x10^3{\rm cal}}=4.2{\rm J/cal}\]
\end{solution}
\end{enumerate}


\subsection{练习三}
\begin{enumerate}
		\item 做功和热传递对改变物体的内能虽然等效,但从能的转化的观点来看是有区别的,这种区别是什么?
		
\begin{solution}
    做功使内能发生变化时,是其他形式的能和内能的转
化。热传递使内能发生变化时,只是内能在物体之间的转移,
没有能量形式的转化。
\end{solution}
	\item 在图2.2所示的焦耳测定热功当量的实验中,什么其他形式的能转化成了水的内能?在历史上,热功当量的确定为建立能的转化和守恒定律提供了坚实的实验基础.你怎样理解这句话?讨论一下这个问题.
		
    \begin{solution}
    是机械能转化成了水的内能。热功当量的确定,表明
    消耗同样多的机械能总会产生相同数量的热,这就意味着热
    是能的一种形式,并且在机械能转化为内能的过程中能量是
    守恒的,所以说热功当量的确定为建立能的转化和守恒定律
    提供了坚实的实验基础。
    \end{solution}
	\item 用活塞压缩气缸里的空气,对空气做了900焦的功,同时气缸向外散热210焦.空气的内能改变了多少?
	
\begin{solution}
由热力学第一定律$W+Q=\Delta E$, 根据题意,活塞压缩
空气,对空气所做的功$W$为正,气缸向外放出的热量$Q$为
负值。代入已知数据,则空气内能的改变
\[\Delta E=W+Q=900-210=690{\rm J}\]
\end{solution}
	\item 空气压缩机在一次压缩中,活塞对空气做了$2\times 10^5$焦的功,同时空气的内能增加$1.5\times 10^5$焦.这时空气向外界传递的热量是多少?
		
    \begin{solution}
因为$W+Q=\Delta E$, 所以
\[Q=\Delta E-W=1.5\x10^5-2\x10^5 =-5\x10^4{\rm J}\]
$Q$为负值,表示空气向外界放出的热量是$5\x10^4$焦.
    \end{solution}
	\item 如果用$Q$表示物体吸收的热量,用$W$表示物体对外界所做的功,热力学第一定律也可以表达为下式:
	\[Q=\Delta E+W\]
	怎样解释这个表达式的物理意义?试根据课文中的表达式推
导出这个表达式.
		
\begin{solution}
    在表达式$W+Q=\Delta E$中,按题意外界对物体所做的
功为$-W$, 代入上式可得$-W+Q=\Delta E$.所以
\[Q=\Delta E+W\]
这个表达式的物理意义是:物体从外界吸收的热量$Q$, 等
于物体内能的增加$\Delta E$, 加上物体对外所做的功$W$.
\end{solution}
\end{enumerate}

\subsection{练习四}

\begin{enumerate}
	\item 试说明下列现象中能量是怎样转化的:
	\begin{enumerate}
		\item 在水平公路上行驶的汽车,发动机熄火之后,速度越来越小,最后停止.
		\item 在阻尼振动中,单摆的振幅越来越小,最后停下来.
		\item 火药爆炸产生燃气,子弹在燃气的推动下从枪膛发射出去,射穿一块钢板,速度减小.
		\item 用柴油机带动发电机发电,供给电动水泵抽水,把水从低处抽到高处.
	\end{enumerate}

\begin{solution}
\begin{enumerate}[(a)]
    \item 机械能转化为内能。\item 机械能转化为内能。\item 化
学能转化为内能,内能再转化为机械能,机械能又转化为内
能。\item 化学能转化为内能,内能转化为机械能,机械能再转化
为电能,电能又转化为机械能。
\end{enumerate}
\end{solution}


\item 取一个不高的横截面积是30${\rm cm^2}$的圆筒,筒内装水0.6千克,用来测量射到地面的太阳能,在太阳光垂直照射2分钟后,水的温度升高了1$^\circ$C.计算在阳光直射下地球表面每平方厘米每分钟获得的能量.

\begin{solution}
圆筒内0.6千克水的内能的增加为
\[\begin{split}
   \Delta E=Q=mc\Delta t&=0.6\x4.2\x10^3\x1\\
&=2.5\x10^3{\rm J} 
\end{split}\]

每分钟0.6千克水增加的能量为$2.5\x 10^3/2=1.25\x
10^3$焦.

因圆筒的横截面积为$3{\rm dm}^2=3\x10^2{\rm cm}^2$. 所以地球
表面每平方厘米每分钟获得的能量为
\[\frac{1.25\x10^3{\rm J}}{3\x10^2}=4.2{\rm J}\]
\end{solution}
\item 从20米高处落下的水,如果水的势能的20\%用来使水的温度升高,水落下后的温度升高多少摄氏度?

\begin{solution}
    依题意可得$0.20mgh=mc\Delta t$, 所以
\[\Delta t =\frac{0.20mqh}{mc}\]
已知$h=20{\rm m}$,取$g=9.8{\rm m/s^2}$, $c=4.2\x10^3{\rm J/(kg\cdot^{\circ}C)}$
代入上式,得
\[\Delta t=\frac{0.20\x 9.8\x 20}{4.2\x 10^3}{\rm ^{\circ}C}=9.3\x 10^{-3}{\rm ^{\circ}C}\]
\end{solution}
\item 用铁锤打击铁钉,设打击时有80\%的机械能转化为内能,其中50\%用来使铁钉的温度升高.打击20次后,铁钉的温度升高多少摄氏度?已知铁锤的质量为1.2千克,铁锤打击铁钉时的速度为10${\rm m/s}$,铁钉的质量为40克,铁的比热为$5.0\times 10^2 {\rm J}/({\rm kg\cdot— ^\circ C})$.

\begin{solution}
铁锤打击铁钉,用来使铁钉温度升高的机械能为
\[0.50\x0.80\x20\x\frac{1}{2}m_1v^2\]
已知铁锤质量$m_1=1.2$kg,铁锤
打击铁钉时的速度$v=10{\rm m/s}$.代入已知数据得
\[0.50\x0.80\x20\x\frac{1}{2}\x1.2\x10^2=4.8\x10^2{\rm J}\]
则铁钉增加的热量$Q=m_2c\Delta t=4.8\x10^2$焦.所以铁钉
升高的温度\[\Delta t=\frac{4.8\x10^2{\rm J}}{m_2c}\]
已知铁钉的质量$m_2=0.040$kg,铁的比热$c=5.0\times 10^2 {\rm J}/({\rm kg\cdot— ^\circ C})$, 代入已知数据,铁
钉升高的温度
\[\Delta t=\frac{4.8\x10^2 }{0.040\x 5.0\x 10^2}—^{\circ}{\rm C}=24^{\circ}{\rm C}\]
\end{solution}
\item 在光滑的桌面上放着一个木块,铅弹从水平方向射中木块,把木块打落在地面上,落地点与桌边的水平距离为0.4米.铅弹射中木块后留在木块中.设增加的内能有60\%使铅弹的温度升高,铅弹的温度升高多少摄氏度?已知桌面高为0.8米,木块的质量为2千克,铅弹的质量为10克,比热为$1.3\times 10^2 {\rm J}/({\rm kg\cdot —^\circ C})$.取$g=10{\rm m}/{\rm s^2}$.

\begin{solution}
    铅弹打中木块后与木块一起做平抛运动的初速度
\[v=\frac{x}{\sqrt{\frac{2y}{g}}}=\frac{0.4}{\sqrt{\frac{2\x 0.8}{10}}}{\rm m/s}=1{\rm m/s}\]

用$M$、$m$分别表示木块和铅弹的质量,根据动量守恒定
律可以求出铅弹射入木块时的速度$v'$
\[v'=\frac{(M+m)}{m}v=\frac{(2+0.010)\x 1}{0.01}{\rm m/s}=2\x 10^2{\rm m/s}\]

系统机械能的损失为
\[\begin{split}
   & \frac{1}{2}{mv'}^2-\frac{1}{2}(M+m)v^2\\
&=\left[\frac{1}{2}\x 0.010\x(2\x10^2)^2-\frac{1}{2}\x(2+0.01)\x 1^2\right]\\
&=2\x 10^2 {\rm J}
\end{split}\]

故增加的内能$\Delta E=2\x10^2$焦.

由$cm\Delta t=0.60\Delta E$, 所以铅弹升高的温度
\[\Delta t=\frac{0.60\Delta E}{mc}=\frac{1.2\x10^2}{0.010\x 1.3\x 10^2}—^{\circ}{\rm C}={\rm 92^{\circ}C}\]
\end{solution}
\end{enumerate}


\section{参考资料}
\subsection{温度的微观含义}
将实验得到的理想气体状态方程跟气体分子运动论的压
强公式相比较,可以找出气体温度与分子的平均平动动能之
间的关系,从而揭示出温度这一宏观物理量的微观含义。

如果用$N$表$M$克气体的分子数目,用$N_0$表示1摩尔气
体的分子数目,$m$表示一个分子的质量,那么$M=mN$, $\mu=mN_0$,
把这些关系代入克拉珀龙方程$pV=\frac{M}{\mu}RT$, 得
\begin{equation}
    p=\frac{mN}{V}\cdot \frac{R}{mN_0}\cdot T=\frac{N}{V}\cdot \frac{R}{N_0}\cdot T
\end{equation}

由于1摩尔任何气体的分子数$N_0$都相同,$R$是普适恒
量,所以$R/N_0$
也是个常数,用$k$表示,叫做玻尔兹曼常数。

$N/V=n$,表示单位体积内的气体分子数,因此(2.1)式可写为
\begin{equation}
    p=nkT
\end{equation}

气体分子运动论指出,气体对于器壁的压强由下式决定
\begin{equation}
    p=\frac{2}{3}n\left(\frac{1}{2}m\bar{v}^2\right)
\end{equation}
式中,$\frac{1}{2}m\bar{v}^2$表示分子的平均平动动能。

比较(2.1)、(2.3)两式可得
\[\frac{1}{2}m\bar{v}^2=\frac{3}{2}kT\]

上式表明,宏观量温度只与气体分子的平均平动动能有
关。它指出气体分子的平均平动动能与热力学温度成正比。
这就是说,上式反映了气体温度的统计意义,即气体的温度
是大量分子平均平动动能的量度,是大量分子热运动的集体
表现,说个别分子有多高的温度,是没有意义的。

从分析温度的微观含义,我们可以得到启示:在热现象的
研究中,宏观量与微观量是描述同一物理现象的两种不同方
法中所用的量,因而它们之间必然是有联系的。

\subsection{热质说简介}

18世纪经典力学和天文学取得的成就,向人们指出了定
量地表示现象,用数学解析来进行处理,是物理学所必须遵循
的方法。在这种思想的影响下,为了开辟定量地处理电、磁、
热等现象的道路,当时的物理学界普遍采用了电流体、磁流体
和热流体(热质)的设想。

最早阐明热质说的是菏兰医生、化学家波哈夫(1668—
1738),而把热质说作为热学理论根据的是英国化学家布莱克
(1728—1799).他于1760年左右确定了热容量的概念,明确
叙述了热平衡的概念,首次区分了温度与热量,从而打下了定
量地研究热现象的基础。可以说,热现象研究的真正发展就
是从这时开始的.1789年,法国著名化学家拉瓦锡(1743—
1794)在他所著《化学纲要》一书中第一次引入了热质一词,到
十八世纪末,热质说已在热学理论中占了统治地位。

从总体上看,热质说自然是一种不正确的理论,但其中仍
包含有一些合理的东西。以热质说为基础的研究,在某些范
围内还是取得了很大的成果,这是因为在某些现象中热量守
恒是成立的。比如,在热量守恒的前提下,热传导的理论研
究和气体比热的测定都有了明显的进展;卡诺(1769—1832)
提出自己的理论时,最初也是站在热质说的立场上展开讨论
的。总之,历史上热质说的出现并在一段时间内得到大多数
科学家的承认,都不是偶然的,它反映了人们对热的本质的认
识经历了一个曲折的过程,19世纪中叶,能的转化和守恒定律
的建立,否定了热质说,为分子运动论的发展开辟了道路。分
子运动论取代热质说的史实表明,人们正是在不断地进行科
学实践中,抛弃原有的不正确的理论,保留其中的合理因素,
才建立起了更接近事实、更加完善的科学理论。

\subsection{能的转化和守恒定律的建立}
在18世纪40—50年代,焦耳(1818—1889)、迈耶(1814
—1878)、亥姆霍兹(1821—1894)等一批科学家,几乎在同
一时期内各自独立地提出了能的转化和守恒的观点,尽管他
们各自研究的范围和深广度并不一样,但在核心问题上的主
张却是相同的。这就意味着能的转化和守恒定律的发现,确
有某种时代因素在起作用。这些因素中,重要的是:
\begin{enumerate}
    \item 经过许多科学家的努力,经典力学在17—18世纪取
得了巨大成就。在经典力学中,已经蕴含着机械能的转化和
守恒的初步思想.18世纪90年代,伦福德(1753—1814)等
人对摩擦生热的实验进行了研究,否定了热质说,揭示了机
械能与物体内能变化的联系,从18世纪末到19世纪40年
代,随着物理学研究范围的扩大,陆续发现了许多现象相互
联系、相互转化的事实.例如,1800年发明了电池,不久就发
现了电流的热效应、磁效应、化学效应以及电磁感应现象等。
跟发明电池同年,还发现了红外线,经过对红外线的研究,人
们又了解到光能转化为内能的情况,在其他方面,如生物学
界,也发现了动物的体温和进行机械活动的能量跟它摄取的
食物的化学能有关。这一切,都为能的转化和守恒定的发
现作了必要的准备。
\item 18世纪中叶以来,在产业革命的推动下,生产技术
有了很大进步,社会对动力的需求也日益增多,尤其是蒸汽机的发展和迅速普及,促使人们普遍关心能的转化问题和提高
动力机的效率问题。正是这样的时代要求,把一些科学家和
工程技术人员引向了能的转化和守恒的研究。例如,焦耳开
始研究能量守恒的目的就是为了提高发动机的效率;而伦福
德是从事火药和武器研究的军官。
\end{enumerate}

以上表明,能的转化和守恒定律的建立,既是时代的要
求,也是自然科学本身发展到一定阶段的必然结果。

据一些历史资料介绍,首先了解和提倡能的转化和守恒
思想的,大都是些年轻人,而且他们的主要职业兴趣都不在物
理学方面.这些人中,迈耶是医生,28岁;亥姆霍兹是生理学
家,32岁;焦耳是酒厂主,25岁;卡诺也是工程师,34岁;伦福
德的年龄最大,45岁。这些情况,如教师认为有必要,可以向
学生介绍,并引导他们从中获得有益的启示。

\subsection{中微子与能量守恒定律}
自1914年直到30年代初,在$\beta$衰变的研究中,最令人困
惑不解的就是$\beta$粒子的连续能谱问题。

实验研究的结果表明,在$\alpha$放射性元素衰变过程中放出
的$\alpha$粒子能量值都是分立的.例如,镭($^{226}_{88}{\rm Ra}$)放出的$\alpha$粒
子有两种不同的能量值,铋($^{263}_{83}{\rm Bi}$)放出的$\alpha$粒子有六种不同
的能量值,钋($^{212}_{84}{\rm Po}$)放出的$\alpha$粒子有四种不同的能量值。这
就是$\alpha$粒子的不连续能谱。这种情况跟原子发出的光谱很相
似,人们由此认识到原子核内部也有能级存在,$\alpha$粒子的能谱
是跟核的能级分布相联系的。

我们以$^{226}_{88}{\rm Ra}$的衰变为例稍微详细地说明一下这个问
题。在
$^{226}_{88}{\rm Ra}\to ^{222}_{86}{\rm Rn}+^{4}_{2}{\rm He}$的衰变反应中,氡核有两个不
同的能级,一个是基态,一个是激发态,当镭核衰变为基态的
氡核时,放出的$\alpha$粒子能量较高:$E_{\alpha_1}=4793$兆电子伏;当镭
核衰变为激发态的氡核时,放出的$\alpha$粒子能量较低:$E_{\alpha_2}=
4.612$兆电子伏.由放出的$\alpha$粒子的能量$E_{\alpha}$, 应用公式$E_d=\left(1+\frac{m}{M}\right)E_{\alpha}$(式中的$m$和$M$分别为$\alpha$粒子和剩余原子核的质
量),可以求出与之对应的衰变能$E_d$(即发出的$\alpha$粒子的能量
与剩余原子核的反冲能量之和):$E_{d_1}=4.879$兆电子伏,$E_{d_2}=
4.695$兆电子伏.这两个衰变能的差值$\Delta E_{d}=E_{d_1}-E_{d_2}=0.184$
兆电子伏就是氡核两个能级间的能量差。当氡核由激发态跃
迁到基态时应该放出能量为0.184兆电子伏的$\gamma$射线
(图2.3),在实验中果然观察到Ra的$\alpha$衰变过程中伴随有
这种$\gamma$射线。所以从$\alpha$放射性的能谱可以获得原子核能级
的数据。
\begin{figure}[htp]
    \centering
\begin{tikzpicture}[>=latex]
    \node at (0,3){能量(MeV)};
\draw[very thick] (0,2)node[left]{4.879}--node[above]{Ra}(2,2);
\foreach \y/\ytext in {-0.5/0.184,-1/0}
{
    \draw[very thick] (3,\y)--(5,\y);
    \draw[dashed] (0,\y)node[left]{\ytext}--(3,\y);
}
\draw[->](4,-.5)--node[right]{$\gamma$,\; 0.189{MeV}}(4,-1)node[below]{Rn};

\draw[->,thick](.8,2)--node[left]{$\alpha$,\; 4.793MeV}(3.1,-1);
\draw[->,thick](1.5,2)--node[right]{$\alpha$,\; 4.612MeV}(3.5,-.5);
\end{tikzpicture}
    \caption{镭的$\alpha$射线与氡核的能级}
\end{figure}

$\beta$衰变中的电子($\beta$粒子),也是由原子核放出来的。根
据原子核内部的能级分布,有理由认为衰变中放出的$\beta$粒子
的能量也应该是不连续的,然而事实与人们的期望相反,$\beta$粒
子具有连续能谱.图2.4是实验测得的$\beta$能谱的一般情形。
\begin{figure}[htp]
    \centering
\includegraphics[scale=.4]{fig/2-4.png}
    \caption{}
\end{figure}

从这个能量分布曲线可以看到:衰变中放出的$\beta$粒子能量有
一个最大值$E_m$; 分布曲线的极大值对应的能量(表明放出的
该能量的$\beta$粒子数最多)约等于$E_m$的$1/3$。
可见,$\beta$粒子能量
的平均值要比$E_m$小。另一方面,实验测量的结果证明了$E_m$
正好等于理论上计算出的衰变能$E_d=(M_2-M_{2+1})c^2$. 式中
的$M_2$和$M_{2+1}$分别为衰变前后原子的质量.

$\beta$粒子能量的连续分布,使当时的物理学面临了严重的
困难:既然衰变过程中放出的粒子能量小于初态和终态之间
的能量差,那么失去的那部分能量到底哪里去了呢?经过多年
的研究,仍然不能揭开失去的能量之谜,致使一些著名的物理
学家(如玻尔)也准备在核领域中放弃能量守恒定律。在看来
毫无希望之时,泡利于1930年提出了一个大胆的设想:如果认
为在$\beta$衰变过程中,同时产生了一种未被察觉粒子,它带走
了一部分衰变能,上述矛盾就可以解决了。为什么这种粒子
没有被记录下来呢?是因为它不带电荷、没有静止质量,很难
跟其他粒子发生作用,因而观察不到它的出现所引起的效应。
泡利把这种粒子称为“中子”。1934年,费米为了把这种粒子
跟存在于原子核内的中子相区别,把它命名为中微子。$\beta$衰
变实际上是核内中子放出一个电子和一个中微子(后来知道
是反中微子)衰变为质子的过程:${\rm n}\to {\rm P}+^0_{-1}{\rm e}+\bar{\nu}$. 中微子的
假设,既保持了能量守恒定律又能圆满地解释$\beta$衰变的连续
能谱,使诞生不久的核物理从困境中摆脱出来。

尽管中微子的假设很成功,因为它跟所有粒子之间的作
用都非常弱(它可以毫无困难地穿过地球),所以要截获它证
明它的存在,是非常困难的,只有利用核反应堆中裂变产物
放出的强大反中微子流和高效率的探测装置,才能看到中微
子被原子核吸收时产生的效应,1956年人们终于在精心设
计的实验中证实了中微子的存在。实验的安排大体如下:

在反应堆附近,用一罐含有镉的水做质子靶,水罐外用一
层液态闪烁物做探测器。一旦反应堆中放出的反中微子被水
中的氢原子核吸收,就会产生反$\beta$衰变:${\rm P}+\bar{\nu}\to {\rm n}+ ^0_{-1}{\rm e}$, 放
出中子和正电子。正电子几乎立即会碰到电子湮成两个光
子。中子在水中穿行几微秒后,也会被镉原子核所俘获,产生
一个较重的处于激发态的镉原子核.这个核会发出3—4个
$\gamma$光子,释放出8兆电子伏的能量.探测器依次探测到电
子-正电子湮灭和镉激发核衰变时放出的光子,就证明了反$\beta$
衰变的发生和反中微子的存在。

\subsection{我国能源建设的成就简介}
1985年,我国煤产量已达85000万吨,由过去居世界第
三位上升为第二位;石油产量已达12500万吨,比1980年增
加1900万吨;发电量已达4073亿千瓦时,比1980年增加
1073亿千瓦时,我国在开发和利用新能源方面也有明显的
进展。例如,“六五”期间核电厂(站)的建设已经起步。目前
浙江秦山核电厂和广东大亚湾核电站的建设工程正在顺利
进行。






\chapter{气体的性质}
\section{教学要求}
从实验出发,建立气体实验定律,对实验定律进行研究而获得理想气体的状态方程,最后从微观上利用气体分子运动论和统计方法解释气体实验定律,本章教材的这种安排,较鲜明地体现了实验的作用和理论的指导意义,体现了宏观和微观的统一,这是本章教材的特点,注意到这一点,无疑对学生有科学研究方法上的启迪。

本章可分为四个单元,第一单元由第一节至第四节,从介绍气体的三个状态参量开始,研究气体的两个实验定律并引出热力学温标,第二单元包括五、六两节,主要研究理想气体的状态方程和克拉珀龙方程,第三单元由七、八两节组成,介绍气体分子运动的特点并对气体的实验定律进行定性的微观解释,第四单元为九、十两节,从能量的角度研究理想气体并讨论了理想气体在几个等值过程中内能的变化。

气体的三个状态参量,体积、压强、温度是研究气体性质的基础,在教学中,要求学生理解这三个状态参量的意义,由于压强的计算稍难一点,因此在教学中要多花一些力气搞清
压强的计算和单位间的关系。

气体三个实验定律反映了气体在温度不太低、压强不太高的条件下状态变化所遵循的规律,是建立理想气体状态方程的依据,因此,要求学生很好的掌握,对气体等温变化的图线和气体等容变化的图线的物理意义要清楚地理解。

教材是利用气体等容变化的$p$-$t$图线交于${\rm -273^{\circ}C}$引出绝对零度从而建立热力学温标的,热力学温标的建立对简化查理定律并导出理想气体状态方程起了重要作用 因此,要使学生掌握热力学温度和摄氏温度的换算关系,理解热力学温标和绝对零度的意义。

理想气体的状态方程反映了一定质量的理想气体三个状态参量间的变化关系,是本章的重点知识,对于进一步学习和生产实践都有重要的意义,要求学生巩固掌握,克拉珀龙方程是本章的重要知识,但不是重点,要求学生掌握,但不要去做过多过深的习题,以免加重学生的负担。

关于气体分子运动的特点,以及对气体实验定律的微观解释,教材在介绍知识的同时,渗透一些统计的观点,由于这部分内容比较抽象,在中学也不可能讲解透彻,因此,不宜讲得过多过深,不宜要求过高。

理想气体的内能一节介绍了理想气体的微观模型,指出了它的内能只跟温度有关,跟气体体积无关,了解了理想气体的微观模型,就能理解为什么实际气体在温度不太低,压强不太高时,其性质跟理想气体接近,并为进一步从能量角度研究理想气体打下基础。关于理想气体内能的变化,应根据学生情况决定讲与不讲,讲到什么程度。这一节内容,对热力学第
一定律用于理想气体的几个等值过程做了粗略的介绍,利于
扩大学生的知识面,但限于学生实际水平,只能要求他们定性地了解这些知识。

这一章的教学要求是:
\begin{enumerate}
 \item 理解描述气体状态的三个参量(体积、压强、温度)的意义,会计算气体的压强。
  \item 掌握玻意耳-马略特定律、查理定律、盖·吕萨克定律,理解气体等温变化的p-V图象和气体等容变化的p-t、p-T图象的意义,会用这三个定律计算有关的问题。
  \item 理解热力学温标和绝对零度的意义,掌握热力学温度和摄氏温度的换算关系。
  \item 巩固掌握理想气体的状态方程,掌握克拉珀龙方程,会用它们来计算有关问题。
  \item 了解气体分子运动的特点,了解气体压强的微观解释,会用气体分子运动论对气体的三个实验定律进行微观解释。
  \item 了解理想气体的微观模型,了解理想气体的内能只跟温度有关,跟体积无关。了解理想气体在几种等值过程中内能的变化情况。
\end{enumerate}
 
\section{教学建议}
\subsection{第一单元}
\subsubsection{气体的状态和状态参量}

本节研究了描述气体状态的三个参量,提出了本章研究的中心课题,即确定三个状态参
量之间的变化规律,指出了研究方法:保持其中一个量不变研究其他两个量的关系,然后再确定三个状态参量之间的关系。让学生在学习本节内容时,明确以上各点能够增强学生学习的目的性。

在介绍描述气体状态的状态参量时,应使学生明确地认识到,在物理学中因研究的问题不同,所用的物理量也不同。在热学中研究气体的性质,所用参量有体积、压强和温度。

在讲解本节内容时,应注意三点:
\begin{enumerate}
\item 气体体积是指气体分子充满的空间,即容器的容积.这个容积不是气体分子本身的体积之和,气体分子之间是有空隙的。
\item 对气体压强的计算方法,单位换算要给予足够的重视。因为它是后面学习气体定律和气态方程的基础,正确地确定气体的压强往往是应用气态方程解决问题的关键,恰当地选择压强的单位可以避免繁杂的计算。在教学中要多花些力气,学生也应做一定练习,务求掌握气体压强的计算方法和单位之间的换算关系。
\item 教材中提到“对一定质量的气体来说,如果体积、压强和温度这三个量都不改变,我们就说气体处于一定的状态中”。这个“状态”指的是气体的热动平衡态,有关热动平衡态,在课堂上无需涉及,在这里,只要求学生理解上面那段话的意思就行了。
\end{enumerate}

\subsubsection{气体的等温变化}
 玻意耳-马略特定律 节利用实验研究一定质量气体的等温变化,总结出玻-马定律,对等温线作了介绍,最后指明了玻-马定律的物理意义和适用
 范围。

 讲好玻-马定律的关键是做好演示实验,用教材中介绍的装置,也可以验证在温度不变的条件下,$pV=$恒量。

 在做演示实验前,要引导学生仔细观察实验装置,明确指出研究对象是$A$管中密闭的气体,提出下列问题让学生思考回答:
 \begin{enumerate}
 \item 如教材39页图3.5所示,$A$管中气体的压强的表达式是怎样的?
 \item $A$管中的气体体积怎样确定?
 \item 为了保证实验中气体恒温,$B$管的升降快慢有什么要求?为什么?实验中的恒温是什么温度?
 \end{enumerate}

 该实验不一定采用书上介绍的装置,也可用其他仪器进行,不论采用哪种办法,其共同点是保证气体的定质量和等温变化的条件。 

 对于等温线,主要讲清它表明了一定质量气体在等温条件下压强和体积的变化关系,不要求利用图线进行定量计算。教材中给出了不同温度下的几条等温线,主要是说明$pV=$恒量这个公式中的恒量跟温度的关系,得出对于一定质量的气体,“恒量随温度的增高而增大”的结论。显然这个恒量不是一个普适恒量。因此,要得出$p$、$V$、$T$三个状态参量之间的关系,还必须研究$P$、$T$或$V$、$T$之间的规律。关于这一点,教师可对比牛顿第二定律的研究过程:在$F$、$m$、$a$三个物理量中先研究加速度和力的关系,再研究加速度和质量的关系,从而确立了三个量之间的关系,引导学生注意这种控制条件的研究方法。

 教材选用了一个漏气的例题,这是一个有利于培养学生分析能力的题目,应该引导学生分析讨论,由于该题是第一
 个利用玻-马定律求解的题目,教师一定要明确指出,确定研究的对象是哪一定量的气体,对它进行变化前后的状态分析是正确解题的前提,教材也为此做了很好的示范,教师要充分注意这个问题,使学生一开始就养成良好的习惯、掌握正确的分析方法,由于本章习题涉及的物理量多,状态往往在两个或两个以上,做到这一点,尤为重要。

\subsubsection{气体的等容变化~~查理定律}

本节利用实验定性地归纳出在体积不变的条件下,一定质量的气体的压强随温度的升高而增大,随温度的降低而减小,然后给查理定律的表述和公式,给出气体等容变化的图线,最后明确查理定律的适用范围。

 教学中应对查理定律演示实验进行精心准备,使实验达到定量的效果,用实验所取得的数据作出如教材第47页的$p$-$t$图线,其延长线在实验误差范围内交于$-273^{\circ}{\rm C}$. 这样做的好处是:
 \begin{enumerate}
     \item 能突出物理规律的实验研究方法;
     \item 由实验图线获得$-273^{\circ}{\rm C}$的外推点,为下节研究热力学温标打下基础;
     \item 查理定律可以由图象总结出来。
 \end{enumerate}

 由于学生已学过斜截式的直线方程,比较容易总结出查理定律的教学表达式:由图线写出方程
\[p=kt+p_0\]
其中$p_0$为$0^{\circ}{\rm C}$时气体的压强,斜率$K=\frac{p_0}{273}$,
所以
\[p=\frac{p_0}{273}\cdot t+p_0\]
即
\[p=p_0\left(1+\frac{t}{273}\right)\]

为了能定量地进行实验,一要用气压表测定实验时的大气压;二要在$0^{\circ}{\rm C}$以下得出一组$p$、$t$数据,才能使$p$-$t$图象较准确地交$t$轴于${\rm -273^{\circ}C}$。为此,可用食盐加碎冰溶化获得零下20多摄氏度的低温;三要注意给出充分地时间让气体跟周围环境达到热平衡。

同讲解玻-马定律一样,教材在讲解查理定律后又强调了定律的适用范围是压强不太大,温度不太低,这点仍须向学生明确指明,物理定律都有它的适用范围。超出了这个范围,物理现象所遵从的规律将发生变化,不再是原来的规律了。对于气体定律的适用范围,学过理想气体的微观模型以后,就可以进一步理解它的道理了。

\subsubsection{热力学温标}

由查理定律外推而引入新的温标,绝不意味着查理定律的适用范围可以“外推”,在低温下,查理定律不适用了,这点要再次强调。

“外推点”定为热力学温标的零度,也绝不是一次实验的结果,而是对不同气体做多次实验所作出的图线都交$t$轴于${\rm -273^{\circ}C}$, 这才赋予它以新的物理意义,引出绝对零度,这一点在教学中要交代清楚。

在引出热力学温标后,要强调“就每一度的大小来说,热力学温度和摄氏温度是相同的”。这就是说,虽然热力学温度跟摄氏温度间的关系为$T=t+273.15$(K), 但就\textbf{同一温度变化}来看,利用热力学温标与利用摄氏温标所表示的数值是相同的,即$\Delta T=\Delta t$. 这一点在计算题中经常用到。

由实验总结出查理定律,又由查理定律“外推”到零压强而引入热力学温标、绝对零度,这一方面体现了实验的重要
性,另方面也体现了理论的指导作用,正是由于这种理论上的外推使人们获得了绝对零度的概念,导致了人类不断向绝对零度进军的历史进程,向学生介绍这点可以启迪思维,加深对物理研究方法的认识,结合讲解一些人类向绝对零度进军的情况,可以大大提高学生兴趣,开阔学生眼界。

\subsection{第二单元}
\subsubsection{理想气体的状态方程}

本节利用玻-马定律和查理定律推出一定质量气体的状态方程,提出了理想气体这一概念,说明了状态方程包含了玻-马定律和查理定律,并由它推出了另一个实验定律——盖·吕萨克定律。

对盖·吕萨克定律,教材中叙述得比较简单,教师可以补做演示,使学生的印象深刻些。

要再次强调状态方程的适用范围,强调理想气体是一种科学的抽象,一个理想的物理模型,要使学生明确地认识到,理想气体虽然并不存在,但它是实际气体在一定程度上的近似,在压强不太大,温度不太高的条件下可把实际气体作为理想气体来处理。

课后布置教材习题4, 根据玻-马定律和盖·吕萨克定律推出一定质量的理想气体的状态方程$pV/T=$恒量,向学生指明,只要初末状态确定,不论经过哪两个等值过程,都可以推出状态方程。这种练习,对于培养学生发散性思维是有帮助的。

\subsubsection{克拉珀龙方程}

克拉珀龙方程是任意质量的理想气体的状态方程,推导克拉珀龙方程时,要注意启发性,注意启
发的层次。教材是按下列层次安排的:
\begin{enumerate}
\item 从$pV/T=$恒量开始讨论,说明这个恒量与气体的质量和气体的种类有关;
\item 研究1摩尔任何气体的$p_0V_0/T_0$值,即摩尔气体恒量$R$, 从而得到1摩尔理想气体的状态方程$pV=RT$;
\item 由标准状态下,$n$摩尔气体占有体积为$nV_0$, 利用摩尔气体恒量$R$导出任意质量的理想气体的状态方程,即克拉珀龙方程$pV=\frac{m}{M}RT$.
\item 最后说明气体满足方程的条件。
\end{enumerate}

教材这样安排,层次是清楚的,启发性也较强。在讲解中,要指出摩尔气体恒量是热学中的一个重要常数,限于学生知识水平,其重要性不可能在课堂上分析,但值得注意的是,在讲解中不要把$R=p_0V_0/T_0$误认为是$R$的物理意义,上式仅仅是$R$的计算方法,由于在标准状态下,1摩尔的任何气体体积$V_0$都是22.4升,上式表明$R$是一个普适恒量,$R$的物理意义是:1摩尔理想气体在压强不变时,温度升高1开对外界所做的功。R的物理意义在讲授中不宜提出,更不能给出错误的解释,这一点应该注意。

在应用克拉珀龙方程解题前,还必须指出,由于$R=8.31$焦/摩·开$=0.082$标准大气压·升/摩·开,因此,$p$、$V$的单位必须与选用的$R$的单位相适应。

利用三个实验定律和两个状态方程解题时,要注意训练学生对气体变化前后的状态进行分析,教师要做出表率。

\subsection{第三单元}
本单元通过对气体分子运动的特点的分析,对气体实验定律进行微观解释,由于本单元的内容较为抽象,因此,一方面要严格控制,不要增加教材以外的内容;另一方面要使学生对分子间距离的大小和分子间的频繁碰撞在头脑中有深刻的、形象的认识。

讲解气体分子运动的特点要渗透一些统计的观点,要使学生知道,个别分子的运动具有偶然性,但大量分子在总体上都服从确定的规律,这种规律是统计性的,叫做统计规律。

为了讲好“气体实验定律的微观解释”,建议在讲课开始时,复习压强的概念、公式及弹性小球对墙壁的碰撞给与墙壁的冲量的计算,为正课的讲授扫清一些障碍。

压强的微观解释是本节的重点,在讲解这个问题时,要使学生体会到,虽然个别分子的运动服从力学规律,个别分子碰撞器壁时所产生的冲力也要用力学规律来计算,但涉及大量分子的集体行为却要用到统计观点,因此,不是一种纯粹的力学问题。

另外,应注意形成气体压强的微观图景,教材是以雨滴对伞的压强为例来说明的,为了加深理解,建议课堂上进行模拟气体压强的演示实验,以增加学生的实感。

\subsection{第四单元}
本单元的必学教材为第九节,该节主要建立理想气体的
微观模型,并指出理想气体的内能就是气体所有分子热运动的动能的总合,不存在分子势能,因此理想气体的内能仅与温度有关而与体积无关,建议在讲解本节内容前,认真复习第二章第一节“物体的内能”,让学生搞清分子动能、分子势能是由什么因素决定的,在此基础上,本节内容是容易理解的。

第十节是选学教材,对于基础较好的班可以选用,本节研究各种等值过程中理想气体的内能的变化,其研究的手段是利用热力学第一定律,该节的全部内容可列为下表。

\begin{table}[htp]
\caption{理想气体等值过程的讨论}
\begin{tabular}{cp{.32\textwidth}p{.18\textwidth}c}
\hline
过程名称& 各等值过程的物理现象 & 热力学第一定律的形式&$p$-$V$图\\
    \hline
等温过程&内能不变$\Delta E=0$,\par 吸热则体积膨胀,对外做功;放热则体积缩小,外界对气体做功
&$W+Q=0$&\raisebox{-.65\height}{\tikz[>=latex, yshift=-3cm]
 {\draw[<->](0,2.5)node[right]{$p$}--(0,0)--(2.5,0)node[above]{$V$};\draw(.34,2) [bend left=-35] to node[right]{等温线} (2,.34);
 }        }
 \\
\hline
等容过程&气体不做功$W=0$,\par 吸热则内能增加;放热则内能减少
&$Q=\Delta E$&\raisebox{-.45\height}{\tikz[>=latex]
 {\draw[<->](0,2.5)node[right]{$p$}--(0,0)--(2.5,0)node[above]{$V$};\draw(1,0) to node[right]{等容线} (1,2.5);
 }        }
\\
\hline
等压过程&膨胀时,气体对外做功,其密度减小,为保持压强不变,必须吸热升温,故内能增加;压缩时,情况相反
&$W+Q=\Delta E$&\raisebox{-.85\height}{\tikz[>=latex]
 {\draw[<->](0,2.5)node[right]{$p$}--(0,0)--(2.5,0)node[above]{$V$};\draw(0,1) to node[above]{等压线} (2.5,1);
 }        }
\\
\hline
绝热过程&气体与外界无热交换,$Q=0$.
体积膨胀,对外做功,内能减少;体积压缩,外界对气体做功,内能增加
&$W=\Delta E$&\raisebox{-.85\height}{\tikz[>=latex]
 {\draw[<->](0,2.5)node[right]{$p$}--(0,0)--(2.5,0)node[above]{$V$};\draw(.34,2) [bend left=-35] to node[right]{绝热线} (2,.34);
 }        }
\\
\hline
\end{tabular}
\end{table}

\section{实验指导}
\subsection{演示实验}
\subsubsection{气体压强的模拟实验}

仪器可选用“气体分子运动模拟实验器”。 该仪器如图3.1所示.
\begin{figure}[htp]\centering
    \begin{minipage}[t]{0.3\textwidth}
    \centering
    \includegraphics[scale=.8]{fig/3-1.png}
    \caption{}
    \end{minipage}
    \begin{minipage}[t]{0.58\textwidth}
    \centering
    \includegraphics[scale=.8]{fig/3-2.png}
    \caption{}
    \end{minipage}
    \end{figure}

其中的钢球模拟气体分子。在玻璃筒内放入50—80个钢球,接通电源,选择适中的电压,由于活塞的振动引起钢球的无规则运动,不断地冲击浮动板使之升起到一定的高度,这就定性地模拟了大量气体分子频繁地碰撞器壁产生持续的均匀的压力,形成对器壁的压强,如图3.2所示.

\begin{figure}[htp]
    \centering
\includegraphics[scale=.8]{fig/3-3.png}
    \caption{}
\end{figure}

如果没有上述仪器,可用托盘天平来演示,如图3.3所
示,将托盘天平中的一盘反扣在支架上,调平后,在正放的托盘里放入适当质量的砝码,用小钢球(如自行车轴承里的钢球,或用豌豆)自反扣托盘的上方徐徐释放,不断地冲击托盘,使天平达到平衡,这就定性地说明,大量气体分子对器壁的频繁碰撞,产生持续的,均匀的压力,形成对器壁的压强。

\subsubsection{温度不变时气体的压强跟体积的关系}

做教材图3.5的实验时,应注意:
\begin{enumerate}
\item 灌注水银时,要利用小漏斗和细铁丝将水银慢慢注入,尽量避免水银中带有空气泡;
\item $A$、$B$管与橡皮管连接处要用细铁丝绑紧,防止由于装入水银而脱落;$A$管的阀门$a$要涂上凡士林,以保证良好的密闭性;
\item 实验中,$A$、$B$管始终保持竖直,$B$管的升降移动要缓慢,保证有足够的时间使$A$管中的气体跟周围空气进行热交换,以便保持等温条件。
\item 水银不浸润玻璃,液面成凸弯月面,读数要读水银凸面顶端的刻度数。
\end{enumerate}

\subsubsection{体积不变时气体的压强跟温度的关系}

做教材图3.9的实验时,其注意事项与玻-马定律注意事项类似,不再赘述,建议在$B$管上套一个橡皮圈,标记水银面的位置,即密闭气体的定体积标志,每次改变容器的水温,让其有足够的时间使烧瓶内气体与水达到热平衡,再调整$A$管的高度,使$B$管中水银面恰好在橡皮圈标志处,这样做可以使学生更清楚地观察到实验在等容条件下进行。

\subsubsection{压强不变时气体的体积跟温度的关系}

定性演示可利用气体在大气压下热胀冷缩实验进行,如图3.4所示的简单仪器即可做这个实验,用双手握住烧瓶,瓶内气体膨胀,水滴右移;用冷湿毛巾包住烧瓶,瓶内气体收缩,水滴左移。为增加可见度,水滴应染色。
\begin{figure}[htp]
    \centering
\includegraphics[scale=.8]{fig/3-4.png}
    \caption{}
\end{figure}

定量演示可利用针筒,其装置同本章的学生实验。针筒竖放固定,内筒上不加任何压载物,当筒内密闭了一定质量的气体后,若忽略内筒所受的重力,则气体的压强恒为当地的大气压值,这就保证了压强不变的条件。将针筒逐次放入温度不同的水中,每次都要保持针筒竖放、水淹没筒内气体并保证必要的热平衡所需的时间,就可以从温度计和针筒上的刻度读出各次实验的$t$、$V$值,将$t$值换为$T$值,可证明在实验误差范围内
\[\frac{V_1}{T_1}=\frac{V_2}{T_2}=\frac{V_3}{T_3}=\cdots\]

上式即盖·吕萨克定律。

\subsection{学生实验}
本章有两个学生实验:“验证玻意耳-马略特定律”和“验证气体状态方程”。 教材上选用针简作为这两个学生实验的主要仪器,优点是在学生实验中取掉了水银,用针筒完全可以较准确地做好这两个实验,下面提出一些注意事项供参考。

\begin{enumerate}
    \item 针筒一般选用50毫升规格为宜,针筒的内外筒的配合应该是“密而不紧”。把注射器小孔打开,竖直拉起内筒,针筒能够在自重的作用下慢慢下落,这就达到“不紧”的要求。将内筒拉起一定高度后,密封针筒小孔,这时再用力将内筒拉出或压入一段距离,放手后,内筒能恢复到原来的位置,这就说明针筒的密闭性是较好的,达到了“密”的要求,不能满足上述条件的针筒是不能选用的。
\item 消除内、外筒间的梗阻,做上项选择实验或正式实验前,都务须用细软的、不留毛屑的布将内外筒擦净,以免尘粒毛屑之类的小物梗阻内筒的运动、
\item 提高针筒的密闭性是两个实验成败的关键,为此,可以在擦净内外筒后,在其间注进少许缝纫机油或纯净的轻机油,千万不要用水代替机油,水蒸气会破坏针筒内气体的定质量条件。
\end{enumerate}

\section{习题解答}

\subsection{练习一}
\begin{enumerate}
	\item 什么叫气体的压强?举出气体对器壁有压力作用的几个实例.

\begin{solution}
    气体作用在器壁单位面积上的压力叫做气体的压强。充氢气的气球,氢气对气球内壁有压力作用;内燃机气缸中的燃气对活塞和气缸壁有压力作用,正是这个压力,推动活塞做功;高压锅中的蒸汽,对锅内壁有压力作用,这个压力随着蒸汽的增多而增大,到一定限度,推开高压阀,使高压蒸汽喷出。
\end{solution}
	\item 大气压为750毫米汞柱时,等于多少帕?

\begin{solution}
\[750{\rm mmHg}=750\x \frac{1.013\x 10^5}{760}=1.00\x 10^6{\rm Pa}\]
\end{solution}
		\item 在教材图3.2中,水银柱的长度为19厘米,大气压为
	760毫米汞柱,玻璃管开口向上竖直放置时,被封闭的气体的压强等于多少毫米汞柱?开口向下竖直放置时,等于多少毫米汞柱?


\begin{solution}
玻璃管开口向上竖直放置时,气体压强为
\[p_1=p_0+p_k=760+190=950{\rm mmHg}\]
玻璃管开口向下竖直放置时,气体压强为
\[p_2=p_0-p_k=760-190=570{\rm mmHg}\]
\end{solution}

	\item 图3.5是测量气体压强的水银压强计,两端开口的U形管内装入水银,$A$管跟容器连接.已知大气压$p_0$和两管中水银面的高度差,就可以知道容器中气体的压强.大气压为$1.013\times 10^5{\rm Pa}$,图甲和图乙中的$h$都是10厘米,分别求出这两种情形中气体的压强是多少帕.
    \begin{figure}[htp]\centering
        \includegraphics[scale=.8]{fig/3-5.png}
        \caption{水银压强计}
    \end{figure}	

\begin{solution}
    依题意,甲图中容器的压强应该等于大气压强和$h$高水银柱产生的压强之和,乙图中容器的压强为上两项之差。即:
\[\begin{split}
    p_{\text{甲}}=p_0+\rho gh
    &=(1.013\x10^5+13.6\x10^3\x9.8\x0.10)\\
    &=1.14\x 10^5{\rm Pa}\\
    p_{\text{乙}}=p_0-\rho gh
    &=(1.013\x10^5-13.6\x10^3\x9.8\x0.10)\\
    &=0.88\x 10^5{\rm Pa}\\
\end{split}\]    
\end{solution}
	\item 在图3.4所示的几种情形中,被封闭的气体$A$的压强分别是多少帕?大气压为$1.013\times 10^5{\rm Pa}$.
\begin{figure}[htp]\centering
	\includegraphics[scale=.8]{fig/3-6.png}
	\caption{}
\end{figure}	

\begin{solution}
\[\begin{split}
    p_{1}=p_0-\rho gh_1
    &=(1.013\x10^5-13.6\x10^3\x9.8\x0.20)\\
    &=7.4\x 10^4{\rm Pa}\\
    p_{2}=p_0+\rho_{\text{水}} gh_2
    &=(1.013\x10^5+1.0\x10^3\x9.8\x 20)\\
    &=2.97\x 10^5{\rm Pa}\\
    p_{3}=p_0+\rho_{\text{水}} gh_3
    &=[1.013\x10^5+1.0\x10^3\x9.8\x(0.85-0.10)]\\
    &=1.87\x 10^5{\rm Pa}\\
\end{split}\]  
\end{solution}
	\item 举出气体状态发生改变的几个实例.

\begin{solution}
 柴油机汽缸中的气体,在第二冲程时被急骤压缩,气体的三个状态量都发生变化;氢气球破裂,氢气溢出,其体积和压强发生明显变化,由于体积膨胀,温度也要发生变化。   
\end{solution}
\end{enumerate}

\subsection{练习二}
\begin{enumerate}
		\item 把打气筒的出口堵住,往下压打气筒的活塞,会感到越往下压越费劲,怎样解释这个现象?

\begin{solution}
打气筒的出口被堵住后,简内密闭了一定质量的空气,往下压打气筒的活塞,可将其近似地看成温度不发生改变(实际上温度有一点变化),越往下压活塞,气体的体积越小。根据玻意耳-马略特定律,这时气体的压强增大,所以越往下压感到越费劲。
\end{solution}
	\item 某个容器的容积是5升,里面所装气体的压强是10标准大气压,如果温度保持不变,把容器的开关打开以后,这些气体会有多大体积?容器里剩下的气体是原来的百分之几?设外界压强为1标准大气压.
	
    \begin{solution}
根据题意,$V_1=5$升,$p_1=10$标准大气压,当容器的开关打开以后,气体的体积为$V_2$,$p_2=1$标准大气压,根据玻意耳-马略特定律:$p_1V_1=p_2V_2$

$\therefore\quad V_2=\dfrac{p_1V_1}{p_2}=\dfrac{10\x 5}{1}=50{\rm L}$

容器内剩下的气体是$V_1=5$升,气体的总体积为$V_2=
50$升,故容器里剩下的气体是原来的
\[\frac{5}{50}=\frac{1}{10}=10\%\]       
    \end{solution}
	\item 在上题里,打开容器的开关以后,气体的密度怎样改变?设上题里容器里剩下的气体的密度是$\rho_2$,原来容器里气体的密度是$\rho_1$,那么,密度之比$\rho_2/\rho_1$是多大?
	
\begin{solution}
 由于原容器内的压强是10标准大气压,外界压强是1标准大气压,打开容器后,气体体积增大,因气体质量不变,故密度变小。
 \[\frac{\rho_2}{\rho_1}=\frac{m/V_2}{m/V_1}=\frac{V_1}{V_2}=\frac{5}{50}=\frac{1}{10}\]   
\end{solution}
	\item 在密闭圆筒的中央有一个活塞(图3.7),活塞两边
	封闭着两部分气体,它们的压强都是750毫米汞柱,现在用
	力把活塞向右移动,使活塞右边气体的体积为原来的一半,那么活塞两边气体的压强差是多大?假定气体的温度不变.
\begin{figure}[htp]\centering
	\begin{tikzpicture}[>=latex]
\draw [pattern=north east lines](2.4,0) rectangle (2.6,2);	
  \pgfsetlinewidth{4pt}
\pgfsetinnerlinewidth{3pt}
\draw  (0,0) rectangle (5,2);
\end{tikzpicture}
\caption{}
\end{figure}

\begin{solution}
设活塞移动前后左边的压强和体积分别为$p_1$、$p'_1$、$V_1$、$V'_1$, 活塞右边的压强和体积分别为$p_2$、$p'_2$、$V_2$、$V'_2$,其中$p_1=p_2=750$毫米汞柱,$V_1=V_2$,
\[V'_1=\frac{3}{2}V_1,\qquad V'_2=\frac{1}{2}V_2\]

根据玻意耳-马略特定律,对左边的气体有
    \[p_1V_1=p'_1V'_1\]
$\therefore\quad p'_1=\dfrac{V_1}{V'_1}p_1=\dfrac{V_1}{\frac{3}{2}V_1}p_1=\dfrac{2}{3}p_1=\dfrac{2}{3}\x 750=500{\rm mmHg}$
    
    对右边的气体有
    \[p_2V2=p'_2V'_2\]
    
$\therefore\quad p'_2=\dfrac{V_2}{V'_2}p_2=\dfrac{V_2}{\frac{1}{2}V_2}p_2=2p_2=2\x 750=1500{\rm mmHg}$
    
    活塞两边的压强差
\[\Delta p=p'_2-p'_1=1500-500=1000{\rm mmHg}\]
    
\end{solution}

	\item 在教材图3.2中,水银柱的长度为19厘米,大气压为760毫米汞柱,玻璃管是粗细均匀的,玻璃管开口向上竖直放置时,被封闭的气体柱长15厘米,当开口向竖直放置时,被封闭的气体柱的长度是多少?
	
\begin{solution}  
设玻璃管开口向上竖直放置时,气体的压强和体积分别为$p_1$、$V_1$, 玻璃管开口向下竖直放置时,气体的压强和体积分别是$p_2$、$V_2$, 玻璃管的横截面积为$S$.
    则:
\[\begin{split}
      p_1&=760+190=950{\rm mmHg}\\
    V_1&=\ell_1S=15{\rm cm}\x S\\
    p_2&=760-190=570{\rm mmHg}\\
    V_2&=\ell_2S  
\end{split}\]    
    
玻璃管在倒置过程中,气体的温度可以认为不变。根据玻意耳-马略特定律:
\[p_1V_1=p_2V_2\]
$\therefore\quad V_2=\dfrac{p_1}{p_2}V_1=\dfrac{p_1}{p_2}\x 15{\rm cm}\x S$

即:$\ell_2S=\dfrac{p_1}{p_2}\x 15{\rm cm}\x S$

$\therefore\quad \ell_2=\frac{950}{570}\x 15=25{\rm cm}$

当开口向竖直放置时,被封闭的气体柱的长度是25厘米。
\end{solution}  
	\item 在下端封闭的竖直玻璃管里有一段4厘米长的水银柱,水银柱下面封闭着6${\rm cm^3}$的空气,玻璃管的横截面积是0.1${\rm cm^2}$.如果再向管里装入27.2克水银,那么,封闭在水银
 柱下面的空气柱有多高?设大气压为760毫米汞柱.

\begin{solution}
当只有4厘米长的水银柱封闭玻璃管时,气体的压强和体积分别为$p_1$、$V_1$,则
\[\begin{split}
    p_1&=76+4=80{\rm cmHg}\\
    V_1&=6{\rm cm^3}
\end{split}\]

再向管里装入27.2克水银后,由于管的横截面积$S=0.1{\rm cm^2}$
,故27.2克水银在管内的高度
\[h'=\frac{m}{\rho S}=\frac{27.2}{13.6\x 0.1}=20{\rm cm}\]
若这时气体的压强和体积为$p_2$、$V_2$,
则$$p_2=76+4+20=100{\rm cmHg}$$
可认为气体的温度不变,由玻意耳-马略特定律:
\[p_1V_1=p_2V_2\]
$\therefore\quad 
V_2=\dfrac{p_1}{p_2}V_1=\dfrac{80}{100}\x 6=4.8{\rm cm^3}$

此时空气柱的长度为$H$,
\[H=\frac{V_2}{S}=\frac{4.8}{0.1}=48{\rm cm^3}\]
\end{solution}
	\item 一个足球的容积是2.5升.用打气筒给这个足球打气,每打一次就把1标准大气压的空气打进去125${\rm cm^3}$,如果足球在打气前内部没有空气,打了40次以后,足球内部空气的压强有多大?假定空气的温度不变.
	
\begin{solution}
将打入足球的那部分空气作为研究对象,打入足球前空气的压强和体积为$p_1$、$V_1$, 打入足球后空气的压强和体积为$p_2$、$V_2$.
则$p_1=1$标准大气压,$V_1=40\x125{\rm cm^3}=5{\rm L}$.$V_2=2.5$升.

由题设条件假定空气的温度不变,根据玻意耳-马略特定律
\[p_1V_1=p_2V_2\]
得:
\[p_2=\frac{V_1}{V_2}p_1=\frac{5\x 1}{2.5}\text{标准大气压}=2\text{标准大气压}\]
\end{solution}
\end{enumerate}

\subsection{练习三}

\begin{enumerate}
	\item 炎热的夏天,打足了气的自行车胎在日光曝晒下有时会胀破,解释这个现象.

\begin{solution}
  自行车胎在炎热的夏天被日光曝晒,车胎里的气体的温度上升。车胎的容积在打足了气后已不能再增大,根据查理定律,气体体积一定时,一定质量的气体的压强跟热力学温度成正比。所以气体的压强将增大。当压强达到车胎所能承受的最大压强时,温度再高车胎就会被胀破。
\end{solution}
\item 乒乓球挤瘪后,放在热水里泡一会,会重新鼓起来.解释这个现象.

\begin{solution}
    挤瘪的乒乓球在热水里浸泡时,乒乓球内的空气温度升高,在一个极短的时间内可以认为体积不变,根据查理定律,球内空气的压强增大,当球内压强到达一定值时,乒乓球就会鼓起一点,温度再升高,根据相同的道理,乒乓球会再鼓起一点……, 一会儿,乒乓球会重新鼓起来。
\end{solution}
\item 一定质量的氢气在0$^\circ$C时的压强是700毫来汞柱,它在30$^\circ$C时的压强是多大?压强为650毫米汞柱时它的温度
是多少摄氏度?保持氢的体积不变.

\begin{solution}
设氢气在$0^{\circ}{\rm C}$时的压强和热力学温度分别为$p_1$、$T_1$, 在$30^{\circ}{\rm C}$时的压强和体积分别为$p_2$、$T_2$, 压强650毫米汞柱时的温度为$T_3$, 则:
$p_1=$700毫米汞柱,$T_1=273$开;$P_2$是未知数,$T_2=30+273=303$开;$p_3=650$毫米汞柱,$T_3$是未知数,由于体积不变,根据查理定律
\[\frac{p_1}{p_2}=\frac{T_1}{T_2}\]
得:
\[p_2=\frac{T_2p_1}{T_1}=\frac{303\x 700}{273}=777{\rm mmHg}\]
又$\dfrac{p_1}{p_3}=\dfrac{T_1}{T_3}$
得:
\[T_3=\frac{p_3T_1}{p_1}=\frac{650\x 273}{700}=254{\rm K}=-19^{\circ}{\rm C}\]
\end{solution}
\item 一定质量的某种气体,在20$^\circ$C时的压强是$1.0\times 10^5$帕,如果保持它的体积不变,温度升高到50$^\circ$C时,它的压强是多大?温度降低到$-7^\circ$C时,它的压强又是多大?

\begin{solution}
 这一定质量的某种气体的三个状态,其压强和温度分别为
\begin{center}
\begin{tabular}{p{.3\textwidth}p{.3\textwidth}}
    $p_1=1.0\x10^5$帕 & $T_1=20^{\circ}{\rm C}=293{\rm K}$\\
    $p_2$是未知数& $T_2=50^{\circ}{\rm C}=323{\rm K}$\\
    $p_3$是未知数& $T_3=-7^{\circ}{\rm C}=266{\rm K}$\\
\end{tabular}
\end{center}
由于气体体积不变,根据查理定律$\dfrac{p_1}{p_2}=\dfrac{T_1}{T_2}$,得:
\[p_2=\frac{T_2p_1}{T_1}=\frac{323\x 1.0\x 10^5}{293}=1.1\x 10^5{\rm Pa}\]
又$\dfrac{p_1}{p_3}=\dfrac{T_1}{T_3}$,
\[p_3=\frac{T_3p_1}{T_1}=\frac{266\x 1.0\x 10^5}{293}=9.1\x 10^4{\rm Pa}\]
\end{solution}
\item 盛有氧气的钢筒,在室内(室温是17$^\circ$C)测得筒内气体的压强是$9.31\times 10^8$帕,当钢筒搬到温度是$-$13$^\circ$C的工地时,筒内气体的压强变为$8.15\times 10^8$帕.钢筒是不是漏了气?为什么?

\begin{solution}
先假定钢筒没有漏气,则气体满足查理定律$\dfrac{p_1}{p_2}=\dfrac{T_1}{T_2}$, 得
\[p_2=\frac{T_2p_1}{T_1}\]
其中$p_1=9.31\x10^6$帕,$T_1=17+273=290$开,$T_2=-13+273=260$开.
将数据代入上式得
\[p_2=\frac{260\x 9.31\x 10^6}{290}=8.35\x 10^6{\rm Pa}\]
而钢筒内实际气压为$p'_2=8.15\x 10^6$Pa,$p'_2<p_2$,故钢筒已漏气。
\end{solution}
\item 装在容器中的气体,体积为4升,压强为$2.0\times 10^5$帕,温度为300开,先让气体发生等容变化,压强增大为原来的2倍,然后让气体发生等温变化,压强又降低到原来的数值.求气体在末状态时的体积和温度.


\begin{solution}
根据题意,装在容器中的气体共经历两个变化过程,一个是等容变化过程,一个是等温变化过程,有三个状态,其参量分别为
\begin{center}
\begin{tabular}{p{.3\textwidth}p{.2\textwidth}p{.3\textwidth}}
$p_1=2.0\x10^5$帕&$V_1=4$升& $T_1=300$开\\
$p_2=4.0\x10^5$帕& $V_2=4$升&$T_2$为未知\\
$p_3=2.0\x10^5$帕& $V_3$为未知& $T_3=T_2$为未知  \\
\end{tabular}
\end{center}

根据查理定律
\[\frac{p_1}{p_2}=\frac{T_1}{T_2}\]
得
\[T_2=\frac{p_2}{p_1}T_1=\frac{4.0\x10^5}{2.0\x10^5}\x 300=600{\rm K}\]
根据玻意耳-马略特定律$p_2V_2=p_3V_3$, 得
\[V_3=\frac{p_2V_3}{p_3}=\frac{4.0\x10^5\x 4}{2.0\x10^5}=8{\rm L}\]
而$T_3=T_2=600$K。

气体末状态的体积是8升,温度是600开.
\end{solution}
\end{enumerate}

\subsection{练习四}

\begin{enumerate}
	\item 对一定质量的气体来说,能否做到:
	\begin{enumerate}
	\item	保持压强和温度不变而改变它的体积?
	\item	保持温度和体积不变而改变它的压强?
	\item	保持体积和压强不变而改变它的温度?	 
	\end{enumerate}

\begin{solution}
对于一定质量的气体,在状态变化过程中,它的三个状态参量始终应满足$\dfrac{pV}{T}=$恒量。如果任意改变了一个状态参量,而其余的两个参量一个也不发生相应的变化,那么$\dfrac{pV}{T}$不再等于原先的结果,这是违反气态方程的。因而题中所说的只改变一个状态参量的三种情况都是不可能做到的。
\end{solution}
\item  对一定质量的气体来说,能否做到:
\begin{enumerate}
\item 保持压强不变,同时升高温度并减小体积?
\item 保持温度不变,同时增加体积并减小压强?
\item 保持体积不变,同时增加压强并降低温度?
\end{enumerate}

\begin{solution}
\begin{enumerate}
    \item 不能做到。根据盖·吕萨克定律,在保持压强不变的情况下,一定质量的气体的体积应随温度的升高而增大;
    \item 能做到。根据玻意耳-马略特定律,在保持温度不变的条件下,
    一定质量的气体体积增大,压强必减小;
    \item 不能做到,根据查理定律,在体积不变的情况下,一定质量的气体的压强应随温度的降低而减小。
\end{enumerate}
\end{solution}
\item  一定质量的空气,27$^\circ$C时的体积为$1.0\times 10^{-2}{\rm m}^3$.计算在压强不变的情况下,温度升高到100$^\circ$C时的体积.

\begin{solution}
气体在状态1时:$T_1=273+27=300$开,$V_1=1.0\x10^{-2}{\rm m^3}$

气体在状态2时:$T_2=273+100=373$开,$V_2$为待求体积。

由盖·吕萨克定律$\dfrac{V_1}{V_2}=\dfrac{T_1}{T_2}$得到
\[V_2=\frac{T_2}{T_1}V_1=\frac{373}{300}\x 1.0\x 10^{-2}{\rm m^3}=1.2\x 10^{-2}{\rm m^3}\]
\end{solution}
\item  某种柴油机的气缸容积为$0.83\times 10^{-3}{\rm m^3}$.压缩前其中空气的温度为47$^\circ$C,压强为$0.8\times 10^5$帕,在压缩冲程,活塞把空气压缩到原体积的1/17,压强增大到$40\times 10^5$帕.求这时空气的温度.

\begin{solution}
    气体压缩前为初状态:
\[V_1=0.83\x10^{-3}{\rm m^3},\quad T_1=273+47=320{\rm K},\quad p_1=0.8\x10^5{\rm Pa}\]
压缩后为末状态:
\[V_2=\frac{1}{17}V_1,\qquad  p_2=40\x10^5{\rm Pa}\]
$T_2$为待求温度,由气态方程$\dfrac{p_1V_1}{T_1}=\dfrac{p_2V_2}{T_2}$得到
\[T_1=\frac{40\x10^5\x320}{0.8\x10^5\x17}=941{\rm K}=668^{\circ}{\rm C}\]
\end{solution}
\item  在容积为25升的容器中,盛有温度为37$^\circ$C、压强为62标准大气压的氧气.求氧气在标准状态(0$^\circ$C,1标准大气压)下的体积,从化学课中学过,在标准状态下,1摩的任何气体的体积都是22.4升.你能不能由此求得容器中氧气的摩尔数并进而求得氧气的质量?怎样求?

\begin{solution}
    
\end{solution}
\item  一个瓶子里装有某种气体,瓶上有一个小孔跟外面大气相通.原来瓶里气体的温度为15$^\circ$C.如果把它加热到207$^\circ$C,瓶里保留的气体的质量是原来质量的几分之几?

\begin{solution}
    
\end{solution}
\item  贮气筒内装有压缩气体,温度是27$^\circ$C,压强是$40\times 10^5$帕,如果从筒内放出一半质量的气体,并使筒内剩余的气体的温度降到12$^\circ$C,这些剩余气体的压强是多大?


\begin{solution}
    
\end{solution}
\end{enumerate}





\chapter{固体和液体的性质}


\section{教学要求}


这一章是介绍性的。关于固体,介绍晶体及其微观结构;关于液体,主要介绍液体的表面现象。这部分内容可以把它看成是第一章“分子运动论基础”的具体应用,本章讲授的知识在生产和生活中有一定的实用意义,对于学习分子物理学的研究方法、培养观察能力和抽象思维能力也是有好处的。

全章共六节,可以划分为两个单元,第一节和第二节为第一单元,讲述固体的性质,第三节到第六节为第二单元,讲述液体的性质。

关于晶体的各向异性,教材只就导热性进行了说明,教材中虽然提到了晶体的力学性质、电学性质、光学性质也是各向异性的,但不要求具体讲解,只要学生知道就可以了。

关于空间点阵,只要求学生知道组成晶体的物质微粒在空间中依照一定的规则排成整齐的行列就可以了,至于用点阵概念来解释晶体的规则外形和各向异性,只要求学生有个大体的了解,不要求做进一步的解释。

为了说明液体表面的收缩趋势,教材提出了表面层的概
念。这里不要求讲解表面层的形成,只要学生知道表面层里的分子比液体内部稀疏即可,教材给出了表面张力系数的概念,是为了说明不同液体表面的收缩趋势并不相同,不要求用表面张力系数进行计算。

为了说明浸润和不浸润,教材提出了附着层的概念,这里也不要求讲解附着层的形成,只要求学生知道附着层里分子比液体内部密或稀是形成浸润或不浸润的原因就可以了。至于为什么固体分子的吸引比较弱(或强),附着层里分子就比较稀(或密),则不要求加以解释。

毛细现象的教学,要求学生能够利用浸润现象和表面张力的知识对毛细现象的产生做出解释,但不要求对液柱上升或下降的高度进行定量的计算。

这一章虽然是介绍性的,但在教学中也不能忽视,讲述这一章,可以扩展知识面,使学生多知道一些物理现象,有利于学好物理,通过这一章的学习,要注意使学生体会到,分子运动论不但能从微观上研究气体,而且能够研究固体和液体;人们的研究从宏观领域进入微观领域,使人类对自然界的认识前进了一大步。

这一章的教学要求是:
\begin{enumerate}
\item 知道固体可分为晶体和非晶体两类,知道晶体有单晶体和多晶体;了解晶体和非晶体在外形上和物理性质上的区别。
\item 知道什么叫空间点阵,了解晶体外形的规则性和各向异性可用晶体物质微粒的规则排列来说明。
\item 了解液体的微观结构情况.
\item 了解液体的表面张力,知道它是怎样产生的;了解浸
润和不浸润,知道它们是怎样产生的;了解毛细现象,知道它是怎样产生的。
\end{enumerate}


\section{教学建议}
\subsection{第一单元}
\subsubsection{晶体的特征}

教材对这部分知识要求较低,只简要讲
述晶体的特征,晶体与非晶体的区别,以及固体的微观结构,而不涉及固体的其他性质,要注意把握教学的深广度,不宜讲得太多、太深。

\subsubsection{固体材料}
 为了扩展学生的眼界,引起他们对研究固体性质的兴趣和求知欲望,在本单元教学开始,可以简述一下有关固体材料和固体物理的发展概况。

 长期以来,在科研、生产和日常生活中都广泛利用固体材料,因此,固体在材料科学技术中占有特殊重要的地位,按指定的性能设计新的固体材料已成为固体物理的重要研究内容,近30年来,固体物理已发展成为一门独立的综合性学科,是物理学的重要分支,由于各种尖端技术对固体材料提出多种多样的要求,因此固体物理同现代尖端技术的发展有非常密切的联系,例如,原子能技术需要耐放射性辐射的固体材料:高速飞行、火箭导弹需要耐高温、耐辐射、强度高、质地轻的合金材料;电子技术需要半导体元件、集成电路、铁氧体元件等新型器件;激光技术需要小巧价廉的半导体激光器等等,在
 科研和生产需要的推动下,新理论和新技术相互促进、相辅相成,使固体物理在近代原子理论的基础上得到巨大的发展。

 \subsubsection{利用实验、挂图、模型讲述晶体、非晶体}
 
 本单元知识
 比较抽象,教学中要加强实验演示,运用挂图、模型,使学生获得鲜明、具体的印象,有条件的学校可以把演示实验让学生自己动手进行实验观察,教学效果会更好些,例如,在讲晶体和非晶体时,可在教师指导下一边让学生分组用放大镜观察食盐、砂糖晶体的外形,按教材介绍的方法比较云母晶体和非晶体玻璃的导热性;一边阅读教材,由学生自己概括晶体和非晶体在外形上和物理性质上的区别,最后,教师作适当补充、归纳,在讲空间点阵一节内容时,最好先让学生观察点阵模型,再对照模型和教材图4-5讲明空间点阵的概念;解释晶体外形的规则性和物理性质各向异性;讲明同一物质的微粒能够形成不同的空间点阵,但应当提醒学生:点阵模型并不代表晶体的真实情况,它只是组成晶体的物质微粒有规则排列的示意图。

\subsubsection{教学中要注意的问题}
\begin{enumerate}
\item 教材中指出晶体具有天然的规则外形,教学时应该强调,这种规则的外形不是人工造成的,是晶体本身具有的.
\item 讲晶体的各向异性时要指出:天然的规则的几何形状和各向异性都是晶体区别于非晶体的性质,但晶体有时不完整、它的外形也容易受到破坏使外部特征不显著,所以各向异性就成为判断晶体和非晶体的一个基本特征.
\item 根据初中学过的知识,可以告诉学生,晶体的另一个基本特征是具有一定的熔点,使他们对晶体的特征有较全面的认识,同时指出,这是从宏观上区分晶体和非晶体的重要依据。当物体已被研成粉末,不能从外形及各向异性来鉴别它是否晶体时,只有根据有无一定熔点才能作出准确判断.
\item 多晶体虽无规则的几何形状,物理性质又各向同性,但是组成多晶体的晶粒却有规则的几何形状,物理性质呈各向异性。这是多晶体和非晶体在内部结构上的区别。多晶体与非晶体的区别还在于多晶体跟单晶体一样具有一定熔点,非晶体则没有.
\item 可按乙种本上册255页的叙述,介绍多形性的概念。有些物质能够生成几种不同的晶体,就是因为它能够生成几种形状不同的空间点阵,这种性质叫多形性。换句话讲,同一化学成分的物质可以结晶成结构不同的晶体的性质叫多形性,还可以简要介绍同一种化学成分的物质,既能以晶体形式存在,又能以非晶体形式存在,例如,天然水晶是晶体,而熔融的水晶(即石英玻璃)就是非晶体,以便扩大学生眼界,避免出现片面的、绝对化的错误认识。
\end{enumerate}



\subsection{第二单元}
这一单元讲述液体的性质。主要是从分子运动论的观点来剖析液体的微观结构,研究液体和气体接触时形成的表面层以及液体和固体接触时形成的附着层发生的现象,然后讨论在表面层与附着层的共同作用下产生的毛细现象。教学中应突出教材的这一思路,使学生能对这部分知识从整体上有所认识,本单元的特点是偏重实验现象的分析,因此,需要加强演示实验,注意培养学生的观察能力和抽象概括能力。

\subsubsection{液体的微观结构}

建议在教学中注意以下问题.

重视新旧知识的联系,先引导学生回顾第一章学过的分子运动论的基本观点和对固、液、气体的分子特点的分析,这对学习新知识是十分有益的。

液体的性质介于固体和气体之间,但更接近固体.因此,应对比固体结构讲解液体微观结构的大致图景,并注意用液体微观结构的特点去认识液体的性质:具有一定体积,不易压缩,各向同性,流动性等。

课本76页上提到“液体中的分子也是密集在一起的”,“液体分子距离很小”。为使学生能比较具体认识这个问题,可举如下事实:一摩尔水有$6.022\x10^{23}$个水分子,在常温下的体积为$18{\rm cm^3}$,每立方厘米中有分子
$\dfrac{6.022\x10^{23}}{18}=3.3\x10^{22}$
个。设两相邻水分子间的距离为$r'$, 可得
\[r'=\sqrt[3]{\frac{1}{3.3\x10^{22}}}=3.1\x 10^{-8}{\rm cm}\]
和分子直径同数量级,所以说液体分子几乎一个挨一个地密集在一起,彼此间距离很小。

要指出非晶体的微观结构跟液体非常类似(见参考资料),非晶体的分子也是处于杂乱无章的结构状态,跟液体中小区域杂乱无章的结构状态非常类同,所以非晶体可以看作是粘滞性极大的液体。可以提示学生,严格说来,只有晶体才是真正的固体。

液晶在近10多年来获得了广泛的应用,因此教材把它安排为阅读材料,如教学时间比较充裕,最好先在课堂上简要介绍一下,再要求学生阅读,激发他们学习的兴趣。

\subsubsection{液体的表面现象}

关于液体表面的收缩趋势,教材是通过实验现象的观察得出结论的,因此,必须做好演示实验(见实验指导),并注意突出每个实验在认识液体表面性质中的作用,回忆荷叶上的小水滴,草叶上的露珠及演示玻璃板上的小水银滴,说明液体与空气接触的表面有收缩的趋势;教材图4.8的实验,一方面说明不仅液体与空气接触的表面有收缩的趋势,而且液体与液体接触的表面也存在收缩的趋势,另一方面对照玻璃板上的大液滴呈扁平形,说明消除重力后这个表面有收缩到最小面积的趋势;教材图4.9和图4.10的实验则是对液面具有收缩趋势的验证.

对教材图4.10所示的实验,要从棉线圈的周长一定所围面积最大的几何图形是圆来说明只有棉线圈被张紧成圆形时肥皂膜面积才会最小。

关于表面层,根据教材的要求,可不讲解它是怎样形成的,只要求学生知道表面层里分子比液体内部稀疏就可以了。这个结论是讲解表面张力产生原因的依据,让学生明确认识表面层分子聚集的特点是进行表面张力教学的关键。

从复习分子间相互作用力的特点,说明表面层内分子间相互作用表现为引力,即表面张力,是不太困难的。在教学中需要注意的是:
\begin{enumerate}
    \item 由于张力概念比较难懂,学生过去又没有学过,在这里只要求学生能接受教材上关于表面张力的讲法即可,不必单独讲解“张力”的概念,以免增加学生负担.
    \item 表面张力不是指个别分子间的相互引力,而是表面层中大量分子间引力的宏观表现,凡液体与气体接触的表面都存在表面张力,例如对一层液膜来说,无论其厚薄程度如何都存在两个表面,每个表面都存在表面张力.
    \item 应当提醒学生,教材图4.11中液面的分界线MN是任意选取的,但不论分界线怎样选,分别作用在分界线两侧液面的表面张力f和f总是一对作用力、反作用力。
\end{enumerate}

至于表面张力的方向,即表面张力跟液面相切、跟液面分界线垂直,只要求学生知道,不要求作进一步解释,但对液面是曲面的情况,可作出表面张力的示意图帮助学生理解。

教材不要求用表面张力系数进行定量计算,不过应当指出,两部分液面间的表面张力大小不仅与液体的表面张力系数有关,还与分界线长度有关,其数值等于表面张力系数与分界线长度的乘积,避免学生把表面张力和表面张力系数混为一谈。关于各种液体的表面张力系数不同,可以安排一些有趣的演示(或布置为课外实验,见实验指导)来加以说明,但不宜从理论上多加解释。

\subsubsection{浸润和不浸润} 

这部分内容研究液体与固体接触时发生的现象,是与液体的表面现象平行的知识,教材是从附着层内分子作用力的不同特征来解释浸润和不浸润现象的。因此,在教学中要抓好两个环节:做好演示实验;让学生懂得怎样以附着层内分子聚集的特殊情况(即附着层内分子比液体内部稀或密)为依据,解释浸润和不浸润现象。

做好演示实验的关键在于提高器材的洁净程度,实验中所用的水银和水要洁净,玻璃和玻璃管应洗涤洁净,锌板需用稀硫酸清洗并擦拭洁净,为增大可见度,可用幻灯投影。

讲解浸润和不浸润现象时,应对什么是浸润和不浸润现象作适当概括,便于学生掌握,应当强调,同一种液体对某些
固体浸润,对另一些固体则不浸润,因此讲某种液体是浸润的或不浸润的液体,一定要指明相应的固体,对此可演示水能浸润玻璃但不能浸润石蜡,水银不能浸润玻璃但能浸润锌的实验,加深学生的认识。

讲解附着层的性质时,可与液体表面层的性质进行对比,要引导学生掌握分析问题的思路。附着层的液体分子除受液体内部分子的吸引外,还受到固体分子的吸引。固体分子吸引作用的强弱决定着附着层内液体分子比液体内部分子稀疏或稠密,也就决定着附着层内分子间表现为引力或斥力,进而决定着附着层是收缩趋势还是扩展趋势。附着层有收缩趋势,就表现为液体不浸润固体,附着层有扩展趋势,就表现为液体浸润固体。

有了前面的基础,讲解弯月面的形成就水到渠成,所以建议把弯月面放在最后讲述。

\subsubsection{毛细现象}

毛细现象的发生是附着层的收缩力或相斥力与表面张力共同作用的结果,由于学生已学过上述两方面知识,建议通过指导学生讨论和阅读的方式进行教学,这样做,有利于培养学生的能力。

首先演示教材图4.14和图4.15的两个实验,可用幻灯投影增大实验的能见度,让学生观察并思考从中能得出什么结论?再由教师概括说明什么是毛细现象,强调毛细管内径越小,毛细现象越明显,即浸润液体在管内上升高度越大,不浸润液体在管内下降越厉害。

为什么会出现上述现象呢?可引导学生思考、讨论以下问题.
\begin{enumerate}
    \item 浸润液体在毛细管内,液面为何成凹弯月面?是什么原因使液体上升?上升到什么时候为止?
\item 为什么不浸润液体在毛细管内会下降?下降到什么时候为止?
\end{enumerate}
然后让学生阅读教材,检验自己的认识。教师再根据具体情况适当归纳,着重分析第二个问题。

教材列举了不少毛细现象的实际应用,但没有作详细的说明,可选择其中一些例子让学生讨论,巩固所学知识。例如,
\begin{enumerate}
    \item 为什么棉灯芯能吸油?是否可用丝线做灯芯?    \item 为什么可以用粉笔来吸干纸上的墨水迹?
\end{enumerate}

本单元教材的题目较少,可以联系实际补充一些解释现象的题目,例如:
\begin{enumerate}
\item 人造卫星中有一个盛液体的容器,如果液体浸润器壁,会发生什么现象:如果液体不浸润器壁,又将出现什么现象?(浸润液体沿器壁上升并沿器壁流散;不浸润液体则呈球形。)
\item 教材练习二第1题的缝衣针如果事先用肥皂水洗得很干净并用清水冲洗擦干,将会出现什么现象?为什么?(针下沉,水能浸润缝衣针,针周围的水分子与针上水分子连成一片,使针处在水面以下。)
\item 为什么带有油脂的抹布不能把湿了的桌面擦干t(水对带有油脂的抹布不浸润,抹布就不能带走桌面上的水。)
\item 车轮在潮湿的土地上滚过以后,车辙中就会渗出水来,这是什么缘故?(车轮压紧地面不仅使土壤里原有的毛细管变得更细,而且还会增添新的毛细管,使毛细现象更加显著,地下水更容易上升到地面上来。)
\end{enumerate}

\section{实验指导}
\subsection{演示实验}
\subsubsection{晶体外形的演示}

演示晶体具有天然规则的几何形状,最好事先培养一些体积大点的晶体,便于观察。其中硫酸铜、重铬酸钾、明矾的大晶体比较容易获得,现以硫酸铜为例,介绍制作方法。

在一个大烧杯中盛大半杯热水,将研成粉末的硫酸铜逐浙倒入水中搅拌,直到饱和,再用细线拴一小粒硫酸铜晶体,悬挂在硫酸铜溶液中作为结晶核,让晶体缓慢生长,几天后可获得一块较大的硫酸铜晶体,若溶液过饱和、室温太高、结晶生长太快,得到晶体的质量就不够好,此方法也适用于培养重铬酸钾和明矾大晶体。

也可用厚一点的纸糊一个一端封底的圆筒,圆筒比小电珠略大、略长一点,在筒外包两层纱布。用线把圆筒开口的一端拴住,悬挂起来,使筒的4/5左右浸入硫酸铜溶液中作为结晶核,从而得到一中空的硫酸铜晶体,然后在晶体内部安上一个小电珠,演示时使小电珠发光,从里面照亮硫酸铜晶体,效果会更加明显。

\subsubsection{晶体传热的各向异性}

比较云母片和玻璃板的传热性,说明晶体导热各向异性而非晶体导热各向同性。
\begin{enumerate}
    \item 云母片要薄,最好撕裂成单层薄片使用.
    \item 用四氯化碳溶液将石蜡溶解,用脱脂棉将溶液均匀涂在云母片上,放在阳光下晒干备用,为使石蜡层成为均匀薄层,也可先在云母片上放一小块石蜡,把云母片放在火焰上方烘一下,使石蜡熔成膜状薄层。
    \item 取一根长约40—50毫米的铁钉,用尖嘴钳夹住放在酒精灯上将钉的尖端烧红,然后用尖端去接触云母片涂蜡层的反面,石蜡熔成椭圆形,应当注意,根据云母晶体的特点,该椭圆的离心率不会太大,要让学生留意观察。
\end{enumerate}

用玻璃板作同样实验,所用玻璃板也要薄。还可以用小号培养皿的底部代替玻璃板。

\subsubsection{液体表面的收缩趋势}

橄榄油呈球形的实验.先将橄榄油滴在酒精里,这时油滴沉底。然后在酒精中缓缓注入清水,直到看见油成球形悬浮在酒精与水的混合液中为止。

也可用苯胺代替橄榄油,滴在食盐水中,逐渐调整食盐水浓度,使苯胺滴成球形悬浮在食盐水中。

还可把内盛有色水的玻璃管的尖嘴浸入重机油中,注意控制上部皮管的开关,使尖嘴处缓缓地形成带色的表面呈球形的小水滴,如图4.1所示.
\begin{figure}[htp]\centering
    \begin{minipage}[t]{0.48\textwidth}
    \centering
\includegraphics[scale=.6]{fig/4-1.png}
    \caption{}
    \end{minipage}
    \begin{minipage}[t]{0.48\textwidth}
    \centering
\includegraphics[scale=.6]{fig/4-2.png}
    \caption{}
    \end{minipage}
    \end{figure}

液膜收缩实验.通常用肥皂水做教材上图4.9和图4.10所示的实验,虽然肥皂水的表面张力系数($40\x10^{-3}{\rm N/s}$)比纯水的表面张力系数($73\x10^{-3}{\rm N/s}$)小得多,但肥皂水的粘滞性强,在重力作用下比纯水流下较缓,所以较容易在表面层之间保持液膜,为了使实验效果较好,肥皂水浓度要
合适,并可滴入几滴甘油增强其粘滞性,铁丝环要平整、光洁,棉线宜细软,事先应将棉线浸湿。

为便于学生看清液膜收缩的过程,可将金属丝做成图4.2所示的U形框架,两侧框杆长约3厘米,相距约4厘米,框架尖端可弯成小钩,将细棉线的一端系在一个小钩上,手执细线的自由端,在另一小钩附近绕一周后拉紧,把框架浸没在肥皂水内,轻轻提起,使框架上形成矩形液膜,并保持液膜处于竖直面内,且线的固定端在上方。然后,放松被拉紧的自由端,可明显看到液膜向U型框架接手柄的一边收缩,把棉线拉成弧形。



\subsubsection{验证表面张力的存在}

用金属丝做成一金属环,把它悬挂在倔强系数很小的纤细弹簧下方,记下弹簧指针的示数$F_1$, 把金属环浸没在肥皂水中,再把它缓缓拉出,使环四周与液面间出现液膜,如图4.3所示,记下这时弹簧指针的示数$F_2$. 比较这两个示数可以看出$F_2>F_1$, 证明液膜上存在表面张力.

\subsubsection{毛细现象}

在缺少口径不一的毛细管时,也可用以下器材代替,将两块平板玻璃一边靠拢,另一边用木棍隔开,再用橡皮筋把两边箍牢。放入有颜色的水中后,可看到在玻璃板夹缝间的水升起,缝越窄的地方,水升得越高。调节木棍在玻璃板间的位置,减小板的间距,板间各处的水会上升得更高(图4.4)。

\begin{figure}[htp]\centering
    \begin{minipage}[t]{0.48\textwidth}
    \centering
\includegraphics[scale=.5]{fig/4-3.png}
    \caption{}
    \end{minipage}
    \begin{minipage}[t]{0.48\textwidth}
    \centering
\includegraphics[scale=.6]{fig/4-4.png}
    \caption{}
    \end{minipage}
    \end{figure}

\subsection{课外实验活动}
\subsubsection{观察各种液体的表面张力系数不同}

在桌上铺一张白纸,在纸上再放一块面积稍大一些的干净平板玻璃,在玻璃板上倒少许红色水形成一层薄薄的水层,再用一团脱脂棉花球沾少量酒精,用手指捏一下棉球,滴几滴酒精到玻璃板上,就可发现玻璃板上的红色水带着酒精向四周移动,出现一块“干”的区域,这是由于滴酒精后,在水和酒精相邻的分界线上有表面张力相互作用,水的表面张力系数比酒精大得多,所以红色水就把酒精拉向四周,留下一块“干”的区域。

让两根火柴相隔一段距离平行地浮在水面上,待稳定后,在火柴间轻轻滴入一两滴肥皂水(或用一小块肥皂轻轻接触火柴之间的水面),便可看到两根火柴立即分别向两边跑开。如果在两火柴之间不滴入肥皂水而改滴入一两滴糖水,则两根火柴立即靠拢,这是因为肥皂水或糖水的表面张力系数分别小于或大于水的表面张力系数。

在水面上放几小块樟脑,就会发现这些樟脑块在水面上进行复杂的、紊乱的运动。这是由于樟脑的形状是不规则的,它们在各方面溶解的程度不同,导致樟脑四周的樟脑水溶液浓度不同,表面张力系数也不相同,结果就引起樟脑块作这种奇怪的运动。


\section{习题解答}
\subsection{练习一}
\begin{enumerate}
    \item 把玻璃管的裂断口放在火焰上烧熔,它的尖端就变圆.这是什么缘故?

\begin{solution}
    玻璃烧熔后,它的表面层在表面张力作用下收缩到最小表面积,从而使玻璃管尖端变圆。    
\end{solution}
\item 在处于失重状态的宇宙飞船中,一大滴水银会呈什么形状?

\begin{solution}
    在处于失重状态的宇宙飞船中,由于消除了重力的影响,一大滴水银的表面将收缩到最小面积——球面,水银滴成为球形。    
\end{solution}
\item 把熔化的铅一滴一滴地滴入水中,凝固后可以得到
球形的小铅弹.为什么?

\begin{solution}
    熔化的铅滴在自由下落中,由于失重,液滴表面将在表面张力作用下收缩为球形,进入水后迅速冷却,就会凝固成球形小铅弹。
\end{solution}
\end{enumerate}

\subsection{练习二}

\begin{enumerate}
   \item 把一根缝衣针小心地放在水面上,针可以把水面压弯而不沉没(试试看).解释这个现象.

   \begin{solution}
 由于针的表面有油脂,不能被水浸润,当针放在水面上把水面压弯时,仍处在水的表面层之上,这是因为水面的表面张力,要使被压弯的水面收缩,使它恢复原来的水平液面,从而对针产生一个向上托的力,这个力与水对针的向上的压力一起跟针所受的重力平衡,使针不致下沉。  
   \end{solution}
   \item 布的雨伞虽然纱线间有可以看得出来的孔隙,却仍然不漏雨水.解释这个现象.

   \begin{solution}
    因为水能浸润纱线,在纱线孔隙中形成向下弯曲的水面,弯曲水面的表面张力,承受住孔隙内水所受的重力,使得雨水不致漏下。
   \end{solution}
\end{enumerate}



\subsection{练习三}
\begin{enumerate}
   \item 要想把凝在衣料上面的蜡或油脂去掉,只要把两层吸墨纸分别放在这部分衣料的上面和下面,然后用熨斗来熨就可以了,为什么这样做可以去掉衣料上的蜡或油脂?

   \begin{solution}
    放在衣料上、下的吸墨纸内有许多细小的孔道起着毛细管的作用。当蜡或油脂受热熔解成液体后,由于毛细现象,它们就会被吸墨纸吸掉。
   \end{solution}
\item 建筑楼房的时候,在砌砖的地基上铺一层油毡防潮层.如果不铺这层油毡,楼房就容易受潮,为什么?

\begin{solution}
    因为土壤和砖块里有许多细小孔道,地下水可以通
    过这些毛细管上升到楼房里,使楼房受潮,铺一层油毡后,由于油毡上涂有煤焦油,堵塞了纸料上的孔隙,不会发生毛细现象,从而阻断了地下水上升到楼房的通道,防止房屋受潮。
\end{solution}
\end{enumerate}

\section{参考资料}
\subsection{固体的微观结构}

晶体和非晶体宏观性质上的不同,是由于它们微观结构上的差别造成的,经过伦琴射线的分析,发现组成晶体的原子(或分子)的排列十分整齐,极有规则,结构是周期性重复的。即在长距离范围内作有秩序排列,称为长程有序,而非晶体内部的原子(或分子)的排列不很整齐或很不整齐,没有明显的规则性,不具有周期性重复的结构,因此不是长程有序的。但在一个原子间距的范围内,原子排列还是有一定规则、一定结构的,这称为短程有序,可见,非晶体是无序结构中存在着有序成分,长程无序而短程有序的固体,正因为这样,所以非晶体在宏观上没有一定天然的规则形状;在较大范围内各方向的原子排列情况平均说来是一样的,非品体在宏观上具有各向同性;另外,由于内部各处原子的结合情况不同,有松有紧,不能在同一温度都达到能挣脱相互束缚的程度,因而非晶体不具有确定的熔点。

从有序性来看,非晶体的固态与液体相似,组成液体的物质微粒在短暂时间内,在一个微小的区域中也可保持有一定规则的排列,也属于短程有序,但是它们也有重要区别,液体的这种短程有序受热运动的影响不断被破坏和改变,而非
晶体中的短程有序性却保持相对的稳定。

\subsection{晶体的类型}

晶体中粒子(分子、原子和离子)之间存在着相互作用力,这种力叫结合力。正是这种力使粒子有规则地聚集在一起形成空间点阵。结合力又称化学键,它是决定晶体基本性质的根本原因。根据化学键的不同,可把晶体分为四类。
\begin{enumerate}
\item 离子晶体,晶体由离子组成,靠正、负离子之间的静电力,即离子键把这些离子结合起来,由离子键的作用组成的晶体,称为离子晶体,由于离子键作用强,因此离子晶体具有高熔点、低挥发性、可压缩性小。食盐晶体就是最典型、最简单的离子晶体,半导体材料中的硫化镉、硫化铝也是重要的离子晶体。
\item 原子晶体.晶体由中性原子组成,靠共有电子产生的结合力,即共价键把这些中性原子结合起来,这种晶体称为原子晶体,共价键的作用很强,所以原子晶体硬度大、熔点高、导电性弱、挥发性低,金刚石就是典型的原子晶体,半导体中的重要材料硅、锗、碲也都是原子晶体。
\item 分子晶体.晶体由分子组成,靠分子间的相互作用产生结合力,即范德瓦尔斯键把这些分子结合起来,由范德瓦尔斯键的作用而组成的晶体称为分子晶体,范德瓦尔斯键的作用很弱,所以分子晶体硬度小、熔点低、易于挥发,碘和低温下的情性气体以及许多有机化合物构成的晶体,都是分子晶体。
\item 金属晶体.晶体内金属正离子排列在点阵的结点上,自由电子为全体离子所具有,自由地在正离子形成的点阵内
运动,自由电子的总体称为电子云。正离子与电子云之间的作用力使各粒子结合在一起,这种结合力称为金属键。由金属键作用所组成的晶体叫金属品体,简称金属,金属键的作用可以很强,因此金属可以具有高熔点、高硬度和低挥发性,金属内由于存在自由电子,因而具有良好的导电性和导热性。
\end{enumerate}

对于大多数晶体来说,晶体的结合往往是几种键共同作用的结果.例如石墨有三种键共同作用,如教材图4.6所示,每一层中每一个碳原子有三个电子以共价键与周围三个碳原子相互作用,另一电子为层中所有碳原子共有,而以金属键与层中所有碳原子相互作用,层与层之间则以范德瓦尔斯键相互作用,因此,共价键、金属键、分子键这三种键在石墨晶体中都起作用,致使石墨的性质与同为碳原子但只由共价键组成的金刚石的性质有很大的不同。

\subsubsection{表面层的收缩趋势}
我们知道,分子间的作用力只有在距离很小时才起作用,这个距离约为$10^{-8}$厘米,由于液体分子在各个方向上都是均匀分布的,可以把分子作用的范围认为是一个半径约为$10^{-8}$厘米的球,这个球叫分子作用球,我们把跟气体交界并且厚度等于分子作用球半径的一薄层液体,叫做液体表面层。
\begin{figure}[htp]
    \centering
\includegraphics[scale=.6]{fig/4-5.png}
    \caption{}
\end{figure}

如图4.5所示,设液体内有一分子$A$, 由于该分子的分子作用球处在液体内部,而球内的分子均匀分布,对这个分子来讲是球对称的,因此球内所有分子对$A$的作用力的合力为
零。设表面层里有一分子$B$, 以$B$为中心的分子作用球一部分在液面外,缺少了这部分分子的引力,就使作用在$B$上的全部分子引力的合力$f$垂直于液面指向液内(因分子间表现为斥力时分子间的距离极小,$B$所受分子斥力仍是球对称的,可不予考虑,只考虑分子间表现为引力的情况即可),如果要把液体分子从内部移到表面层去,就必须克服力$f$的作用做功,使分子的势能增加,故分子在表面层比在液体内具有较大的势能。表面层中全体分子因上述原因所具有的势能的总和,叫做表面能,液体的表面越大,具有较大势能的分子数也越多,表面能也越大。力学原理指出,物体系在稳定平衡时,它的势能必须是一切可能值中的最小值,因此,液体在稳定平衡时,它的表面能应当最小,即液体表面应尽可能收缩,直到表面积最小。

\subsubsection{弯曲液面下的压强}

由于表面张力作用,弯曲液面有一个特征,就是在它的下面产生附加压强。
\begin{figure}[htp]
    \centering
\includegraphics[scale=.6]{fig/4-6.png}
    \caption{}
\end{figure}

我们以液面是半径为$R$的球体的一部分为例,求出附加压强的数值,取球面的一部分$\Delta S$(图4.6),作用在这部分边线上的表面张力处处都是与这球面相切的,设表面张力系数
为$\alpha$, 通过边线上每一微段$\Delta\ell$作用在液块上的表面张力$\Delta f=\alpha\cdot \Delta\ell$. 这个力的一个分力垂直于球面半径$OC$, 整个边线各个微段的表面张力的这个分力对$OC$具有轴对称性,合力为零。$\Delta f$的另一个分力平行于$OC$, 如球面是凸的,则中心$C$在液体之内,力$\Delta f_1$压缩$\Delta S$下面的液体形成一正压力;如球面是凹的,则中心$C$在液体之外,力$\Delta f_1$对液体形成一负压力,由图可见,$\Delta f_1=\Delta f\sin\phi=\alpha\Delta \ell\sin\phi$, 因此平行于半径$OC$施加于整个球面部分$\Delta S$上的力
\[f_1=\sum \Delta f_1=\alpha\sin\phi\cdot \sum \Delta \ell\]
但$\sum \Delta \ell$是包围球面$\Delta S$的边线长度,以$r$表示其半径,则$\sum \Delta \ell=2\pi r$, 由此可得 $f_1=\alpha\cdot 2\pi r\sin\phi$.

又由图可知$\sin\phi=r/R$,则:
\[f_1=\frac{\alpha\cdot 2\pi r^2}{R}\]
所以附加压强
\[p=\frac{f_1}{\pi r^2}=\frac{2\alpha}{R}\]

显然,该附加压强和表面张力系数成正比,和表面的半径$R$成反比。表面弯曲越厉害,其半径$R$越小,因而附加压强$p$就越大。对凸液面,液面内压强大于液面外压强,附加压强$p$向下;对凹液面,液面内压强小于液面外压强,附加压强$p$向上。

对一个球形液膜(如肥皂泡)来说,它的液膜有两个表面,且一凸一凹,由于膜很薄,其半径可认为相等,都是$R$(图4.7). 在液膜内取一点$B$, 用$p_A$、$p_B$、$p_C$分别表示$A$、$B$、$C$三点的压强,故
\[p_B-p_A=\frac{2\alpha}{r},\qquad p_C-p_B=\frac{2\alpha}{r}\]
两式相加得
\[p_C-p_A=\frac{4\alpha}{r}\]
由于这种附加压强的存在,使肥皂泡内的压强比泡外的大气压强大。泡的半径越小,泡内外的压强差越大。如在一玻璃管的两端吹制半径不同的肥皂泡$A$和$B$(图4.8), 让两泡相通。因小泡内的压强大于大泡内的压强,空气就会从小泡不断流向大泡,使小泡不断变小,大泡不断增大。泡内压强的这种关系在吹制玻璃器皿时要用到,开始时吹气的压强要比较大,吹大后就要减小压强。

\begin{figure}[htp]\centering
    \begin{minipage}[t]{0.48\textwidth}
    \centering
    \includegraphics[scale=.8]{fig/4-7.png}
    \caption{}
    \end{minipage}
    \begin{minipage}[t]{0.48\textwidth}
    \centering
    \includegraphics[scale=.8]{fig/4-8.png}
    \caption{}
    \end{minipage}
    \end{figure}


\subsubsection{液体在毛细管中上升高度的计算}


如图4.9所示,设毛细管内径为$r$, 液体表面张力系数为$\alpha$, 液体密度为$\rho$, 液柱上升高度为$h$. 由于毛细管内径很小,可以粗略认为沿着管壁的液面是竖直的,那么,竖直向上作用到液柱上的表面张力就是$2\pi r\alpha$, 因竖直向下作用到液柱上的重力为$\pi r^2h\rho g$, 当液体保持平衡时
\[2\pi r\alpha=\pi r^2h\rho g\]
由此可以得到
\[h=\frac{2\alpha}{r\rho g}\]

\begin{figure}[htp]\centering
    \begin{minipage}[t]{0.48\textwidth}
    \centering
    \includegraphics[scale=.8]{fig/4-9.png}
    \caption{}
    \end{minipage}
    \begin{minipage}[t]{0.48\textwidth}
    \centering
    \includegraphics[scale=.8]{fig/4-10.png}
    \caption{}
    \end{minipage}
    \end{figure}

所以,在毛细管内浸润液体上升的高度跟表面张力系数成正比,跟毛细管内部的半径和液体的密度成反比。

用类似的方法也可推导出不浸润液体在毛细管中下降的
高度同样是$h=\dfrac{2\alpha}{r\rho g}$.

更为一般的方法是根据弯月面内外压强差来推导,如图4.10所示.当毛细管刚插入液体时,由于液体浸润管壁,所以沿管壁上升,使液面成凹弯月面,这将使液面下方的压强小于液面上方的大气压强,而在管外平液面下与$B$点同高的$C$点的压强仍等于液面上方大气压。根据流体静力学的基本原理,$B$、$C$压强应相等。因此液体不能平衡而要在管内上升,一直升到$B$、$C$两点压强相等为止。

设管截面为圆形,对凹弯月面可近似看作半径为$R$的球面,弯月面下$A$点的压强比大气压$p_0$小$2\alpha/R$, 即
\[p_A=p_0-\frac{2\alpha}{R}\]

因$B$点与$A$点高度差为$h$, 所以
\[p_B=p_A+\rho gh=p_0-\frac{2\alpha}{R}+\rho gh\]
又因$p_B=p_C=p_0$, 所以有
\[p_0-\frac{2\alpha}{R}+\rho gh=p_0\]
则:
\[\frac{2\alpha}{R}=\rho gh,\qquad h=\frac{2\alpha}{\rho gR}\]
由图可知:
\[R=\frac{r}{\cos\theta}\]
其中$\theta$为接触角,即固、液接触处,液体表面的切线与固体表面的切线在液体内部所成的角度。将本式代入上式,可得
\[h=\frac{2\alpha\cos\theta}{\rho gr}\]


\chapter{物态变化}
\section{教学要求}
本章讲解的知识,是对初中学过的物态变化知识的扩展和加深,与生产、科研和日常生活实际有着密切的联系。

教材对熔解、凝固、汽化等现象,和熔解热、汽化热等概念,从分子运动论的观点和能量的观点作了定性的解释,是为了使学生对现象和概念理解得深刻一些;这也有利于培养学生的抽象思维能力。这一章的现象多、概念多,讲述时除了加强实验、增加感性知识外,还要多举生活中的实例,帮助学生理解。还要注意弄清楚某些具有共同特点的概念间的区别(如熔解热与汽化热)。

本章的教学可分为三个单元:第一单元,包括第一、二节,讲熔解和凝固的知识,第二单元,包括第三节到第七节,讲解蒸发、沸腾及饱和汽和气体的液化的知识,第三单元,包括第八、九节,讲解空气的湿度和露点。

下面对这一章的教学内容作一些具体说明。

晶体和非晶体在熔解时的区别初中已学过,只要求作简单的复习,课本对此从分子结构不同所作的解释,以及对熔
解热用分子运动论和能量转化的知识所作的说明,可以使学生从理论上加深对这些现象的认识,同时使他们了解理论的应用,从而巩固理论知识。

饱和汽的知识是理解液体的沸腾、气体的液化、空气的湿度、露点等许多现象的基础,需要重点讲解,它又是学生新接触的理论性较强的内容,也是教学的难点,理解饱和汽的知识,关键在于讲好动态平衡的概念。

液体沸腾的条件和沸点跟压强的关系,课本是通过实验观察得出结论的。至于为什么当饱和汽压等于外界压强时液体才沸腾,这个问题在中学阶段很难讲清楚,教材中也没有进一步分析这个问题,希望在教学中注意掌握,不做过高的要求。

课本对汽化热也用分子运动论和能量转化的知识作了说明,液体汽化时用于克服外界压强所做的功不能忽略,沸点变化时汽化热差别较大,这些与熔解热的不同点,要向学生指明。

空气的湿度和露点,在生产和生活实际中常常用到,课本讲解这些知识,就是为了给学生理解一些实际现象打下一个初步基础。

根据以上分析,本章的教学要求是:
\begin{enumerate}
\item 了解晶体和非晶体熔解和凝固时的不同,掌握熔解热的概念。
\item 知道蒸发和沸腾的区别,了解沸腾的条件及沸点跟压强的关系,掌握汽化热的概念。
\item 知道什么是饱和汽,理解动态平衡的概念.了解饱和汽的压强(或密度)跟温度有关,跟体积无关,知道气体液化的方法,了解临界温度的概念。
\item 理解绝对湿度、相对湿度和露点等概念.
\end{enumerate}

\section{教学建议}
\subsection{第一单元}
\subsubsection{熔解和凝固} 

可以先让同学观察萘的熔解和凝固曲线(图5.1和图5.2),思考以下几个问题:
\begin{enumerate}
\item 两个图各是什么曲线?    \item 曲线的哪一段表示物质的固态、液态或固液共存的状态?    \item 曲线的哪一段是熔解和凝固的过程?    \item 熔点和凝固点各是多少度?它们有何关系?
\end{enumerate}
这样就可以把熔解和凝固的过程以及熔点的概念搞清楚了。
\begin{figure}[htp]\centering
    \begin{minipage}[t]{0.31\textwidth}
    \centering
    %\includegraphics[scale=.8]{fig/5-1.png}
    \caption{}
    \end{minipage}
    \begin{minipage}[t]{0.31\textwidth}
    \centering
    %\includegraphics[scale=.8]{fig/5-2.png}
    \caption{}
    \end{minipage}
    \begin{minipage}[t]{0.31\textwidth}
        \centering
        %\includegraphics[scale=.8]{fig/5-3.png}
        \caption{}
        \end{minipage}
    \end{figure}

同样,通过对比萘的熔解曲线和松香的熔解曲线(图5.1和图5.3),可以使学生认识晶体和非晶体熔解过程的不同特点。

对晶体的熔解和凝固过程,运用分子运动论从晶体和液体的微观结构进行解释时,可以先结合熔解曲线对熔解过程作出示范分析,对凝固过程则可让学生自己进行分析说明。

要在讲清多数物质熔解时体积变大、少数物质熔解时体积变小的前提下,讲解物质的熔点与压强的关系,可以先引导学生看课本上各种物质的熔点表,并使他们懂得这些熔点是在一标准大气压下测得的。进而讨论熔点跟压强的关系,还须指出,熔解时物体的体积无论是增大还是减小,都需要吸收热量,用来克服分子力做功,破坏空间点阵,掺杂对熔点的影响主要通过冰、盐混合物熔点降低等实例说明。

\subsubsection{熔解热}

要从熔解过程中能量转换的角度说明熔解时吸收热量,用于克服分子力做功,破坏晶体的空间点阵,增加物体的分子势能。同时,还要说明,由于不同晶体的空间点阵不同,单位质量不同的物质熔解时吸收的热量也不同,为表征物质的这一性质,引入熔解热的概念。可以把熔解热跟比热、热量等概念从物理意义、定义、单位等方面加以对
比区别,加深对熔解热的认识,还应指出,熔解热与凝固时单位质量的物质放出的热量相等,是能量守恒定律的必然结果。

课本上的测定熔解热的方法,需要用到热量的计算、热平衡方程和量热器的构造等初中已学过的知识,需要适当的复习。为了具体地掌握测量的方法,可以一边进行实验一边讲解,把测量熔解热的方法具体化,结合熔解热的测定,还要讲清分析和解决包含物态变化过程在内的热平衡问题的基本思路:首先要明确研究对象,知道参加热量交换的物体是哪些?它们各自的初状态和应该达到的末状态的物态和状态参量(温度、压强、体积)是怎样的?状态变化的过程经历了哪些阶段?中间状态的物态和状态参量又是怎样的?然后,分别列出所有放热物体放出热量和所有吸热物体吸收热量的数学表达式。最后,由热平衡方程列式求解,在不知道末状态物质处于什么物态时,需要先判断末状态是什么物态。可以熔点温度为界线,结合熔解热进行热量的估算,具体判定。比如水与冰混合,可以把水温降低到0℃放出热量跟冰上升到$0^{\circ}{\rm C}$和完全熔解所需的热量进行比较,来判断末状态是水、是冰还是冰水混合物。

\subsection{第二单元}
\subsubsection{蒸发}

蒸发现象、影响蒸发快慢的因素、蒸发的致冷作用等,在初中学过,可以通过举例复习、巩固这些知识,教材中增加的内容是:用分子运动论解释蒸发现象、介绍蒸发
致冷的应用实例,这些学生不难理解。可以让他们自己阅读,然后组织讨论,把学习搞得生动活泼一些。

\subsubsection{饱和汽与饱和汽压}

需要帮助学生复习一些已有的知识为学习新课做准备.可以通过提问,复习以下问题:
\begin{enumerate}
\item 描述气体状态的三个参量(温度、体积、压强)的意义及微观解释是怎样的?
\item 影响蒸发快慢的因素是什么,并用分子运动论加以说明。
\end{enumerate}

正确理解动态平衡的意义是掌握饱和汽概念的关键。可引导学生通过课本图5.2从微观的角度想象,密闭容器内分子逸出液面和返回液面的运动状况,着重让学生理解这两种运动宏观上达到平衡时(即汽体的密度不变,液体不再减少),并非意味着分子运动的停止,而是单位时间内逸出液面的分子数与回到液面内的分子数相等,即处于动态平衡。在此基础上讲清什么是饱和汽和未饱和汽,以及“在一定温度下,未饱和汽密度小于饱和汽密度”的道理。

动态平衡是有条件的,对此要讲解清楚。由于外界条件变化的影响,原来的动态平衡状态被破坏,经过一段时间才能达到新的平衡。比如,温度升高时,分子平均动能增大,单位时间内逸出液面的分子数增多。于是原有的动态平衡状况被破坏;空间汽分子密度逐渐增大,导致单位时间内返回液面的分子数增多,从而达到新的条件下的动态平衡。由此得出,饱和汽的密度随着温度的升高而增大,掌握了动态平衡状态变化的条件,才能更好地理解饱和汽的性质。

饱和汽压,从微观上讲仍然决定于分子密度和分子的平均速率。讲清这一点,能消除对饱和汽压的陌生感,帮助学生
认识饱和汽压的实质。要做好测定饱和汽压值的实验,这不仅能使学生对饱和汽压获得感性认识,也能使他们了解一种测量饱和汽压的方法。

研究饱和汽压跟温度的关系和跟体积的关系,要先从演示实验得出结果,再从理论上加以说明,使学生对结论理解得更确切、深刻些,课本上的实验也可以用其他实验代替,但对课本上的实验也要讲一讲。

在得出“饱和汽压随温度的升高而增大”的结论后,还要进一步讲解饱和气压随温度怎样变化,其关系如课本图5.4所示,可以跟理想气体的等容线作对比,一定质量的理想气体在体积不变时,其压强与绝对温度成正比,从微观上讲,是质量一定、体积一定,因而分子密度未变;但温度升高,分子平均动能变大,平均速率变大,导致压强增大,对饱和汽来说,温度升高时,不仅分子平均动能变大,分子平均速率变大;同时液面进入空间的分子数增多,分子的密度也增大,决定压强的这两个微观因素都变大,这就使饱和气压的值增大得更多。

在讲解饱和汽压跟体积变化的关系时,要注意强调前提条件:温度不变。再从饱和汽密度与温度的关系和动态平衡状态的条件,讲清体积变化时饱和汽压值不变的道理,由于温度不变,分子平均动能不变,分子平均速率不变,设若汽的体积变大,分子密度变小,小于这一温度下饱和汽应有的密度,成为未饱和汽,破坏了原有的动态平衡,液体就会继续蒸发,使汽恢复到原有的分子密度,成为饱和汽,建立起新的动态平衡。因此,决定压强的两个微观因素都没有变化,因而压强不变。这里还应结合温度一定,饱和汽体积变小的过程,讲
清一部分汽会凝结成液态达到减小体积后的新的动态平衡。

要注意归纳饱和汽的特点,说明它与理想气体不同,不能用理想气体定律来解决饱和汽的问题。

\subsubsection{沸腾}

用课本图5.6的实验现象,讲解什么是沸腾以及沸腾的过程,要把它与蒸发对比,让学生认识这两种汽化现象的区别,由观察现象让学生了解:沸腾前,液体内气泡在上升过程中体积变小;沸腾时,液体内气泡在上升过程中体积变大。对这一现象不要作过细分析,为了利于学生理解沸腾的条件,可以讲到这样的程度:沸腾前,容器内壁吸附的空气在器壁和器底形成小气泡,由于周围的水向气泡内蒸发,气泡中除了空气外还有水的饱和汽。当器底的气泡受到液体的浮力上升时,因为液体底部的温度高于上部,气泡在上升过程中温度下降,气泡内部的饱和汽不断液化,气泡的体积不断变小。当液体内各部分的温度都达到某一温度,气泡内的饱和汽压值等于外部大气压强时,气泡在上升过程中体积不再减小,而且由于周围液体不断向泡内蒸发,体积还会继续变大。到达液面时破裂放出饱和汽,液体就沸腾了,这样就可以使学生理解液体沸腾的条件是它的饱和汽压等于外界的压强。

关于沸点,除了引导学生理解课本对沸点的解释外,要强调跟外界压强相等的饱和汽压对应的温度,就是液体的沸点。这样才能帮助学生理解沸点随外界压强变化的关系,对此,除了道理上要讲清楚,还要用演示实验来验证,再用这一关系去说明一些现象,如离地面越高沸点越低,高压锅、蒸汽锅炉用增大压强的办法来提高沸点等。

\subsubsection{汽化热} 

先以蒸发时致冷、沸腾时液体继续吸热而温
度不变的现象,讲解液体汽化时要从周围吸收热量,还要让学生从能量方面了解,固体熔解时,由于体积变化较小,吸收的热量主要用来克服分子间的引力做功;而液体汽化时,由于体积明显增大,吸收的热量,一部分用来克服分子间的引力做功,另一部分用来克服外界压强做功。这是熔解过程和汽化过程的不同之处。

关于汽化热,可以对照熔解热,讲清它的定义,公式和单位,再要求学生阅读课本第101页的两个表格,思考这两个表格各自说明什么问题,怎样用分子运动论作出解释?然后通过讨论,了解不同物质在同一压强下汽化热不同,同种物质在不同温度下汽化热不同的道理。比如,不同物质分子间的距离和作用力的性质、大小不同,汽化时分子逸出液体所做的功不同,因而汽化热不同,再用分子运动论和能量转化的知识,说明在某一温度下气体凝结为液体放出的热量,与同一温度下汽化过程中吸收的热量是相等的。

关于汽化热的测定,要讲好或做好课本图5.8所示的实验。要使学生了解测定汽化热的装置及各部分的作用,实验的原理和方法,引导学生用所测出的各个物理量写出计算汽化热的表达式。当然,也可以把要讲解的内容以问题的形式提出来,让学生结合实物和实验过程展开讨论,得出结论。这样,既能加深对汽化热的理解,又能帮助学生学会应用汽化热等知识,分析解决包括汽化和液化过程在内的热平衡问题。分析解决这一问题的基本思路,可以参照解决熔解热问题的有关步骤,在分析参加热交换的各个物体的物态和温度时,要把汽化或凝结过程中汽液共存的状态考虑在内,在计算热量
时,不要漏掉计算汽化或凝结时吸收或放出的热量。

\subsubsection{气体的液化}

首先要引导学生回忆课本图5.5所示的实验:管内的饱和汽在温度不变、体积变小时,要凝为液滴,管内积存的液体增多,让学生体会到,气体液化的关键是把未饱和汽变为饱和汽。

关于把未饱和汽变为饱和汽的方法,可以先让学生思考两个问题:
\begin{enumerate}
    \item 相同温度下,饱和汽与未饱和汽的密度有何不同?
    \item 在温度不变时可否采取改变体积的办法来使未饱和汽转变为饱和汽?
\end{enumerate}
组织学生讨论得出:在温度不变时,减小未饱和汽的体积,使它的密度增大到这个温度下饱和汽的密度,未饱和汽就变成饱和汽了,再让学生思考以下问题:
\begin{enumerate}
\item 不同温度的饱和汽的密度相同吗?饱和汽的密度与温度有何关系?    \item 把高温下的未饱和汽的温度降低,能使它变为某个低温下的饱和汽吗?
\end{enumerate}
组织学生讨论得出:降低温度可以使未饱和汽变为饱和汽,在这过程中,容器内汽的密度没有发生变化,只是高温下的未饱和汽密度等于某个低温下的饱和汽密度,概括起来,把未饱和汽变为饱和汽的方法:一是减小体积(增大压强),二是降低温度。

使饱和汽凝结为液体,仍可采取减小体积和降低温度的方法,那么,只采用增加压强(减小体积)的办法能否使所有的气体都液化呢?用历史事实说明是不可能的,由此引入临界温度的概念,要强调在这个温度以上,物质只能处于气态,不能单纯用增大压强的方法来使它液化,临界温度是每种气体都具有的一个特殊温度,它的物理含义是:临界温度是物质以液态存在的最高温度,还可以通过临界管实验观察乙醚、水
等物质的临界状态,再通过物质的临界温度的表,弄懂氢、氧、氮等气体在历史上为什么会被当成“永久气体”的原因。

要重视液态气体和低温技术应用的介绍,讲清楚一、两个典型例子,以使学生眼界开阔、思维活跃。

\subsection{第三单元}
这一单元围绕湿度和湿度的测定,讲解了绝对湿度,相对湿度、露点的概念,介绍了几种湿度计的原理和使用方法,是已学基础知识的应用,与实际有紧密的联系。由于又引入了一些新概念,教师要注意引导启发学生用已学知识去理解新知识。

\subsubsection{空气的湿度}

要联系生活事例,说明空气的干湿程度是经常变化的,再讲解绝对湿度的定义,绝对湿度的初始定义-空气中所含水蒸气的密度,即单位体积中所含水蒸气的质量,学生是可以接受的。过渡到“空气中所含水蒸气的压强,叫空气的绝对湿度”时,须引导学生从决定气体压强的两个微观因素:分子的密度和分子运动的平均速率,得出:温度一定时,气体压强与分子密度成正比。从而理解绝对湿度的定义。

再通过实例和课本列举的数据,说明湿度的影响取决于空气中的水蒸气离饱和状态的远近,引入相对湿度的概念,对相对湿度的计算,学生易于接受,可让学生阅读教材及不同温度下饱和汽压的表格,做点练习题,理解这些知识。

\subsubsection{露点}

复习用降低温度使未饱和汽变为饱和汽的方
法及其道理,引入并讲清露点的概念:设温度为$t_1$时,空气中未饱和汽的密度为$\rho$; 降低它的温度至$t_2$, 若$t_2$温度下的饱和汽密度也是$\rho$, 在$t_2$温度下,原来的未饱和汽就变成饱和汽了.温度$t_2$就是空气的露点.

测定露点的实验,瓶面出现的凝结现象不易观察,温度计的可见度小,在课堂上演示时,可请学生参与操作和观察,把结果告诉大家。

由露点来求相对湿度,可以先讲一讲根据露点与原来气温的差值可以大致判断相对湿度是大还是小,再讲清根据露点计算相对湿度的方法。先要明确露点温度水的饱和汽压值,就是原来温度下水的未饱和汽压值,即原来温度下的绝对湿度,这可在不同温度下水的饱和汽压表中查出,再在这个表中查出原来温度下水的饱和汽压值,即可计算相对湿度。

\subsubsection{湿度计}

教师可以对干湿泡湿度计的使用作演示和示范讲解,再让学生阅读教材,了解它的构造、原理和优缺点。阅读前可以提出一些思考题,如为什么湿泡温度计的示数要低于干泡温度计的示数?使用时要读出哪些数据?怎样才能得出相对湿度?还可以给出一些简单的练习题,帮助学生了解怎样用湿度计测定相对湿度。

\section{实验指导}
\subsection{演示实验}
\subsubsection{研究饱和汽性质的实验}

实验装置如图5.4所示.$A$是一根竖直固定在木板上的
玻璃管,上下口用橡皮塞塞紧。$B$是一根上端有进液口$D$和进液阀门$E$的长直细玻管,直穿上下两橡皮塞的正中央,下端口通过橡皮管与长颈漏斗$C$相连。长颈漏斗$C$夹持在木板上,可以上下移动,木板上画有均匀刻度,圆筒上端橡皮塞装有一进水口$F$和插入一温度计$G$. 圆筒下端橡皮塞上装有一带阀门的出水管。以上装置可用J2257型气体定律演示器代替。
\begin{figure}[htp]
    \centering
      %\includegraphics[scale=.7]{fig/5-4.png}
    \caption{}
\end{figure}

\paragraph{饱和汽压强的测量}
启开阀门$E$, 从长颈漏斗$C$的上端灌入清洁水银,提起漏斗,让水银排出$B$管内空气,使水银面上升至$B$管上端的阀门$E$, 恰有少量水银过阀门后,关闭阀门$E$. 在进液口$D$装入适量的乙醚(或其他被测物质),降低长颈漏斗,使$B$管内出现一段真空,两管水银面的高度差应为当地的大气压强$p_0$. 缓慢启开进液阀门$E$, 滴入$B$管适量的乙醚,再关上阀门$B$. 乙醚在$B$管水银面上方空间中蒸发;至$B$管水银面上剩有少许乙醚液为止,这时管内空间充满乙醚的饱和汽,读出此时$B$、$C$两管水银面的高度差$h$。$p=p_0\pm \rho gh$即为乙醚的饱和汽压值(当$B$管的水银面高于$C$管的水银面时,取“$-$”号,反之,取“$+$”号)。

\paragraph{饱和汽压不随体积变化}
保持室温不变,将长颈漏斗缓慢提升,可以看到饱和汽所在空间的体积变小,$B$管中水银面上乙醚液增加,而$B$、$C$管水银面高度差不变,这说明在温度一定时,饱和汽压与体积变化无关。

\paragraph{饱和汽压跟温度变化的关系}
把长颈漏斗置于适当位置,读出温度$t_1$和$B$、$C$水银面的高度差$h_1$, 这时的饱和汽压$p_1=p_0\pm\rho gh_1$. 从进水口$F$向圆筒中注满热水,在温度升高的过程中,$B$管水银面上方乙醚液将蒸发,若蒸发完了,则可启开进液阀门$E$, 继续滴入乙醚,待蒸发到水银面上留有少许乙醚为止,读出这时的温度$t_2$及$B$、$C$管水银面的高温差$h_2$, 饱和汽压为$p_2=p_0\pm \rho gh_2$, 得出饱和汽压随温度的升高而增大。

\subsubsection{温水在低压下沸腾}
如图5.5所示,在广口瓶中装入大半瓶温度约$95^{\circ}{\rm C}$左右的水,瓶口塞紧装有开口弯玻管的橡皮塞。玻管上口用橡皮管与注射器连接。用注射器抽气,使瓶内温水上方的气压降低,可看到温水沸腾现象。
\begin{figure}[htp]\centering
    \begin{minipage}[t]{0.48\textwidth}
    \centering
%\includegraphics[scale=.7]{fig/5-5.png}
    \caption{}
    \end{minipage}
    \begin{minipage}[t]{0.48\textwidth}
    \centering
%\includegraphics[scale=.7]{fig/5-6.png}
    \caption{}
    \end{minipage}
    \end{figure}

如图5.6所示,在烧瓶塞中插一个温度计$C$, 一个三
通管和一个直角弯管,三通管的一个支管带有阀门$A$的,另一支管用橡管接一大容量的注射器$B$; 直角弯管经橡皮管与
U型液体(水银或水)气压计$D$连接。

演示时,将阀门$A$启开与大气相通,用酒精灯将烧瓶内的水加热至沸腾,撤去酒精灯,待水中气泡消失、水温降至沸点以下时,关闭阀门$A$. 用注射器缓慢地抽气,可以看到U型气压计$D$左管液面上升,说明烧瓶内液面上方的压强减小;到一定时候,水重新沸腾。可以读出这时U型气压计两管液面的高度差$h$及温度计的示数$t$, 另测出当时的大气压强$p_0$, 由$p_0$和$h$可算出沸腾时对应的压强值$p$. 这样,可以定量研究$p$、$t$间的关系。

\subsection{学生实验}
\subsubsection{测定水的熔解热}

由于初中的学习,对这个实验已有一定基础。在明确实验目的、原理、操作步骤的前提下,还要注意以下几点。
\begin{enumerate}
\item 由于记录和计算涉及的物理量较多,学生要作充分的准备,分清哪些是实验中测得的?哪些是用表格查得的?哪些是计算得出的?便于正确处理数据,避免错记、漏重记。
\item $0^{\circ}{\rm C}$的冰块要在已准备好的冰水混合物中取出,尽量少带水分,迅速投入量热筒。
\item 温水的温度和质量、冰块的质量等要参照课本给出的参考数据,不要偏离过大,影响实验效果。
\item 搅动小筒中的水时,要用力适度,估计到温度计在水中的位置,以免损坏器材。实验时,温度计的示数从$t$. 下降到最低温度后,又缓慢回升,最低温度$t$就是热平衡时的温度。所以要对温度计密切注视,连续观察记录温度示数,才能正确确定平衡时的温度。
\end{enumerate}

\subsubsection{测定空气的相对湿度}

要明确这里使用的实验装置是一个简易的露点湿度计,通过测出空气的露点,计算相对湿度,实验的研究对象是金属盒外的空气。盒内的冰在水中熔解吸热,起降低温度的作用。

实验装置中光滑的环形金属片是为了易于观察金属
盒上的露滴,与之对比而设置的。要保证它的表面光亮清洁,跟金属盒绝热。

如果冰块的备取有困难,可以准备适量的铵盐(或尿素)代替,并准备一些浓硫酸。

投入水中的碎冰(不要太大块)或铵盐要适量.过少时,温度尚未降到露点,而溶质已熔解完毕,实验不能成功;过多时,温度降到露点以后的回升时间太长,搅拌的快慢,应视温度变化的具体情况而定。若用铵盐代替做实验时,发现温度回升太慢,可以滴入几滴浓硫酸,使温度回升加快。

为及时观察露滴的出现或消失,可用手指(或棉纱)在金属盒的同一地方来回擦动,将擦过的地方与环形金属片表面对比,以期观察是否有露滴出现或消失。一经发现,及时记下温度的示数。

\subsection{课外实验活动}
\subsubsection{测定水的汽化热}

实验误差的主要来源是设计原理的不够完善。这个实验把铝锅里的水每秒钟吸收的热量看作是不变的;实际上,水在升温和汽化(沸腾)的过程中,单位时间内吸收的热量是不同的。因为热交换物体间的温度差不同,单位时间传递的热量也不同,温差越大,传递的热量越多,温差越小,传递的热量越少,另外,随着水温的升高,水跟周围空气的温差增大,在单位时间内放出的热量也增加了。由于这两方面的原因,随着水温的升高,水在单位时间内获得的净热量就会减少,还有,水在
温度没有达到沸点之前就不断蒸发,达到沸点时水的质量已经比原来的少了,这些因素在本实验中都没有考虑,再则,水的温度在未升到沸点前已开始汽化,即水的质量随着温度的升高不断减少,实验误差的另一个来源是由测量的水的质量、水的初温和沸点,以及加热水至沸腾和全部汽化的时间,这是测量都可能产生误差。

\subsubsection{估计水升高的温度}

估计水升高的温度为摄氏几度。

\begin{enumerate}
    \item 测火柴杆的质量,把火柴杆看成是正四方柱体,测出火柴杆正方形横截面的边长$a$, 测出它的长度$b$, 则火柴的体积$V=a^2b$. 查出一般木材的密度$\rho$的值为$(0.4\sim 0.9)\x10^3{\rm kg/m^3}$, 火柴杆的质量$m=\rho a^2b$.
    \item 计算火柴杆燃烧放出的热量,可以认为,火柴杆完全燃烧放出的热量$Q_{\text{放}}=qm$. $q$是木材的燃烧值$1.26\x10^7{\rm J/kg}$。
    \item 估计热传递的效率.考虑到玻璃是热的不良导体,火柴杆燃烧时间内向周围空间传递的热量较多,估计$\eta=20\%$.
    \item 计算水的温度的升高$\Delta t$. 由于
    \[Q_{\text{吸}}=m_{\text{水}}C_{\text{水}}\Delta t,\qquad Q_{\text{放}}=q\rho a^2b\]
    由热平衡方程 $Q_{\text{吸}}=\eta Q_{\text{放}}$, 得
    \[\Delta t=\frac{\eta q\rho a^2b}{m_{\text{水}}C_{\text{水}}}\]
    代入数据,即可算得$\Delta t$.
\end{enumerate}

\subsection{练习一}

\begin{enumerate}
    \item 把玻璃放在火上加热,观察它的熔解情况,看看玻璃是不是先变软,再流动.玻璃是不是晶体?
    
\begin{solution}
 玻璃是先变软,再流动。由于它没有固液共存温度不变的熔点,所以玻璃不是晶体。
\end{solution}
    \item 解释下面的现象:把一块冰放在支承物上(图5.1),
    将两端各挂一个重物的铁丝搭在冰块上.过一段时间后可以看到,铁丝切进冰块,但是铁丝穿过冰块的地方并没有留下切口,冰仍然是完整的一块,铁丝
    为什么能切进冰块?铁丝穿过后上面的冰为什么又成了完整的一块?如果有条件,自己做这个实验.
\begin{figure}[htp]
\centering
%\includegraphics[scale=.8]{fig/5-1.png}
\caption{细铁丝穿过冰块而不留下切口
}
\end{figure}
    
\begin{solution}
因为铁丝与冰块接触面积小,在重物通过铁丝对冰的接触面的压力作用下,产生的压强很大,使冰的熔点降低,即在$0^\circ$C以下就可以熔解.因此接触处的冰熔解成水,铁丝切进冰块,切口处的水在铁丝切过以后,压强又恢复为1标准大
气压,而水温还在$0^\circ$C以下,切口处的水又结成冰,冰块又成了完整的。
\end{solution}
    \item   冬季在菜窖里放上几桶水,可以使窖内的温度不致降低得很多,防止把莱冻坏.这是什么道理?如果在窖内放入200千克10$^\circ$C的水,试计算这些水结成0$^\circ$C的冰时放出的热量.这相当于燃烧多少千克干木柴所放出的热量?于木柴的
    燃烧值约为$1.26\times 10^4$千焦/千克.
    
    \begin{solution}
      
    \end{solution}
\item  铜制量热器小简的质量是160克,装入200克20$^\circ$C的水.向水里放进30克0$^\circ$C的冰,冰完全熔解后水的温度是多少摄氏度?
    
\begin{solution}
  
\end{solution}
\item  量热器的铜制小筒里盛有200克15$^\circ$C的水,小筒的质量为160克,向水里放入50克0$^\circ$C的冰,求量热器里的末温度是多少摄氏度?
    
\begin{solution}
  
\end{solution}
\end{enumerate}


\subsection{练习二}

\begin{enumerate}
    \item 举出几个蒸发致冷的例子来.
    
    \begin{solution}
在注射针药前,护士要在注射部位涂碘酒(或酒精)
消毒,涂上后有一股凉爽的感觉,这是由于酒精易于蒸发,蒸发时从涂抹部位的皮肤吸热,致使温度降低,感觉凉爽。

在夏季高温的地区,常常向地面酒一些水,水在蒸发的过程中要吸收热量,从而达到室内降温的目的。      
    \end{solution}
\item 液面上的汽达到饱和时,还有没有液体分子从液面飞出?为什么这时从宏观上看来液体不再蒸发?
    
\begin{solution}
有分子从液面飞出,只不过这时,单位时间内飞出液面和回到液面的分子数是相等的,所以从宏观上看来液体不再蒸发。
\end{solution}
\item 饱和汽的密度怎样随温度而变化?饱和汽的压强怎祥随温度而变化?为什么这样变化?
    
\begin{solution}
饱和汽的密度随温度的升高而增大。因为温度升高后,液体分子的平均动能增大,单位时间内逸出液面的分子数增多,破坏了原有的动态平衡,液体继续蒸发,汽的密度不断增大,直到建立起新的动态平衡为止。

饱和汽的压强随着温度的升高而增大,气体的压强从微观上决定于分子密度和分子的平均速率。当温度升高时,饱和汽分子平均动能增大,分子平均速率也增大;同时,逸出液面进入空间的分子数增多,分子密度随之增大。决定压强的这两个微观因素都变大,因此饱和汽的压强变大。
\end{solution}
\item 在温度不变的情况下,增大液面上饱和汽的体积时,下面的说法哪些是正确的:
\begin{enumerate}
    \item 饱和汽的质量不变,饱和汽的密度减小;
    \item 饱和汽的密度不变,饱和汽的压强也不变;
    \item 饱和汽的密度不变,饱和汽的压强增大;
    \item 饱和汽的质量增大,饱和汽的压强也增大;
    \item 饱和汽的质量增大,饱和汽的压强不变.
\end{enumerate}

    
\begin{solution}
  (b)(e)是正确的。
\end{solution}
\item 解释下面的现象:密闭容器中装有少量液态乙醚,当容器的温度升高时,液态乙醚逐渐减少;容器升高到一定温度
时,液态乙醚消失;容器冷却时,容器中又出现液态乙醚.
    
\begin{solution}
  密闭容器内有液态乙醚,说明空间的乙醚蒸气处于饱和状态,当温度升高时,饱和汽变为未饱和汽,乙醚液要继续蒸发,因而逐渐减少。升高至一定温度时,乙醚液蒸发完毕,液态乙醚消失,容器冷却时温度降低,饱和汽的密度随之减小,空间中一部分乙醚汽的分子凝聚为乙醚液滴,积存下来又出现液态乙醚。
\end{solution}
\end{enumerate}




\subsection{练习三}
\begin{enumerate}
 \item 在1标准大气压下,乙醚的沸点是35$^\circ$C,这个温度时乙醚的饱和汽压是多大?
    
 \begin{solution}
这个温度时乙醚的饱和汽压是1标准大气压。
 \end{solution}
  \item 锡的熔点是232$^\circ$C,但是用锡焊的水壶盛着水放在1000$^\circ$C以上的火上烧,锡并不熔解.为什么?
    
  \begin{solution}
    由于锡焊水壶内盛有水,水壶把从火上吸收的热量及时地传递给水,因而水壶的温度也只能稍高于水的温度。通常情况下,水的沸点是100$^\circ$C, 即使在水沸腾时,水壶的温度也只能是100$^\circ$C多一点,不会达到锡的熔点232$^\circ$C, 所以,锡并不熔解。
  \end{solution}
  \item 在蒸汽暖室装置的散热器里,每小时有20千克100$^\circ$C的水蒸气液化成水,
    并且水的温度降低到80$^\circ$C.求散热器每小时供给房间的热量.
    
    \begin{solution}
      
    \end{solution}
    \item 某人在做测定水的汽化热实验时,得到的数据如下:
   钢制量热器小筒的质量为200克,通入水蒸气前筒内水的质量为350克,温度为14$^\circ$C;
    通入100$^\circ$C的水蒸气后水的温度为36$^\circ$C,水的质量为364克,他测得的水的汽化热是多少?
    
    \begin{solution}
      
    \end{solution}
    \item 容器里装有0$^\circ$C的冰和水各500克,向里面通入100$^\circ$C的水蒸气后,
    容器里的水升高到了30$^\circ$C.假设容器吸收的热量很少,可以忽略不计,并且容器是绝热的,
    计算一下通入的水蒸气有多少?
    
    \begin{solution}
      
    \end{solution}
\end{enumerate}




\subsection{练习四}

\begin{enumerate}
	\item 说明使未饱和汽变为饱和汽的方法和道理.
    
  \begin{solution}
使未饱和汽变为饱和汽的方法有两种。

一是降低温度,体积不变的情况下,温度下降,未饱和汽
的密度不变,由于饱和汽的密度跟温度有关系,温度低时,饱和汽的密度小,当温度下降至某一数值时,原来温度下未饱和汽的密度恰好等于该温度下饱和汽的密度,于是未饱和汽就变成了饱和汽。

二是增加压强,温度不变的情况下增加压强是由减小未饱和汽的体积来获得的。体积减小时未饱和汽的密度增大。当汽的密度增大至这一温度下所对应的饱和汽密度时,原来的未饱和汽就变为饱和汽。
  \end{solution}
\item 潮温的天气里,湿衣服不容易干,为什么?
    
\begin{solution}
  潮湿的天气里,空气的相对湿度大,空气里的水蒸气压强已接近饱和汽压值,衣服上的水蒸发较难进行,所以湿衣服不易干。
\end{solution}
\item 在绝对湿度相同的情况下,冬天和夏天的相对湿度哪个大?为什么?
    
\begin{solution}
冬天的相对湿度大,由公式$B=\dfrac{p}{P}\x 100\%$可知,在绝对温度$p$相同时,水的饱和汽压$P$大的,相对湿度$B$小;$p$小的,$B$大。从不同温度下水的饱和汽压表中可知,夏天温度高,$P$值大;冬天温度低,$P$值小,所以冬天的相对湿度$B$值大。
\end{solution}
\item 空气的绝对湿度是9毫米汞柱,气温是16$^\circ$C,相对湿度是多少?
    
\begin{solution}
  
\end{solution}
\item 教室里空气的相对湿度是60\%,温度是18$^\circ$C,绝对温度是多少?
    
\begin{solution}
  
\end{solution}
\end{enumerate}





\subsection{练习五}

\begin{enumerate}
	\item 在北方,冬天戴着眼镜从寒冷的室外进入温暖的空内时,镜片上常出现一层细小的露滴.这是为什么?
    
  \begin{solution}
在室外镜片温度较低,进入室内时,温度较低的镜片使周围温度下降至露点,空气中的水蒸气,便在镜片上凝成露滴。
  \end{solution}
\item 白天空气的绝对湿度是13.7毫米汞柱.天气预报夜里的最低温度是14$^\circ$C,如果空气的绝对湿度保持不变,夜里会不会出现露水?
    
\begin{solution}
  白天的绝对湿度,即空气中水蒸气的压强为13.7毫米汞柱,把这个压强值作为水的饱和蒸气压所对应的温度——露点,经查表得知约为16$^\circ$C, 而夜里的温度是14$^\circ$C, 已在露点以下,所以会出现露水。
\end{solution}
\item 如果干湿泡湿度计上两支温度计的指示数字相同,这时空气的相对湿度是多少?
    
\begin{solution}
若两支温度计的示数相同,即干湿泡温度差为零,湿泡不再蒸发水分,说明空气的水汽压强(绝对湿度)等于这一温度下的饱和水汽压,相对湿度为100\%.
\end{solution}
\item 空气的温度是20$^\circ$C,露点是12$^\circ$C,这时的绝对湿度和相对湿度是多少?
    
\begin{solution}
  
\end{solution}
\item 空气的温度是25$^\circ$C,相对湿度是50\%,气温降低到多少摄氏度时,才会有露出现?
    
\begin{solution}
  
\end{solution}
\end{enumerate}

\section{参考资料}
\subsection{关于现有的物态}

当大量微观粒子在一定温度和压力下,相互聚合为一种稳定的结构状态时,称为“物质的一种状态”,简称“物态”。

在本世纪以前,人们还只能从物体的宏观特征(体积、形状)来区别物质的状态;把它们分为固态、液态和气态,但从物质的内部结构来看就远不止这三态了。

对固态而言,就分为结晶态和非晶固态,从宏观看,一切晶体的基本特征有三点:外观上两对应的晶面夹角恒等;物理性质表现为各向异性;相变时有确定的温度,从微观看,晶体分子规则的排列,组成空间点阵,这些是区分结晶态和非晶固态的标志。

在结晶态和液态之间,有不少有机物质,既具有流动性(无一定形状),又具有类似晶体的光学性质,这种物态被称为液晶态。

等离子体,被称为物质的第四态。它是气体电离成为带正电的离子和带负电的电子所组成的集合体,而且正负电量相等,这两种离子的集聚状态叫等离子态,它是1925年美国的兰米尔在研究气体放电时发现的。通过加热到万度以上的高温、辐射、放电等方式,使气体电离就变为等离子体,等离子体存在的温度极高。例如喷焊工艺常用的等离子弧中,氢
等离子体温度为5400K以上,氩等离子体温度为14700K, 这也只能叫“冷”等离子体,灼热的等离子体可达几百万至几千万度高温,等离子体的研究,是目前研究受控热核反应的重要方面,此外,在天体物理、气体电离、微波和超声速流体力学等方面,都有重要的应用。

另外,还有几种在超低温、超高压、超高温下的物态。

在超低温的条件下,某些金属的直流电阻将超近于零,这叫做超导态,1911年荷兰物理学家昂尼斯,在温度降低至
4.2K时,发现水银直流电阻消失,为此他获得1913年的诺贝尔奖。超导材料的制成,将会引起电工技术的巨大变革,但由于需要超低温,目前技术上限制了它的应用。

在超低温的条件下,有的液体的粘滞性也完全消失,这叫做超流态,1930年荷兰科学家基索姆发现,当温度降至2.1K时,液态氮的性质发生突变,其粘滞系数为零,能够无阻地流动,如果在其中插上一根内径为$10^{-5}$厘米的毛细管,它就会象喷泉一样,溢出管外;而流速与液面的压强差和毛细管的长度无关,这种特性,他把它命名为超流。

在超高压条件下,氢可以转变成具有金属特性的固态,称为金属氢。有人估算,在几百万大气压下,氢可以转化为金属。可以推知,非金属也能在高压下转化为金属;再把这些变为金属的物质,在极低的温度下实现超导态,为提高超导转变温度开辟新路,有人在理论上计算,固态氢在压强为$1. 35\x10^5$帕的条件下可以转变为金属氢,而且金属氢的超导转
变温度在80K。目前,世界上有100多个实验室在研制金属氢。近两年来,苏联、日本相继宣布在实验室中研制成功。

在超高压、超高温条件下,物质原子的所有电子都脱离原子核而成为自由电子,所有的裸原子核高度紧密地堆积,自由电子在其间混乱运动,由于密度很高,被称为超固态,根据光谱分析说明,在一种叫白矮星的恒星上,物态处于超固态,其平均密度为水的几万至一亿倍,中心密度可达$10^{10}{\rm g/cm^3}$, 温度为1千万度.

设想的物态总图如图5.8所示。
\begin{figure}[htp]
  \centering
 % \includegraphics[scale=.8]{fig/5-8.png}
  \caption{}
\end{figure}

\subsubsection{水在结冰时体积增大}

水和冰的物理性质不同,与分子的结构有关,水分子属于非直线型的极性分子。其中的氧原子与两个氢原子通过共价键连接,两个O$-$H键间的夹角为$104^{\circ}40'$, 一个氢原子除了用仅有的一个电子与另一原子以共价键结合外,它的原子核(因没有电子)还会被其他原子的电子层吸引而形成“氢键”。 它是含氢化合物之间的一种相互作用,将在产生“缔合分子”的群体和影响晶体结构形状和稳定性方面起重要作用。
\begin{figure}[htp]
  \centering
 % \includegraphics[scale=.8]{fig/5-9.png}
  \caption{}
\end{figure}

水在液态时,由于水分子间氢键的作用,使得水分子不全以单分子存在,而是三三两两地缔合在一起的,如图5.9所示,如水的温度降低时,分子平均动能减小,缔合分子数增多.到$4^{\circ}{\rm C}$, 缔合分子排列最紧密,此时分子的密度最大,水的体积最小,温度继续降低时,出现更多的缔合分子;到$0^{\circ}{\rm C}$时,全部水分子因氢键结合,排列成规则的六方晶系,这是一
种以氢键作为桥梁架设起的较松散的结构,两个相邻的水分子中心的距离约为$2. 76\x10^{-10}$米,其间有很大的空洞,几乎可以容纳一个水分子,因而冰的外观体积较大,这就是水从液态转变为结晶态时体积增大的原因。


\subsubsection{水沸腾时气泡体积的变化}

水中空气泡的存在,是沸腾的重要条件。它是由附着在器壁上、未溶于水的微量气体形成的。

沸腾前,气泡还附着在器壁上时,随着水温升高,气泡周围的水不断向泡内蒸发,气泡的体积及泡内水的饱和汽压随温度升高而变大。当气泡体积增大到某一程度,水对它的浮力大于它对器壁的附着力时,气泡脱离器壁,在浮力作用下上升(图5.10)。
\begin{figure}[htp]
  \centering
 % \includegraphics[scale=.8]{fig/5-10.png}
  \caption{}
\end{figure}

由于沸腾前水的上、下层温度不同,离液面越近,水温越低,这时,气泡表面内,外压强的力学平衡条件是:
\begin{equation}
  p_0+p_{\text{静}}+p_{\text{表}}=p_{\text{饱}}+p_{\text{气}}
\end{equation}
其中气泡外部的压强:$p_0$是液面上方的大气压强;$p_{\text{静}}=\rho gh$是容器内液体的静压强;$p_{\text{表}}=2\sigma/r$是气泡表面张力引起的附加压强,$\sigma$是表面张力系数。气泡内部的压强有:
\[p_{\text{气}}=\frac{mRT}{\mu V}\]
$p_{\text{气}}$
是气泡内空气的压强,$m/\mu$是空气的摩尔数;$p_{\text{饱}}$是泡内水汽的饱和气压。







\subsubsection{干湿泡温度计的干湿泡温度差与相对湿度的关系}

在干湿泡湿度计中,湿球水分汽化(蒸发),要从湿泡温度计吸热,使湿泡温度计的示数低于干泡温度计的示数,湿球水分的汽化,也从它周围的空气中吸热,使它周围空气的温度降低,在达到热平衡时,湿泡的温度跟它周围空气的温度相同。

设在干泡温度计的示数为$t_1$, 湿泡温度计的示数为$t_2$时达到热平衡,我们来求空气的相对湿度。

含有水的未饱和汽的空气与湿球热交换,温度由原来的温度$t_1$(即干泡温度计的示数)降低到湿球的温度$t_2$。根据热量交换理论可知,空气放出的热量$Q_1$跟干湿泡温度差成正比,即$Q_1=h(t_1-t_2)$. 式中$h$为热量交换系数,它的量值决定于风速大小。

设在$t_1$温度下,湿球蒸发的水蒸气的质量为$m$, 在这温度下水蒸气的饱和汽压为$p$, 周围空气中水蒸气的压强(绝对湿度)为$p$, 大气压强为$P_0$, 汽化热为$L$. 根据蒸发的理论研究可知,在湿球温度下蒸发的水蒸气的质量,跟饱和汽压与绝对湿度的差$(P-p)$成正比,跟大气压强成反比,即$m=k(P-p)/p_0$. $k$为比例系数.这部分水蒸发所需的热量















\chapter{电场}\minitoc[n]
\section{教学要求}
这一章讲授静电学,从电菏在电场中受力和电荷在电场中具有能量两个角度出发来研究电场的基本性质.本章内容是电学的基础知识,也是学习后面各章的准备知识.

基本概念多而且抽象,是这一章的突出特点,针对这个特点,教材注意从具体情况出发引入概念,而不过于强调抽象的论证;注意加强演示实验,力求使学生获得感性知识;注意讲清楚概念的物理意义.这一章的另一个特点是许多知识要在力学知识基础上学习,教材在内容选择、确定讲述方法时注意了这个特点,希望教学中也给予注意,把新旧知识联系起来.

这一章的教材内容,是从两种电荷出发,学习电荷守恒定律和库仑定律,以此为基础,认识表征电场的力的性质和能的性质的物理量-电场强度和电势,以及它们之间的相互联系,构成较完整的静电场基础知识,作为上述知识的应用,学习了电场中的导体、带电粒子在电场中的加速和偏转、密立根实验.作为基础理论的引伸,学习了电容的概念及电容器的连接.

这一章教材可分为四个单元:第一单元包括第一节到第五节,讲述库仑定律和电场强度,第二单元包括第六节到第十节,讲述电势能、电势和电势差跟场强的关系.第三单元包括第十一节和第十二节,讲述带电粒子在电场中的运动,介绍密立根实验,第四单元包括第十三节到第十五节,讲述电容器及电容器的连接.

下面对这一章的教学内容作些具体说明.库仑定律是这一章的基础.教材详细介绍了库仑定律的实验,目的是使学生了解一些重要的物理实验方法,活跃学生的思维.教材介绍了电介质中的库仑定律,只要求学生了解什么是电介质,以及电介质中电荷间的作用力比真空中小,而不给予微观上的解释.

电场的概念,对于学生是个新概念,开始只要求大体有所了解,不要求深入地解释电场为什么是物质的一种特殊形式.电场强度和电势就是从电荷在电场中受力并具有能量这一客观事实的基础上,建立的两个重要的表征电场性质的物理概念,通过学习这两个概念以及它们的相互联系,使学生对电场的物质性有进一步的认识,由于这两个概念较抽象,又是这一章的两个教学难点,因此,在讲解电场强度之前,教材讲述了检验电荷在电场中不同位置所受的电场力的大小和方向不同的实验,使学生先对电场的强弱有所认识,便于引入电场强度的概念,对于电势的讲述,教材结合具体的电场定性地说明电势能跟电量的比值为一恒量,从而引入电势的概念,不要求作定量的论证,其目的是为了使学生容易接受一些,为了把场强和电势这两个难点分散开,在它们之间,安排讲授电场中的导体.

教材经过分析推导得出带电粒子在电场中的侧移距离和偏角公式,但不要求学生记忆这些公式,应该注意培养学生运用物理规律对具体问题进行具体分析的能力.讲述这一内容时,还要综合运用力学和电学知识,有利于发展学生综合运用知识的能力,教学中要予以重视.

密立根实验是物理学的经典实验之一,教材介绍了这个实验的基本思路,限于学生知识基础,不要求进一步讲解密立根是如何测定油滴半径的,不要求讲解密立根实际所做的实验.这个实验只讲给学生,不要求实际去做.

电容的概念比较抽象.教材讲述了平行板电容器的电容,并给出了公式,其目的在于让学生领会电容是由电容器本身的因素决定的,不要求用公式进行定量计算.

教材介绍了静电的应用及其危害,是为了加强理论知识与实际的联系,以利于扩展学生眼界.

这一章的教学要求是:
\begin{enumerate}
\item 了解电荷守恒定律,掌握库仑定律,能够计算点电荷间的相互作用.
\item 了解电场的概念,理解电场强度和电力线,掌握电场强度的公式和单位,了解匀强电场的特点.
\item 理解导体处于静电平衡时的特点.
\item 理解电势能、电势、电势差的物理意义,了解等势面.掌握匀强电场中场强和电势差的关系.
\item 掌握带电粒子在电场中的运动规律,能够分析解决加速和偏转方面的问题.
\item 了解密立根实验的简单原理和测定基本电荷的意义.
\item 理解电容器的电容概念、掌握电容器串联和并联的公式.
\item 了解静电的应用.
\end{enumerate}

\section{教学建议}
\subsection{第一单元}
\subsubsection{两种电荷}

教学中首先要利用实验演示电荷的种类、电荷的相互作用、电荷的相互增强和减弱、抵消等现象.比如,用一绝缘线悬挂一导体球,使其带上某种电荷.用毛皮摩擦后的橡胶棒接近它时,相互排斥;用丝绸摩擦后的玻璃棒接近它时,相互吸引.这说明橡胶棒和玻璃棒上带两种不同的电荷.再通过讨论问题:
\begin{enumerate}
    \item 什么叫物体带电?电荷有哪两种?它们之间有怎样的相互作用?
    \item 什么叫电荷的相互增强和中和?
    \item 摩擦起电的过程是怎样的?
\end{enumerate}
使学生回顾和复习初中的静电知识.

然后,结合演示实验讲解什么是静电感应和感应起电,这里只说清楚现象,理论解释留在电场中的导体一节讲述.至此,可以归纳已学过的起电的各种方法(接触起电、摩擦起电和感应起电)的特点,使之对起电过程,即使物体中电荷分离的过程有较深的理解.

\subsubsection{电荷守恒定律}

充分运用前面演示实验的事例:课本
图6.1所示的电荷相互增强和中和的现象,摩擦起电的过程,课本图6.2所示的静电感应现象的分析;说明以下问题:电荷的中和是否正、负电荷都消失了?物体呈中性,是否说明物体中没有电荷?感应起电中,是否导体两端新产生了正、负电荷?从而得出荷守恒定律的内容,强调在电荷的分离和转移的过程中总量保持不变.

\subsubsection{库仑定律}

从初中对电荷间相互作用的认识过渡到对相互作用力的定量研究,首先就要弄清楚电荷间相互作用力的大小和方向.库仑定律就是反映这种关系的物理规律,是电学的基础规律.

点电荷的概念,可以从质点的概念出发来理解.指出这是一个理想化模型,明确在哪种实际情况下,可以把带电体科学地抽象为点电荷,强调研究点电荷间的相互作用,是库仑定律成立的前提条件.

得出库仑定律,是以库仑扭秤实验为基础的,通过教学,不仅可以了解库仑定律的建立过程,而且能学到一些物理实验的方法,这个实验不要求实际去做,但应运用模型或挂图,以增强教学的直观性.同时,要引导学生联想卡文迪许实验验证万有引力定律的过程,以帮助学生了解库仑扭秤实验装置和原理.教学中可采用讲解或自学讨论的方法,如果用后一种方法,可提出以下问题:库仑是用什么办法测算出点电荷间的作用力的?当时还没有电量的单位,库仑是怎样解决作用力与电量的关系的?让学生带着问题阅读教材,展开讨论,明确库仑扭秤实验的过程.当然,要指出两个金属球必须是大小相同的,但不必说明理由.

对库仑定律的内容,要把文字表述和数学表达式结合起来理解,还要把库仑定律和万有引力定律作对比,以利于记忆和应用,要再次强调定律成立的条件是点电荷,指出它的适用范围可以推广到静止的电荷与运动电荷之间的相互作用(例如原子核对运动电子的作用);运动电荷间的相互作用则不适用了.

电介质中的库仑定律,可以用两个相互作用的带电小球间插入一电介质(如塑料板)后,其作用力减小的现象推理得出:如果这电介质充满空间,两个带电小球的相互作用力比在真空中的要小,这样对公式$F=kQ_1Q_2/\varepsilon r^2$的理解就会容易一些,对电介质的意义和介电常数,只作介绍,不作深入探讨.

在应用定律进行运算时,电量要用绝对值代入,力的方向由是引力或斥力具体确定,公式中各物理量的单位都统一使用国际单位制的单位.

在解决多个点电荷间作用问题时,要注意每两个点电荷间就有一对库仑力,它们遵从牛顿第三定律.

在解决综合力学问题时,带电体不仅受到库仑力,还可能受到万有引力(重力)、弹力、摩擦力的作用.教材的例题计算结果说明,在研究微观粒子的相互作用时,库仑力比万有引力大得多,因而万有引力可以忽略;但在其他带电体的平衡或运动的问题中,是否可以忽略万有引力(重力),应视具体情况而定.

以上各点,应结合例题、习题的教学,尽可能启发学生自行得出,使之易于理解、记忆和应用.

\subsubsection{电场和电场强度}

电场是学生新接触的抽象概念,教
学时要加强直观性,做好课本图6.4甲的演示实验,可以采用讲解法,与磁极间的相互作用比较,由学生已有的磁场概念形成对电场的认识,再结合对电磁场、电磁波及其应用于实际的广播、电视的介绍,帮助学生理解电场是一种特殊的物质.并指出电场的物质性的表现之一,就是它对其中的电荷具有力的作用.

关于电场强度的引入,首先要讲清检验电荷的意义.为了感知电场的存在及其性质,要用检验电荷进行探测,只有点电荷,由它来确定电场中某点的位置才有确切的意义;只有带电量很小,它自身的电场对源电荷电场的影响才能忽略.然后,做好课本图6.4甲所示的实验,定性地说明检验电荷在电场中的不同位置,所受电场力大小不同,进而说明电场各点的强弱不同,为此引入电场强度这个物理量.

讲解电场强度的定义时,可提出“能否用电荷所受电场力的大小表示电场的强弱”这一问题来思考,引导学生回忆密度的概念:对某种物质,体积大的质量大,用单位体积的质量、即密度来表示物质的这种特性;在电场中某点,电荷的电量大受到的电场力就大,用单位电量的电荷受到的电场力、即电场力与电量的比值,表示电场的力的性质——电场的强弱,再利用课本图6.4乙,以点电荷电场中的$A$点为例,分析电场力随电量的变大而增大,概括出电场力与电量的比值相同;对不同的点则比值不同,由此抽象出这个比值与检验电荷的电量无关,表示电场本身的性质:在比值大的点电场强,在比值小的点电场弱.并把点电荷电场的定义推广到任何电场.

场强的方向,可以在课本图6.4乙的基础上来说明,以$+Q$为圆心、以$r_1$为半径作圆,在圆上取任一点$D$, 说明$A$、$D$两点场强的大小相等,同一电荷分别在这两点所受的电场力力大小相等,但力的方向是不同的.从而说明场强是有方向的,是矢量,然后再进一步指出,场强的方向是怎样规定的.由于场强是矢量,对几个电场的叠加,其合场强要用平行四边形法则进行矢量运算;对此,只要求简单介绍,不要求定量计算.

可以把电场强度与电荷所受的电场力,从意义、公式、方向、单位等方面用列表的形式,加以比较,加深对场强的理解.还应该通过解答习题,帮助学生认识公式$E=F/q$与公式$E=kQ/r^2$和$E=kQ/\varepsilon r^2$的区别和联系.

\subsubsection{电力线}

先通过讲解,明确引入电力线可以形象地表示电场分布.再演示悬浮在蓖麻油中的木屑在各种电场(正、负点电荷的电场,两个等量异种电荷和等量同种电荷的电场)中排成一系列直线或曲线的情况.再说明电力线是人为设想的,不是真实存在的.最后讲述电力线的定义,说明线上各点的切线方向,与该点的场强方向一致.

匀强电场是最简单而又很重要的电场,要在演示木屑微粒在带正,负异种电荷的平行板间的分布形状的基础上,讲清什么是匀强电场?它的场强的特点和电力线分布的特点是什么?实际的匀强电场有哪些?

\subsubsection{电场中的导体}

教材先用分析推理的方法,讲述导体在电场中的一些基本性质,然后用实验验证,希望能发展学生的推理能力.为了激发兴趣、设置悬念,加强直观,可以先演示带电小球在电场中受力、小球罩上金属空腔后又不受电场力的现象.再提出问题:电场中的金属空腔为什么能起这样的作用:把研究金属空腔,转变为研究电场中的导体.

引导学生复习金属导体的组成和导电的微观机制,有层次地讲解:
\begin{enumerate}
\item 导体在电场$E$中,内部的自由电子受电场力作用作逆电场方向的定向移动,导致垂直于场强方向的两端面出现等量异种电荷;
\item 同时,这种电荷将在导体内形成与外电场$E$方向相反的附加电场$E'$;
\item 在$E'<E$时,自由电子的定向移动按原方向继续进行,两端面积累的电荷增多,附加电场$E'$增强;
\item 当$E'=E$时,导体内部自由电子的宏观定向移动停止,处于动态平衡状态,两端面积累的电荷稳定;由于$E$与$E'$叠加的结果,导体内的合场强为零,因而没有电力线分布.
\end{enumerate}
在以上的讲解的基础上,说明静电平衡状态的意义,以及在这种状态下场强的特点.

导体在静电平衡时其表面上任一点的场强方向与导体表面垂直和电荷只能分布在导体表面,教材都是用反证法来说明的,学生不易理解;教师既要讲清知识,又要讲清思路.之后,再用演示实验验证.

\subsection{第二单元}
\subsubsection{电势能}

教材是直接与重力势能类比,引出电势能概念的,不要求论证移动电荷做功与路径无关.具体地以异种的场源电荷和检验电荷为例,把场源电荷与地球、检验电荷与物体类比,从重力势能和重力做功跟重力势能变化的关系,来认识电势能和电场力做功跟电势能变化的关系.教学中,可以在复习重力势能、正功、负功等知识的基础上,引导学生列表对比进行讨论.应指出,由于电场力可以是引力也可以是斥力,因此电场力做功的问题要复杂一些.

教材中结合正、负电荷的电场来讲解电势能,对正电荷在电场中不同位置时电势能的大小、数值、正负分别作了说明,以期学生对电势能的认识更加具体,讨论正电荷$q$在$+Q$和$-Q$的电场中不同点电势能的大小时,可按以下顺序进行:
\begin{enumerate}
\item 确定$q$所受电场力的方向和它在电场中两点间的位移方向;    \item 根据两者的方向关系,判定电场力做正功还是负功;    \item 由电场力做功跟电势能变化的关系,判定电荷$q$在电场中两点电势能的大小;    \item 得出结论:在$+Q$的电场中,电荷$q$离$+Q$越近,电势能越大;在$-Q$的电场中,电荷$q$离$-Q$越近,电势能越小.在教学中,第一种情形由教师示范讲解,第二种情形可由学生自行推出结果,以利培养学生的推理能力.
\end{enumerate}

在讨论电势能的数值时,可先复习重力势能的数值是怎样确定的,再与重力势能类比,以正电荷$q$在$+Q$和$-Q$点电荷的电场中为例,讲请确定电势能数值的过程.要先确定电势能为零的位置,一般取电荷$q$在无穷远处的电势能为零,把电荷$q$从电场中某点移到零电势能处电场力做功的大小,即为电荷$q$在电场中某点电势能的数值.

在讨论电势能的正负时,要先复习重力势能正负的意义.然后,仍以$+Q$和$-Q$点电荷的电场为例,讲清在$+Q$的电场中,将正电荷$q$从任何一点移向无穷远(零电势能)处,电场力做正功,电势能减小至零;因而正电荷$q$在$+Q$电场中任
何一点的电势能都是正值.同理,可以说明正电荷q在$-Q$的电场中任一点的电势能都是负值.需要强调,这结论是检验电荷为正的条件下得出的;若检验电荷为负电荷,则结论不同.

\subsubsection{电势}

在引入电势之前,可复习场强的引入过程,说明在电场中某点,随着检验电荷电量的增大,所受电场力成正比地增大,但电场力与电量的比值是确定的,这就是该点的场强.它与检验电荷的电量无关,是表示电场力的性质的物理量.与场强的引入过程类比,结合具体的电场定性地说明:在电场中某点,检验电荷的电势能随电量的增大而成正比地增大,但电势能与电量的比值是确定的,它就是表示电场的能的性质的物理量,从而引入电势.电势的高低也与检验电荷的电量无关,由电场本身的性质决定.

与电势能对应,讲明零电势位置的确定;以及它确定后,电场中各点电势的大小才有确定的数值,电势的正负才有确切的意义.再应用上述认识,说明点电荷$+Q$和$-Q$的电场中各点电势的高低,从而得出结论:在$+Q$的电场中,电势都为正值,离$+Q$越近电势越高;在$-Q$的电场中,电势都为负值,离$-Q$越近电势越低.

根据电力线的方向可以判定电场中各点电势的高低,为帮助学生理解,可先列举电荷在如课本图6.16的电场中的各点间移动时,根据电场力做功的正负,判定电势能的变化,进而判定电势的高低,再考察电力线的方向与这些点的电势高低的联系.最后概括得出:顺着电力线方向,电势越来越低.

要及时帮助学生小结已学过的知识,可以把电场强度与
电势、电势与电势能,分别列出表格,从意义、公式、方向性、单位等方面进行比较,巩固掌握这些概念,也发展了分析、比较的思维能力.

\subsubsection{等势面}

首先,要明确什么是等势面?可以运用地图上等高线作比喻,以点电荷的电场和匀强电场为例,配合模型教具,讲清等势面的意义.

电力线与等势面的关系,可先用反证法得出:等势面一定跟电力线垂直;再在复习电力线方向与电势高低的关系的基础上得出:电力线是由电势较高的等势面指向电势较低的等势面.注意运用这些知识,分析具体的各种电场中电力线与等势面的关系,了解和想象这些电场的等势面的形状.

在讲解带电导体(课本图6.22)周围的电场中等势面的分布时,要讲清“处于静电平衡状态的导体是等势体,其表面是等势面”,使学生对处于静电平衡状态的导体的力的性质(场强)和能的性质(电势)的特点,有全面的认识.

\subsubsection{电势差}

引入电势差时,除用地理位置的高度差不变
来比喻外,还可以列举数据阐述或指导学生运算,说明在匀强电场中,选定不同的零电势位置,其中任意两点的电势之差保持不变,进而指出电势差的应用比电势更普遍,它的物理意义和公式也就顺势得出了.讨论电势差在$U_A>U_B$和$U_A<U_B$的两种表示式,是为了表明:两点间的电势差要取绝对值,又要明确哪点的电势高.还要用公式说明,电场中某点的电势,就是该点相对于零电势位置的电势差.

用电势差来计算电场力做功时,公式中各物理量取绝对值,电场力做功的正负要根据电荷移动方向与所受电场力方
向的具体情况来判定.

\subsubsection{电势差和场强的关系}

讨论电势差和场强的方向的关系时,为了便于学生接受,可以举例说明,自山顶上从坡度不同的两个方向下到同一水平面,坡度陡的方向,单位长度的水平距离上高度下降大,即高度下降得快,再结合课本图6.24的匀强电场讲解,$AB$、$AC$、$AD$三个方向,电势下降的差值都相同;$AB$的距离最小,单位长度上电势降落最大,即电势降落最快;而$AB$的方向就是场强的方向,因而得出:场强的方向是指向电势降低最快的方向.

在得出电势差和场强在数值上的关系式$E=U/d$后,要强调理解公式的意义和适用条件,$d$是两点间在场强方向上的距离,$U$是所对应的两点间的电势差;它只对匀强电场才适用.

\subsection{第三单元}
本单元的内容,是以带电粒子在电场中的运动和密立根实验为例,应用电场和力学的知识解决综合性问题.由于这部分内容主要是已学知识的运用,可较多地放手让学生自行探讨、研究,以利于对电场知识的深人理解和巩固,培养学生分析和解决问题的能力.

\subsubsection{带电粒子的加速和偏转}

为了增强学生的感性认识,可以演示阴极射线管,说明热电子在加速电场作用下,形成电子束;演示阴极射线管或其他带电粒子在电场中偏转的现象.要复习力学中匀加速直线运动和平抛运动的有关知识,以便
为学生学习这一部分知识作好准备.

结合课本图6.27和图6.28, 可引导学生应用已学的电学和力学知识自己去探讨研究.对带正电粒子的加速,要求学生分别用牛顿运动定律和动能定理,推导出带电粒子达到负极板时的速度$v=\sqrt{2qU/m}$, 对带电粒子的偏转,要求学生对照物体在重力场中的平抛运动,从受力、加速度、飞行时间、侧移距离、末速度$v$与初速度$v_0$的夹角$\phi$, 逐步推导出侧位移$y$和夹角$\phi$的表达式,并讨论决定$\phi$的因素.

还可以深入讨论下述一些问题.比如,在带正电粒子加速的现象中,可以提出:
\begin{enumerate}
\item 若粒子以速率$v_0$从正极板小孔沿场强方向进入电场,则从负极板小孔穿出时的速率$v$有多大?
\item 若粒以初速度$v_0$从负极板小孔沿场强方向进入电场,粒子作什么运动?为什么?
\item 如果不是匀强电场,能否用匀变速动公式求出粒子到达负极板的速率?为什么?应该怎么办?
\end{enumerate}


要概括讲解分析和解决问题的思路:对带电粒子进行受力分析(包括电场力),明确粒子的运动状态及过程,运用相应的力学规律列式求解.

\subsubsection{密立根实验}

这个实验是较精确地测定基本电荷数值的重要物理实验.应向学生讲解基本电荷的概念,简述电子发现的过程及有关测定电子电荷量的实验.这个实验只要求学生懂得道理和方法,不要求实际去做.

要先介绍实验的设想:如果电子的电量是基本电荷,那么,测出若干个带电微粒的电量,如果这些电量都是某一电量的整数倍,这个电量,即为基本电荷.接着,讨论下面的问题:
\begin{enumerate}
\item 实验研究中用的是什么微粒1怎样使它带电的?
\item 根据什么原理、使用怎样的装置、采取什么方法测得带电微粒的电量?
\end{enumerate}
通过讨论,明确测微粒电量的原理和方法.对计算式$q=mgh/U$中,油滴质量$m$的计算,应作简要的说明.在对数千个油滴电量的实验数据进行分析研究后,才得出基本电荷的数值$e$.

在上述讨论、说明的基础上,讲述实验原理.实验是根据带电油滴在电场中平衡时,所受合力为零来计算油滴的荷电量的,进而应用数学知识进行分析研究,求得基本电荷数值的.带电微粒在电场中平衡时,受力分析一般要考虑重力.

\subsection{第四单元}
\subsubsection{电容}

可以从电视机、电子仪器中广泛使用电容器,
引入对电容器的学习,让学生观察废旧的电容器的外形及电极,再把它拆开,看两个极板(金属箔)和中间的电介质,了解电容器是由两个彼此绝缘又相互靠近的导体构成的.再演示电容器充电和放电的现象,配合讲解,说明充电后,电容器两极板带上等量异种电荷,电容器的功能就是能够容纳(储存)电荷.要强调指出:电容器所带的电量,是指一个极板带电量的绝对值.

不同电容器容纳电荷的本领是不相同的,可从怎样表征这一性质,过渡到电容概念的教学,电容器这个概念比较抽象,教材把它与容器的容量作了对比,既然是对比,就不可能完全相同,达到有助于理解电容的目的就可以了.

还可以用场强、电势等用比值定义的物理量作类比,加深对电容概念的理解.

\subsubsection{平行板电容器}

这里讨论的是决定它的电容大小的因素及相互关系,使之对电容的物理意义有深入的认识.讲清这一问题的前提是做好教材要求的演示实验,在介绍实验装置时,应讲清静电计指针张角的大小为什么能够表示电容器两极板间电压的大小.在进行操作前,可以先把观察现象的记录表格板画出来,使学生观察有目标,并在演示中及时记载结果.根据实验结果,概括出平行板电容器电容公式$C=\dfrac{\varepsilon S}{4\pi kd}$, 可以用已学过的知识,对公式作定性说明,还要指出,这个公式与电容定义式的区别和联系,以及它只适用于平行板电容器这一条件.

要通过练习或例题说明,使用公式$C=Q/U$讨论平行板电容器的有关问题时要注意:
\begin{enumerate}
    \item 若电容器两极板与电源相连接,则极板间的电压不变;
    \item 若电容器极板与电源断开,则极板的荷电量不变.
\end{enumerate}



\subsubsection{电容器的连接}

讲述电容器的串联和并联,可从连接方法、电量关系、电压关系和电容计算等方面通过讲解或指导学生推证,用列表的方法加以比较.

根据静电感应的原理,讲清串联电容器中的每个电容器极板带电量相等,且等于电容器组的总电量,根据电荷守恒定律,讲清并联电容器组的总电量等于每个电容器所带电量之和.根据电容器各级板的电势关系,分析得出:串联电容器
的总电压等于各个电容器极板电压之和;并联电容器中各个电容器极板间的电压相等,且等于总电压.上述结果,还可以用电压表对实际电路中各个电压的测定数据进行验证.

两种连接法的电容的计算,或教师讲解、或指导学生用公式推导,得出
\[\frac{1}{C_{\text{串}}}=\frac{1}{C_1}+\frac{1}{C_2}+\cdots,\qquad C_{\text{并}}=C_1+C_2+\cdots\]
说明$C_{\text{串}}$小于每个电容器的电容,$C_{\text{并}}$大于每个电容器的电容,作出定性解释,并指明使用串联或并联电容器的条件.

\subsubsection{静电的防止和应用}

可通过实例使学生认识静电与人类生活的紧密关系.列举吸尘、干扰、火花引爆等事例说明静电的危害,指明接地、调节空气湿度等防止静电危害的方法.

重点是介绍静电的利用,要做好两、三个演示实验,如静电除尘、静电植绒等;还可以补充一些事例,以扩展学生眼界.对现象要用静电知识作简要的说明,注意突出物理原理,而不涉及技术细节.

结合本节的教学内容,可对学生进行“认识自然,改造自然,建设祖国,造福人民”的思想教育.

\section{实验指导}
\subsection{演示实验}
\subsubsection{摩擦起电时产生等量的异种电荷}

拿两块相同的玻璃圆板(装有绝缘手柄),在一块板面上贴上丝绸如图6.1中的$B$. 两块板相互摩擦,然后分别把它
们接近金属箔验电器,金属箔张开大致相同的角度,可以证明两块板带了等量的电荷.如果这两块带电板紧密接触(合拢),合在一起靠近验电器,金属箔不再张开.这是由于两板带的电,合拢以后的作用相互抵销了.这说明两个相互摩擦的物体同时分别带上等量异种电荷.
\begin{figure}[htp]\centering
    \begin{minipage}[t]{0.48\textwidth}
    \centering
    \includegraphics[scale=.4]{fig/6-1.png}
    \caption{}
    \end{minipage}
    \begin{minipage}[t]{0.48\textwidth}
    \centering
    \includegraphics[scale=.4]{fig/6-2.png}
    \caption{}
    \end{minipage}
    \end{figure}


\subsubsection{电力线谱}

可以用投影装置演示电力线.如图6.2所示,将玻璃皿洗净烘干,注入厚约2—3毫米的蓖麻油.在蓖麻油中放入两个金属圆片作为电极(圆片正中先焊一小段铜管,以备插入接起电机的导线),再撒进一些羊毛绒(羊毛脱脂后染上颜色),用玻璃棒搅匀.演示时,将这个装置放在幻灯机的平台上,两极板接到起电机上,中速均匀地摇动起电机使两极带电,羊毛绒在电场作用下有规则地排列,并在屏幕上显示出它的投影.电力线还可以用验电羽来演示.

用投影装置演示电力线,能使学生看清羊毛绒在电场中是怎样慢慢地按一定规律排列起来,用验电羽演示,可以看出电力线的空间分布,形状容易控制,有利于建立清晰的电力线谱的图形.

用验电羽演示匀强电场的电力线时,可先把两板分开到
一定的距离,起电后让学生看清孤立导体板两侧都有电场分
布,两侧的验电羽都“飞”起来了.然后将两板慢慢靠近,让学生观察两板间及边缘的验电羽呈相互吸引状态,两外侧的验电羽随两板距离的靠近而逐渐垂下来的现象.这一现象的观察,有利于学生认识两靠近的(不是远离的)带电金属板间(不是外侧),除边缘附近外的中央区域的电场才是匀强电场.

同种或异种点电荷的电场,也可用同法演示,将静态与动态的演示结合起来,使学生能从物理现象的发展变化的过程中,去把握现象产生的条件.

\subsubsection{平行板电容器的电容跟哪些因素有关}
按图6.3装置好以后,用摩擦后的玻璃棒或用起电机使$A$板带上电荷(带电后,必须拆去与起电机连接的导线),就可以进行两板距离增大,两板正对面积减小,以及插入电介质后电容器的电容变化的演示.
\begin{figure}[htp]
    \centering
   \includegraphics[scale=.4]{fig/6-3.png}
    \caption{}
\end{figure}

对实验现象的解释,先要明确静电计为什么能测电势差.静电计本身相当于一个电容器,指针$C$(包括固定针、金属杆、金属球)相当于电容器的一个电极,外壳$D$相当于另一个电极,当可动指针位置变化时,整个电容器的电容变化极小,因而可
以认为静电计是一个电容不变的电容器,静电计带电时,可动指针与固定指针带同种电荷而相互排斥,使可动针张开一个角度,带电越多,张角越大.由$C=Q/U$可知,电容一定,带电越多则电压越大,因此,指针张角大小标志了指针与外壳间的电势差的大小,也标志着与之相连的$A$、$B$板间电势差的大小.

还可以更严密地考虑这一问题.静电计的指针与外壳构成一个电容器,其电容$C_{CD}$为定值,它与极板$A$、$B$连接后,两个电容器是并联的,即$C_{AB}$与$C_{CD}$并联,当$A$、$B$距离增大时,电荷由$A$移到$C$上,但$A$、$C$所带总电量是不变的,即$Q_{\text{并}}$为定值.由$C_{\text{并}}=Q_{\text{并}}/U_{\text{并}}$可知,当$A$、$B$板距离增大时,指针偏角增大说明$U_{\text{并}}$增大,故$C_{\text{并}}$必然减小.再由$C_{\text{并}}=C_{AB}+C_{CD}$可知,$C_{AB}$将随之减少,定性地说明平行板电容器的电容随板间距离增大而减小.同理,可以分析其余两种情况,当然,这些道理不一定都向学生讲解.

\subsubsection{静电除尘}

在一直径约5厘米、长约30厘米的玻璃筒的外面用铝芯线绕十几匝.另用一长将近30厘米的直裸导线,穿过一硬纸片的中心,让硬纸片刚好能盖在玻璃简上端口,使裸导线处于玻璃筒的轴线上,将中央裸导线和绕在外面的铝芯线的一端,分别接在感
应圈的两个电极上,将废橡胶点燃后吹灭,放在玻璃筒下端
口,使冒出的白烟上升到玻璃筒中,待白烟充满玻璃筒后,启动感应圈,筒中烟雾立即消失(图6.4).

\begin{figure}[htp]
    \centering
   \includegraphics[scale=.4]{fig/6-4.png}
    \caption{}
\end{figure}

\subsection{学生实验}
\subsubsection{电场中等势线的描绘}

这个实验是利用稳恒的电流场模拟静电场.当给$A$、$B$两极接上直流电源后,导电纸上就有稳恒电流流过,虽然电荷在不断的流动,但由于是稳恒电流,因而导电纸上的电荷分布是不随时间改变的,形成了一个稳恒电流场.稳恒电流场与静电场一样都是势场.但是,它们又有根本的区别,如维持
静电场不需要消耗任何其他形式的能量,而维持稳恒电流场必须有非静电力做功.

以上原理,不宜向学生讲解,但应让学生通过这个实验,了解模拟实验这样一种物理实验方法.

这个实验,用课本上介绍的方法很容易做成功,其中一些器材也很容易找到代用品,导电纸可以用誊印机上用的导电纸,也可以自制,如在白纸上均匀地涂抹上一层“铅笔芯末”,就做成了导电纸;用吸过食盐溶液或硫酸铜溶液的吸水纸或粗布片,也可代替导电纸.(在桌上铺的白纸,按顺序迭铺上复写纸、塑料薄膜、吸水纸,铺好后再往吸水纸上均匀地洒上足够的食盐水.)

电极可用图钉代替,先将电源线绕在图钉脚上,然后将图钉从铺好的导电纸、复写纸、白纸上按下去,直到图钉能牢固地固定在桌子上.

探针可用万用表笔代替,也可用固定在木条上的大头针代替.

电流表可用万用表的微安档代替,也可用伏特计代替.其方法是将一探针直接接电极,另一探针寻找各个使伏特计读数相同的点,这些点都是某一电势上的等势点.

\subsubsection{利用电容器放电测电容}

学生在初中没有接过分压电路,这是第一次在实验中用分压电路取得所需的电压值.实验前应指导学生对电路进行必要的分析,实验时接好电路、检查无误后,才能合上开关$K'$, 进行实验.

若有适合的直流电源,可以不用分压电路,按图6.5所示的电路做实验也是简便易行的.

\begin{figure}[htp]
    \centering
    \includegraphics[scale=.4]{fig/6-5.png}
    \caption{}
\end{figure}

充电电流和放电电流都满足
\[i=\frac{E}{R}e^{-\frac{t}{r}}\]的关系,故放电的
最大电流强度
\[I_m=\frac{E}{R}\approx \frac{12}{ 27\x10^3}=440{\rm mA}\]但由于表头指针及线圈的惯性,其最大示数将明显超过这个值,因此放电时应等指针返回到这个值后,再开始计时,同时
记取放电电流数值.

$i_c$-$t$图象中曲线与横轴间包围的面积,在数值上等于电
容器所带电量.计算面积时,可用数坐标小方格的办法来求数值(不到半格的舍去,超过半格算一格).

由于一般电解电容器的标称值本身误差常常很大,故不必求测量值的误差.

\subsection{课外实验活动}
\subsubsection{用自制的验电器做静电实验}

自制验电器不太难,关键在于瓶盖要有良好的绝缘性能.自制好验电器以后,可以做一些静电实验.

\begin{enumerate}
    \item 检验物体是否带电.验电器先不带电,把塑料尺接触瓶盖上的金属丝,下面的两条金属箔处于下垂状态,说明塑料尺不带电,将摩擦后的塑料尺再跟瓶盖上面的金属丝接触一下,若金属箔因此而张开,则说明摩擦后的塑料尺带了电;若不张开,则未带电.
    \item 检验物体带何种电荷.让验电器先带上已知的电荷(正电荷或负电荷),然后把需检验的带电体(如摩擦后的塑料尺)接近瓶盖上的金属丝,若验电器金属箔的张角增大,则被检验的电荷与验电器先带的电荷是同种电荷;若金属箔张角减小,则被检验电荷与验电器先带的电荷是异种电荷.    
    \item 检验物体是导体还是绝缘体.让物体接触带电的验电器,若其金属箔张角明显减小,则物体是导体;反之,是绝缘体.
    \item 感应起电.把带电体(如摩擦后的塑料尺)接近(不接
    触)不带电的验电器的金属丝,当接近到一定程度时,验电器的金属箔张开,说明此时验电器的金属丝和金属箔内的电荷在带电体电场作用下重新分布,发生了静电感应,用手指接触一下金属丝后,随即脱离,再移开带电体,金属箔仍然张开,这说明导体(在这里是验电器的金属丝和金属箔)由于静电感应而带电.
\end{enumerate}

    参照课本的有关内容,用自制验电器、金属小筒、金属网还可以做导体上电荷的分布、静电屏蔽等实验.

\section{习题解答}
\subsection{练习一}
\begin{enumerate}
    \item 把支在绝缘座上的不带电的导体$A$移近带电体$B$, 用手指接触一下$A$, 然后移开手指,握住绝缘座移开导体$A$, 导体$A$就带电了,如果带电体$B$原来带正电,导体$A$将带什么电?做这个实验并作出解释,实验时可用验电器来检查导体$A$是否带电和带什么电?
    
    \begin{solution}
        导体$A$带负电,因为在导体$A$移近带正电的物体$B$时,由于静电感应而在距$B$近的一端出现与带电体$B$异种的电荷即负电荷,距$B$远的一端出现与带电体$B$同种的电荷即正电荷.用手指接触导体时,导体$A$、人和大地连成一个导体.不管手指接触$A$的什么部位,对$B$而言,导体$A$始终是近端,大地始终是远端,由于静电感应,导体$A$带负电,大地带正电,所以,移开手指后,导体$A$上余下的净电荷是负电;再把 导体$A$移开,负电荷就分布在导体$A$上.
    \end{solution}
   
    \item 在课本图6.1中,先让验电器带上少量正电荷,然后拿一个带负电的带电体逐渐接近验电器的金属球,可看到这样的现象:金属箔张开的角度先是减小,以至闭合,然后又张开了.解释这个现象.
    
\begin{solution}
  验电器带少量正电荷时,下端金属箔带正电,相互排斥,呈一定张角.当带负电的导体球逐渐接近验电器上端的金属球时,由于静电感应而使电子向下端的金属箔移动.与原来的少量正电荷逐步中和,金属箔带正电荷减少以至为零,故金属箔的张角逐渐减小以至闭合;之后,随着金属箔感应负电荷的增多,张角又逐渐增大.   
\end{solution}
\end{enumerate}



\subsection{练习二}

\begin{enumerate}
	\item 1库仑的电量是电所带电量的多少倍?

    \begin{solution}
        每一个电子的电量$e=1.6\x10^{-18}$库.
        $Q=1$库时,它为$e$的倍数为
      \[  n=\frac{Q}{e}=\frac{1}{1.6\x10^{-18}}=6.25\x10^{18}{\text{(倍)}}\]
    \end{solution}
    
	\item 在真空中有两个点电荷,电量分别为$+4.0\x10^{-9}$库
和$-2.0\x10^{-9}$库,相距10厘米,这两个点电荷间的作用力
是多大?用电荷的绝对值代入进行计算,求出力的大小,然后
根据电荷的正负确定是引力还是斥力.

\begin{solution}
    由库仑定律公式计算两个点电荷间相互作用力的
    大小
\[F=\frac{kQ_1Q_2}{r^2}=\frac{9.0\x10^9\x4.0\x10^{-9}\x2.0\x10^{-9}}{(0.10)^2}=7.2\x 10^{-6}{\rm N}\]
由于两个电荷是异种电荷,它们之间的作用力是引力.
\end{solution}

\item 在真空中有两个点电荷,保持它们的距离不变,它们
间的相互作用力在下列情况下将如何变化?
\begin{enumerate}
	\item 一个电荷的电量变为原来的2倍;
	\item 两个电荷的电量都变为原来的1/2.
\end{enumerate}

\begin{solution}
    设原来两点电荷电量分别为$Q_1$、$Q_2$, 相距为$r$. 相互
    作用力$F=kQ_1Q_2/r^2$.
\begin{enumerate}
    \item 当$Q_1$、$r$不变,$Q'_2=2Q_2$时,$F\propto Q_2$,
    \[\frac{F'}{F}=\frac{Q'_2}{Q_2}=2\]
    即一个电荷变为原来的2倍时,相互作用力的大小为原
    来的2倍.
    \item 当$r$不变,$Q'_1=\frac{1}{2}Q_1$, $Q'_2=\frac{1}{2}Q_2$
    时,$F\propto Q_1Q_2$,
    \[\frac{F'}{F}=\frac{Q'_1Q'_2}{Q_1Q_2}=\frac{1}{2}\x \frac{1}{2}=\frac{1}{4}\]
    即两个电荷的电量都变为原来的1/2时,相互作用力的
    大小变为原来的1/4.
\end{enumerate}
\end{solution}

\item 原子核的半径大约为$10^{-14}$米,假定核中两个质子相距这么远,其间的静电力大约有多大?

\begin{solution}
    每个质子的带电量$Q=1.6\x10^{-19}$库.

    两个质子间的静电力
    \[F=\frac{kQ^2}{r^2}=\frac{9.0\x10^9\x(1.6\x10^{-19})^2}{(10^{-14})^2}=2.3{\rm N}\]
\end{solution}

\item 两个带电小球在煤油中相距0.5米,其中一个小球
带电$5.0\x10^{-9}$库,另一个带电$3.0\x10^{-9}$库,求小球间的作
用力.

\begin{solution}
    查得煤油的介电常数$\varepsilon=2$, 由电介质中的库仑定律公
    式
    \[F=\frac{kQ_1Q_2}{\varepsilon r^2}=\frac{9.0\x10^{-9}\x 5.0\x10^{-9}\x 3.0\x10^{-9}}{2\x (0.5)^2}=3\x 10^{-7}{\rm N}\]
    它们间的相互作用力为斥力.
\end{solution}

\item 当两个点电荷相距为$r$时,它们间的斥力为$F$.改
变电荷间的距离,当斥力为$16F$时,相距为多少?当斥力为
$\dfrac{1}{4}F$时,相距为多少?

\begin{solution}
    在电量不变时,由$F=kQ_1Q_2/r^2$可知,$F\propto 1/r^2$
    
    当$F_1=16F$时,
    \[\frac{F_1}{F}=\frac{r^2}{r^2_1}=16,\qquad r_1=\frac{r}{4}\]
    故当斥力为$16F$时,两电荷相距为$r/4$.
    
    当$F_2=F/4$时,
\[\frac{F_2}{F}=\frac{r^2}{r^2_2}=\frac{1}{4},\qquad r_2=2r\]
    故当斥力为$F/4$
    时,相距为$2r$.
\end{solution}

\end{enumerate}

\subsection{练习三}

\begin{enumerate}
	\item 在正电荷$Q$的电场中的某一点放一个电荷,它的电
	量$q=10^{-8}$库,$q$受到的电场力为$10^{-8}$牛.求这一点的电场
	强度$E$,并指出电场强度的方向.如果取走$q$,$E$有无变化?
	为什么?

    \begin{solution}
        这点的场强
\[E=\frac{F}{q}=\frac{10^{-8}}{10^{-8}}=1{\rm N/C}\]
其方向在$Q$、$q$的连线上,离开$Q$而去.

取走$q$, 场强$E$无变化.因为场强是由场源电荷$Q$和该点
离$Q$的距离$r$决定的,与检验电荷无关.
    \end{solution}
    
	\item 在以水为介质的负点电荷$Q$的电场中,离$Q$0.5米处
	的电场强度$E$是1$\NC$,求负电荷Q的电量是多少库.

    \begin{solution}
        查得水的介电常数$\varepsilon=81$, 由电介质中的点电荷的
        场强公式$E=\dfrac{kQ}{\varepsilon r^2}$得
   \[     Q=\frac{\varepsilon r^2 E}{k}=\frac{81\x(0.5)^2\x1}{9.0\x10^9}
 =2\x10^{-9}{\rm C}\]
    \end{solution}
    
	\item 电场中某点的场强是$0.2\x10^5\NC$.求电量为
	$2\x10^{-8}$库的正电荷在该点受到的电场力是多大.

    \begin{solution}
        由场强的定义式$E=F/q$,得:
 \[       F=qE=2\x10^{-8}\x0.2\x10^5=4\x10^{-4}{\rm N}\]
    \end{solution}
    
	\item 在氢原子中,电子和质子的平均距离是$5.3\x10^{-11}$
	米.质子在这个距离处产生的场强是多大?方向如何?电子
	受到的力是多大?方向如何?

    \begin{solution}
        在氢原子中,质子和电子的电量大小均为$e=1.6\x10^{-19}$
        库.由点电荷的场强公式,质子在电子轨道半径处的场
        强为
\[E=\frac{ke}{r^2}=\frac{9.0\x 10^9\x 1.6\x 10^{-19}}{(5.3\x 10^{-11})^2}=5.1\x 10^{11}{\rm N/C}\]
其方向在质子与该点的连线上,离质子而去.

电子受到的电场力
\[F=eE=1.6\x10^{-18}\x5.1\x10^{11}=8.2\x10^{-8}{\rm N}\]
其方向在电子与质子连线上,指向质子.
    \end{solution}
    
	\item 物理学上常把重力作用的空间叫做\textbf{重力场}.如果把
	单位质量的物体受到的重力叫做重力场强度,试写出重力场
	强度的定义式.重力场强度的方向如何?从重力场强度的方
	向来看,重力场是跟正电荷形成的电场相似,还是跟负电荷形
成的电场相似?

\begin{solution}
    重力场强度的定义式$g=G/m$, 其方向与重力的方向
    相同.重力场跟负电荷的电场相似.
\end{solution}

\item 电场强度的定义式$E=F/q$在电介质中要不要改
写?为什么?

\begin{solution}
    不需要改写.因为$E=F/q$是定义式,它是普遍适用
    的.事实上,公式$F=\dfrac{kQ_1Q_2}{\varepsilon r^2}$中含有介电常数,已经考虑
    了电介质的存在.
\end{solution}

\end{enumerate}


\subsection{练习四}

\begin{enumerate}
	\item 有人说电力线就是带电粒子在电场中运动的轨迹.这种说法对吗?为什么?

    \begin{solution}
不对.电力线是为形象地表示电场特性而人为地引
人的曲线.曲线上每点的切线方向都跟该点的场强方向一
致,也与放在该点的正电荷的受力方向一致(与负电荷受力方
向相反).但电荷的运动轨迹不但与它的受力大小有关,还与
它的初速度有关,即使受力情况相同,如初速度不同,其运动
轨迹也是不同的.因此,电力线与运动轨迹是两回事.只有
当电力线为直线,且带电粒子初速度方向也在这直线上或初
速为零时,其运动轨迹才恰好与电力线重合.
    \end{solution}
    
	\item 在课本图6.9到图6.11中,所有的电力线都不相交,我们能否断言,电场中任何两条电力线都不相交,为什么?

    \begin{solution}
        能断言.因为在电场中任意一点的场强方向只有一
        个;如果有两条电力线在电场中的某一点相交,则说明交点的
        场强有两个方向,这实际上是不可能的.
    \end{solution}
    
\end{enumerate}


\subsection{练习五}

\begin{enumerate}
	\item 把两个异种电荷的距离增大一些,电场力做正功还
是做负功?电势能是增加还是减小?把两个同种电荷的距离增
大一些,情况又怎样?

\begin{solution}
    把两个异种电荷的距离增大一些,电场力做负功,电
    势能增加,把两个同种电荷的距离增大一些,电场力做正功,
    电势能减小.
\end{solution}

\item 在课本图6.16中,把负电荷$-q$放在$A$、$B$点,它在哪一点
的电势能较大?无限远处的电势能为零,负电荷$-q$在这个
电场中的电势能是正值还是负值?

\begin{solution}
    在课本图6.16中,$-q$放在$B$点的电势能比放在$A$
    点的大;取无限远处的电势能为零时,$-q$在这电场中的电势
    能为负值.
\end{solution}

\item 在课本图6.17中,把负电荷$-q$放在$C$、$D$点,它在哪一点的电势能较大?取无限远处的电势能为零,负电荷$-q$在这
个电场中的电势能是正值还是负值?

\begin{solution}
    在课本图6.17中,$-q$放在$C$点的电势能比放在$D$
    点的大;取无限远处的电势能为零时,$-q$在电场中的电势能
    为正.
\end{solution}

\item 在图6.6所示的电场中,如果把正电荷$q$由$N$点移到$M$点,$q$的电势能增加还
是减小?如果移动的是负电荷$-q$,电势能又怎样变化?
\begin{figure}[htp]\centering
    \begin{tikzpicture}[>=latex]
\foreach \x in {1,2,3}
{
    \draw[->] (0, \x)--(4,\x);
\node at (1, 2.4){$M$};  \node at (3, 2.4){$N$};
\fill (1,2.1) circle (1pt); \fill (3,2.1) circle (1pt);
}
        
    \end{tikzpicture}

    \caption{}
\end{figure}	


\begin{solution}
    在图6.6中,把正电荷由$N$移到$M$, 电势能增加;如果
    移动的是负电荷,电势能减小.
\end{solution}

\item  电子在原子核附近运动时,电子的电势能是正值还
是负值?取无限远处的电势能为零,把这个电子由原子核附
近移到无限远处,电子的电势能是增加还是减小?

\begin{solution}
    电子在原子核附近的电势能为负;把它移到无穷远处
    电势能增加.
\end{solution}

\end{enumerate}



\subsection{练习六}

\begin{enumerate}
	\item 电场中A点的电势是3伏,求;
	\begin{enumerate}
		\item 电量为5库的电荷在$A$点的电势能;
		\item 电量为10库的电荷在$A$点的电势能;
		\item 电量为$-5$库的电荷在$A$点的电势能;
		\item 电量为$-10$库的电荷在$A$点的电势能.
	\end{enumerate}

    \begin{solution}
\begin{enumerate}
    \item 由$U=\mathcal{E}/q$, 得$\mathcal{E}=qU=5\x3=15{\rm J}$;
    \item 同理,$\mathcal{E}=qU=10\x3=30{\rm J}$;
    \item 同理,$\mathcal{E}=qU=-5\x3=-15{\rm J}$;
    \item 同理,$\mathcal{E}=qU=-10\x3=-30{\rm J}$.
\end{enumerate}
    \end{solution}
    
	\item 在课本图6.16中$A$、$B$两点哪一点电势高?在图6.17中
$C$、$D$两点哪一点的电势高?说明理由.


\begin{solution}
    在课本图6.16中,$A$点的电势高;在课本图6.17中,$D$点的电势高.因为电力线的方向分别是由$A$到$B$和由$D$
    到$C$, 而电势是顺着电力线的方向降低的.
\end{solution}

\item 在图6.7所示的匀强
电场中,如果$A$板是接地的,
$M$、$N$两点哪点电势高?电势是
正值还是负值?如果$B$板是接
地的,结果又怎样?取大地的电
势为零.

\begin{figure}[htp]\centering
    \begin{tikzpicture}[>=latex]
\foreach \x in {0,2}
{
    \draw[very thick] (0,\x) --(5,\x);
}

\node at (5.25,0){$B$};
\node at (5.25,2){$A$};

\foreach \x in {1, 2,3,4}
{
    \draw[->, dashed](\x,2)--(\x,0);
}   
\node at (2.5,1.5){$M$};    \node at (2.5,.5){$N$};
\fill (2.2,1.5) circle (1.5pt);\fill (2.2,.5) circle (1.5pt);
    \end{tikzpicture}
    \caption{}
\end{figure}	


\begin{solution}
    $A$板接地时,$M$点电势高,电势为负;$B$板接地时,仍
    是$M$点电势高,但电势为正.
\end{solution}

\item 一个初速度为零的正电荷放在电场中,只在电场力
作用下,它向电势高的地方跑还是向电势低的地方跑?一个初
速度为零的电子放在电场中,它向电势高的地方跑还是向电
势低的地方跑?说明理由.

\begin{solution}
    初速度为零的物体,运动方向与受力方向相同.正电
    荷受电场力的方向,就是场强方向,即电势降低的方向.所
    以,初速度为零的正电荷放在电场中,向着电势低的地方跑.
    初速度为零的电子放在电场中,所受电场力方向与场强方向
    相反,即电子的运动方向与场强方向相反,它向着电势高的地
    方跑.
\end{solution}

\item 一个初速度为零的电荷放在电场中,不论是正电荷
还是负电荷,都向着电势能低的地方跑,试说明理由.

\begin{solution}
    因为初速度为零的电荷在电场中移动时,必然是电场力做正功,电荷的电势能总是减少的,所以都是向着电势低的
    地方跑.
\end{solution}

\item 电场中某点的电势是否跟检验电荷的正负有关?讨
论一下这个问题.

\begin{solution}
    跟检验电荷的正负无关.因为势是由场源电荷决
    定的.若某点电势为$U$, 则$+q$电荷在该点的电势能为$+qU$,
    它与$+q$的比值为$U$; 而$-q$在该点的电势能为$-qU$, 它与$-q$
    的比值仍为$U$.
\end{solution}

\item 电场中两个电势不同的等势面能不能相交?为什么?

\begin{solution}
    电场中两个电势不同的等势面不能相交,因为在电
    场中一个确定的点,只有一个确定的电势,若有两个电势不同
    的等势面相交,则在交线上的点,就有两个电势值,这是不符
    合实际的.
\end{solution}

\end{enumerate}



\subsection{练习七}

\begin{enumerate}
	\item 把带电体从电势为300伏的$A$点移到电势为100伏
的$B$点,电场力做了$3.0\x10^{-8}$焦的负功,带电体带哪种电
荷?电量是多少?

\begin{solution}
    由$A$指向$B$的方向是电势降低的方向,即为电场的
    方向.在这个方向上移动带电体电场力作负功,说明电场力
    方向与电场方向(即带电体移动方向)相反,因此,带电体带
    负电荷.

    由$U_{AB}=W/q$,得:
    \[q=\frac{W}{U_{AB}}=\frac{W}{U_A-U_B}=\frac{3.0\x 10^{-6}}{300-100}=1.5\x 10^{-8}{\rm C}\]
    带电体的电量为$1.5\x 10^{-8}$库.
\end{solution}

\item 电场中$M$、$N$两点的电势$U_M=800$伏、$U_N=-200$
伏,把电量是$1.5\x10^{-8}$库的负电荷从$M$点移到$N$点,电场力
做了多少功?做正功还是负功?

\begin{solution}
    电场力做负功,其大小
  \[  W=qU_{MN}=q(U_M-U_N)=1.5\x10^{-8}\x[800-(-200)]=1.5\x10^{-5}{\rm J}\]
\end{solution}

\item 在电场中把电量为$2.0\x10^{-8}$库的正电荷从$A$点移
到$B$点,电场力做了$1.5\x10^{-7}$焦的正功,再把这个正电荷从
$B$点移到$C$点,电场力做了$4.0\x10^{-7}$焦的负功、$A$、$B$、$C$三点
中哪点的电势最高,哪点的电势最低?$A$、$B$间,$B$、$C$间和$A$、$C$
间的电势差各是多大?

\begin{solution}
    将正电荷$q$由$A$点移到$B$点时,电场力做正功,
   \[ U_{AB}=\frac{W_1}{q}=\frac{1.5\x 10^{-7}}{2.0\x 10^{-9}}=75{\rm V},\qquad U_A>U_B\]
    将它由$B$点移到$C$点时,电场力作负功,
    \[ U_{CB}=\frac{W_2}{q}=\frac{4.0\x 10^{-7}}{2.0\x 10^{-9}}=200{\rm V},\qquad U_C>U_B\]  
    由以上结果比较得出:$A$、$B$、$C$三点中$C$点电势最高,$B$
    点电势最低.$U_{AB}=75$伏,$U_{BC}=-U_{CB}=-200$伏,而
\[\begin{split}
    U_{AC}=U_A-U_C=(U_A-U_B)+(U_B-U_C)&=U_{AB}+U_{BC}\\
    &=75+(-200)=-125{\rm V}
\end{split}\]
\end{solution}

\item 一个原来静止的电子,从电场中的$A$点被加速移到
$B$点.$A$、$B$两点间的电势差是2000伏,电场力所做的功是多
少电子伏?电势能的变化是多少电子伏?设电子是在真空中
移动的,电子在$B$点获得的动能是多少电子伏?

\begin{solution}
    电场力做功$W=qU_{AB}=1\x2000=2000{\rm eV}$

    加速电子时,电场力做正功,而电势能的减少等于电场力
    做的正功,故电势能减少了2000电子伏.

    在真空中电子只受电场力,由动能定理$W=\Delta E_k$,电子
    获得的动能等于电场力做的功,也为2000电子伏.
\end{solution}

\end{enumerate}


\subsection{练习八}

\begin{enumerate}
	\item 两块相距0.05米的带电平行板之间的电场是匀强
	电场,两板的电势差为$10^4$伏.求作用在两板之间的一个电
	子上的电场力.

    \begin{solution}
        作用在电子上的电场力
\[F=qE=q\cdot \frac{U}{d}=1.6\x 10^{-19}\x \frac{10^4}{0.05}=3\x 10^{-14}{\rm N}\]
    \end{solution}
    
	\item 平行的带电金属板$A$、$B$间是匀强电场(图6.8),场
	强为$1.2\x10^8\NC$.两板间的距离为5厘米,两板间的电
	势差有多大?电场中有两点$P_1$和$P_2$,$P_1$点离$A$板的距离是
	0.5厘米,$P_2$点离$B$板的距离也是0.5厘米.$P_1$和$P_2$两点间
	的电势差有多大?

    \begin{figure}[htp]\centering
        \begin{tikzpicture}[>=latex]
    \draw (0,0)--(6,0)node [right]{$B$};
    \draw (0,3)--(6,3)node [right]{$A$};        
    
    \foreach \x in {.5,1.5,...,5.5}
    {
        \node at (\x, 0.25){$+$};
        \node at (\x, 2.75){$-$};
    }
    
    \fill (2,2.7) circle (2pt);
    \fill (5,.3) circle (2pt);
    \node at (2,2+.4){$P_1$};
    \node at (5,1-.4){$P_2$};
    
        \end{tikzpicture}
        \caption{}
    \end{figure}	
    \begin{solution}
两板间的电势差
\[U=Ed=1.2\x10^3\x0.05=60{\rm V}\]
$P_1$、$P_2$之间在场强方向上的距离
\[d'=0.05-2\x0.005=0.04{\rm m}\]
$P_1$、$P_2$之间的电势差
\[U'=E\cdot d'=1.2\x10^3\x0.04=48{\rm V}\]
    \end{solution}
    
\end{enumerate}


\subsection{练习九}
\begin{enumerate}
	\item 在真空中有一对平行金属板,相距6.2厘米,加上90
伏的电压,两价的氧离子从静止出发被加速,从一板到达另
一板时,它的动能是多大?

这道题有几种解法?哪种解法比较简便?

\begin{solution}
\begin{enumerate}
    \item 用牛顿运动定律求解.
    \[a=\frac{F}{m}=\frac{qE}{m}=\frac{qU}{md}\]
    因为 $U_0=0$, $U^2=2ad=2qU/m$

    所以
    \[E_k=\frac{1}{2}mU^2=\frac{1}{2}m\cdot \frac{2qU}{m}=qU\]
    \item 由动能定理求解.
   \[ W=\Delta E_k=E_k-E_{k0}\]
   因为$U_2=0$, $E_{k0}=0$; 而$W=qU$, 所
    以$E_k=qU$. 
\end{enumerate}
代入数据,算得
    \[E_k=2\x1.6\x10^{-18}\x90=2.9\x10^{-17}{\rm J}\]
    由上面的两种方法比较可知,用动能定理求解较为简便.
\end{solution}

\item 两价离子在90伏的电压下从静止加速后,测出它的
动量是$1.24\x10^{-21}{\rm kg}\cdot \ms$,这种离子的质量是多大?


\begin{solution}
    由动能定理$W=\Delta E_k$, 又$U_0=0$, 得$E_k=W=qU$.
    而$E_k=\dfrac{p^2}{2m}$, 所以
 \[\begin{split}
     m=\frac{p^2}{2E_k}&=\frac{p^2}{2qU}\\
     &=\frac{(1.24\x10^{-21})^2}{2\x 2\x1.6\x10^{-19}\x 90}\\
     &=2.7\x10^{-26}{\rm kg}
 \end{split}\]   
\end{solution}

\item 经1000伏加速电压加速后的电子,沿着与电场垂直
的方向进入匀强偏转电场,场强为5000$\NC$.已知偏转电
极长为6厘米,求电子离开偏转电场时的速度.

\begin{solution}
    经电压$U$加速后的电子速度$U_0$满足$eU=\frac{1}{2}mU^2_0$的
    关系,由此得
 \[   U_0=\sqrt{\frac{2eU}{m}}\]
    通过偏转电场的时间
    \[t=\frac{\ell}{v_0}=\ell\sqrt{\frac{m}{2eU}}\]
    在离开偏转电场时的侧向速度
\[    v_{\bot}=at=\frac{eE}{m}\cdot \ell\sqrt{\frac{m}{2eU}}=\ell E\sqrt{\frac{e}{2mU}}\]

    离开偏转电场时电子的速度大小
   \[\begin{split}
      v&=\sqrt{v^2_0+v^2_{\bot}}=\sqrt{\frac{2eU}{m}+\ell^2E^2\frac{e}{2mU}}\\
&=\sqrt{\frac{e}{m}\cdot \frac{4U^2+\ell^2E^2}{2U}}\\
&=\sqrt{\frac{1.6\x 10^{-19}}{9.1\x 10^{-81}}\cdot \frac{4\x 1000^2+0.06^2\x 5000^2}{2\x 1000}}\\
&=1.9\x 10^7\ms
   \end{split}\]

    这时速度方向与水平方向的夹角的正切值
    \[\tan\phi=\frac{v_{\bot}}{v_0}=\ell E\frac{\sqrt{\dfrac{e}{2mU}}}{\sqrt{\dfrac{2eU}{m}}}=\frac{\ell E}{2U}=\frac{0.06\x 5000}{2\x 1000}=0.15 \]
    查表得$\phi=8^{\circ}32'$.
\end{solution}

\item 计算一下本节例题中的电子离开偏转电场时侧向移
动的距离.

\begin{solution}
    电子离开偏转电场时的侧向移动距离
    \[\begin{split}
 y=\frac{1}{2}at^2 &=\frac{1}{2}\cdot \frac{eU}{md}\cdot \left(\frac{\ell }{v}\right)^2\\
 &=\frac{1}{2}\x \frac{1.6\x 10^{-19}\x 2.0}{9.1\x 10^{-31}\x 2\x 10^{-3}}\cdot \left(\frac{0.060}{3.0\x 10^7}\right)^2\\
 &=3.5\x 10^{-4}{\rm m}      
    \end{split}\]
\end{solution}

\item 图6.9所示的实验装置可以用来验证电场对带电
粒子的加速作用只跟电压有关.左边的非匀强电场使电子加
速,右边的匀强电场使电子减
速,设非匀强电场的电压为$U$,
匀强电场的电压为$U'$.实验结
果是:只要$U'<U$,电流计的指
针就偏转;只要$U'>U$,电流计
的指针就不偏转.你从这个实
验结果可以得出什么结论?
\begin{figure}[htp]\centering
	\includegraphics[scale=.8]{fig/6-9.png}
	\caption{}
	\end{figure}

    \begin{solution}
        在图6.9的实验中,左边的非匀强磁场使电子加速,把电子的电势能转化为动能,右边的匀强电场,使电子减速,
        把电子的动能转化为电势能.当加速电压$U>U'$时,电流计
        中有电流通过,这说明电子在加速电场中获得的动能能使它
        通过减速电场,即大于它通过减速电场减少的动能,当加速
        电压$U<U'$时,电流计中没有电流通过,这说明电子在加速电
        场中获得的动能不足以使它通过减速电场,即小于一个电子
        通过减速电场时应减少的动能.所以,这个实验说明,电场对
        带电粒子的加速作用只跟加速电压有关.
    \end{solution}
    
\end{enumerate}



\subsection{练习十}
\begin{enumerate}
	\item 电容器带电后电势差增大的情形,跟物体吸收热量后温度升高的情形也很相似,试对这两种现象作一比较.

    \begin{solution}
        电容器带电后电势差的增量,相当于物体吸热后温度
        升高的度数;其所带的电量,相当于物体吸收的热量;电容器
        的电容,相当于物体的质量与构成这个物体的物质的比热的
        乘积.
    \end{solution}
    
	\item 一个由圆板制成的平行板电容器,圆板的半径为3.0
厘米,两板的距离为2.0毫米,中间充满介电常数为6.0的电
介质,这个电容器的电容是多少?

\begin{solution}
    平行板电容器的电容
\[\begin{split}
    C=\dfrac{\varepsilon S}{4\pi kd}=\dfrac{\varepsilon\cdot  \pi r^2}{4\pi kd}=\frac{\varepsilon r^2}{4kd}
    &=\frac{6.0\x (3.0\x 10^{-2})^2}{4\x 9\x 10^9\x 2.0\x 10^{-3}}\\
    &=7.5\x 10^{-11}{\rm F}=75{\rm pF}
\end{split}\]
\end{solution}

\item 一个电容器的电容是$1.5\x10^{-2}$微法,把它的两个极
板接在90伏的电源上,求每个极板上的电量.

\begin{solution}
    由$C=Q/U$推得,每个极板上的电量
   \[ Q=CU=1.5\x10^{-2}\x10^{-6}\x90=1.35\x10^{-6}{\rm C}\]
\end{solution}

\item 有一个电容器,在带了电量$Q$以后,两导体间的电势
差是$U$,如果使它带的电量增加$4.0\x10^{-8}$库,两导体间的电
势差就增大20伏,这个电容器的电容是多少微法?

\begin{solution}
    电容器增加电量$\Delta Q$前、后有
    \begin{align}
    C&=\frac{Q}{U}\\
    C&=Q+\frac{\Delta Q}{U}+\Delta U    
    \end{align}
    由(6.1)、(6.2)联立可得
\begin{equation}
    \frac{Q}{U}=\frac{\Delta Q}{\Delta U}
\end{equation}
    将(6.3)式代入(6.1)式中,有
\[C=\frac{\Delta Q}{\Delta U}=\frac{4\x10^{-8}}{20}=2\x10^{-9}{\rm F}=2\x10^{-3}{\rm \mu F}\]


\end{solution}

\item 如图6.39所示,闭合
电键$K$使平行板电容器$C$充
电,然后断开电键,当增大电容
器两板间的距离时,下述各量
是否改变,怎样改变?
\begin{enumerate}
	\item 电容器所带电量;
	\item 电容器的电容;
	\item 电容器两板间的电势差.
\end{enumerate}

\begin{solution}
\begin{enumerate}
    \item 由于与电源断开了,电容器所带电量不变;
    \item 由
    $C=\dfrac{\varepsilon S}{4\pi kd}$可知,$d$变大,其余量不变时,则电容器的电容$C$变小;
    \item 又由$C=Q/U$推出$U=Q/C$可得,当$Q$不变时,$C$变
    小,则$U$变大.
\end{enumerate}
\end{solution}

\item 在上题中,充电后如果保持电键$K$闭合,那么,增大
电容器两板间的距离时,下述各量是否改变,怎样改变?
\begin{enumerate}
	\item 电容器两板间的电势差;
	\item 电容器的电容;
	\item 电容器所带的电量.
\end{enumerate}

\begin{solution}
\begin{enumerate}
    \item 由于电键闭合,电容器与电源相连接,其两极板间
    的电势差始终等于电源电动势,保持不变;
    \item 由$C=\dfrac{\varepsilon S}{4\pi kd}$
    可知,$d$变大,其余各量不变时,$C$变小;
    \item 由公式$C=Q/U$推
    出$Q=CU$可知,在$U$不变时,$C$变小,则极板带电量$Q$也
    变小.
\end{enumerate}
\end{solution}

\end{enumerate}



\subsection{练习十一}

\begin{enumerate}
\item 两个相同的电容器,标有“100皮法、600伏”,串联后
接到900伏的电路上,每个电容器带多少电?加在每个电容器
上的电压是多少?电容器是否会击穿?

\begin{solution}
    电容相同的两个电容器串联,每个电容器两极板间的
    电压相等,又$U=U_1+U_2$, 所以
   \[ U_1=U_2=U/2=900/2=450{\rm V}\]
 
   每个电容器的带电量
   \[Q_1=Q_2=CU/2=1.0\x10^{-10}\x450=4.5\x10^{-8}{\rm C}\]
    由于每个电容器所承受的实际电压为450伏,小于额定
    电压600伏,故不会被击穿.
\end{solution}

\item 把“100皮法、600伏”和“300皮法、300伏”的电容器
串联后接到900伏的电路上,这样连接是否合适?为什么?

\begin{solution}
    串联电容器的总电容
\[    C=\frac{C_1C_2}{C_1+C_2}=\frac{100\x 300}{100+300}=75.0{\rm pF}\]
    每个电容器所带的电量
\[    Q_1=Q_2=Q=CU=7.5\x10^{-11}\x900=6.75\x10^{-8}{\rm C}\]
    第一个电容器极板间的实际电压
    \[U_1=\frac{Q_1}{C_1}=\frac{6.75\x10^{-8}}{100\x10^{-12}}=675{\rm V}\]
它大于
这个电容器的额定电压600伏,所以先被击穿.

第二个电容器极板间的实际电压
\[U_2=U-U_1=900-675=225{\rm V}\]
虽然它小于这个
电容器的额定电压;但当第一个电容器被击穿之后,900伏电
压全部加在第二个电容器上,以致大于它的额定电压300伏,
所以第二个电容器随后也被击
穿.因此,这样的连接不合适.
\end{solution}

\item 平行板电容器的正对面积为$S$,
两板距离为$\ell$,电介质是真空.如果在两
板之间插入一厚度为$d$的金属板(图6.10),试证明它的电容为
\[C=\frac{S}{4\pi k(\ell-d)}\]
\begin{figure}[htp]\centering
    \begin{tikzpicture}[>=latex]
\draw[ultra thick] (-0.5,0)--(0,0);
\draw[ultra thick] (1,0)--(1.5,0);
\draw [ultra thick] (0,-1.8)--(0,1.8);
\draw[ultra thick]  (1,-1.8)--(1,1.8);
\fill [pattern=north east lines, draw ](.4,-1.8) rectangle (.6,1.8);
\draw (0,-2)--(0,-2.5);\draw (1,-2)--(1,-2.5);
\draw [<->](0,-2.25)--node[fill=white]{$\ell$}(1,-2.25);
\draw (.4,2)--(.4,2.5);\draw (.6,2)--(.6,2.5);
\draw [->](0,2.25)--(.4,2.25);
\draw [->](1,2.25)--(.6,2.25);
\node at (.5,2.5){$d$};
    \end{tikzpicture}

    \caption{}
\end{figure}

\begin{proof}
    插入导体板后,相当于板间距离为$d_1$、$d_2$的两个平
    行板电容器串联.它们的电容分别为
    $C_1=\dfrac{S}{4\pi kd_1}$和 $C_2=\dfrac{S}{4\pi kd_2}$, 串联电容器的电容$C=\dfrac{C_1C_2}{C_1+C_2}$,将上式代入,得
    \[C=\frac{S}{4\pi k(d_1+d_2)}\]
    因为$d_1+d_2=\ell-d$, 所以$C=\dfrac{S}{4\pi k(\ell-d)}$.
\end{proof}

\item 电容分别为20微法和50微法的
两个电容器并联后,接在电压为100伏的
电路上,它们共带多少电?

\begin{solution}
    并联电容器的电容$C=C_1+C_2=20+50=
    70{\rm \mu F}$.所带的总电量
    \[Q=CU=70\x10^{-6}\x100=7.0\x  10^{-3}{\rm C}\]
\end{solution}

\item 电容为 3000 皮法的电容器带电
$1.8\x10^{-6}$库后,撤去电源,再把它跟电容为1500皮法的电容
器并联,求每个电容器所带电量.

\begin{solution}
    3000皮法的电容器原来所带电量,即为并联电容器
    所带的总电量$Q=1.8\x10^{-6}$库,而并联电容器的总电容$C=
    C_1+C_2=3000+1500=4500{\rm pF}$

    每个电容器两极板间的电压
    \[U_1=U_2=U=\frac{Q}{C}=\frac{1.8\x10^{-6}}{4500\x10^{-12}}=4.0\x 10^2{\rm V}\]
    
    3000皮法的电容器带电量
    \[Q_1=C_1U_1=3000\x10^{-12}\x4.0 \x10^2=1.2\x10^{-6}{\rm C}\]
    1500皮法的电容器带电量
    \[Q_2=Q-Q_1=1.8\x10^{-6}-1.2 \x10^{-6}=0.6\x10^{-6}{\rm C}\]
\end{solution}

\end{enumerate}





\subsection{习题}
\begin{enumerate}
\item 有一个绝缘的金属筒,上面开个小孔,通过小孔放入
一个悬挂在丝线上的带正电的小球,在下列各种情况里,金
属筒外壁各带什么电荷?
\begin{enumerate}
\item 小球跟筒的内壁不接触;
\item 小球跟筒的内壁接触;
\item 小球跟筒的内壁不接触,但用手指接
触一下金属筒,然后移开手指,再把小球移出筒外.
\end{enumerate}


\begin{solution}
\begin{enumerate}
    \item 由于静电感应,远离带正电小球的金属筒外壁带正
    电;
    \item 原来带正电的小球与金属筒成为一体,小球所带正电
    荷分布在导体(金属筒)表面,故金属筒外壁带正电;
    \item 由于感
    应接地起电,金属筒带负电,且分布在筒的外壁.
\end{enumerate}
\end{solution}

\item 有两个带电小球,电量分别为$+Q$和$+9Q$,在真空
中相距0.4米.如果引进第三个带电小球,正好使三个小球都
处于平衡状态,第三个小球带的是哪种电荷?应放在什么地
方?电量是$Q$的几倍.

\begin{solution}
根据引入第三个小球后各电荷的变力情况判断,若使
三个小球都可能平衡,第三个小球应带负电荷.放在$+Q$与
$+9Q$的连线之间,设它的电量为$Q_x$, 离$+Q$距离为$x$(图6.11)
\begin{figure}[htp]
    \centering
\begin{tikzpicture}[>=latex]
\tkzDefPoints{0/0/A, 2/0/B, 6/0/C}
\tkzDrawPoints(A,B,C)
\draw[<->, thick] (0-.6,0)node[above]{$F_3$}--(0+.6,0)node[above]{$F_1$};
\draw[<->, thick] (2-.6,0)node[above]{$F'_1$}--(2+.6,0)node[above]{$F_2$};
\draw[<->, thick] (6-.6,0)node[above]{$F'_2$}--(6+.6,0)node[above]{$F'_3$};
\tkzLabelPoint[below](A){$+Q$}
\tkzLabelPoint[below](B){$Q_x$}
\tkzLabelPoint[below](C){$+9Q$}
\draw[dashed](3,0)--(5,0);
\end{tikzpicture}
    \caption{}
\end{figure}

$+Q$与$Q_x$间的作用力$F_1$、$F'_1$大小相等
\begin{equation}
   F_1=F_1'=\frac{kQQ_x}{x^2} 
\end{equation}
$+9Q$与$Q_x$间的作用力$F_2$、$F_2'$大小相等
\begin{equation}
    F_2=F_2'=\frac{9kQQ_x}{(\ell-x)^2}
\end{equation}
$+Q$与$+9Q$间的作用力$F_3$, $F_3'$大小相等
\begin{equation}
    F_3=F_3'=\frac{9kQ^2}{\ell^2}
\end{equation}

若要三个点电荷都处于平衡状态,则有$F_1=F_3$、$F_1'=F_2$、
$F_2=F_3'$.

由$F_1=F_3$得
\begin{equation}
    \frac{kQQ_x}{x^2} =\frac{9kQ^2}{\ell^2} \quad \Rightarrow\quad Q_x=9Q\frac{x^2}{\ell^2}
\end{equation}
由$F_2'=F_3'$得
\begin{equation}
    \frac{9kQQ_x}{(\ell-x)^2}=\frac{9kQ^2}{\ell^2}  \quad \Rightarrow\quad Q_x=Q\frac{(\ell-x)^2}{\ell^2}
\end{equation}
由(6.7)、(6.8)两式得:$9x^2=(\ell-x)^2$
\[x=\frac{\ell}{4}=\frac{0.4}{4}=0.1{\rm m}\]
再代入(6.7)式中,
\[Q_x=9Q\cdot \left(\frac{0.1}{0.4}\right)^2=\frac{9}{16}Q\]
\end{solution}

\item 如图6.12所示,有两
个挂在丝线上的小球,带有等
量的同种电荷,由于电荷彼此
推斥,丝线都偏离竖直线$\theta$角,
已知两小球的质量都为$m$,两丝线长都为$\ell$,求每个小球上所带的电量.

\begin{figure}[htp]
\centering
\begin{minipage}[t]{0.48\textwidth}
\centering
\begin{tikzpicture}[scale=.7]
	\draw (0,0)--(2,-5);
	\draw (0,0)--(-2,-5);	
	\draw [dashed](0,0)--(0,-6);
	\draw  (2,-5)[fill=gray] circle (5pt);
	\draw  (-2,-5)[fill=gray] circle (5pt);
	\node at (-1.5, -2.5){$\ell$};	\node at (1.5, -2.5){$\ell$};
	\node at (-2.5, -5){$m$};	\node at (2.5, -5){$m$};
	\draw (0,-1) arc (-90:-112:1) node [below]{$\theta$};
	\draw (0,-.9) arc (-90:-68:.9) node [below]{$\theta$};
\end{tikzpicture}
\caption{}
\end{minipage}
\begin{minipage}[t]{0.48\textwidth}
\centering
\begin{tikzpicture}[>=latex]
\tkzDefPoints{0/0/O, -1/-3/A, 1/-3/B, 0/-2/C}
\draw(O)--node[left]{$\ell$}(A);
\draw(O)--node[right]{$\ell$}(B)node[right]{$m$};
\draw[dashed](O)--(C);
\tkzDrawPoints(A,B)
\draw[<->, thick](-1.5,-3)node[above]{$F$}--(A)--(-1,-4.5)node[right]{$mg$};
\draw[dashed](-1.5,-3)--  (-1.5,-4.5)--(-1,-4.5);
\draw[dashed,->](A)--(-1.5,-4.5)node[left]{$T'$};
\draw[thick,->](A)--(-.5,-1.5)node[right]{$T$};
\draw (0,-1) arc (-90:-108:1) node [below]{$\theta$};
\draw (0,-.9) arc (-90:-72:.9) node [below]{$\theta$};

\end{tikzpicture}
\caption{}
\end{minipage}
\end{figure}

\begin{solution}
    小球在重力、库仑力和线的拉力作用下平衡.由共点
    力的平衡条件,$mg$与库仑力$F$的合力$T'$与线的拉力$T$大小
    相等(图6.13),即$T=T'$.

    根据作图 $mg\tan\theta=F$, 又由库仑定律可知
\[F=\frac{kQ^2}{r^2}=\frac{kQ^2}{(2\ell\sin\theta)^2}\]
于是:
\[mg\tan\theta=\frac{kQ^2}{(2\ell\sin\theta)^2},\qquad Q=2\ell\sin\theta\sqrt{\frac{mg\tan\theta}{k}}\]
\end{solution}


\item 在氢原子中,可以认为核外电子沿圆形轨道绕原子
核(质子)旋转,轨道半径为$5.3\x10^{-11}$米,电子沿轨道运动
的动能是多大?

\begin{solution}
    氢核与核外电子之间的库仑力$F_e=ke^2/r^2$, 
    电子绕核运动所需的向心力$F_n=mv^2/r=2E_k/r$.
    而$F_n=F_e$,
    \[\frac{ke^2}{r^2}=\frac{2E_k}{r}\]
\[E_k=\frac{ke^2}{2r}=\frac{9.0\x10^9\x(1.6\x10^{-19})^2}{2\x 5.3\x 10^{-11}}=2.2\x 10^{-18}{\rm J}\]
\end{solution}

\item 下面一些说法哪个正确,哪个错误?说明理由.
\begin{enumerate}
\item 在匀强电场中电势处处相等.
\item 沿着电力线的方向场强越来越小.
\item 正电荷在电场中只能由电势高的地方向电势低的地
方跑.
\item 电荷在电场中只能向着电势能低的地方跑.
\item 初速度为零的电荷在电场中一定向着电势能低的地
方跑.
\end{enumerate}

\begin{solution}
\begin{enumerate}
    \item 错误.电势沿场强方向降低.
    \item 错误.场强大小决定于电力线的疏密,不决定于电力
    线的指向.
    \item 错误.若正电荷具有与场强反方向的初速度,在一段
    时间内,它将逆着电势降低的方向作减速运动.
    \item 错误.若电荷具有与电荷所受电场力方向相反的初
    速度,在一段时间内,电场力作负功,在电荷的运动方向上,
    电势能是增大的.
    \item 正确.对初速度为零的电荷,其运动方向与电场力方
    向相同,电场力做正功,在运动方向上电势能减小.
\end{enumerate}
\end{solution}

\item 两块靠近的平行金属板,在两板之间为真空时,使它
们分别带上等量的异种电荷,保持两板带的电量不变,如果
将两板间的距离减小为原来的1/3,两板间的电势差是原来
的多少倍?两板间匀强电场的场强是原来的多少倍?

\begin{solution}
    由$C=Q/U$和$C=\dfrac{S}{4\pi kd}$得$U=4\pi kdQ/S$. 若其余
    各量不变时,$U$与$d$成正比,故当两板间的距离减小为原来的
    1/3时,两板间的电势差也减小为原来的1/3.

    又由$E=U/d$, 得$E=4\pi kQ/S$, 等式右边各量均为不变
值,可知两板间的电场强度保持不变.
\end{solution}

\item 上题中,保持两板带的电量不变,而在两板间充满介
电常数为8的电介质,两板间的电势差和匀强电场的场强将
如何改变?

\begin{solution}
    仿照上题的推导可得
    \[U=\frac{4\pi kdQ}{eS},\qquad E=\frac{4\pi kQ}{eS}\]
    在$Q$和$d$不变的条件下,两极板间充满电介质后,电势差和场强都变
    为原来的$1/e$.
\end{solution}

\item 有一个电容器,电容是$1.5\x10^{-4}$微法,把它的两板
分别跟直流电源的正负极相连,使两板分别带电$6\x10^{-8}$库
和$-6\x10^{-8}$库,如果两板的距离为1毫米,电容器两板间的
电场强度是多大?

\begin{solution}
    由$C=Q/U$和$E=U/d$可得
\[E=\frac{Q}{Cd}=\frac{6\x 10^{-8}}{1.5\x 10^{-10}\x 10^{-3}}=4\x 10^{5}{\rm V/m}\]
\end{solution}

\item 两个相当大的平行金属板相
距10厘米,两板分别跟电池组的正
负极连接,两板间的一个小电荷受到
的电场力为$3\x10^{-4}$牛,现在把两板
的距离增加到15厘米,如果连接的电
池组不变,小电荷受到的力变为多大?
如果在增大两板距离时把所连电池组
换成3倍电压的电池组,小电荷受到
的力又将变为多大?

\begin{solution}
    小电荷受的电场力$F=qE=qU/d$. 在$q$、$U$不变时,
    $F\propto 1/d$, 即$F_2/F_1=d_1/d_2$.
\[F_2=F_1\cdot \frac{d_1}{d_2}=3\x 10^{-4}\x \frac{10}{15}=2\x 10^{-4}{\rm N}\]
在保持$q$、$d_2$不变时,$F$与$U$成正比,即$F_3/F_2=U_2/U_1$,
\[F_3=F_2\cdot \frac{U_2}{U_1}=2\x 10^{-4}\x \frac{3}{1}=6\x 10^{-4}{\rm N}\]
\end{solution}


\item 如图 6.14,质量为$4.5\x
10^{-3}$千克的带电小球用2.0米长的线
悬挂在带等量异种电荷的平行板之
间,平衡时小球偏离竖直位置 2.0厘
米.
\begin{enumerate}
	\item 小球受到的电场力是多大?
	\item 如果两板间的电压是$1.5\x10^4$伏,两板的距离是10厘米,
那么,小球带有多少个多余的电子?
\end{enumerate}
	
\begin{figure}[htp]
    \centering
\begin{tikzpicture}[>=latex]
\draw [|<->|](0,0)--node [fill=white]{2.0m}(0,5);
\draw [|-|, dashed](-1,0)--(-1,5);        
\draw [thick](-1.4,0)--(-1,5); 
\draw[fill=gray] (-1.4,0) circle (3pt);

\draw[very thick] (-2, -1)--(-2, 1)node [left]{$+$}; \draw[very thick] (-.5, -1)--(-.5, 1)node [right]{$-$};
\draw [|<->|] (-1.4,-.25)--node [below]{2.0cm}(-1,-.25);


    \end{tikzpicture}
    \caption{}
\end{figure}

\begin{solution}
\begin{enumerate}
\item 带电小球在重力$mg$、电场力
    $F$和线的拉力$T$作用下静止.由共点力
    的平衡条件,得$F=mg\tan\theta$.$\theta$角是线与
    竖直方向的夹角,所以$\tan\theta=0.02/2.0=
    0.01$, 由此得$F=4.5\x10^{-3}\x9.8\x0.01=4.4\x10^{-4}{\rm N}$
    \item 由$F=qE=qU/d$, 得出小球所带负电量$q=Fd/U$. 由
    此可知小球上多余的电子数
    \[n=\frac{q}{e}=\frac{Fd}{Ue}=\frac{4.4\x10^{-4}\x 0.1}{1.5\x10^4\x 1.6\x 10^{-19}}=1.8\x 10^{10}\text{(个)}\]
\end{enumerate}
\end{solution}


\item 在课本图6.29中,先让一束电子,后让一束氢核通过偏
转电场,设电子和氢核的初速度相同,电子和氢核原来
的动能相同,试分别求出两种情况下电子的偏角$\phi_e$和氢核
的偏角$\phi_H$的正切之比.

\begin{solution}	
带电粒子垂直于场强方向进入匀强电场,经偏转离开
电场时偏离入射方向的偏角$\phi$的正切值可由下式推出:
\[\tan\phi=\frac{v_{\bot}}{v_0}=\frac{at}{v_0}=\frac{qU}{md}\cdot \frac{\ell}{v^2_0}\]
\begin{enumerate}
\item 电子与氢核以相同的初速度$v_0$垂直进入电场.在表
示偏角$\phi$的公式中,除质量$m$外其余各量($q$、$U$、$d$、$\ell$)都
相同.因此:
\[\frac{\tan\phi_e}{\tan\phi_H}=\frac{m_H}{m_e}=1840:1\]

\item 由前面$\tan\phi$的表达式可得
\[\tan\phi=\frac{qU\ell}{2dE_k}\]
电子与氢核原来的动能$E_k$相同,且其余各量($q$、$U$、$d$、$\ell$)也相同,可
得
\[\frac{\tan\phi_e}{\tan\phi_H}=1:1\]
\end{enumerate}
\end{solution}

\item 图6.15是用来使带正电的粒子加速和偏转的装
置,如果让氢和氦进入并电离,我们将得到一价氢离子,一价
氦离子和二价氦离子的混合物.它们经过同一电场加速后,
在同一电场里偏转,它们是否会分为三股,从而到达荧光屏后
产生三个亮点?回答中要说明理由.

\begin{figure}[htp]\centering
    \includegraphics[scale=.7]{fig/6-15.png}
        \caption{}
        \end{figure}

\begin{solution}
    带电粒子在通过偏转电场后,偏角$\phi$的正切值
    \[\tan\phi=\frac{qU\ell}{mdv_0^2}=\frac{qU\ell}{2dE_k}\]
式中$E_k$为进入偏转电场时带电粒子
    的初动能,$\ell$、$d$、$U$分别为偏转电极的长度、距离和电压,$q$和$m$
    分别为带电粒子的电量和质量.

    设若三种离子在加速电场中的初速度都为零,则由动能
    定理可得$E_k=qU_{\text{加}}$. 代入$\tan\phi=\dfrac{U\ell }{2dU_{\text{加}}}$. 可见,偏角的正
    切值与带电离子的电量、质量无关.所以三种离子的混合物
    不会分为三股,荧光屏上也不会产生三个亮点.
\end{solution}

\end{enumerate}


























\section{参考资料}
\subsection{做好静电实验的几个有关问题}
\subsubsection{如何防止漏电}

静电实验有两个特点:一是电压高,
二是电量少,由于电压高,以至平时的绝缘体失去了绝缘性
能,结果电荷极易流失;加之电量又少,一经漏电,就很快漏
完.因此静电实验要解决的主要问题是防止漏电.

首先,要保持仪器良好的绝缘性能.要选用有机玻璃、塑
料等材料作绝缘体;对于用旧了的塑料部件,应将其表面清洗
干净,晾干后涂上一层熔化了的干净石蜡,待石蜡凝固后再
使用.必要时,仪器不要直接放在桌上做实验,可用发泡塑料
垫上.

其次,教室要通气,要避免空气中有过多的二氧化碳、水
蒸气或各种离子,这样可防止电荷从空气中漏掉.

为使带电体所带电量不变,也可采用随漏随补,漏多少、
补多少的方法来达到目的.例如,演示两个带电小球在库仑
力作用下相互排斥,最后达到平衡状态的现象.就可以用棉
线作导线,接到起电机的电极上,轻摇起电机逐渐达到转速不
变,就可以使两个小球排斥到一定位置后静止不动.此时小
球上漏掉的电量与补充的电量相当.


\subsubsection{如何使带电体带上较多的电荷}
用毛皮摩擦橡胶棒
后使一绝缘导体球带电,怎样操作才能使导体球带上较多的
电荷呢?

首先要使摩擦起的电多.为此,毛皮和橡胶棒要清洁、干
燥,毛皮不脱毛,摩擦要握得紧,使摩擦作用加剧.分离时要
快,避免边摩擦边中和的现象发生.

还要使带电后的橡胶棒以不同位置多次与导体球接触
(但不要与导体球摩擦,以免产生异种电荷).重复摩擦、接触
几次,就可以使导体球带上较多的电荷.

\subsubsection{防止“反常”静电现象的产生}

做静电实验时,有时
会出现与预期结果相反的“反常”现象.

用摩擦后的玻璃棒接近一不带电的验电器时,验电器的
箔片张开,移走玻璃棒后验电器箔片重新闭合.这说明静电
感应时,外电场作用迫使导体中电荷重新分布,而其正负电荷
的总量没有发生变化,但演示时,也可能发生移去玻璃棒后,
验电器箔片不闭合的现象.这里,是放电造成了这一“反常”
情况的出现.摩擦后的玻璃棒的电势可达几千伏,当它接近
不带电的验电器到某一程度时,其间就可能发生放电现象,使
验电器带上同种电荷,箔片自然不会闭合了.

用课本图6.14演示静电平衡电荷只分布在导体的外表
面时,不带电的金属球与带电的空心圆筒内壁接触后、再与验
电器接触,验电器箔片应该不张开.但也可能出现箔片张开
的“反常”现象,这同样是放电造成的.

为了达到预期的效果,做这些实验时,带电体不应与验电
器离得太近;从圆筒顶部圆孔取出金属球时,特别注意使金属
球从孔的正中位置取出,防止金属球与圆孔边沿之间发生
放电.

把带电体移近被金属网罩住的验电器,箔片不张开,说明
金属网能起静电屏蔽作用.但演示时,也可能出现箔片略有
张开的“反常”现象.这里是由于没有满足验电器与金属网间
的距离要比网眼线度大数倍的条件所致.演示时要注意这
一点.

这些反常现象的出现,会分散学生观察现象的注意力,扰
乱学生的思维,实验时要尽可能避免.

\subsection{电介质}
电介质就是绝缘物质.如果点电荷在电介质中,它们之
间的库仑力小于在真空中的库仑力,平行板电容器极板间充
入电介质后,其电容也会变化.原因何在呢?

电介质的分子有两种.一种叫无极分子,如$\rm H_2$, $\rm N_2$,
$\rm CCl_4$等分子.另一种叫有极分子,如水分子.我们知道,组
成物质分子的原子,是由带正电的原子核和绕核高速旋转的
电子组成的.平常分子不显电性,其中的正电荷总量和负电
荷总量是相等的.但是,由于分子中的正负电荷不是集中于
一点,我们把与分子中所有正电荷等效的、某个单独的正点电
荷的位置,叫分子中正电荷的电“重心”;同理,分子中负电荷
的电“重心”,也就是与分子中所有负电荷等效的、某个单独的
负点电荷的位置.在无外电场时,电介质中分子的正负电荷
的电重心是重合的,这类分子叫无极分子.在无外电场时,电
介质分子的正负电荷的电重心不重合,这类分子叫有极分子.
每个有极分子相当于一个电偶极子,由于热运动,它在电介
质中的排列是混乱的,故无论在电介质内部或表面,均不显
电性.

电介质在电场作用下,其垂直于场强方向的两端表面出
现束缚电荷的现象,叫极化.束缚电荷,是指不能自由移动的
电荷.电介质的极化分为两类.一类是无极分子在外电场作
用下,电重心发生相对位移(由于电子质量远小于原子核的质
量,相对位移主要是电子的位移).对整块电介质,垂直于电
场方向的两端面上出现束缚电荷.这类极化,叫位移极化.另
一类是在外电场作用下,电介质中的有极分子受到电力转矩,
电偶极子的取向趋向于外电场方向,使电介质在这个方向上
的两个端面出现束缚电荷.这类极化,叫转向极化,位移极
化在有极分子中同样存在,只不过它比转向极化的效应小得
多.因此,有极分子组成的电介质的极化,主要是转向极化.

两种极化的宏观效果是相同的,即在垂直于外电场方向
上的两个端面上出现束缚电荷.它产生的电场$E_{\text{束}}$, 与外电场
$E_0$的方向相反,削弱$E_0$的作用.设电介质极化后的电场强
度为$E$, 则$E=E_0-E_{\text{束}}$, 小于$E_0$.

在理论上可以得出$E_0/E=\varepsilon$, $\varepsilon$叫相对电介常数(现称为
介电常数),是一个无量纲的量,它是表征电介质极化程度的
特性的物理量,各种电介质的$\varepsilon$值都大于1.

点电荷所在的空间充满电介质后,场强为真空时的$\varepsilon$分
之一,故库仑力减小为真空中的$\varepsilon$分之一.

\subsection{零电势的选择}
在理论研究中常把无穷远处电势规定为零.因为在普通
物理学中,点电荷电场中任两点间的电势差
\[U_1-U_2=Q\left(\frac{1}{r_1}-\frac{1}{r_2}\right)\]
若$r_2\to \infty$, 即$U_2=U_{\infty}$, 那么$U_1=U_{\infty}+Q/r_1$. 取$U_{\infty}=0$, $U_1=Q/r_1$, 表达较为简洁.所以,在理论研究中,取无穷远(即电
场外)电势为零,能简化公式,方便研究.

在实际中取地球电势为零,各地电势相等.当然,严格地
讲地球不是一个等势体,研究表明,地球表面有电流存在,即
地球表面上,电势不等,沿表面的切向方向场强不为零.但由
于这个场强值$E_{\tau}$很小,在处理一些日常静电问题时,带电体
的线度$\Delta \ell$远小于地球的线度.由$E_{\tau}=-\Delta U/\Delta \ell$
可知,$\Delta U$也很小,所以,在很大程度上是完全可以将地球视为等势体,人们
在各地从事生产,科研,若取地球电势为零,就有一个统一标
准,带来许多方便.

那么,无穷远处的电势与大地的电势是否相等呢?根据分
析论证认为,它们是相等的,即$U_{\text{地}}=U_{\infty}$. 限于篇幅,这里不
作说明,请参考《物理教学》1984年第5期:“无限远处电势跟
地球电势的等电势问题”一文.

\subsection{密立根油滴实验}
1917年美国科学家密立根成功地测定了电子的电荷量.
他是利用水平放置的平行板电容器,以其中的一个带电油滴
为研究对象,通过测定它匀速下降及匀速上升的速度,计算出
油滴所带的电量;在取得大量的油滴所带电量的数据后,通过
数学处理,算出了基本电荷$e$的值.

他的实验装置如图6.16所示.$MN$为两块水平放置的
黄铜板,上板$M$开有一小孔.由喷雾器喷出的油滴通过这小
孔进入两板间;由于摩擦作用,这些油滴分别带有正电荷或负
电荷.光源发出的光,照亮了油滴,再用显微镜来观察和测量
油滴的运动.显微镜的目镜中装有分划板(图6.17),分划板
上每小格的距离是已知的,因此利用分划板可以很方便地测
量油滴运动的距离;再配合使用秒表,就可测定油滴运动的速
度.$K$是一个单刀双掷开关,当$K$接1时,$MN$板间短路;
当$K$接2时,$MN$板间接一连续可调的平衡电压$U$.

\begin{figure}[htp]\centering
    \begin{minipage}[t]{0.48\textwidth}
    \centering
    \includegraphics[scale=.8]{fig/6-16.png}
    \caption{}
    \end{minipage}
    \begin{minipage}[t]{0.48\textwidth}
    \centering
    \includegraphics[scale=.8]{fig/6-17.png}
    \caption{}
    \end{minipage}
    \end{figure}

实验时,先将开关$K$接1处.此时$M$、$N$板未带电,作
为研究对象的带负电的油滴受三个力作用而下落.这三个力
中,一是重力,方向向下,大小为
\[G=\frac{4}{3}\pi r^3\rho_{\text{油}}g\]
(静止的油滴
由于表面张力而呈球形,当它以不大的速度运动时,仍可当
作球形).其中$r$为油滴半径,$\rho_{\text{油}}$为油的密度,$g$为当地的重
力加速度.二是空气的浮力,方向向上,大小为
\[F=\frac{4}{3}\pi r^3 \rho_{\text{空}}g\]
$\rho_{\text{空}}$为空气的密度.三是空气对油滴的粘滞阻力,方
向向上.根据斯托克斯定律,粘滞阻力跟油滴的运动速度成
正比,故油滴在板间作加速度变小的加速运动,最后达一收
尾速度而匀速下落.匀速运动的速度为$v_{\text{下}}$,则可知它所受的
粘滞阻力$f=6\pi\eta rv_{\text{下}}$. $\eta$为空气的粘滞系数.对这个油滴运
用平衡条件,有
\[G-F-f=0\]
即:
\[\frac{4}{3}\pi r^3\rho_{\text{油}}g-\frac{4}{3}\pi r^3\rho_{\text{空}}g-6\pi\eta rv_{\text{下}}=0\]
测出$v_{\text{下}}$,再由已知的$\rho_{\text{油}}$、$\rho_{\text{空}}$和、$\eta$,可得这个油滴半径
\[r=\frac{3\sqrt{\eta v_{\text{下}}}}{\sqrt{2(\rho_{\text{油}}-\rho_{\text{空}})g}}\]

然后,把单刀双掷开关扳向2. 这时,电极板间形成一匀
强电场,于是这个油滴除受重力、浮力、粘滞阻力外,还受到向
上的电场力$qU/d$的作用.调节两板间的电压$U$, 使这个油
滴恰好能匀速上升,测出它上升的速度$v_{\text{上}}$.再运用平衡条
件,则有
\[qU/d+F-G-f'=0\]
即
\[qU/d+\frac{4}{3}\pi r^3\rho_{\text{空}}g-\frac{4}{3}\pi r^3\rho_{\text{油}}g-6\pi\eta rv_{\text{上}}=0\]
将前面$r$的表达式代入,整理后得油滴电量
\[q=\frac{q\sqrt{2}\pi v_{\text{下}}^{1/2} \eta^{3/2}(v_{\text{下}}+v_{\text{上}})d}{U(\rho_{\text{油}}-\rho_{\text{空}})^{1/2}g^{1/2}}\]

密立根用了三年多的时间,取得了数千个油滴电量的数
据,并对数据进行分析研究,发现这些电量都是某个最小电荷
的整数倍,这不仅证实了基本电荷的存在,还测得了它的值
是$e=1.59\x10^{-19}$库,这是当时密立根实验的测量值.

\subsection{静电的危害及其防止、应用}

\subsubsection{静电的危害}

表现有以下方面:

第一,静电力的作用.
比如,静电吸引尘埃,使纺织品色泽灰暗,使半导体产品质量
下降.摩擦使纤维带电,互相排斥,给加拈成纱造成困难.印
刷时,受滚筒挤压而摩擦带电的纸张,常吸在铅板或滚筒上,
影响连续印刷.

第二,静电场的干扰.比如,汽车上的收音机,因轮胎与
路面摩擦起电的干扰,而无法接收信号.片卷砂砾摩擦带电,
使无线电通信中断.

第三,静电的放电,缓慢的放电,会使有机纺织品纤维的
强度、韧性降低,使半导体器件和电子仪器的元件变质.更为
严重的是急剧的放电,会造成火灾、爆炸,危害人身安全.


\subsubsection{静电危害的防止}

概括起来有以下措施:

第一,抗静电,控制静电不产生或少产生,由于摩擦的
不可避免,需要从产生静电的物质本身,采取措施.比如,易
产生静电的机械零件,可尽量采用导电材料;必须采用高绝缘
电阻材料(橡胶、塑料)的零件,可以在材料的配方中加入抗静
电剂,制成导电的橡胶、塑料、纤维等;还可以给这些材料穿上
“导电衣”,涂上一层金属粉末或导电漆.如聚乙烯塑料,包上
一层叫做“隆斯推特-P”的导电薄膜,就能使原来一、两千伏
的电势基本降至零.

第二,消静电,产生静电后,防止聚集,及时逸散,采用
导体(线)接地,把聚集的电荷放走.比如,输油管每隔100—
300米就要接地一次,油罐汽车运油时要安上一条拖地的金
属链.调节空气湿度,以利放电,根据实验,油罐汽车装卸
时,空气湿度为35—40\%,油罐电势达1100伏;当空气湿度
为72—75\%时,电势基本降为零、设置“静电消止器”,使它产
生并注入与积累电荷电性相反的离子;或用放射性同位素(如
钋),放射$\alpha$粒子使空气分子电离后,把积累的静电导走.避
雷针是利用尖端放电,把产生的电荷随时释放在空气中,从而
避免雷击.

第三,及时预防静电的危害,还可以采用能自动记录和报
警的仪器.

\subsubsection{静电的应用}

从原理上,可分为两大方面:

利用静电力控制带电物体(粒子)的运动.又分为:静电
除尘;静电摄影,包括静电复印、制版、照相;静电喷涂,包括静
电喷漆,静电植绒,静电喷撒,油墨喷射印刷;静电分离:分选
种子、矿石等;粒子束:离子束注入,电子爆光,等离子刻蚀;示
波管及应用于带电粒子的测量仪器:加速度计、重力计、陀螺
仪等.

在能量转换方面的应用,以静电场作媒介来实现机械能
与电能之间的相互转化.如范德格喇夫起电机.由于离子或带
电胶体粒子在真空中高电压作用下而加速,以极大的速度喷
出产生强大的推力,制成离子推进器.

以上资料的详细内容,参见《物理教师》1983年第6期:
“静电力的应用”一文.




\chapter{稳恒电流}\minitoc[n]
\section{教学要求}
这一章是电学部分的重点章,本章讲授的直流电路基本
知识,有较大的实用价值,是学习电工和电子技术的基础,对
日常生活和生产中使用电器也很有用.

教材注意了联系前章学过的电场知识,注意了联系实际.
在讲电流的产生、电功、电路中的电势降落等知识时,注意用
前章学过的电场、电势能、电势等知识给予说明,以加深学生
的理解.在讲串联和并联电路、电路的分析和计算、闭合电路
的欧姆定律等知识时,注意它们的实际应用,在培养能力方
面,注意教给学生分析解决电路问题的方法.

全章教材共十四节,可以划分为五个单元,第一节到第
三节为第一单元,讲述直流电路的一些基本概念和规律.第
四节和第五节为第二单元,从功和能的观点研究电路中的电
能转化.第六节到第九节为第三单元,讲述串联电路和并联
电路,以及简单的混联电路.第十节到第十二节为第四单元,
讲述闭合电路的欧姆定律及其应用.第十三节和第十四节为
第五单元,讲述测量电阻的方法.

部分电路欧姆定律和闭合电路欧姆定律是电学中最基本
的规律之一,它们不仅是解决直流电路有关计算的依据,而且
对学习交流电路有重要作用.所以,欧姆定律是全章的中心
和贯穿全章教材的主线.电流、电压、电动势和电阻是稳恒电
流的几个基本理量,欧姆定律正反映了这些量之间的相互
关系;电功、电功率、焦耳定律都和欧姆定律有着密切联系,研
究了电能与其他形式能的转化,有重要的实际意义;串联电路
和并联电路是电路连接的基本形式,掌握串、并联电路的基本
特点,是应用欧姆定律进行电路计算的关键,这些知识都很
重要,是研究直流电路的基础和工具,电池组在生活、生产和
实验中都常用.学生应该知道并联电池组和串联电池组的
特点,并能根据实际需要选用和组成电池组.电阻的测量在
电学实验、业余无线电活动以及生活、生产中经常遇到.教材
讲解电阻的测量主要是综合运用本章学过的知识,以提高学
生运用知识的能力.值得说明的是,电动势的概念尽管重要,
但由于这个概念比较复杂,限于学生的知识水平,中学阶段很
难讲清楚,不要求涉及电源内部的情况和电动势是怎样产
生的.

这一章的教学要求是:
\begin{enumerate}
\item 了解电流的产生条件,理解电流强度的概念,理解电
阻和电阻率的概念,掌握电阻定律.巩固掌握欧姆定律.
\item 理解电功和电功率的概念,掌握电功和电功率的公
式,掌握焦耳定律,了解电功和电热的关系
\item 巩固掌握串联电路和并联电路的特点,会分析、解决
直流电路中的混联问题.
\item 了解电动势的概念,巩固掌握闭合电路的欧姆定律.
掌握串联电池组及并联电池组的计算和应用,了解混联电池
组的应用.
\item 理解用伏安法和欧姆表法测电阻的原理.
\end{enumerate}

\section{教学建议}
本章教学的重点是闭合电路欧姆定律,难点是电动势概
念以及电动的几个公式的意义和运用条件.

由于全章教材内容具有鲜明的实践性,教学中应当加强
演示实验和学生实验,培养学生的实验能力.要特别注意教
给学生分析问题的思路和方法,培养他们的思维能力及应用
物理规律解决实际问题的能力.


\subsection{第一单元}
这一单元主要复习初中教材的电流强度、电阻、以及欧姆
定律,并在复习的基础上应用电场理论适当予以深化、提高.
同时补充学习电阻定律、电阻率等新知识.初中受学生知识
水平和思维能力发展的限制,只能从生活经验及实验事实出
发讲得比较简单,考虑到不学好这些概念、规律,很难进一步
学好其他电学知识,而在学过电场知识以后又可能讲得更深
入一些,因此教材仍把它们作为重要内容认真讲述.教学中
切不可因为学生已有一定的基础而掉以轻心,既要重视突出
知识的要点,又应当针对学生容易出错的问题,采用适当的教
学方法予以强调.

\subsubsection{电流}

电流产生的条件可按教材讲述的层次,利用课
本图7.1和图7.2, 逐步引导学生认识:
\begin{enumerate}
\item 形成电流
的条件是既要有能自由移动的电荷-自由电荷,又必须在
导体中建立电场.    \item 由于导体内存在自由电荷,所以导体中
存在持续电流的条件是保持导体两端有电势差.电源的作用
正是为了保持电路两端的电势差,使电路中存在持续电流.
\item 在学完稳恒电流与直流电的区别后应明确:要得到大小、
方向都不随时间变化的稳恒电流,导体两端的电势差必须保
持恒定.
\end{enumerate}

教材指出“自由电子除了做无规则的热运动外,还要在电
场力的作用下做定向移动,”对这段叙述,学生往往缺乏清晰
的认识,应当指出,自由电子作定向运动时,其无规则热运动
并未消失.它们一方面继续作无规则热运动,另一方面又沿
某一方向定向运动,即在无规则热运动速度的基础上叠加了
一个定向移动速度.为了避免学生从生活经验出发认为自由
电子定向移动速率非常大的错误,可以把第八章中关于自由
电子热运动的平均速率,定向移动速率和电场传播速率的数
量级分别为105$\ms$、$10^{-5}\ms$,$3\x10^8\ms$提前介绍.强
调电流传导速率实际上是电场传播速率,而不是自由电子定
向移动速率.

在讲解电流强度的概念及电流方向时,可以提出一些问
题让学生讨论,如:
\begin{enumerate}
\item 当通过不同横截面积的几根导线的电
流强度相等时,在相同时间内通过各导线横截面的电量跟截
面积大小有无关系?为什么?    
\item 为什么说导体中电流的方向总是从电势高的一端流向电势低的一端?
\item 电解液中电流强
度的大小和方向怎样确定?通过讨论加深学生对这部分知识
的理解,澄清一些模糊认识.
\end{enumerate}


\subsubsection{欧姆定律}

电阻是电路中的三个基本物理量之一,正
确理解电阻的物理意义对理解和掌握欧姆定律有重要作用.
因此,教材首先对电阻的概念作了认真的剖析.$R=U/I$
是电
阻的定义式,表明了一种量度电阻的方法.但学生在学习中
往往出现以下错误,如“电阻跟电压成正比,跟电流强度成反
比.”“既然电阻是表示导体对电流的阻碍作用的物理量,所以
导体中没有电流时导体就不存在电阻.”教学中应当有针对性
地将这些问题提出来讨论,以阐明$R=U/I$
的物理意义,纠正学
生的错误认识.要强调对金属和电解液电阻R是一个只与导
体本身有关的量.当导体的材料、粗细、长度、温度等因素一
定后,导体的电阻就确定了.电压和电流变化了,它们的比值$U/I$
即$R$是不会变的.同时应当指出,不管有无电流通过导体,
反映导体对电流阻碍作用大小的电阻$R$是客观存在的,只不
过当有电流通过导体时,这个作用才具体表现出来.

在进行欧姆定律$I=U/R$
的教学时,应通过具体题目的分
析,反复强调公式中的$U$、$I$、$R$是表示同一段电路的三个物
理量,决不能把属于不同电路的$U$、$I$、$R$值代入公式计算.

伏-安特性曲线是说明物质导电特性常用的一种表示方
法,在电工学和无线电电子学中有着广泛应用,要让学生对
此有所了解.应引导学生认识金属导体和电解液的伏-安特
性曲线是通过坐标原点的一条直线,具有这种性质的电阻叫
线性电阻,直线的斜率等于导体电阻的倒数.因此,通过比
较直线斜率的大小可以判断电阻的大小,但是有的导体和元
件,如热敏电阻、晶体二极管、真空二极管等的伏-安特性曲
线都不是直线而是特殊形状的曲线.这些导体和元件的电阻
就不是线性电阻而是非线性电阻了.它们的电阻$R$不是一个
常量,其大小与电压、电流的值有关.所以,它们不遵从欧
姆定律.

从线性电阻和非线性电阻的伏-安特性曲线的分析,我
们可以看出,尽管欧姆定律$I=U/R$
和电阻的定义式$R=U/I$
从数学形式上并无本质不同,但是物理意义却有区别.线性
电阻遵从欧姆定律,非线性电阻不遵从欧姆定律,然而,无论
线性还是非线性电阻都可以用$R=U/I$
来定义.这个问题只要
点到就行,不必展开讲解.

\subsubsection{电阻定律和电阻率}

电阻定律是一个实验定律.由
于在初中只作过定性实验,为了增加学生的感性认识,熟悉运
用实验总结物理规律的研究方法,在教学中有必要通过定量
实验得出电阻定律$R=\rho\ell/S$.
并且说明电阻定律不是对所有
导体都适用的电阻定义式,它适用于粗细均匀的金属导体及
浓度均匀一致的电解液.它定量揭示了这类导体的电阻由自
身的哪些物理条件决定,也反映了这类导体的电阻大小跟加
在导体两端的电压和通过的电流强度无关.

材料的电阻率$\rho$是描述材料导电性好坏的一个物理量,
可按教材安排的顺序分以下几步讲解.
\begin{enumerate}
\item 从分析电阻定律$R=\rho\ell/S$
中比例系数$\rho$的意义入手,讲清电阻率的意义.
\item 
强调指出电阻率与导体的大小、形状无关,仅取决于导体材料
的性质和导体所处的条件(如温度).公式$\rho=RS/\ell$
表示材料
的电阻率在数值上等于这种材料制成的长1m,横截面积
$1{\rm m^2}$的导体的电阻.
\item 说明在国际单位制中电阻率单位为
什么是欧姆·米.
\item 引导学生讨论电阻和电阻率有什么区
别,使他们认识到电阻率反映物质对电流阻碍作用的特性,
电阻则反映物体对电流阻碍作用的特性,因此,前者仅与物
质种类有关,后者却不但与构成物体的物质种类有关还与物
体的长度、横截面积有关.教材练习二第3题有助于学生理
解这个问题,可安排课堂讨论.
\item 指导学生阅读教材180页
的表列数值,使他们对常用金属的导电性能强弱有一个清楚
的认识.
\end{enumerate}

各种物质的电阻率都和温度有关,因此,各种导体的电阻
也都随温度变化.实验表明,所有纯金属的电阻率都随温度
升高而增大.为了加强教学的直观性,可以增加如下一个演
示实验:把一个废白炽灯泡的玻璃外壳小心去掉,取其内部一
段几厘米长的钨丝接在灯芯的灯丝支架上,与一个1.5伏电
源,一个安培表接成串联电路,测出电流强度,然后用酒精
灯把灯丝烧红,将看到电流强度明显减小.从而定性说明温
度越高,金属电阻率越大.精确实验证明,当温度变化范围不
大时,电阻率跟温度近似有如下线性关系$\rho=\rho_0(1+\alpha t)$. 式
中$\rho$和$\rho_0$分别是温度为$t$和$0^{\circ}{\rm C}$时材料的电阻率,$\alpha$为材料的
温度系数.多数纯金属的$\alpha$值近似等于$4\x10^{-3}/$度.对电
阻率跟温度的这个定量关系式,一般不宜在课堂上讲授,如:
果教师认为有必要介绍,也只需让学生知道就行,不要作过多
的阐述.但是,应当向学生指出:今后在涉及电阻值的计算
时,为使问题简化,如果没有特殊要求,我们通常不考虑温度
对电阻率的影响.

关于超导体的讲授,可引用后面参考资料提供的材料,适
当充实教学内容,以开阔学生眼界,激发他们进行科学探索的
志趣.

\subsection{第二单元}
这一单元仍属复习提高性质,目的在于使学生从电场力
做功和电势能变化的角度加深对电流做功的理解,搞清楚电
路中的能量转化关系.

\subsubsection{电功和电功率}

根据教材的逻辑顺序,从复习电场力
移动电荷做功$W=qU$结合电流强度的定义式$I=q/t$
推导出
电流做功的计算式$W=UIt$; 从复习电场力做功与电势能变
化的关系,用能量转化的观点阐明电流做功的实质,学生理解
起来并不太困难,但是他们往往对电流是怎么做功实现能量
转化的具体物理过程提出疑问.限于学生的现有知识水平,
一时不可能讲清.因此,建议最好先以能量转化关系明显,学
生容易接受的电炉、电动机等实际用电器为例,讲清通过电场
力移动电荷做功,电势能减少,电能转变为内能、机械能等,再
概括为一般结论:电流通过用电器做功的过程实际上是电能
转化为其他形式的能的过程.电流做了多少功,就有多少电
能转化为其他形式的能.

讲解电功率的概念时,应强调$P=UI$是电功率的一般表
达式,适用于任何用电器,表示用电器消耗的全部电功率,为
学习下一节内容奠定基础,要针对学生的常见错误,着重讲清
用电器的额定功率和实际功率,一方面通过实例的计算、分
析,具体说明它们的区别.如举这样的例题:把标有“220V
100W”的灯泡接到220伏的电路中,过灯丝的电流多大?
灯泡的功率多大?如果接到110伏电路中电流和功率又分别
是多少?假定灯丝电阻不变.另一方面可通过演示某一灯泡
在额定电压(需用伏特表观察)、低于额定电压、适当高于额定
电压等三种情况下的实际功率(亮度),以鲜明的直观加深学
生的印象,也可采用讨论式进行教学.首先提出以下两个问
题供学生看书讨论.
\begin{enumerate}
\item 公式$P=UI$表示的物理意义是什
么?讲一个用电器的电功率大,表示什么意思?   
 \item 什么叫用电
器的额定功率?什么又叫用电器的实际功率?标有“220V
100W”的灯泡是否一定比标有“220V40W”的灯泡亮?为什
么?在什么情况下,前者才会比后者亮?
\end{enumerate}
然后师生共同归纳这
部分知识的要点:电功率的概念及数学表达式,额定功率及实
际功率,最后通过上述演示实验强化用电器的额定功率和实
际功率的区别.

\subsubsection{焦耳定律}

在讲解焦耳定律时,应指出焦耳定律是反
映电流热应的定量规律,在国际单位制中,热量和功都用
焦耳作单位,故焦耳定律的数学表达式为$Q=I^2Rt$. 但是,如
果热量采用卡作单位,$I$、$R$、$t$用安培、欧姆、秒作单位,上式
两边的单位就不统一,需要通过热功当量$1{\rm Cal}=4.2{\rm J}$改写
为$Q=0.24I^2Rt$. 还应当强调$Q=I^2Rt$是焦耳定律的基本形
式.无论对任何电路,只要有电阻$R$存在,由电流热效应产生
的热量都可以通过这个公式计算,只有在纯电阻电路中,由
于$U=IR$, 焦耳定律才能表示成$Q=U^2t/R$
或$Q=UIt$的形式.

混淆电功与电热的概念,把$W=IUt$和$W=I^2Rt$、
$W=U^2t/R$
等同看待,是学生的一个常见错误,因此,阐明电功
和电热的关系是本节教学的难点.具体教法可采用理论分析
与实验演示相结合的形式.

首先从能量转化和守恒的观点分析:
\begin{enumerate}
\item 对只含纯电阻用
电器(如电灯、电炉等)的电路——纯电阻电路,电流做的功等
于$UIt$, 电流热效应产生的热量等于$I^2Rt$. 由于这时电路两
端的电压$U=IR$, 因此$W=UIt=U^2t/R=I^2Rt=Q$, 即电功等-
于电热,电能全部转化为内能.
\item 对于包含电动机、电解槽
等用电器的非纯电阻电路,尽管电功仍为$UIt$, 电热仍为
$I^2Rt$, 但它们的能量转化关系跟纯电阻电路不同.在电动机.
里,电能$\to $机械能$+$内能;在电解槽里,电能$\to $化学能$+$
内能.说明电能除部分转化为内能外还要转化为其他形式能
量.因此,$UIt>I^2Rt$, 即电功大于电热.这时电路两端电压
$U>IR$.
\item 综合上述分析得出结论:在纯电阻电路中$W=UIt$
和$W=I^2Rt$等效,电功等于电热;在非纯电阻电路中电功大
于电热,只能用$UIt$和$I^2Rt$分别计算电功和电热.并通过
教材最后一段的例子用具体数据说明这个结论.
\item 向学生
指出用电器消耗的全部电功率$P=UI$跟用电器因发热而消
耗的功率$P=I^2R$也存在着上述区别,同样不能混为一谈.
\end{enumerate}

其次,通过实验演示(见实验指导,演示实验1)以加深学
生的印象,帮助他们确信教材最后一段所举例子的真实性,直
观地认识电功与电热的区别.

第一、二单元的教材,大部分属于复习、巩固初中知识的
内容.实践证明采用指导学生自学的教学方式可以取得较好
的效果,如果过去对学生的自学能力培养不够,可在每节课
通过布置自学提纲或思考、讨论题等形式指导学生自学、讨
论,但要注意及时归纳小结.如果学生的自学习惯已基本形
成,不妨分单元按以下四个环节组织教学:
\begin{enumerate}
\item 教师概要介绍
全单元知识要点和知识间联系,提出自学要求.    
\item 学生自学,写出读书笔记,用课本上的练习题自我检查学习效果.教
师则个别指导、答疑,并收集学生中出现的问题.    
\item 教师根据学生的问题和教学重点,有针对性地讲解.    
\item 讲练结合,辅之必要的小组讨论,以达到复习、巩固的目的.
\end{enumerate}


\subsection{第三单元}
这一单元是前两个单元中基本概念、基本规律的具体应
用.学生在初中通过实验认识了串联电路和并联电路的基本
特点,学习了这两种电路总电阻的计算,高中进一步研究它们
的电压分配、电流分配、功率分配等.并且通过例题讲述了混
联电路的实际意义和分析、计算方法,教学中应突出分析的
思路,即首先搞清楚电路各部分间的串、并联关系,然后运用
有关的知识进行计算.应当注意,要求学生解决的混联问题
应该是比较简单和有实际意义的.所谓比较简单,是指电路
各部分间的串、并联关系比较容易辨认出来;所谓有意义,是
指电路应是实际中存在的或者是实际电路的简化或抽象.让
学生去解那些实际意义不大的繁难题目,教学中应该避免,至
于那些不能最终化为串、并联的电路,需要用基尔霍夫定律才
能求解的复杂电路(网路),更是不必涉及.

\subsubsection{串联电路}

在初中是通过实验得出串联电路的电流、
电压特点的.现在联系前一章学过的电场知识和稳恒电流的
特点作进一步说明,以加深学生的理解.

在讲解串联电路两端的总电压等于各部分电路两端的电
压之和时,可从电场力移动电荷做功跟电势能变化的关系出
发,分析为什么电流通过串联电路各电阻时,沿电流方向每通
过一个电阻,电势要降低一个数值,并结合教材图7.6说明
\[U=U_1+U_2+U_3\]

在上述的基础上,运用欧姆定律不难推出串联电路的总
电阻,电压分配、功率分配等三个具体规律.教学中要着重讲
清并让学生掌握推导的思路.只有真正理解了这些规律的意
义及相互关系,才不会孤立地死记硬背,也才能在解决实际问
题时灵活运用.

进行这部分知识的教学,应注意以下三个问题.

\begin{enumerate}
    \item 讲清“等效”的意义.所谓等效,就是指作用效果相
同.例如在力学中,某一个力对物体的作用效果与另外几个
力对物体的共同作用效果相同,我们就可以认为这个力是那
几个力的等效力,又叫那几个力的合力,用等效力去代替原
来那几个力,在处理某些问题时会更简便.在串联电路中,等
效电阻就是指串联电路的总电阻,它在电路中起作用效果
跟原来几个电阻的共同作用效果相当,在电路计算中,我们
常用总电阻去代替原来的几个电阻,使问题简化.还应指出,
等效方法十分有用,在今后的学习中还会遇到,如电路分析中
的等效电路.
\item 讲解滑动变阻器作分压器使用时,最好配合实验演
示,以加强教学的直观性,可按教材图7.7接线,并在$cd$间
接上一个伏特表,尤其要对照电路图和实物突出滑动变阻器
的连接方法.当改变滑动端在两个固定端之间的位置时,让
学生观察伏特表示数的变化,再分析发生这种变化的原因,阐
明分压器原理.建议在讲完并联电路后,安排一次学生分组
实验“研究串联电路和并联电路”通过实验巩固串联、并联
电路的规律,进一步训练学生正确地使用安培表、伏特表、滑
动变阻器、电键等基本器材,为后面的学生实验作准备,实验
应把区分滑动变阻器的两种接法(制流电路和分压电路)作为
重要内容,具体电路如图7.1所示,比较这两种电路中滑动变
阻器的连接有什么不同?灯泡两端电压变化的范围有何不同?

\begin{figure}[htp]
    \centering
  \includegraphics[scale=.7]{fig/7-1.png}
    \caption{}
\end{figure}
\item 对串联电路的电压分配和功率分配关系可通过下面
的实验定性验证,把电阻值不同的灯泡(如“220V100W”和
“220V25W”)串联起来接入照明电路,每个灯泡两端各并联
一只交流伏特表,将使学生观察到阻值大的灯泡两端电压
大,消耗的功率多(灯泡亮);阻值小的灯泡两端电压小,消耗
的功率少(灯泡暗).教学中还应通过实例介绍运用串联电路
电压分配公式和功率分配公式的比例解题法的优越性.
\end{enumerate}

\subsubsection{并联电路}

推导出并联电路的总电阻计算公式以后,
教材根据
\[\frac{1}{R}=\frac{1}{R_1}+\frac{1}{R_2}+\cdots+\frac{1}{R_n}\]
直接得出结论:并联电路的
总电阻比每一个电阻都小.可引导学生从不同角度去认识这
个问题,以培养他们的发散性思维.一方面可根据电阻定律
$R=\rho\ell/S$
定性说明:几个电阻并联相当于增大了导体的横截
面积,总电阻当然减小.另一方面直接用总电阻计算公式讨
论.如果两个电阻$R_1$和$R_2$并联,由
$\dfrac{1}{R}=\dfrac{1}{R_1}+\dfrac{1}{R_2}+\cdots+\dfrac{1}{R_n}$
得总电阻
\[R=\frac{R_1R_2}{R_1+R_2}\]
再改写为
\[R=\frac{R_1}{R_1+R_2}R_2\quad \text{或}\quad R=\frac{R_2}{R_1+R_2}R_1\]
因为$R_1+R_2>R_1$, $R_1+R_2>R_2$, 所以$R<R_1$, $R<R_2$. 由此可类推多个
电阻并联后它们的总电阻必小于其中任何一支路的电阻.但
应注意防止学生把
\[\frac{1}{R}=\frac{1}{R_1}+\frac{1}{R_2}+\frac{1}{R_3}\]
误写为
\[R=\frac{R_1R_2R_3}{R_1+R_2+R_3}\]
通过以上分析再次强调无论串联电路还是并联电路,总电阻
实际上是指这个电路的等效电阻.

并联电路的电流分配关系和功率分配关系,可以用下面
的实验定性地验证.把阻值不同的灯泡(如“220V100W”
和“220V25W”)各串一个交流安培表后并联在照明电路
里,可观察到电阻值小的灯泡亮(功率大),通过的电流强;阻
值大的灯泡暗(功率小),通过的电流弱,同时应通过实例让
学生掌握运用并联电路的电流分配式和功率分配式的比例
解题法.

最后,应要求学生把串联电路和并联电路的特点作一比
较,使他们对串、并联电路能有一个全面的认识.

\subsubsection{分压和分流在伏特表和安培表中的应用}

这部分内
容是串联电阻分压和并联电阻分流的具体应用.因为学生已
具有一定的知识基础,为了使他们学习更主动,课堂教学的形
式最好活跃一点.例如采用教师引导下的讨论式教法.

首先指导学生观察大型演示用电流表,让他们对电流表
的大体结构以及测量电流、电压的基本原理有一个初步了解.
要注意强调电流表的内阻$R_g$和满度电流$I_g$仅决定于表的
内部结构.因此,对给定的一个电流表,它的这两个参量及最
大允许电压(满度电压)$V_g=I_gR_g$就是确定不变的.

其次,启发学生对下列问题展开讨论.
\begin{enumerate}
    \item 为什么电流表不能测量较大电压和较强电流?
    \item 如果要用电流表去测量较
大电压,并且能从刻度盘上直接读出待测电压值应当怎么办?假设有一个电流表,内阻$R_g=1000$欧,满度电流$I_g=100$微
安,要把它改装为量程是3伏的伏特表,试具体说明改装的方
案.
\item 如果要用电流表去测量较大电流,并且能从刻度盘上
直接读出待测电流值,应采取哪些措施?假如用上一问题中的电流表改装为量程是1安的安培表,试具体说明改装的方案.
\item 试分别概括电流表改装为伏特表和改装为安培表的原理及具体方法.
\end{enumerate}
通过学生讨论得出结论后,指导他们阅读教材
196页到198页的有关部分,检验,完善自己的看法,加深对问
题的理解.

然后,教师归纳小结,着重讲明以下几点.
\begin{enumerate}
\item 将电流表
改装成伏特表或改装为安培表,需分别串联一只分压电阻或
并联一只分流电阻,串联或并联的实际电阻值应根据分压原
理或分流原理,根据欧姆定律及需要扩大的量程进行计算,要
改装的伏特表量程越大,需串联的分压电阻值就应越大;要改
装的安培表量程越大,需并联的分流电阻值就要越小.要求
学生掌握的是改装的原理,分析、计算的方法,不必死记具体
计算公式,包括练习六第(2)题的关系式.
\item 电流表和串联
的分压电阻所构成的整体才叫伏特表,因此,伏特表刻度盘
上标出的伏特值,不表示加在电流表上的电压,而是直接表示
加在伏特表上的电压,在用伏特表测量一段电路的电压时,
伏特表指针的示数表示伏特表两端的电压,也表示待测电路
两端的电压.电流表和并联的分流电阻所构成的整体才叫安
培表,因此,安培表刻度盘上标出的安培值,不表示通过电流
表的电流,而是直接表示通过安培表的电流.
\item 电流表改装
为伏特表和改装为安培表的原理及计算方法同样适用于扩大
伏特表的量程和扩大安培表的量程.
\item 由于伏特表的内阻
很大,安培表的内阻很小,把它们接入待测电路时,一般对
原电路的电流、电压的分配影响不大.因此,题目如果没有明
确要求考虑伏特表的内阻和安培表的内阻时,通常可以认为
伏特表的内阻无穷大,安培表内阻为零.
\end{enumerate}

最后,还可以提出一些有启发性的问题供学生思考、讨
论.例如:用安培表和伏特表测量一段电路的电流强度和电压
时,若不慎将安培表并入电路或将伏特表串入电路,将会产生
什么后果?又如:试定性说明由于安培表、伏特表有内阻,将对
实际测量值带来什么影响?

\subsubsection{电路的分析和计算}

本节教材运用前面已学过的串
联电路和并联电路的知识,通过例题分析,介绍简单混联电路
的分析计算方法.对中学生来说,主要是教会他运用知识
分析问题的基本思路及基本方法,以利于巩固知识和培养能
力,不宜把过多力量用来教给学生某些具体的解题方法和
技巧.

掌握电路分析的方法是进行电路计算的基础.可以通过
举例,介绍识别简单混联电路中各部分间串、并联关系的基本
方法.如根据电流的分支与汇合,判断各电阻的串、并联关
系.若电流通过各电阻时没有分支,则这些电阻为串联;若电
流有分支,则从分流点到汇流点之间的各支路为并联.如果
电路中没有电源,可假设一个电流方向去判断.不过,应控制
教学深度,不宜要求学生掌握串、并联关系较难识别的电路
图,以免脱离教学要求,加重学生负担.

讲解教材上的两个例题,要重视启迪学生的思维.例如,
可采用教师启发讲解和学生讨论、练习相结合的教学方式.

在讲例题1时,教师首先应引导学生分析题意,画出电路
图,弄清电路中所有灯泡、输电线之间的串、并联关系.接着,
可依次提出以下问题启发学生积极思考,学习分析问题的方
法:怎样求每盏灯的实际功率1怎样求每盏灯的电阻?怎样求
每盏灯的实际电压?怎样求输电线的电压?怎样求输电线上的
电流强度?怎样求整个电路的总电阻?如何才能求出各灯泡并
联的电阻?分析问题的思路清楚了,可让学生自己进行计算.
然后,教师根据学生的计算结果提出问题供他们讨论.例如:
在电压不变的条件下,为什么多开灯比少开灯时每盏灯消耗
的功率小?在这两种情况中电源输出的功率是否相等?为什
么?使学生通过例题明确当部分电路的电阻发生变化时,必
然会引起整个电路的电流、电压、电功率发生变化的道理,懂
得这时应对电流、电压、电功率的分配重新计算.

讲解例题2时,先应引导学生认识接入伏特表后电路发
生了什么变化,认识伏特表的读数表示伏特表与10k$\Omega$电阻组
成的并联电路两端的电压.启发学生从例题1的结论-部
分电路的电阻值发生变化将引起整个电路的电压重新分配出
发,理解接入伏特表后,$a$、$b$间电压发生变化的原因,接着
指导学生自己根据串、并联电路的特点和欧姆定律进行具体
计算.然后再提出问题让学生讨论.伏特表接入电路时,测量
值比真实值偏大还是偏小?产生这种现象的原因是什么?为什
么选用内阻大的伏特表进行测量比较准确?教师应对学生讨
论的结果作出小结,使学生有一个完整的认识.伏特表接入
电路时,由于其内阻与被测电路并联的等效电阻小于被测电
路的电阻,若电源电压和另一个串联电阻不变,整个电路的电
压分配将发生变化,使并联电路分得的电压减小.
因此,伏特表的测量值将小于被测电路电压的真实值.伏特表的内阻比
被测电路的电阻大得越多,其并联等效电阻就越接近被测电
路电阻.使整段电路上电压分配的比例关系变化越小,伏特
表测量的误差就会越小.

综合两道例题的分析,最后应归纳进行电路计算的基本
思路:
\begin{enumerate}
\item 弄清电路中各部分之间的串、并联关系.
\item 根据题
目的已知条件对电路的局部和整体进行分析,从串、并联电路
的特点出发,找出该段电路和相邻电路的电流、电压关系.特
别要注意分析当电路连结方式改变或某个外电阻发生变化
时,电流、电压、功率分配的变化情况.
\item 正确选用基本公式
列出方程求解.
\end{enumerate}

\subsection{第四单元}
这一单元在引入电动势概念的基础上,把欧姆定律扩展
到包括电源在内的整个闭合电路,着重讲述闭合电路的欧姆
定律及其应用.这是本章的新知识,也是重点.对学生的分
析能力,推理能力要求较高.教学中应加强实验演示,强调物
理思考,注意训练分析问题的方法.

\subsubsection{电动势、闭合电路的欧姆定律} 

电动势是电学中的
一个重要概念,但它比较抽象难懂.为了减少学生学习的困
难,教材只要求学生知道电势反映了电源的一种特性,它的
大小等于外电路断开时两极间的电压,也等于外电路接通时
内、外电路上的电压之和,还从能量转化的角度讲解了电动势
的物理意义.教学中要掌握好分寸,不要在理论上补充、加深.

教材图7.25所示的实验有承上启下的作用,一是介绍
内电路、内电阻的概念;二是分析电源跟外电路接通后,两极
间电压小于电源电动势的原因;三是提出内、外电路上电势降
落之和应等于电源电动势的设想.

教材图7.26所示实验有助于学生进一步理解电动势的
意义,为讲述闭合电路欧姆定律奠定基础,因此,做好这个实
验(详见后面演示实验部分)是本节教学成功的关键,学生的
一个常见错误是认为接在电源两极间的伏特表的示数表示电
源内电压,在介绍实验装置时,可针对这个问题强调电源内
电压的测量方法.

在从能量转化角度来阐明电动势的物理意义时,应当进
一步指出:电源电动势是表示电源本身属性的物理量,即反映
电源把其他形式能量转化为电能的本领的大小,电动势越
大,表示电源把其他形式能量转化为电能的本领越大.因此,
对一个给定的电源,电动势有确定的数值,其大小只由电源的
性质和结构决定,而与外电性质以及是否接通没有关系.可
提出这样的问题启发学生思考、讨论.如:电动势为1伏,电
势降落为1伏,电势为1伏,各表示什么意思?这三个物理量
有什么区别?使他们弄清为什么$U+U'$不叫电动势,而只是
数值上等于电动势的道理.

闭合电路欧姆定律是关于电路的一条重要定律,要引导
学生理解它的物理意义,要指出闭合电路欧姆定律跟部分电
路欧姆定律的适用条件不同,前者适用于包括电源的整个闭
合电路,后者适用于不含电源的某一部分电路.应当通过教
材208页例题的分析引导学生认识:当外电路发生变化时,将
引起电路各部分的电流、电压重新分配.但是,电源电动势和
内电阻是保持不变的.这个特点对解电路计算问题十分重要.

在讲过闭合电路欧姆定律之后,应引导学生对整个电路
中能的转化情况作全面认识,学会从能的转化观点分析有关
电路的问题,如理解$I\mathcal{E}=IU+IU'$的物理意义,了解电源的
总功率、输出功率和内电路消耗的功率的概念和它们间的
关系.

\subsubsection{路端电压}

路端电压随外电阻而变化是一个重要问
题,讨论这个问题对巩固和运用闭合电路的欧姆定律很有好
处,所以教材把它单列一节讲述.

应用演示实验,给学生以鲜明的直观印象是十分重要的.
演示实验可用教材7.26的装置,在外电路上再串联一个安培
表,也可用教材图7.30所示电路演示.如果由于干电池内阻
小,实验效果不明显;不妨用一段阻值约十几欧的电炉丝把两
节干电池串联起来作为电源,并且使用的滑动变阻器的阻值
不要大(以0—50欧为宜).

在教学的具体安排上,先研究路端电压变化的一般规律,
再从一般到特殊,讨论断路、短路两种情况,讲解一般规律
时,一种办法是先通过实验演示,从实验现象的观察中总结出
路端电压变化的规律.再从理论上分析原因;另一种办法是
先从理论上用闭合电路欧姆定律分析路端电压的变化规律,
然后用实验来验证,并把理论分析的整个物理过程完整地表
现出来进行归纳小结.两种讲法各有其特点,但无论采用哪
种方法都应当引导学生认识以下几个问题.
\begin{enumerate}
\item 电源有内电
阻是路端电压变化的根本原因.
\item 不能用$U=IR$讨论路端电
压的变化.这是因为在闭合电路中决定电流强度的不是$U$和
$R$, 而是$\mathcal{E}$、$R$和$r$. 在电源确定的情况下,$\mathcal{E}$、$r$不变,由
\[I=\frac{\mathcal{E}}{R+r}\]
可看出外电阻$R$的变化直接改变电路中的电流,引
起内电压$Ir$的变化,从而影响路端电压,用$U=\mathcal{E}-Ir$来讨
论,容易看出路端电压的变化情况,若用$U=IR$来讨论,则
由于$R$增大,$I$就减小,到底$U$如何变化,无法判断.

\item 电源的外特性曲线($U$-$I$图象)是一条直线,它反映了整个闭合电
路中路端电压随电流的增大而线性减小,直线的斜率表示电
源的内电阻,直线在$U$轴上的截距表示电源电动势的大小.因
此,$U$-$I$图象反映了电源的特性.它与欧姆定律一节讲述的
反映导体导电性能的伏安特性曲线($I$-$U$图象)是有区别的.
\item 外电路某部分电阻发生变化时,判断路端电压及整个电路
的电流、电压分配的变化情况,可遵循以下步骤:弄清外电路
的串、并联关系,分析外电路总电阻怎样变化;由$I=\dfrac{\mathcal{E}}{R+r}$
确定闭合电路的电流强度如何变;由$U=\mathcal{E}-Ir$确定路端电压的
变化情况;用欧姆定律$U=IR$及分流、分压原理讨论各部分
电阻的电流、电压变化情况.让学生掌握从部分同整体的关
系上来分析电路的方法.
\end{enumerate}



断路和短路是路端电压随外电阻变化的两种特殊情况.
可提出以下问题引导学生通过讨论,自己得出结论.
\begin{enumerate}
\item 断路
和短路有什么不同?
\item 为什么用伏特表测出断路时的路端电
压并不准确等于电动势?怎样才能提高测量的精确程度?
\item 
有人说:断路时$I=0$, 由$U=IR$得出路端电压$U=0$, 这种说
法是否正确?为什么?
\item 短路时电源有什么危害?这时电源电
动势是否为零?在$U$-$I$图中怎样确定短路电流的大小?
\end{enumerate}


\subsubsection{电池组}

讲解这个问题可通过实例从使用电池组的
必要性出发引入新课.应当指出,电池存在允许通过的最大
电流是由于电源存在内电阻,电流流经内电阻会发热,若电
流太强,温升过高,将损坏电池.

讲解串联电池组的总电动势$\mathcal{E}_{\text{总}}=n\mathcal{E}$时,可把理论分析与
实验演示结合起来,边演示边分析.按教材7.33所示,用伏
特表测定每个电池的电动势,示数均为$\mathcal{E}$.说明断路时路端
电压等于电源电动势,每一电池正极的电势比它的负极高$\mathcal{E}$.
再用导线把伏特表正接线柱与第一个正极相连,将与伏特表
负接线柱的导线分别与第一个电池的负极和第二个电池的正
极相连,可观察到伏特表示数相同,说明前一个电池的负极
与后一个电池的正极电势相同.接着依次把伏特表负极与第
二个电池负极,第三个电池正极……等相连,逐一观察分析,
得出结论.

对串联电池组的使用,要引导学生认识以下几点,第一,
使用串联电池组的目的是为了获得较大的电动势,使用的条
件是:用电器额定电流小于单个电池允许通过的最大电流.第
二,不要把某些电池接反.可按教材图7.34演示、分析.弄
清这个问题将有助于今后理解矩形线框在垂直于线框平面
的匀强磁场中平动的总电动势为零的道理.第三,不要把新
旧电池混合起来串联使用.因为一节干电池电动势为1.5伏
时,内阻约0.5欧,在使用一段时间后,随电动势减小,内阻将
增大,当电动势降到1.1伏左右,内阻甚至可增大到几百欧.
这样一来,在内电阻上损耗的功率过大,是很不合算的.

要启发学生从用导线连接起来的所有极板电势都相等这
一事实去理解为什么并联电池组的总电动势等于单个电池的
电动势,应当注意,在中学阶段,我们仅研究相同电池的并
联,不涉及电动势和内电阻不同的电池的并联问题.使学
生理解,使用并联电池组的目的是为了给用电器提供较强的
电流.使用的条件是用电器的额定电压低于单个电池电
动势.

如果用电器的额定电压和额定电流都大于单个电池的电
动势和允许通过的最大电流应当怎么办呢?在讲解混联电池
组时,可先提出这样的问题组织学生利用刚学过的串联和并
联电池组的知识思考、讨论.要启发他们认识,需根据用电器
的额定电压和单个电池的电动势来确定串联电池的个数;根
据用电器的额定电流和单个电池允许通过的最大电流来确定
并联的组数,从而连成混联电池组.可通过具体例子加深学
生对混联电池组的理解.

\subsection{第五单元}

这一单元运用稳恒电流的基本理论讨论电阻的测量问
题.测量电阻的方法很多,这里主要介绍伏安法、欧姆表和惠
斯通电桥,着重研究它们的测量原理,分析测量误差产生的
原因,并提出减小误差的方法.教学中要特别注意启发、诱导
学生应用已学知识主动地分析、解决遇到的新问题,借以提高
他们运用知识的能力.

\subsubsection{伏安法}

伏安法测电阻的原理是欧姆定律,学生在初
中已学过,本节教材着重于运用串、并联电路的知识去分析
伏安法测电阻的误差.应提醒学生,这里所讲的误差不是指伏
特表、安培表的精度及读数引起的误差,而是由于伏特表、安
培表存在内阻,把它们连入电路中不可避免地要改变电路本
身,给测量结果带来的误差.

首先应引导学生了解伏安法测电阻有安培表外接(如教
材图7.37甲)和安培表内接(如图7.37乙)两种连接方式.其
次,引导学生讨论:这两种连接方法测电阻产生误差的原因是
什么?在什么条件下采用哪种接法,测量误差较小?并通过归
纳小结,使学生有一个较完整的认识.

采用安培表外接方式测量电阻时,伏特表的示数反映了
待测电阻两端电压的真实值,但由于伏特表有内阻,安培表的
示数$I$并不表示通过待测电阻的电流的真实值$I_R$, 而是表示
通过待测电阻和伏特表的总电流,即$I=I_R+I_V$. 这时待测
电阻的真实值$R=U/I_R$
和测量值$R'=U/I$就存在差异.因为$I>
I_R$, 所以$R'<R$. 可见,这种接法必然使测量值小于待测电阻
的真实值.引起误差的根本原因在于伏特表内阻的分流作
用,导致安培表示数大于待测电阻的电流.如果$R_V$越大,其
分流作用就会越小,安培表示数将越接近待测电阻的电流,
测量误差就会越小.当$R_V\gg R$时,$I_V\ll I_R$, 这时可认为
$I=I_R+I_V\approx I_R$, 则$R\approx R'$. 因此当$R_V\gg R$, 即测量小电阻
时,宜用安培表外接法(详见参考资料),采用安培表内接法测
电阻时,安培表的示数反映了待测电阻的电流真实值.但安
培表内阻的存在使伏特表示数$U$表示待测电阻两端电压与安
培表上电压降之和$U=U_R+U_A$. 这时待测电阻真实值$R=U_R/I_R$
就不等于测量值 $R'=U/I_R$.
因$U>U_R$, 故$R'>R$. 可见这
种接法必然使测量值大于待测电阻真实值.引起误差的根本
原因是安培表内阻的分压作用,导致伏特表示数大于待测电
阻两端电压,若$R_A$越小,其分压作用越小,伏特表示数就接
近待测电阻两端电压,测量误差就会越小,当$R_A\ll R$时,
$U_A\ll U_R$, 这时可认为$U=U_R+U_V\approx U_R$, 则$R\approx R'$. 因此,当
$R_A\ll R$, 即测量大电阻时,宜用安培表内接法.

\subsubsection{欧姆表}

讲解欧姆表的原理和构造时,应当注意以下
几个问题.
\begin{enumerate}
\item 讲解中需要有演示实验配合,以加深学生印
象,并对欧姆表的使用方法有一个初步了解.    
\item 应引导学生
认识欧姆表的设计思路.用伏安法测电阻比较麻烦,不仅需用
伏特表和安培表,而且要对测量结果进行计算才能得到待测
电阻值.如果在测量时将所用电压固定,直接用电流表的示
数来表示待测电阻的阻值大小,就能大大简化实验过程.欧
姆表就是基于这个思想设计的.   
 \item 欧姆表的工作原理是闭
合电路欧姆定律$$I=\dfrac{\mathcal{E}}{R_g+r+R+R_x}$$
式中$R_g+r+R$为欧姆
表内电阻.要强调欧姆表应有内电源,要装上干电池才能使
用.在万用表欧姆挡,红、黑表笔尽管分别插入标有“$+$”、“$-$”
号的插孔中,但这个“$+$”、“$-$”号并不表示内部电源的正、负
极.这两个符号表示万用表无论在欧姆挡、直流电流挡还是直
流电压挡,电流都应该从“$+$”插孔流入,从“$-$”插孔流出,以
保证表头指针向顺时针方向偏转.
\item 关于欧姆表的刻度,教
材中没有涉及.教师可根据学生实际酌情处理.一般说来,可
讲到这个程度:第一,红、黑表笔短路$R_x=0$, $I=\dfrac{\mathcal{E}}{R_g+r+R}$
此时电流最大.可调整$R$使指针满度,即$I=\dfrac{\mathcal{E}}{R_g+r+R}=I_g$,
指针所指的表盘上的满度位置可定为0欧,红、黑表笔断开
$R_x=\infty$, $I=0$, 指针不动,这时指针在表盘上所指的位置可定
为“$\infty$”,表示电阻无穷大.因此,欧姆表的刻度与其他表的刻
度相反.第二,由$I=\dfrac{\mathcal{E}}{R_g+r+R+R_x}$
可以看出电流强度$I$与
待测电阻阻值$R_x$之间不是线性变化关系.所以,欧姆表表盘
刻度是不均匀的(见参考资料).
\end{enumerate}

\subsubsection{惠斯通电桥}

惠斯通电桥的教学可以充分发挥学生
学习的主动性,在教师启发、引导下,多让学生活动.

首先引导学生回顾伏安法和欧姆表测电阻的误差原因、
然后提出设计任务,组织学生讨论.要求设计一个电路既能
避免伏特表分流、安培表分压的影响,又能消除电源电动势和
内电阻的变化对测量的影响,提高测量的精确程度,教师要
适时启发他们,在设计的新电路中应当没有伏特表和安培表,
也不能用电流的强弱来反映电阻的大小.逐步把学生的思路
引导到采用将待测电阻跟已知电阻相比较的方法上来,并指
出惠斯通电桥就是根据这一指导思想设计的.

在介绍惠斯通电桥电路图及电桥平衡条件的基础上,可
让学生自己利用电桥平衡条件和欧姆定律推导出确定待测电
阻的公式
\[R_x=\frac{R_2R_3}{R_1}\]

讲解惠斯通电桥测电阻的精确程度时,可组织学生议论:
影响惠斯通电桥测量精确程度的因素有哪些?理由是什么?然
后指出,要使测量结果精确,应选用准确程度高的电阻,如电
阻箱作为已知电阻;应选用表头灵敏度高的电流表作检流计.
在讲解滑线式电桥时应当把实物演示与讲述有机结合,

既分析构造和工作原理,又介绍操作程序和注意事项.要告
诉学生,教材图7.40所示电路图中的滑动变阻器有两个作
用:一是在开始实验时起限流作用,保护电流表;二是在实验
中通过减小它的电阻以提高$A$、$C$间电压,检验在电流表灵敏
度范围内电桥是否真正平衡,保证测量的精确度.但也要注
意,不能使流过各臂的电流太大以致各臂电阻发热,影响其阻
值的准确性,甚至烧坏桥臂.尤其要提醒学生,按下滑动触头
与电阻线$AC$只能瞬时接触,以免当$D$的位置跟恰使电桥平
衡的位置相距较大时,一开始就可能因电流过大使电流表损
坏.更不允许按下$D$后在$AC$上移动$D$的位置,这样才不致
破坏电阻丝的均匀性.

练习十一第5题和习题10、11题对培养学生解决实
际问题的能力很有训练价值,可视具体情况选择其中一、二题
作为课堂讨论,并用实验验证讨论结果.


\section{实验指导}
\subsection{演示实验}
\subsubsection{电功和电热}
将玩具电动机(1.5V—6V)、J2355型(0—50$\Omega$)滑动变阻
器、电键、大型演示用伏特表和安培表、干电池组或蓄电池组
(电源电动势视电动机规格确定).按图7.2所示电路连接.

先用手捏住电动机转轴使其不动,调节变阻器阻值让安
培表示数不要过大,记下此时安培表及伏特表示数$I_1$、$U_1$, 计
算出电动机电枢线圈的电阻$R=U_1/I_1$
松开手指让电动机转动,
可以看到安培表示数减小、伏特表示数增大,记下它们的值
$I_2$、$U_2$. 不难看出电动机两端电压$U_2>I_2R$. 通过实际计算还
可以看出,在相等时间$t$内,电功$U_2I_2t$大于电热$I_2^2Rt$.
\begin{figure}[htp]\centering
    \begin{minipage}[t]{0.48\textwidth}
    \centering
 \includegraphics[scale=.7]{fig/7-2.png}
    \caption{}
    \end{minipage}
    \begin{minipage}[t]{0.48\textwidth}
    \centering
 \includegraphics[scale=.7]{fig/7-3.png}
    \caption{}
    \end{minipage}
    \end{figure}

\subsubsection{闭合电路欧姆定律}
这个实验是要验证$\mathcal{E}=U+U'$. 由于普通蓄电池和伏打
电池的内阻很小,内电阻不易测出,因此做好这个实验的关键
在于增大电源内电阻.具体作法如下.

\paragraph{电源装置(图7.3所示)}

蓄电池式.在两个500毫升的烧杯内装入适量的浓
度为20\%的稀硫酸,再将盛满稀硫酸的U型玻璃管倒插入两
烧杯中(也可用过滤纸或浸透酸液的塑料海绵条,把两端分别
浸入两烧杯的酸液中,改变过滤纸或海绵条的数目,就能改变
电源内电阻),使两烧杯连通.取旧蓄电池极板或纯铅板两块
放在同一稀硫酸槽中组成内阻小的电源进行充电(这样可节
省充电时间).待充电结束,把两极板分别放入两烧杯酸液内
作为电源正、负极,组成电源.它的电动势约2伏.

取两支废圆珠笔芯去掉笔尖,在其塑料管上每隔2—3毫
米用针扎一个孔,把它们分别捆在正,负极板上,作为插入探
针的针管.用能插入针管的铜丝作探针,要在测量内电压时
才插进针管.

伏打电池式.电源装置和上述蓄电池式相仿,不同的
是用铜板和锌板作电源的正、负板.为了减小极板的极化,可
在稀硫酸中加入5\%左右的重铬酸钾或高锰酸钾作为去极剂.
这个装置不用充电,使用起来也比较方便.但由于伏打
电池电动势约为1.02伏,为使实验效果显著,应把普通大型
演示用电表作适当改装,即在灵敏电流计上串一只适当的分
压电阻,使伏特表的满刻度电压为1伏左右.

\paragraph{注意事项}
\begin{enumerate}
    \item 所使用的两个示教用大型伏特表应有相同的准确度
和合适量程.
\item 连接电路应注意两个伏特表的正、负极不要
接反.特别要注意内电路,靠近负极板的探针电势高于靠近
正极板的探针电势,所以前根探针应与伏特表$V_2$的正极相
连,后根探针则应与$V_2$的负极相连.
\item 滑动变阻器(或电阻
箱)的最大阻值应大于电源内阻,否则外电压不易测出.
\item 实
验前应将铅板(或铜板、锌板)、铜丝擦去氧化物.
\item 实验后要
将全部零件从酸液中取出,(对伏打电池,更需立即取出),用
清水冲洗干净,晾干备用.
\end{enumerate}

\subsection{学生实验}
\subsubsection{测定金属的电阻率}
这个实验涉及电阻定律、欧姆定律等基本规律,是具有长
度测量和电学测量综合性质的实验.上述知识都是学生已学
过的,因此教材要求学生自选器材、自行设计来进行实验.教
学中可以通过一系列问题启发学生思考、讨论,形成完整的设
计方案,再实际操作,方案设计建议分以下几步:
\begin{enumerate}
\item 弄懂实验原理.可提出下列问题,让学生思考:要
测出一段金属丝的电阻率,应当测量哪些物理量?怎样求出
电阻率?

\item 初步拟订实验方案.考虑到学校使用电表的量程,
教材提出,要测出一段长约0.5米,直径约0.3毫米.阻值约
3欧姆的金属导线(这是指锰铜丝,如选用其他材料或其他规
格的金属丝作待测对象,教师应先初测一下,不可照搬教材上
的数据)的电阻率.要求学生自己提出一个设计方案,包括选
用哪些实验器材、画出实验电路图、简要说明实验步骤,启发
学生相互交流、讨论.对电阻的测量,学生一般会根据初中的
知识,提出用伏特表和安培表来测量.如学生还提出其他方
法,可让他们进行讨论,判断是否可行.

\item 完善设计方案.从提高实验精确程度出发,引导学
生统一认识.用米尺(最小刻度为毫米)量金属丝的长度,用
千分尺(或游标卡尺)测金属丝直径(应及时复习一下有关测
量的读数方法),用伏安法测电阻.应指出用安培表外接形式
测量金属丝电阻精确程度较高(其道理可视学生情况作简要
分析,或完全回避,指出今后再学),可引导学生讨论:根据金
属丝电阻的大概数值结合教材提出的注意事项-电路中电
流不宜过大(要启发学生弄懂为什么),应选电动势为多大
的电源?量程为多大的伏特表和安培表?采用什么办法可以
控制电流不致过大?最后归纳学生意见并画出实验线路如图7.4所示.再简要概括出实验
步骤及实验注意事项.要提醒
学生,在测量各物理量时均应
读出估计数字来.
\end{enumerate}

\begin{figure}[htp]
    \centering
\begin{circuitikz}[european, >=latex]
\draw(0,0) to [battery2] (2,0) to [cute open switch](4,0) to [R] (6,0)--
(6,3) to [rmeter, t=V](0,3)--(0,2) to [rmeter, t=A](0,0);
\draw(0,2)to [R, *-*](6,2);

\node at (3,0)[below]{$K$};    
\draw[->](5.8,0)--(5.8,.7)--(5,.7)--(5,.2);
\end{circuitikz}

    \caption{}
\end{figure}

\subsubsection{把电流表改装为伏特表}
整个实验分三步进行,第一步,测定电流表的内电阻$r_g$.
第二步,根据测出的$r_g$, 计算分压电阻.并实际组装成伏特
表,第三步,把改装的伏特表跟标准伏特表校对,并计算相对
于满刻度的百分误差.

在指导学生实验时,需要注意以下问题:

应介绍电位器的使用方法.电位器有三个接线片,
可引导学生把电位器跟滑动变阻器比较:中间接线片相当于
滑动变阻器金属棒的接线柱,两边的接线片相当于滑动变阻
器中线圈的两接线柱.让学生知道电位器作分压器或可变电
阻使用应如何接线.

要引导学生弄懂用半偏法测电流表内阻的原理和条
件.在教材图9.7所示电路中,当断开$K_2$、闭合$K_1$调整
$R$使电流表指针恰偏转满度$I_g$时,电路中总电流强度$I=I_g$.
闭合$K_2$后,必然有一部分电流要流经$R'$, 使通过电流表的电
流减小,如能保持电路中总电流$I$不变,调节$R'$使通过电流
表的电流恰为$\frac{1}{2}I_g$, 由分流原理可知,这时$R'$的阻值一定等
于电流表内阻$r_g$, 只要读出电阻箱的电阻值就能得出电流表
内阻.

怎样才能保持总电流强度$I$不变呢?这就必须保持外电
路的总电阻不变.因为$K_2$闭合前外电路总电阻为$R+r_g$,
$K_2$闭合后调$R'$至$R'=r_g$时,外电路总电阻为$R+\dfrac{R'}{2}$,
所以
只有$R\gg R'$(即$R\gg r_g$) 时,才能认为外电路的总电阻基本没
有改变,也才能使电路的总电流强度在$K_2$闭合前后变化不
大,保证测量$r_g$的精确程度.因此,用半偏法测电流表内阻对
器材的要求就是$R\gg R'$.

由于一般电流表的内阻约100欧左右(如杭州电表厂的
J-DB$_2$XA型电流表,量限为$\pm300\mu {\rm A}$, 动圈内阻$92{\Omega}\pm 10$),
教材上说$R$可用470千欧的电位器,完全满足$R\gg R'$这个
条件.在准备学生实验时,根据学校现有器材怎样选用,才
能满足$R\gg R'$呢?事实上,只要$R\ge 100R'$就可以用$R'$代替
$r_g$; 它产生的相对误差不大于1\%,若受实验条件限制,取$R\ge 
50R'$, 测量的相对误差也不会超过2\%(见参考资料).这对
中学生分组实验的误差要求已基本可以了.

在讲解把改装成的伏特表跟标准伏特表校对时,应
启发学生认识教材图9.9所示电路中,滑动变阻器作
分压器使用是为了用一个阻值较小的滑动变阻器就能保证伏
特表两端的电压变化可以从零增加到2伏特,满足校对表的
要求.要提醒学生注意搞清改装后电流表上刻度的每一小格
表示的电压数,注意相对于满刻度的百分误差的计算方法.

在学生动手实验前,应告诫他们:闭合$K_1$前$R$应
调到最大值,以免闭合$K_1$后通过电流表的电流过大而损坏
表头;$K_2$闭合后,在调节$R'$时不能再改变$R$值,否则将改变
电路的总电流强度,影响实验的精确程度.

\subsubsection{用安培表和伏特表测定电池的电动势和内阻}
这个实验应注意以下几个问题.

实验器材的选择.要引导他们分析教材图
9.9所示电路,了解实验时应选用较大内阻的伏特表和较小
阻值的变阻器.伏特表内阻越大,变阻器阻值越小,安培表的
读数就越接近通过电池的真实电流,实验误差就会越小.但变
阻器阻值又不能太小,以免电流大于电池或变阻器允许通过
的最大电流,教师在准备实验器材时,干电池需选用已使用
过一段时间的干电池.以使测得的内阻适当大一些.由于现
在一般学校配备的安培表量程为0—0.6—3A,伏特表量程为
0—3—15V,为使读数比较准确,可用两节干电池串联作电
源,安培表量程选用0—0.6A挡,伏特表量程选0—3V挡,这
时伏特表内阻约1千欧,所以滑动变阻器宜选用J2355型
(电阻值约为0—50$\pm 10\%\Omega$, 额定电流1.5A).若变阻器阻值
选得过小(如用0—10$\Omega$的J2354型)则阻值变化范围太小,作
图表时示数值的点会过于密集,影响实验的准确性.

如何作图.应引导学生认识用作图法求电池电动势
和内电阻的优越性在于:可以省去解方程、求平均值的运算,
比较简捷地求出答案.要引导学生了解,在数据准确的情况
下怎样作图才能使实验结果更准确.第一,实验时滑动变阻
器的阻值应在几欧到几十欧的较大范围内变化,使各组数据
差别大一点,作图时才能使数据点适当拉开一些.第二,实验
数据要多取几组(至少5组).由于读数的偶然误差,描出的
点不在一条直线上,在作图时应通过尽可能多的点画一条直
线,并使不在直然上的各点分布在直线两侧的数目大致相等,
个别偏离直线太远的点,则舍去,不予考虑.这样才能使各次
实验的偶然误差得到部分抵消,以提高实验的精确程度,第
三,要适当选取横坐标$I$和纵坐标$U$的标尺比例和坐标起点,
使实验数据点大致布满整个图纸,不要集中在一边或一角.由
于这个实验的$U$值不宜过小,因此纵坐标$U$的起点可根据实
测数据从不为零的某一数值开始,但因为要用$I=0$时图线
在$U$轴上的截距来求电动势$\mathcal{E}$, 横坐标$I$必须以零为起点.

实验的注意事项.第一,接通电源前应将变阻器的
阻值调到最大值,避免$K$闭合时通过安培表的电流过大;在实
验中决不允许将滑动触头滑在使变阻器阻值为零的位置,造
成外电路短路.第二,利用$U$-$I$图象计算电池的内电阻时,
不能简单地在实验数据中任选两组$U$、$I$值分别相减再相除.
这样达不到取平均值,减小偶然误差的目的.也不能用量出
图线与横轴的夹角,通过查正切函数表求$r$的值.而应当在
图线上任选两个相距较远的点,计算图线的斜率$\Delta U/\Delta I$
以求得内阻$r$.


\subsubsection{练习使用万用表}
教材是以供中学实验用的J0411型万用表为例来介绍万
用表的使用方法,如果学校没有这种型号的万用表,可根据
教材的要求结合自己的具体情况做此实验.

本实验应该注意以下几个问题.

在介绍万用表的外形及测量前的准备工作时,可按
教材图9.10制作一张幻灯片或挂图,配合讲解.

应特别强调并严格要求学生遵守教材提出的测量
前,测量时的有关注意事项和操作程序.除此之外,还要提醒
学生:第一,测量电流和电压时应在用表笔接触测量点的同
时,注视电表指针的偏转情况,并随时准备在出现不正常现象
时,使表笔离开测量点.第二,用欧姆挡测电阻时,不得测额
定电流极小的电阻(如灵敏电流计的内阻);不得测带电的电
阻;如果待测电阻跟别的电路元件相连,应当把待测电阻同它
们断开,否则测出的阻值将包括待测电阻和其他元件在内的
等效电阻,第三,使用完毕,务必将万用表选择开关拨离欧姆
挡,应拨到空挡或最大交流电压量程处.

具体练习使用万用表时,建议通过测量直流电压,直
流电流和电阻,结合验证串、并联电路的特点.下面给出参考
电路图,实验器材的规格可根据各校实际自行选择,但注意不
要超过万用表各挡的量程.

测量电流时,按图7.5甲所示电路实验,分别测出干路的
总电流强度$I$和各支路电流强度$I_1$、$I_2$, 验证$I=I_1+I_2$.

\begin{figure}[htp]
    \centering
\begin{circuitikz}[european]
      \begin{scope}
\draw(0,0) to [battery2] (2,0) to [cute open switch](4,0)--(4,3) to [R=$R_1$](0,3)--(0,0);
\draw(4,1.5)to [R=$R_2$, *-*](0,1.5);
\node at (3,0)[below]{$K$};
\node at (2,-1){甲};
\end{scope}  
\begin{scope}[xshift=5cm]
\draw(0,0) to [battery2] (2,0) to [cute open switch](4,0)--(4,3) to [R=$R_2$](2,3) to [R=$R_1$](0,3)--(0,0);
\node at (3,0)[below]{$K$};
\node at (2,-1){乙};
\end{scope}
\end{circuitikz}    
    \caption{}
\end{figure}

测量电压时,先按图7.5乙所示电路分别测出$R_1$、$R_2$两
端的电压$U_1$和$U_2$以及总电压$U$, 验证$U=U_1+U_2$. 再按图
7.5甲所示电路分别测出$R_1$、$R_2$两端电压,验证并联电路两
端电压相等,$U_1=U_2$.

测电阻时,先分别测出$R_1$、$R_2$的阻值,接着把$R_1$跟$R_2$串
联测出总电阻$R$, 验证$R=R_1+R_2$. 再把$R_1$跟$R_2$并联,测出
总电阻$R'$, 验证$R'=\dfrac{R_1R_2}{R_1+R_2}$.
为了提高测量电阻的精确程度,
选取的欧姆挡量程应使指针的偏转角度尽量靠近表盘中间
(见参考资料).例如,测量1千欧的电阻时,可供选用的量程
有“$\x$1k”、“$\x$100”、和“$\x$10”,但以选用“$\x$100”的量程最为
适宜.因为这时指针的位置相对而言最靠近表盘中心.

\subsubsection{用惠斯通电桥测电阻}
对于没有滑线式电桥供学生实验的学校,可以选用以下
器材自行组装.滑线采用直径0.3毫米,有效长度0.5米(阻
值约3.2$\Omega$)的锰铜丝,张挂在0.5米长的刻度尺上,刻度尺的
最小分度为1毫米.滑键采用铜片自制.滑动变阻器要选用
阻值比锰铜丝电阻大得多的规格(如0—50$\Omega$). 已知电阻$R$
用0—9999.9$\Omega$的电阻箱.待测电阻$R_2$选用阻值约几千欧
的定值电阻,电源用3—4.5伏干电池组.把它们和灵敏电
流表、电键K按教材图9.11接线,组成电桥.

应结合实物采用边讲边演示的方式,引导学生弄懂教材
280页到281页提出的问题;强调实验操作程序和要求,并严
格要求学生按这些规定进行实验.

实验中应注意的问题.
\begin{enumerate}
\item 通过锰铜丝的电流不宜太大,
连续使用的时间不能过久,否则,锰铜丝会明显发热变长,影
响测量效果.因此,滑动变阻器的阻值不要减小太多.    
\item 滑
动触头要能与锰铜丝接触良好,以减小因接触电阻引起的误
差.    
\item 锰铜丝要保持粗细均匀,如表面生锈或有碰伤,应更换新的锰铜丝.
\end{enumerate}

\subsection{课外实验活动}
\subsubsection{自制电池}
不同金属具有不同的标准电极电位,只要把不同的金属
放在电解质溶液中,由于化学作用,就会产生电动势,成为一
个化学电源——电池.从原则上讲,任何两种不同金属都可
以作为电池的两个电极,一般说来,活动性较大的金属是负
极,活动性较小的金属是正极.但是,要获得好的实验效果,
应该选择金属活动性相差较大的两种作为电池的两极.这是
因为它们的标准电极电位相差较大,产生的电动势较大.教
材介绍用锌片、铜片及吸满盐水的吸水纸自制电池,实际上就
是一个最原始的伏打电堆.这种电池的作用机理超出了高中
化学的内容,不必从化学角度对这个问题作过多的阐述.

\subsubsection{研究电灯泡的电阻}

这个实验是要使学生进一步认识,金属电阻率随温度升
高要增大.用电灯泡的额定功率和额定电压计算出的电阻,
是灯泡在正常发光的温度下灯丝具有的电阻值.如果灯泡两
端的电压不等于额定电压,灯丝的电阻也会相应发生变化.


\section{习题解答}

\subsection{练习一}
\begin{enumerate}
    \item 导线中的电流强度为10安,20秒钟内有多少电子通过导线的横截面?
    
\begin{solution}
    根据
    $I=q/t,\quad q=ne$
    可得电子数
    \[n=\frac{It}{e}=\frac{10\x20}{1.60\x10^{-19}}
    =1.3\x10^{21}\text{(个)}\]
\end{solution}
    \item 手电筒小灯泡上的电压是3伏时,电阻为8.5欧,求通过小灯泡的电流强度.
    
    \begin{solution}
        由欧姆定律
       $ I=U/R$
        得:通过小灯泡的电流强度
    \[I=\frac{3}{8.5}=0.35{\rm A}\]    
    \end{solution}
    \item 人体通过50毫安的电流时,就会引起呼吸器官麻痹,如果人体的最小电阻为800欧,求人体的安全工作电压.
    
    \begin{solution}
        由欧姆定律
       $ I=U/R$,可知人体的安全工作电压为
       \[U=IR=50\x 10^{-3}\x 800=40{\rm V}\]
    \end{solution}
    \item 根据上题中所给的数字说明:为什么人体触到220伏的电线时会发生危险,而接触干电池的两极(电压为1.5伏)时却没有感觉?
    
    \begin{solution}
          人体触到220伏的电线时,通过人体的电流强度
可达
\[I=\frac{U}{R}=\frac{220}{800}=0.275{\rm A}=275{\rm mA}\]
远大于50毫安,所以会发生危险.

人体接触干电池的两极时,通过人体的电流强度仅为
\[I=\frac{U}{R}=\frac{1.5}{800}=0.0019{\rm A}=1.9{\rm mA}\]
远小于50毫安,所以没有感觉.      
    \end{solution}
    \item 电路中有一电阻,测得通过它的电流强度是2毫安时,电阻两端的电压是50毫伏,在通过它的电流强度为15毫安时,它两端的电压是多大?
    
    \begin{solution}
        由欧姆定律$ I=U/R$,求出电阻的阻值
        \[R=\frac{U_1}{I_1}=\frac{50\x 10^{-3}}{2\x 10^{-3}}=25\Omega\]
        再由欧姆定律求出在通过该电阻的电流强度为15毫安时,电
        阻两端的电压
        \[U_2=I_2R=15\x10^{-3}\x25=375\x10^{-3}{\rm V}=
        375{\rm mV}\]

        也可用比例关系解此题.由欧姆定律知:当导体电阻一
        定时,通过导体的电流强度跟导体两端电压成正比.
     \[   I_1:I_2=U_1:U_2\]
        则
  \[      U_2=\frac{I_2U_1}{I_1}=\frac{15\x50}{2}{\rm mV}=375{\rm mV}\]
    \end{solution}
    \item 画出电限为5欧的导体的伏安特性曲线,当导体的电阻增大为10欧时,图线持怎样变化?电阻减小为2.5欧时呢?
    
    \begin{solution}
        电阻为5欧、10欧和2.5欧时的伏安特性曲线分别
        如图7.6所示.
\begin{figure}[htp]
    \centering
\begin{tikzpicture}[>=latex]
\draw[<->, thick](0,4)node[right]{$I(A)$}--(0,0)--(4,0)node[right]{$U(V)$};
\foreach \x/\xtext in {1/5,2/10,3/15}
{
    \draw(\x,0)node[below]{$\xtext$}--(\x,.1);
    \draw (0,\x)node[left]{$\x$}--(.1,\x);
}
\draw[dashed](0,2)--(2,2)--(2,0);
\draw[dashed](1,0)--(1,2);
\draw[dashed](0,1)--(2,1);
\draw[domain=0:1.7, samples=10, thick] plot(\x, {2*\x})node[above]{$R=2.5\Omega$};
\draw[domain=0:2.5, samples=10, thick] plot(\x, {\x})node[right]{$R=5\Omega$};
\draw[domain=0:3, samples=10, thick] plot(\x, {.5*\x})node[right]{$R=10\Omega$};
\node at (-.2,-.2){$O$};
\end{tikzpicture}
    \caption{}
\end{figure}
从图中可以看出:当导体电阻由5欧增大为10欧时,图
线的斜率变小;当电阻减小为2.5欧时,图线斜率增大.
    \end{solution}
\end{enumerate}




\subsection{练习二}

\begin{enumerate}
    \item 图7.7是滑动变阻器的结构图,涂有绝缘漆的电阻丝密绕在瓷管上,$A$、$B$是它的两个端点,滑动端$P$可在金属杆上移动,它通过金属片与电阻丝接触,把电阻丝和金属杆连接起来.如果把固定端$A$和接线柱$C$接入电路中,当滑动端从$B$向$A$移动时,电路中的电阻就随着变小,说明其道理.
    \begin{figure}[htp]\centering
       \includegraphics[scale=1.5]{fig/7-4.PDF}
        \caption{滑动变阻器}
        \end{figure}

\begin{solution}
    根据电阻定律$R=\rho\ell/S$,
    可知均匀电阻丝的电阻跟它
    的长度成正比.当滑动端$P$从$B$向$A$移动时,接入电路中的
    电阻丝长度变短,所以电阻随着变小.
\end{solution}

    \item 一卷铝导线长100米,横截面积为1${\rm mm}^2$,这卷导线的电阻是多大?

    \begin{solution}
        由电阻定律得$R=\rho\ell/S$,
        查表可知铝的电阻率$\rho=
        2.9\x10^{-8}\Omega\cdot {\rm m}$,则这卷铝线的电阻
\[R=2.9\x 10^{-8}\x\frac{100}{1\x 10^{-6}}=2.9\Omega\]
    \end{solution}
    
    \item 有一段导线,电阻是4欧,把它对折起来作为一条导线用,电阻是多大?如果把它均匀拉长到原来的两倍,电阻又是多大?

    \begin{solution}
已知$R=\rho\ell/S=4$欧,把它对折后$\ell_1=\frac{1}{2}\ell$,
$S_1=2S$,则这时导线电阻
\[R_1=\rho\frac{\frac{1}{2}\ell}{2S}=\frac{1}{4}R=1\Omega\]

把它均匀括长到原来的两倍后,$\ell_2=2\ell$, $S_2=\frac{1}{2}S$,则这时导线电阻
\[R_2=\rho\frac{2\ell}{\frac{1}{2}S}={4}R=16\Omega\]
    \end{solution}
    
    \item 一根做电学实验用的铜导线,长度是60厘米,横截面积是0.5${\rm mm}^2$,它的电阻是多少欧?一根输电用的铜导线,长度是10千米,横截面积是1${\rm cm}^2$,它的电阻是多少欧?为什么做电学实验时可以不考虑连接用的铜导线的电阻,而对输电线路的导线的电阻则需要考虑?

    \begin{solution}
做电学实验用的铜导线的电阻
\[R_1=\rho\frac{\ell_1}{S_1}=1.7\x 10^{-8}\x \frac{0.6}{0.5\x 10^{-6}}=0.02\Omega\]

输电铜导线的电阻
\[R_2=\rho\frac{\ell_2}{S_2}=1.7\x 10^{-8}\x \frac{10\x 10^{3}}{1\x 10^{-4}}=1.7\Omega\]
从计算结果可以看出:做电学实验时,连接用的铜导线的电阻很小,对电路中电流强度的影响甚微,因此可以不予考
虑;而输电线的电阻较大,足以影响电路中的电流强度,所以
需要考虑.
    \end{solution}
    
    \item 一根电阻丝长10米,横截面积是0.2${\rm mm}^2$,两端加上10伏电压时,通过的电流强度是0.2安.这根电阻丝的电阻率是多大?它是用什么材料制作的?

    \begin{solution}
由欧姆定律$I=U/R$,知电阻丝的电阻
\[R=\frac{U}{I}=\frac{10}{0.2}=50\Omega\]
又因$R=\rho\ell/S$,则电阻丝的电阻率
\[\rho=\frac{RS}{\ell}=\frac{50\x0.2\x10^{-6}}{10}=1.0\x 10^{-6}\Omega\cdot {\rm m}\]
查表得知它是用镍铬合金制作的.
    \end{solution}
    
\end{enumerate}



\subsection{练习三}
\begin{enumerate}
    \item 额定电压相同、额定功率不同的两只灯泡,哪个的额定电流大?哪个的电阻大?

    \begin{solution}
    由$P=UI$, 有$I=P/U$.
可见额定电压相同时,灯泡的
额定电流与额定功率成正比,额定功率大的灯泡额定电流大.
因灯泡是纯电阻元件,故$P=U^2/R$, 
即$R=U^2/P$.
可见电阻与额定功率成反比,额定电压相同的两只灯泡,额定功率小的灯泡电阻大.
    \end{solution}
    
    \item 用户保险盒中安装的保险续允许通过的最大电流一般都不大(几个安培),如果在电路中接入功率在1000瓦以上的用电器,如电炉等,就会把保险丝烧断,这是为什么?

    \begin{solution}
因为功率为1000瓦以上的用电器的工作电流为
$I=P/U=\frac{1000}{220}=4.55{\rm A}$以上,把它接入电路后,如果电路中的
电流强度超过保险丝允许通过的最大电流(几安培),就会把
保险丝烧断.
    \end{solution}
    
    \item 日常使用的电功单位是“度”,等于功率为1千瓦的电流在1小时内做的功,又叫千瓦时,1度等于多少焦?

    \begin{solution}
        因$W=Pt$, 故$1\text{度}=1000{\rm W}\x3600{\rm s}=3.6\x10^6{\rm J}$
    \end{solution}
    
    \item 有一个1千瓦、220伏的电炉,正常工作时电流是多少?如果不考虑温度对电阻的影响,把它接在110伏的电压上,它的功率将是多少?

    \begin{solution}
因$P=UI$, 故正常工作时,通过电炉的电流是
\[I=\frac{P}{U}=\frac{1000}{220}=4.55{\rm A}\]
该电炉的电阻为$R=U/I=\frac{220}{4.55}=48.4\Omega$.
把它接到110伏的电压上,它的功率将是
\[P'=\frac{U'}{R}=\frac{110^2}{48.4}=250{\rm W}\]
    \end{solution}
    
    \item 输电线的电阻共计1.0欧,输送的电功率是100千瓦,用400伏的低压送电,输电线上发热损失的功率是多少千瓦?改用1万伏的高压送电呢?

    \begin{solution}
因$P=UI$, 故输电线上电流为$I=P/U$,则输电线上发热损失的功率
\[P_{\text{热}}=I^2R_{\text{线}}=\left(\frac{P}{U}\right)^2R_{\text{线}}\]
当用400伏低压送电时,
\[P_{\text{热}}=\left(\frac{100\x 10^3}{400}\right)^2\x 1.0=6.25\x 10^4{\rm W}=62.5{\rm kW}\]
当用1万伏高压送电时
\[P'_{\text{热}}=\left(\frac{100\x 10^3}{10^4}\right)^2\x 1.0=100{\rm W}=0.1{\rm kW}\]
    \end{solution}
    
    \item 用功率为2千瓦的电炉把2千克的水从20$^\circ$C加热到100$^\circ$C,如果电炉的效率为30\%,需要多少时间?水的比热为$4.2\x10^3{\rm J}/({\rm kg}\cdot {}^\circ {\rm C})$.

    \begin{solution}
把2千克水从$20^{\circ}{\rm C}$加热到$100^{\circ}{\rm C}$需吸收热量
$Q=cm\Delta T$.

设电炉需通电的时间为$t$秒,则电功为
$W=Pt$.

由题设条件电炉的效率$\eta=0.3$, 故
$\eta Pt=cm\Delta T$
所以
\[t=\frac{cm\Delta T}{\eta P}=\frac{4.2\x10^3\x2\x(100-20)}{0.3\x 2000}=1120{\rm s}\]
    \end{solution}
    
\end{enumerate}


\subsection{练习四}
\begin{enumerate}
    \item 电炉和导线是串联的,把它们接入电源后,导线和电炉丝中通过的电流强度是否一样?为什么这时电炉丝热得发红,导线并不热?

    \begin{solution}
因为流过串联电路各电阻的电流强度相等,所以导线
和电炉丝中通过的电流强度是一样的.但由于电流通过导体
产生的热量$Q=I^2Rt$与导体的电阻$R$成正比,电炉丝的电阻
远比导线电阻大且散热又不好,故电炉丝热得发红,而导线并
不热.    
    \end{solution}
    \item 某同学要为游艺晚会准备一棵装有彩色电灯的小松树,如果所用的每只灯泡的额定电压是8伏,用220伏的市电做电源,那么需要将多少只灯泡串联在一起才能接在电源上?

    \begin{solution}
 设需将$n$只灯泡串联起来接在电源上.
因为$U=nU_1$,则
\[n=\frac{U}{U_1}=\frac{220}{8}\approx 28\text{(只)}\]   
    \end{solution}
    \item 一个量程为150伏的电压表,内阻为20千欧,把它与一高电阻串联后接在110伏的电路上,电压表的读数是5伏.求高电阻的阻值是多少?(这是测量高电阻的一种方法)

    \begin{solution}
电压表的读数就是电压表两端的电压.

设电压表内阻为$R_1$, 其两端电压为$V_1$; 高电阻的阻值为
$R_2$, 两端的电压为$U_2$, 串联电路的电流强度为$I$.

因为串联电路两端电压等于各部分电路电压之和,$U=
U_1+U_2$, 所以可求出$R_2$上的电压
\[U_2=U-U_1=110-5=105{\rm V}\]
从$U=IR$, 可得串联电路中的电流强度
\[I=\frac{U_1}{R_1}=\frac{5}{20\x10^3}=2.5\x 10^{-4}{\rm A}\]
于是,高电阻
    \[R_2=\frac{U_2}{I}=\frac{105}{2.5\x 10^{-4}}=4.2\x 10^{5}\Omega=420{\rm k}\Omega\]

    $R_2$也可以利用串联电路的电压分配公式求得.因为
\[\frac{U_1}{R_1}=\frac{U_2}{R_2},\qquad U=U_1+U_2\]
由此可得:
\[R_2=\frac{(U-U_1)R_1}{U_1}=\frac{(110-5)\x 20}{5}{\rm k}\Omega=420{\rm k}\Omega\]
    \end{solution}
    \item 直流电动机线圈的电阻很小,所以起动时的电流很大,这对电动机本身和接在同一电源上的其他用电器都产生不良的后果.为了减小电动机起动时的电流,需要给电动机串联一个起动电阻$R$,如图7.8所示,电动机起动后再将$R$逐渐减小,如果电源电压$U=220$伏,电动机的线圈电阻$r_0=2$欧,那么,
    \begin{enumerate}
        \item 不串联电阻$R$时的起动电流是多大?
        \item 为了使起动电流减小为20安,起动电阻应为多大?
    \end{enumerate}

\begin{figure}[htp]
\centering
\begin{minipage}[t]{0.48\textwidth}
\centering
        \begin{circuitikz}[european,>=stealth]
            \draw(0,0)--(5,0) to  (5,3);
            \draw (0,3) to [vR=$R$] (5,3);
            
            \draw[<->](0,.1) to node [fill=white]{$U$} (0,3-.1);
            \draw (5,1.5) node[elmech]{};
            \draw [fill=white](0,0) circle (2pt);
            \draw [fill=white](0,3) circle (2pt);  
            \node at (5.7, 1.5){$r_0$};
        \end{circuitikz}
    
        \caption{}
        \end{minipage}
\begin{minipage}[t]{0.48\textwidth}
\centering
\begin{circuitikz}[european,>=latex]
    \draw (0,0)--(2,0) to [R=$R_3$] (2,2) to [R=$R_2$] (2,4) to [R=$R_1$] (2,6)--(0,6);
    \draw (2,0)--(4,0)node[right]{$b$};
    \draw [<-](2.2,3)--(4,3)node[right]{$a$};
    \draw [fill=white](4,3) circle (2pt);
    \draw [fill=white](4,0) circle (2pt);      
    \draw [fill=white](0,0) circle (2pt);
    \draw [fill=white](0,6) circle (2pt);   
    \draw[<->](0,.1) to node [fill=white]{$U$} (0,6-.1);
    \node at (2.4,2.7){$P$}; 
        \end{circuitikz}
\caption{}
\end{minipage}
\end{figure}

    \begin{solution}
根据欧姆定律,不串联起动电阻$R$时的起动
电流
\[I=\frac{U}{r_0}=\frac{220}{2}=110{\rm A}\]

串联起动电阻后,电路的总电阻
\[R_{\text{总}}=\frac{U}{I}=\frac{220}{20}=11\Omega\]
因$R_{\text{总}}=R+r_0$, 故起动电阻$R=R_{\text{总}}-r_0=11-2=9\Omega$
    \end{solution}
    \item 图7.9是一个变阻器分压电路,如果电压$U=12$伏,$R_1=350$欧,$R_2=270$欧,$R_3=550$欧,那么,滑动端$P$从$R_2$下端向上移动时,$a$、$b$间的电压将怎样变化?当$P$在$R_2$最下端和最上端时,$a$、$b$间的电压各是多少?

    \begin{solution}
   滑动端$P$从$R_2$下端向上移动,$a$、$b$间的电压将
增大.

当$P$在$R_2$最下端时,$a$、$b$间电压即为$R_3$两端电压.因为
串联电路的电流强度 
\[I=\frac{U}{R_1+R_2+R_3}\]
故\[U_{ab}=IR_3=\frac{UR_3}{R_1+R_2+R_3}=\frac{12\x 550}{350+270+550}=5.6{\rm V}\]

当$P$在$R_2$最上端时,$a$、$b$间电压即为$R_2+R_3$两端的
电压,则
\[U'_{ab}=I(R_2+R_3)=\frac{U(R_2+R_3)}{R_1+R_2+R_3}=\frac{12\x (270+550)}{350+270+550}=8.4{\rm V}\]
    \end{solution}	
\end{enumerate}





\subsection{练习五}
\begin{enumerate}
    \item 电路中需要一个阻值为15千欧的电阻,现在手边只有几只10千欧的电阻,怎样才能组成一个15千欧的电阻?
    

    \begin{solution}
        两只10千欧的电阻并联,其等效电阻
        \[R_{\text{并}}=\frac{1}{2}R=\frac{1}{2}\x 10=5{\rm k}\Omega\]
        再把它跟一只10千欧的电阻串联,总电阻
        \[R_{\text{总}}=R+R_{\text{并}}=10+5=15{\rm k}\Omega\]
        符合题目的要求.   
    \end{solution}
    
    \item 在图7.10所示的电路中,$R_1=10$欧,$R_2=30$欧,$U=6$伏,电键$K$合上前后,
    \begin{enumerate}
        \item 电路中的总电阻各是多少?
        \item 通过$R_1$、$R_2$的电流强度各是多少?
        \item $R_1$、$R_2$上消耗的功率各是多少?
    \end{enumerate}
    \begin{figure}[htp]\centering
        \begin{circuitikz}[european, >=latex]
            \draw (0,0) -- (5,0) to [R=$R_2$] (5,2.5) to [cute open switch] (5,4)-- (0,4);
            \draw (3,0) to [R=$R_1$, *-*] (3,4);
            \draw [fill=white](0,4) circle (2pt);
            \draw [fill=white](0,0) circle (2pt);
    \draw[<->](0,.1)--node [fill=white]{$U$} (0,4-.1);
            \node at (5.3, 3.5){$K$};
        \end{circuitikz}
    
        \caption{}
    \end{figure}	

    \begin{solution}
\begin{enumerate}
    \item 电键$K$闭合前,电路中只有电阻$R_1$, 故总电阻
\[R=R_1=10\Omega\]
电键$K$闭合后,$R_1$与$R_2$并联,总电阻
\[R'=\frac{R_1R_2}{R_1+R_2}=\frac{10\x 30}{10+30}=7.5\Omega\]
\item 电键$K$闭合前,通过$R_1$的电流
\[I_1=\frac{U_1}{R_1}=\frac{6}{10}=0.6{\rm A}\]
通过$R_2$的电流$I_2=0$

电键$K$闭合后,通过$R_1$的电流
\[I'_1=\frac{U}{R_1}=\frac{6}{10}=0.6{\rm A}\]
通过$R_2$的电流
\[I'_2=\frac{U}{R_2}=\frac{6}{30}=0.2{\rm A}\]
\item 电键$K$闭合前,$R_1$上消耗的功率
\[P_1=I_1^2R_1=0.6^2\x10=3.6{\rm W}\]
$R_2$上消耗的功率$P_2=0$.

电键$K$闭合后,$R_1$上消耗的功率
\[P_1'={I_1'}^2R_1=0.6^2x10=3.6{\rm W}\]
$R_2$上消耗的功率 
\[P_2'={I_1'}^2R_2=0.2^2\x30=1.2{\rm W}\]
\end{enumerate}
    \end{solution}
	


    \item 在图7.11所示的电路中,要使通过$R_1$的电流强度不超过5毫安,分流电阻$R_2$应为多大?

    \begin{solution}
因为$R_2$与$R_1$并联,所以
\[U_2=U_1=I_1R_1=5\x10^{-3}\x100=0.5{\rm V}\]
又因通过$R_2$的电流强度$I_2=I-I_1=3-5\x10^{-3}=3{\rm A}$
故分流电阻
\[R_2=\frac{U_2}{I_2}=\frac{0.5}{3}=0.167\Omega\]
    \end{solution}

    \begin{figure}[htp]
        \centering
        \begin{minipage}[t]{0.48\textwidth}
        \centering
                \begin{circuitikz}[european, >=latex, scale=.8]
            \draw (0,0) -- (5,0) to [R=$R_1$] (5,4)--(0,4);
            \draw (3,0) to [R=$R_2$, *-*] (3,4);
            \draw [->] (.5,4)--(1.5,4)node [above]{3A};
            \node at (6.2, 2.4) {5mA};
            \node at (6.2, 1.6) {100$\Omega$};
            \draw [fill=white](0,4) circle (2pt);
            \draw [fill=white](0,0) circle (2pt);
                \end{circuitikz}
            
                \caption{}
                \end{minipage}
        \begin{minipage}[t]{0.48\textwidth}
        \centering
            \begin{circuitikz}[european, >=latex, scale=.8]
                \draw (0,0) -- (5,0) to [R=$R_2$] (5,4) to  [R=$R_1$]  (0,4);
        \draw (5, 1)--(6,1)--(6,3)--(5,3);
        \draw[<->](0,.1)--node [fill=white]{$U$} (0,4-.1);
                \node at (6.3, 2){$L$};
                \draw [fill=white](0,4) circle (2pt);
                \draw [fill=white](0,0) circle (2pt);
                \draw [fill=black](5,1) circle (2pt);
                \draw [fill=black](5,3) circle (2pt);
            \end{circuitikz}
            \caption{}
        \end{minipage}
            \end{figure}

    \item 在图7.12所示的电路中,$L$是跟$R_2$并联的一条导线,下列说法哪些是正确的?

    
    \begin{enumerate}
        \item 通过$R_1$、$R_2$的电流强度$I$相等:
        \[I=\frac{U}{R_1+R_2} \]
        \item $R_1$上的电压$U_1=R_1I_1$,$R_2$上的电压$U_2=R_2I$,导线$L$中的电流为零.
        \item $R_1$上的电压$U_1=U$,$R_2$上的电压为零.
        \item $R_1$中的电流强度$I_1=U/R_1$,导线$L$中的电流强度等于$I_1$;$R_2$中的电流强度为零.
\item $R_2$去掉后,电路中的电阻和电流强度不发生变化.
    \end{enumerate}

    \begin{solution}
(c)、(d)、(e)是正确的.

提示:导线$L$的电阻可忽略不计,即认为$R_L=0$.
    \end{solution}
\end{enumerate}



\subsection{练习六}

\begin{enumerate}
    \item 有一个电流表,内阻为100欧,满度电流为3毫安,要把它改装成量程为3安的安培表,需并联多大的分流电阻?要把它改装为6伏的伏特表,需串联多大的电阻?

    \begin{solution}
在测量了安培的电流时,分流电阻$R$上通过的电流
应该是
\[I_R=I-I_g=3-3\x10^{-3}=2.997{\rm A}\]
由于并联电路中电流强度跟电阻成反比
\[I_gR_g=I_RR\]
所以,需并联的分流电阻值
\[R=\frac{I_g}{I_R}R_g=\frac{3\x10^{-3}}{2.997}\x100=0.1\Omega\]

在测量6伏的电压时,分压电阻两端的电压应该是
\[U_R=U-I_gR_g=6-3\x10^{-3}\x100=5.7{\rm V}\]
所以,需串联的分压电阻值
\[R=\frac{U_R}{I_g}=\frac{5.7}{3\x10^{-3}}=1.9\x10^3\Omega\]
    \end{solution}
    
    \item 电流表的内阻为$R_g$,满度电流为$I_g$.试证明:
     \begin{enumerate}
         \item 要把它改装成量程为$U=nI_gR_g$的伏特表,串联电阻的阻值应为$R_{\text{串}}=(n-1)R_g$;
         \item 要把它改装成量程为$I=nI_g$的安培表,并联电阻的阻值应为$R_{\text{并}}=R_g/(n-1)$.
     \end{enumerate}

     \begin{proof}
\begin{enumerate}
    \item 电流表改装为伏特表时,表头能承受的最大电压
    为$U_g=I_gR_g$.

    由于改装伏特表的量程为$U=nI_gR_g$, 故分压电阻应承受
    的电压$U_R=U-U_g=(n-1)I_gR_g$.

    根据串联电路的电压分配关系
    \[\frac{U_g}{R_g}=\frac{U_R}{R_{\text{串}}}\]
    得串联电阻的阻值
\[R_{\text{串}}=\frac{U_R R_g}{U_g}=\frac{(n-1)I_gR_g}{I_gR_g}R_g=(n-1)R_g\]

    \item 电流表改装为安培表时,表头能承受的最大电流
    为$I_g$.

    由于改装安培表的量程为$I=nI_g$, 故分流电阻应分担的
    电流 $$I_{\text{并}}=I-I_g=(n-1)I_g$$.

    根据并联电路的电流分配关系 $I_gR_g=I_R R_{\text{并}}$, 得并联电阻的阻值
\[R_{\text{并}}=\frac{I_g}{I_R}R_g=\frac{I_g}{(n-1)I_g}R_g=\frac{1}{n-1}R_g\]
\end{enumerate}
     \end{proof}
     
     \item 在第1题中,改装后的安培表的量程比电流表原来的电流量程扩大的倍数$n$是多少?改装后的伏特表比电流表原来的电压量程扩大的倍数$n$又是多少?利用第2题中推出的两个公式,重新计算第1题中的并联电阻和串联电阻,比较两次计算的结果是否相同.

     \begin{solution}
        改装安培表的量程比电流表原来的电流量程扩大的倍数
\[n=\frac{I}{I_g}=\frac{3}{3\x 10^{-3}}=1000\]

改装伏特表的量程比电流表原来的量程扩大的倍数
\[n=\frac{U}{U_g}=\frac{U}{I_gR_g}=\frac{6}{3\x 10^{-3}\x 100}=20\]
由$R_{\text{并}}=\frac{1}{n-1}R_g$,得:
\[R_{\text{并}}=\frac{1}{1000-1}\x 100=0.1\Omega\]
由$R_{\text{串}}=(n-1)R_g$,得:
\[R_{\text{串}}=(20-1)\x 100=1900\Omega=1.9{\rm k}\Omega\]
计算结果均与第1题的计算结果相同.
     \end{solution}
     
     \item 有一安培表,内阻为0.03欧,量程为3安,测量电阻$R$中的电流强度时,本应与$R$串联,如果不注意,错把安培表与$R$并联了(图7.13),将会产生什么后果?假设$R$两端的电压为3伏.
     
\begin{figure}[htp]\centering
    \begin{circuitikz}[european]
\draw (0,0)node[left]{$-$}--(4,0) to [R=$R$] (4,3)--(0,3) node[left]{$+$};
\draw (2.5,0) to [rmeter, t=A, *-*] (2.5, 3);
\draw [fill=white](0,0) circle (2pt);
\draw [fill=white](0,3) circle (2pt);
    \end{circuitikz}

    \caption{}
\end{figure}	

     \begin{solution}
因安培表与$R$并联,故
安培表两端电压跟$R$两端电压
相等,即 $U_A=U_R=3{\rm V}$

这时通过安培表的电流强度
\[I_A=\frac{U_A}{R_g}=\frac{3}{0.03}=100{\rm A}\]
远大于安培表的满度电流3安,安培表将烧毁.
     \end{solution}
     
\end{enumerate}



\subsection{练习七}
\begin{enumerate}
    \item 图7.21是某电路中的一部分,$R_2$、$R_3$、$R_4$是三个电子管的灯丝电阻,阻值分别为$R_2=R_3=84$欧,$R_4=21$欧,分压电阻$R_1=31$欧,$AB$间的电压是28伏,求通过各电子管灯丝的电流强度.
    \begin{figure}[htp]\centering
        \begin{circuitikz}[european, thick]
    \draw (-2,0) node[above]{$A$} to [R=$R_1$] (2,0) to [bulb=$R_2$] (4,0) to [bulb=$R_4$] (7,0) -- (8,0) node[above]{$B$};
    
    \draw (1.5,0)--(1.5,-2) to [bulb=$R_3$] (4.5,-2)--(4.5,0);
    \draw [fill=white](-2,0) circle (2pt);
    \draw [fill=white](8,0) circle (2pt);
    \draw [fill=black](1.5,0) circle (2pt);
    \draw [fill=black](4.5,0) circle (2pt);
    \node at (-2-.3,0){$+$};
    \node at (8+.3,0){$-$};
        \end{circuitikz}
        \caption{}
    \end{figure}	


    \begin{solution}
电路的总电阻
\[R=R_1+R_4+\frac{R_2}{2}=31+21+\frac{84}{2}=94\Omega\]
通过$R_4$的电流强度即电路的总电流强度
\[I=\frac{U_{AB}}{R}=\frac{28}{94}=0.30{\rm A}\]
通过$R_2$和$R_3$的电流强度相等.
\[I_2=I_3=\frac{1}{2}I=\frac{1}{2}\x 0.30=0.15{\rm A}\]
    \end{solution}
    
    \item 图7.22中,电源电压为220伏,各段输电导线上的电阻$r$都为2欧,用电器$R_1$的电阻为200欧,$R_2$的电阻为196欧,$R_1$、$R_2$两端的电压各是多少?
    \begin{figure}[htp]\centering
        \begin{circuitikz}[european, >=latex]
    \draw (0,0)--node [below]{$r$} (3,0)--node[below]{$r$}(5,0) to [R=$R_2$] (5,3)
    --node[above]{$r$}(3,3) --node[above]{$r$}(0,3);
            \draw (3,0) to [R=$R_1$, *-*] (3,3);
    
    
    \draw [<->](0,0.1)--node [fill=white]{220V} (0,3-.1);
    \draw [fill=white](0,0) circle (2pt);
    \draw [fill=white](0,3) circle (2pt);
    
        \end{circuitikz}
    
        \caption{}
    \end{figure}


    \begin{solution}
从图看出$R_2$与两个$r$串联后与$R_1$并联,再与两个$r$
串联,最后接到220伏电源上,并联部分的电阻
\[R_{\text{并}}=\frac{R_1(2r+R_2)}{R_1+2r+R_2}=\frac{200\x (2\x 2+196)}{200+2\x 2+196}=100\Omega\]
电路的总电阻
\[R=2r+R_{\text{并}}=2\x2+100=104\Omega\]
电路的总电流强度
\[I=\frac{U}{R}=\frac{220}{104}=2.12{\rm A}\]
则用电器$R_1$两端的电压
\[U_1=U_{\text{并}}=IR_{\text{并}}=2.12\x 100=212{\rm V}\]

因通过$R_2$的电流强度
\[I_2=\frac{U_{\text{并}}}{2r+R_2}=\frac{212}{2\x 2+196}=1.06{\rm A}\]
故用电器$R_2$两端的电压
\[U_2=I_2R_2=1.06\x196=208{\rm V}\]
    \end{solution}
    
    \item 一个600欧的电阻和一个400欧的电阻串联后接在电压为90伏的电源上,用伏特表测得600欧电阻上的电压为45伏.
    \begin{enumerate}
        \item 伏特表接入电路后,600欧电阻上的电压改变了多少?
        \item 这个伏特表的内阻是多少?
    \end{enumerate}
    
\begin{solution}
\begin{enumerate}
    \item 在未接入伏特表时,电路的电流强度
\[I=\frac{U}{R_1+R_2}=\frac{90}{600+400}=0.090{\rm A}\]
    这时600欧电阻上的电压
 \[   U_1=IR_1=0.090\x600= 54{\rm V}\]
 因伏特表接入电路后,$R_1$两端电压变为45伏,故600欧
 电阻上的电压改变为$54-45=9{\rm V}$.
\item  伏特表接入电路后,600欧电阻上的电压之所以改变,
 是因为伏特表有内阻$R_V$, 它与$R_1$并联,
 因$U_{\text{并}}=45$伏,$U_2=U-U_{\text{并}}=90-45=45{\rm V}=U_{\text{并}}$,故
 \[R_1=R_2=400\Omega\]
由于
\[\frac{1}{R_{\text{并}}}=\frac{1}{R_V}+\frac{1}{R_1}\]
 则伏特表内阻
 \[R_V=\frac{R_{\text{并}}R_1}{R_1-R_{\text{并}}}=\frac{600\x 400}{600-400}=1.20\x 10^3\Omega\]
\end{enumerate}
\end{solution}

    \item 在图7.16中滑动变阻器作分压器使用,负载电阻$R$一端接在变阻器的固定端$A$上,另一端接在滑动端$P$上,滑动端$P$在$AB$间移动时,$R$上就能得到不同的电压.
    \begin{figure}[htp]\centering
        \begin{circuitikz}[european,>=latex]
    
     \draw (0,0) to [R] (5,0) to [cute open switch] (6,0)--(7,0) ;      
     \draw (0,0)--(0,-2);
     \draw (7,0)--(7,-2);
    
     \draw [fill=white](0,-2) circle (2pt);
     \draw [fill=white](7,-2) circle (2pt);
     \draw [fill=black](0,0) circle (2pt);
    
    \draw (0,0) -- (0,1) to [R=$R$] (2.5,1); 
    \draw [->](2.5,1)--node [right]{$P$} (2.5,.2);
    \node at (2-.1,-.4){$A$};
    \node at (3+.1,-.4){$B$};
    \node at (5.5,-.4){$K$};
    
    
    
    \draw [<->](0+.1,-2)--node [fill=white]{$U$} (7-.1,-2);
    
    
       \end{circuitikz}
    
        \caption{}
    \end{figure}

    \begin{enumerate}
        \item 当滑动端从$A$向$B$移动时,$R$上的电压怎样变化?
        \item 如果电压$U=6$伏,变阻器的电阻$R_{AB}=50$欧,负载电阻$R=100$欧,当滑动端$P$在$A$点、$R$点时,$R$上的电压各是多少?
        \item 当$P$在$AB$中点时,$R$上的电压是否为3伏?通过变阻器各部分的电流是否相等?
    \end{enumerate}
    

\begin{solution}
    从图看出,$R$与$R_{AP}$并联后与$R_{PB}$串联.
\begin{enumerate}
    \item 由于
    \[R_{\text{并}}=\frac{R\cdot R_{AP}}{R+R_{AP}}=\frac{R}{1+R/R_{AP}}\]
    当$P$从$A$向$B$移动
时$R_{AP}$由小变大,所以$R_{\text{并}}$也由小变大.又因$R_{\text{并}}$跟$R_{PB}$串
联,根据串联电路中各电阻两端的电压跟它的阻值成正比,可
知$U_{\text{并}}$(即$R$上的电压)将随$R_{\text{并}}$的变大而变大.
\item 当滑动端$P$在$A$点时,$R$被导线短路,故$R$上的电压
为零.

当$P$在$B$点时,$R$与$R_{AB}$并联起来接到$U=6$伏的电源
电压上,故$R$上电压为6伏.
\item 当$P$在$AB$中点时,电器的总电阻
\[R_{\text{总}}=\frac{RR_{AP}}{R+R_{AP}}+R_{PB}=\frac{100\x \frac{1}{2}\x 50}{100+\frac{1}{2}\x 50}+\frac{1}{2}\x 50=45\Omega\]
$R$上的电压
\[U_{\text{并}}=U-U_{PB}=U-\frac{U}{R_{\text{总}}}R_{PB} =6-\frac{6}{45}\x \frac{50}{2}=2.7{\rm V}\]
\end{enumerate}

因通过变阻器$PB$段的电流等于电路的总电流强度,而
通过变阻器$AP$段的电流只是总电流的一部分,总电流的另
一部分经$R$分流.所以$P$在$AB$中点时,通过变阻器各部分
的电流不相等.
\end{solution}

\end{enumerate}



\subsection{练习八}
\begin{enumerate}
    \item 电源的电动势为1.5伏,内电阻为0.12欧,外电路的电阻为1.28欧,求电路宁的电流强度.

    \begin{solution}
根据闭合电路欧姆定律$I=\dfrac{\mathcal{E}}{R+r}$,
可求电路中电流
强度
\[I=\dfrac{\mathcal{E}}{R+r}=\frac{1.5}{1.28+0.12}=1.07{\rm A}\]
    \end{solution}
    
    \item 把一个定值电阻和电源连成图7.17所示的电路,可以测得电源的内电阻,定值电阻$R$为10欧,合上开关$K$时,伏特表的读数为5.46伏,打开$K$时,伏特表的读数为6.0伏.求电源的内电阻为多少.
\begin{figure}[htp]
    \centering
    \begin{circuitikz}[european]
        \draw(0,0)--(5,0) to [R=$R$](5,3) to  [cute open switch] (2,3)--(0,3) to[battery2] (0,0);
        \draw (2,0) to [rmeter, t=V, *-*] (2,3);    
        \node at (7/2,3)[above] {$K$}    ;
            \end{circuitikz}    
    \caption{}
\end{figure}

    \begin{solution}
因断开$K$时,伏特表
的读数为电源电动势,故电源
电动势$\mathcal{E}=6.0$伏.

合上开关$K$时,伏特表的
读数为定值电阻$R$两端的电压.由闭合电路欧姆定律$I=\dfrac{\mathcal{E}}{R+r}$
和部分电路欧姆定律$U=IR$,有
\[U=\frac{\mathcal{E}}{R+r}R\]
则电源内电阻
\[r=\frac{\mathcal{E}R}{U}-R=\frac{6.0\x 10}{5.46}-10=1\Omega\]
    \end{solution}
    
    \item 在图7.18所示的电路中$\mathcal{E}=9.0$伏,$r=3.0$欧,$R=15$欧,当$K$闭合时,$U_{AB}$是多少?当$K$打开时,$U_{AB}$又为多少?

    \begin{solution}
        当$K$闭合时,电路的电流强度为
\[I=\dfrac{\mathcal{E}}{R+r}=\frac{9.0}{15+3.0}=0.50{\rm A}\]
则\[U_{AB}=IR=0.50\x 15=7.5{\rm V}\]
当$K$断开时,$U_{AB}=\mathcal{E}=9.0{\rm V}$
    \end{solution}

    \begin{figure}[htp]\centering
        \begin{minipage}[t]{0.48\textwidth}
    \centering
    \begin{circuitikz}[european, xscale=.8]
        \draw(0,0) to [battery2] (6,0) node [below]{$B$} to[cute open switch] (6,2) to[R=$R$] (0,2)node [above]{$A$}--(0,0) ;
        
        \node at (6,1)[right]{$K$};
        
        \node at (3,-.5)[below]{$\mathcal{E}\;  r$};
            \end{circuitikz}
        
            \caption{}
        \end{minipage}
    \begin{minipage}[t]{0.48\textwidth}
    \centering
            \begin{circuitikz}[european,>=latex, yscale=.8]
        \draw (0,0)to [battery2] (6,0)--(6,2)to [rmeter, t=V, *-*] (0,2) to [rmeter, t=A] (0,0);
        \draw (0,2)--(0,4)--(2,4) to [R] (4,4)--(4.5,4)to [cute open switch] (6,4)--(6,2);
                \draw (1.5,4) to (1.5,4.7)--(3,4.7);
                \draw [->](3,4.7)--(3,4.2);
            \end{circuitikz}
        
            \caption{}\end{minipage}
        \end{figure}    
    
    \item 利用图7.19所示的电路可以测出电源的电动势和内电阻.当变阻器的滑动端在某一位置时,安培表和伏特表的读数分别是0.20安和1.98伏,改变滑动端的位置后,两表的读数分别是0.40安和1.96伏,求电池的电动势和内电阻.

    \begin{solution}
        根据闭合电路的欧姆定律,可列出方程组
\[\begin{split}
    \mathcal{E}&=I_1R_1+I_1 r=U_1+I_1r\\
    \mathcal{E}&=I_2R_2+I_2 r=U_2+I_2r\\
\end{split}\]
消去$\mathcal{E}$,可得
\[U_1+I_1r=U_2+I_2r\]
所以,电源的内电阻
\[r=\frac{U_1-U_2}{I_2-I_1}=\frac{1.98-1.96}{0.40-0.20}=0.1\Omega\]
把$r$的值代入$\mathcal{E}=U_1+I_1r$中,可求得电源电动势
\[\mathcal{E}=1.98+0.20\x 0.1=2.00{\rm V}\]
    \end{solution}
    
\end{enumerate}


\subsection{练习九}

\begin{enumerate}
    \item 试分析说明:外电路中的电阻发生变化时为什么会影响路端电压的变化.

    \begin{solution}
根据闭合电路欧姆定律$I=\dfrac{\mathcal{E}}{R+r}$
可知,就某个电源
来说,由于电动势$\mathcal{E}$和内电阻$r$都是一定的,因此外电路电阻
$R$发生变化时电路的电流强度$I$将发生变化.而路端电压$U=\mathcal{E}-Ir$, 所以当$R$变化导致$I$变化时,$U$必然变化.
    \end{solution}
    
    \item 在两个电路中,电源的电动势相同,但内电阻不同,当它们的外电路中流过的电流相同时,哪个电路的路端电压大?

    \begin{solution}
        由路端电压$U=\mathcal{E}-Ir$可知,若两个电路的$\mathcal{E}$相同、$I$
        相同时,内电阻$r$小的电路路端电压较大.
    \end{solution}
    
    \item 在图7.32所示的电路中,当$P$由左向右滑动时,安培表和伏特表的读数怎样变化?$P$的位置在何处,伏特表的
指示更接近于电源的电动势?

\begin{figure}[htp]\centering
    \begin{circuitikz}[>=latex, european]
\draw (0,0) to [battery2] (6,0)--(6,1) to [rmeter, t=V, *-*] (0,1)--(0,0);
\draw (0,1)--(0,2) to [rmeter, t=A] (2,2) to[R] (6,2)--(6,1);
\draw (6,2)--(6,3)--(4,3);
\draw [->](4,3)--node [left]{$P$}(4,2.2);
\fill [white] (6-.02,2.2) rectangle (4.58,1.8);
    \end{circuitikz}
    \caption{}
\end{figure}


\begin{solution}
    当$P$由左向右滑动时,电路中的电阻$R$变大,由$I=\dfrac{\mathcal{E}}{R+r}$
    可知,安培表的读数即电路中的电流强度将变小;由$U=\mathcal{E}-Ir$可知,伏特表的读数即路端电压将变大.因$R$越大,
    $I$越小,$U$就越大,也就越接近电源电动势;所以当$P$在最右
    端时,$R$值最大,伏特表的示数更接近于电源电动势.
\end{solution}

\item 发电机的电动势为240伏,内电阻为0.40欧,给200盏电阻均为1210欧的电灯供电,电灯上的电压是多大?如果
再接入100盏同样的电灯,电灯上的电压又是多大?利用所得的结果说明:电路中的用电器增多时,加在用电器上的电压将
怎样变化?

\begin{solution}
给200盏相同电灯供电时,外电路总电阻
\[R=\frac{1210}{200}=6.05\Omega\]
这时电路中总电流强度
\[I=\dfrac{\mathcal{E}}{R+r}=\frac{240}{6.05+0.40}=37.2{\rm A}\]
则电灯上的电压
\[U=\mathcal{E}-Ir=240-37.2\x0.40=225{\rm V}\]
如再接入100盏同样的电灯,外电路总电阻
\[R'=\frac{1210}{300}=4.03\Omega\]
同理可求这时电灯上的电压
\[U'=\mathcal{E}-I'r=\mathcal{E}-\frac{\mathcal{E}}{R'+r}r=240-
\frac{240}{4.03+0.40}\x 0.40=218{\rm V}\]
从计算结果可以看出:电路中用电器增多时,加在用电器
上的电压将减小.
\end{solution}

\end{enumerate}




\subsection{练习十}
\begin{enumerate}
    \item 每个铅蓄电池的电动势为2.0伏,想用它们给一个额定电压为6伏的用电器供电,应该怎么办?

    \begin{solution}
由于用电器的额定电压高于单个电池的电动势,可采用串联电池组供电.

依题意蓄电池内阻不计,有$U=\mathcal{E}_{\text{串}}=n\mathcal{E}$,
故
\[n=\frac{U}{\mathcal{E}}=\frac{6}{2.0}=3\]
应当用3个2.0伏的蓄电池串联起来供电.
    \end{solution}
    
    \item 每节干电池的电动势为1.5伏,允许通过的最大电流为0.05安.现在需要一个电动势为6伏,最大电流为0.1安的电源,应该怎么办?

    \begin{solution}
因所需总电动势大于单个电池电动势,应采用串联
电池组.需由$p$个电池串联而成,$\mathcal{E}_{\text{串}}=p\mathcal{E}$.
则
\[p=\frac{\mathcal{E}_{\text{串}}}{\mathcal{E}}=\frac{6}{1.5}=4\]
又因所需电流大于单个电池允许通过的最大电流,应把
几个跟上述串联电池组相同的电池组并联起来组成混联电池
组.设需由$q$个串联电池组并联而成,$I_{\text{总}}=qI$.
则
\[q=\frac{I_{\text{总}}}{I}=\frac{0.1}{0.05}=2\]
所需电池总数 $m=pq=4\x2=8$

要实现题目提出的要求,可用8节干电池,每4节串联成
一组,再将两组并联起来供电.
    \end{solution}
    
    \item 有10个相同的蓄电池,每个蓄电池的电动势为2.0伏,内电阻为0.04欧,把这些蓄电池接成串联电池组,外接它阻为3.6欧.求电路中的电流强度和电池组两端的电压.

    \begin{solution}
因$\mathcal{E}_{\text{串}}=n\mathcal{E}$, $r_{\text{串}}=nr$, 由闭合电路欧姆定律得
\[I=\frac{\mathcal{E}_{\text{串}}}{R+r_{\text{串}}}=\frac{n\mathcal{E}}{R+nr}=\frac{10\x 2.0}{3.6+10\x 0.04}=5.0{\rm A}\]
电池组两端电压
\[U=\mathcal{E}_{\text{串}}-Ir_{\text{串}}=n\mathcal{E}-Inr=10\x 2.0-5.0\x 10\x 0.04=18{\rm V}\]
    \end{solution}
    
    \item 找一个半导体收音机,打开看看里面有几节干电池,是怎样连接的,算一算这个收音机的电源电压是多少.

    \begin{solution}
        一般半导体收音机的电源是由4节1.5伏的干电池
        串联而成.电源电压为$n\mathcal{E}=4\x1.5=6{\rm V}$

        注意:此题答案不是唯一的,例如有不少半导体收音机的电源是由3节干电池串联而成.
    \end{solution}
    
    \item 把两节干电池串联起来组成电池组,用伏特表量出电池组的电动势.再把三个小灯泡照图7.21依次连入电路中,注意每增加一个小灯泡时伏特表读数的变化,说明伏特表的读数为什么会发生变化.这个变化表明了什么?

    \begin{figure}[htp]\centering
        \begin{circuitikz}[european]
    %\draw (0,3)--(6,3) to[lamp] (6,0)--(0,0) to [battery](0,3);
    \draw (0,0)--(6,0) to [lamp] (6,3)--(0,3) to [battery](0,0);
    
    \draw (1.5,0) to [rmeter,t=V, *-*] (1.5,3);
    \draw (4,0) to [lamp, *-*] (4,3);
    \draw (5,0) to [lamp, *-*] (5,3);
    
        \end{circuitikz}
    
        \caption{}
    \end{figure}

    \begin{solution}
        每增加一个灯泡,伏特表的读数就减小一次.因为电
        池组的总电动势和总内阻保持不变,当电源的并联负载不断
        增加时,外电路的总电阻将不断减小,电路的电流强度
$I=\dfrac{\mathcal{E}_{\text{总}}}{R+r_{\text{总}}}$
        将不断增大,导致伏特表的读数(即路端电压)$U=\mathcal{E}_{\text{总}}-Ir_{\text{总}}$不断减小.这个变化表明电路的路端电压随外电阻
        减小而减小.
    \end{solution}
    
\end{enumerate}






\subsection{练习十一}
\begin{enumerate}
    \item 在课本图7.37甲中,如果安培表的读数是0.2安,伏特表的读数是30伏,根据这些数据算出的$R$的阻值是多大?如果已知伏特表的内阻是3千欧,那么,$R$的真实值是多大?采用这种接法时,算出的$R$值比真实值大还是小?

    \begin{solution}
        $R$的计算值
        \[R'=\frac{U}{I}=\frac{30}{0.2}=150\Omega\]
        如考虑伏特表的内阻,则伏特表与$R$并联的等效电阻值
\[R_{\text{并}}=\frac{U}{I}=\frac{30}{0.2}=150\Omega\]
因为$R_{\text{并}}=\dfrac{R_VR}{R_V+R}$,所以,$R$的真实值
\[R=\frac{R_VR_{\text{并}}}{R_V-R_{\text{并}}}=\frac{3000\x 150}{3000-150}=158\Omega\]
可见算出的$R$值比真实值小.
    \end{solution}
    
    \item 在课本图7.37乙中,如果伏特表的读数是5伏,安培表的读数是0.5安,根据这些数据算出的$R$的阻值是多大?如果已知安培表的内阻是0.2欧,那么,$R$的真实值是多大?采用这种接法时,算出的$R$值比真实值大还是小?

    \begin{solution}
$R$的计算值
\[R'=\frac{U}{I}=\frac{5}{0.5}=10\Omega\]
如考虑安培表的内阻,则安培表与R串联的等效电阻值
\[R_{\text{串}}=\frac{U}{I}=\frac{5}{0.5}=10\Omega\]
因为$R_{\text{串}}=R+R_A$, 
所以,$R$的真实值
\[R=R_{\text{串}}-R_A=10-0.2=9.8\Omega\]
可见算出的$R$值比真实值大.
    \end{solution}
    
    \item 已知伏特表的内阻为5千欧,安培表的内阻为0.2欧,如果用它们来测量一个线圈的电阻,估计这个线图的电阻大约为几个欧姆,那么,怎样连接电路测得的结果误差较小?画出电路图.

    \begin{solution}
        从题目已知条件可以看出待测电阻的阻值远小于伏
        特表的内阻,所以采用安培表外接的方式测得结果误差较小.
        如图7.22所示.
\begin{figure}[htp]\centering
    \begin{circuitikz}[european, scale=.8]
\draw (0,0) to [rmeter, t=A] (2,0) to (2.5,0) to [R=$R$, *-*] (5,0)--(6,0);
\draw (2.5,0)--(2.5,1.5) to [rmeter, t=V] (5,1.5)--(5,0);
    \end{circuitikz}
    \caption{}
\end{figure}

    \end{solution}
    
    \item 在课本图7.40中,$R$为15欧,电桥平衡时,$\ell_1$为0.45米,$\ell_2$为0.55米,求待测电阻$R_x$.

    \begin{solution}
        因电桥平测时有
\[\frac{R_x}{R}=\frac{\ell_2}{\ell_1}\]
        所以,待测电阻$R_x$的值
\[R_x=\frac{\ell_2}{\ell_1}R=\frac{0.55}{0.45}\x 15=18\Omega\]
    \end{solution}
    
    \item 在课本图7.40中,如果$AB$支路发生断路,当滑动触头
    $D$从$A$移向$C$时,电流表$G$中的电流如何变化?如果$BC$支路发生断路,当$D$从$A$移向$C$时,电流表$G$中的电流又如何变化?

    \begin{solution}
        可以认为$A$、$C$间电压保持不变.设电阻线$AD$段
        电阻值为$R_1$, $DC$段电阻值为$R_2$.

        当$AB$支路断路时,$R_{DC}=\dfrac{(R_x+R_g)R_2}{R_x+R_gR_2}$
        与$R_1$串联,随滑动
        触头$D$从$A$移向$C$, $R_1$增大,$R_2$减小使$R_{DC}$减小.根据串联
        电路中各电阻两端电压跟它的阻值成正比,$R_{DC}$两端的电压
$U_{DC}$将减小,导致电流表G中电流$I=\dfrac{U_{DC}}{R_x+R_g}$
减小.

当$BC$支路断路时,$R_{AD}=\dfrac{(R+R_g)R_1}{R+R_g+R_1}$
与$R_2$串联,随滑
动触头$D$从$A$移向$C$, $R_2$减小,$R_1$增大使$R_{AD}$增大,根据串
联电路中各电阻两端电压跟它的阻值成正比,$R_{AD}$两端的电
压$U_{AD}$将增大,导致电流表G中电流$I=\dfrac{U_{AD}}{R+R_g}$增大.
    \end{solution}
    
\end{enumerate}






\subsection{习题}

        \begin{figure}[htp]
            \centering
            \begin{minipage}[t]{0.48\textwidth}
            \centering
            \begin{circuitikz}[european, scale=.7, >=stealth]
        
                \draw (0,0) --(4,0) to [R=$R$] (4,5)--(3,5); 
            \draw(1,5)--(0,5)to [battery2] (0,0);
        \draw (2,0) to [C=$C$, *-] (2,4);
        \node at (2,-.5){$b$};
        \draw [ultra thick ] (2,5.25)--(2,4);
        \draw [->] (2,5.25) arc (90:60:1.5);
        \draw [->] (2,5.25) arc (90:120:1.5);
        
        
        \draw (2,4) [fill=white] circle (1.5pt)node[left]{$a$};
        \draw (1,5) [fill=white] circle (1.5pt)node[above]{1};
        \draw (3,5) [fill=white] circle (1.5pt)node[above]{2};\draw (2,5.25) [fill=white] circle (1.5pt);
        
                \end{circuitikz}
            \caption{}
            \end{minipage}
            \begin{minipage}[t]{0.48\textwidth}
            \centering
            \begin{circuitikz}[european, scale=1.5]
    
                \draw (0,0) --(1,0) to [rmeter, t=mA] (2,0); 
                \draw [dashed](2,0)--(3,0)--node[right]{$B$}(3,2)--(2,2);
                \draw (0.8,0) to [rmeter, t=V, *-*] (.8,2); 
            \draw(2,0)--node[right]{$E$}(2,2)--(0,2)to [battery] (0,0);
            \node at (-.5,1){$A$};
                \end{circuitikz}
            \caption{}
            \end{minipage}
            \end{figure}



\begin{enumerate}
    \item 在图7.23中,电键可以向左扳将$a$与1接通,也可以向右扳将$a$与2接通,电键接通1的瞬间,电流的方向怎样?
接通1后再向右扳电键,将$a$与2接通,接通2的瞬间,电流的方向又怎样?

\begin{solution}
    电键$K$接通1的瞬间,电源给电容器$C$充电.充电
    电流的方向从电源正极流出,负极流入.电键接通1后再接
    通2的瞬间,已充电的电容器$C$通过电阻$R$放电.放电电流
    的方向从$C$的上极板流出经$R$流入$C$的下极板.
\end{solution}

\item $A$和$B$两地相距40千米,从$A$到$B$的两条输电线的总电阻为800欧,如果在$A$、$B$之间的某处$E$两条电线发生短路(图7.24),可用伏特表、毫安表和电池组检查出发生短路的地点.如果在$A$处测得伏特表的读数是10伏,毫安表的读数是40毫安,求短路处$E$到$A$的距离.

\begin{solution}
    从$A$到$E$的两条输电线的电阻为
\[R=\frac{U}{I}=\frac{10}{40\x 10^{-3}}=250\Omega\]
因为$\dfrac{AE}{AB}=\dfrac{R}{800\Omega}$,所以,短路处$E$到$A$的距离
\[AE=\frac{R}{800\Omega}AB=\frac{250}{800}\x 40=12.5{\rm km}\]
\end{solution}

\item 有两个灯泡,一个是110伏、100瓦,一个是110伏、
40瓦,把它们串联后接入220伏的电路中使用行不行?为什么?有一个变阻器,把它怎样连入电路中可以使两灯泡正常发光?这时变阻器的阻值应调至多大?

\begin{solution}
110伏、100瓦灯泡的电阻
\[R_1=\frac{U^2}{P_1}=\frac{110^2}{100}=121\Omega\]
110伏、40瓦灯泡的电阻
\[R_2=\frac{U^2}{P_2}=\frac{110^2}{40}=303\Omega\]
由于$R_2>R_1$, 它们串联起来接入220伏电路中使用时,
$U_2$将大于$U_1$, 
\[U_2=\frac{R_2}{R_1+R_2}U=\frac{303}{121+303}\x 220=157{\rm V}\]
可见,40瓦灯泡实际承担的电压已大于自身的额定电压,会
被烧毁,所以这样连接不行.

如要使两灯串联后又都能正常发光,可将变阻器与40瓦
灯泡并联,使它们的等效电阻与100瓦灯泡的电阻相等.这
样,两灯各分担110伏电压,均等于各自的额定电压.设这时
变阻器的阻值为$R$, 则有
\[R_1=R_{\text{并}}=\frac{R_2R}{R_2+R}\]
所以,变阻器的阻值应调到
\[R=\frac{R_1R_2}{R_2-R_1}=\frac{121\x 303}{303-121}=201\Omega\]
\end{solution}

\item 有一用电器$W$,额定电压为100伏,额定功率为150瓦,用120伏的电源供电.为了使用电器能正常工作,用一
电阻为210欧的变阻器进行分压(图7.25).$R_1$、$R_2$为多大时,用电器才能正常工作?

\begin{solution}
先求出用电器$W$的电阻值.
\[R_W=\frac{U^2_W}{P_W}=\frac{100^2}{150}=66.7\Omega\]
因为$R_2$与$R$并联后再与$R_1$串联,根据
串联电路的电压分配关系可得:
\[U_1:U_W = R_1:\frac{R_WR_2}{R_W+R_2}\]
把$R_1=R-R_2$和$U_1=U-U_W$代入上式,整理后得
\[R_2^2+\left(\frac{U}{U_m}R_W-R\right)R_2-RR_W=0\]
将$U=120$伏,$U_m=100$伏,$R_W=66.7$欧,$R=210$欧代入上式
得
\[R_2^2-130\cdot R_2-14000=0\]
解方程(舍去负根)得$R_2=200$欧,$R_1=210-200=10$欧.

\end{solution}

\begin{figure}[htp]
	\centering
	\begin{minipage}[t]{0.48\textwidth}
	\centering
	\begin{circuitikz}[european, scale=1.3, >=stealth]
		\draw [->](0,0)--(2,0)--(3,0)to [R=$W$](3,2)--(2.15,2);
		\draw (2,0)--(2,1) to [R] (2,3)--(0,3);
		
		\draw[<->] (0,.05)--node[fill=white]{$U$}(0,3-.05);
		\draw (2,0) [fill=black] circle (1.2pt);
		\draw (0,0) [fill=white] circle (1.2pt);
		\draw (0,3) [fill=white] circle (1.2pt);
		\node at (1.5,1.5){$R_2$};\node at (1.5,2.5) {$R_1$};
			\end{circuitikz}	\caption{}
	\end{minipage}
	\begin{minipage}[t]{0.48\textwidth}
	\centering
	\begin{circuitikz}[european, scale=1, >=stealth]
		\draw (-2,0) to [R=$R_1$, *-*] (0,0) to  [R=$R_2$, *-*] (2,0);
		\draw (-2,-1.5)--(-2,1.5) to [rmeter, t=G] (2,1.5)--(2,-1.5);
		\draw (0,0)--(0,-1.5);
		\draw (0,-1.5) [fill=white] circle (1.5pt) node[right]{$b$};
		\draw (-2,-1.5) [fill=white] circle (1.5pt)node[right]{$a$};
		\draw (2,-1.5) [fill=white] circle (1.5pt)node[right]{$c$};
		\node at (0,-1.5)[below]{1A}; \node at (0,1.1)[below]{$R_3$};
		\node at (2,-1.5)[below]{0.1A};
		\draw [dashed](2.5,2.2) rectangle (-2.5, -1);
		
			\end{circuitikz}
	\caption{}
	\end{minipage}
	\end{figure}

\item 图7.26所示的是有两个量程的安培表,当使用$a$、$b$两端点时,量程为1安,当使用$a$、$c$两端点时,量程为0.1安.已知电流表的内阻$R_g$为200欧,满度电流$I_g$为2毫安,求电阻$R_1$和$R_2$.


    \begin{solution}
当使用$a$、$b$两端点时,$R_g$与$R_2$串联后与$R_1$并联,
则有
\[I_g(R_g+R_2)=(I-I_g)R_1\]
当使用$a$、$c$两端点时,$R_1$与$R_2$串联后与$R_g$并联,则有
\[I_gR_g=(I'-I_g)(R_1+R_2)\]
代入题设数据得
\[\begin{cases}
    2\x10^{-3}\x(200+R_2)=(1-2\x10^{-3})\x R_1\\
2\x10^{-3}\x200=(0.1-2\x10^{-3})\x(R_1+R_2)
\end{cases}\]
化简后有
\[\begin{cases}
   200+R_2=499R_1\\
200=49\x(R_1+R_2) 
\end{cases}\]

上述二式联立求解可得:$R_1=0.41$欧,$R_2=3.67$欧.
    \end{solution}
    
\item 在课本图7.27中,不用安培表,改用伏特表能不能测出电源的电动势和内电阻?画出伏特表应怎祥接入电路,说明要取得哪些数据,写出计算电动势和内电阻的公式.

\begin{solution}
    不用安培表,改用伏特表能够测出电源电动势和内
    电阻.电路图如图7.27所示.

    将开关先后扳到位置1和位置2, 分别测出$R_1$和$R_2$接
    入电路时的伏特表读数$U_1$和$U_2$. 计算电动势和内电阻的公
    式可根据闭合电路欧姆定律列出方程组
\[\begin{cases}
    \mathcal{E}=U_1+\dfrac{U_1}{R_1}r\\
    \mathcal{E}=U_2+\dfrac{U_2}{R_2}r\\
\end{cases}\]
联立求解得出:
\[\mathcal{E}=\frac{U_1U_2(R_1-R_2)}{U_2R_1-U_1R_2},\qquad r=\frac{(U_1-U_2)R_1R_2}{U_2R_1-U_1R_2}\]
\end{solution}

\begin{figure}[htp]\centering
    \begin{minipage}[t]{0.48\textwidth}
    \centering
\begin{circuitikz}[european]

    \draw (.5,2.5)--(0,2.5)--(0,0) to [battery2] (5,0)--(5,3);
\draw(0,1) to [rmeter, t=V, *-*] (5,1);
\draw(1.5,3) to [R=$R_1$] (5,3);
\draw(1.5,2) to [R=$R_2$] (5,2);

\draw[thick](.5,2.5)node[below]{$K$}--(1.5,2.9);
\draw(1.5,3) [fill=white]circle(1.5pt)node[above]{$1$};
\draw(1.5,2) [fill=white]circle(1.5pt)node[below]{$2$};
\draw(.5,2.5) [fill=white]circle(1.5pt);
\node at (2.2,0)[below]{$\mathcal{E}$};
\node at (2.8,0)[below]{$r$};
    \end{circuitikz}
    \caption{}
    \end{minipage}
    \begin{minipage}[t]{0.48\textwidth}
    \centering
    \begin{circuitikz}[european, xscale=.8]
\draw (0,0)--(8,0) to [lamp] (8,3)--(3,3) to [R=$R$] (0,3) to (0,0);
\draw (3,0)--(3,3);
\draw (5,0)to [lamp] (5,3);
\draw [dashed](6.5,0)to [lamp] (6.5,3);        
\draw (0,1.5) node[elmech]{$F$};
\draw (3,1.5) node[elmech]{$M$};
\node at (1,1.5){$\mathcal{E},\; r$};
\draw [fill=black](3,3) circle (2pt);
\draw [fill=black](5,3) circle (2pt);
\draw [fill=black](6.5,3) circle (2pt);
\draw [fill=black](3,0) circle (2pt);
\draw [fill=black](6.5,0) circle (2pt);
\draw [fill=black](5,0) circle (2pt);
    \end{circuitikz}
    \caption{}
    \end{minipage}
    \end{figure}


\item 在图7.28中,用一台直流发电机$F$给一台电动机$M$和一些电灯供电.已知发电机的电动势$\mathcal{E}=240$伏,内电阻$r=1$欧,输电线的总电阻$R=3$欧,电动机的工作电流为3安,供给电灯的总电流为9安,求电灯和电动机两端的电压.

\begin{solution}
电路的总电流强度
\[I=I_M+I_{\text{灯}}=3+9=12{\rm A}\]
路端电压
\[U=\mathcal{E}-Ir=240-12\x1=228{\rm V}\]
则电灯和电动机两端的电压
\[U'=U-U_R=U-IR=228-12\x3=192{\rm V}\]
\end{solution}


\item 现有电动势为1.5伏,内电阻为1欧的电池若干,每个电池允许输出的电流为0.05安,又有不同阻值的电阻可作为分压电阻,试设计一种电路,使额定电压为6伏、额定电流为0.1安的用电器正常工作,画出电路图,并标明分压电阻的值.

\begin{solution}
    按题意要使用电器正常工作,应采用混联电池组,设
    它由$p$个电池串联为一组,$q$组并联而成.

    要使用电器两端的电压等于额定电压6伏,必须让每组
    串联电池输出的电压$U$满足以下关系:
\[U=p\mathcal{E}-0.05\x pr\ge 6\]
即:\[p\x 1.5-0.05\x r\ge 6\]
则$p$至少要取5.

要使用电器通过的电流等
于额定电流,又不致损坏电池,应满足$0.1=q\x0.05$,则$q=2$.

因此,需将5只电池串联成一组,再将两组并联使用.这
种混联电池组的电动势$\mathcal{E}_{\text{总}}=7.5$伏,内阻
\[r_{\text{总}}=\frac{5\x 1}{2}=2.5\Omega\]

因电池组的输出电压
\[U=5\x1.5-0.05\x5\x1=7.25{\rm V}\]
大于用电器的额定电压,故应给用电器串联一个分压电阻$R$. 

由于$\mathcal{E}_{\text{总}}=Ir_{\text{总}}+IR+U_{\text{额}}$,
式中的$U_{\text{额}}=6$伏,是用电器的额定电压,所以分压电阻
\[R=\frac{\mathcal{E}_{\text{总}}-Ir_{\text{总}}-U_{\text{额}}}{I}=\frac{7.5-0.1\x 2.5-6}{0.1}=12.5\Omega\]
整个电路如图7.29所示.

\begin{figure}[htp]
    \centering
\begin{circuitikz}[european,scale=.8]
\draw (0,4)--(0,0) to [R](3,0) to [R](5,0)--(5,4);
\draw(0,4) to [battery] (5,4);
\draw(0,2) to [battery, *-*] (5,2);
\node at (1.5,-.3)[below]{$R=12.5\Omega$};
\node at (4,-.3)[below]{用电器};

\end{circuitikz}
    \caption{}
\end{figure}

\end{solution}

\item 用伏安法测电阻,如果所用的安培表的内阻$R_A=0.1$
欧,伏特表的内阻$R_V=1000$欧,那么,用课本图7.37所示的两种不同接法测量$R=1$欧的电阻时,哪种方法产生的误差较小?测量$R=500$欧的电阻时,哪种方法产生的误差较小?测量较小的电阻和较大的电阻,各应采用什么方法?

\begin{solution}
    由题设条件可看出:测量$R=1$欧的电阻时,$R\ll R_V$,
    采用课本图7.37甲的接法产生的误差较小;测量$R=500$欧
    的电阻时,$R\gg R_V$, 采用课本图7.37乙的接法产生的误差较
    小.测量较小的电阻应采用课本图7.37甲所示的安培表外
    接法,测量较大电阻应采用课本图7.37乙所示的安培表内
    接法.
\end{solution}

\item 在课本图7.40中,滑动触头$D$从$A$向$C$移动时,电流
表中都有电流通过,但电流强度逐渐减小,这时可能是什么地方发生了断路?如果电流表指示的电流强度逐渐增大,又可能是什么地方发生了断路?

\begin{solution}
    电流表示数逐渐减小,可能$AB$支路发生断路;电流
    表示数逐渐增大,可能$BC$支路发生断路.(具体分析过程可
    参考练习十一第5题的解答)
\end{solution}

\item 在课本图7.40中,无论在从$A$经电源到$C$的电路中发
生断路,还是从$B$经电流表到$D$的电路中发生断路,在按下滑动触头时都没有电流通过电流表.用什么办法可以把这两种情况区别开来?

\begin{solution}
    可将滑动触头$D$处接线断开,然后把$D$处与电流表相
    连的接线线端跟电源正、负极瞬时接触.如果电流表都不偏
    转,说明$B$经电流表到$D$的电路发生了断路;如接触电源正极时,电流表偏转,说明从$A$到电源正极的电路发生了断路;如
    接触电源负极时电流表偏转,说明从电源负极到$C$的电路发
    生了断路.
\end{solution}

\end{enumerate}



\section{参考资料}
\subsection{超导电现象}
1911年荷兰科学家昂尼斯发现,水银的电阻率并不象预
料的那样随温度降低逐渐减小,而是在绝对温度4.15K附近
突然减小到测不出来.现在用最精确的方法估计,其电阻率
小于$10^{-25}$欧·米,而良导体之一的铜,即使是最纯的铜,在温
度为4.15K时的电阻率也有$10^{-11}$欧·米,因此,完全可以认为
在4.15K附近水银的电阻已趋消失.某些金属、合金和化合
物,在温度降到绝对零度附近某一特定温度时,它们的电阻率
会突然减小到无法测量的现象叫超导现象,处于这种状态的
导体叫超导体,每一种超导物质都在特定的温度下转变为超
导体.这个温度$T_c$, 叫转变温度.现已发现大多数金属元素
(除铁、镍、钴、铜、银、金等之外)以及数以千计的合金、化合物
都能在不同条件下显示出超导性.如钨的转变温度为0.012K,
锌为0.75K, 铝为1.196K, 锡为3.722K, 铅为7.193K, 银为
9.25K, 铌三锗为23K.

超导体的电阻是怎样消失的?从微观上回答这一问题,花
费了固体物理学家们近半个世纪的心血,直到1957年才由巴
丁、库伯、施里弗在前人的基础上建立了通称为BCS理论的
超导性理论.为此,他们荣获1972年诺贝尔物理学奖.

如果用超导材料做成一个闭合回路,那么在这个回路里
电流一经激发就可以无需电源持续几个星期之久而不减弱,
并且也不会象在具有电阻的普通导体回路中那样发热.

我们知道,在大的电磁铁或电机中,通过线圈的电流很
强,为了避免产生过多的热量,线圈就必须用较粗的导线绕或
采取冷却措施.这就使电磁铁和电机既笨重耗电又多.如果
用超导体作线圈,就可以避免这种缺点.现在用超导体产生
强磁场和制造电机方面的研究工作已获得较大的进展.

例如,使用超导发电机能将通常发电机的单机输出率极
限从200千瓦提高5—10倍.目前,数千千瓦的超导发电机
已安全运行,数万千瓦的试验超导发电机正在研制中.

超导电缆的研究和应用,近年来也有很大进展.超导电缆
埋在地下,损耗也小,有利于节约能量,保护环境和节约土地.

超导现象在高能物理领域也有重要应用.用超导线圈制
成电磁铁能产生强大的磁场,对于核聚变时约束等离子体
和粒子加速器实验装置都有很大用处.

阻碍超导现象大规模应用的主要问题是它要求低温.假
如找到能在液氮(沸点为77K)工作的超导材料,就可以用液
氮作致冷剂,则可能会使电力工业乃至整个工业发展发生巨
大的变化.1986年底,我国的物理工作者发现了一种转变
温度达48.6K的镧钡铜氧化物,已引起了国际学术界的注意.

\subsection{非纯电阻电路中的电功和电热}
在非纯电阻电路中,为什么要分别用$UIt$和$I^2Rt$来计算
电功和电热呢?根本原因在于欧姆定律只适用于不存在电动
势的一段电路,即只适用于仅有静电力做功的一段电路.在
一段含有电动势的电路中,除了静电力外,还有非静电力,欧
姆定律$U=IR$就不适用了,因此就不能用$UIt$推出$I^2Rt$和
$U^2t/R$
来.这时静电力移动电荷运动的过程中还需反抗非静电
力作功,把电能的一部分转化为其他形式的能.

\subsection{电源电动势}
电源的作用在于维持电源两极$A,B$之间的电压恒定不
变.当外电路接有电阻$R$时(图7.30)正电荷在电场力作用
下沿电阻从$A$极流向$B$极,使两极板积累的正、负电荷减少,
我们若能将流到$B$板的正电荷不断取
走,并把它送回$A$板,以保持$A$、$B$两板
上电量不变,那么两板之间的电压就能
保持不变.但是,将正电荷从$B$板取走
送回$A$板,与正电荷从$A$板经$R$流向$B$板是两个完全不同的过程.在后一过程中,$A$、$B$板上的电荷
所产生的电场力是使正电荷从电势高处向电势低处运动的动
力;而在前一过程中,两极堆积的电荷产生的电场力,是正
电荷要从低电势处向高电势处运动的阻力,正电荷必须克服
这一阻力才能到达$A$板.因此,当只有静电场存在时,前一
过程是不能实现的.只有非静电性质的作用,才能将流至$B$
板的正电荷不断取走,取回$A$板去.通常用非静电力来表
示这种非静电性质的作用,只有电源内部存在非静电力,如
图7.30箭头所示,才能使两极间维持恒定的电压.

温差电源的非静电力是与温度差和电子的浓度相联系的
扩散作用.一般发电机的非静电力是电磁感应作用,尽管各
种电源中的非静电力各不相同,但其作用都是使正电荷重新
聚积到电源的正极上,负电荷重新聚积到负极上.

再从能量转化观点来分析.非静电力只存在于电源内部,
但电场力在电源的内部和外部都存在着.当负载和电源接通
时,正电荷在电场力作用下从电势高的正极出发,经过负载到
达势低的负极,然后在非静电力的作用下,克服电场力的阻
碍作用,又从负极经电源内部回到正极.由于电荷克服电场
力做功,系统的电势能要增加,因此正电荷从负极经电源内部
回到正极的过程是使电能增加的过程.这个过程之所以能实
现,正是由于非静电力耗费了其他形式的能对正电荷做功的
结果.由此可见,电源的电能是通过非静电力对电荷做功的
方式,从其他形式的能量转化来的,没有非静电力就不能实现
这种能量的转化,也不能获得电能.至于电场力所作的功只决
定于起点和终点的位置,而与路径无关.当正电荷从电源正极
出发,绕整个电路一周又回到正极后,电场力作的总功为零.
因此总的来讲,电场力并没有对电荷提供能量作出贡献.但是
在电能的传递上,静电力的作用是重要的,电源内部的能量正
是靠了静电力在电路上引起的电流才传递到负载上去的.

设非静电力将正电荷$q$从电源负极$B$移到正极$A$所作
的功为$W_{BA}$, 则移送单位正电荷所做的功为$W_{BA}/q$,
这个量就
称为电源的电动势,通常用符号$\mathcal{E}$来表示
\[\mathcal{E}=\frac{W_{BA}}{q}\]
由于非静电力所做的功的大小等于电源内其他形式的能量转
化成电能的量,用此可以把电动势理解成单位正电荷通过电
源时所发生的其他形式的能量转化成电能的量.

在讨论电磁感应等现象时,会遇到整个闭合回路上都有
非静电力的情形,还要用到回路电动势的概念,回路电动势是
指单位正电荷在绕回路一周的过程中,非静电力所做的功.

\subsection{用半偏法测电流表内阻的实验条件}
在教材图9.7所示电路中,设电源电动势为$\mathcal{E}$,
内电阻为$r$, 电流表内阻的真实值为$r_g$. 当闭合电键$k_1$调整
电位器$R$至电流表指针满偏时,由闭合电路欧姆定律有
\begin{equation}
    \mathcal{E}=I_g(R+r+r_g)
\end{equation}

当再闭合$k_2$调整电阻箱$R'$使电流表指针半偏时,设电
路中总电流为$I$, 应有
\begin{equation}
    \mathcal{E}=I\left(R+r+\frac{R'r_g}{R'+r_g}\right)
\end{equation}

因电流表与$R'$并联,故
\begin{equation*}
    \frac{1}{2}I_g r_g=I\left(\frac{R'r_g}{R'+r_g}\right)
\end{equation*}
解出$I$, 可得
\[I=\frac{I_g(R'+r_g)}{2R'}\]
代入(7.2)式中得
\begin{equation}
    \mathcal{E}=\frac{I_g(R'+r_g)}{2R'}\left(R+r+\frac{R'r_g}{R'+r_g}\right)
\end{equation}

比较(7.1)式和(7.3)式可得
\[R+r+r_g=\frac{R'+r_g}{2R'}\left(R+r+\frac{R'r_g}{R'+r_g}\right)\]
则
\begin{equation}
    r_g=\frac{R+r}{R+r-R'}R'
\end{equation}

(7.4)式表示实验中测算电流表内阻真实值的表达式.这
时电流表内阻的测量值$r'_g=R'$

实验的相对误差
\[\frac{\Delta r_g}{r_g}=\frac{|r_g-r'_g|}{r_g}\]
将(7.4)式和$r'_g=R'$代入并化简,最后得到:
\begin{equation}
    \frac{\Delta r_g}{r_g}=\frac{R'}{R+r}
\end{equation}

由于$R\ll r$, 可将(7.4)、(7.5)两式简化为
\[r_g=\frac{R}{R-R'}R',\qquad     \frac{\Delta r_g}{r_g}=\frac{R'}{R}\]

从这两个式子可以看出,当$R\gg R'$时,才有$r_g=R'$,
$\Delta r_g/r_g=0$, 这就是用半偏法测电流表内阻时必须满足的条件.
做实验时,如果取$R=100R'$, 则$r_g=\frac{100}{99}R'$,用$R'$代替$r_g$所
产生的相对误差
$\Delta r_g/r_g=1\%$;
如果取$R=50R'$, 则$r_g=\frac{50}{49}R'$
用$R'$代替$r_g$所产生的相对误差
$\Delta r_g/r_g=2\%$.

以上讨论没有涉及电阻箱和电流表精确程度对实验的
影响.


\subsection{伏安法测电阻的误差分析}
\subsubsection{安培表外接(教材图7.37甲)}
在这种连接方式中,伏特表示数$U$等于待测电阻$R$两端
的电压,而安培表的示数$I=I_R+I_V$, 因此$R$的真实值
$R=U/I_R$,
测量值$R'=\dfrac{U}{I}=\dfrac{RR_V}{R+R_V}$,
即测量值实际上为$R$与
$R_V$并联的等效电阻值.

测量的绝对误差
\[R=|R-R'|=R-\dfrac{RR_V}{R+R_V}=\dfrac{R^2}{R+R_V}\]
相对误差
\[\frac{\Delta R}{R}=\frac{R}{R+R_V}\]

可见当待测电阻$R\ll R_V$时,采用安培表外接,可以使误
差很小,相反若$R$很大,且接近$R_V$时,相对误差可达50\%.

\subsubsection{安培表内接(教材图7.37乙)}
在这种连接方式中,安培表示数等于通过待测电阻$R$中
的电流,而伏特表的示数为$U=U_R+U_A$, 因此,$R$的真实值
$R=U_R/I$,
测量值$R'=\dfrac{U}{I}=R+R_A$, 即测量值实际为$R$与$R_A$
串联的等效电阻值.

测量的绝对误差$\Delta R=|R-R'|=R_A$,相对误差$\dfrac{\Delta R}{R}=\dfrac{R_A}{R}$

可见当待测电阻 $R\gg R_A$时,采用安培表内接,可以使误
差很小,相反若$R$很小,且接近$R_A$时,相对误差可达100\%.

从上面的讨论可以得出,当待测电阻$R$满足下述关系式
时,两种接法的相对误差相等.
\[\frac{R_A}{R}=\frac{R}{R+R_V}\]
整理得$R^2-R_AR-R_AR_V=0$.

解这个方程,并考虑待测电阻只取正值,得
\[R=\frac{R_A+\sqrt{R^2_A+4R_AR_V}}{2}\]

\subsection{欧姆表表盘的刻度}

在教材图7.38丙中,当红、黑表笔间接入某一
电阻$R_2$时,通过电流表的电流强度
\[I=\frac{\mathcal{E}}{R_g+r+R+R_x}\]

由于干电池的内阻$r$跟$R_g$、$R$相比非常小,可以忽略不
计,所以上式变为
\begin{equation}
    I=\frac{\mathcal{E}}{R_g+R+R_x}
\end{equation}

当电源电动势$\mathcal{E}$,电流表内阻$R_g$, 以及限流电阻$R$确定
后,$I$随$R_x$改变,电流表指针偏转的角度与$R_x$的值相对应,如
果在刻度盘上直接标出与$I$相对应的$R_x$值,就成为欧姆表.

如果把红、黑表笔短接($R_x=0$)调节$R$的值使电路中的电
流强度等于电流表的满偏电流$I_g$, 则
\begin{equation}
    I_g=\frac{\mathcal{E}}{R_g+R}
\end{equation}
即欧姆表的零阻值刻度应标在电流表满刻度处.

如果把红、黑表笔断开($R_x=\infty$),电路中电流强度为
$I=0$, 电流表指针不发生偏转,因此欧姆表的无穷大电阻刻
度应在电流表的电流为零处.

表盘上$0<R<\infty$之间的刻度又怎样确定呢?

首先确定中值电阻,比较(7.6)式和(7.7)式可知,当$R_x=R_g+R$时,$I_x=\frac{1}{2}I_g$,
即电表指针只偏转满刻度的一半,也就是
在欧姆表表盘的中心刻度处.此刻度值叫中值电阻$R_{\text{中}}$, 它等
于电流表内阻$R_g$和限流电阻之和,$R_{\text{中}}=R_g+R$.

然后根据
\[I=\frac{\mathcal{E}}{R_{\text{中}}+R_x}=\frac{R_{\text{中}}}{R_{\text{中}}+R_x}I_g\]
确定其他刻度.当$R_x$
分别为中值电阻$R_{\text{中}}$的2倍、3倍、4倍……时,电路中的电流
强度$I$分别为满偏电流$I_g$的$\dfrac{1}{3},\dfrac{1}{4},\dfrac{1}{5},\ldots$即电表指针的偏转角度为满偏时的
$\dfrac{1}{3},\dfrac{1}{4},\dfrac{1}{5},\ldots$.
当$R_x$的值分别为$R_{\text{中}}$的
$\dfrac{1}{2},\dfrac{1}{3},\dfrac{1}{4},\ldots$时,电表指针的偏转角度则分别为满偏时的$\dfrac{2}{3},\dfrac{3}{4},\dfrac{4}{5},\ldots$
所以欧姆表表盘的刻度是不均匀的.

为了使欧姆表各挡共用一个标尺,一般都以$R\x1$挡的中
值电阻为标准,成10倍地扩大.如$R\x1$挡中值电阻$R_{\text{中}}=
10\Omega$, 此时表的总内阻$r_{\text{总}}=10\Omega$. 当$r_{\text{总}}=100\Omega$时,电表的$R_{\text{中}}=
100\Omega$. 若$R_x=100\Omega$, 电表指针应指标尺正中刻度“10”的位
置,读数应乘以10, 这就是$R\x10$挡.依此类推.可见,扩大
欧姆表量程就是扩大表的总内阻$r_{\text{总}}$的值,实际上是通过欧
姆表内的另一附加电路来实现的.

由$I=\dfrac{R_{\text{中}}}{R_{\text{中}}+R_x}I_g$还可看出,当$R_x\ll R_{\text{中}}$时,有$I\approx I_g$, 此时
指针偏转接近满偏,随$R_x$的变化,$I$的变化太小,表头受本身
灵敏度限制,不易分辨,因而测量误差很大.当$R_x\gg R_{\text{中}}$时,
$I\approx 0$, 同理可知,测量误差也很大.所以在实用上通常只用欧
姆表中间一段来测量,一般取$0.1R_{\text{中}}$—$10R_{\text{中}}$这段范围.实际上
欧姆表都有几个量程,每个量程的$R_{\text{中}}$都不同,但每个量程的
可用范围都是$0.1R_{\text{中}}$—$10R_{\text{中}}$. 如$R_{\text{中}}=10\Omega$, 则测量范围为
$1\Omega$—$100\Omega$, 如$R_{\text{中}}=100\Omega$, 则测量范围为$10\Omega$—$1000\Omega$.

欧姆表的刻度是对设计电源电动势$\mathcal{E}$计算出来的.由
于实际上电源电动势不可能总是正好等于$\mathcal{E}$,所以在欧姆表
中还装有调零旋钮,以保证刻度正确.




\chapter{物质的导电性}
\section{教学要求}
这一章将有关物质导电性的知识集中起来,系统地阐述
了金属。液体、气体和半导体等各种物质的导电性,以及真空
中的电流,这样,可以使学生对物质导电性有一个较为全面
的认识;同时也使学生有可能通过比较学习,更容易理解和掌
握。教材对各种不同的导电现象在技术上的实际应用的介
绍,可以丰富学生的知识,开阔他们的眼界。

这一章从物质的微观结构来讨论各种物质的导电机理,
对一些宏观现象要作出微观解释,从而使学生对宏观现象的
认识深入一步。当然这种认识还是很初步的,为了不加重学
生的负担,教学中一般不要再补充,但这种把宏观和微观统一
起来讨论问题的方法很重要,应当使学生有进一步的认识。

本章教材对于金属导电、电解液导电和电子电量的确定
等内容进行了定量讨论,这些内容应作为重点来讲授。除此
而外,由于气体导电和半导体导电的过程都比较复杂,有些是
无法用经典理论加以解释的,因此只作定性叙述,不涉及定量
研究,有些演示实验,也以定性观察为主。

本章内容可以分为五个单元:第一单元包括引言和第一
节,讲述金属的导电性;第二单元从第二节到第四节,讲述液
体的导电性;第三单元从第五节到第七节,讲述气体的导电
性;第四单元包括第八节和第九节,讲述真空中的电流和示波
管;第五单元从第十节到第十三节,讲述半导体的导电性及晶
体二极管和三极管。

第一节金属的导电性,教材从金属的微观结构出发,介绍
了金属导电的微观机理,导出了电流强度和自由电子平均定
向移动速率的关系式和欧姆定律的微观表达式。对金属导电
的微观解释,以及作为下一章推导洛仑兹力基础的电流强度
与自由电子平均定向移动速率的关系,决定了这节是重点知
识。但不要求学生记忆欧姆定律的微观表达式和掌握推导
过程。

第二节在学生已有的化学知识的基础上讲述液体的离子
导电,并分析了离子导电的电解过程,在讲述电解的应用时,
着重介绍了电解电容器(电解、电镀在化学课有专门的讲述)。
这一节教学主要是要求学生知道电解质导电是怎样形成的,
以及离子导电与电子导电的区别。

第三节是重点,知识讲述电解质导电的重要规律——法
拉第电解第一定律和第二定律,学生应当掌握这两个定律,理
解法拉第恒量的物理意义。

第四节介绍了一种确定电子电量的方法,这里导出的计
算基本电荷的公式$e=F/N$十分重要,它把法拉第恒量$F$、基
本电荷$e$、阿伏伽德罗常数$N$这三者联系起来,对于测量$e$
和$N$这两个微观量有重要意义。

第五节《气体的导电性》中,教材首先通过演示,使学生知
道,空气在火焰的作用下发生电离,变成了导体,可以导电。接
着介绍了气体导电跟电解质导电、金属导电的区别,以加深学
生对气体导电机理的认识,最后介绍了被激放电和自激放
电,并对形成自激放电的过程作了具体地分析,从而得出了产
生自激放电的条件:气体电离和阴极发射电子。对这个条件
应当要求学生清楚地了解,下节讲述几种自激放电现象,虽然
各有特点,但都要用这个条件给以说明。

第六节介绍了几种自激放电现象,要求学生对它们有所
了解,并且知道它们各自的特点,关于这几种放电的形成,教
材只作了简要介绍,不要求仔细地展开讲解,也不要求把它们
进行对比,学生能够对自激放电的条件大致有所了解就可以
了。关于辉光放电,教材没有把辉光分成若干个区,教学中也
不要求补充介绍。关于弧光放电,教材讲述的主要是大气压下
的弧光放电现象,对其形成过程作了简单介绍,弧光放电不
是仅在大气压下才能发生的放电现象,另外还有低气压下的
弧光放电,例如汞弧整流管中就是低压下的弧光放电,对此,
教师可以酌情向学生简单介绍。在讲火花放电和电晕放电放
时,前后面处提到了避雷针,但这两处所要讲述的问题是不同
的,前面提出了为什么要安装避雷针,后面则说明了为什么
避雷针能够避免雷击。跟弧光放电一样,火花放电和电晕放
电也不仅是在大气压下才能发生的放电现象。

第七节《气体电光源》是选讲内容。这里介绍的霓虹灯、
日光灯,高压水银灯等都是常见的气体放电光源,也是前两节
知识的应用,让学生对这些光源的工作原理有一个初步的
认识,是十分有益的。

第八节《真空中的电流》要求学生清楚阴极射线的产生和
为什么说阴极射线是从阴极发射出来的电子流。

第九节讲述示波管。要求学生知道示波管的基本构造和
工作原理。要求学生清楚扫描过程和什么是同步,以及荧光
屏上的图线是怎样形成的。所有这些,也都是做学生实验《练
习使用示波器》的基础,对于荧光屏上显示的图线,教材只介
绍了一条完整的正弦曲线的情形。由此不难推知显示出若干
个完整的正弦曲线的条件。

第十节《半导体的导电性》和第十一节《N型半导体和P
型半导体》是第五单元的基础,应当使学生清楚半导体中产生
自由电子和空穴的物理过程,认识半导体的导电机理及其与
金属导电的不同,以及N型半导体和P型半导体的导电
机理。

第十二节《PN结晶体二极管》,对PN结的形成讲解比
较简单,要求比较低,没有提及空间电荷区,也没有说明漂移
运动。要求学生对PN结的形成有一个粗浅的认识。当然,
形成稳定的阻挡层时,并不是没有扩散运动了,而是处于动态
平衡。

第十三节<晶体三极管》,只要求学生知道当基极电流稍
有变化时,集电极电流就有较大的变化,即三极管有放大作
用。不要求根据PN结的性质对放大作用作出解释。

这一章的教学要求是:
\begin{enumerate}
    \item 了解金属的导电机理,知道欧姆定律的微观解释。
    \item 了解液体的导电机理,掌握法拉第电解定律,理解法拉第恒量的意义,知道怎样根据这个恒量确定电子的电量。
    \item 了解气体的导电机理,知道什么是被激放电和自激放
    电,知道几种自激放电现象。
    \item 了解阴极射线,了解示波管的简单原理。
    \item 了解半导体的导电机理,了解PN结的作用和二极管
    的单向导电性,知道三极管有放大作用。
\end{enumerate}

\section{教学建议}
\subsection{第一单元}
课前可布置学生复习教材第四章固体的性质中第74页
一、二、三自然段以及第七章稳恒电流中第174页第二、三、四
自然段,为复习引入本节内容作好准备。

\subsubsection{金属的导电结构}

学习这节教材时,学生往往认为金
属的导电机构是自由电子,乃是已知的事实,而没有注意到本
节是从金属的微观结构出发来讨论的。前面第四章固体的性
质,只笼统地提到晶体的微观结构是由分子、原子或离子形成
空间点阵,电流一节也仅仅提到金属中的自由电荷是自由电
子。如果不理解金属的微观结构和自由电子的无规则热运动,
还会产生一些错误认识,如有的学生认为金属原子离解为正
离子和自由电子后,金属就显正电;或自由电子运动到某部
分,某部分就显负电,金属中出现了电源等。这类看法需要从
自由电子热运动特点,用统计观点加以澄清。要让学生了解,
由于自由电子向各个方向运动的几乎完全相同,可以认为大
量电子在金属导体内是均匀分布的,从统计观点考虑,导体中
任何一部分体积内的自由电子数目都与该体积内金属正离子
所带电荷数相等,所以任何一段导体都不显电性,整个导体
也是不带电的。在没有外电场时,通过金属导体内部任何一个
横截面积的总电量都等于零,这种热运动不会引起任何一个
方向上产生电流,还可以补充两幅图来形象地说明,图8.1
甲中,小黑点表示自由电子,虚线箭头表示自由电子热运动速
度,金属正离子用符号$\oplus$表示。可以看出,在这部分导体中,
自由电子的速度朝哪个方向的都有,它们的数目也跟金属正
离子相等,有电场存在时,自由电子除具有热运动速率。还
具有定向运动速率,如图8.1乙中实线箭头所示(应该说明,
图中所画的自由电子运动速率,跟它们的热运动速率相比,是
夸大了的)。导体中出现了自由电子整体的定向运动,因而形
成了电流。

\begin{figure}[htp]
    \centering
\includegraphics[scale=.6]{fig/8-1.png}
    \caption{}
\end{figure}

\subsubsection{关于公式$I=neSv$的推导}

可分成几个步骤,提出
问题启发引导学生自己推导,让学生再次学习这种把宏观与
微观统一起来的讨论方法,体会其优越性。有了这个关系式,
微观量$v$就可由容易测得的宏观量$I$、$S$及已知的$n$、$e$计算
出来,教师可给出书上的数据,让学生动手算算$v$的数值,记
下$v$的数量级。

对于自由电子平均定向移动速率、自由电子热运动平均
速率、电场的传播速率三者的区别,一方面要引导学生从它们
的物理意义和数值大小两个方面加以对比区别;另一方面可
用类比、比喻来帮助学生理解并加深印象。

\subsubsection{欧姆定律的微观解释}

可采用下述的思路推导:

欧姆定律的表达式$I=U/R$
中,已经知道$I=neSv$这个
与微观量有关的关系式。其中$e$、$S$、$v$均为确定值,只有大量
自由电子定向移动的平均速率$v$需进一步讨论,而自由电子
的定向移动速率是受电场力作用作加速运动积累起来的,由
此我们应抓住$v$与外电场$E,U$间存在关系这一点,作定量
讨论。

根据力学和静电场的知识可以推得一个自由电子定
向移动的平均速率:
\[v=\frac{0+v_{\tau}}{2}\to v=\frac{1}{2}v_{\tau}\to v_{\tau}=a\tau\to a=\frac{Ee}{m}=\frac{Ve}{\ell m}\]
得\[v=\frac{1}{2}\frac{e\tau }{m\ell }U\]

对大量自由电子来说,可以认为每个自由电子都以这
个平均速率作定向运动,因此大量自由电子的平均定向移动
速率$v=\dfrac{e\tau }{2m\ell }U$。

代入$I=neSv$, 得
\[I=\frac{ne^2\tau S}{2m\ell}U\]

紧接着可以讨论教材第230页第3题.得出
\[R=\frac{2m\ell}{ne^2\tau S}\]
对照电阻定律$R=\rho\ell /S$
可知
\[\rho=\frac{2m}{ne^2\tau}\]
其中$\tau$是自由电子两次碰撞时间的平均值,它与温度高低有关系,温度高,
自由电子热运动剧烈,碰撞频繁,$\tau$值小,这就从微观的角度
解释了$\rho$与温度有关。

最后教材指出了经典电子论的不足,说明金属导电的理
论经历了由经典理论到量子理论的发展过程。这里可对学生
渗透辩证唯物主义的认识论和真理观的思想教育。

\subsection{第二单元}
\subsubsection{液体的导电性}
\begin{figure}[htp]
    \centering
\includegraphics[scale=.8]{fig/8-2.png}
    \caption{}
\end{figure}
可以增加一个如图8.2所示的演示
实验来引入新课。容器中放入
蒸馏水,按下电键,小灯泡并不
发光,说明该液体不导电。再加入细盐末,用玻璃棒轻轻搅动
或稍等一些时间后,按下电键,
小灯泡发光。说明盐溶液能够
导电。进而可以对金属导电与液体导电加以比较:
\begin{enumerate}
    \item 一切金
属都能导电,而液体则不然,只有某些物质如酸、碱、盐(电解
质)的水溶液或它们熔解成液体时才能导电;
\item 金属的导电是
电子导电,液体的导电是离子导电,当有外电场时,正、负离子
同时做方向相反的定向运动形成电流。在一段时间内,通过电
解质某一截面的电量等于通过该截面的正、负离子的电量绝
对值之和。
\item 金属导电时,金属本身并不变化,而液体导电时,电解质要发生化学变化,极板上有物质析出。讲这一点
时,考虑到化学上还未讲电解、电镀,可作电解的演示实验,使
学生对液体导电极板上有物质析出这一点,有直观的、生动的
印象。
\end{enumerate}

如上所述,进行本节教学,抓住液体导电与金属导电的比
较,不仅复习、巩固了金属导电的知识,而且使学生对液体导
电知识了解得更清楚,密切了这两部分知识的联系。

\subsubsection{法拉第电解定律}

关于法拉第电解第一定律的教学,
首先是复习引入课题——析出物质的质量跟通电的电流强度
和时间有什么关系呢?让学生思考,从何着手解决这一问题?
当学生提出用实验方法研究解决这一问题之后,再让学生思
考回答这个实验该怎么做?步骤如何?然后指导学生阅读教材
233页第二行至该段末,让学生了解实验的结果。要求学生
逐步写出表达式:$m\propto I$, $m\propto t$, 故$m\propto It$, 写成等式需要引入
比例系数,有$m=kIt=kq$. 这样可以锻炼学生运用数学语言
进行表达的能力。关于电化当量$k$的教学,可以指导学生阅
读教材233页倒数第5行至234页第5行,设计几个问题用
以检查、巩固学生的阅读效果。例如:不同物质的电化当量是
否相同?电化当量与化学当量同不同?电化当量的大小,在数
值上等于什么?单位是什么?铜的电化当量是$0.3294\x10^{-6}{\rm kg/C}$表示什么意思?测定某物质的电化当量应测哪些量?用
什么式子计算?测定极板上析出物质质量,不用天平应测哪
些量?

关于法拉第电解第二定律的教学,可以指导学生阅读教
材234页第6行至倒数第4行.然后指定每行学生各根据
233页表中数据计算出一个$F$的值,再抽问各行计算的$F$值,
记录在黑板上,以加深恒量$F$与物质种类无关的印象,并知
道$F$的大小和单位,教材235页第2至3行,叙述$F$数值
大小的一段话也可用数学语言表达:由(3)式
$m=\dfrac{Mq}{Fn}$有
\[F=\frac{1}{m}\cdot \frac{M}{n}\cdot q\]
当$m=M/n$时,有$F=q$.

尽管法拉第电解第一、第二定律都是实验定律,但是也可
以从微观角度进行解释,这样可以帮助学生理解和掌握这两
条重要定律。

\subsubsection{电子电量的确定}

教材上第一、二、三自然段叙述十
分清楚,第四自然段采用叙述推理,也可考虑采用如下的推导
进行表述:
\begin{enumerate}
    \item 当$m=M/n$时,由教材234页(3)式有
    \begin{equation}
        q=F
    \end{equation}

\item 通过电解质的电量$q$是离子携带电量的总和。当
$m=M/n$时,电解质含有的离子数是$N/n$,每个离子所带电量是
$q_n$,有
\begin{equation}
    q=\frac{N}{n}q_n
\end{equation}
\item 由(1)、(2),有
\begin{equation}
    F=\frac{N}{n}q_n
\end{equation}
当$n=1$时,任何一个一价离子都带有一分基本电荷的电
量$e$, 有$q_n=e$. 代入(3)式得$F=Ne$.

$\therefore\quad $有 $e=F/N$。
\end{enumerate}

这个式子的重要意义在于,测出一个微观量$e$(或$N$)可
利用法拉第恒量$F$与它们的关系,算出另一个微观量$N$(或
$e$)的值。在阅读教材里介绍了测定$N$和$e$的历史情况。通
过这两节的讲述,应当使学生体会到人们是怎样通过宏观测
定来确定微观恒量的,微观恒量间有联系,而且用不同方法
得到的恒量数值相符,证明了人类的认识正确地反映了微观
世界的规律。

\subsection{第三单元}
这一单元的教学,要注意做好数量较多的演示实验,使学
生对气体的导电性和各种自激放电现象有一个生动的印象,
另外,也需要将气体的导电体理、条件与金属导电和液体导电
进行对比。

\subsubsection{气体的导电性}

在用教材图8.4所示的演示来说明
气体可以导电时,不要对这个导电过程作具体分析。因为这
种情形下的空气电离是一种暂时发生的过程,情况比较复杂。
更不能把这种情形下的空气电离解释为是热电离。所谓热电
离是中性气体分子在高温下运动加剧,相互碰撞而发生的电
离。对于最容易发生热电离的碱金属蒸气,发生热电离的温
度为3000K左右.在大气压下,空气电弧发生热电离时,温度
约为5000K—6000K或更高一些,教材图8.4所示的电离,
是用酒精灯加热的,远低于空气热电离所需要的温度,因而
这种电离不是热电离。

关于气体导电跟金属导电,电解质导电的区别,可以提出
问题让同学讨论,最后得出结论:气体导电既有电子导电,又
有离子导电,从而培养学生分析问题的能力。气体的电离过
程,也可以结合介绍下面要介绍的电子碰撞电离,利用教材图
8.6来加以说明.

教材是在说明什么是电离剂之后介绍被激放电和自激放
电的.在利用教材图8.5来说明被激放电时,可以对图中的
电离剂作具体的说明,这里可以是紫外线,放射性元素发出的
射线等。

教材在注释里说明了正离子很少使气体发生电离,这一
点应让学生知道。正离子很少使气体发生电离的原因,也比
较复杂,不需要向学生介绍。

教材在介绍自激放电时提到了电子发射、正离子轰击发
射和热电子发射,对这些问题的进一步了解,可推荐学生参
看这节教材后面的阅读材料:电子发射。

\subsubsection{几种自激放电现象}

教材里讲述的各种自激放电现
象,都是通过具体的产生过程来介绍的,关键是要讲清楚各种
放电现象的特征。应该通过演示让学生观察,最后再对各种
放电现象加以比较,从而使学生对各种放电现象有比较清楚
的认识。

所谓各种放电现象的特征,可以从放电的名称,以及教材
对各种放电现象的描述来认识。例如,火花放电,它放电时发
出的既不是辉光,也不是弧光或电晕,而是象教材所描述的火
花。即“是一束明亮的、曲折而分叉的细丝。这些明亮的细丝
很快地穿过两极间的气体,一个接着一个地出现,并且伴有爆
炸声,这就是火花放电。”

对于各种放电现象的产生,可按照教材里介绍的,联系上
一节介绍的自激放电的条件来讲述,关于教材里介绍的产生
各种放电现象的具体过程,不必作进一步的深究,使学生有一
个大致的了解就可以了。可以告诉学生,教材里介绍的这些
过程,即弧光放电、火花放电、电晕放电等,都不仅是在大气压
下才能产生的放电现象。

\subsubsection{气体电光源}

这一节是选讲内容,重点是要使学生了
解霓虹灯、日光灯和高压水银灯的工作原理。这一节的学习,
可以先让学生自己阅读教材,然后提出一些问题让学生回答
和讨论,培养学生的阅读能力和分析问题的能力。下面是可
以让学生回答和讨论的问题。
\begin{enumerate}
\item 日光灯里的电子发射属于什么电子发射?
\item 为什么日光灯发出的光比较柔和,并且发光效率高?
\item 日光灯的优缺点是什么?
\item 为什么日光灯又叫做低压水银灯?
\item 高压水银灯为什么采用两层玻璃壳?里面的一层为什么
\item 采用耐高温的石英玻璃管?
\item 高压水银灯的起动过程怎样?
\item 高压水银灯的优缺点是什么?
\end{enumerate}

如有破碎的灯管让学生观察了解其构造时,最好是把破
的灯管放在盘子里,不要让学生直接用手去拿,以免把手划破
和接触水银。

\subsection{第四单元}
\subsubsection{真空中的电流}

介绍真空中的电流时,由辉光放电复
习引入后,要做好教材彩图5甲、乙、丙三个演示,启发学生注
意观察现象并进行分析,认识阴极射线的产生及其性质。

实验甲,先要说明真空管内不可能产生气体电离发光,再
从对着阴极的玻璃壁发出荧光和十字形阴影,说明是阴极发
出的一种射线引起的,由阴极发出的射线就叫阴极射线。实
验乙中,阴极射线能使叶轮转动,说明它具有很大动能,有很
高的速度。实验丙,电场能使阴极射线偏转而且偏向正极,说
明它带负电。如果用强蹄形磁铁来作,也可看到阴极射线的
偏转,根据左手定则,也可以推知阴极射线带负电,最后指出
由英国科学家汤姆生的实验测定知道阴极射线就是高速电
子流。

另外要指出,真空中的电流与气体导电是不同的。气体
导电是气体自身出现带电微粒——正离子、电子以及电子附
着在中性分子上形成的负离子。这些带电微粒在电场力作用
下定向运动使气体变为导体。其中电子的作用,一方面做定
向运动参与导电,另一方面与中性原子或分子碰撞,产生碰撞
电离。在阴极射线管中,管内抽成真空,谈不上有什么导体导
电,阴极发射出的电子在电场作用下高速奔向阳极,既碰不到
气体分子,也不可能产生电离作用。

\subsubsection{示波管} 

介绍示波管的构造时,可先板画出示意图讲
解,再出示实物,(如有废示波管,可敲掉玻璃壳让学生传观),
使学生认识示波管的外形,并识别出电子枪、偏转电极、荧光
屏三大部分。

介绍示波管的工作原理前,可提问检查上节课后布置复
习的带电粒子在电场中的运动的知识,或在本节课内带领学
生一起重温静电学中这节的主要内容:带电粒子在加速场中
被加速,在偏转场中被偏转,带电粒子在偏转场中侧移距离求
法,合运动及分运动的性质等等。在带电粒子偏转装置的基
础上,再增加热电子发射装置,一对偏转板和一个荧光屏,就
成了示波管。要使学生明确电子束的偏转情况是由$XX'$偏
转板上的扫描电压和$YY'$偏转板上的电压共同决定的,是水
平与竖直两个方向上运动的合运动。当这两个偏转电极上电
压变化的周期相同,起始时刻也相同时,则在荧光屏上可以观
查到要研究的电压的一个完整而稳定的波形。

\begin{figure}[htp]
    \centering
\includegraphics[scale=.6]{fig/8-3.png}
    \caption{}
\end{figure}

为了帮助学生理解可以增加几幅图.第一幅是教材252
页上讲的扫描电压(锯齿波)的图象和扫描时荧光屏上呈现的
水平亮线(图8.3甲、乙).第二幅是教材253页上讲的加在
竖直偏转板上的正弦交流电的电压图象以及在这个电压作用
下荧光屏上呈现出来的竖直亮线(图8.4甲、乙).第三幅是
说明上述两个电压分别加在两对偏转板上而且保持同步时,
荧光屏上的亮点将怎样变化的图象(图8.5)。如果$XX'$上
电压的周期是要研究的$YY'$上电压周期的二倍、三倍……而
起始时刻又相同,那末在荧光屏上将观察到二个、三个……完
整而又稳定的波形。
\begin{figure}[htp]
    \centering
\includegraphics[scale=.6]{fig/8-4.png}
    \caption{}
\end{figure}

\begin{figure}[htp]
    \centering
\includegraphics[scale=.6]{fig/8-5.png}
    \caption{}
\end{figure}

在介绍电子枪的作用时可以指出,电子枪不仅发射电子,
而且还可以通过改变加在电子枪中控制极与阴极之间的电压
来改变电子枪射出的电子束的强弱,从而使荧光屏上亮点的
亮度发生变化(为下一节课介绍示波器面板上“辉度旋扭”的
作用作准备),改变加在电子枪中第一阳极与阴极之间电压
的大小,还可以改变电子枪射出电子聚成光点的大小(为下节
课介绍示波器面板上“聚焦旋钮”的作用作准备)。

在介绍荧光屏起着显示电子束位置的作用时,要让学生
知道,即使荧光屏上出现一个小亮点也是大量电子连续不断
射到荧光屏上作用的结果。


\subsection{第五单元}
这一单元教材中共有十二幅图,分别显示了半导体的微
观结构,掺杂质后生成的P型、N型半导体的微观结构,以及
PN结的微观结构,给PN结加上正向或反向电压后阻挡层
的变化情况,画出了二极管、三极管的结构图,符号图和三极
管的电流分配情况。讲解时充分利用这些图,可以帮助学生
理解并加深印象。

\subsubsection{半导体导电机理}

半导体的导电性一节重点是要讲
清楚电子导电和空穴导电,半导体的电子导电跟金属的电子
导电一样,学生不难理解。空穴导电是半导体导电的特点,也
是理解半导体导电的关键。这里要着重说清楚束缚电子的填
补运动跟自由电子的移动是不同的。在外电场作用下,束缚
电子逆着电场方向的填补运动,从效果上看好象空穴顺着电
场方向移动,这种空穴顺着电场方向定向移动形成的电流,就
是半导体的空穴导电。

半导体的热敏特性和光敏特性可以通过演示来说明,这
里介绍半导体的热敏特性和光敏特性是为了说明半导体有着
它特殊的导电性质,并不需要对热敏特性和光敏特性产生的
机理作出解释。

在理解了半导体的电子导电和空穴导电机理的基础上,
进一步认识N型半导体和P型半导体还是比较容易的。可
以只讲述其中的N型半导体,P型半导体让学生自己阅读课
本,以培养学生的阅读能力。

\subsubsection{晶体二极管和PN结} 

晶体二极管的单向导电性是
与电阻比较而言的,做教材图8.26演示时,可把二极
管换成电阻元件再作一次。讲二极管、三极管时,可提供一
些实物让学生观察,并让学生画一画、记一记二极管、三极管
的符号。

需要向学生说明的是,PN结不是由P型半导体材料和
N型半导体材料机械拼凑而成的,要用一定的工艺对P型材
料和N型材料加工,如教材图8.25右图所示的
面接触型二极管,它的PN结是用合金烧结法或扩散法工艺
加工制成的,晶体三极管也不能由两只二极管接拢凑成。要
按不同类型管子的要求加工做成。

形成PN结的微观过程比较复杂,为了帮助学生理清线
索,可归纳小结如下:
\begin{enumerate}
\item 扩散运动——复合——阻挡层开始形成。
\item 继续扩散——阻挡层变厚——阻碍扩散,扩散减弱。
\item 扩散减弱——稳定的阻挡层即PN结生成。
\end{enumerate}

\subsubsection{晶体三极管}

介绍晶体三极管的结构时,主要是让学
生知道晶体三极管并不等于两个二极管的简单组合。这里提
到的三个区、三个电极和两个PN结的名称,学生很难一下子
都记住,可以只对照着三极管的符号图记忆三个电极的名称。
这样有利于讲述三极管的放大作用;并且实际应用中,更多的
情况是要知道三极管的三个电极。

三极管的放大作用最好利用教材图8.30所示的电路实
际测量的数据纪录来说明.根据数据,先结合教材图8.31说
明三极管的电流分配关系,即$I_e=I_b+I_c$和$I_b\ll I_c$; 再说明三
极管的放大作用。

\section{实验指导}
\subsection{演示实验}
\subsubsection{液体导电中离子的运动}
实验装置如图8.6所示,将蛋壳洗净晾干后仔细地在开口
处用三根棉线系牢,以便固定在铁架上,演示前将氢氧化钠溶
液置入蛋壳内,并插入一根炭棒,炭棒用导线与电源负极相
连,再取一个烧杯,内盛硫酸钠溶液,同时滴入几滴酚酞作指
示剂,并插一根洗净的炭棒作电极,用导线与电源正极相连,
最后将蛋壳小心浸在硫酸钠溶液中。
\begin{figure}[htp]\centering
    \begin{minipage}[t]{0.48\textwidth}
    \centering
    \includegraphics[scale=.6]{fig/8-6.png}
    \caption{}
    \end{minipage}
    \begin{minipage}[t]{0.48\textwidth}
    \centering
    \includegraphics[scale=.6]{fig/8-7.png}
    \caption{}
    \end{minipage}
    \end{figure}

演示分两步进行:
\begin{enumerate}
    \item 断开电键$K$, 可观察到烧杯内无变
色的现象,说明蛋壳内的氢氧根离子并未进入烧杯.
\item 接通
电键$K$, 立即会发现蛋壳外靠近电极这面的液体逐渐变红,而
且红色区域逐渐向正极扩展,说明有氢氧根离子进入溶液,而
氢氧根是带负电的,从而说明溶液在通电时有带电离子定向
运动。
\end{enumerate}

\subsubsection{气体的导电性}
教材图8.4所示的实验,跟一般的静电实验一样,
关键在于两个静电计的金属杆与外壳之间、静电计与空气和
桌面之间有良好的绝缘,这个实验也可以用如图8.7所示的
实验代替。$A$、$B$是两块绝缘金属板,高压电源选用感应圈,
与演示用灵敏电流计串联组成回路。$A$、$B$间的间距选择到
这样的程度,在高压电源接通后灵敏电流计中恰好无电流。

演示时,将酒精灯火焰放在$A$、$B$两板之间,给空气加热,
可观察到电流计的指针偏转,说明空气已成为导体。

\subsubsection{稀薄气体的辉光放电}
实验装置采用如课本图8.9所示的低气压放电管。

低气压放电管的两个电极,一个呈圆片形,为阴极,使用
时接电源的负极,一个呈圆棒形,为阳极,使用时接电源的正
极,高压电源选用感应圈。抽气机应通过一个三通与放电管
气压计相连,连接的胶管应选用硬胶管(软胶管在抽气中会
被大气压扁而不能继续抽气)。

实验时,先给放电管通电,然后抽气,随着气压的变化,放
电现象也随之变化。下面列出空气在不同气压下的放电
现象:
\begin{itemize}
\item 760—50mmHg: 不产生放电现象;
\item 40mmHg: 出现紫色线形光条纹;
\item 10mmHg: 紫色光条变宽,几乎充满全管.阴极周围出现
彩色辉光;
\item 3mmHg: 阴极周围紫色辉光,阳极发生红色辉光充满全
管,并开始出现克鲁克斯暗区;
\item 1mmHg: 阴极区兰色辉光鲜明,阳极出现鳞片状辉光,并
开始出现法拉第暗区;
\item 0.1mmHg: 法拉第暗区加长,由阳极发出的鳞片状辉光
减少,管内出现灰白色棉状辉光,在阴极附近开始出现荧光;
\item 0.02mmHg: 阴极发出射线,管壁出现亮的荧光。
\end{itemize}

实验时使用施片式真空泵,其极限真空度为$5\x10^{-4}$mmHg, 可以观察到全部辉光现象。若使用手摇抽气机,其极
限真空度为0.4mmHg, 则不能观察到全部辉光现象。

为了使学生能更清晰地观察低气压放电现象,在做了上
述实验,表明辉光现象随气压的变化而变化后,最好再用低气
压放电管组进行演示,低气压管组内的空气气压,正好选用了
上面六个数值,即40mmHg、10mmHg、3mmHg、1mmHg、
0.1mmHg和0.02mmHg. 观察到的放电现象清晰而稳定。

实验要在暗室中进行。

\subsection{学生实验}
\subsubsection{测定铜的电化当量}
低压直流电源(输出电流3A),安培表(3A),计时表,电
键,硫酸铜溶液,电极(铜片或碳棒),天平(学生天平,感量20
mg)。

电解质溶液的配制:
要获得良好的实验效果,电解质溶液的配制十分重要,在
硫酸铜溶液中加入适量的浓硫酸,增强溶液的导电性;最好还
加入少量的葡萄糖作为添加剂,改善阴极板上铜的沉积质量
(用60克硫酸铜和300毫升水配制的硫酸铜溶液,可加入17
毫升浓硫酸和10克葡萄糖).

极板特别是阴极板不能有毛刺和污迹,否则,阴极板
各处的电流密度不一样,电流密度大的地方铜沉积得快,结
果会出现结瘤或须状物。因此事先要用细砂纸将阴极板擦
干净。

为了减少误差,要增大电流强度和通电时间.在电源
功率允许的条件下,可适当增大电流强度,但电流不宜过分
大,否则阴极板析出的铜太疏松容易脱落。该实验以做两次
为宜。因此,需准备两个阴极板,在第一次实验通电结束后,
立即接上另一个阴极板做第二次实验,使称量第一次的阴极
板铜片质量与第二次通电电解同时进行,这样可大大节约
时间。

\subsubsection{练习使用示波器}

该实验使用J2459示波器旋钮开关共16个,学生是
第一次接触这样复杂的仪器,因此,在实验前,可以指导学生
将16个开关和旋钮分为三组(图8.8),逐个讲解和演示它们
的作用。

\begin{figure}[htp]
    \centering
\includegraphics[scale=.6]{fig/8-8.png}
    \caption{}
\end{figure}

第一组亮斑调节:包括辉
度调节旋钮、聚焦调节旋钮、辅
助聚焦调节旋钮及电源开关。

第二组竖直方向调节:包
括垂直位移旋钮、Y增益旋钮、
衰减旋钮、“Y输入”和“地”旋
钮以及“DC.AC”选择开关。

第三组水平方向调节:包
括水平位移旋钮、X增益旋钮、
扫描范围旋钮、扫描微调旋钮、“X输入”旋钮以及“同步”选择
开关。

这三组旋钮在面板上的位置大致如图8.8所示.

可向学生指出,这个实验有六个方面的练习内容.
\begin{enumerate}
    \item 开机练习; \item 寻找光点、调节光点的亮度与大小的练
    习; \item 寻找扫描线、调节水平幅度大小的练习; \item 在竖直方向
    加一直流电压观察光点向上、向下偏移的练习; \item 测量一节干电池电压的练习;\item 关机的练习。
\end{enumerate}


这样做,可以使学生较快地熟悉示波器,逐步做到调节有
序,测量有方。

\section{习题解答}

\subsection{练习一}
\begin{enumerate}
    \item 在金属导体中,自由电子的热透动速率和定向移动逮率之间有什么区别?这两种速率哪个大?

    \begin{solution}
        金属导体中自由电子的热运动不需要外加条件,是
        不停地永远进行着的,热运动的速率大小与金属的温度有关。
        由于热运动的不规则性。这种运动不能形成电流;金属中自
        由电子的定向运动。必须有电场存在才能发生。运动方向总
        是逆电场方向的,这种运动形成金属中的电流。自由电子的
        平均定向运动速率比热运动速率小得多。
    \end{solution}
    
    \item 电路接通后,为什么整个电路中几乎同时形成电流?

    \begin{solution}
        电路接通后,电路里便以光速在各处极迅速地建立
        起电场,整个电路中的自由电子几乎同时受到电场力的作用
        作定向运动,所以整个电路中同时形成电流。
    \end{solution}
    
    \item 利用公式$I=\dfrac{e^2 nS\tau}{2m\ell}U$,求出这段导体的电阻$R$和制
成这段导体的材料的电阻率$\rho$.

\begin{solution}
    将公式与欧姆定律$I=U/R$
    相比较,则
    \[R=\frac{U}{I}=\frac{2m\ell}{e^2 nS\tau}\]
又$R=\rho\ell/S$,比较上面两个式子,
\[\rho=\frac{2m}{e^2 n\tau}\]
\end{solution}

\end{enumerate}

\subsection{练习二}
\begin{enumerate}
    \item 通过硫酸铜溶液的电量是$2.0\x10^4$库,在阴极上能析出多少克铜?

    \begin{solution}
根据法拉第电解第一定律$m=kq$, 查表得铜的电化
当量$k$为$0.3294\x10^{-6}{\rm kg/C}$,代入上式得
\[m=0.3294\x10^{-6}\x2.0\x10^4{\rm kg}=6.6{\rm g}\]
    \end{solution}
    
    \item 如果要在表面积是5${\rm cm^2}$的器件上,镀上一层20微米厚的银层,需要通过多少库仑的电量?

    \begin{solution}
    根据法拉第电解第一定律$m=kq$. 得
    \[q=\frac{m}{k}=\frac{\rho V}{k}\]
查表得银的密度$\rho=10.49{\rm g/cm^3}=10.49\x10^3{\rm kg/m^3}$; 
银的$k$值为$1.118\x10^{-6}{\rm kg/C}$,代入上式得
\[q=\frac{\rho Sh}{k}=\frac{10.49\x10^3\x 5\x 10^{-4}\x 20\x 10^{-6}}{1.118\x10^{-6}}=94{\rm C}\]
    \end{solution}
    
    \item 有一个学生电解硫酸铜溶液来测定铜的电化当量.他在通电以前称一次阴极板,通电25分钟以后再称一次,从而知道析出的铜的质量是0.29克,还知道通过溶液的电流强度是0.6安,从这些数据算出铜的电化当量是多少?

    \begin{solution}
        根据法拉第电解第一定律$m=kIt$,得
        \[k=\frac{m}{It}=\frac{0.29\x 10^{-3}}{0.6\x 25\x 60}=3\x 10^{-7}{\rm kg/C}\]
    \end{solution}
    
    \item 锌的摩尔质量是0.06538kg/mol,它的化合价是2,求锌的电化当量.

    \begin{solution}
        根据法拉第电解第二定律
\[k=\frac{M}{Fn}=\frac{0.06538}{2\x9.65\x10^4}=0.339\x10^{-6}{\rm kg/mol}
\]
    \end{solution}
    
    \item 金的摩尔质量是0.1972kg/mol,它的化合价是3,要想使金电解池的阴极上析出1克金,需要通过多少库的电量?

    \begin{solution}
        根据法拉第电解第一、第二定律有
        \[m=kq=\frac{Mq}{Fn}\]
        则:
\[q=\frac{mnF}{M}=\frac{10^{-3}\x 3\x 9.65\x 10^4}{0.1972}=1.47\x 10^3{\rm C}\]
    \end{solution}
    
\end{enumerate}

\subsection{练习三}
\begin{enumerate}
    \item 放电管里气体的自激放电是怎样形成的?

    \begin{solution}
        放电管里的气体中含有少量的自由电子和正离子,
        在高电压强电场作用下分别奔向阳极和阴极的过程中动能增
        加,如果电压足够高,电场足够强,电子的动能增大到一定程
        度时,电子跟中性原子碰撞。会从原子中打出电子,发生气体
        电离,而动能足够大的正离子轰击阴极表面时,能使阴极发
        射电子。这些电子又会使气体发生电子碰撞电离,产生新的
        电子和正离子,于是形成自激放电,也就是说形成自激放电
        的条件是气体电离和阴极发射电子。
    \end{solution}
    
    \item 为什么安装了避雷针能够避免雷击?

    \begin{solution}
        避雷针是尖端状导体,位置高过建筑物。当带电云
        层与建筑物接近时,放电通过避雷针和接地导体这条通路不
        断进行,避免了用电荷积累,使云层和大地之间产生高压放
        电现象。所以不再发生雷击。
    \end{solution}
    
    \item 示波管荧光屏上显示出的一条完整的正弦曲线(图8.17),是在什么条件下得到的?这时,如果使扫描电压的周期增大为原来的二倍,那么,荧光屏上将显示出什么样的图线?

    \begin{solution}
        扫描电压的周期与竖直极板上电压周期相等,并且
        起始时间也相同,在荧光屏上就会显示出一条完整的正弦曲
        线。如果扫描电压的周期增大为原来的二倍,荧光屏上将显
        示出一条稳定的有两个完整周期的正弦曲线。
    \end{solution}
    
\end{enumerate}


\section{参考资料}
\subsection{关于金属电子论对电阻的解释}
金属具有优良的导电性和导热性,这一点早就为人们所
知,但要从理论上说明金属导电、导热的规律及二者间的联
系,却象力学一样经历了经典理论到量子理论的发展过程,正
是因为把电子作为纯经典粒子处理,把适用于宏观物体运动
的牛顿力学以及适用于理想气体分子的玻尔兹曼和麦克斯韦
统计分布律用于自由电子,才建立了经典电子论,特鲁德和
洛仑兹为这个理论的建立作出了重要的贡献。

1887年汤姆生发现电子,1900年特鲁德提出自由电子理
论。特鲁德假设金属中的电子与理想气体分子一样,称做电
子气。电子气能在金属中自由运动,和金属中的正离子碰撞
在一定温度下达到热平衡状态。

1904年洛仑兹进一步提出电子气服从麦克斯韦-玻
尔兹曼统计分布规律,能解释欧姆定律、热传导规律等。但
在解释金属的热容量和电阻与温度的关系方面遇到了困难。
这里只谈一谈金属的电阻问题。

根据洛仑兹的经典电子理论来解释欧姆定律,已经得到
\[I=\frac{n e^{2} S \tau}{2 m \ell} U , \qquad R=\frac{2 m \ell}{e^{2} n S \tau} ,\qquad \rho=\frac{2 m}{e^{2} n \tau}\]
式中$\tau$是自由电子相继,两次碰撞的平均时间。因为自由电子的平均定向移动速率远
远小于它的热运动平均速率$\bar v$,所以平均定向移动速率可以
略去不计。这样用电子的平均自由程$\bar\lambda$除以电子热运动的平
均速率$\bar v$,就得到自由电子经过一个平均自由程的平均时间
$\tau=\bar\lambda/\bar v$,代入上式,有:
\[\rho=\frac{2m\bar v}{e^2n\bar\lambda}\]
由于$\bar v=\sqrt{\dfrac{8kT}{\pi m_A}}$,所以导体
的电阻率$\rho$与热力学温度$T$的平方根成正比。而实验结果却
是电阻跟热力学温度$T$成正比,理论与实验不符,这暴露了经
典理论的缺陷。

1928年以后,在量子力学的基础上提出了自由电
子量子理论,才解决了上述矛盾。由于电子是自旋为
半整数的费米子,它不服从玻尔兹曼分布,而服从费
米-狄拉克统计规律。应用量子理论讨论的结果,由
经典电子论得到的电阻率$\rho$的表达式中的$\bar v$, 要用费米速度$v_F$\footnote{根据经典理论,$T=0$时,所有电子的动能都等于零.在量子
理论中,即使在$T=0$(基态)时,电子也不可能具有零动能.不相容原
理,又不允许两个(具有相反自旋)以上的电子处于同一能级。因此,电
子将由低到高依次填满各个相应的能级,我们把电子气处于基态时的
最高电子能量,称为电子气的费米能$E_F$. 与费米能对应的电子速度
\[v_F=\sqrt{\frac{2}{m}E_F}\]
称为费米速度。 }
(它只与金属自由电子的数密度有关)来代替;而电子的
平均自由程则由金属中电子的散射过程所决定,其数值与金
属中原子由于热运动而离开平衡位置的位移的均方成反比,
因而与绝对温度$T$成反比。由此可得出金属电阻率$\rho\propto T$的
结论。

\subsection{电解液中的电流}
电解液中的正负离子在没有外加电场时,做无规则的热
运动。在有外加电场时,正负离子在电场力的作用下,分别向
两个电极作定向移动。这种电解液中正负离子的定向移动即
是电解液中的电流。

各种离子在电场作用下迁移的速度一般是不同的,离子
所带的电荷越多,受外加电场的作用力越大,带电量相同的正
负离子,虽然它们在电场中受到的电场力大小相等,由于质量
不同,迁移速度$v_+$与$v_{-}$也不相同。因而,在相同的时间里通
过电解液某一截面的正负离子数也不相等,通过这截面的正
负电荷的电量$Q_+$与$Q_-$也不相等。

假如单位体积内有$n_1$个正离子,这些正离子以迁移速度
$v_+$通过截面$S$, 每个离子所带电量为$q_+$, 则在$t$时间内每个
离子前进的距离为$v_+t$, 在$t$时间内通过这个截面的正电荷
的电量为$Q_+=n_1q_+v_+tS$.

同理,如果单位体积内有$n_2$个负离子,它们以迁移速度
$v_-$跟正离子的方向相反通过同一截面$S$, 每个离子所带电量
为$q_-$, 则在$t$时间内每个离子前进的距离为$v_-t$, 在$t$时间内
通过这个截面的负电荷的电量为$Q_-=n_2q_-v_-tS$. 

正负离子的运动都形成电流,正离子的定向移动相当于
负离子的反方向运动,两种电荷的运动具有同样的效果。因
此,在$t$时间内通过截面$S$的总电量$Q=Q_+ +Q_-$. 通过这个
截面的电流强度$I=\frac{Q}{t}=\frac{Q_+ +Q_-}{t}$。

\subsection{正离子碰撞电离}
气体放电中的正离子在电场作用下跟中性原子碰撞时,
很少使原子发生电离。这个实验事实可作如下的解释:

我们把气体中两个粒子的碰撞看作是在一个孤立系统中
发生的,这个系统的运动是由它们的质心的运动和每一个粒
子相对质心的运动所组成的。中性原子在碰撞中发生电离就
是动能转化为系统内部的势能而实现的;并且在这种转化过
程中,质心运动的动能保持不变。

为了使问题简化,只讨论正碰的情况、假设碰撞粒子的
质量为$m_1$, 碰撞前的速度为$v_1$; 被碰撞粒子的质量为$m_2$, 碰
撞前是静止的。取被碰粒子中心为坐标原点,碰撞前碰撞粒
子的坐标为$r$, 系统质心的坐标为$x$,则根据质心的定义有
$x(m_1+m_2)=rm_1$,即
\[x=r\frac{m_1}{m_1+m_2}\]

\input{9ref.tex}




%\input{app1.tex}
\backmatter
%\input{app2.tex}



%\begin{appendix}
%	\input{app3.tex}
%\end{appendix}







\end{document}