\chapter{物质的导电性}\minitoc[n]
\section{教学要求}
这一章将有关物质导电性的知识集中起来,系统地阐述
了金属.液体、气体和半导体等各种物质的导电性,以及真空
中的电流,这样,可以使学生对物质导电性有一个较为全面
的认识;同时也使学生有可能通过比较学习,更容易理解和掌
握.教材对各种不同的导电现象在技术上的实际应用的介
绍,可以丰富学生的知识,开阔他们的眼界.

这一章从物质的微观结构来讨论各种物质的导电机理,
对一些宏观现象要作出微观解释,从而使学生对宏观现象的
认识深入一步.当然这种认识还是很初步的,为了不加重学
生的负担,教学中一般不要再补充,但这种把宏观和微观统一
起来讨论问题的方法很重要,应当使学生有进一步的认识.

本章教材对于金属导电、电解液导电和电子电量的确定
等内容进行了定量讨论,这些内容应作为重点来讲授.除此
而外,由于气体导电和半导体导电的过程都比较复杂,有些是
无法用经典理论加以解释的,因此只作定性叙述,不涉及定量
研究,有些演示实验,也以定性观察为主.

本章内容可以分为五个单元:第一单元包括引言和第一
节,讲述金属的导电性;第二单元从第二节到第四节,讲述液
体的导电性;第三单元从第五节到第七节,讲述气体的导电
性;第四单元包括第八节和第九节,讲述真空中的电流和示波
管;第五单元从第十节到第十三节,讲述半导体的导电性及晶
体二极管和三极管.

第一节金属的导电性,教材从金属的微观结构出发,介绍
了金属导电的微观机理,导出了电流强度和自由电子平均定
向移动速率的关系式和欧姆定律的微观表达式.对金属导电
的微观解释,以及作为下一章推导洛仑兹力基础的电流强度
与自由电子平均定向移动速率的关系,决定了这节是重点知
识.但不要求学生记忆欧姆定律的微观表达式和掌握推导
过程.

第二节在学生已有的化学知识的基础上讲述液体的离子
导电,并分析了离子导电的电解过程,在讲述电解的应用时,
着重介绍了电解电容器(电解、电镀在化学课有专门的讲述).
这一节教学主要是要求学生知道电解质导电是怎样形成的,
以及离子导电与电子导电的区别.

第三节是重点,知识讲述电解质导电的重要规律——法
拉第电解第一定律和第二定律,学生应当掌握这两个定律,理
解法拉第恒量的物理意义.

第四节介绍了一种确定电子电量的方法,这里导出的计
算基本电荷的公式$e=F/N$十分重要,它把法拉第恒量$F$、基
本电荷$e$、阿伏伽德罗常数$N$这三者联系起来,对于测量$e$
和$N$这两个微观量有重要意义.

第五节《气体的导电性》中,教材首先通过演示,使学生知
道,空气在火焰的作用下发生电离,变成了导体,可以导电.接
着介绍了气体导电跟电解质导电、金属导电的区别,以加深学
生对气体导电机理的认识,最后介绍了被激放电和自激放
电,并对形成自激放电的过程作了具体地分析,从而得出了产
生自激放电的条件:气体电离和阴极发射电子.对这个条件
应当要求学生清楚地了解,下节讲述几种自激放电现象,虽然
各有特点,但都要用这个条件给以说明.

第六节介绍了几种自激放电现象,要求学生对它们有所
了解,并且知道它们各自的特点,关于这几种放电的形成,教
材只作了简要介绍,不要求仔细地展开讲解,也不要求把它们
进行对比,学生能够对自激放电的条件大致有所了解就可以
了.关于辉光放电,教材没有把辉光分成若干个区,教学中也
不要求补充介绍.关于弧光放电,教材讲述的主要是大气压下
的弧光放电现象,对其形成过程作了简单介绍,弧光放电不
是仅在大气压下才能发生的放电现象,另外还有低气压下的
弧光放电,例如汞弧整流管中就是低压下的弧光放电,对此,
教师可以酌情向学生简单介绍.在讲火花放电和电晕放电放
时,前后面处提到了避雷针,但这两处所要讲述的问题是不同
的,前面提出了为什么要安装避雷针,后面则说明了为什么
避雷针能够避免雷击.跟弧光放电一样,火花放电和电晕放
电也不仅是在大气压下才能发生的放电现象.

第七节《气体电光源》是选讲内容.这里介绍的霓虹灯、
日光灯,高压水银灯等都是常见的气体放电光源,也是前两节
知识的应用,让学生对这些光源的工作原理有一个初步的
认识,是十分有益的.

第八节《真空中的电流》要求学生清楚阴极射线的产生和
为什么说阴极射线是从阴极发射出来的电子流.

第九节讲述示波管.要求学生知道示波管的基本构造和
工作原理.要求学生清楚扫描过程和什么是同步,以及荧光
屏上的图线是怎样形成的.所有这些,也都是做学生实验《练
习使用示波器》的基础,对于荧光屏上显示的图线,教材只介
绍了一条完整的正弦曲线的情形.由此不难推知显示出若干
个完整的正弦曲线的条件.

第十节《半导体的导电性》和第十一节《N型半导体和P
型半导体》是第五单元的基础,应当使学生清楚半导体中产生
自由电子和空穴的物理过程,认识半导体的导电机理及其与
金属导电的不同,以及N型半导体和P型半导体的导电
机理.

第十二节《PN结晶体二极管》,对PN结的形成讲解比
较简单,要求比较低,没有提及空间电荷区,也没有说明漂移
运动.要求学生对PN结的形成有一个粗浅的认识.当然,
形成稳定的阻挡层时,并不是没有扩散运动了,而是处于动态
平衡.

第十三节<晶体三极管》,只要求学生知道当基极电流稍
有变化时,集电极电流就有较大的变化,即三极管有放大作
用.不要求根据PN结的性质对放大作用作出解释.

这一章的教学要求是:
\begin{enumerate}
    \item 了解金属的导电机理,知道欧姆定律的微观解释.
    \item 了解液体的导电机理,掌握法拉第电解定律,理解法拉第恒量的意义,知道怎样根据这个恒量确定电子的电量.
    \item 了解气体的导电机理,知道什么是被激放电和自激放
    电,知道几种自激放电现象.
    \item 了解阴极射线,了解示波管的简单原理.
    \item 了解半导体的导电机理,了解PN结的作用和二极管
    的单向导电性,知道三极管有放大作用.
\end{enumerate}

\section{教学建议}
\subsection{第一单元}
课前可布置学生复习教材第四章固体的性质中第74页
一、二、三自然段以及第七章稳恒电流中第174页第二、三、四
自然段,为复习引入本节内容作好准备.

\subsubsection{金属的导电结构}

学习这节教材时,学生往往认为金
属的导电机构是自由电子,乃是已知的事实,而没有注意到本
节是从金属的微观结构出发来讨论的.前面第四章固体的性
质,只笼统地提到晶体的微观结构是由分子、原子或离子形成
空间点阵,电流一节也仅仅提到金属中的自由电荷是自由电
子.如果不理解金属的微观结构和自由电子的无规则热运动,
还会产生一些错误认识,如有的学生认为金属原子离解为正
离子和自由电子后,金属就显正电;或自由电子运动到某部
分,某部分就显负电,金属中出现了电源等.这类看法需要从
自由电子热运动特点,用统计观点加以澄清.要让学生了解,
由于自由电子向各个方向运动的几乎完全相同,可以认为大
量电子在金属导体内是均匀分布的,从统计观点考虑,导体中
任何一部分体积内的自由电子数目都与该体积内金属正离子
所带电荷数相等,所以任何一段导体都不显电性,整个导体
也是不带电的.在没有外电场时,通过金属导体内部任何一个
横截面积的总电量都等于零,这种热运动不会引起任何一个
方向上产生电流,还可以补充两幅图来形象地说明,图8.1
甲中,小黑点表示自由电子,虚线箭头表示自由电子热运动速
度,金属正离子用符号$\oplus$表示.可以看出,在这部分导体中,
自由电子的速度朝哪个方向的都有,它们的数目也跟金属正
离子相等,有电场存在时,自由电子除具有热运动速率.还
具有定向运动速率,如图8.1乙中实线箭头所示(应该说明,
图中所画的自由电子运动速率,跟它们的热运动速率相比,是
夸大了的).导体中出现了自由电子整体的定向运动,因而形
成了电流.

\begin{figure}[htp]
    \centering
\includegraphics[scale=.6]{fig/8-1.png}
    \caption{}
\end{figure}

\subsubsection{关于公式$I=neSv$的推导}

可分成几个步骤,提出
问题启发引导学生自己推导,让学生再次学习这种把宏观与
微观统一起来的讨论方法,体会其优越性.有了这个关系式,
微观量$v$就可由容易测得的宏观量$I$、$S$及已知的$n$、$e$计算
出来,教师可给出书上的数据,让学生动手算算$v$的数值,记
下$v$的数量级.

对于自由电子平均定向移动速率、自由电子热运动平均
速率、电场的传播速率三者的区别,一方面要引导学生从它们
的物理意义和数值大小两个方面加以对比区别;另一方面可
用类比、比喻来帮助学生理解并加深印象.

\subsubsection{欧姆定律的微观解释}

可采用下述的思路推导:

欧姆定律的表达式$I=U/R$
中,已经知道$I=neSv$这个
与微观量有关的关系式.其中$e$、$S$、$v$均为确定值,只有大量
自由电子定向移动的平均速率$v$需进一步讨论,而自由电子
的定向移动速率是受电场力作用作加速运动积累起来的,由
此我们应抓住$v$与外电场$E,U$间存在关系这一点,作定量
讨论.

根据力学和静电场的知识可以推得一个自由电子定
向移动的平均速率:
\[v=\frac{0+v_{\tau}}{2}\to v=\frac{1}{2}v_{\tau}\to v_{\tau}=a\tau\to a=\frac{Ee}{m}=\frac{Ve}{\ell m}\]
得\[v=\frac{1}{2}\frac{e\tau }{m\ell }U\]

对大量自由电子来说,可以认为每个自由电子都以这
个平均速率作定向运动,因此大量自由电子的平均定向移动
速率$v=\dfrac{e\tau }{2m\ell }U$.

代入$I=neSv$, 得
\[I=\frac{ne^2\tau S}{2m\ell}U\]

紧接着可以讨论教材第230页第3题.得出
\[R=\frac{2m\ell}{ne^2\tau S}\]
对照电阻定律$R=\rho\ell /S$
可知
\[\rho=\frac{2m}{ne^2\tau}\]
其中$\tau$是自由电子两次碰撞时间的平均值,它与温度高低有关系,温度高,
自由电子热运动剧烈,碰撞频繁,$\tau$值小,这就从微观的角度
解释了$\rho$与温度有关.

最后教材指出了经典电子论的不足,说明金属导电的理
论经历了由经典理论到量子理论的发展过程.这里可对学生
渗透辩证唯物主义的认识论和真理观的思想教育.

\subsection{第二单元}
\subsubsection{液体的导电性}
\begin{figure}[htp]
    \centering
\includegraphics[scale=.8]{fig/8-2.png}
    \caption{}
\end{figure}
可以增加一个如图8.2所示的演示
实验来引入新课.容器中放入
蒸馏水,按下电键,小灯泡并不
发光,说明该液体不导电.再加入细盐末,用玻璃棒轻轻搅动
或稍等一些时间后,按下电键,
小灯泡发光.说明盐溶液能够
导电.进而可以对金属导电与液体导电加以比较:
\begin{enumerate}
    \item 一切金
属都能导电,而液体则不然,只有某些物质如酸、碱、盐(电解
质)的水溶液或它们熔解成液体时才能导电;
\item 金属的导电是
电子导电,液体的导电是离子导电,当有外电场时,正、负离子
同时做方向相反的定向运动形成电流.在一段时间内,通过电
解质某一截面的电量等于通过该截面的正、负离子的电量绝
对值之和.
\item 金属导电时,金属本身并不变化,而液体导电时,电解质要发生化学变化,极板上有物质析出.讲这一点
时,考虑到化学上还未讲电解、电镀,可作电解的演示实验,使
学生对液体导电极板上有物质析出这一点,有直观的、生动的
印象.
\end{enumerate}

如上所述,进行本节教学,抓住液体导电与金属导电的比
较,不仅复习、巩固了金属导电的知识,而且使学生对液体导
电知识了解得更清楚,密切了这两部分知识的联系.

\subsubsection{法拉第电解定律}

关于法拉第电解第一定律的教学,
首先是复习引入课题——析出物质的质量跟通电的电流强度
和时间有什么关系呢?让学生思考,从何着手解决这一问题?
当学生提出用实验方法研究解决这一问题之后,再让学生思
考回答这个实验该怎么做?步骤如何?然后指导学生阅读教材
233页第二行至该段末,让学生了解实验的结果.要求学生
逐步写出表达式:$m\propto I$, $m\propto t$, 故$m\propto It$, 写成等式需要引入
比例系数,有$m=kIt=kq$. 这样可以锻炼学生运用数学语言
进行表达的能力.关于电化当量$k$的教学,可以指导学生阅
读教材233页倒数第5行至234页第5行,设计几个问题用
以检查、巩固学生的阅读效果.例如:不同物质的电化当量是
否相同?电化当量与化学当量同不同?电化当量的大小,在数
值上等于什么?单位是什么?铜的电化当量是$0.3294\x10^{-6}{\rm kg/C}$表示什么意思?测定某物质的电化当量应测哪些量?用
什么式子计算?测定极板上析出物质质量,不用天平应测哪
些量?

关于法拉第电解第二定律的教学,可以指导学生阅读教
材234页第6行至倒数第4行.然后指定每行学生各根据
233页表中数据计算出一个$F$的值,再抽问各行计算的$F$值,
记录在黑板上,以加深恒量$F$与物质种类无关的印象,并知
道$F$的大小和单位,教材235页第2至3行,叙述$F$数值
大小的一段话也可用数学语言表达:由(3)式
$m=\dfrac{Mq}{Fn}$有
\[F=\frac{1}{m}\cdot \frac{M}{n}\cdot q\]
当$m=M/n$时,有$F=q$.

尽管法拉第电解第一、第二定律都是实验定律,但是也可
以从微观角度进行解释,这样可以帮助学生理解和掌握这两
条重要定律.

\subsubsection{电子电量的确定}

教材上第一、二、三自然段叙述十
分清楚,第四自然段采用叙述推理,也可考虑采用如下的推导
进行表述:
\begin{enumerate}
    \item 当$m=M/n$时,由教材234页(3)式有
    \begin{equation}
        q=F
    \end{equation}

\item 通过电解质的电量$q$是离子携带电量的总和.当
$m=M/n$时,电解质含有的离子数是$N/n$,每个离子所带电量是
$q_n$,有
\begin{equation}
    q=\frac{N}{n}q_n
\end{equation}
\item 由(1)、(2),有
\begin{equation}
    F=\frac{N}{n}q_n
\end{equation}
当$n=1$时,任何一个一价离子都带有一分基本电荷的电
量$e$, 有$q_n=e$. 代入(3)式得$F=Ne$.

$\therefore\quad $有 $e=F/N$.
\end{enumerate}

这个式子的重要意义在于,测出一个微观量$e$(或$N$)可
利用法拉第恒量$F$与它们的关系,算出另一个微观量$N$(或
$e$)的值.在阅读教材里介绍了测定$N$和$e$的历史情况.通
过这两节的讲述,应当使学生体会到人们是怎样通过宏观测
定来确定微观恒量的,微观恒量间有联系,而且用不同方法
得到的恒量数值相符,证明了人类的认识正确地反映了微观
世界的规律.

\subsection{第三单元}
这一单元的教学,要注意做好数量较多的演示实验,使学
生对气体的导电性和各种自激放电现象有一个生动的印象,
另外,也需要将气体的导电体理、条件与金属导电和液体导电
进行对比.

\subsubsection{气体的导电性}

在用教材图8.4所示的演示来说明
气体可以导电时,不要对这个导电过程作具体分析.因为这
种情形下的空气电离是一种暂时发生的过程,情况比较复杂.
更不能把这种情形下的空气电离解释为是热电离.所谓热电
离是中性气体分子在高温下运动加剧,相互碰撞而发生的电
离.对于最容易发生热电离的碱金属蒸气,发生热电离的温
度为3000K左右.在大气压下,空气电弧发生热电离时,温度
约为5000K—6000K或更高一些,教材图8.4所示的电离,
是用酒精灯加热的,远低于空气热电离所需要的温度,因而
这种电离不是热电离.

关于气体导电跟金属导电,电解质导电的区别,可以提出
问题让同学讨论,最后得出结论:气体导电既有电子导电,又
有离子导电,从而培养学生分析问题的能力.气体的电离过
程,也可以结合介绍下面要介绍的电子碰撞电离,利用教材图
8.6来加以说明.

教材是在说明什么是电离剂之后介绍被激放电和自激放
电的.在利用教材图8.5来说明被激放电时,可以对图中的
电离剂作具体的说明,这里可以是紫外线,放射性元素发出的
射线等.

教材在注释里说明了正离子很少使气体发生电离,这一
点应让学生知道.正离子很少使气体发生电离的原因,也比
较复杂,不需要向学生介绍.

教材在介绍自激放电时提到了电子发射、正离子轰击发
射和热电子发射,对这些问题的进一步了解,可推荐学生参
看这节教材后面的阅读材料:电子发射.

\subsubsection{几种自激放电现象}

教材里讲述的各种自激放电现
象,都是通过具体的产生过程来介绍的,关键是要讲清楚各种
放电现象的特征.应该通过演示让学生观察,最后再对各种
放电现象加以比较,从而使学生对各种放电现象有比较清楚
的认识.

所谓各种放电现象的特征,可以从放电的名称,以及教材
对各种放电现象的描述来认识.例如,火花放电,它放电时发
出的既不是辉光,也不是弧光或电晕,而是象教材所描述的火
花.即“是一束明亮的、曲折而分叉的细丝.这些明亮的细丝
很快地穿过两极间的气体,一个接着一个地出现,并且伴有爆
炸声,这就是火花放电.”

对于各种放电现象的产生,可按照教材里介绍的,联系上
一节介绍的自激放电的条件来讲述,关于教材里介绍的产生
各种放电现象的具体过程,不必作进一步的深究,使学生有一
个大致的了解就可以了.可以告诉学生,教材里介绍的这些
过程,即弧光放电、火花放电、电晕放电等,都不仅是在大气压
下才能产生的放电现象.

\subsubsection{气体电光源}

这一节是选讲内容,重点是要使学生了
解霓虹灯、日光灯和高压水银灯的工作原理.这一节的学习,
可以先让学生自己阅读教材,然后提出一些问题让学生回答
和讨论,培养学生的阅读能力和分析问题的能力.下面是可
以让学生回答和讨论的问题.
\begin{enumerate}
\item 日光灯里的电子发射属于什么电子发射?
\item 为什么日光灯发出的光比较柔和,并且发光效率高?
\item 日光灯的优缺点是什么?
\item 为什么日光灯又叫做低压水银灯?
\item 高压水银灯为什么采用两层玻璃壳?里面的一层为什么
\item 采用耐高温的石英玻璃管?
\item 高压水银灯的起动过程怎样?
\item 高压水银灯的优缺点是什么?
\end{enumerate}

如有破碎的灯管让学生观察了解其构造时,最好是把破
的灯管放在盘子里,不要让学生直接用手去拿,以免把手划破
和接触水银.

\subsection{第四单元}
\subsubsection{真空中的电流}

介绍真空中的电流时,由辉光放电复
习引入后,要做好教材彩图5甲、乙、丙三个演示,启发学生注
意观察现象并进行分析,认识阴极射线的产生及其性质.

实验甲,先要说明真空管内不可能产生气体电离发光,再
从对着阴极的玻璃壁发出荧光和十字形阴影,说明是阴极发
出的一种射线引起的,由阴极发出的射线就叫阴极射线.实
验乙中,阴极射线能使叶轮转动,说明它具有很大动能,有很
高的速度.实验丙,电场能使阴极射线偏转而且偏向正极,说
明它带负电.如果用强蹄形磁铁来作,也可看到阴极射线的
偏转,根据左手定则,也可以推知阴极射线带负电,最后指出
由英国科学家汤姆生的实验测定知道阴极射线就是高速电
子流.

另外要指出,真空中的电流与气体导电是不同的.气体
导电是气体自身出现带电微粒——正离子、电子以及电子附
着在中性分子上形成的负离子.这些带电微粒在电场力作用
下定向运动使气体变为导体.其中电子的作用,一方面做定
向运动参与导电,另一方面与中性原子或分子碰撞,产生碰撞
电离.在阴极射线管中,管内抽成真空,谈不上有什么导体导
电,阴极发射出的电子在电场作用下高速奔向阳极,既碰不到
气体分子,也不可能产生电离作用.

\subsubsection{示波管} 

介绍示波管的构造时,可先板画出示意图讲
解,再出示实物,(如有废示波管,可敲掉玻璃壳让学生传观),
使学生认识示波管的外形,并识别出电子枪、偏转电极、荧光
屏三大部分.

介绍示波管的工作原理前,可提问检查上节课后布置复
习的带电粒子在电场中的运动的知识,或在本节课内带领学
生一起重温静电学中这节的主要内容:带电粒子在加速场中
被加速,在偏转场中被偏转,带电粒子在偏转场中侧移距离求
法,合运动及分运动的性质等等.在带电粒子偏转装置的基
础上,再增加热电子发射装置,一对偏转板和一个荧光屏,就
成了示波管.要使学生明确电子束的偏转情况是由$XX'$偏
转板上的扫描电压和$YY'$偏转板上的电压共同决定的,是水
平与竖直两个方向上运动的合运动.当这两个偏转电极上电
压变化的周期相同,起始时刻也相同时,则在荧光屏上可以观
查到要研究的电压的一个完整而稳定的波形.

\begin{figure}[htp]
    \centering
\includegraphics[scale=.6]{fig/8-3.png}
    \caption{}
\end{figure}

为了帮助学生理解可以增加几幅图.第一幅是教材252
页上讲的扫描电压(锯齿波)的图象和扫描时荧光屏上呈现的
水平亮线(图8.3甲、乙).第二幅是教材253页上讲的加在
竖直偏转板上的正弦交流电的电压图象以及在这个电压作用
下荧光屏上呈现出来的竖直亮线(图8.4甲、乙).第三幅是
说明上述两个电压分别加在两对偏转板上而且保持同步时,
荧光屏上的亮点将怎样变化的图象(图8.5).如果$XX'$上
电压的周期是要研究的$YY'$上电压周期的二倍、三倍……而
起始时刻又相同,那末在荧光屏上将观察到二个、三个……完
整而又稳定的波形.
\begin{figure}[htp]
    \centering
\includegraphics[scale=.6]{fig/8-4.png}
    \caption{}
\end{figure}

\begin{figure}[htp]
    \centering
\includegraphics[scale=.6]{fig/8-5.png}
    \caption{}
\end{figure}

在介绍电子枪的作用时可以指出,电子枪不仅发射电子,
而且还可以通过改变加在电子枪中控制极与阴极之间的电压
来改变电子枪射出的电子束的强弱,从而使荧光屏上亮点的
亮度发生变化(为下一节课介绍示波器面板上“辉度旋扭”的
作用作准备),改变加在电子枪中第一阳极与阴极之间电压
的大小,还可以改变电子枪射出电子聚成光点的大小(为下节
课介绍示波器面板上“聚焦旋钮”的作用作准备).

在介绍荧光屏起着显示电子束位置的作用时,要让学生
知道,即使荧光屏上出现一个小亮点也是大量电子连续不断
射到荧光屏上作用的结果.


\subsection{第五单元}
这一单元教材中共有十二幅图,分别显示了半导体的微
观结构,掺杂质后生成的P型、N型半导体的微观结构,以及
PN结的微观结构,给PN结加上正向或反向电压后阻挡层
的变化情况,画出了二极管、三极管的结构图,符号图和三极
管的电流分配情况.讲解时充分利用这些图,可以帮助学生
理解并加深印象.

\subsubsection{半导体导电机理}

半导体的导电性一节重点是要讲
清楚电子导电和空穴导电,半导体的电子导电跟金属的电子
导电一样,学生不难理解.空穴导电是半导体导电的特点,也
是理解半导体导电的关键.这里要着重说清楚束缚电子的填
补运动跟自由电子的移动是不同的.在外电场作用下,束缚
电子逆着电场方向的填补运动,从效果上看好象空穴顺着电
场方向移动,这种空穴顺着电场方向定向移动形成的电流,就
是半导体的空穴导电.

半导体的热敏特性和光敏特性可以通过演示来说明,这
里介绍半导体的热敏特性和光敏特性是为了说明半导体有着
它特殊的导电性质,并不需要对热敏特性和光敏特性产生的
机理作出解释.

在理解了半导体的电子导电和空穴导电机理的基础上,
进一步认识N型半导体和P型半导体还是比较容易的.可
以只讲述其中的N型半导体,P型半导体让学生自己阅读课
本,以培养学生的阅读能力.

\subsubsection{晶体二极管和PN结} 

晶体二极管的单向导电性是
与电阻比较而言的,做教材图8.26演示时,可把二极
管换成电阻元件再作一次.讲二极管、三极管时,可提供一
些实物让学生观察,并让学生画一画、记一记二极管、三极管
的符号.

需要向学生说明的是,PN结不是由P型半导体材料和
N型半导体材料机械拼凑而成的,要用一定的工艺对P型材
料和N型材料加工,如教材图8.25右图所示的
面接触型二极管,它的PN结是用合金烧结法或扩散法工艺
加工制成的,晶体三极管也不能由两只二极管接拢凑成.要
按不同类型管子的要求加工做成.

形成PN结的微观过程比较复杂,为了帮助学生理清线
索,可归纳小结如下:
\begin{enumerate}
\item 扩散运动——复合——阻挡层开始形成.
\item 继续扩散——阻挡层变厚——阻碍扩散,扩散减弱.
\item 扩散减弱——稳定的阻挡层即PN结生成.
\end{enumerate}

\subsubsection{晶体三极管}

介绍晶体三极管的结构时,主要是让学
生知道晶体三极管并不等于两个二极管的简单组合.这里提
到的三个区、三个电极和两个PN结的名称,学生很难一下子
都记住,可以只对照着三极管的符号图记忆三个电极的名称.
这样有利于讲述三极管的放大作用;并且实际应用中,更多的
情况是要知道三极管的三个电极.

三极管的放大作用最好利用教材图8.30所示的电路实
际测量的数据纪录来说明.根据数据,先结合教材图8.31说
明三极管的电流分配关系,即$I_e=I_b+I_c$和$I_b\ll I_c$; 再说明三
极管的放大作用.

\section{实验指导}
\subsection{演示实验}
\subsubsection{液体导电中离子的运动}
实验装置如图8.6所示,将蛋壳洗净晾干后仔细地在开口
处用三根棉线系牢,以便固定在铁架上,演示前将氢氧化钠溶
液置入蛋壳内,并插入一根炭棒,炭棒用导线与电源负极相
连,再取一个烧杯,内盛硫酸钠溶液,同时滴入几滴酚酞作指
示剂,并插一根洗净的炭棒作电极,用导线与电源正极相连,
最后将蛋壳小心浸在硫酸钠溶液中.
\begin{figure}[htp]\centering
    \begin{minipage}[t]{0.48\textwidth}
    \centering
    \includegraphics[scale=.6]{fig/8-6.png}
    \caption{}
    \end{minipage}
    \begin{minipage}[t]{0.48\textwidth}
    \centering
    \includegraphics[scale=.6]{fig/8-7.png}
    \caption{}
    \end{minipage}
    \end{figure}

演示分两步进行:
\begin{enumerate}
    \item 断开电键$K$, 可观察到烧杯内无变
色的现象,说明蛋壳内的氢氧根离子并未进入烧杯.
\item 接通
电键$K$, 立即会发现蛋壳外靠近电极这面的液体逐渐变红,而
且红色区域逐渐向正极扩展,说明有氢氧根离子进入溶液,而
氢氧根是带负电的,从而说明溶液在通电时有带电离子定向
运动.
\end{enumerate}

\subsubsection{气体的导电性}
教材图8.4所示的实验,跟一般的静电实验一样,
关键在于两个静电计的金属杆与外壳之间、静电计与空气和
桌面之间有良好的绝缘,这个实验也可以用如图8.7所示的
实验代替.$A$、$B$是两块绝缘金属板,高压电源选用感应圈,
与演示用灵敏电流计串联组成回路.$A$、$B$间的间距选择到
这样的程度,在高压电源接通后灵敏电流计中恰好无电流.

演示时,将酒精灯火焰放在$A$、$B$两板之间,给空气加热,
可观察到电流计的指针偏转,说明空气已成为导体.

\subsubsection{稀薄气体的辉光放电}
实验装置采用如课本图8.9所示的低气压放电管.

低气压放电管的两个电极,一个呈圆片形,为阴极,使用
时接电源的负极,一个呈圆棒形,为阳极,使用时接电源的正
极,高压电源选用感应圈.抽气机应通过一个三通与放电管
气压计相连,连接的胶管应选用硬胶管(软胶管在抽气中会
被大气压扁而不能继续抽气).

实验时,先给放电管通电,然后抽气,随着气压的变化,放
电现象也随之变化.下面列出空气在不同气压下的放电
现象:
\begin{itemize}
\item 760—50mmHg: 不产生放电现象;
\item 40mmHg: 出现紫色线形光条纹;
\item 10mmHg: 紫色光条变宽,几乎充满全管.阴极周围出现
彩色辉光;
\item 3mmHg: 阴极周围紫色辉光,阳极发生红色辉光充满全
管,并开始出现克鲁克斯暗区;
\item 1mmHg: 阴极区兰色辉光鲜明,阳极出现鳞片状辉光,并
开始出现法拉第暗区;
\item 0.1mmHg: 法拉第暗区加长,由阳极发出的鳞片状辉光
减少,管内出现灰白色棉状辉光,在阴极附近开始出现荧光;
\item 0.02mmHg: 阴极发出射线,管壁出现亮的荧光.
\end{itemize}

实验时使用施片式真空泵,其极限真空度为$5\x10^{-4}$mmHg, 可以观察到全部辉光现象.若使用手摇抽气机,其极
限真空度为0.4mmHg, 则不能观察到全部辉光现象.

为了使学生能更清晰地观察低气压放电现象,在做了上
述实验,表明辉光现象随气压的变化而变化后,最好再用低气
压放电管组进行演示,低气压管组内的空气气压,正好选用了
上面六个数值,即40mmHg、10mmHg、3mmHg、1mmHg、
0.1mmHg和0.02mmHg. 观察到的放电现象清晰而稳定.

实验要在暗室中进行.

\subsection{学生实验}
\subsubsection{测定铜的电化当量}
低压直流电源(输出电流3A),安培表(3A),计时表,电
键,硫酸铜溶液,电极(铜片或碳棒),天平(学生天平,感量20
mg).

电解质溶液的配制:
要获得良好的实验效果,电解质溶液的配制十分重要,在
硫酸铜溶液中加入适量的浓硫酸,增强溶液的导电性;最好还
加入少量的葡萄糖作为添加剂,改善阴极板上铜的沉积质量
(用60克硫酸铜和300毫升水配制的硫酸铜溶液,可加入17
毫升浓硫酸和10克葡萄糖).

极板特别是阴极板不能有毛刺和污迹,否则,阴极板
各处的电流密度不一样,电流密度大的地方铜沉积得快,结
果会出现结瘤或须状物.因此事先要用细砂纸将阴极板擦
干净.

为了减少误差,要增大电流强度和通电时间.在电源
功率允许的条件下,可适当增大电流强度,但电流不宜过分
大,否则阴极板析出的铜太疏松容易脱落.该实验以做两次
为宜.因此,需准备两个阴极板,在第一次实验通电结束后,
立即接上另一个阴极板做第二次实验,使称量第一次的阴极
板铜片质量与第二次通电电解同时进行,这样可大大节约
时间.

\subsubsection{练习使用示波器}

该实验使用J2459示波器旋钮开关共16个,学生是
第一次接触这样复杂的仪器,因此,在实验前,可以指导学生
将16个开关和旋钮分为三组(图8.8),逐个讲解和演示它们
的作用.

\begin{figure}[htp]
    \centering
\includegraphics[scale=.6]{fig/8-8.png}
    \caption{}
\end{figure}

第一组亮斑调节:包括辉
度调节旋钮、聚焦调节旋钮、辅
助聚焦调节旋钮及电源开关.

第二组竖直方向调节:包
括垂直位移旋钮、Y增益旋钮、
衰减旋钮、“Y输入”和“地”旋
钮以及“DC.AC”选择开关.

第三组水平方向调节:包
括水平位移旋钮、X增益旋钮、
扫描范围旋钮、扫描微调旋钮、“X输入”旋钮以及“同步”选择
开关.

这三组旋钮在面板上的位置大致如图8.8所示.

可向学生指出,这个实验有六个方面的练习内容.
\begin{enumerate}
    \item 开机练习; \item 寻找光点、调节光点的亮度与大小的练
    习; \item 寻找扫描线、调节水平幅度大小的练习; \item 在竖直方向
    加一直流电压观察光点向上、向下偏移的练习; \item 测量一节干电池电压的练习;\item 关机的练习.
\end{enumerate}


这样做,可以使学生较快地熟悉示波器,逐步做到调节有
序,测量有方.

\section{习题解答}

\subsection{练习一}
\begin{enumerate}
    \item 在金属导体中,自由电子的热透动速率和定向移动逮率之间有什么区别?这两种速率哪个大?

    \begin{solution}
        金属导体中自由电子的热运动不需要外加条件,是
        不停地永远进行着的,热运动的速率大小与金属的温度有关.
        由于热运动的不规则性.这种运动不能形成电流;金属中自
        由电子的定向运动.必须有电场存在才能发生.运动方向总
        是逆电场方向的,这种运动形成金属中的电流.自由电子的
        平均定向运动速率比热运动速率小得多.
    \end{solution}
    
    \item 电路接通后,为什么整个电路中几乎同时形成电流?

    \begin{solution}
        电路接通后,电路里便以光速在各处极迅速地建立
        起电场,整个电路中的自由电子几乎同时受到电场力的作用
        作定向运动,所以整个电路中同时形成电流.
    \end{solution}
    
    \item 利用公式$I=\dfrac{e^2 nS\tau}{2m\ell}U$,求出这段导体的电阻$R$和制
成这段导体的材料的电阻率$\rho$.

\begin{solution}
    将公式与欧姆定律$I=U/R$
    相比较,则
    \[R=\frac{U}{I}=\frac{2m\ell}{e^2 nS\tau}\]
又$R=\rho\ell/S$,比较上面两个式子,
\[\rho=\frac{2m}{e^2 n\tau}\]
\end{solution}

\end{enumerate}

\subsection{练习二}
\begin{enumerate}
    \item 通过硫酸铜溶液的电量是$2.0\x10^4$库,在阴极上能析出多少克铜?

    \begin{solution}
根据法拉第电解第一定律$m=kq$, 查表得铜的电化
当量$k$为$0.3294\x10^{-6}{\rm kg/C}$,代入上式得
\[m=0.3294\x10^{-6}\x2.0\x10^4{\rm kg}=6.6{\rm g}\]
    \end{solution}
    
    \item 如果要在表面积是5${\rm cm^2}$的器件上,镀上一层20微米厚的银层,需要通过多少库仑的电量?

    \begin{solution}
    根据法拉第电解第一定律$m=kq$. 得
    \[q=\frac{m}{k}=\frac{\rho V}{k}\]
查表得银的密度$\rho=10.49{\rm g/cm^3}=10.49\x10^3{\rm kg/m^3}$; 
银的$k$值为$1.118\x10^{-6}{\rm kg/C}$,代入上式得
\[q=\frac{\rho Sh}{k}=\frac{10.49\x10^3\x 5\x 10^{-4}\x 20\x 10^{-6}}{1.118\x10^{-6}}=94{\rm C}\]
    \end{solution}
    
    \item 有一个学生电解硫酸铜溶液来测定铜的电化当量.他在通电以前称一次阴极板,通电25分钟以后再称一次,从而知道析出的铜的质量是0.29克,还知道通过溶液的电流强度是0.6安,从这些数据算出铜的电化当量是多少?

    \begin{solution}
        根据法拉第电解第一定律$m=kIt$,得
        \[k=\frac{m}{It}=\frac{0.29\x 10^{-3}}{0.6\x 25\x 60}=3\x 10^{-7}{\rm kg/C}\]
    \end{solution}
    
    \item 锌的摩尔质量是0.06538kg/mol,它的化合价是2,求锌的电化当量.

    \begin{solution}
        根据法拉第电解第二定律
\[k=\frac{M}{Fn}=\frac{0.06538}{2\x9.65\x10^4}=0.339\x10^{-6}{\rm kg/mol}
\]
    \end{solution}
    
    \item 金的摩尔质量是0.1972kg/mol,它的化合价是3,要想使金电解池的阴极上析出1克金,需要通过多少库的电量?

    \begin{solution}
        根据法拉第电解第一、第二定律有
        \[m=kq=\frac{Mq}{Fn}\]
        则:
\[q=\frac{mnF}{M}=\frac{10^{-3}\x 3\x 9.65\x 10^4}{0.1972}=1.47\x 10^3{\rm C}\]
    \end{solution}
    
\end{enumerate}

\subsection{练习三}
\begin{enumerate}
    \item 放电管里气体的自激放电是怎样形成的?

    \begin{solution}
        放电管里的气体中含有少量的自由电子和正离子,
        在高电压强电场作用下分别奔向阳极和阴极的过程中动能增
        加,如果电压足够高,电场足够强,电子的动能增大到一定程
        度时,电子跟中性原子碰撞.会从原子中打出电子,发生气体
        电离,而动能足够大的正离子轰击阴极表面时,能使阴极发
        射电子.这些电子又会使气体发生电子碰撞电离,产生新的
        电子和正离子,于是形成自激放电,也就是说形成自激放电
        的条件是气体电离和阴极发射电子.
    \end{solution}
    
    \item 为什么安装了避雷针能够避免雷击?

    \begin{solution}
        避雷针是尖端状导体,位置高过建筑物.当带电云
        层与建筑物接近时,放电通过避雷针和接地导体这条通路不
        断进行,避免了用电荷积累,使云层和大地之间产生高压放
        电现象.所以不再发生雷击.
    \end{solution}
    
    \item 示波管荧光屏上显示出的一条完整的正弦曲线(图8.17),是在什么条件下得到的?这时,如果使扫描电压的周期增大为原来的二倍,那么,荧光屏上将显示出什么样的图线?

    \begin{solution}
        扫描电压的周期与竖直极板上电压周期相等,并且
        起始时间也相同,在荧光屏上就会显示出一条完整的正弦曲
        线.如果扫描电压的周期增大为原来的二倍,荧光屏上将显
        示出一条稳定的有两个完整周期的正弦曲线.
    \end{solution}
    
\end{enumerate}


\section{参考资料}
\subsection{关于金属电子论对电阻的解释}
金属具有优良的导电性和导热性,这一点早就为人们所
知,但要从理论上说明金属导电、导热的规律及二者间的联
系,却象力学一样经历了经典理论到量子理论的发展过程,正
是因为把电子作为纯经典粒子处理,把适用于宏观物体运动
的牛顿力学以及适用于理想气体分子的玻尔兹曼和麦克斯韦
统计分布律用于自由电子,才建立了经典电子论,特鲁德和
洛仑兹为这个理论的建立作出了重要的贡献.

1887年汤姆生发现电子,1900年特鲁德提出自由电子理
论.特鲁德假设金属中的电子与理想气体分子一样,称做电
子气.电子气能在金属中自由运动,和金属中的正离子碰撞
在一定温度下达到热平衡状态.

1904年洛仑兹进一步提出电子气服从麦克斯韦-玻
尔兹曼统计分布规律,能解释欧姆定律、热传导规律等.但
在解释金属的热容量和电阻与温度的关系方面遇到了困难.
这里只谈一谈金属的电阻问题.

根据洛仑兹的经典电子理论来解释欧姆定律,已经得到
\[I=\frac{n e^{2} S \tau}{2 m \ell} U , \qquad R=\frac{2 m \ell}{e^{2} n S \tau} ,\qquad \rho=\frac{2 m}{e^{2} n \tau}\]
式中$\tau$是自由电子相继,两次碰撞的平均时间.因为自由电子的平均定向移动速率远
远小于它的热运动平均速率$\bar v$,所以平均定向移动速率可以
略去不计.这样用电子的平均自由程$\bar\lambda$除以电子热运动的平
均速率$\bar v$,就得到自由电子经过一个平均自由程的平均时间
$\tau=\bar\lambda/\bar v$,代入上式,有:
\[\rho=\frac{2m\bar v}{e^2n\bar\lambda}\]
由于$\bar v=\sqrt{\dfrac{8kT}{\pi m_A}}$,所以导体
的电阻率$\rho$与热力学温度$T$的平方根成正比.而实验结果却
是电阻跟热力学温度$T$成正比,理论与实验不符,这暴露了经
典理论的缺陷.

1928年以后,在量子力学的基础上提出了自由电
子量子理论,才解决了上述矛盾.由于电子是自旋为
半整数的费米子,它不服从玻尔兹曼分布,而服从费
米-狄拉克统计规律.应用量子理论讨论的结果,由
经典电子论得到的电阻率$\rho$的表达式中的$\bar v$, 要用费米速度$v_F$\footnote{根据经典理论,$T=0$时,所有电子的动能都等于零.在量子
理论中,即使在$T=0$(基态)时,电子也不可能具有零动能.不相容原
理,又不允许两个(具有相反自旋)以上的电子处于同一能级.因此,电
子将由低到高依次填满各个相应的能级,我们把电子气处于基态时的
最高电子能量,称为电子气的费米能$E_F$. 与费米能对应的电子速度
\[v_F=\sqrt{\frac{2}{m}E_F}\]
称为费米速度. }
(它只与金属自由电子的数密度有关)来代替;而电子的
平均自由程则由金属中电子的散射过程所决定,其数值与金
属中原子由于热运动而离开平衡位置的位移的均方成反比,
因而与绝对温度$T$成反比.由此可得出金属电阻率$\rho\propto T$的
结论.

\subsection{电解液中的电流}
电解液中的正负离子在没有外加电场时,做无规则的热
运动.在有外加电场时,正负离子在电场力的作用下,分别向
两个电极作定向移动.这种电解液中正负离子的定向移动即
是电解液中的电流.

各种离子在电场作用下迁移的速度一般是不同的,离子
所带的电荷越多,受外加电场的作用力越大,带电量相同的正
负离子,虽然它们在电场中受到的电场力大小相等,由于质量
不同,迁移速度$v_+$与$v_{-}$也不相同.因而,在相同的时间里通
过电解液某一截面的正负离子数也不相等,通过这截面的正
负电荷的电量$Q_+$与$Q_-$也不相等.

假如单位体积内有$n_1$个正离子,这些正离子以迁移速度
$v_+$通过截面$S$, 每个离子所带电量为$q_+$, 则在$t$时间内每个
离子前进的距离为$v_+t$, 在$t$时间内通过这个截面的正电荷
的电量为$Q_+=n_1q_+v_+tS$.

同理,如果单位体积内有$n_2$个负离子,它们以迁移速度
$v_-$跟正离子的方向相反通过同一截面$S$, 每个离子所带电量
为$q_-$, 则在$t$时间内每个离子前进的距离为$v_-t$, 在$t$时间内
通过这个截面的负电荷的电量为$Q_-=n_2q_-v_-tS$. 

正负离子的运动都形成电流,正离子的定向移动相当于
负离子的反方向运动,两种电荷的运动具有同样的效果.因
此,在$t$时间内通过截面$S$的总电量$Q=Q_+ +Q_-$. 通过这个
截面的电流强度$I=\frac{Q}{t}=\frac{Q_+ +Q_-}{t}$.

\subsection{正离子碰撞电离}
气体放电中的正离子在电场作用下跟中性原子碰撞时,
很少使原子发生电离.这个实验事实可作如下的解释:

我们把气体中两个粒子的碰撞看作是在一个孤立系统中
发生的,这个系统的运动是由它们的质心的运动和每一个粒
子相对质心的运动所组成的.中性原子在碰撞中发生电离就
是动能转化为系统内部的势能而实现的;并且在这种转化过
程中,质心运动的动能保持不变.

为了使问题简化,只讨论正碰的情况、假设碰撞粒子的
质量为$m_1$, 碰撞前的速度为$v_1$; 被碰撞粒子的质量为$m_2$, 碰
撞前是静止的.取被碰粒子中心为坐标原点,碰撞前碰撞粒
子的坐标为$r$, 系统质心的坐标为$x$,则根据质心的定义有
$x(m_1+m_2)=rm_1$,即
\[x=r\frac{m_1}{m_1+m_2}\]
