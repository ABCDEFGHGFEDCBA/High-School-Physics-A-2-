\documentclass[b5paper, openany]{ctexbook}

%\usepackage{anysize}
%\papersize{26cm}{18.5cm}
%\marginsize{2.25cm}{2.25cm}{2cm}{2cm}
\usepackage[margin=2.5cm]{geometry}


% 修改脚注的编号为加圈样式,并且各页单独编号

\usepackage{pifont}
\usepackage[perpage,symbol*]{footmisc}
\DefineFNsymbols{circled}{{\ding{192}}{\ding{193}}{\ding{194}}
{\ding{195}}{\ding{196}}{\ding{197}}{\ding{198}}{\ding{199}}{\ding{200}}{\ding{201}}}
\setfnsymbol{circled}



\usepackage{amsmath,amsfonts,mathrsfs,amssymb}
\usepackage{graphicx}

\usepackage[font=bf,labelfont=bf,labelsep=quad]{caption}
\usepackage{circuitikz}
\usepackage{tikz}



\usepackage{physics}
%\usepackage{amsthm}
\usepackage{ntheorem}
\theoremseparator{\;}



\usepackage{blkarray}
\usepackage{bm}
\usepackage[colorlinks=true, linkcolor=black]{hyperref}

%\usepackage{enumerate}


\theoremstyle{plain}
\theoremheaderfont{\normalfont\bfseries} 
\theorembodyfont{\normalfont}


\newtheorem{example}{\bf 例题}[chapter]
\newenvironment{solution}{\noindent {\bf 解答:}}{}%{\hfill $\blacksquare$\par}


\renewcommand{\proofname}{\bf 证明:}
\newenvironment{proof}{{\noindent \bf 证明:}}{}%{\hfill $\square$\par}

\newcommand{\E}{\mathbb{E}}
\renewcommand{\Pr}{\mathbb{P}}
\newcommand{\EP}{\mathbb{E}^{\mathbb{P}}}
\newcommand{\EQ}{\mathbb{E}^{\mathbb{Q}}}
\newcommand{\dif}{\,{\rm d}}
\newcommand{\Var}{{\rm Var}}
\newcommand{\Cov}{{\rm Cov}}



 \usepackage{tcolorbox}
 \tcbuselibrary{breakable}
 \tcbuselibrary{most}


\setcounter{tocdepth}{1}

\setcounter{secnumdepth}{3}



\ctexset {
section = {
	name = {第,节},
	number = \chinese{section}},
subsection = {
	name = {\hspace{2em},、\hspace{-1em}},
	number = \chinese{subsection}
},
subsubsection = {
	name = {\hspace{2em}(,)\hspace{-1em}},
	number = \chinese{subsubsection},
}
}


\renewcommand{\contentsname}{目~~录}

\usepackage{paralist}
\let\itemize\compactitem
%\let\enditemize\endcompactitem
\let\enumerate\compactenum
%\let\endenumerate\endcompactenum
\let\description\compactdesc
%\let\enddescription\endcompactdesc


\usepackage{titlesec}
\titlespacing{\chapter}{0pt}{*1}{*1}
\titlespacing{\section}{0pt}{*1}{*1}
\titlespacing{\subsection}{0pt}{*1}{*1}

\titlespacing{\subsubsection}{0pt}{*1}{*1}

\renewcommand{\le}{\leqslant}
\renewcommand{\ge}{\geqslant}
\usepackage{mathtools}

\setlength{\abovecaptionskip}{0.cm}
\setlength{\belowcaptionskip}{-0.cm}

%\setmainfont{Times New Roman}
\usetikzlibrary{decorations.pathmorphing, patterns}
\usetikzlibrary{calc, patterns, decorations.markings}
\usetikzlibrary{positioning, snakes}

%\cover{cover.pdf}

\usepackage{yhmath}
\newcommand{\ms}{\text{m}/\text{s}}
\newcommand{\cms}{\text{cm}/\text{s}}
\newcommand{\msq}{\text{m}/\text{s}^2}
\newcommand{\cmsq}{\text{cm}/\text{s}^2}
\newcommand{\kmh}{\text{km}/\text{h}}
\newcommand{\NC}{\text{N}/\text{C}}









\newcommand{\x}{\times}

\begin{document}
%\fontsize{10.5}{11}\selectfont







\title{\Huge\bfseries 高中物理(甲种本)\\
第二册\vspace*{2cm}}



\author{\Large 张同恂\and \Large 方玉珍\and \Large 马淑美\and \Large 郭连壁\and \Large 李福利 }
\date{\Large 1984年7月 \\~\\~\\	\includegraphics[scale=1.2]{fig/front.pdf}}

\maketitle

\tableofcontents


\frontmatter

\input{preface.tex}

\mainmatter

\chapter{分子运动论基础}
从这一章开始我们学习热学知识,热学是物理学的一部
分,它研究热现象的规律.热现象跟力学现象不同,描述热现
象的一个基本概念是温度,温度发生变化的时候,物体的许
多性质都发生变化.例如物体的温度升高,它的体积要膨胀.
在1标准大气压下,水在0$^\circ$C以下是固体(冰),在0$^\circ$C以上才
是液体.一段橡皮管冷却到-100$^\circ$C以下会变得象玻璃一样
地易碎,轻轻打一下就碎裂成许多小块.凡是跟温度有关的
现象都叫做\textbf{热现象}.

热学知识在实际中有重要的应用.各种热机和制冷设备
的研制,化工、冶金、气象的研究,都离不开热学知识.

研究热现象有两种不同的方法.一种是从能量的观点来
研究,确认热是能的一种形式,叫做热能,并把热能跟其他形
式的能联系起来,建立了能的转化和守恒定律.另一种是从
物质微观结构的观点来研究,建立了分子运动论,说明热现象
是大量分子无规则运动的表现,这两种方法相辅相成,使人
们对热现象的研究越来越深入.

这一章讲述分子运动论,下一章讲述热能以及能的转化
和守恒定律,然后以此为基础分别研究气体、液体和固体的
性质,气体比较简单,研究得比较透彻,我们的学习将以气体
作为重点.

\section{分子运动论的建立}
    早在古希腊的时候,就有人提出物质的微粒结构的思想.
两千多年以前,古希腊的著名思想家德谟克利特说过,万物都
是由极小的微粒构成的,并把这种微粒叫做原子.这种古代
的原子学说虽然没有实验根据,却包含着原子理论的萌芽.

    在十七世纪到十八世纪期间,随着热学的发展,人们开始
探讨热现象的本质,出现了分子运动论的学说,伽森第提出
物质是由分子构成的,设想分子是一种硬的粒子,能向各方向
运动,并用来解释固液气三种物质状态.胡克和伯努利发展
了这个学说.罗蒙诺索夫继续发展了这个学说,明确提出了
热是分子无规则运动的表现.但是,这个学说当时还不能定
量地解释热现象.更重要的是,认为热是一种运动的表现,当
时得不到公认,因而这个学说未能得到发展.另一种学说,即
认为热是一种特殊物质的热质说,占据着统治地位.

   十九世纪中叶,建立了能的转化和守恒定律,确认热是能
的一种形式,而不是一种特殊物质,能的转化和守恒定律的
建立否定了热质说,为分子运动论的发展开辟了道路.此后,
定量而系统的分子运动论飞速发展起来,在差不多半个世纪
的时间里就建立起完善的分子运动论.克劳修斯认为气体对
器壁的压强是由大量气体分子碰撞器壁而产生的,他由此算
出了气体的压强,解释了有关气体的实验定律.麦克斯韦认识
到气体分子的速率各不相同,而分子的速率是按着一定规律
分布的.玻耳兹曼进一步研究分子运动论,使分子运动论达
到了完善的程度.

分子运动论的基本内容是:物体是由大量分子\footnote{构成物质的单位是多种多样的,或是原子(如金属)或是离子(如盐类)或是分子(如有机物).为了简化,这里把构成物质的单位统称为分子.}组成的,
分子永不停息地做无规则运动,分子之间存在着相互作用的
引力和斥力.按照分子运动论,热现象是大量分子无规则运
动的表现,温度表示分子无规则运动的激烈程度,热能是大量
做无规则运动的分子具有的能.用分子运动论可以说明很多
热现象和物质的性质.首先详细地研究了气体,建立了气体
分子运动论,说明了气体的宏观性质.随后又用分子运动论
研究了液体和固体,也获得很大成果.

    分子和分子的运动虽然看不见,但分子运动论也跟其他
物理理论一样,是建立在一定的实验基础之上的.下面我们
介绍分子运动论的基本内容,要着重说明它的实验基础.

\section{物体是由分子组成的}
    物体是由分子组成的,这在化学中已经学过了.这一节
讲讲分子的大小和阿伏伽德罗常数.

    分子的大小分子看不见,摸不到,怎样能知道分子的大
小呢?

    一种粗略地测定分子大小的方法是油膜法,把油滴滴到
水面上,油在水面上要尽可能地散开,形成单分子油膜(图1.1).如果把分子看成球形,单分子油膜的厚度就可以认为等
于油分子的直径.事先测出油
滴的体积,再测出油滴在水面
上散开的面积,就可以算出单
分子油膜的厚度,这样就测出
了分子的直径.测定结果表明,
分子直径的数量级是$10^{-10}$米.

\begin{figure}[htp]
\centering
\begin{tikzpicture}[>=stealth]
\draw (0,0)--(6,0);

\foreach \x in {-.1, -.25, -.5, -1, -1.5, -2}
{ 
   \draw [densely dashed](0,\x)--(6,\x);
}

\foreach \y in {1,2,...,10}
{
   \draw (\y/2+.5,.25) circle [radius=.25];
}

\draw [thick] (5.5,0)--(7.5,0);
\draw [thick] (5.5,.5)--(7.5,.5);
\draw [->,thick](6.5,1.5)--(6.5,.5);
\draw [->,thick](6.5,-1)--(6.5,0);

\end{tikzpicture}

\caption{水面上的单分子油膜的示意图}
\end{figure}

    现在有了能放大上百万倍的离子显微镜,用它可以看到
钨针针尖上原子分布的图样,并且可以测出
钨原子间的距离大约是$2\times 10^{-10}$米,设想钨原子是一个挨着
一个排列的,那么,可以认为钨原子间的距离$2\times 10^{-10}$米就是
钨原子的直径.

    物理学中有各种不同的方法来测定分子的大小.用不同
方法测出的分子的大小并不完全相同,但是数量级是相符的.
测定的结果表明,一般分子直径的数量级都是$10^{-10}$米.例
如水分子的直径是$4.0\times 10^{-10}$米,氢分子的直径是$2.3\times 10^{-10}$
米.

    需要指出的是:把分子看作小球,是分子运动论中对分子
的简化模型;实际上,分子有它复杂的内部结构,并不真是小
球.因此,说分子的直径有多大,一般知道数量级已经可以
了,它提供了关于分子大小的一个数量观念,使我们了解分子
是多么微小.

    \subsection{阿伏伽德罗常数}

我们在化学课中学过,1摩尔\footnote{摩尔简称摩,国际符号是mol}
的任
何物质,其中含有的粒子数相同,都等于12克碳-12中含有
的原子数,这个数叫做\textbf{阿伏伽德罗常数}.

    知道分子的大小,可以粗略地算出阿伏伽德罗常数.例
如1摩的水,质量是$1.8\times 10^{-2}$千克,体积是$1.8\times 10^{-5}{\rm m}^3$.水
分子的直径是$4.0\times 10^{-10}{\rm m}$,体积大约是$3\times 10^{-29}{\rm m}^3$.设想
水分子是一个挨一个排列的,我们可以算出1摩的水中所含
的分子数:
\[N=\frac{1.8\times 10^{-5}{\rm m^3/mol}}{3\times 10^{-20}{\rm m^3}}=6\times 10^{23}{\rm mol}^{-1} \]

    早期测定阿伏伽德罗常数的一种方法,就是利用油膜法
测出分子直径,得出这个常数的.这种测定方法比较粗略,但
得出的数量级是正确的.

    我们看到,阿伏伽德罗常数是一个十分巨大的数字.为
了说明这个数字有多么大,我们设想有一个极小的动物来喝
水,它每秒钟喝进100亿个分子,要二百万年才能把1摩的水
喝完.

    反过来,知道了阿伏伽德罗常数,对液体和固体很容易估
算分子的大小.知道液体和固体的摩尔体积,设想其中的分
子是一个挨一个排列的,利用阿伏伽德罗常数就可以算出一
个分子所占的体积,从而估算出它的直径.

    知道了阿伏伽德罗常数,还可以算出分子的质量.水的
摩尔质量是$1.8\times 10^{-2}{\rm kg/mol}$,1摩的水中含有$6\times 10^{23}$个
分子,所以一个水分子的质量是
\[m_{\rm H_2 O}=\frac{1.8\times 10^{-2}{\rm kg/mol}}{6\times 10^{23}{\rm mol}^{-1}}=3\times 10^{-26}{\rm kg} \]
可见水分子的质量是很小的.除了包含几千个原子的有机物
大分子而外,一般分子的质量也是这个数量级.

    反过来,知道分子的质量,也可以算出阿伏伽德罗常数,
物理中有办法测出分子的质量,例如精确测得一个碳原子的
质量是$1.995\times 10^{-36}$千克,由此不难得出阿伏伽德罗常数.

    阿伏伽德罗常数是微观世界的一个重要常数,用分子运
动论定量地研究热现象经常要用到它,它是联系微观世界和
宏观世界的桥梁.从上面所讲的我们可以看出,阿伏伽德罗
常数把摩尔质量或摩尔体积这种宏观物理量跟分子质量或分
子大小这种微观物理量联系起来了.

    正因为阿伏伽德罗常数这样重要,所以物理学家们想出
各种办法来测定它,一百多年以来不断努力来更精确地测定
它,后面我们讲到电学的时候,就要提到一种测定阿伏伽德
罗常数的方法.现在测得的阿伏伽德罗常数的精确值是
\[N=6.022045\times 10^{23}{\rm mol^{-1}} \]
通常可取作
\[N=6.02\times 10^{23}{\rm mol^{-1}} \]

\section*{阅读材料:离子显微镜}
    课文中提到,用离子显微镜可以测出钨原子的直径.现
在简单介绍一下离子显微镜的构造和原理.

    离子显微镜由半径约为10厘米的球形玻璃容器和一根
钨针组成,钨针的针尖放在容器的中心(图1.2).针尖的表面
可以看作是半径非常小的球面,近代金属加工技术可以做到
使这个半径约为$5\times 10^{-6}$厘米.在球形容器的内表面涂上一
薄层导电物质,象电视荧光屏那样,在快速粒子打击下可以发
光.在导电层和针尖之何加上高电压,使导电层带负电,针尖
带正电.

\begin{figure}
\centering
\includegraphics[scale=.8]{fig/1-2.png}
\caption{离子显微镜的构造原理
}
\end{figure}


    在球形容器中充满低压的氦气.当无规则运动的氦原子
与针尖上的钨原子碰撞时,由于氦原子失去电子成为正离子,
氦离子在电力作用下就离开针尖,以很大速度沿着球半径运
动,打到球形容器的内表面上使之发光.这样,就出现了钨针
针尖上原子分布的图样(图1.3).

\begin{figure}
\centering
    \begin{tikzpicture}[>=latex, scale=.8]
\draw[dashed] (0,2) arc (90:150:2);
\draw[dashed] (0,2) arc (90:50:2);
\draw (0,6) arc (90:140:6);
\draw (0,6) arc (90:70:6);
\draw (0,0) [fill=black] circle (2.5pt)node[below]{$O$};
\draw (90:2) [fill=black] circle (2.5pt)node[right]{$b$};
\draw (90:6) [fill=black]circle (2.5pt)node[above]{$B$};
\draw (110:2) [fill=black]circle (2.5pt)node[left]{$a$};
\draw (110:6) [fill=black]circle (2.5pt)node[above]{$A$};

\draw[->] (0,0)--node[below]{$R$}(130:6);
\draw[very thick]  (0,0)--(90:6);
\draw[very thick]    (0,0)--(110:6);
\draw[very thick,->] (0,0)--(90:3.8);
\draw[very thick, ->] (0,0)--(110:3.8);
\draw [<->] (0,3) arc (90:110:3)node[right]{$\alpha$};
\draw[->] (0,0)--node[above]{$r$}(60:2);

\draw (1,4)--(90:4);
\draw (1,4)node[right]{离子的轨迹}--(110:3.3);

\draw (90:2)--(1,3)node[right]{钨原子};
\draw (110:2)--(1,3);

    \end{tikzpicture}
\caption{计算离子显微镜的
放大倍数,图中的
$a$和$b$表示钨针针
尖上的两个钨原子;
$A$和$B$分别表示它
们在球形容器内表
面上的像.$R$是球
形容器的半径,$r$表
示针尖的半径.
}
\end{figure}

    图1.3中弧长$ab$表示相邻两个钨原子间的距离,弧长$AB$
表示它们在球形容器内表面上的像之间的距离.因为$AB=R\alpha$, $ab=r\alpha$,
所以放大倍数
\[K=\frac{AB}{ab}=\frac{R}{r}=\frac{10{\rm cm}}{5\times 10^{-6}{\rm cm}}=2\times 10^5 \]
即放大二百万倍.已知放大倍数,测出弧长$AB$,就可以
求出原子间的距离$ab$.设想钨原子是一个挨一个排列的,可
以认为距离$ab$等于钨原子的直径.测定结果表明,钨原子的
直径是$2\times 10^{-10}$米.


\subsection*{练习一}
\begin{enumerate}
\item  一般分子的直径,以厘米作单位时数量级是多大?
\item  把体积为1$\rm mm^3$的石油滴在水面上,石油在水面上
形成面积为3$\rm m^2$的单分子油膜.试估算石油分子的直径.
\item  设想把分子一个挨一个地排起来,要多少个分子才
能排满1米的长度?
\item  1$\rm cm^3$水中含有多少个水分子?10克氧中含有多少
个氧分子?
\item  一个氧分子、一个氮分子的质量各是多少千克?
\item  已经测得一个碳原子的质量是$1.995\times 10^{-26}$千克,
求阿伏伽德罗常数.
\item  已知金刚石的密度是$3500{\rm kg/m^3}$,有一小块金刚
石,体积是$5.7\times 10^{-8}{\rm m^3}$.这小块金刚石中含有多少个碳原
子?设想金刚石中碳原子是紧密地堆在一起的,估算碳原子的
直径.

\end{enumerate}







\section{布朗运动}
物体里的分子永不停息地做无规则运动,这个结论也是在实验事实的基础上得到的.我们在初中学过的扩散现象表明分子在不停地运动.现在我们再讲一种现象,它可以更明
显地证实分子的无规则运动,这种现象叫做布朗运动.

1827年英国植物学家布朗用显微镜观察悬浮在水中的花粉,发现花粉颗粒不停地做无规则运动,后来把颗粒的这种无规则运动叫做布朗运动.不只是花粉,悬浮在液体中的微粒,都做布朗运动,把少量墨汁用水稀释,取一滴这样的液体放在显微镜下来观察(图1.4),就可以看到碳粒做无规则
的布朗运动,图1.5是做布朗运动的三个微粒的运动路线,从图中可以看出,布朗运动是毫无规则的.这个图只画出了
每隔30秒观察到的微粒的位置,并用直线依次把这些位置连接了起来.实际上,即使在这短短的30秒内,微粒的运动也是极不规则的.

\begin{figure}[htp]
\centering
\includegraphics[scale=.45]{fig/1-4.png}
\caption{观察布朗运动的装置的示意图(左),
右图是显微镜下看到的微粒
}
\end{figure}

\begin{figure}[htp]\centering
	\begin{minipage}[t]{0.48\textwidth}
		\centering
\includegraphics[scale=1.2]{fig/1-5.pdf}
\caption{做布朗运动的微粒的运动路线
}
	\end{minipage}
	\begin{minipage}[t]{0.48\textwidth}
		\centering
\includegraphics[scale=1.2]{fig/1-6.pdf}
\caption{}
	\end{minipage}
\end{figure}






布朗运动是怎样产生的呢?起初,人们认为是由外界影响如震动、液体的对流等引起的.但是实验表明,在尽量排除外界影响的情况下布朗运动仍然存在.只要微粒足够小,在任何液体中都可以观察到布朗运动.布朗运动决不会停止.我们可以连续观察许多天甚至几个月,也看不到这种运动会停下来.
可见布朗运动的原因不在外界,而在液体内部.

甚至在显微镜下看起来是连成一片的液体,实际上也是由许许多多做不规则运动的分子组成的,悬浮在液体中的微粒不断地受到液体分子的撞击,图1.6描绘了一个微粒受
到液体分子撞击的情景.当微粒足够小时,它受到的来自各个方向的液体分子的撞击作用是不平衡的.在某一瞬间,微粒在某个方向受到的撞击作用强,它就沿着这个方向运动,在下一瞬间,微粒在另一方向受到的撞击作用强,它又向着另一方向运动,这样,就引起了微粒的无规则的布朗运动.悬浮在液体中的颗粒越小,在某一瞬间跟它相撞的分子数越少,撞击作用的不平衡性就表现得越明显,因而布朗运动越明显.悬浮在液体中的颗粒越大,在某一瞬间跟它相撞的分子数越多,撞击作用的不平衡性就表现得越不明显,以至可以认为撞击
作用相互平衡,因而布朗运动越不明显以至观察不到.

可见,液体分子永不停息的无规则运动是产生布朗运动的原因.分子的运动我们是看不见的,做布朗运动的微粒是由成千上万个分子组成的,微粒的布朗运动并不是分子的运动,但是微粒的布朗运动的无规则性,却反映了液体内部分子运动的无规则性.

实验表明,布朗运动随着温度的升高而愈加微烈.在扩散现象中,也是温度越高,扩散进行得越快,这表示分子的无规则运动跟温度有关系,温度越高,分子的无规则运动越激烈.正因为分子的无规则运动跟温度有关系,所以通常把分子的这种运动叫做热运动.


\subsection*{练习二}
\begin{enumerate}
\item 有人说布朗运动就是分子的运动,这种说法对吗?为什么?
\item 为什么悬浮在液体中的颗粒越小,它的布朗运动越明显?为什么悬浮在液体中的颗粒越大,它的布朗运动越不明显以至观察不到?
\item 为什么说布朗运动的无规则性反映了液体内部分子运动的无规则性?设想液体内部分子的运动是有规则的,比如在任何时刻所有分子都向着某个方向运动,还能不能产生布朗运动?
\item  图1.5中所示的不同小颗粒的布朗运动的情况并不相同,人们由此考虑到布朗运动不可能是由外界影响引起的.为什么?找几位同学一起讨论一下,并说明你的理由.
\end{enumerate}

\section{分子间的相互作用力}
布朗运动和扩散现象不但说明分子不停地做无规则运动,同时也说明分子间是有空隙的,否则分子便不能运动了.气体容易被压缩,水和酒精混合后的体积小于两者原来体积之和,说明气体分子之间、液体分子之间都有空隙.有人曾用两万标准大气压的压强压缩钢筒中的油,发现油可以透过筒壁逸出,说明钢的分子之间也有空隙.前面讲述分子的大小时,认为面体分子和液体分子是一个挨一个排列的,那只是为估算分子直径的数量级而作的设想.

分子间虽然有空隙,大量分子却能聚集在一起形成固体或液体,说明分子之间存在着引力.用力拉伸物体,物体内要产生反抗拉伸的弹力,就是因为分子间存在着引力的缘故.把两块纯净的铅压紧,由于分子间的引力,两块铅就合在一起,甚至下面吊一个重物也不能把它们拉开.把两块光学玻璃的表面磨得很光滑又相吻合,把表面处理干净,施加一定的压力就可以把它们粘合在一起,这也是利用了分子间的引力.

分子间有引力,而分子间又有空隙,没有紧紧吸在一起,这说明分子间还存在着斥力.面体和液体很难被压缩,即使气体,压缩到一定程度后再继续压缩也很困难,就是因为分子间存在着斥力的缘故,用力压缩物体,物体内要产生反抗压缩的弹力,就是分子间的斥力的表现.

研究表明,分子间同时存在着引力和斥力,它们的大小都跟分子间的距离有关.图1.7甲中的两条虚线分别表示两个
分子间的引力和斥力随距离变化的情形,实线表示引力和斥力的合力即实际表现出来的分子间的作用力随距离变化的情形.我们看到,引力和斥力都随着距离的增大而减小,当两分子间的距离等于$r_0$时,分子间的引力和斥力相互平衡,分子间的作用力为零.$r_0$的数量级约为$10^{-10}$米,相当于距离为$r_0$的位置叫做平衡位置.当分子间的距离小于$r_0$时,引力和斥力虽然都随着距离的减小而增大,但是斥力增大得更快,因而分子间的作用力表现为斥力.当分子间的距离大于$r_0$时,引力和斥力虽然都随着距离的增大而减小,但是斥力
减小得更快,因而分子间的作用力表现为引力,它随着距离的增大迅速减小,当分子间的距离的数量级大于$10^{-9}$米时,已
经变得十分微弱,可以忽略不计了.
\begin{figure}[htp]\centering
	\includegraphics[scale=.85]{fig/1-7.pdf}
	\caption{分子间的作用力跟距离的关系.图甲中斥
力用正值来表示,引力用负值来表示.合力
$F$为二者的代数和.合力为正值时,表示合
力为斥力;合力为负值时,表示合力为引力.
}
\end{figure}

我们知道,分子是由原子组成的,原子内部有带正电的原子核和带负电的电子.分子间这样复杂的作用力就是由这些带电粒子的相互作用引起的.

上面我们讲了分子运动论的基本内容,分子不停地做无规则运动,它们之间又存在相互作用力,分子力的作用使分子聚集在一起,分子的无规则运动将使它们分散开来,由大量分子组成的物体可以处于气、液、固三种不同的物质状态,正是由这两种相反的因素决定的,在面体中,分子力的作用比较强大,绝大多数分子被束缚在平衡位置附近做微小的振动.温度升高,分子的无规则运动加剧,加剧到一定限度,分子力的作用已经不能把分子束缚在固定的平衡位置附近,但分子还不能分散远离,于是物体表现为液体状态.温度再升高,分子的无规则运动更加剧,到一定限度,分子分散远离,分子力的作用很微弱,分子可以到处移动,物体就表现为气体状态.

\subsection*{练习三}
\begin{enumerate}
	\item 什么事例说明分子间有引力?什么事例说明分子间有斥力?
	\item 当分子间的距离大于$r_0$时,随着距离的增大,引力和斥力哪个减小得快?当分子间的距离小于$r_0$时,随着距离的减小,引力和斥力哪个增加得快?
	\item 物体为什么能够被压缩,但又不能无限地被压缩?
\item	从图1.7看出,当分子中心间的距离小于$r_0$时,分
子间的作用力表现为斥力,它随着距离的减小而很快地增大.分子间作用力的这一特点,可以借助于下述模型想象出来.设想分子为弹性钢球,当两个钢球相撞时,它们都发生微小的形变,因而在它们之间产生相互推斥的弹力,如同分子间的作用力表现为斥力一样.钢球发生微小形变就可以产生很大的弹力,所以这个弹力随着钢球中心间距离的减小而很快地增大.利用这一模型可以粗略地估计出分子直径的数量级为$10^{-10}$米.这是怎样估计的?

\end{enumerate}

\section*{复习题}
\begin{enumerate}
\item 分子运动论的基本内容是什么?
\item 就你所知道的,测定分子的大小和阿伏伽德罗常数有什么方法?
\item 什么叫布朗运动?布朗运动是怎样产生的?为什么把大量分子的无规则运动叫做热运动?
\item 仔细研究图1.7,说明分子间作用力的特点.
\end{enumerate}


















\chapter{内能~~能的转化和守恒定律}

\section{物体的内能}
\subsection{分子的动能~~温度} 
既然组成物体的分子不停地做无规则运动,那么,象一切运动着的物体一样,做热运动的分子也具有动能.

物体里分子运动的速率是不同的,有的大,有的小,因此各个分子的动能并不相同、在热现象的研究中,我们所关心的不是物体里每个分子的动能,而是所有分子的动能的平均值.这个平均值叫做分子热运动的平均动能.

温度升高,物体分子的热运动加剧,分子热运动的平均动能也增加,温度越高,分子热运动的平均动能越大,温度越低,分子热运动的平均动能越小,从分子运动论的观点看来,\textit{温度是物体分子热运动的平均动能的标志}.这样,分子运动论使我们懂得了温度的微观含义.

\subsection{分子势能} 

分子间存在相互作用力,因此分子间具有由它们的相对位置决定的势能,这就是分子势能.

分子间的距离大于$r_0$(见图1.7)的时候,分子间的相互作用表现为引力,要增大分子间的距离必须克服引力做功,因此分子势能随着分子间的距离的增大而增大.这种情形同弹簧被拉长时弹性势能的变化相似.分子间的距离小于$r_0$的
时候,分子间的相互作用表现为斥力,要减小分子间的距离必须克服斥力做功,因此分子势能随着分子间的距离减小而增大.这种情形同弹簧被压缩时弹性势能的变化相似.

物体的体积发生变化时,分子间的距离也发生变化,因而分子势能随着发生变化.可见分子势能跟物体的体积有关系.

气体分子间的距离较大,分子的相互作用是引力.对气体来说,体积增大,分子间的距离增大,分子势能增加;体积缩小,分子间的距离减小,分子势能减少.

\subsection{物体的内能} 

物体中所有分子的热运动的动能和分子势能的总和,叫做物体的\textbf{内能}.一切物体都是由不停地做无规则热运动并且相互作用着的分子组成的,因此任何物体都具有内能.

由于分子热运动的平均动能跟温度有关系,分子势能跟体积有关系,因此物体的内能跟物体的温度和体积有关系.温度升高时,分子的动能增加,因而物体的内能增加.体积变化时,分子势能发生变化,因而物体的内能发生变化.

任何物体都具有内能,它同时还可以具有机械能.例如正在空中飞行的炮弹,除了具有内能,还具有机械能——动能和重力势能.下面我们要研究内能的变化,在作这种研究的时候,我们暂时不考虑作为研究对象的那个物体的机械能的变化.

顺便指出:我们过去常常提到热能,学过内能后应该知道,所谓热能不过是内能的一种通俗的说法.

\section*{阅读材料:热的本质}
热的本质是什么?为了弄清这个问题,人类经历了一段
曲折的认识过程.在二百多年以前,人们普遍认为热是一种特殊的物质——热质,热质是一种没有质量的流质,它既不能产生,也不能消失,总保持守恒,一个地方的热质多了,另一个地方的热质要变少.热质流入一个物体,物体含有的热质多了,温度就升高;热质从一个物体流出,物体含有的热质少了,温度就降低.这就是热质说,热质说成功地说明了有关热传导和热量测定的一些实验事实,直到十九世纪初大多数学者都支持热质说.

热质说碰到的最大困难是对摩擦生热现象的解释.1798
年,本杰明·汤普森(伦福德伯爵)在慕尼黑指导军工生产时发现:用钻头加工炮筒时,摩擦可以产生大量的热,使炮筒的温度升得很高,而且只要钻孔继续进行,就会不断地产生出大量的热来,好象物体里含有的热质是取之不尽的,热质并不守
恒,维护热质说的人解释说:炮筒温度升高,是由于钻下来的铜屑的比热减小了,铜屑放出的热质被炮筒所吸收.伦福德测定了钻下来的铜屑的比热,证明比热一点也没减小.伦福德的实验给热质说一个致命的打击,伦福德从大量实验中得出结论:热不可能是一种物质,只能认为热是一种运动.

后来还有许多人研究了热和机械功的关系,十九世纪中叶建立了能的转化和守恒定律,确认热是能的一种形式,它可以跟机械能、电能等相互转化,并在转化中守恒,而不存在守
恒的热质.能的转化和守恒定律的建立,彻底否定了热质说,同时为分子运动论的发展开辟了道路.而分子运动论进一步从微观上研究热现象,说明热现象是大量分子做无规则运动的表现,热这种形式的能是大量做无规则运动的分子具有的能,即课文中讲的内能.这样,人们对热的本质获得了正确认识.


\subsection*{练习一}
\begin{enumerate}
	\item 壶里的水被加热而温度升高,水的内能怎样改变?液体的热膨胀很小,可不予考虑.
\item 一根烧红了的铁棍逐渐冷却下来,铁棍的内能怎样改变?固体的热膨胀很小,可不予考虑.
\item 客器里装着一定质量的气体,在保持体积不变的条件下使它的温度升高,气体的内能怎样改变?在保持温度不变的条件下把气体压缩,气体的内能怎样改变?
\item 设想我们对固体进行压缩.当分子间的距离小于$r_0$时,随着固体被压缩分子势能怎样改变?
\item 一颗炮弹在高空中以某一速度$v$飞行,由于炮弹中所有分子都具有这一速度,所以分子具有动能,又由于所有分子都在高处,所以分子具有势能.所有分子的上述动能和势能的总和就是炮弹的内能.上述说法正确不正确?为什么?
\end{enumerate}

\section{改变内能的两种方式}
在热学研究中所涉及的总是内能的变化.那么,什么物理过程可以改变物体的内能呢?

做功可以改变物体的内能.用锯条锯木头,我们克服摩
擦力做了功,锯条和木头的温度升高,内能增加.这类所谓摩擦生热的现象,是大家都知道的.物体在非弹性碰撞中做功,可以使它们的温度升高,内能增加,用搅拌器在水中搅拌做功,可以使水的温度升高,内能增加.气体被压缩或膨胀时做功,气体的内能就发生变化,在
一个厚壁玻璃筒里放一块浸过
乙醚的棉花,迅速压下活塞对筒内空气做功,空气的内能增加,温度升高,达到乙醚的着火点,浸了乙醚的棉花就燃烧起来(图2.1).柴油机就是利用这个道理来点火,使喷入气缸内的雾状柴油燃烧的.热机气缸内高温高压的气体膨胀时做功,气体的温度降低,内能减少.热机就是利用这个道理对外做功的.

\begin{figure}[htp]
\centering\includegraphics[scale=.7]{fig/2-1.png}
\caption{压缩气体做功,气体内能增加
}
\end{figure}

但做功并不是改变物体内能的唯一方式.灼热的火炉使它上面和周围的物体温度升高,这些物体的内能增加,火炉熄灭后,这些物体的温度降低,内能又减少,在这样的过程中,物体的内能改变了,但是并没有做功,这种没有做功而使物体内能改变的物理过程叫做\textbf{热传递}.

可见,\textit{能够改变物体内能的物理过程有两种:做功和热传递}.

做功使物体的内能发生变化的时候,内能的变化就用功的数值来量度,外界对物体做多少功,物体的内能就增加多少;物体对外界做多少功,物体的内能就减少多少.

热传递使物体的内能发生变化的时候,内能的变化是用热量来量度的.外界传递给物体多少热量,或者说物体吸收了多少热量,物体的内能就增加多少;物体传递给外界多少热量,或者说物体放出了多少热量,物体的内能就减少多少.

一杯水可以用热传递的方式传给它一定的热量,使它从
某一温度升高到另一温度;也可以用做功的方式,比如用搅拌器在水中搅拌,使它升高同样的温度.两种方式不同,得到的结果却相同,除非事先知道,我们将无法区别是哪种方式使这杯水的内能增加的.可见,做功和热传递对改变物体的内能是等效的.

\section{热功当量}
既然做功和热传递对改变物体的内能是等效的,功和热量都可以用来量度内能的变化,那么功和热量之间就应该有确定的数量关系.

我们在初中学过,热量的单位是卡,使1克水的温度升高1$^\circ$C所需的热量就是1卡.如果功和热量之间有确定的数量关系,1卡的热量相当于多少焦耳的功?相当于单位热量的功的数值叫做\textbf{热功当量}.历史上第一个用实验来测定热功当量的人是英国物理学家焦耳.他用各种不同的方法测定了热功当量,下面我们介绍其中最著名的一种(图2.2).

\begin{figure}[htp]
\centering\includegraphics[scale=.8]{fig/2-2.png}
\caption{左图是焦耳测定热功当量的装置;
右图是量热器的纵截面图.}
\end{figure}


量热器里装着水,重物$P$和$P'$下落时带动量热器中的轴转动,轴上的叶片就带动周围的水随着转动,量热器内壁上也固定着叶片,它们的作用是阻碍水的运动,增大摩擦.叶片搅动水做功,使水的内能增加,温度由$t_1$升高到$t_2$.已知每个重物的质量$M$和落下的高度$h$,我们可以算出这个功:
$$W=2Mgh $$
假定水的内能的增加不是由于做功而是由于热传递的结果,我们也可以算出使水的温度由$t_1$升高到$t_2$所需的热量:
\[Q=(m_1c_1+m_2c_2)(t_2-t_1)\]
其中$m_1$, $m_2$, $c_1$, $c_2$分别表示水和量热器的质量和比热.这样,就可以求出热功当量:
\[J=W/Q\]

这个实验焦耳做过多次,测得的热功当量的数值相同.他
又用水银代替水,重做上述实验,也得到相同的结果,他还用其他方法来测定,结果仍然相同.焦耳同时代的和以后的许多科学家用不同的方法来测定,结果都相同.热功当量的数值通常可取为
\[ J=4.2\text{焦/卡} \]

热功当量的数值的确定,证明功和热量之间存在着确定的数量关系,即1卡=4.2焦,或者1焦=0.24卡.这进一步定量地证明做功和热传递对改变物体的内能是等效的.

热量的单位卡,是过去人们对热的本质认识不清楚的情况下规定的.既然功和热量之间有确定的数量关系,那么,功、热量和能量使用相同的单位,是很自然也很合理的,现在,国际单位制中规定它们统一用焦耳作单位,并建议逐步取消卡这个单位.

\subsection*{练习二}
\begin{enumerate}
\item 举出几个实例来说明:做功可以改变物体的内能.
\item 锅炉中盛有150千克的水,由20$^\circ$C加热到100$^\circ$C,水的内能增加多少?
\item 一个物体的内能增加了20焦.如果物体跟周围环境不发生热交换,周围环境需要对物体做多少焦的功?如果周围环境对物体没有做功,需要传给物体多少焦的热量?
\item 设想在测定热功当量的不同实验中得到的结果并不相同,还能不能得到结论说:做功和热传递对改变物体的内能是等效的?讨论一下这个问题.
\item 在图2.2所示的实验中,已知重物$P$和$P'$的质量都是14千克,水的质量是7千克,重物连续从2米高处落下20次后,水的温度升高0. 37$^\circ$C.不考虑传给量热器和外界的热量,试根据这些数据计算热功当量.
\end{enumerate}



\section{能的转化和守恒定律}
\subsection{热力学第一定律} 

现在我们来研究功、热量跟内能变化之间的定量关系.

一个物体,如果它跟外界不发生热交换,也就是它既没有
吸收也没有放出热量,那么,外界对它做多少功,它的内能就增加多少,设外界对物体所做的功为$W$,内能的增加为$\Delta E$,那么,$W=\Delta E$.在物体对外界做功的情况下,上式同样适用.这时$W$为负值,内能的增加$\Delta E$也是负值,表示内能减少.

如果外界既没有对物体做功,物体也没有对外界做功,那么物体吸收了多少热量,它的内能就增加多少.设物体吸收的热量为$Q$,内能的增加为$\Delta E$,那么,$Q=\Delta E$.在物体放出热量的情况下,上式同样适用,这时$Q$为负值,内能的增加$\Delta E$也是负值,表示内能减少.

在一般情况下,如果物体跟外界同时发生做功和热传递的过程,那么,外界对物体所做的功$W$加上物体从外界吸收的热量$Q$,等于物体内能的增加$\Delta E$.即
\[W+Q=\Delta E\]
上式所表示的功、热量跟内能变化之间的定量关系,在物理学中叫做\textbf{热力学第一定律}.

\subsection{能的转化和守恒定律} 
现在我们从能的转化的观点来考察热力学第一定律.我们知道,功是能的转化的量度,做功使内能发生变化时,其他形式的能和内能发生相互转化.在摩擦生热的现象中,克服摩擦力做多少功,就有多少机械能转
化成等量的内能.在图2.1所示的压缩气体做功的过程中,做多少功,就有多少机械能转化成等量的内能.气体膨胀做功的时候,做多少功,就有多少内能转化成等量的机械能.热传递使内能发生变化时,只是内能在物体之间的转移,而没有能量形式的转化.一个物体从外界吸收了多少热量,就有多少内能从外界转移给这个物体.这里我们看到,做功和热传递对改变物体的内能虽然等效,但从能的转化的观点来看却是有区别的,热力学第一定律表示,做功和热传递提供给一个物体多少能量,物体的内能就增加多少,能量在转化或转移中守恒.

不但机械能,其他形式的能也可以和内能相互转化.通过电流的导线变热,电能转化成内能.燃料燃烧生热,化学能转化成内能,炽热的灯丝发光,内能转化成光能,实验证明,在这种转化中能量也守恒.

大量的事实证明:各种形式的能都可以相互转化,并且在转化中守恒.


\textbf{能量既不会凭空产生,也不会凭空消失,它只能从一种形式转化为别的形式,或者从一个物体转移到别的物体.}这就是能的转化和守恒定律.

现在我们利用这个定律来分析一下射到地球上的太阳能的转化.太阳把地面晒热,把空气晒热,把水面晒热并使一部分水蒸发.变热的空气上升,使空气流动而形成风,太阳能转化成空气的机械能.蒸发的水蒸气上升到空中形成云,以雨雪等形式落下来,通过江河流入海洋,太阳能转化成水的机械
能.太阳能的一部分被植物叶子吸收,发生光合作用,生成各
种有机化合物,太阳能转化成植物的化学能.植物作为食物被动物吃掉,植物的化学能转化成动物的化学能.人们以植物和动物为食物,从中获得了维持生命活动的能量.古代的植物和动物在地质变迁中转化成煤、石油、天然气,成为我们现代工农业生产的主要能源,在水力发电站和火力发电站里,水的机械能,煤、石油和天然气的化学能转化成电能,在工厂、农村和住宅中,电能通过各种电器转化成机械能、内能、光能等等.

物质有许多不同的运动形式,每种运动形式都有一种对应的能.跟机械运动对应的是机械能,跟热运动对应的是内能,跟其他运动形式对应的还有电能、磁能、化学能、原子能等等.能的不断转化表现了物质的运动不断地由一种形式转化为另一种形式,能的转化和守恒定律就是关于自然界的这种转化过程的一条普遍定律.

\subsection{永动机不可能制成} 
历史上有不少人希望设计一种机器,这种机器不消耗任何能量,却可以源源不断地对外做功.这种机器被称为永动机.虽然经过多次尝试,作了各种努力,但无一例外地归于失败.这种尝试的失败导致了能的转化和守恒定律的发现.而能的转化和守恒定律则明确指出:任何一部机器只能使能量从一种形式转化为另一种形式,要对外
界做功必须消耗能量,不消耗能量便无法对外界做功,因而永动机不可能制成.我们利用自然,必须遵循自然界的规律,违反这种规律,就要失败.制造永动机只能是一种永远无法实现的幻想.

\begin{example}
	一定量的气体从外界吸收热量$6.36\times 10^4$卡,
内能增加$4.25\times 10^5$焦,气体对外界做了功,还是外界对气体做了功?做多少功?	
\end{example}

\begin{solution}
利用热力学第一定律的公式进行计算,$W$、$Q$和$\Delta E$要统
一用焦耳作单位.
\[Q=6.36\times 10^{4}{\rm Cal}=2.66\times 10^5{\rm J}\]
由于$W+Q=\Delta E$,所以
\[\begin{split}
W=\Delta E-Q&=4.25 \times 10^5{\rm J}-2.66\times 10^5{\rm J}\\
&=1.59\times 10^5{\rm J}
\end{split} \]
$W$为正值,表示外界对气体做了$1.59\times 10^5$焦的功.
\end{solution}

\subsection*{练习三}
\begin{enumerate}
		\item 做功和热传递对改变物体的内能虽然等效,但从能的转化的观点来看是有区别的,这种区别是什么?
	\item 在图2.2所示的焦耳测定热功当量的实验中,什么其他形式的能转化成了水的内能?在历史上,热功当量的确定为建立能的转化和守恒定律提供了坚实的实验基础.你怎样理解这句话?讨论一下这个问题.
	\item 用活塞压缩气缸里的空气,对空气做了900焦的功,同时气缸向外散热210焦.空气的内能改变了多少?
	\item 空气压缩机在一次压缩中,活塞对空气做了$2\times 10^5$焦的功,同时空气的内能增加$1.5\times 10^5$焦.这时空气向外界传递的热量是多少?
	\item 如果用$Q$表示物体吸收的热量,用$W$表示物体对外界所做的功,热力学第一定律也可以表达为下式:
	\[Q=\Delta E+W\]
	怎样解释这个表达式的物理意义?试根据课文中的表达式推
导出这个表达式.
\end{enumerate}

\section{能的转化和守恒定律的建立及其意义}
能的转化和守恒定律,是在工业革命的直接影响下,经过许多物理学家的长期探索,在十九世纪确立的,对这一定律的发现有重大贡献的物理学家中,特别值得我们纪念的有英国物理学家焦耳(1818—1889)、德国医生兼物理学家迈尔(1814—1878)、德国物理学家亥姆霍兹(1824—1894).

焦耳一生致力于实验研究,他从1840年起在将近四十年的时间里,做了四百多次实验,用各种不同的方法测定了热功当量,为能的转化和守恒定律的建立提供了坚实的实验基础.迈尔指出能的转化的广泛存在,是世界上首先阐述能的转化和守恒思想的人,他还从理论上计算出热功当量的数值,亥姆霍兹分析了不同形式的能的转化和守恒,并且把这个结果跟永动机不可能制成联系起来,他在1847年的论文中对能的转化和守恒定律作了清晰而全面的论述,使这个定律为人
们广泛接受.恩格斯曾经把这一定律称为“伟大的运动基本定律”,并把这一定律和细胞学说、达尔文的生物进化论一起称为十九世纪自然科学的三大发现.

能的转化和守恒定律自从建立以来,就是人们认识自然和改造自然的有力武器.从物理、化学、生物到天文、地质,以及各种工程技术,这一定律都发挥了重要的作用.可以说,没有别的定律能象这一定律那样把如此广泛的科学技术领域联
系起来,使不同领域的科学工作者具有一系列的共同语言.

人们利用能的转化和守恒定律来研究自然取得了许多重
大的成就,下面我们简单介绍一个近代物理学研究中的例子.在三十年代初期,人们发现在某些原子核反应中能量似乎并不守恒,一部分能量消失了.当时一些人就认为这些实验事实表明能量并不是普遍守恒的.但是另一些人认为在这里能量也是守恒的,奥地利物理学家泡利(1900—1958)在1933年提出,可能存在着一种当时并不知道的极其微小的粒子,所谓消失了的能量,就是被它们带走了,后来意大利物理学家费密(1901—1954)把这种粒子叫做中微子,并且发展了有关中微子的理论,认为它是一种不带电的、质量极其微小的粒子.以后人们一面继续从理论上研究这种假设的中微子,一面想办法用实验来探测中微子的存在,直到1956年,在人们已经拥有核反应堆后,才在实验中证实了中微子是的确存在的,这样,能的转化和守恒定律就直接导致了中微子的发现.

我们正在为实现我国的工业、农业、国防和科学技术的现代化而斗争,在这一伟大事业中,能的转化和守恒定律将是我们手中的强有力的武器之一.

\section{能源的利用和开发}
机器的运转,金属的冶炼,火车的行驶,飞机的飞行,都需要能量.日常生活中做饭、取暖、照明等,也需要能量,凡是能够提供能量的东西,都可以叫做能源.煤、石油、天然气在燃烧时可以提供能量,它们是能源.水力和力可以提供能
量,它们也是能源,能源是提高人民生活水平和进行社会主义现代化建设的重要物质基础.

人类在生产和生活中需要各种形式的能,其中用量最大的是内能、机械能和电能.我们可以按照需要把能源的能量转化成各种形式的能,以供应用.能源的利用过程,实质上是能的转化和传递的过程.

煤、石油、天然气等燃料,在各种各样的燃烧炉中燃烧,它们的化学能转化成内能.内能可以直接使用,以满足生产和生活的需要;也可以通过各种热机转化成机械能,然后再被使用.电能便于输送而且用起来方便,因此还可以用热机带动发电机进一步把能量转化成电能,这就是火力发电.

水力可以通过水轮机做功,风力可以通过风车做功,变为便于应用的机械能.还可以用水轮机或风车带动发电机进一步把能量转化成电能,这就是水力发电或风力发电.

能源的能量在转化中虽然保持守恒,但最后被有效利用的仅是其中的一部分,因此,提高能源的利用率,有效地利用能源,是十分重要的.

燃料在燃烧炉中燃烧时,理论上虽然化学能可以全部转化成内能,实际上却只有一部分转化成可供利用的内能.原因是:燃烧不完全,一部分化学能没有发生转化,还有许多能量以散热的形式散失了.因此,要改进燃烧条件,促使燃料完全燃烧,并且要做好炉体、管道等设备的绝热保温,以减少热量的散失.

同燃料的化学能转化成内能相比,把内能转化成机械能要困难得多.从理论上讲,任何热机都不可能把内能全部转
化成机械能,一定要有一部分内能被废气所带走,热机工作物质(蒸汽或燃气)的温度越高,内能转化成机械能的部分就越大,因此,使用高温的蒸汽或燃气,是提高热机效率的有效途径.充分利用废气带走的内能,可以大幅度提高燃料的总利用率,在火力发电站中,可以利用蒸汽轮机排出的蒸汽给工厂和居民供热,这就是电热并供的热电站.

电能便于输送,但是远距离输电按目前的技术水平能量损失也不小.各种用电设备的效率虽然比较高,但都有能量损失,因此,提高电能输送和用电设备的效率也是节能的重要措施.

\begin{figure}[htp]
\centering\includegraphics[scale=.6]{fig/2-3.pdf}
\caption{能源能量的转化和利用}
\end{figure}

现在人类社会使用的能源主要是煤、石油和天然气.但是煤、石油、天然气的储量是有限的,因此在合理开发和有效利用煤、石油、天然气等常规能源的同时,要不断探索能的转
化的新途径,大力开发和利用新能源.核能、太阳能、地热能、海洋能等都属于新能源.

核能也叫原子能,是原子核发生变化时释放出的能量.核能的发现虽然只有四十几年的历史,但是核能的利用已获得了巨大进展.原子核发生变化的方式有裂变和聚变.这方面的知识我们将在高中三年级学习,目前裂变技术已经成熟.核燃料的核能通过反应堆可以转化成内能,再通过热机和发电机可以转化成电能,这就是核电站.对相同质量的燃料来说,核能比化学能要大几百万倍.1千克铀裂变时释放的能量相当于2400吨标准煤燃烧时释放的能量.因此,兴建核电站在经济上是合算的.目前我们正在兴建大型的核电站.核聚变能的利用还处于研究阶段,一旦成功将使人类享有可供长期使用的能源.

我们讲过,煤、石油和天然气的化学能归根到底来自太阳能,作为新能源的太阳能,是指直接利用射到地球上的太阳能,利用太阳灶可以把太阳能转化成内能.利用光电池还可以把太阳能直接转化成电能.太阳能是取之不尽,用之不竭的,但是目前利用太阳能存在着成本高、效率低等问题,要想大规模利用还需要取得技术上的突破.

地热能和海洋能的利用,目前处于试验研究阶段,大规模利用需要解决一系列技术问题.

我国是一个能源资源比较丰富的国家.煤、石油、天然气的储量丰富,水力资源居世界第一.但我国的能源利用率较低,浪费较大,能源供应比较紧张,我国正大力加强能源的科学研究,以掌握有关的先进科学技术,抓好能源的开发,开展
以节能为中心的技术改造.这样我们将能够逐步克服能源供应的紧张,有条件依靠自己的能源资源实现社会主义四个现代化.同学们要好好学习,将来在能源科学技术中可以大显身手.

\subsection*{练习四}

\begin{enumerate}
	\item 试说明下列现象中能量是怎样转化的:
	\begin{enumerate}
		\item 在水平公路上行驶的汽车,发动机熄火之后,速度越来越小,最后停止.
		\item 在阻尼振动中,单摆的振幅越来越小,最后停下来.
		\item 火药爆炸产生燃气,子弹在燃气的推动下从枪膛发射出去,射穿一块钢板,速度减小.
		\item 用柴油机带动发电机发电,供给电动水泵抽水,把水从低处抽到高处.
	\end{enumerate}

\item 取一个不高的横截面积是30${\rm cm^2}$的圆筒,简内装水0.6千克,用来测量射到地面的太阳能,在太阳光垂直照射2分钟后,水的温度升高了1$^\circ$C.计算在阳光直射下地球表面每平方厘米每分钟获得的能量.
\item 从20米高处落下的水,如果水的势能的20\%用来使水的温度升高,水落下后的温度升高多少摄氏度?
\item 用铁锤打击铁钉,设打击时有80\%的机械能转化为内能,其中50\%用来使铁钉的温度升高.打击20次后,铁钉的温度升高多少摄氏度?已知铁锤的质量为1.2千克,铁锤打击铁钉时的速度为10${\rm m/s}$,铁钉的质量为40克,铁的比热为$5.0\times 10^2 {\rm J}/({\rm kg\cdot ^\circ C})$.

\item 在光滑的桌面上放着一个木块,铅弹从水平方向射中木块,把木块打落在地面上,落地点与桌边的水平距离为0.4米.铅弹射中木块后留在木块中.设增加的内能有60\%使铅弹的温度升高,铅弹的温度升高多少摄氏度?已知桌面高为0.8米,木块的质量为2千克,铅弹的质量为10克,比热为$1.3\times 10^2 {\rm J}/({\rm kg\cdot ^\circ C})$.取$g=10{\rm m}/{\rm s^2}$.

\end{enumerate}

\section*{复习题}
\begin{enumerate}
	\item 从分子运动论的观点看来,温度标志着什么?什么是分子的动能?什么是分子势能?什么是物体的内能?物体的内能跟什么有关系?
	\item 改变物体的内能有哪两种方式?从能的转化的观点来看,它们有什么区别?
\item 什么是热功当量?热功当量的测定在物理学的发展中有什么重要意义?
\item 热力学第一定律的内容是什么?写出它的数学表达式.
\item 能的转化和守恒定律的内容是什么?这个定律有什么重要意义?
\end{enumerate}















\chapter{气体的性质}
\section{气体的状态和状态参量}
我们研究物理学问题时,经常要用一些物理量来描述研
究对象.问题不同,所用的物理量也不同.在力学中我们用位置、速度等物理量来描述物体的运动状态,现在研究气体的热学性质,我们要用体积、压强、温度等物理量来描述气体的状态.描述气体状态的这几个物理量叫做气体的\textbf{状态参量}.

气体分子可以自由移动,因而气体总要充满整个容器,气体的体积就是指气体所充满的容器的容积,在国际单位制中,体积的单位有${\rm m}^3$、${\rm dm}^3$、${\rm cm}^3$等.日常生活和生产中还常用升作单位,升的国际代号是L.
\[1{\rm L}=10^{-3}{\rm m}^3=1{\rm dm}^3\]

气体对器壁有压力的作用,这是气体分子频繁地碰撞器壁而产生的.用打气筒把空气打到自行车的车胎里去,会把车胎胀得很硬,就是因为空气对车胎有压力而造成的.气体作用在器壁单位面积上的压力叫做气体的压强.

在国际单位制中,压强的单位是帕斯卡,简称帕,国际代号是Pa,$1{\rm Pa}=1{\rm N}/{\rm m}^2$.气体的压强还常用标准大气压和毫米汞柱作单位.
\[1\text{标准大气压}=760\text{毫米汞柱}=1.013\times 10^5{\rm Pa}\]
\[1\text{毫米汞柱}=133.3{\rm Pa}\]

\begin{figure}[htp]\centering
	\begin{tikzpicture}[>=latex]
\draw [pattern=north east lines](2.5,-2) rectangle (2.7,0);	
\draw[->](1.5,-1) node [below]{$p$}--(2.5,-1);
\draw[->](3.7,-1)node [below]{$p_0$}--(2.7,-1);

  \pgfsetlinewidth{4pt}
\pgfsetinnerlinewidth{3pt}
\draw [rounded corners=0.2cm](4,0)--(0,0)--(0,-2)--(4,-2);
\end{tikzpicture}
\caption{气体的压强等于大气压}
\end{figure}



在图3.1中,容器内的气体被活塞封闭着,当活塞静止不动时,容器内的气体对活塞的压力跟大气压对活塞的压力平
衡,所以这时容器内的气体的压强$p$等于大气压$p_0$,即$p=p_0$.如图3.2所示,用长为$h$的一小段水银柱把气体封闭在玻
璃管里.玻璃管水平放置时,被封闭的气体的压强$p_1$等于
大气压$p_0$,即$p_1=p_0$.玻璃管开口向上竖直放置时,气体的压强$p_2$等于大气压$p_0$加上这小段水银柱产生的压强$p_h$,即$p_2=p_0+p_h$.玻璃管开口向下竖直放置时,气体的压强$p_s$加上这小段水银柱产生的压强$p_h$等于大气压$p_0$,即$p_s+p_h=p_0$,由此得到$p_3=p_0-p_h$.

\begin{figure}[htp]\centering
	\begin{tikzpicture}[>=latex]
\draw[rounded corners=0.1cm](-2,0)--(-6,0)--(-6,-.25)--(-2,-.25);	
\draw[rounded  corners=0.1cm, fill=black](-4,-.25) rectangle (-3,0);		
\node at (-3.5,-.5){$h$};
\draw[<-](-2,-0.125)--node[below]{$p_0$}(-1.5,-.125);

\draw[rounded corners=0.1cm](0,2)--(0,-2)--(.25,-2)--(.25,2);

\draw[rounded  corners=0.1cm, fill=black](0,-.25) rectangle (.25,.75);
\node at (.5,.25){$h$};
\draw[<-](0.125,2)--node[right]{$p_0$}(0.125,2.5);


\draw[rounded corners=0.1cm](1,-2)--(1,2)--(1+.25,2)--(1+.25,-2);

\draw[rounded  corners=0.1cm, fill=black](1,-1.75) rectangle (1+.25,.75-1.5);
\node at (1.5,-1.25){$h$};
\draw[<-](1.125,-2)--node[right]{$p_0$}(1.125,-2.5);

	\end{tikzpicture}
	\caption{}
\end{figure}

温度这个物理量大家都很熟悉.温度是表示物体冷热程度的物理量,是物体分子热运动的平均动能的标志.温度的数值表示法叫做温标,我们在初中学过摄氏温标,用摄氏温标表示的温度叫做摄氏温度,在国际单位制中采用热力学温标(又常叫绝对温标),这种温标将在第四节中讨论.

研究气体的性质,首先引起我们注意的是描述气体状态的这三个物理量的变化.举例来说,地面附近的空气变热以后向空中上升时,它的体积、压强和温度都发生变化.把氧气装入钢筒时,或者用户(工厂、医院)把氧气从钢筒中放出来使用时,氧气的体积、压强和温度都发生变化.内燃机气缸里的燃料混合物爆发时,这三个物理量也都发生变化.对一定质量的气体来说,如果体积、压强和温度这三个量都不改变,我们就说气体处于一定的状态中,如果这三个物理量同时发生变化或者其中有两个发生变化,我们就说气体的状态改变了.对一定质量的气体来说,只有一个量改变而其他两个都不改变的情况,是不会发生的.那么,在气体的状态改变时,这三个物理量的变化是任意的,还是相互关联,遵循一定的规律?如果遵循一定的规律,这个规律又是什么?这就是本章讨论的中心课题.

下面,我们用实验方法先研究一定质量的气体在分别保持温度、体积不变时,其他两个量的变化规律,然后在此基础上确定三个状态参量的变化规律.


\subsection*{练习一}
\begin{enumerate}
	\item 什么叫气体的压强?举出气体对器壁有压力作用的几个实例.
	\item 大气压为750毫米汞柱时,等于多少帕?
		\item 在图3.2中,水银柱的长度为19厘米,大气压为
	760毫米汞柱,玻璃管开口向上竖直放置时,被封闭的气体的压强等于多少毫米汞柱?开口向下竖直放置时,等于多少毫米汞柱?
\begin{figure}[htp]\centering
	\includegraphics[scale=.55]{fig/3-3.png}
	\caption{}
\end{figure}


	\item 图3.3是测量气体压强的水银压强计,两端开口的U形管内装入水银,$A$管跟容器连接.已知大气压$p_0$和两管中水银面的高度差,就可以知道容器中气体的压强.大气压为$1.013\times 10^5{\rm Pa}$,图甲和图乙中的$h$都是10厘米,分别求出这两种情形中气体的压强是多少帕.
	\item 在图3.4所示的几种情形中,被封闭的气体$A$的压强分别是多少帕?大气压为$1.013\times 10^5{\rm Pa}$.
	\item 举出气体状态发生改变的几个实例.
\begin{figure}[htp]\centering
	\includegraphics[scale=.55]{fig/3-4.png}
	\caption{}
\end{figure}

\end{enumerate}

\section{气体的等温变化~~玻意耳-马略特定律}
我们先来研究温度不变时,一定质量的气体的压强随着
它的体积而变化的情形.这种变化叫做\textbf{等温变化}.
\begin{figure}[htp]\centering
	\includegraphics[scale=.55]{fig/3-5.png}
	\caption{}
\end{figure}

实验装置如图3.5所示,玻璃管$A$和$B$通过一条橡皮管连在一起,$A$管上端有一个阀门$a$,$B$管上端是开口的.$A$管固定在有刻度的竖直板上,$B$管可以上下移动,实验开始时,打开阀门$a$,从$B$管注入水银,然后关闭阀门,把一定质量的空气封闭在$A$管里.当两管中的水银面一样高时(图甲),$A$管里空气的压强等于作用在$B$管水银而上的大气压.

把$B$管慢慢提高,使$A$管里空气的体积缩小.这时$B$管里的水银面比$A$管里的高(图乙),$A$管里气体的压强等于大
气压加上$B$管水银面高出$A$管水银面的那段水银柱的压强.实验表明,在温度不变的条件下,气体的体积缩小到原来的几分之一,它的压强就增大到原来的几倍.

把$B$管慢慢放低,使$A$管里气体的体积增大,这时$B$管里的水银面比$A$管里的低(图丙),$A$管里气体的压强等于大气压减去$A$管水银面高出$B$管水银面的那段水银柱的压强.实验表明,在温度不变的条件下,气体的体积增大到原来的几倍,它的压强就减小为原来的几分之一.

改用其他气体做这个实验,得到的结果相同.

英国科学家玻意耳(1627—1691)和法国科学家马略特(1620—1684)各自独立地用实验研究了气体的压强和体积的关系,得到下面的结论:

\textbf{温度不变时,一定质量的气体的压强跟它的体积成反比}.这个结论叫做\textbf{玻意耳-马略特定律}.

玻意耳-马略特定律可以用公式来表示,保持一定质量气体的温度不变,设体积为$V_1$时压强为$p_1$,体积为$V_2$时压强为$p_2$,那么
\[\frac{p_1}{p_2}=\frac{V_2}{V_1} \]
或
\[p_1V_1=p_2V_2 \]

由此式可以看出,玻意耳-马略特定律也可以叙述为:\textbf{温度不变时,一定质量的气体的压强跟它的体积的乘积是不变的},
\[pV=\text{恒量}\]

气体的等温变化也可以用图线来表示.用直角坐标系的
横轴表示气体的体积$V$,用纵轴表示气体的压强$p$.设在一定温度下,一定质量的某种气体在$V_1=2$升时,$p_1=1$标准大气压,在图3.6中由$A$点表示.根据玻意耳-马略特定律可以
得出:$V_2=4$升时,$p_2=1/2$标准大气压,由$B$点表示;$V_3=1$升时,$p_3=2$标准大气压,由$C$点表示;$V_4=1/2$升时,$p_4=4$标准大气压,由$D$点表示.当然还可以使气体的体积等于其他许多不同的数值,并计算出相应的压强的数值,从而得到其他许多点.由这些点连成的平滑曲线,叫做气体的等温线.从等温线可以清楚地看出温度不变时气体的压强跟体积的关系.
\begin{figure}[htp]\centering
	\begin{tikzpicture}[>=stealth, domain=.45:4.5, samples=100]
\draw[<->] (0,5)node [right]{$p$(标准大气压)}--(0,0)--(5,0) node [right]{$V$(升)};

\foreach \x in {1,2,3,4}
{
   \draw (0,\x)node [left]{\x}--(.1,\x);
\draw (\x,0)node [below]{\x}--(\x,.1);
}

\draw[color=black, ultra thick] plot (\x,{2/\x}) ;

\draw [dashed] (1,0)--(1,2)--(0,2);
\draw [dashed] (2,0)--(2,1)--(0,1);
\draw [dashed] (4,0)--(4,.5)--(0,.5);
\draw [dashed] (0.5,0)--(.5,4)--(0,4);

\draw [fill=black] (1,2)node [right]{$C$}  circle (2pt);
\draw [fill=black] (2,1)node [above]{$A$}  circle (2pt);
\draw [fill=black] (4,.5)node [above]{$B$}  circle (2pt);
\draw [fill=black] (.5,4)node [right]{$D$} circle (2pt);

\node at (-.2,-.2){$0$};

\end{tikzpicture}
	\caption{气体等温变化的图线}
\end{figure}

\begin{example}
某个容器的容积是10升,所装气体的压强是$20\times 10^5$帕,如果温度保持不变,把容器的开关打开以后,容器
里剩下的气体是原来的百分之几?设大气压是$1.0\times 10^5$帕.
\end{example}

\begin{solution}
这个题目可以这样来分析.容器里装着一定质量的气体,取这一定质量的气体作为我们的研究对象.气体在初状态时,$p_1=20\times 10^5$帕,$V_1=10$升,打开开关以后,由于气体压强大于外界大气压,于是气体发生膨胀(等温膨胀),有一部分
气体跑出容器,随着气体的膨胀,气体的压强降低,最后,当气体压强等于外界大气压时,气体停止膨胀而达到末状态.这时,气体一部分在容器内,一部分在容器外.如果我们知道气体在末状态时占有多大体积,就可以知道容器里剩下的气体为原来的百分之几.

气体在初状态时,$p_1=20\times 10^5$帕,$V_1=10$升,在末状态时,$p_2=1.0\times 10^5$帕,$V_2$为待求的体积.由玻意耳-马略特定律$p_1V_1=p_2V_2$得到
\[V_2=\frac{p_1V_1}{p_2}=\frac{20\times 10^5\times 10}{1.0\times 10^5}{\rm L}=200{\rm L} \]

这时容器里剩下10升气体,所以剩下的气体是原来气体的$10{\rm L}/200{\rm L}=5\%$.
\end{solution}

从这里我们看到,利用玻意耳-马略特定律来解题,先要明确研究对象以及它的初末两个状态,然后才能利用公式来求解,用玻意耳-马略特定律解题时,还要注意等式两边的$p$或$V$必须采用相同的单位,至于具体采用什么单位,可以根据解题方便来决定.

玻意耳-马略特定律表示一定温度下气体的压强跟体积的关系,因此我们可以预料这个定律的表达式
\[pV=\text{恒量} \]
中的恒量跟温度有关系.实验表明,温度不同,这个恒量也不同,图3.7中画出了不同温度下的几条等温线,从此可以知道:一定质量的气体,保持它的体积不变,温度越高,压强越大;保持它的压强不变,温度越高,体积越大,可见,表达式中的恒量随温度而增大.这样看来,要确定体积、压强、温度这
三个物理量的变化规律,我们还需要研究压强怎样随着温度而变化或者体积怎样随着温度而变化.
\begin{figure}[htp]\centering
	\includegraphics[scale=1]{fig/3-7.pdf}
	\caption{不同温度下的几条等温线}
\end{figure}



玻意耳-马略特定律是在压强不太大(和大气压比较)、温度不太低(和室温比较)的条件下总结出来的.在这种条件下,不论什么气体都近似地符合这个定律.当压强很大、温度很低时,由这个定律得出的结果跟实际测量的结果有很大差别,这个定律就不适用了.举例来说,有一定质量的氦气,压强为1标准大气压时,体积为1${\rm m^3}$.压强为500标准大气压时,按照玻意耳-马略特定律体积应该是1/500${\rm m^3}$,而实际测量的结果是1. 36/500${\rm m^3}$,二者之间已经显示出不小的差别.压强为1000标准大气压时,按照玻意耳-马略特定律体积应该是1/1000${\rm m^3}$,而实际测量的结果是2.0685/1000${\rm m^3}$,二者相差一倍多,很本无法应用玻意耳-马略特定律了.


\subsection*{练习二}
\begin{enumerate}
		\item 把打气筒的出口堵住,往下压打气筒的活塞,会感到越往下压越费劲,怎样解释这个现象?
	\item 某个容器的容积是5升,里面所装气体的压强是10标准大气压,如果温度保持不变,把容器的开关打开以后,这些气体会有多大体积?容器里剩下的气体是原来的百分之几?设外界压强为1标准大气压.
	\item 在上题里,打开容器的开关以后,气体的密度怎样改变?设上题里容器里剩下的气体的密度是$\rho_2$,原来容器里气体的密度是$\rho_1$,那么,密度之比$\rho_2/\rho_1$是多大?
	\item 在密闭圆筒的中央有一个活塞(图3.8),活塞两边
	封闭着两部分气体,它们的压强都是750毫米汞柱,现在用
	力把活塞向右移动,使活塞右边气体的体积为原来的一半,那么活塞两边气体的压强差是多大?假定气体的温度不变.
\begin{figure}[htp]\centering
	\begin{tikzpicture}[>=latex]
\draw [pattern=north east lines](2.4,0) rectangle (2.6,2);	
  \pgfsetlinewidth{4pt}
\pgfsetinnerlinewidth{3pt}
\draw  (0,0) rectangle (5,2);
\end{tikzpicture}
\caption{}
\end{figure}

	\item 在图3.2中,水银柱的长度为19厘米,大气压为760毫米汞柱,玻璃管是粗细均匀的,玻璃管开口向上竖直放置时,被封闭的气体柱长15厘米,当开口向竖直放置时,被封闭的气体柱的长度是多少?
	\item 在下端封闭的竖直玻璃管里有一段4厘米长的水银柱,水银柱下面封闭着6${\rm cm^3}$的空气,玻璃管的横截面积是0.1${\rm cm^2}$.如果再向管里装入27.2克水银,那么,封闭在水银
 柱下面的空气柱有多高?设大气压为760毫米汞柱.
	\item 一个足球的容积是2.5升.用打气筒给这个足球打气,每打一次就把1标准大气压的空气打进去125${\rm cm^3}$,如果足球在打气前内部没有空气,打了40次以后,足球内部空气的压强有多大?假定空气的温度不变.
\end{enumerate}

\section{气体的等容变化~~查理定律}
气体在体积不变的情况下发生的变化叫做等体积变化,也叫\textbf{等容变化}.现在我们用实验来研究一定质量的气体,在体积保持不变的情况下它的压强怎样随着温度而变化.
\begin{figure}[htp]\centering
\includegraphics[scale=.6]{fig/3-9.png}
\caption{}
\end{figure}

实验装置如图3.9所示,烧瓶上连一根玻璃管,用橡皮管把它跟水银压强计连在一起,这样便在烧瓶中封入了一定质量的气体.调节压强计的可动管$A$,使两管中的水银面一样高,这时瓶里气体的压强就等于当时的大气压强(图甲),用记号标出$B$管中水银面的位置.

把烧瓶放进盛着冰水混合物的容器里,经过一段时间,瓶里气体的温度跟冰水混合物的温度一样,等于0$^\circ$C.调节压强计的$A$管,使$B$管中水银面恢复到原先标出记号的位置,也就是使气体恢复原来的体积.从压强计$B$管的水银面比$A$管的水银面高可以知道,气体压强减小了(图乙).

把烧瓶放进盛有热水的容器中,调节压强计的$A$管,使$B$管中水银面恢复到原先标出记号的位置,使气体恢复原来的体积,从压强计$B$管的水银面比$A$管的水银面低可以知道,这时气体压强增大了(图丙).

实验表明,在保持气体的体积不变的情况下,一定质量气体的压强随度的升高而增大.

1787年法国科学家查理(1746—1823)通过实验研究,发
现所有气体都遵从下述规律:

\textbf{一定质量的气体,在体积保持不变的情况下,温度每升
高(或降低)$1^\circ{\rm C}$,增加(或减小)的压强等于它在0$^\circ$C时压强的$1/273$.}这就是\textbf{查理定律}.

设一定质量的某种气体,在体积不变的条件下,0$^\circ$C时的压强为$p_0$, $t^\circ$C时的压强为$p_t$.温度升高$t^\circ$C增加的压强为$p_t-p_0$,每升高1$^\circ$C增加的压强$(p-t-p_0)/t$等于$p_0$的$1/273$,即
\[\frac{p_t-p_0}{t}=\frac{p_0}{273} \]
整理后得到
\[p_1=p_0 \left(1+\frac{t}{273}\right)\]
这就是查理定律的数学表达式.

查理定律也可以用图线来表示.用直角坐标系的横轴表示气体的温度$t$,纵轴表示气体的压强$p$.査理定律表明压强是温度的一次函数,而一次函数的图线是一条倾斜的直线,它在纵轴上的截距等于0$^\circ$C时的压强$p_0$,如图3.10所示.

查理定律也是在压强不太大、温度不太低的条件下总结出来的,在这种条件下,不论什么气体都近似地符合这个定律,当压强很大、温度很低时,每升高1$^\circ$C增加的压强不再等于$p_0$的$1/273$,而且这个数值对不同的气体也不再相同.这时查理定律就不适用了.


\section{热力学温标}
图3.10表明了气体的压强跟温度之间的关系,我们看到,图中的直线并未通过原点,说明气体的压强不是直接与摄氏温度成正比的.但是如果我们改用一个新的温标,那就可以得到压强和温度之间的简单的正比关系.

\begin{figure}[htp]\centering
	\begin{tikzpicture}[>=stealth ]
\draw[<->] (0,6.5)node [right]{$p$}--(0,1)--(6,1) node [right]{{$t^\circ$C}};

\draw[color=black, ultra thick] (0,3)--(5,6+1/3);

\draw [dashed] (1.5,+1)--(1.5,4)--(0,4);
\draw [dashed] (3,+1)--(3,5)--(0,5);
\draw [dashed] (4.5,+1)--(4.5,6)--(0,6);

\node at (-.2,-.2+1){$0$};
\node at (1.5,-.25+1){$t_1$};\node at (3,-.25+1){$t_2$};\node at (4.5,-.25+1){$t_3$};
\node at (-.25,4){$p_1$};\node at (-.25,5){$p_2$};\node at (-.25,6){$p_3$};
\node at (-.25,3){$p_0$};
\end{tikzpicture}
	\caption{气体等容变化的图线}
\end{figure}

把图3.10中的直线向左方延长,交横轴于$D$点(图3.11),$D$点表示气体的压强等于零时的温度.这个温度是多少度呢?

设$p_t=p_0 \left(1+\frac{t}{273}\right)=0$,由于$p_0\ne 0$,所以必须要求$1+\frac{t}{273}=0$,由此得出$t=-273^\circ{\rm C}$.

精确的实验证明,上节查理定律数学表达式中的273应该是273.15.这样,气体压强等于零时的温度就应该是$-273.15^\circ{\rm C}$.

\begin{figure}[htp]\centering
	\begin{tikzpicture}[>=stealth, scale=.7 ]
\draw[<->] (0,6.5)node [right]{$p$}--(0,0)--(6,0) node [right]{{$t^\circ$C}};
\draw(-6,0)--(0,0);


\draw[color=black, ultra thick] (0,2.5)--(5,5);
\draw [dashed] (0,2.5)--(-5,0);
\node at (0,-.4){$O$}; \node at (-5,-.4){$-273$};
\node at (-5,.4){$D$};
\end{tikzpicture}
	\caption{}
\end{figure}

英国科学家威廉·汤姆孙(开尔文)(1824—1907)创立了把$-273.15^\circ{\rm C}$作为零度的温标,叫做\textbf{热力学温标}(或\textbf{绝对温标}),用热力学温标表示的温度叫做\textbf{热力学温度}(或\textbf{绝对温度}).

热力学温度是国际单位制中七个基本量之一,用符号$T$表示,它的单位是开尔文,简称为开,国际符号为K.热力学温度的零度是$-273.15^\circ{\rm C}$,叫做绝对零度.就每一度的大小来说,热力学温度和摄氏温度是相同的,所以热力学温度跟摄氏温度间的关系为
\[T=t+273.15 \]
为了简化,可以粗略地取$-273^\circ{\rm C}$为绝对零度,这样就有
\[T=t+273 \]
例如,在1标准大气压下,冰的熔点为0$^\circ{\rm C}$即273K,水的沸点为100$^\circ{\rm C}$即373K.

利用热力学温标可以使查理定律的表述简化.设在体积不变的情况下,一定质量的气体温度为$t_1$时压强为$p_1$,温度为$t_2$时压强为$p_2$,那么
\[\begin{split}
p_1&=p_0\left(1+\frac{t_1}{273}\right)=p_0\frac{273+t_1}{273}\\
p_2&=p_0\left(1+\frac{t_2}{273}\right)=p_0\frac{273+t_2}{273}\\
\end{split} \]
其中$p_0$表示0$^\circ$C时的压强.把上面两式相除得到
\[\frac{p_1}{p_2}=\frac{273+t_1}{273+t_2} \]
用热力学温度$T_1$和$T_2$分别代换$(273+t_1)$和$(273+t_2)$,得到
\[\frac{p_1}{p_2}=\frac{T_1}{T_2} \]
可见查理定律可以表述为:\textbf{体积不变时,一定质量的气体的压强跟热力学温度成正比}.

上面是把查理定律“外推”到零压强而引入热力学温标的,这种“外推”是可以理解的.随着温度的降低,气体分子热运动减弱,分子对器壁的撞击作用也减弱,因而压强减小.由此推想,在某一个温度下,气体压强变为零,这个温度就是绝对零度.实际上,在达到绝对零度之前,任何气体都已液化甚至变为固体,查理定律早已不适用了.虽然如此,由“外推”得到的绝对零度仍具有物理意义,它是低温的极限,能够无限接近,但不可能达到.

\subsection*{练习三}

\begin{enumerate}
	\item 炎热的夏天,打足了气的自行车胎在日光曝晒下有时会胀破,解释这个现象.
\item 乒乓球挤瘪后,放在热水里泡一会,会重新鼓起来.解释这个现象.
\item 一定质量的氢气在0$^\circ$C时的压强是700毫来汞柱,它在30$^\circ$C时的压强是多大?压强为650毫米汞柱时它的温度
是多少摄氏度?保持氢的体积不变.
\item 一定质量的某种气体,在20$^\circ$C时的压强是$1.0\times 10^5$帕,如果保持它的体积不变,温度升高到50$^\circ$C时,它的压强是多大?温度降低到$-7^\circ$C时,它的压强又是多大?
\item 盛有氧气的钢筒,在室内(室温是17$^\circ$C)测得筒内气体的压强是$9.31\times 10^8$帕,当钢筒搬到温度是$-$13$^\circ$C的工地时,筒内气体的压强变为$8.15\times 10^8$帕.钢筒是不是漏了气?为什么?
\item 装在容器中的气体,体积为4升,压强为$2.0\times 10^5$帕,温度为300开,先让气体发生等容变化,压强增大为原来的2倍,然后让气体发生等温变化,压强又降低到原来的数值.求气体在末状态时的体积和温度.

\end{enumerate}

\section{理想气体的状态方程}
前面我们研究了一定质量的气体在温度不变时压强跟体积的关系以及体积不变时压强跟温度的关系,分别得出了玻意耳-马略特定律和查理定律.现在我们从这两个实验定律出发,确定一定质量的气体的体积、压强、温度这三个状态参量在变化中的相互关系.

设有一定质量的气体,在初状态时的压强、体积和温度分别为$p_1,V_1,T_1$,经过某个变化过程,到末状态时这三个量分别变成$p_2,V_2,T_2$.气体从初状态到末状态可以经过各种不同的变化过程,现在设想有一个变化过程是分两个阶段进行的.在第一个阶段中,保持温度$T_1$不变,体积从$V_1$变成$V_2$,压强从$p_1$变成另一个值$p_2$(图3.12中的1和2).在第二个阶段
中,保持体积$V_2$不变,温度从$T_1$变成$T_2$,压强从$p_c$变成$p_2$(图3.12中的2和3).

\begin{figure}[htp]\centering
	\begin{tikzpicture}[>=stealth, xscale=2, yscale=.8 ]
\draw [rounded corners=0.2cm, very thick](-3,2)--(-3,-2)--(-1.5,-2)--(-1.5,2);
\draw [pattern=north east lines](-3,0) rectangle (-1.5,0.5);	
\node at (-4.5/2, -1){$p_1\quad V_1\quad T_1$};
\node at (-4.5/2, -2.5){1};
\draw[->](-1.4,0)--node [above]{等温}(-.6,0);



\draw [rounded corners=0.2cm, very thick](-.5,2)--(-.5,-2)--(1,-2)--(1,2);
\draw [pattern=north east lines](-.5,-1) rectangle (1,-0.5);	
\node at (.5/2, -1.5){$p_c\quad V_2\quad T_1$};
\draw [fill=gray!30](0.2,-.5) rectangle(0.4,.5);
\node at (.5/2, -2.5){2};
\draw[->](1.1,0)--node [above]{等容}(1.9,0);

\draw [rounded corners=0.2cm, very thick](2,2)--(2,-2)--(3.5,-2)--(3.5,2);
\draw [pattern=north east lines](2,-1) rectangle (3.5,-0.5);
\draw [fill=gray!30](2.5,-.5) rectangle(2.7,.5);
\draw [fill=gray!30](2.9,-.5) rectangle(3.1,.5);	
\node at (5.5/2, -1.5){$p_2\quad V_2\quad T_2$};
\node at (5.5/2, -2.5){3};


\end{tikzpicture}
	\caption{}
\end{figure}

第一个阶段是等温变化,根据玻意耳-马略特定律有
\begin{equation}
p_1V_1=p_2V_2
\end{equation}
第一个阶段是等容变化,根据查理定律有
\begin{equation}
\frac{p_c}{T_1}=\frac{p_2}{T_2}
\end{equation}
由(3.1)式解出$p_c$,代入(3.2)式,整理后得到
\begin{equation}
\frac{p_1V_1}{T_1}=\frac{p_2V_2}{T_2}
\end{equation}

上式说明,一定质量的气体从初状态$(p_1,V_1,T_1)$变到末状态$(p_2,V_2,T_2)$,压强和体积的乘积与热力学温度的比值是不变的,即
\begin{equation}
\frac{pV}{T}=\text{恒量}
\end{equation}

我们知道,玻意耳-马略特定律和查理定律是在压强不太大、温度不太低的条件下总结出来的.在这种条件下,不论什么气体都近似地符合这两个实验定律.(3.4)式是从上述两个实验定律推导出来的,因此,也只有在这种条件下,不论什么

气体才近似地符合(3.4)式.尽管如此,为了研究的方便,我们还是可以设想出一种气体,能够在任何温度和压强下都遵守(3.4)式,这样的气体叫做\textbf{理想气体}.(3.4)式叫做一定质量的理想气体的状态方程.

当然,理想气体是不存在的,它只是实际气体在一定程度上的近似.有许多实际气体,特别是那些不易液化的气体,如氢气、氧气、氮气、空气、氦气等,在通常的温度和压强下,它们的性质很近似于理想气体,可以把它们当作理想气体来处理.这样处理的结果,误差很小,可是计算起来却简便多了.

理想气体状态方程实际上包含了前面讲的两个气体实验
定律,如果保持温度$T$不变,便得到$pV=\text{恒量}$,这就是玻意耳-马略特定律.如果保持体积$V$不变,便得到$p/T=\text{恒量}$,这就是查理定律.

从理想气体状态方程还可以知道,压强保持不变时,一定质量气体的体积怎样随着温度而变化,这种变化叫做\textbf{等压变化}.在保持压强$p$不变时,得到$V/T=\text{恒量}$,这表示\textbf{压强不变时,一定质量气体的体积跟热力学温度成正比}.这个关系最初是法国科学家盖·吕萨克(1778—1850)研究气体热膨胀时得到的实验定律,叫做\textbf{盖·吕萨克定律}.在压强不太大、温度不太低时,不论什么气体都近似地符合这个定律.


\subsection*{练习四}

\begin{enumerate}
	\item 对一定质量的气体来说,能否做到:
	\begin{enumerate}
	\item	保持压强和温度不变而改变它的体积?
	\item	保持温度和体积不变而改变它的压强?
	\item	保持体积和压强不变而改变它的温度?	 
	\end{enumerate}
\item  对一定质量的气体来说,能否做到:
\begin{enumerate}
\item 保持压强不变,同时升高温度并减小体积?
\item 保持温度不变,同时增加体积并减小压强?
\item 保持体积不变,同时增加压强并降低温度?
\end{enumerate}

\item  一定质量的空气,27$^\circ$C时的体积为$1.0\times 10^{-2}{\rm m}^3$.计算在压强不变的情况下,温度升高到100$^\circ$C时的体积.
\item  某种柴油机的气缸容积为$0.83\times 10^{-3}{\rm m^3}$.压缩前其中空气的温度为47$^\circ$C,压强为$0.8\times 10^5$帕,在压缩冲程,活塞把空气压缩到原体积的1/17,压强增大到$40\times 10^5$帕.求这时空气的温度.
\item  在容积为25升的容器中,盛有温度为37$^\circ$C、压强为62标准大气压的氧气.求氧气在标准状态(0$^\circ$C,1标准大气压)下的体积,从化学课中学过,在标准状态下,1摩的任何气体的体积都是22.4升.你能不能由此求得容器中氧气的摩尔数并进而求得氧气的质量?怎样求?
\item  一个瓶子里装有某种气体,瓶上有一个小孔跟外面大气相通.原来瓶里气体的温度为15$^\circ$C.如果把它加热到207$^\circ$C,瓶里保留的气体的质量是原来质量的几分之几?
\item  贮气筒内装有压缩气体,温度是27$^\circ$C,压强是$40\times 10^5$帕,如果从筒内放出一半质量的气体,并使筒内剩余的气体的温度降到12$^\circ$C,这些剩余气体的压强是多大?

\end{enumerate}


\section{克拉珀龙方程}
\subsection{摩尔气体恒量} 

上一节讲的气体状态方程
\[\frac{pV}{T}=\text{恒量} \]
是一定质量的理想气体状态方程,其中的恒量跟气体的质量有关系.在体积和温度相同的情况下,气体的质量越多,气体的压强就越大,因而上式中的恒量就越大.自行车车胎里打进的空气越多,车胎胀得越硬,这是大家都知道的.

实验表明,上式中的恒量还跟气体的种类有关系.在体积和温度相同的情况下,质量相同的不同种类气体,它们的压强并不相同,因而上式中的恒量也不相同.

那么,怎样在一般情况下应用上式呢?我们先把上式用于质量限定的各种气体,而且质量就限定为1摩尔.这是因为,在标准状态下,即$p_0=1$标准大气压,$T_0=273$开,1摩的任何气体的体积都是$V_0=22.4$升.由此我们可以求得一个适用于1摩尔任何气体的恒量,叫做\textbf{摩尔气体恒量},它通常用$R$来表示,即
\[R_0=\frac{p_0V_0}{T_0} \]

$R$的数值跟$p,V,T$的单位有关.在国际单位制中,$p_0=
1.013\times 10^5{\rm Pa}=1.013\times 10^5{\rm N}/{\rm m^2}$,$V_0=22. 4\times 10^{-3}{\rm m^3}/{\rm mol}$,$T_0=273{\rm K}$,代入上式得到
\[\begin{split}
R&=\frac{1.013\times 10^5{\rm N}/{\rm m^2}\times  22. 4\times 10^{-3}{\rm m^3}/{\rm mol}}{273{\rm K} }\\
&=8.31{\rm J}/({\rm mol\cdot K})
\end{split} \]

对于1摩的理想气体,因为$pV/T=p_0V_0/T_0=R$,所以
$$pV=RT$$
这就是1摩的理想气体的状态方程,它对任何气体都适用.

摩尔气体恒量是热学中又一个重要常数,不仅在研究气
体的热学性质中,而且在研究其他热现象中,它与阿伏伽德罗常数共同起着重要作用.

\subsection{克拉珀龙方程} 

知道了1摩的理想气体的状态方程,我们不难得到任意质量的理想气体的状态方程.设有质量为$m$千克的某种理想气体,它的摩尔质量为$M$千克/摩,它的摩尔数$n=m/M$摩.既然1摩的理想气体在标准状态下占有体积$V_0$($=22.4$升),那么$n$摩的理想气体在标准状态下占有的体积应为$V'_0=nV_0$.由理想气体的状态方程可得:
\[\begin{split}
\frac{pV}{T}&=\frac{p_0V'_0}{T_0}\\
&=n\frac{p_0V_0}{T_0}=nR
\end{split} \]
由此得到
\[pV=nRT \]
或\[pV=\frac{m}{M}RT \]
这就是任意质量的理想气体的状态方程,又叫做\textbf{克拉珀龙方程}.只要温度不太低,压强不太大,这个方程对一切气体都适用,这个方程在实际中有广泛的应用,可以用来解决有关气体的各种问题.

\begin{example}
容积为30升的瓶内装有氢气.假定在气焊过程中,温度保持27$^\circ$C不变,当瓶内压强由$4. 9\times 10^6$帕降为$9.8\times 10^5$帕时,共用去多少氢气?
\end{example}

\begin{solution}
用国际单位制来计算,把已知各个量的数值用相应的单位表示出来.
$p_1=4.9\times 10^6{\rm Pa}$;$V=30{\rm L}=30\times 10^{-3}{\rm m^3}$; $p_2=9.8\times 10^5{\rm Pa}$;$T=(27+273){\rm K}=300{\rm K}$;$M=2\times 10^{-3}{\rm kg}/{\rm mol}$.

这个例题可以这样来解:根据克拉珀龙方程先计算瓶内原有氢气的质量$m_1$,再计算气体状态改变后瓶内氢气的质量$m_2$,二者之差$m_1-m_2$就是用去的氢气的质量.

气体的初状态和末状态的体积$V$和温度$T$保持不变,压强$p$和质量$m$发生了变化,压强由$p_1$变到$p_2$,质量由$m_1$变到$m_2$.

由$p_1V=\dfrac{m_1}{M}RT$ 得到 $m_1=\dfrac{p_1VM}{RT}$
;
由$p_2V=\dfrac{m_2}{M}RT$ 得到 $m_2=\dfrac{p_2VM}{RT}$
.
所以
\[m_1-m_2=\frac{VM}{RT}(p_1-p_2) \]
代入数值得到
\[\begin{split}
m_1-m_2&=\frac{30\times 10^{-3}\times 2\times 10^{-3}}{8.31\times 300}(4.9\times 10^6-9.8\times 10^5){\rm kg}\\
	&=9.4\times 10^{-2}{\rm kg}
\end{split} \]
\end{solution}

如果就一定质量的气体来考虑气体的状态变化,即压强由$p_1$降低到$p_2$,而体积由$V_1$膨胀到$V_2$,能不能解出这个题目?你来试一下,并把两种解法加以比较.

利用理想气体状态方程解题,首先要明确我们所研究的对象是哪部分气体,以及气体状态发生变化时它的初状态和末状态,然后才能用状态方程来求解,计算时要注意物理量的单位,$T$必须采用热力学温度,根据$p_1V_1/T_1=p_2V_2/T_2$解题时,公式两边的$p$和$V$的单位必须统一,根据$pV=mRT/M$解题时,$R$的单位必须与$p$、$V$的单位相适应.

\subsection*{练习五}

\begin{enumerate}
	\item 如果压强用标准大气压作单位,体积用升作单位,试通过计算证明:$R=0.082$标准大气压·升/(摩·开).
\item 一个容器内装有氧气100克,压强为10标准大气压,温度为47$^\circ$C,容器的容积是多少${\rm m^3}$?
\item 1克的气体,温度为27$^\circ$C、压强为600毫米汞柱时,体积为5升,2克的同种气体,温度为127$^\circ$C、压强为400毫米汞柱时,体积是多少升?
\item 容积是10升的钢筒里盛有90标准大气压、$-13^\circ$C的氧气,求钢筒中氧气的质量.已知氧气在标准状态下的密度$\rho_0=1.43{\rm kg}/{\rm m^3}$.
\item 有0.612克的某种氮氧化合物,在293开和1标准大气压时体积为480${\rm cm^3}$,这是一种什么气体?写出它的分子式.
\item 给汽车轮胎打气,使胎内空气达到所需的压强,冬天和夏天打入胎内的空气质量是否相同?冬天还是夏天打入的空气质量多?
\item 有两种不同种类的气体,它们的温度和体积都相同.如果它们的质量也相同,气体的压强是否相同?如果它们的质
量不同,但摩尔数相同,气体的压强是否相同?
\item 理想气体的状态方程可写成$pV/T=C$(恒量),对于这个恒量$C$,下面哪种说法正确,哪种说法错误,并说明理由.
\begin{enumerate}
\item 对质量相同的任何气体,$C$都相同.\item 对质量不同的同种气体,$C$都相同.	\item 对摩尔数不同的同种气体,$C$都相同.\item 对摩尔数相同的任何气体,$C$都相同.
\end{enumerate}
\end{enumerate}

\section{气体分子运动的特点}
\subsection{分子间的距离较大} 
气体很容易被压缩,说明气体分子间的距离比较大,气体凝结成液体时,体积要缩小上千倍,而液体不容易被压缩,可以认为其中的分子几乎是紧密排列的,可见气体分子之间的距离大约是分子直径的$\sqrt[3]{1000}$倍,即10倍.由于气体分子间的距离比较大,所以在处理某些问题时可以把气体分子看作是没有大小的质点.也是由于气体分子间的距离比较大,分子间的相互作用力十分微弱,所以通常可以认为,气体分子除了相互碰撞或者跟器壁碰撞外不受力的作用,可以在空间里自由移动.由此可以说明:气体能充满它所能达到的空间,既没有一定的体积,没有一定的形状.

\subsection{分子间的碰撞频繁} 
比起固体和液体来,气体中的分子是比较稀疏的,但是单位体积中的分子数还相当大.在标准状态下,1${\rm cm^3}$气体中仍含有$2.7\times 10^{10}$个分子,大量分子永不停息地运动,分子之间不断地发生碰撞,在标准状态下,一个空气分子在1秒内与其他空气分子的碰撞竞达65亿次之多.频繁的碰撞使得每个分子的速度的大小和方向频繁地改 
变.设想我们追随某个气体分子的运动(图3.13),我们将看到这个分子的运动是忽左忽右,忽前忽后,时快时慢,运动轨迹是一条极不规则的折线,频繁的碰撞造成气体分子做杂乱无章的热运动.
\begin{figure}[htp]\centering
\includegraphics[scale=.6]{fig/3-13.png}
\caption{}
\end{figure}

通常假定分子之间或分子与器壁之间的碰撞是完全弹性碰撞.

\subsection{分子沿各方向运动的机会均等} 

气体分子做杂乱无章的热运动,就某一个分子来说,它在某一时刻的速度具有怎样的大小和方向,完全是偶然的.但是,对大量分子的整体来说,分子的运动却表现出一定的规律,先来讨论分子运动的方向.正因为大量分子的运动十分混乱,在某一时刻向任一方向运动的分子都有,因而可以想见,在任一时刻分子沿各方向运动的机会是均等的,没有任何一个方向,沿着它运动的分子的数目更多,设想真有这么一个方向,那么,由于气体分子的频繁碰撞,分子的运动越来越混乱,这个方向也不会存在了,这就是说,气体分子沿各个方向运动的数目应该是相等的.

这里所说的数目相等,是对大量分子用统计方法得到的一个统计平均数,与实际数目会有微小的出入,分子数越多,这种用统计方法得到的结果跟实际情况越符合,用分子运动论的观点研究热现象,涉及的总是大量分子,统计方法非常有用.

\subsection{分子速率按一定规律分布} 
大量分子做无规则运动,速率有的大,有的小,但分子的速率却按照一定的规律分布.

研究表明,气体的大多数分子,速率都在某个数值附近,离开这个数值越远,分子数越少,表现出“中间多,两头少”的
分布规律,下表是氧气分子速率的分布情况.我们看到,在0$^\circ$C时速率在300—400$\ms$这一速率区间的分子数最多,速率大于400$\ms$和小于300$\ms$的分子数依次递减,速率很大和很小的分子实际上很少,温度升高时,这种“中间多,两头少”的分布规律虽然不变,可是与分子数的最大值相对应的速率区间却移向速率大的一方,也就是说,温度升高时,速率小的分子数减少,速率大的分子数增加,这种速率分布规律是一种统计规律,表中的在某一速率区间的相对分子数,也是
对大量分子用统计方法得到的统计平均数,与实际数值会有微小的出入.

\begin{table}
	\centering\caption{氧气分子的速率分布}
	\begin{tabular}{ccc}
\hline
按速率大小划分的区间 & \multicolumn{2}{c}{各速率区间的分子数占总分子数的百分率}\\
($\ms$)& 0$^\circ $C & 100$^\circ $C\\
\hline
100以下 & 1.4 & 0.7\\
100—200 &	8.11 &	5.4
\\
200—300 &	7.0	 &11.9
\\
300—400 &	21.4 &	17.4
\\
400—500 &	20.4 &	18.6
\\
500—600 &	15.1 &	16.7
\\
600—700 &	9.2 &	12.9
\\
700—800  &	4.5 &	7.9
\\
800—900 &	2.0 &	4.6
\\
900 以上 &	0.9 &	3.9\\
\hline
	\end{tabular}
\end{table}

既然在一定温度下,某种气体的分子速率分布是确定的,我们就可以求出在这个温度下该种气体分子的平均速率,即所有分子的速率的平均值.温度升高时,速率大的分子数增
加,分子的平均速率增大.例如氮气分子的平均速率在$-150^\circ $C时为320$\ms$,在$0^\circ $C时为493$\ms$,在$1000^\circ $C时
为1194$\ms$.这里我们又一次看到,温度越高,分子的热运动越激烈.

\section{气体实验定律的微观解释}
\subsection{气体压强的微观解释} 
从气体分子运动论的观点看来,气体压强是大量的气体分子频繁地碰撞器壁而产生的.雨滴打在雨伞上,使伞面受到冲力,单个雨滴对伞面的冲力是短暂的,但大量密集的雨滴接连不断地打在伞面上,对伞面就形成一个持续的均匀的压力,同样,单个分子碰撞器壁的冲力是短暂的,但是大量分子频繁地碰撞器壁,就对器壁产生持续的均匀的压力,下面我们从气体分子运动论的观点定性地讨论一下气体的压强.

设想容器内只有一个分子,我们可以利用以前学过的力学知识算出这个分子碰撞器壁时对器壁产生多大的冲力,现在的问题是:容器中有大量分子,它们的速度的大小和方向又不断地在改变.怎样才能算出大量分子碰撞器壁时对器壁产生的冲力呢?

我们知道,气体分子做无规则运动,它们沿各个方向运动的机会是均等的,也就是说,在上下、前后、左右各个方向中没有哪个方向的运动占优势,因此我们可以认为各有1/6的分子向着上下前后左右这六个方向运动,气体分子速度的大小也不相同,但我们可以认为所有分子都以平均速率向着各个
方向运动.

\begin{figure}[htp]\centering
	\begin{tikzpicture}[>=stealth, scale=.9 ]
\draw (5,2)--(5,-2);
\fill [pattern=north east lines](5,-2) rectangle(5.25,2);

\draw [shade](0,0) circle (4mm);
\draw [dashed] (3.5,0) circle (4mm);
\draw[->](0,0)--(1,0) node [above]{$v$};
\draw[->](3.5,0)--(2.5,0)node [above]{$v'=v$};


\end{tikzpicture}
	\caption{气体分子每碰撞一
次器壁,就给器壁$2mv$的冲量.}
\end{figure}



现在设想有一个向右运动的分子与器壁发生碰撞(图3.14),碰撞前的动量是$mv$,其中$v$是分子的平均速率,碰撞后向左运动,速率是$v'$,动量是$-mv'$.这个分子碰撞前后的动量变化是$-mv'-mv$.气体分子与器壁之间的碰撞是完全弹性碰撞,这样,分子碰撞前后的速率相等,即$v'=v$,因而动量变化是$-2mv$.从动量定理知道,这个动量变化$-2mv$等于器壁对分子的冲量.从牛顿第三定律知道,这时分子对器壁也有一个大小相等方向相反的冲量.可见气体分子每碰撞一次器壁,就给器壁$2mv$的冲量.

在一段时间内,大量分子与器壁碰撞多少次,分子给器壁
的总冲量就是$2mv$的多少倍,而在单位时间内分子给器壁的总冲量就等于器壁所受的压力,单位面积器壁所受的压力就等于气体的压强.

这样,如果知道单位时间内分子对单位面积器壁的碰撞次数,就可以求得气体的压强.这个碰撞次数跟单位体积内气体的分子数有关,单位体积内气体的分子数越多,即气体分子越密,这个碰撞次数就越多,这是容易理解的.这个碰
撞次数还跟分子的平均速率有关.如图3.15所示,我们在气体内部设想一个柱体,底面积为单位面积,高为平均速率的数值.在单位时间内,这个柱体中向右运动的分子都会运动到器壁而发生碰撞.平均速率越大,这个柱体越高,其中的分子越多,分子与器壁发生碰撞的次数就越多.可见,单位时间内分子对单位面积器壁的碰撞次数是由单位体积内的分子数和分子的平均速率决定的.由此我们将不难理解气体压强也是由单位体积内的分子数和分子的平均速率决定的,单位体积内的分子数越多,分子的平均速率越大,气体的压强就越大.
\begin{figure}[htp]\centering
	\begin{tikzpicture}[>=stealth, scale=.9 ]
\draw (5,2)--(5,-2);
\fill [pattern=north east lines](5,-2) rectangle(5.25,2);

\draw[dashed, pattern=crosshatch dots](1,-.3) rectangle (5,.3);
\draw(1,-.3)--(1,-1);
\draw[<->](1,-.65)--node [below]{$v$}(5,-.65);

\end{tikzpicture}
	\caption{单位时间内分子对单位
面积器壁的碰撞次数跟
分子的平均速率有关.}
\end{figure}

\subsection{气体实验定律的微观解释} 
知道了气体压强是由什么决定的,我们就能够用气体分子运动论对气体实验定律作出微观解释.

一定质量的气体,温度保持不变,也就是分子的总数和分
子的平均速率保持不变.在这种情况下,气体的体积减小到原来体积的几分之一,单位体积内的分子数就增大到原来的几倍,气体的压强就增大到几倍,气体体积增大时,情况恰好相反,结果是气体的压强与体积成反比,这就是玻意耳-马略特定律.

用气体分子运动论也可以解释查理定律,一定质量的气体,体积保持不变而温度升高时,分子的平均速率增大,因而气体的压强增大,温度降低时,情况恰好相反.

怎样解释盖·吕萨克定律呢?从气体分子运动论可以说明:一定质量的气体温度升高时,要保持压强不变,只有让气体的体积增大才行.这时,一方面由于温度升高,分子的平均速率增大,以致每次碰撞给器壁的冲量增加,同时单位时
间内对单位面积器壁的碰撞次数增多,使压强有增大的倾向;另一方面,由于体积增大,单位体积内的分子数减少,以致单位时间内分子对单位面积器壁的碰撞次数减少,使压强有减小的倾向,当体积增大到一定程度时,这两种倾向抵消,所以压强保持不变.

气体分子运动论不仅能够解释上述气体实验定律,而且能够解释气体的其他一些性质,如气体的比热、扩散、热传导等,气体分子运动论是热学和分子物理学的重要组成部分,它使人们对气体的研究从宏观领域进入微观领域,扩展和加深了人们对气体性质的认识.


\subsection*{练习六}

\begin{enumerate}
	\item 现在我们用另一种方法估算一下气体分子间的距离与分子直径的关系.在标准状态下,1摩的气体占有22.4升的体积.我们设想其中的每个分子都位于一个小立方体的中心.这个小立方体的边长是多少?分子直径的数量级为$10^{-10}$米.把小立方体的边长跟分子直径相比较,结果怎样?
\item 根据第七节表中的数据能不能估算出0$^\circ$C和100$^\circ$C时氧气分子的平均速率?怎样估算?结果怎样?
\end{enumerate}

\section{理想气体的内能}
从气体分子运动论的观点看来,所谓理想气体,是指分子间没有相互作用和分子可以看成没有大小的质点的气体.这就是理想气体的微观模型,一定质量的气体,温度越高,压强
越小,因而气体越稀薄,气体分子间的距离越大,就越接近于理想气体.在温度较低和压强较大的情况下,气体不那么稀薄,在研究气体的性质时必须考虑到分子的大小和分子间的相互作用,而它们跟气体的种类有关,这时气体不再遵守理想气体状态方程,并且显示出不同气体在性质上的差异.

理想气体的分子之间既然没有相互作用,就不存在分子势能,因此,理想气体的内能就是气体所有分子热运动的动能的总合,分子的动能跟气体的温度有关,分子势能跟气体的体积有关,现在不存在分子势能,因而理想气体的内能只跟温度有关,跟体积无关.这就是说,只要温度保持不变,气体的体积增大一些因而气体分子疏一些,或者气体的体积减小一些因而气体分子密一些,不仅分子动能保持不变,分子势能仍旧不存在,因此理想气体的内能保持不变.

\section{理想气体的内能变化}
前面我们讲过了理想气体的等温、等容和等压变化,现在我们分析一下在这三种等值变化中内能的变化.

设一定质量的理想气体在温度不变的情况下发生膨胀,由初状态变到末状态,由于温度保持不变,所以气体的内能也不变,即$\Delta E=0$.气体发生膨胀时对外做功,所以$W$为负值,即$W<0$.从热力学第一定律$W+Q=\Delta E=0$知道,$Q$应为正值,即$Q>0$,而且$W$和$Q$的绝对值相等.可见,在等温膨胀的过程中,理想气体要从外界吸收热量,吸收的热量并没有增加气体的内能,而全部用来对外做功.

在体积不变的情况下,对一定质量的理想气体加热,使它的温度升高,压强增大,由初状态变到末状态.末状态的温度比初状态高,所以内能增加,即$\Delta E>0$.气体的体积不变,外界既没有对气体做功,气体也没有对外界做功,所以$W=0$.根据热力学第一定律我们得到$Q=\Delta E$.可见,在等容变化中,如果理想气体从外界吸收热量,这个热量就全部用来增加气体的内能.

在压强不变的情况下,对一定质量的理想气体加热,使它的温度升高,体积增大,由初状态变到末状态,末状态的温度比初状态高,所以内能增加,即$\Delta E>0$,气体膨胀对外做功,$W<0$.从热力学第一定律$W+Q=\Delta E>0$知道,这时气体吸收的热量$Q$的绝对值大于$W$的绝对值,这就是说,在等压膨胀的过程中,理想气体从外界吸收的热量,一部分用来增加气体的内能,一部分用来对外做功.

除了上述三种等值变化外,还有一种所谓绝热变化在实际中常常遇到,物体在状态的变化过程中如果跟外界没有热交换,这种变化就叫做\textbf{绝热变化},绝热变化的特点是:$Q=0$.用绝热良好的材料把容器包起来,让气体发生膨胀或者对气体进行压缩,这时的变化就可以看作绝热变化,气体的膨胀或压缩进行得很迅速,从初状态到末状态所用的时间很短,气体来不及跟外界发生热交换,这种迅速的变化也可以看作绝热变化,图2.1所示的压缩气体的演示,热机气缸内气体膨胀做功,过程进行得很迅速,都可以看作绝热变化.在绝热压缩的过程中,外界对气体所做的功完全用来增加气体的内能,使气体的温度升高.在绝热膨胀的过程中,气体对外界做功
完全靠气体内能的减少,因而气体的温度降低.


\subsection*{练习七}
\begin{enumerate}
	\item 一定质量的理想气体在温度不变的情况下被压缩,气体的内能是否改变?外界对气体是否做功?气体从外界吸热还是向外界放热?功和热量有什么关系?
\item 一定质量的理想气体在体积不变的情况下压强减小,这时外界对气体是否做功?气体的内能是否改变,怎样改变?气体放出的热量跟内能的改变有什么关系?
\item 一定质量的理想气体在压强不变的情况下体积减小,外界对气体是否做功?气体的内能是否改变,怎样改变?气体放热还是吸热?这个热量跟内能的改变有什么关系?
\end{enumerate}

\section*{习题}
\begin{enumerate}
	\item 下面几种说法,哪个正确,哪个错误,并说明理由.
	\begin{enumerate}
		\item 有两个相同的容器,内装同种气体,它们的压强相同,因而它们的温度一定相同.
		\item 有两个相同的容器,内装质量相同的不同气体,它们的压强不同,因而它们的温度一定不同.
		\item 有两个相同的容器,内装摩尔数相同的气体,它们的压强相同,因而它们的温度一定相同.
	\end{enumerate}
\item 一定质量的理想气体,处在某一初始状态,现在要使气体的温度经过状态变化后回到初始状态的温度,用下列哪些过程可能实现?
\begin{enumerate}
	\item 先保持压强不变而使它的体积膨胀,接着保持体积不变而减小压强.
\item 先保持压强不变而使它的体积减小,接着保持体积不变而减小压强.
\item 先保持体积不变而增大压强,接着保持压强不变而使它的体积膨胀.
\item 先保持体积不变而减小压强,接着保持压强不变而使它的体积膨胀.	 
\end{enumerate}
\item 盖·吕萨克定律如果用摄氏温标$t$来表示,可以写成下式:
\[V_t=V_0\left(1+\frac{t}{273}\right) \]
其中$V_0$和$V_t$分别表示气体在0$^\circ$C和t$^\circ$C时的体积.试推导出上式.
\item 能不能根据玻意耳-马略特定律和盖·吕萨克定律推出一定质量的理想气体的状态方程$PV/T=$恒量?实际推导一下.
\item 当温度为27$^\circ$C、压强为$2.0\times 10^5$帕时,32克氧气的体积是多大?密度是多大?另有48克氧气,温度和压强跟上述数值相同,氧气的密度又是多大?	
\item   试根据克拉珀龙方程推导出用压强和温度来表示的气体密度的表达式.
\item  水银气压计中混入了一个空气泡,上升到水银柱的上方,使水银柱上方不再是真空,因而气压计的读数比实际的大气压小些,当实际大气压为768毫米汞柱时,气压计的读数只有750毫米汞柱,此时管中水银面到管顶的距离为80毫米,当气压计读数为740毫米汞柱时,实际大气压为多少?设温度保持不变.
\item  在湖面下50米深处(温度为4$^\circ$C)有一个体积为10
${\rm cm^3}$的气泡升到湖面上来,湖面的温度为17$^\circ$C,求它升到湖面时的体积,大气压强为$1.013\times 10^5$帕.
\item  有两个容积相等的器,里面盛有同种气体,用一段水平玻璃管把它们连结起来.在玻璃管的正中央有一段水银柱,当一个容器中气体的温度是0$^\circ$C,另一个容器中气体的温度是20$^\circ$C时,水银柱保持静止,如果使两容器中气体的温度都升高10$^\circ$C,管中的水银柱会不会移动?如果移动的话,向哪个方向移动?试根据学过的气体定律加以说明.
\item  一个容器,如果其中气体十分稀薄,通常就说这个容器为“真空”.有一个容积为10${\rm cm^3}$的电子管,在温度为300开时用真空泵把它抽成真空,使管内气体压强为 $5\times 10^{-8}$毫米汞柱,这时管内有多少个气体分子?
\item  氧气瓶的容积是32升,其中氧气的压强是130标准大气压.规定瓶内氧气压强降到10标准大气压时就要重新充氧.有一个车间,每天需用1标准大气压的氧气400升.这瓶氧气能用几天?假定温度保持不变.
\begin{figure}[htp]\centering
	\begin{tikzpicture}[>=stealth, scale=.7 ]


\draw[very thick] (-2, -1.5)--(2, -1.5);
\draw[very thick] (-2, 1.5)--(2, 1.5);
\draw[very thick] (2,.2)--(5,.2); \draw[very thick] (2,-.2)--(5,-.2);
\draw[very thick] (5, 2.5)--(10, 2.5); \draw[very thick] (5, -2.5)--(10, -2.5);
\draw[very thick] (-2, 1.5)--(-2, -1.5);
\draw[very thick] (2, 1.5)--(2, .2);
\draw[very thick] (2, -.2)--(2, -1.5);
\draw[very thick] (5, 2.5)--(5, .2);
\draw[very thick] (5, -.2)--(5, -2.5);
\draw[dashed](-3,-2.5) rectangle (3,2.5);

\fill [pattern=north east lines, draw](8,-2.5) rectangle(8.5,2.5);
\node at (0,2){127$^\circ$C}; \node at (9,2){$D$}; 
\node at (0,0){$B$};  \node at (7,0){$A$};

\draw [->](10,0)node [below]{$p_0$}--(8.5,0);

\draw(3.5,-.5)--(3.5,.5);
\draw(3.2,.5)--(3.7,.5)node[right]{$K$};

\end{tikzpicture}
\caption{}
\end{figure}


\item  如图3.16所示,气缸$A$和容器$B$由一细管经阀门
$K$相联.$A$和$B$的壁都是透热的.$A$放在27$^\circ$C、1标准大气压的大气中,$B$浸在127$^\circ$C的恒温槽内.开始时$K$是关断的,
$B$内没有气体,容积$V_B=2.4$升;$A$内装有气体,体积$V_A=
4.8$升.打开$K$,使气体由$A$流入$B$,等到活塞$D$停止移动
时,$A$内气体的体积是多大?假设活塞$D$与气缸壁之间没有摩擦,细管的容积忽略不计.
\end{enumerate}















\chapter{固体和液体的性质}
\section{晶体和非晶体}
固体可以分成晶体和非晶体两类.在常见的固体中,石英、云母、明矾、食盐、硫酸铜等都是\textbf{晶体},玻璃、蜂蜡、松香、沥青、橡胶等都是\textbf{非晶体}.晶体和非晶体在外形上和物理性质上都有很大区别.

晶体具有天然的规则的几何形状,它的外形是由若干个平面围成的多面体.例如食盐的晶体是立方体(图4.1甲),明矾的晶体是八面体(图4.1乙),石英的晶体(透明的石英晶体叫水晶)中间是六面棱柱,两端是六面棱锥(图4.1丙),冬季的雪花是水蒸气在空气中冻结时形成的冰的晶体,它们的形状虽然不同,但都呈六角形的规则图案.非晶体则没有规则的外形.
\begin{figure}[htp]
\centering
\includegraphics[scale=1.1]{fig/4-1.pdf}
\caption{晶体的外形}
\end{figure}


从实验知道,晶体在不同方向上的物理性质(力学性质、热学性质、电学性质、光学性质等)是不同的,这种现象叫做晶体的各向异性.以导热性为例,我们来看下面的实验.在一张云母片上涂上很薄一层石蜡,然后用烧热的钢针的针尖
接触云母片,接触点周围的石蜡就熔化了,而熔化了的石蜡成椭圆形(图4.2).这表明云母晶体在不同方向上的导热性是
不同的,用玻璃板代替云母片重做上面的实验,熔化了的石蜡则成圆形(图4.3),表明非晶体玻璃在不同方向上的导热性是相同的,即各向同性.各向异性是晶体区别于非晶体的一个基本特征,我们可以借助于物体是否具有各向异性来判
断它是不是晶体.

\begin{figure}[htp]\centering
	\begin{tikzpicture}[>=stealth, scale=1 ]
\fill [gray](-3,-2) rectangle(3,2);
\draw [fill=white] (0,0) ellipse [x radius=2, y radius=1];

\end{tikzpicture}
\caption{云母片上熔化了的石蜡成椭圆形}
\end{figure}

\begin{figure}[htp]\centering
	\begin{tikzpicture}[>=stealth, scale=1 ]
\fill [gray](-3,-2) rectangle(3,2);
\draw [fill=white] (0,0) circle [radius=1];

\end{tikzpicture}
\caption{玻璃板上熔化了的
石蜡成圆形}
\end{figure}

晶体可以分为单晶体和多晶体,如果整个物体就是一个晶体,这样的物体就叫做单晶体.上面说的晶体就是指单晶体.单晶体是科学技术上的重要原材料,制造各种晶体管就要用纯度很高的单晶硅或单晶锗.

如果整个物体是由许多杂乱无章地排列着的小晶体(晶粒)组成的,这样的物体就叫做\textbf{多晶体}.平常见到的各种金属材料就是多晶体.把纯铁做成的样品放在显微镜下观察,可以
看到它是由许许多多晶粒组成的.晶粒有大有小,最小的只有$10^{-5}$厘米那样大,最大的也超不过$10^{-3}$厘米.每个晶粒都是一个小单晶体,具有各向异性.由于晶粒在多晶体里杂乱无章地排列着,所以多晶体没有规则的几何形状,也不显示各向异性,它在不同方向上的物理性质是相同的,即各向同性.

\section{空间点阵}
十九世纪中叶,人们根据晶体外形的规则性和各向异性提出了一种假说,认为晶体内部的微粒是有规则排列着的,从1912年开始的应用X射线对晶体结构进行的研究,证实了这种假说的正确.现在,人们用电子显微镜对晶体内部结构进行直接观察和照相,进一步证实了这种假说的正确.

组成晶体的物质微粒(分子、原子或离子)依照一定的规律在空间中排成整齐的行列,构成所谓\textbf{空间点阵}.如果沿着这些物质微粒的行列画出直线来,可以得到若干组平行线,物质微粒就在这些组平行线的交点上.这些交点叫做空间点阵的
\textbf{结点}.

晶体中物质微粒的相互作用是很强的,微粒的热运动不足以克服它们的相互作用而远离,因而形成了空间点阵的结构.微粒的热运动主要表现为以结点为平衡位置不停地做微小的振动.
\begin{figure}[htp]\centering
	\begin{tikzpicture}[>=stealth, scale=1.5 ]
%\draw [->](0,0,0)--(0,0,10)node [right]{$z$};
%\draw [->](0,0,0)--(0,10,0)node [right]{$y$};
%\draw [->](0,0,0)--(10,0,0)node [right]{$x$};

%\draw (0,0,0) circle (2pt);
%\draw (0,0,2) circle (2pt);

\draw (0,0,0)--(0,0,2);
\draw (0,0,0)--(0,2,0);
\draw (0,0,0)--(2,0,0);
\draw (0,1,0)--(0,1,2);
\draw (0,1,0)--(2,1,0);
\draw (0,2,0)--(0,2,2);
\draw (0,2,0)--(2,2,0);
\draw (1,1,0)--(1,1,2);
\draw (1,2,0)--(1,2,2);
\draw (2,1,0)--(2,1,2);
\draw (2,2,0)--(2,2,2);
\draw (1,0,0)--(1,0,2);
\draw (1,0,0)--(1,2,0);
\draw (2,0,0)--(2,0,2);
\draw (2,0,0)--(2,2,0);
\draw(0,2,1)--(2,2,1);
\draw(0,2,2)--(2,2,2);
\draw(0,1,1)--(2,1,1);
\draw(0,1,2)--(2,1,2);
\draw (0,2,2)--(0,0,2);
\draw (2,0,2)--(0,0,2);
\draw (2,0,1)--(0,0,1);
\draw (2,0,2)--(2,2,2);
\draw (2,0,1)--(2,2,1);
\draw (1,0,1)--(1,2,1);
\draw (1,0,2)--(1,2,2);
\draw (0,0,1)--(0,2,1);


\foreach \x in {0,1,2}
\foreach \y in {0,1,2}
\foreach \z in {0,1,2}
{
\draw [fill=black](\x,\y,\z) circle (2pt);
}


\draw [fill=white] ( 0,0,0 )  circle (2pt);
\draw [fill=white] ( 0,0,2 )  circle (2pt);
\draw [fill=white] ( 0,1,1 )  circle (2pt);
\draw [fill=white] ( 1,0,1 )  circle (2pt);
\draw [fill=white] ( 2,0,0 )  circle (2pt);
\draw [fill=white] ( 2,0,2 )  circle (2pt);
\draw [fill=white] ( 2,1,1 )  circle (2pt);
\draw [fill=white] ( 1,2,1 )  circle (2pt);
\draw [fill=white] ( 0,2,0 )  circle (2pt);
\draw [fill=white] ( 0,2,2 )  circle (2pt);
\draw [fill=white] ( 1,1,0 )  circle (2pt);
\draw [fill=white] ( 1,1,2 )  circle (2pt);
\draw [fill=white] ( 2,2,0 )  circle (2pt);
\draw [fill=white] ( 2,2,2 )  circle (2pt);

\end{tikzpicture}
\caption{食盐晶体的空间点阵}
\end{figure}

图4.4是食盐的空间点阵示意图.食盐的晶体是由钠离子Na$^+$和氯离子Cl$^-$组成的,它们等距离地交错地排列在三
组相互垂直的平行线上,每个Na$^+$的周围有六个Cl$^-$,每个Cl$^-$的周围有六个Na$^+$.

晶体外形的规则性可以用物质微粒的规则排列来解释.
同样,晶体的各向异性也是由晶体的内部结构决定的.
\begin{figure}[htp]\centering
	\begin{tikzpicture}[>=stealth, scale=1 ]

\foreach \x in {0,1,2,...,5}
\foreach \y in {0,1,2,...,5}
{
\draw [fill=black](\x,\y) circle (2pt);
}

\draw(0,5)node [left]{$B$}--(5,5)node [right]{$A$};
\draw (5,5)--(0,3-.5)node [left]{$C$};
\draw (5,5)--(.5,.5)node [left]{$D$};

\end{tikzpicture}
\caption{各向异性的微观解释}
\end{figure}

图4.5表示在一个平面上晶体物质微粒的排列情况.从图上可以看出,沿不同方向所画的等长直线$AB$、$AC$、$AD$上,物质微粒的数目不同.直线$AB$上物质微粒较多,直线$AD$上较少,直线$AC$上更少.正因为在不同方向上物质微粒的排列情况不同,才引起晶体在不同方向上物理性质的不同.

有的物质能够生成种类不同的几种晶体,是因为它们的物质微粒能够形成不同的空间点阵.例如,碳原子如果按图4.6那样排列就成为石墨,按图4.7那样排列就成为金刚石.石墨是层状结构,层与层之间距离较大,作用力较弱,沿着这个方向容易把石墨一层层地剥下.金刚石中碳原子间的作用力很强,所以金刚石有很大的硬度.
\begin{figure}[htp]\centering
	\begin{minipage}[t]{0.48\textwidth}
		\centering
\includegraphics[scale=1.3]{fig/4-6.pdf}
\caption{石墨的空间点阵}
	\end{minipage}
	\begin{minipage}[t]{0.48\textwidth}
		\centering
\includegraphics[scale=1.3]{fig/4-7.pdf}
\caption{金刚石的空间点阵}
	\end{minipage}
\end{figure}


\section{液体的微观结构}
液体的性质介于气体和固体之间.液体一方面象固体,具有一定的体积,不易压缩;另一方面又象气体,没有一定的形状,具有流动性.液体汽化时体积改变上千倍,凝固时体积改变不过百分之十.液体更接近于固体.

跟固体一样,液体中的分子也是密集在一起的,因而液体具有一定的体积,不易压缩.液体分子在很小的区域内作有规则的排列,这种区域是由分子暂时形成的,边界和大小随时改变,有时瓦解,有时又重新形成.液体由大量这种小区域构成,这种小区域杂乱无章地分布着,因而液体表现出各向同性.

液体分子间的距离小,相互作用力还很大,因此液体分子的热运动与固体类似,主要表现为在平衡位置附近做微小的振动,跟固体不同的是,液体分子没有长期固定的平衡位置,
在一个平衡位置附近振动一小段时间以后,又转到另一个平
衡位置附近去振动,即液体分子可以在液体中移动,这就是液体具有流动性的原因.

非晶体的微观结构跟液体非常类似,可以看作是粘滞性
极大的液体.所以严格说来只有晶体才能叫做真正的固体.

\section*{阅读材料:液晶}
某些有机化合物(现已发现有几千种)具有一种特殊的物质状态,叫做液晶.液晶一方面象液体,具有流动性;另一方面又象晶体,光学性质具有各向异性.液晶是介于液体和固体之间的过渡状态,微观结构也介于固液之间.

液晶是不稳定的,外界影响的微小变动都会引起液晶分子排列的变化,改变它的光学性质.有一种液晶,在外加电压的影响下,会由透明状态变成混浊状态,不再透明.去掉电压,又恢复透明,利用这一性质,可以制成显示元件.在两电极间将液晶涂成文字或数码,加上适当电压,透明的液晶变得混浊了,文字或数码就显示出来了.这种显示元件用于电子手表、电子计算器、微电脑以及其他仪器中.

还有一种液晶,具有灵敏的温度效应,温度改变时会改变颜色.只要温度升高1$^\circ$C,液晶就会按红、橙、黄、绿、蓝、靛、紫的顺序改变颜色;温度降低,又按相反顺序改变颜色.液晶的
这种性质,可以用来探测温度.例如在医学上可用来检查肿瘤.在皮肤表面涂上一层液晶,由于肿瘤部分的温度与周围正常组织的温度不一样,液晶会显示出不同的颜色,电路中的短路点温度高,用同样的办法可检查出短路点.

液晶早在十九世纪八十年代就被发现,直到电子技术和其他一些技术迅速发展起来以后,近十几年来,人们对液晶的研究才有了重要的进展,使它获得了广泛的应用.

\section{液体的表面现象}
荷叶上的小水滴、草叶上的露珠、玻璃板上的小水银滴都是近于球形的.大液滴呈扁平形状,是因为它本身的重量较大,它的形状受到重力的影响也比较大.如果设法消除重力的影响,大液滴也会呈球形.配制水和酒精的混合液,使它的密度跟橄榄油的密度相等.把橄榄油引入这种混合液里,可以看到橄榄油呈球形(图4.8).
\begin{figure}[htp]\centering
	\begin{tikzpicture}[>=stealth, scale=1 ]

\foreach \x in {0,0.1,0.2,0.4,.6,.9,1.2,1.6,2,2.4,2.9,3.4,3.9}
{
  \draw [dashed](0,\x)--(5,\x);
}

\draw [shade] (2.5,2) circle (4mm);


\end{tikzpicture}
\caption{橄榄油在水和酒精的混合液里呈球形}
\end{figure}

我们知道,在体积相同的各种形状的物体中,球形物体的表面积最小,上述实验表明,液体表面有收缩到最小面积的趋势.

我们还可以用肥皂水做实验来证明液面具有收缩趋势.把一根棉线拴在铁丝环上(棉线不要张紧),把环在肥皂水里浸一下,使环上布满肥皂水的薄膜(图4.9甲).如果用热针刺破棉线左边的薄膜,棉线就被右边的薄膜拉成弧形(图4.9乙);如果刺破右边的薄膜,棉线就被左边的薄膜拉成弧形(图4.9丙).
\begin{figure}[htp]\centering
\includegraphics[scale=.45]{fig/4-9.png}
\caption{薄膜的收缩使棉线成弧形}
\end{figure}

\begin{figure}[htp]\centering
\includegraphics[scale=.45]{fig/4-10.png}
\caption{薄膜的收缩使棉线圈成圆形}
\end{figure}

把一个棉线圈拴在铁丝环上,使环上布满肥皂水的薄膜(图4.10甲).如果用热针刺破棉线圈里那部分薄膜,外边的薄膜就把棉线圈张紧成圆形(图4.10乙).

这些实验表明:液体的表面就好象张紧的橡皮膜一样,具有收缩的趋势.

为什么液体表面具有收缩的趋势呢?原来液体跟气体接触的表面形成一个薄层,叫做\textbf{表面层},表面层里的情况跟液体内部有所不同.研究表明,表面层里的分子要比液体内部稀疏些,也就是分子间的距离要比液体内部大些.在液体内部分子间既存在着引力,又存在着斥力.引力和斥力的数量级相同,在通常的条件下可以认为它们的大小是相等的.在表面层里分子间的距离大,引力和斥力都减小,但斥力减小得更快,所以分子间的相互作用表现为引力,如果在液面上划一条分界线$MN$(图4.11),把液面分为(1)和(2)两部分,那么,由于表面层中分子间的引力,液面(1)对液面(2)有引力$f_1$的作用,液面(2)对液面(1)
有引力$f_2$的作用,$f_1$和$f_2$大小相等
方向相反.象这种液面各部分间相互吸引的力,叫做\textbf{表面张力}.液体的表面张力使液面具有收缩的趋势.
\begin{figure}[htp]\centering
	\begin{tikzpicture}[>=latex, scale=1 ]

\draw [very thick](45:.5)node[below]{$M$}--(45:8.5)node[above]{$N$};

\foreach \x in {1,2,...,5}
{ 
   \draw[->] (\x+.2,\x-.2)--(\x-1,\x+1);
   \draw[->] (\x+.5-.2,\x+.5+.2)--(\x+1.5-.2,\x-.5+.2);
}

\node at (1,5){(1)}; \node at (5,1){(2)};
\node at (4,2.5){$f_2$}; \node at (2.5,4){$f_1$};

\end{tikzpicture}
\caption{液体的表面张力}
\end{figure}

表面张力是跟液面相切的,如果液面是平面,表面张力
就在这个平面上;如果液面是曲面,表面张力就在这个曲面的切面上,作用在任何一部分液面上的表面张力,总是跟这部分液面的分界线垂直.

作用在液体表面单位长度分界线上的表面张力,叫做\textbf{表面张力系数}.不同液体的表面张力系数不同.分子间作用力大的液体,表面张力系数大,液态金属的表面张力系数很大,液态气体的很小.

\subsection*{练习一}
\begin{enumerate}
    \item 把玻璃管的裂断口放在火焰上烧熔,它的尖端就变圆.这是什么缘故?
\item 在处于失重状态的宇宙飞船中,一大滴水银会呈什么形状?
\item 把熔化的铅一滴一滴地滴入水中,凝固后可以得到
球形的小铅弹.为什么?
\end{enumerate}

\section{浸润和不浸润}
在洁净的玻璃板上放一滴水银,它能够在玻璃板上滚来滚去,而不附着在上面.把一块洁净的玻璃片浸入水银里再取出来,玻璃上也不附着水银,这种现象叫做\textbf{不浸润}.对玻璃来说,水银是不浸润液体.

在洁净的玻璃板上放一滴水,它要附着在玻璃板上形成薄层,把一块洁净的玻璃片浸入水里再取出来,玻璃片的表面是带有一层水的,这种现象叫做\textbf{浸润}.对玻璃来说,水是浸润液体.

同一种液体,对一些固体来说是浸润的,对另一些固体来说是不浸润的.水能浸润玻璃,但不能浸润石蜡,水银不能浸润玻璃,但能浸润锌.

把浸润液体装在容器里,例如把水装在玻璃容器里,由于水浸润玻璃,器壁附近的液面向上弯曲(图4.12),把不浸润液体装在容器里,例如把水银装在玻璃容器里,由于水银不浸润
玻璃,器壁附近的液面向下弯曲(图4.13),在内径较小的容器里,这种现象更显著,液面形成凹形或凸形的弯月面.

\begin{figure}[htp]\centering
	\begin{minipage}[t]{0.48\textwidth}
		\centering
	\begin{tikzpicture}[>=stealth, scale=.8]
\fill [pattern=north east lines](-.25,0) rectangle(0,5);
\fill [gray!20] (0,0) rectangle (3,3);
\fill [gray!20] (0,3)--(1,3)--(0,4);
\draw [fill=white, thick](0,4) arc(180:270:1);
\draw[thick](1,3)--(3,3);
\draw (0,0)--(0,5);
\fill [gray!20] (4, 0) rectangle (4.5, 5);
\draw [fill=white, rounded corners=0.2cm](4.5, 5)--(4.5, 4)--(4,4)--(4,5);
\draw (4, 0)--(4,5);
\draw (4.5, 0)--(4.5,5);
\end{tikzpicture}
\caption{液体浸润固体}
    \end{minipage}
\begin{minipage}[t]{0.48\textwidth}\centering
	\begin{tikzpicture}[>=stealth, scale=.8]

\fill [gray!50] (0,0) rectangle (3,2);
\fill [gray!50] (1,2) rectangle (3,3);
\fill [gray!50] (0,2)--(1,3)--(1,2);
\draw[ thick ](1,3)--(3,3);
\draw [fill=gray!50, thick ](1,3) arc(90:180:1);
\fill [white] (4, 0) rectangle (4.5, 5);
\draw [fill=gray!50, rounded corners=0.2cm](4.5, 0)--(4.5, 4)--(4,4)--(4,0);
\draw (4, 0)--(4,5);
\draw (4.5, 0)--(4.5,5);
\fill [pattern=north east lines](-.25,0) rectangle(0,5);
\draw (0,0)--(0,5);
\end{tikzpicture}
\caption{液体不浸润固体}
    \end{minipage}
\end{figure}

浸润和不浸润现象也是分子力作用的表现.当液体与固体接触时,在接触处形成一个液体薄层,叫做\textbf{附着层}.附着层里的分子既受到固体分子的吸引,又受到液体内部分子的吸引.如果受到的固体分子的吸引比较弱,附着层里的分子就比液体内部稀疏,在附着层就出现跟表面张力相似的收缩力,这时跟固体接触的液体表面有缩小的趋势,形成不浸润现象.相反,如果受到的固体分子的吸引相当强,附着层里的分子就比液体内部更密,在附着层里就出现液体相互推斥的力,这时跟固体接触的液体表面有扩展的趋势,形成浸润现象.

\subsection*{练习二}

\begin{enumerate}
   \item 把一根缝衣针小心地放在水面上,针可以把水面压弯而不沉没(试试看).解释这个现象.
   \item 布的雨伞虽然纱线间有可以看得出来的孔隙,却仍然不漏雨水.解释这个现象.
\end{enumerate}

\section{毛细现象}
把几根内径不同的细玻璃管插入水中,可以看到管里的水面比容器里的水面高.管的内径越小,管里的水面越高(图4.14).如果把这些细玻璃管插入水银中,发生的现象正好相
反,管里的水银面比容器里的水银面低.管的内径越小,管里
的水银面越低(图4.15).
\begin{figure}[htp]\centering
	\begin{minipage}[t]{0.48\textwidth}\centering
	\begin{tikzpicture}[>=stealth, scale=.8 ]
\draw [fill=gray!20, rounded corners=0.2cm, thick](0,5)--(0, 0)--(6,0)--(6,5);
\draw [fill=white, rounded corners=0.15cm](0,5)--(0,2)--(1.5,2)--(1.5,5);
\draw [fill=white, rounded corners=0.15cm](2,5)--(2,2)--(3.35,2)--(3.35,5);
\draw [fill=white, rounded corners=0.15cm](3.75,5)--(3.75,2)--(4.75,2)--(4.75,5);
\draw [fill=white, rounded corners=0.15cm](5,5)--(5,2)--(6,2)--(6,5);


\draw [fill=white, rounded corners=0.1cm](1.5, 5)--(1.5,2.8)--(2, 2.8)--(2,5);
\draw [fill=white, rounded corners=0.1cm](3.35, 5)--(3.35,3.5)--(3.75, 3.5)-- (3.75, 5);
\draw [fill=white, rounded corners=0.1cm](4.75, 5)--(4.75, 4.5)--(5, 4.5)--(5, 5);

\foreach \x in {1.5,2,3.35,3.75,4.75,5}
{
  \draw (\x, 1)--(\x, 5);

}
\end{tikzpicture}
\caption{浸润液体在毛细管里上升
}
  \end{minipage}
\begin{minipage}[t]{0.48\textwidth}\centering
	\begin{tikzpicture}[>=stealth, scale=.8 ]
\draw [fill=gray!50, rounded corners=0.2cm, thick](0,1)--(0, 0)--(6,0)--(6,1);

\draw [fill=gray!50, rounded corners=0.15cm](0,1)--(0,2)--(1.5,2)--(1.5,1);
\draw [fill=gray!50, rounded corners=0.15cm](2,1)--(2,2)--(3.35,2)--(3.35,1);
\draw [fill=gray!50, rounded corners=0.15cm](3.75,1)--(3.75,2)--(4.75,2)--(4.75,1);
\draw [fill=gray!50, rounded corners=0.15cm](5,1)--(5,2)--(6,2)--(6,1);


\draw [fill=gray!50, rounded corners=0.1cm](1.5, 1)--(1.5,1.8)--(2, 1.8)--(2,1);
\draw [fill=gray!50, rounded corners=0.1cm](3.35, 1)--(3.35,1.5)--(3.75, 1.5)-- (3.75, 1);
\draw [fill=gray!50, rounded corners=0.1cm](4.75, 1)--(4.75, 1.2)--(5, 1.2)--(5, 1);

\foreach \x in {0,1.5,2,3.35,3.75,4.75,5,6}
{
  \draw (\x, 1)--(\x, 5);

}



\end{tikzpicture}
\caption{不浸润液体在毛细管里下降
}
    \end{minipage}
\end{figure}

象这样浸润液体在细管里上升的现象和不浸润液体在细管里下降的现象,叫做\textbf{毛细现象}.发生毛细现象的管叫做毛细管.

浸润液体为什么能在毛细管里上升呢?原来,浸润液体跟毛细管内壁接触时,引起液面的弯曲,使液面变大.而表面张力的收缩作用使液面减小,于是管内液体随着上升,以减小液面.直到表面张力向上的拉引作用和管内升高的液柱的重量达到平衡时,管内液体停止上升,稳定在一定的高度.利用类似的分析,也可以解释不浸润液体在毛细管里下降的现象.

纸张、棉花、毛巾、粉笔、木材、土壤、砖块等物体,内部有
许多细小的孔道,起着毛细管作用,所以它们能够吸水,这就是用毛巾擦汗、灯芯吸油、粉笔吸墨水的道理.

毛细现象在农业生产上有非常重要的意义,土壤里有很多毛细管,地下的水分可以沿着它们上升到地面.如果要保存地下的水分来供植物的根吸收,就要把地面的土壤锄松,破
坏这些土壤里的毛细管.相反,如果想把地下的水分引上来,就不仅要保持土壤里的毛细管,而且还要使它们变得更细,这时就要用滚子压紧土壤.

\subsection*{练习三}
\begin{enumerate}
   \item 要想把凝在衣料上面的蜡或油脂去掉,只要把两层吸墨纸分别放在这部分衣料的上面和下面,然后用熨斗来熨就可以了,为什么这样做可以去掉衣料上的蜡或油脂?
\item 建筑楼房的时候,在砌砖的地基上铺一层油毡防潮层.如果不铺这层油毡,楼房就容易受潮,为什么?
\end{enumerate}

\section*{复习题}

\begin{enumerate}
\item 晶体和非晶体在外形和物理性质上有什么区别?
\item 什么叫空间点阵?怎样从微观上解释晶体具有规则的外形和各向异性?
\item 液体的微观结构是怎样的?
\item 什么叫表面张力?为什么液体表面具有收缩趋势?
\item 什么叫浸润,什么叫不浸润?怎样解释浸润現象和不浸润现象?
\item 什么叫毛细现象?为什么会发生毛细现象?
\end{enumerate}






\chapter{物态变化}
固体、液体和气体是通常存在的三种物质状态.在一定条件下,这三种物质状态可以相互转化,即发生物态变化,水结成冰或变成水蒸气就是常见的物态变化的例子.物态变化跟日常生活和工农业生产有密切关系,有许多实际应用,在初中我们学过一些物态变化的知识,这一章进一步扩大和加深这方面的知识,同时学习一些新知识.

\section{熔解和凝固}
物质从固态变成液态叫做熔解,从液态变成固态叫做\textbf{凝固}.

我们知道,固体分晶体和非晶体.晶体物质和非晶体物质在熔解和凝固时情况是不同的.晶体有一定的熔解温度——熔点.给晶体加热,当温度升高到熔点时,晶体开始熔解,在熔解过程中温度保持不变,到全部熔解后,温度才继续上升.它们的液体冷却时,温度降低到熔点时开始凝固,在凝固过程中温度也保持不变,到全部凝固后,温度才继续下降.

非晶体没有一定的熔点,温度升高时,非晶体先是由硬变软,再逐渐变成粘稠状液体,最后逐渐变成液体.在整个过程中,温度不停地上升,没有一定的熔解温度,冷却时,随着温度
的下降,液体由稀变稠,由软变硬,最后成为固体,在整个过程中,温度不停地下降,没有一定的凝固温度.

晶体物质熔解和凝固时的特点,可用晶体的微观结构来解释.在晶体中,微粒排列成有规则的空间点阵,维持这种规则排列的是微粒之间的相互作用;微粒的热运动不足以克服这种相互作用,微粒一般只能在平衡位置附近做无规则的振动.给晶体加热时,晶体从外界得到能量,微粒的热运动加剧.达到一定的温度时,一部分微粒具有了足够的动能,能够克服微粒间的作用力,离开平衡位置.这时晶体的点阵结构被破坏,晶体开始熔解,在熔解过程中,外界供给晶体的能量,全部用来破坏晶体的点阵结构,增加分子间的势能,所以温度不发生变化.凝固时,情况正好相反.微粒排列成点阵结构时,微粒间的势能减小,因此虽然放出能量,温度却保持不变,直到全部凝固成晶体.

非晶体的微观结构本来就跟液体类似,非晶体在熔解过
程中不必为破坏点阵结构而消耗能量,所以温度不停地上升.

实验表明,大多数物质熔解时体积膨胀,凝固时体积缩小.然而也有少数物质跟上述情况相反,例如水、灰铸铁、锑、铋等,它们在熔解时体积缩小,凝固时体积膨胀.用铸铁浇铸成的工件,形状跟铸模完全相似,正是利用了铸铁凝固时体积膨胀的特点.水结冰时体积膨胀,所以在冬季水管和盛水的容器常会冻裂,需要加以防止.

物质的熔点跟压强有关系,熔解时体积膨胀的物质,如果所受的压强增大,熔解将受到阻碍,只有在更高的温度下才
能熔解,所以熔点升高,熔解时体积缩小的物质,如果所受的压强增大,会促进熔解的进行,所以熔点降低.冰在熔解时体积缩小,因而冰在受到巨大的压强时,在0$^\circ$C以下也能熔解.不过压强对熔点的影响并不显著,例如,每增如1标准大气压,冰的熔点才降低0.0075$^\circ$C.

下表列出了几种物质在1标准大气压下的熔点.

\begin{center}
    \begin{tabular}{cccc}
\hline
物质 & 熔点($^\circ$C) & 物质 & 熔点($^\circ$C) \\
\hline
氦 & $-$272 & 铅 & 327\\
氢& $-$259 & 铝 & 660\\
酒精& $-$117 & 银 & 962\\
氨& $-$77.7 & 金 & 1064\\
水银& $-$39 & 铜 & 1083\\
冰& 0 & 铁 & 1535\\
萘&80 & 钨 & 3410\\
锡&232   & 碳 & 3550\\
\hline
  &  \end{tabular}
\end{center}

一般说来,纯物质中掺进另一种物质,熔点要降低.例如
海水比淡水的熔点低.冰和食盐的混合物,熔点可以降低到零下二十多摄氏度,冰和氯化钙的混合物,熔点可以降低到零下五十多摄氏度.

某些合金的熔点较低,例如铅锑合金的熔点为246$^\circ$C,锡铅合金的熔点为170$^\circ$C,由铋、镉、锡、铅组成的伍德合金,其熔点仅 70$^\circ$C.这些低熔点合金在生产技术中具有广泛的应用.

熔点是物质的重要性质之一,在实际中常常要根据需要选用不同熔点的物质,例如,制做白炽灯丝要用熔点高的钨,
而焊接电路则用熔点较低的铅锡合金.

\section{熔解热}
晶体在熔解过程中不断从外界吸收热量,但温度却保持不变.吸收的热景绝大部分用于破坏品体的点阵结构,增加分子势能,对大多数熔解时体积增大的物质来说,还有一部分热量用来克服外界压强做功.不过体积变化不大,这个功很小.

质量相等的不同晶体溶解时吸收的热量是不同的.\textit{单位质量的某种物质熔解成同温度的液体时吸收的热量,叫做这种物质的熔解热}.在国际单位制中,熔解热的单位是焦/千克.

液体凝固时要放出热量.单位质量的某种物质凝固时放出的热量等于它的熔解热.显然这是符合能量守恒定律的.

物质的熔点跟压强有关系,在不同的熔解温度下,物质的熔解热也略有不同.下表列出了几种物质在1标准大气压下的熔解热.

\begin{center}
    \begin{tabular}{cccc}
  \hline
 物质 & 熔解热(${\rm J}/{\rm kg}$) & 物质 & 熔解热(${\rm J}/{\rm kg}$) \\
  \hline
  铝 & $3.96\times 10^5$ & 萘 & $1.51\times 10^5$\\
  冰& $3.35\times 10^5$ & 锡 & $0.6\times 10^5$\\
  镍& $2.99\times 10^5$ & 铅 & $0.25\times 10^5$\\
   铁& $2.67\times 10^5$ & 水银 & $0.11\times 10^5$\\
  铜& $2.05\times 10^5$ & 二氧化碳 & $1.81\times 10^5$\\
   钨& $1.92\times 10^5$ & 金 & $0.64\times 10^5$\\
    \hline
   \end{tabular}
\end{center}

熔解热常用字母$\lambda$表示.知道了熔解热,就可以算出质量为$m$的物质熔解时吸收的热量$Q$:
\[Q=\lambda m \]

熔解热可以用量热器测定.现在以冰为例来说明.在量热器中装入已知质量的水,测出水的温度,然后把正在熔解的冰块(温度为0$^\circ$C)放入量热器的水中,冰将继续熔解,水的温度下降.测出冰块全部熔解后水的末温度和的质量.根据前后两次测得的水的质量,可以求出冰块的质量,冰块熔解成水并从0$^\circ$C升高到末温度吸收的总热量,等于量热器中原来的水和小筒从初温度下降到末温度放出的热量,从这个关系就可以求出冰的熔解热.

例如,如果已知铜制量热器小筒的质量是150克,里面装着100克16$^\circ$C的水.放入9克0$^\circ$C的冰,冰完全熔解后水的温度是9$^\circ$C.利用这些数据就可以求出冰的熔解热.

9克0$^\circ$C的冰熔解为0$^\circ$C的水,再升高到9$^\circ$C,总共吸收
的热量
\[Q_{\text{吸}}=m_{\text{冰}}\lambda +m_{\text{冰}}c_{\text{水}} (9^\circ {\rm C}-0^\circ {\rm C}) \]
量热器中的水和量热器小筒从16$^\circ$C降到9$^\circ$C放出的热量
\[Q_{\text{放}}=m_{\text{水}}c_{\text{水}} (16^\circ {\rm C}-9^\circ {\rm C}) +m_{\text{筒}}c_{\text{铜}} (16^\circ {\rm C}-9^\circ {\rm C}) \]
因为$Q_{\text{吸}}=Q_{\text{放}}$,所以
\[m_{\text{冰}}\lambda +m_{\text{冰}}c_{\text{水}} (9^\circ {\rm C}-0^\circ {\rm C})=\left(m_{\text{水}}c_{\text{水}}  +m_{\text{筒}}c_{\text{铜}}\right) (16^\circ {\rm C}-9^\circ {\rm C}) \]

统一单位后,把数值代入上式[铜的比热$c_{\text{铜}}=3.9\times 10^2{\rm J}/({\rm kg}\cdot ^\circ{\rm C})$],可得
\[\lambda=3.3\times 10^5{\rm J}/{\rm kg}   \]

冰的熔解热很大,1千克0$^\circ$C的冰熔解成0$^\circ$C的水吸收的热量,相当于把1千克0$^\circ$C的水升高到80$^\circ$C需要的热量.冰的这一特点对自然界有重要的意义,它使得初冬时,一个寒冷的夜晚不会把江河湖全部封冻起来,气温也不会骤然下降;初春时,一个阳光灿烂的晴天不会使冰雪全部熔解,造成江河泛滥,气温也不会骤然升高.在日常生活中,人们利用冰熔解热大的特点来冷藏食品、冰镇饮料等.

\subsection*{练习一}

\begin{enumerate}
    \item 把玻璃放在火上加热,观察它的熔解情况,看看玻璃是不是先变软,再流动.玻璃是不是晶体?
    \item 解释下面的现象:把一块冰放在支承物上(图5.1),
    将两端各挂一个重物的铁丝搭在冰块上.过一段时间后可以看到,铁丝切进冰块,但是铁丝穿过冰块的地方并没有留下切口,冰仍然是完整的一块,铁丝
    为什么能切进冰块?铁丝穿过后上面的冰为什么又成了完整的一块?如果有条件,自己做这个实验.
\begin{figure}[htp]
\centering
\includegraphics[scale=.8]{fig/5-1.png}
\caption{细铁丝穿过冰块而不留下切口
}
\end{figure}
    \item   冬季在菜窖里放上几桶水,可以使窖内的温度不致降低得很多,防止把莱冻坏.这是什么道理?如果在窖内放入200千克10$^\circ$C的水,试计算这些水结成0$^\circ$C的冰时放出的热量.这相当于燃烧多少千克干木柴所放出的热量?于木柴的
    燃烧值约为$1.26\times 10^4$千焦/千克.
\item  铜制量热器小简的质量是160克,装入200克20$^\circ$C的水.向水里放进30克0$^\circ$C的冰,冰完全熔解后水的温度是多少摄氏度?
\item  量热器的铜制小筒里盛有200克15$^\circ$C的水,小筒的质量为160克,向水里放入50克0$^\circ$C的冰,求量热器里的末温度是多少摄氏度?
\end{enumerate}


\section{蒸发}
物质从液态变成气态,叫做汽化.汽化有两种方式:蒸发和沸腾.蒸发是在液体表面进行的汽化现象.这一节我们先来研究蒸发.

我们知道,液体中的分子都在不停地运动着,它们的平均动能跟温度有关系.但在任何温度下,总有一部分分子的动能比平均动能大.那些处在液体表面层附近的动能足够大的分子,能够挣脱周围分子的引力,飞出液面,形成蒸气,这就是\textbf{蒸发},蒸气也常叫做汽.

液体温度越高,分子的平均动能就越大,具有足够大的动能因而能够飞出液面的分子也就越多.所以,温度越高,蒸发得越快.泼在地上的水夏天干得快,冬天干得慢,就是这个原因.

液体的表面积越大,处在表面层中的分子就越多,能够从液面飞出的分子也就越多.所以,表面积越大,蒸发得越快.洗过的衣服抻开晾晒比团在一起干得快,原因就在这里.

飞出液面的分子如果停留在液面附近.由于分子的热
运动,有的分子会撞到液面,被液体分子重新拉回到液体中去,这样蒸发就变慢了.如果设法把液面上形成的蒸气吹散,使汽分子不能回到液体中去,蒸发就可以加快.所以,蒸发的快慢还跟液面上气体流动的快慢有关系.气体流动得越快,蒸发得也越快,这就是吹风能够使湿东西干得快的原因.

在同样的条件下,不同液体蒸发得快慢不同.水比食油容易蒸发,汽油比水容易蒸发.容易蒸发的液体,我们常说它的挥发性大.液体的这种差别,跟它们分子间的作用力有关系.分子间作用力大的液体不容易蒸发.

在蒸发过程中,从液体中飞出的是动能较大的分子,这些分子飞出后,留在液体中的分子的平均动能必然减小,所以蒸发时液体的温度降低,这时它就要从周围的物体吸收热量,因而液体蒸发有致冷作用.穿湿衣服比穿干衣服感到冷,夏天搧扇子感到凉快,出汗后站在通风处容易着凉,都是由于蒸发致冷的缘故.蒸发致冷作用在实际中有许多应用.用火车运送容易腐烂变质的食品时,常用液态氨或液态二氧化碳的蒸发来降低车厢内的温度.在医疗中,可用液态氮迅速蒸发时的冷却作用使病灶处的细胞冷冻坏死.导弹在大气中高速飞行时,由于跟空气摩擦,会达到极高的温度.为了保护弹壳,常在弹壳表面涂上防护层,防护层的物质受热熔解和蒸发时,要
吸收大量的热量,从而降低了导弹表面的温度.

\section{饱和汽与饱和汽压}
\subsection{饱和汽} 
装在敞口容器里的液体,由于蒸发出来的汽分子能够分散到周围空间里去,所以过一段时间后液体会全部蒸发完,盛在密闭容器里的液体,即使过很长时间,也不会蒸发完,这是什么原因呢?原来在容器中的液面上同时进行着两种相反的过程:一方面分子从液面飞出来;另一方面由于液面上的汽分子不停地做无规则的热运动,有的汽分子撞到液面上又会回到液体中去(图5.2).在密闭的容器中,随着液体的
不断蒸发,液面上汽的密度不断增大,回到液体中的分于数也逐渐增多,最后,当汽的密度增大到一定程度时,就会达到这样的状态:在单位时间内回到液体中的分子数等于从液面飞出去的分子数.这时汽的密度不再增大,液体也不再减少,
液体和汽之间达到了平衡状态,这种平衡叫做\textbf{动态平衡}.我们把跟液体处于动态平衡的汽叫做\textbf{饱和汽}.把没有达到饱和状态的汽叫做\textbf{未饱和汽}.在一定温度下,饱和汽的密度是一定的,未饱和汽的密度小于饱和汽的密度.

\begin{figure}[htp]
\centering
\includegraphics[scale=.6]{fig/5-2.png}
\caption{}
\end{figure}

饱和汽的密度随着温度而改变.温度升高时,液体分于的平均动能增大,单位时间里从液面飞出的分子数增多,原来的动态平衡被破坏,液体就继续蒸发,汽的密度继续增大,直
到达到新的动态平衡为止.所以饱汽的密度随着温度的升高而增大.温度降低时,饱和汽的密度减小.其原因请同学们自己来解释.

\subsection{饱和汽压} 

某种液体的饱和汽所具有的压强,叫做这种液体的饱和汽压.用下面的实验可以测量饱和汽压.把两根装满水银的细长玻璃管倒立在水银槽中(图5.3).$a$管用作气压计.用弯曲玻璃管向$b$管中注入一些乙醚.乙醚的密度比水银小,便浮到水银面上,并在水银面上方的真空中蒸发,
同时$b$管中的水银面开始下降,当乙醚汽的密度达到饱和时,水银面上还剩有一点乙醚,$b$管中的水银面就稳定在一定的高度不再下降.$a$、$b$两管中水银面的高度差$h$就表示乙醚饱和汽压的大小.如果不用乙醚,而是向$b$管中注入水或酒精,就可以测出水或酒精的饱和汽压.
\begin{figure}[htp]
\centering
\includegraphics[scale=.6]{fig/5-3.png}
\caption{测量饱和汽的压强
}
\end{figure}

实验表明,在相同的温度下,不同液体的饱和汽压一般是不同的.挥发性大的液体,饱和汽压大.例如,20$^\circ$C时,乙醚的饱和汽压为440毫米汞柱,水为18毫米汞柱,水银的饱和汽压很小,20$^\circ$C时仅为0.0012毫米汞柱.所以水银气压计水银柱上方的空间可以认为是真空.

\subsection{饱和汽压跟温度的关系} 

用热毛巾把图5.3中的$b$管包住,使管内的温度升高,可以看到管中的水银面下降.这表示温度升高时饱和汽压增大.如果使管内的温度降低,管中的水银面就上升,表示温度降低时饱和汽压减小.可见,\textit{饱和汽压随温度的升高而增大,随温度的降低而减小.}

饱和汽压随温度的升高而增大,这是由两方面的原因引起的,一个原因是温度升高时,饱和汽的密度增大,因此压强增大,另一个原因是温度升高时,汽分子热运动的平均速率增大,这也使得压强增大.图5.4是水的饱和汽压与温度的关系图线.从图线可以看出,温度升高时,饱和汽压增大,但是饱和汽压与温度的关系不是线性的,饱和汽压随温度的升高增大得很快.
\begin{figure}[htp]\centering
	\begin{tikzpicture}[>=stealth, scale=1.1,domain=0:5.5]

\draw[<->] (0,5)node [right]{$P$(mmHg)}--(0,0)--(6,0)node [right]{$t$($^\circ$C)};

\foreach \x in{1,2,3,4}
{
   \draw(0,\x)--(.2, \x);
}
\foreach \x in{1,2,3,4,5}
{
   \draw(\x,0)--(\x, .2);
}

\node at (-.3,1){200};\node at (-.3,2){400};\node at (-.3,3){600};\node at (-.3,4){800};
\node at (1,-.3){20};\node at (2,-.3){40};\node at (3,-.3){60};\node at (4,-.3){80};\node at (5,-.3){100};\node at (0,-.3){0};

\draw[color=black, ultra thick] plot (\x,{0.025*\x*\x*\x}) ;

\end{tikzpicture}
\caption{水的饱和汽压与温度的关系图线}
\end{figure}

饱和汽压随温度的降低而减小,其原因同学们可以自己分析一下.

\subsection{饱和汽压跟体积的关系} 

现在我们用实验来研究饱和汽
压跟体积的关系.象图5-5那样,把水银倒在一个深的容器里,再把装满水银的玻璃管倒立在这个容器中.向玻璃管里移入一些乙醚,使乙醚蒸发后水银面上还留有少量液态乙醚.这时管内乙醚的饱和汽压等于$p_0-p_k$(图5.5甲),其中$p_0$是大气压强.
\begin{figure}[htp]
\centering
\includegraphics[scale=.6]{fig/5-5.png}
\caption{饱和汽压跟体积无关
}
\end{figure}

现在把管子往上提高一些,使水银面上乙醚汽的体积增大(图5.5乙),可以看到,水银面上的液态乙醚减少了,
但是管里水银柱的高度还跟原来一样.这表明,在温度不变的情况下,体积增大时饱和汽压不改变.

把提高的管子放下一些,使乙醚汽的体积减小(图5.5丙). 这时水银面上的液态乙醚增多,但是管中水银柱的高度仍保持不变.这表明,在温度不变的情况下,体积减小时饱和汽压也不改变.

在温度不变的情况下,饱和汽的压强不随体积而变化,可作如下的解释:当体积增大时,容器中汽的密度减小,原来的饱和汽变成了未饱和汽,于是液体继续蒸发,直到未饱和汽成为饱和汽为止;由于温度没有改变,新的饱和汽的密度跟原来的一样,汽分子热运动的平均速率也跟原来一样,所以压强不
改变.体积减小时,容器中汽的密度增大,回到液体中的分子数多于从液面飞出的分子数,于是一部分汽变成液体,直到汽的密度减小到等于该温度下饱和汽的密度为止;由于温度跟原来相同,饱和汽的密度不变,汽分子热运动的平均速率也跟原来相同,所以压强也不改变.

从上面讲的可以看出,饱和汽的压强跟体积没有关系,饱和汽的压强与温度的关系不是线性的,这些都跟理想气体的情况不同.因此,第三章所讲的理想气体定律对饱和汽是不适用的.


\subsection*{练习二}

\begin{enumerate}
    \item 举出几个蒸发致冷的例子来,
\item 液面上的汽达到饱和时,还有没有液体分子从液面飞出?为什么这时从宏观上看来液体不再蒸发?
\item 饱和汽的密度怎样随温度而变化?饱和汽的压强怎祥随温度而变化?为什么这样变化?
\item 在温度不变的情况下,增大液面上饱和汽的体积时,下面的说法哪些是正确的:
\begin{enumerate}
    \item 饱和汽的质量不变,饱和汽的密度减小;
    \item 饱和汽的密度不变,饱和汽的压强也不变;
    \item 饱和汽的密度不变,饱和汽的压强增大;
    \item 饱和汽的质量增大,饱和汽的压强也增大;
    \item 饱和汽的质量增大,饱和汽的压强不变.
\end{enumerate}

\item 解释下面的现象:密闭容器中装有少量液态乙醚,当容器的温度升高时,液态乙醚逐渐减少;容器升高到一定温度
时,液态乙醚消失;容器冷却时,容器中又出现液态乙醚.
\end{enumerate}

\section{沸腾}
给液体加热,当液体升高到一定温度时,液体内部涌现出大量的气泡,升到液面破裂开,放出汽.这时整个液体发生剧烈的汽化,这种现象叫做\textbf{沸腾}.

现在我们来研究一下水的沸腾过程,给盛水的烧杯加热,原来吸附在杯底和杯壁上的空气以及溶解在水里的空气就分离出来,形成一些小气泡.由于周围的水向气泡里蒸发,所以气泡里包含的是饱和水汽和空气.杯底受热温度升高时,气泡膨胀,当体积胀大到一定程度时,气泡就脱离杯底浮起.在达到沸腾温度以前,气泡在上升过程中体积是逐渐缩小的(图5.6甲),这些小气泡升到液面破裂时,放出的主要是空气.
\begin{figure}[htp]
\centering
\includegraphics[scale=.6]{fig/5-6.png}
\caption{}
\end{figure}

当杯内水的温度都升高到某一温度,气泡内的饱和汽压等于外界压强时,气泡在上升过程中体积就不再缩小.并且
由于在上升过程中周围的水还在不断向泡内蒸发,所以体积还会继续增大,直到升到液面破裂开(图5.6乙).这时从气泡里放出的主要是水蒸气,这时水就沸腾了.沸腾时,在液体表面和液体内部同时发生汽化,这是沸腾和蒸发的主要区别.

从上面水的沸腾过程可以看出,\textit{液体只有在它的饱和汽压等于外界压强时才能沸腾.}在外界压强是1标准大气压的情况下,水在100$^\circ$C沸腾,就是因为100$^\circ$C时水的饱和汽压为760毫米汞柱,等于外界的大气压强.

液体沸腾时的温度叫做\textbf{沸点}.外部压强改变时,液体的沸点也改变.在外部压强增大时,液体需要升到较高的温度时饱和汽压才能等于外部压强,所以沸点升高.在外部压强
减小时,液体在较低的温度下饱和汽压就能等于外部压强,所以沸点降低.上述结论可以用实验来证明.把一杯温度低于100$^\circ$C的热水放在抽气机的玻璃罩里,抽出罩里的空气,当罩里空气的压强降低到一定程度时,热水就沸腾起来(图5.7).
\begin{figure}[htp]
\centering
\includegraphics[scale=.6]{fig/5-7.png}
\caption{压强减小时水的沸点降低}
\end{figure}

离海平面越高,大气压强越小,所以高山上水的沸点较低.珠穆朗玛峰顶的气压大约只有1/3标准大气压,在那里水烧到约72$^\circ$C就沸腾了,在海拔高的山上,由于沸点低,用
普通锅煮不熟饭,需要用高压锅.高压锅内的水蒸气不容易泄出来,锅内的压强可以达到2标准大气压,锅内水的沸点接近120$^\circ$C,所以用高压锅煮饭热得快.工业上用的蒸汽锅炉也是利用增大压强来提高沸点,获得高温高压蒸汽的.如果锅炉中压强接近大气压强的10倍,水的沸点就可提高到180$^\circ$C左右.

在相同的压强下,各种物质的沸点不同.例如,在1标准大气压下,乙醚的沸点为35$^\circ$C,酒精的沸点为78$^\circ$C.利用物质的这一性质,可以对液体混合物进行分馏:给液体混合物加热,其中沸点较低的成份先蒸发出来,沸点较高的成份后蒸发出来.汽油、煤油、柴油就是在炼油厂中对石油进行分馏而得到的.

\section{汽化热}
水沸腾后,虽然继续给它加热,但是水的温度并不上升,只是有更多的水变成了水蒸气.可见,水在沸腾过程中要吸收热量.液体变为蒸气时,体积膨胀,分子间的距离增大.沸腾过程中吸收的热量,一部分用来克服分子间的引力做功增大分子势能,另一部分用于体积膨胀时克服外界压强做功.

\textit{单位质量的某种液体,变为同温度的气体时吸收的热量,叫做这种液体的汽化热.}

汽化热常用字母$L$表示.在国际单位制中,汽化热的单位是焦/千克.

不同物质的汽化热不同.下表列出了一些物质在1标准
大气压下沸点时的汽化热.

\begin{center}
  \begin{tabular}{cccccc}
\hline
物质  & 沸点&汽化热&  物质  & 沸点 &汽化热\\
&($^\circ$C)&(${\rm J}/{\rm kg}$)&& ($^\circ$C)&(${\rm J}/{\rm kg}$)\\
 \hline
液态氦 &$-269$& $2.50\times 10^4$ &乙醚 &35& $3.52\times 10^5$\\
液态氢&$-253$& $4.53\times 10^5$ &酒精&78& $8.55\times 10^5$\\
液态氧&$-183$& $2.14\times 10^5$ &水&100& $2.26\times 10^6$\\
液态二氧化碳&$-78.5$& $2.30\times 10^5$ &水银&357& $2.89\times 10^5$\\
液态氨&$-33$& $1.37\times 10^6$ &液态铁&2750& $6.30\times 10^6$\\
 \hline
\end{tabular}
\end{center}

汽凝结为液体时要放出热量.实验表明,单位质量的汽凝结为液体时放出的热量,
等于在同一温度下液体变为相同温度的汽时的汽化热.

知道了物质的汽化热$L$,就能计算出质量为$m$的液体变为同温度的汽时吸收的
热量或质量为$m$的汽变为同温度的液体时放出的热量:
\[Q=Lm\]

在不同温度下,同一物质的汽化热不同.温度升高时,物质的汽化热变小,
这是因为温度升高时液体分子间的距离增大,液体变为汽时克服分子间的引力需
要做的功减小.

下表是水在不同温度下的汽化热.

\begin{center}
\begin{tabular}{cccc}
\hline
温度($^\circ$C)&汽化热(${\rm J}/{\rm kg}$)
&温度($^\circ$C)&汽化热(${\rm J}/{\rm kg}$)\\
\hline
0& $2.50\times 10^6$ & 200 & $1.96\times 10^6$ \\
50& $2.38\times 10^6$ & 300 & $1.38\times 10^6$ \\
100& $2.26\times 10^6$ & 370 & $4.14\times 10^5$ \\
\hline
\end{tabular}
\end{center}

\begin{figure}[htp]
\centering
\includegraphics[scale=.7]{fig/5-8.png}
\caption{测定水的汽化热}
\end{figure}

水的汽化热可以用图5.8所示的实验装置来测定.用酒
精灯把烧瓶A里的水加热至沸腾.从烧瓶A里出来的蒸汽,经过试管$B$
进入装着水的量热器小筒$C$中.蒸汽经过试管$B$后,其中所含的小水滴留下来,
使得进入量热器的蒸汽中不含有水滴.蒸汽在小筒$C$中凝结时放出热量,
使小筒里的水和小筒本身的温度升高,到水温升高到相当的度数时,停止通入蒸汽.
小筒$C$的质量$m$以及其中水的质量$m_1$和温度$t_1$已在通入蒸汽前测出,
停止通入蒸汽后,再测出水的末温度$t_2$和质量$m_2$,$m_2-m_1$就是通入水中的蒸汽的质量.
水和小筒从温度$t_1$升高到$t_2$吸收的热量等于100$^\circ$C的蒸汽变为$t_2$的水放出的热量,
利用这个关系就可以求出水在100$^\circ$C时的汽化热.

\subsection*{练习三}
\begin{enumerate}
 \item 在1标准大气压下,乙醚的沸点是35$^\circ$C,这个温度时乙醚的饱和汽压是多大?
  \item 锡的熔点是232$^\circ$C,但是用锡焊的水壶盛着水放在1000$^\circ$C以上的火上烧,锡并不熔解.为什么?
  \item 在蒸汽暖室装置的散热器里,每小时有20千克100$^\circ$C的水蒸气液化成水,
    并且水的温度降低到80$^\circ$C.求散热器每小时供给房间的热量.
    \item 某人在做测定水的汽化热实验时,得到的数据如下:
   钢制量热器小筒的质量为200克,通入水蒸气前筒内水的质量为350克,温度为14$^\circ$C;
    通入100$^\circ$C的水蒸气后水的温度为36$^\circ$C,水的质量为364克,他测得的水的汽化热是多少?
    \item 容器里装有0$^\circ$C的冰和水各500克,向里面通入100$^\circ$C的水蒸气后,
    容器里的水升高到了30$^\circ$C.假设容器吸收的热量很少,可以忽略不计,并且容器是绝热的,
    计算一下通入的水蒸气有多少?
\end{enumerate}

\section{气体的液化}
我们知道,当饱和汽的体积减小或温度降低时,它就凝结
为液体.因此,对于未饱和汽,如果能使它变为饱和汽,就能使它液化了,怎样才能使未饱和汽变为饱和汽呢?

\subsection{把未饱和汽变为饱和汽的方法}

在一定温度下,饱和汽的密度大于未饱和汽的密度.在保持温度不变的情况下,如果用增大压强的办法来减小未饱和汽的体积,增大它的密度,那么,当密度增大到等于该温度下饱和汽的密度时,未饱和汽
就成了饱和汽.这时进一步减小汽的体积,就能使饱和汽凝结成液体.

饱和汽的密度还跟温度有关系.温度高时,饱和汽的密度大;温度低时,饱和汽的密度小.在较高温度下由于密度小而未达到饱和的未饱和汽,在保持体积不变的情况下,如果降低它的温度,那么,当温度降至未饱和汽的密度等于该温度下饱和汽的密度时,未饱和汽就成了饱和汽,这时如果继续降低温度,饱和汽就会凝结成液体.

\subsection{临界温度}

利用增大压强和降低温度的方法可以把未饱和汽变成饱和汽,从而使它变为液体.用这种方法是否能使所有的气体液化呢?

十九世纪法拉第和其他一些科学家们在这方面进行了大量的工作,他们运用增大压强和冷却的办法,把许多气体都液化了,其中有氨、氯、二氧化硫、氯化氢、硫化氢、二氧化碳等.但是,研究中发现,有几种气体,例如氧、氢、氮等,一直不能被液化,于是当时便以为这些气体是不能液化的所谓“永久气体”.

后来,进一步的研究表明,各种气体都有一个特殊的温度,在这个温度以上,无论怎样增大压强也不能使气体液化,这个温度叫做\textbf{临界温度}.氧、氢、氮等气体所以没有被液化,就是因为它们的临界温度很低,当时的低温技术尚
未获得这样低的温度.于是科学家们便努力提高低温技术,结果在二十世纪初,所有的气体都被液化了,最后一个被液化的气体是氦,它于1908年被液化,后来还被凝固成了固态.

下表是一些物质的临界温度.

\begin{center}
    \begin{tabular}{cccc}
\hline
  物质 &临界温度($^\circ$ C)&物质 &临界温度($^\circ$ C)\\
\hline
  氦&$-268$  & 氨 & 132 \\
氢&$-240$  & 氯 & 144 \\
氮&$-147$  & 乙醚 & 194 \\
氧&$-119$  & 酒精 & 243 \\
二氧化碳&31  & 水 & 374 \\
\hline
  \end{tabular}
\end{center}

从上表可以看出,二氧化碳、氨、氯等气体的临界温度较高,都在室温以上,所以容易液化.而氧、氮、氢、氦的临界温度很低,所以较难液化.

液态气体有许多重要的应用.相同质量的液态气体的体积比气态小得多,便于贮存和运输,所以作为燃料的天然气常常液化后供应用户;火箭中用液态氢作燃料,用液态氧作氧化剂,液态空气蒸发时,沸点较低的氮气先蒸发,沸点较高的氧气后蒸发,因此利用液态空气可以分离出氧气和氮气.使容器中的空气液化,还可以获得压强很低的真空.

氮、氢、氦等气体的沸点很低,因此利用液态氮、氢、氦可以得到低温,低温在科学技术、医疗等方面都有重要的应用.例如,人造卫星在宇宙空间中运行时,向着太阳的一面温度很高,但是背着太阳的一面温度却很低,只有几开尔文,在这样低的温度下对卫星的材料性能有一些特殊的要求,这就要在地球上制造低温环境,进行模拟试验.低温可以治疗疾病,贮藏生物制品如血液、皮肤等,低温贮藏的细胞的活力可以保持近千年.在很低的温度下,某些物质具有特殊的性质,例如超
导电性、超流动性等,研究物质在低温下的各种性质,对认识物质的结构以及发展新技术都有重要的意义.

\section{空气的湿度}
泼在地上的水和江河湖海里的水都在蒸发,动植物的表皮和动物的呼吸也在不断地散发出水蒸气,所以我们周围的空气总含有水蒸气.一定体积的空气中含的水蒸气越多,空气就越潮湿;含的水蒸气越少,空气就越干燥.空气的干湿程度跟我们的生活和生产有密切的关系.空气太潮湿,我们会感到沉闷和窒息,东西也容易发霉;空气太干燥,我们的口腔和鼻腔会感到干燥难受,植物容易枯萎.在某些生产部门以及贮藏物品和保存名贵书画等艺术品的地方,如纺织厂、博物馆等,都要求空气保持一定的湿度.

空气的湿度可以用空气中所含水蒸气的密度,即单位体积的空气中所含水蒸气的质量来表示.由于直接测量空气中水蒸气的密度比较困难,而水蒸气的压强是随水蒸气密度的增大而增大的,所以通常都用空气中水蒸气的压强来表示空气的湿度,空气中所含水蒸气的压强叫微空气的绝对湿度.例如,空气里水蒸气的压强是15毫米汞柱,这时空气的绝对湿度就是15毫米汞柱.

空气湿度对蒸发的快慢、植物的枯萎、动物的感觉的影响不是由空气的绝对湿度来决定,而是跟空气中的水蒸气离饱和状态的远近有关系.由于饱和水蒸气的压强随温度的升高而增大,所以在空气的绝对湿度一定的情况下,气温高时,水蒸气离饱和状态远;气温低时,水蒸气离饱和状态近.例如空气
的绝对湿度是9毫米汞柱,在气温是20$^\circ$C时,水燕气离饱和状态较远(20$^\circ$C时水的饱和汽压是17.5毫米汞柱),我们就感到空气比较干燥;在气温是10$^\circ$C时,水蒸气接近饱和(10$^\circ$C时水的饱和汽压是9.2毫米汞柱),我们就感到空气很潮湿.为了表示空气中水蒸气离饱和状态的远近,物理学中引入了相对湿度的概念.

某温度时空气的绝对湿度跟同一温度下水的饱和汽压的百分比,叫做这时空气的\textbf{相对湿度}.

如果气温为20$^\circ$C时绝对湿度$p=9$毫米汞柱,因为20$^\circ$C时水的饱和汽压$P=17.5$毫米汞柱,所以这时空气的相对湿度
\[B=\frac{9}{17.5}\times 100\%=51\% \]
用公式来表示就是
\[B=\frac{p}{P}\times 100\% \]

不同温度下水的饱和汽压可以从查表得到.这样,知道了空气的绝对湿度,利用上面的公式就可以求出空气的相对湿度.反过来,如果知道了某一温度下的相对湿度,也可以算出绝对湿度.

在绝对湿度一定的情况下,气温降低时,相对湿度将增大.因此,在夏季有时感到白天比较干燥,夜晚比较湿润.

在住人的房间里,相对湿度为60—70\%比较适宜.

\begin{table}
    \centering
    \caption{不同温度下水的饱和汽压(单位:毫米汞柱)}
    \begin{tabular}{cccccccccccc}
\hline
    $t$ & $P$ &$t$ & $P$ &$t$ & $P$ &$t$ & $P$ &$t$ & $P$ &$t$ & $P$ \\
($^\circ$C)&&($^\circ$C)&&($^\circ$C)&&($^\circ$C)&&($^\circ$C)&&($^\circ$C)\\
\hline
$-$15  &  1.44  &  $-$3  &  3.67  &  9  &  8.61  &  21  &  18.65  &  33  &  37.73  &  45  &  71.88\\
$-$14  &  1.56  &  $-$2  &  3.96  &  10  &  9.21  &  22  &  19.83  &  34  &  39.9  &  50  &  92.51\\
$-$13  &  1.69  &  $-$1  &  4.26  &  11  &  9.84  &  23  &  21.07  &  35  &  42.18  &  60  &  149.38\\
$-$12  &  1.83  &  0  &  4.58  &  12  &  10.52  &  24  &  22.38  &  36  &  44.56  &  70  &  233.7\\
$-$11  &  1.99  &  1  &  4.93  &  13  &  11.23  &  25  &  23.76  &  37  &  47.07  &  80  &  355.1\\
$-$10  &  2.15  &  2  &  5.29  &  14  &  11.9  &  26  &  25.21  &  38  &  49.7  &  90  &  525.8\\
$-$9  &  2.33  &  3  &  5.69  &  15  &  12.79  &  27  &  26.74  &  39  &  52.44  &  100  &  760\\
$-$8  &  2.51  &  4  &  6.1  &  16  &  13.63  &  28  &  28.35  &  40  &  55.82  &  150  &  3570.5\\
$-$7  &  2.72  &  5  &  6.54  &  17  &  14.53  &  29  &  30.04  &  41  &  58.34  &  200  &  11659.2\\
$-$6  &  2.93  &  6  &  7.01  &  18  &  15.48  &  30  &  31.82  &  42  &  61.5  &  250  &  29817.8\\
$-$5  &  3.16  &  7  &  7.51  &  19  &  16.48  &  31  &  33.7  &  43  &  64.8  &  300  &  64432.8\\
$-$4  &  3.41  &  8  &  8.05  &  20  &  17.54  &  32  &  35.66  &  44  &  68.26  &  350  &  124001.6\\
\hline
    \end{tabular}
\end{table}

\subsection*{练习四}

\begin{enumerate}
	\item 说明使未饱和汽变为饱和汽的方法和道理.
\item 潮温的天气里,湿衣服不容易干,为什么?
\item 在绝对湿度相同的情况下,冬天和夏天的相对湿度哪个大?为什么?
\item 空气的绝对湿度是9毫米汞柱,气温是16$^\circ$C,相对湿度是多少?
\item 教室里空气的相对湿度是60\%,温度是18$^\circ$C,绝对温度是多少?
\end{enumerate}

\section{露点~~湿度计}
\subsection{露点} 
空气里的未饱和水蒸气,当气温逐渐降低时将逐
渐接近饱和,当气温降低到某一温度时,水蒸气达到饱和状态,这时将有水蒸气凝结成水,在物体表面上形成一层细小的露滴.

\textbf{使空气里的水蒸气刚好达到饱和时的温度,称为露点.}

空气中含的水蒸气多,气温只要少许降低一点,就达到露点,水蒸气就达到饱和;反之,空气中含的水蒸气少,气温要降低较多,才能达到露点,水蒸气才达到饱和,因此,根据露点和气温的差值,可以大致判断出空气中水蒸气的饱和程度,从而判断出相对湿度的大小.
\begin{figure}[htp]
\centering
\includegraphics[scale=.6]{fig/5-9.png}
\caption{测定露点}
\end{figure}

露点可以用图5.9所示的装置来测定.玻璃杯里装入乙醚,杯盖上有三个孔,分别插入温度计$A$和两根弯曲的玻璃管$B$、$C$.管$C$的一端插在乙醚中,另一端连接打气球.管$B$是出气用的.用打气球向乙醚里打气,乙醚就迅速蒸发,使杯子和周围空气的湿度降低,当降低到某一温度时,杯子周围空气中的水蒸气达到饱和,杯壁上就出现一层露滴,这时温度计指示的湿度就是露点,在这个装置中如果用表面光亮的金属
杯代替玻璃杯,更容易观察到露滴的出现,效果会更好.

测出了露点,从水的饱和汽压表中查出露点时的饱和汽压,这个饱和汽压就是空气在原来温度时的绝对湿度,知道了绝对湿度,再查出原来温度下的饱和汽压,就可以求出相对湿度.

\subsection{湿度计} 
既然测出露点就能求出空气的绝对湿度和相对湿度,所以测定露点的仪器就是一种湿度计.这种湿度计叫做\textbf{露点湿度计}.
\begin{figure}[htp]
\centering
\includegraphics[scale=.6]{fig/5-10.png}
\caption{干湿泡湿度计}
\end{figure}

还有两种常用的湿度计,一种叫做\textbf{干湿泡湿度计}(图5-10).它由两支完全相同的湿度计组成,湿度计$A$叫做干泡
湿度计,用来测量空气的湿度;湿度计$B$叫做湿泡湿度计,它的水银泡上包着棉纱,棉纱的下端浸在水中,由于水的蒸发,湿度计$B$指示的湿度总是低于$A$的.$A$、$B$的湿度差叫做干湿泡湿度差,空气的相对湿度越小,即空气越干燥,湿泡湿度计$B$上的水蒸发得越快,$B$的湿度就降得越低,两支湿度计的湿度差就越大;反之,空气的相对湿度越大,即空气越潮湿,湿度计$B$上的水蒸发得就越慢,$A$、$B$的湿度差就越小.所以,干湿泡湿度差的大小跟空气的相对湿度有直接关系,如果把不同湿度时相应于不同的干湿泡湿度差的相对湿度计算出来,绘制成表或画成曲线,那么根据干湿泡湿度计上$A$、$B$两支湿度计的读数,从表或曲线上很快就可以得出空气的相对湿度.

另一种常用的湿度计叫做\textbf{毛发湿度计}(图5.11).它是利用人的头发在脱脂以后,其长度会随着空气的相对湿度而变化制成的.毛发湿度计由一根或一束脱脂的毛发、指针和刻度盘组成.空气的相对湿度增大时,毛发伸长;相对湿度减小时,毛发缩短.毛发长度的变化控制指针的偏转,从刻度盘上就可以直接读出相对湿度.
\begin{figure}[htp]
\centering
\includegraphics[scale=.6]{fig/5-10.png}
\caption{毛发湿度计}
\end{figure}

三种湿度计各有不同的优缺点.露点湿度计测量准确,但是结构比较复杂,测出露点后要进行查表、计算等,使用起来不太方便.干湿泡湿度计使用比较方便,也比较准确,所以生活中大都使用这种湿度计.毛发湿度计结构简单,不易损坏,可以直接读数,还可以和自动记录装置联合使用,缺点是不太准确,要经常进行校准.

\subsection*{练习五}

\begin{enumerate}
	\item 在北方,冬天戴着眼镜从寒冷的室外进入温暖的空内时,镜片上常出现一层细小的露滴.这是为什么?
\item 白天空气的绝对湿度是13.7毫米汞柱.天气预报夜里的最低温度是14$^\circ$C,如果空气的绝对湿度保持不变,夜里会不会出现露水?
\item 如果干湿泡湿度计上两支温度计的指示数字相同,这时空气的相对湿度是多少?
\item 空气的温度是20$^\circ$C,露点是12$^\circ$C,这时的绝对湿度和相对湿度是多少?
\item 空气的温度是25$^\circ$C,相对湿度是50\%,气温降低到多少摄氏度时,才会有露出现?
\end{enumerate}

\section*{阅读材料:过热液体、过冷液体和过饱和汽}
\subsection*{过热液体} 

从液体沸腾现象知道,吸附在容器壁上的空气和溶解在液体中的空气对沸腾起着重要的作用.这些空气在液体被加热时形成气泡,使得周围的液体能够向里面蒸发,成为液体内部的汽化核.在液体沸腾,气泡上升到液面破裂时,空气被放出,这样液体中的空气便越来越少,经过多次煮沸的液体,由于里面的空气已经放尽,所以加热到沸点以上也不沸腾,这种液体叫做过热液体.过热液体是不稳定的,如果有尘屑进入,或由于液体分子的运动,液体内部自发地产生极小的气泡,周围的高温液体就会迅速地向气泡内蒸发,使液
体突然剧烈地沸腾起来,发生暴沸.暴沸时甚至会使容器爆炸.为了避免这种情况,锅炉中的水在加热前应加进一些溶有空气的新水,或放进一些吸附有空气的无釉陶瓷碎块,在高能物理研究中常应用过热液体来探测高能粒子的运动径迹,使容器中的透明液体(例如液体氦、氢、丙烷、戊烷等)处于过热状态,当带电粒子通过液体时,就发生沸腾,结果在粒子经过的地方产生大量气泡,从而显示出粒子的径迹.这种仪器叫做气泡室.

\subsection*{过冷液体} 
液体凝固为晶体,需要有晶核存在.液体中的尘埃、杂质等微粒都可以作为晶核,纯净的液体由子没有晶核,在温度降低到凝固点以下时也不会凝固,这种液体叫做过冷液体.过冷液体是不稳定的,如果有尘屑进入,就立即开始凝固.

在0$^\circ$C以下的云雾中,常有一些过冷水滴.飞机在过冷水滴组成的云层中飞行时,由于机翼、机身等与水滴相碰撞,就会在这些部位结成冰层,因而增加飞机的重量,甚至使某些部位的机械操纵失灵.因此,需要采取措施防止结冰.

\subsection*{过饱和汽} 
在通常情况下,当温度降低到使未饱和汽变为饱和汽时,蒸气就凝结为液体.这是因为一般蒸气中都含有尘埃和杂质,成为蒸气凝结的凝结核,对于纯净的蒸气,即使温度降低到使它超过饱和的程度,蒸气也不凝结为液体.这种超过饱和汽密度而仍不液化的蒸气,叫做过饱和汽.过饱和汽也是不稳定的,如果其中有凝结核出现,就会发生凝结,使蒸气回到饱和状态.在原子核物理研究中观测微观粒子径迹的仪器“威尔逊云室”就利用了过饱和汽.这种仪器我们将
在高中物理第三册原子核一章中讲到.


\section*{复习题}
\begin{enumerate}

	\item 品体和非品体在熔解和凝固时有什么不同?怎样从它们的微观结构来说明这种不同?
	\item 什么是物质的熔解热?
		\item 什么是蒸发?怎样用分子运动论的观点来解释影响蒸发快慢的各种因素?蒸发为什么能致冷?
	\item 什么是沸腾?液体沸腾的条件是什么?为什么外界压强增大时沸点升高,压强减小时沸点降低?
	\item 什么是物质的汽化热?
	\item 什么叫饱和汽?什么叫未饱和汽?饱和汽的密度和压强跟温度有什么关系?跟体积有什么关系?怎样解释这种关系?
	\item 怎样才能使气体液化?什么叫做气体的临界温度?
	\item 什么叫做空气的绝对湿度和相对湿度?在绝时湿度保持不变的情况下,气温不同时相对湿度是否相同?为什么?
	\item 什么叫做露点?测出露点,怎样求出空气的绝对湿度和相对湿度?
	\item 干湿泡湿度计和毛发湿度计各是利用什么现象制做的?
\end{enumerate}



\chapter{电场}
人类很早就认识了磁现象和电现象,例如,我国在战国末期就发现了磁铁矿吸引铁的现象,在东汉初年就有带电的琥珀吸引轻小物体的文字记载.但是,人类对电磁现象的系统研究却是在欧洲文艺复兴之后逐渐开展起来的,到十九世纪才建立了完整的电磁学理论.电磁学及其应用对人类的影响十分巨大.电力工业和电子技术是四个现代化建设的重要部门,电磁学理论是人们探索客观世界的有力武器.所以我们应当学好电磁学.

\section{两种电荷~~电荷守恒定律}
\subsection{两种电荷} 
在初中已经学过,用毛皮摩擦过的硬橡胶棒,或用丝绸摩擦过的玻璃棒都能吸引轻小物体,即它们都带上了电荷.玻璃棒上带的电荷叫正电荷,硬橡胶棒上带的电荷叫负电荷.自然界只存在两种电荷,而且同种电荷互相排斥,异种电荷互相吸引.电荷是有多有少的,荷的多少叫做电量.
\begin{figure}[htp]\centering
	\includegraphics[scale=.6]{fig/6-1.png}
	\caption{}
	\end{figure}

如图6.1甲所示,让验电器带上适量正电荷,这时验电器的金属箔张开.如果用带正电的玻璃棒接触验电器的金属球,把正电荷传给验电器,金属箔张开的角度就变大(图6.1乙);
如果用带负电的硬橡胶棒接触验电器的金属球,把负电荷传给验电器,金属箔张开的角度就变小(图6.1丙).可见,同种电荷放在一起互相增强,异种电荷放在一起互相减弱或抵消.通常,正电荷的电量用正数来表示,负电荷的电量用负数来表示.

等量的异种电荷完全相互抵消的现象叫做中和.我们知道,物体是由原子组成的,原子是由带正电的原子核和带负电的电子组成的,在通常的情况下,原子核所带的正电荷的电量(绝对值)等子所有电子所带负电荷的电量的总和(绝对值),原子呈中性状态,物体也呈中性状态,即对外表现为不带电的状态,任何不带电的物体,其中都有等量的正负电荷,因而处于中性状态.

使物体带电叫做起电.起电的过程,实际上是使物体中的正负电荷分开的过程.在摩擦起电中,其中一个物体因失去一些电子而带正电,同时另一个物体因得到这些电子面带等量的负电,摩擦起电并不是创造了电荷,只是电荷从一个
物体转移到另一个物体.

\subsection{静电感应} 

还有一种常见的使物体带电的方法叫感应起电,取一对用绝缘柱支持的金属导体$A$和$B$,导体上都贴有金属箔,让$A$和$B$彼此接触,这时$A$和$B$上的金属箔闭合,表示它们都没有带电.把另一个带正电的金属球$C$移近导体$A$(图6.2甲),这时$A$、$B$上的金属箔都张开了,表示它们都带了电.实验表明,靠近$C$的导体$A$带的电荷与$C$异号,远离$C$的导体$B$带的电荷与$C$同号,这种现象叫做\textbf{静电感应}.如果
先把$A$和$B$分开,然后移去$C$,则发现$A$和$B$仍带有电荷(图6.2乙).如果再让$A$和$B$重新接触,它们就呈现不带电的状态.这说明:$A$和$B$分开后所带的异种电荷是等量的,重新接触后等量异种电荷相互抵消.利用静电感应使物体带电的方法叫做\textbf{感应起电}.

\subsection{电荷守恒定律} 

静电感应也是使物体中的电荷分开,当我们把带正电的导体$C$移近绝缘导体时(图6.2乙),绝缘导体里的自由电子被吸引过来,因此导体两端分别带上等量异种电荷.可见,静电感应也不是创造了电荷,只是电荷从物体的一部分转移到另一部分.
\begin{figure}[htp]\centering
	\includegraphics[scale=.6]{fig/6-2.png}
	\caption{静电感应}
	\end{figure}

大量事实说明:\textit{电荷既不能创造,也不能被消灭,它们只
能从一个物体转移到另一个物体,或者从物体的一部分转移到另一部分},这个结论叫做\textbf{电荷守恒定律}.它是物理学中重要的基本定律之一.

\subsection*{练习一}
\begin{enumerate}
\item 把支在绝缘座上的不带电的导体$A$移近带电体$B$,用手指接触一下$A$,然后移开手指,握住绝塚座移开导体$A$,导体$A$就带电了.如果带电体$B$原来带正电,导体$A$将带什么电?做这个实验并作出解释,实验时可用验电器来检查导体$A$是否带电和带什么电.
\item 在图6.1中,先让验电器带上少量正电荷,然后拿一个带负电的带电体逐渐接近验电器的金属球,可以看到这
样的现象:金属箔张开的角度先是减小,以至闭合,然后又张开了.解释这个现象.
\end{enumerate}


\section{库仑定律}
\subsection{库仑定律} 

两个电荷间的相互作用力,跟它们的电量有关系,还跟电荷间的距离有关系,法国物理学家库仑(1736—1806)用实验研究了静止的点电荷间的相互作用力,于1785年发现了后来用他的名字命名的定律.

什么是点电荷呢?如果带电体间的距离比它们的大小大得多,以致带电体的形状和大小对相互作用力的影响可以忽略不计,这样的带电体就可以看成是点电荷,跟力学中质点的概念类似,点电荷这个概念也是一种科学的抽象,是一种理
想化的模型.

\begin{figure}[htp]\centering
	\includegraphics[scale=.7]{fig/6-3.pdf}
	\caption{库仑扭秤}
	\end{figure}

库仑是用图6.3所示的扭秤来做实验的.扭秤的主要部分是在一根细金属丝下面悬挂一根玻璃棒,棒的一端有一个
金属小球$A$,另一端有一个平衡小球$B$,在离$A$球某一距离的地方再放一个同样的金属小球$C$,如果$A$球和$C$球带同种电荷,它们间的斥力将使玻璃棒转过一个角度.向相反方向扭转旋钮$M$,使玻璃棒回到原来的位置并保持静止,这时金属丝扭转弹力的力矩跟电荷间斥力的力矩平衡.因此从旋钮$M$转过的角度可以计算出电荷间作用力的大小.

库仑的实验是要研究电荷间的相互作用力跟它们间的距离和电量的关系.作用力跟距离的关系比较好办,保持两球的电量不变,改变两球的距离并测出作用力,就可以找出作用力跟距离的关系.困难在于作用力跟电量的关系,因为当时还不知道怎样测量电量,甚至连电量的单位也没有确定.库仑找到了一个简单办法巧妙地解决了这个问题.他把一个带电的金属球跟同样的但不带电的金属球相碰,两球带的电量一定相等,都是原有电量的1/2.同样可以得到原有电量的
1/4、1/8等等的电量.这样就可以用扭秤来研究电荷间的作用力跟电量的关系了,库仑实验是在空气中做的,其结果跟在真空中相差很小,库仑实验的结果是:\textbf{在真空中两个点电荷间的作用力跟它们的电量的乘积成正比,跟它们间的距离的平方成反比,作用力的方向在它们的连线上},这就是\textbf{库仑定律}.电荷间的这种作用力叫做\textbf{静电力},又叫做\textbf{库仑力}.

如果用$Q_1$、$Q_2$表示两个点电荷的电量,用$r$表示它们间的距离,用$F$表示它们间的静电力,库仑定律就可以写成下面的公式:
\begin{equation}
F=k\frac{Q_1Q_2}{r^2}
\end{equation}
式中$k$是比例恒量,叫\textbf{静电力恒量}.它的数值和单位由式中各量的单位决定.

我们很容易看出,库仓定律和万有引力定律很相似,它们都是平方反比定律,人们现在还不能说明为什么这两个定律如此相似,但这种相似使我们可以用力学的比喻来理解许多电学问题,给我们的学习带来不少方便.

\subsection{电介质中的库仑定律} 

所谓电介质,就是我们在初中学过的绝缘体.空气、煤油、水、玻璃、橡胶、瓷器等都是电介质.

如果把两个电荷放在电介质里,例如放在煤油里,电荷间的作用力就比在同样情形下在真空里的作用力小,小多少,依电介质的不同而不同.这时库仑定律用下面公式来表示:
\begin{equation}
F=k\frac{Q_1Q_2}{\varepsilon r^2}
\end{equation}

每一种电介质的$\varepsilon$的数值是一定的,叫做那种物质的\textbf{介电常数},下表是几种电介质的介电常数.实用上,通常把空
气的介电常数取为1,即认为电荷间的作用力在空气中跟在真空中一样.

\begin{center}
	\begin{tabular}{c|ccccccc}
\hline
电介质 & 空气&煤油&石蜡&陶瓷&玻璃&云母&水\\
\hline
介电常数 & 1.0005&2&2.0—2.1&6&4—11&6—8&81\\
\hline
	\end{tabular}
\end{center}

\subsection{电量的单位} 
在国际单位制中,电量的单位就是我们在初中学过的\textbf{库仑},简称\textbf{库},国际符号是C.

采用国际单位制,在库仑定律的公式中,力的单位用牛,距离的单位用米,电量的单位用库,各个量的单位都已经分别确定,这时静电力恒量的数值要由实验来确定,实验指出:公式中的静电力恒量$k=9.0\times 10^9{\rm N}\cdot {\rm m}^2/{\rm C^2}$.

\begin{example}
比较电子和质子间的静电引力和万有引力,已
知电子质量是$0.91\x 10^{-30}$千克,质子质量是$1.67\x 10^{-27}$
千克,电子和质子的电量都是$1.60\x10^{-19}$
库.
\end{example}

	
\begin{solution}
电子和子间的静电引力$F_{\text{电}}$和万有引力$F_{\text{引}}$分别是
\[F_{\text{电}} =k\frac{Q_1Q_2}{r^2} ,\qquad   F_{\text{引}}=G\frac{m_1m_2}{r^2} \]
因此,
\[\frac{F_{\text{电}}}{F_{\text{引}}}=\frac{kQ_1Q_2}{Gm_1m_2} \]

上式中$m_1$、$m_2$和$Q_1$、$Q_2$已知,$k=9.0\x 10^9{\rm N}\cdot {\rm m^2}/{{\rm C}^2}$,
$G=6.67\x10^{-11}{\rm N}\cdot {\rm m^2}/{{\rm kg}^2}$.把数值代入进行计算,得
\[\begin{split}
	\frac{F_{\text{电}}}{F_{\text{引}}}&=\frac{9.0\x 10^9\x1.60\x10^{-19}\x1.60\x10^{-19}}{6.67\x10^{-11}\x 1.67\x10^{-27}\x 0.91\x10^{-30}}\\
	&=2.3\x 10^{39}
\end{split}\]
\end{solution}


从这个例题可以看出,电子和质子间的万有引力比它们
的静电引力小得多.正是因为这个缘故,在研究微观带电粒
子间的(如原子中电子和原子核间的)相互作用时,经常把万
有引力忽略不计.



\subsection*{练习二}

\begin{enumerate}
	\item 1库仑的电量是电所带电量的多少倍?
	\item 在真空中有两个点电荷,电量分别为$+4.0\x10^{-9}$库
和$-2.0\x10^{-9}$库,相距10厘米,这两个点电荷间的作用力
是多大?用电荷的绝对值代入进行计算,求出力的大小,然后
根据电荷的正负确定是引力还是斥力.
\item 在真空中有两个点电荷,保持它们的距离不变,它们
间的相互作用力在下列情况下将如何变化?
\begin{enumerate}
	\item 一个电荷的电量变为原来的2倍;
	\item 两个电荷的电量都变为原来的1/2.
\end{enumerate}
\item 原子核的半径大约为$10^{-14}$米,假定核中两个质子相距这么远,其间的静电力大约有多大?
\item 两个带电小球在煤油中相距0.5米,其中一个小球
带电$5.0\x10^{-9}$库,另一个带电$3.0\x10^{-8}$库,求小球间的作
用力.
\item 当两个点电荷相距为$r$时,它们间的斥力为$F$.改
变电荷间的距离,当斥力为$16F$时,相距为多少?当斥力为
$\dfrac{1}{4}F$时,相距为多少?
\end{enumerate}


\section{电场~~电场强度}

\subsection{电场}
电荷间的相互作用是怎样发生的呢?

经过长期的科学研究,人们认识到:电荷之间的相互作用
是通过电场发生的,只要有电荷存在,电荷周围就存在着电
场;电场的基本性质是它对放入其中的电荷有力的作用,这种
力叫做电场力,两个电荷$A$和$B$,电荷$A$受到的电荷$B$的作用,
实际上是电荷$B$的电场对电荷$A$的作用.同样,电荷$B$受到的
电荷$A$的作用,实际上是电荷$A$的电场对电荷$B$的作用.

历史上人们对场的认识是逐步深入的,我们理解场这个
概念也要有个过程,要逐步体会它的意义.引入场这个概念,
是对物理学的重要贡献.场的概念引入物理学之后取得了巨
大的成果,除了电场,我们在初中还学过磁场,电场和磁场是
有联系的,常常总称为电磁场,关于电磁场的研究导致了发现
电磁波.我们大家都熟悉的广播和电视就是借电磁波来传播
的,电磁波可以脱离电荷而独立存在并以光速传播,它跟由
原子、分子组成的物质一样具有能量和动量.这样,人们逐渐
认识到:电磁场包括电场和磁场是物质的一种特殊形态.

电场这种物质跟由分子、原子组成的物质不同,看不见,
摸不到,好象不好理解.其实,电场跟其他物质一样,都是不
依赖于我们的感觉而客观存在的东西.在一位现代物理学家
看来,电场正象他所坐的椅子一样是客观存在.电场是在跟
电荷的相互作用中表现出自己的特性的.我们从电场所表现
出来的特性出发,加以分析研究,就可以懂得电场,认识
电场.

\subsection{电场强度} 

刚刚说过,电场的基本性质是它对放入其中
的电荷有电场力的作用,现在来分析研究这个问题.

要研究电场,必须在电场中放入电荷,而这个电荷应该
是一个电量很小的点电荷.电量很小,是为了使它放入之后,
不致影响原来要研究的电场.体积很小,是为了便于用它来
研究电场各点的性质.这样的电荷常常叫做检验电荷.
\begin{figure}[htp]\centering
	\includegraphics[scale=.6]{fig/6-4.png}
	\caption{}
	\end{figure}

如图6.4所示,电场是由正电荷$Q$产生的,用挂在丝线下
端的带正电的小球作检验电荷,把它先后放在电场中不同的
位置,观察它在电场中的受力情况,力的大小可以从丝线对竖
直线偏角的大小看出.实验表明,检验电荷在电场中的位置
不同,受到的电场力的大小和方向也不同.检验电荷受到的
电场力大,说明那点的电场强;检验电荷受到的电场力小,说
明那点的电场弱.图6.4乙中$A$点的电场强,$B$点的电场弱,
$C$点更弱.

物理学中怎样来表示电场的强弱呢?把检验正电荷$q$放
到电场中的$A$点,电荷$q$受到电场力$F_A$的作用(图6.4乙).设
$A$点跟$Q$的距离为$r_1$,从库仑定律知道$F_A=kQq/r^2_1$.同样,如
果把正电荷$q'$放入$A$点,$q'$受到的电场力$F'_A=kQq'/r_1^2$.可
以看出,
\[\frac{F_A}{q}=\frac{F'_A}{q'}=\frac{kQ}{r^2_1} \]
这就是说,放入$A$点的电荷受到的电场力跟它的电量的比值,
是一个跟放入该点的电荷无关的恒量.

如果把电荷放入电场中的$B$点和$C$点,设$B$、$C$跟$Q$的距
离分别为$r_2$和$r_3$,同样可以证明,电荷在$B$和$C$受到的电场
力跟它的电量的比值分别是$kQ/r^2_2$、$kQ/r_3^2$,都是跟放入的电
荷无关的恒量.

可见,电荷在电场中某一点受到的电场力跟它的电量的
比值,由该点在电场中的位置所决定,跟放入的电荷无关.这
个比值越大的地方,放入那里的单位电荷受到的电场力越大,
电场就越强.这一点不仅对正电荷$Q$产生的电场是适用的,
对任何电场都是适用的.这就是说,对任何电场,都要用上述
比值来表示电场的强弱.

放入电场中某一点的电荷受到的电场力跟它的电量的比
值,叫做这一点的\textbf{电场强度},简称为\textbf{场强}.跟力一样,电场强
度也是矢量,如果用开表示电场强度,用$F$表示检验电荷$q$受
到的电场力,那么
\begin{equation}
	E=\frac{F}{q}
\end{equation}

由(6.3)式可以知道,如果$q$为单位正电荷,那么$E$和$F$在
数值上相等,可见电场中某一点的场强在数值上等于单位正
电荷在那一点所受的电场力.正负电荷在电场中某点所受电
场力的方向相反,我们规定场强的方向是正电荷受力的方向.
这样,负电荷受力的方向跟场强的方向相反.

场强的单位是牛/库.电场中的某一点,如果1库的点电
荷在该点受到的电场力是1牛,这点的场强就是1${\rm N}/{\rm C}$.

从上面讲的很容易知道,点电荷$Q$在真空中形成的电场
中,在距离$Q$为$r$的$P$点的场强$E$的大小为
\begin{equation}
	E=\frac{kQ}{r^2}
\end{equation}

这个公式只适用于真空,如果在充满电介质的空间里,这
个公式应改写成
\begin{equation}
	E=\frac{kQ}{\varepsilon r^2}
\end{equation}

如果$Q$是正电荷,$E$的方
向就背离$Q$;如果$Q$是负电荷,
$E$的方向就指向$Q$(图6.5).
\begin{figure}[htp]
	\centering
\begin{tikzpicture}[>=stealth, thick]
\draw[->] (0,0)node[left]{$Q$}--(1,2)node[left]{$P$}--(1.5,3)node[above]{$E$};
\draw (0,0) [fill=black] circle (1.5pt);
\draw (1,2) [fill=black] circle (1.5pt);
\end{tikzpicture}
\qquad 	\begin{tikzpicture}[>=stealth, thick]
		\draw[->] (0,0)node[left]{$Q$}--(1,2)node[left]{$P$}--(.5,1)node[below]{$E$};
		\draw (0,0) [fill=black] circle (1.5pt);
\draw (1,2) [fill=black] circle (1.5pt);
	\end{tikzpicture}
	\caption{场强的方向}
	\end{figure}

应该注意,(6.3)式、(6.4)式和
(6.5)式虽然都表示电场中某点
的场强,但它们的意义是不同
的.(6.3)式是场强的定义式,对任何电场都适用.(6.4)式是点电荷在真空中各点场强的计算
式,只适用于点电荷在真空中的电场.(6.5)式是点电荷在充满
电介质空间里各点场强的计算式.

如果有几个点电荷同时存在,它们的电场就互相叠加,形
成合电场,这时某点的场强,就等于各个点电荷在该点产生
的场强的矢量和.这样,知道了点电荷的场强,原则上我们就
可以知道任一带电体的场强,因为任何带电体都可以看作是
由许多点电荷组成的.

\section*{阅读材料:用比值定义物理量}
在物理学中,常常用比值来定义一个物理量.

我们在初中学过密度,在那里我们学到:单位体积的某
种物质的质量,叫做这种物质的密度.其实,密度也可以用比
值来定义,某种物质的物体,它的质量与它的体积成正比.因
此我们可以这样表达密度的定义:某种物质组成的物体,它的
质量和它的体积的比值,叫做这种物质的密度,对某种物质
来说,这个比值是恒定的;对不同的物质来说,这个比值一般
并不相同,因此,密度表示物质的一种特性.

初中学过的电阻是用比值来定义的.一段导体中的电流
强度跟加在这段导体上的电压成正比,导体两端的电压跟通
过导体的电流强度的比值,叫做这段导体的电阻.当温度保
持不变时,对某段导体来说,这个比值是恒定的;对不同的导
体来说,这个比值一般并不相同.这个比值越大,表示导体对
电流的阻碍作用越大.因此,电阻是表示导体对电流阻碍作
用的物理量.

刚刚学过的电场强度也是用比值来定义的.放入电场中
某点的电荷受到的电场力跟它的电量成正比.放入电场中某
点的电荷受到的电场力跟它的电量的比值,叫做这一点的电
场强度.对电场中的某一点,这个比值是恒定的;对电场中的
不同的点,这个比值一般并不相同.这个比值越大,表示那一
点的电场越强.电场强度是表示电场的力的性质的物理量.

不再一一列举.我们看到这里有一个共同点,那就是要
在实验的基础上寻求一个只与所研究的物体或场有关的比
值,来表示物体或场的某种性质,并由这个比值定义一个新
的物理量.在定义这个新的物理量的同时,也就确定了这个
新的物理量与已有物理量之间的关系.例如,定义了密度,
也就确定了密度、体积和质量这三者的关系.这里我们一定
要注意到这个比值是反映物体或场的什么性质,这样才能很
好地理解它的物理意义.

在今后的学习中,还会遇到用比值来定义的物理量.因
此,我们要很好地体会这种定义物理量的方法.

\subsection*{练习三}

\begin{enumerate}
	\item 在正电荷$Q$的电场中的某一点放一个电荷,它的电
	量$q=10^{-8}$库,$q$受到的电场力为$10^{-8}$牛.求这一点的电场
	强度$E$,并指出电场强度的方向.如果取走$q$,$E$有无变化?
	为什么?
	\item 在以水为介质的负点电荷$Q$的电场中,离$Q$0.5米处
	的电场强度$E$是1$\NC$,求负电荷Q的电量是多少库.
	\item 电场中某点的场强是$0.2\x10^5\NC$.求电量为
	$2\x10^{-8}$库的正电荷在该点受到的电场力是多大.
	\item 在氢原子中,电子和质子的平均距离是$5.3\x10^{-11}$
	米.质子在这个距离处产生的场强是多大?方向如何?电子
	受到的力是多大?方向如何?
	\item 物理学上常把重力作用的空间叫做\textbf{重力场}.如果把
	单位质量的物体受到的重力叫做重力场强度,试写出重力场
	强度的定义式.重力场强度的方向如何?从重力场强度的方
	向来看,重力场是跟正电荷形成的电场相似,还是跟负电荷形
成的电场相似?
\item 电场强度的定义式$E=F/q$在电介质中要不要改
写?为什么?
\end{enumerate}


\section{电力线}
\subsection{电力线}

研究电场,重要的是要知道电场中各点场强的
大小和方向.如果能够用图形把电场中各点场强的大小和方
向形象地表示出来,这对我们认识电场是很有好处的.图6.6
是在正电荷和负电荷的电场中画出的一组矢量,表示出场内
一些点的场强的大小和方向.但是,更好的办法是用英国物
理学家法拉第(1791—1867)引入的电力线来形象地表示
电场.
\begin{figure}[htp]\centering
	\includegraphics[scale=.6]{fig/6-6.png}
	\caption{}
	\end{figure}

在任何电场中,每一点的场强$B$都有一定的方向,所以
我们可以在电场中画出一系列的从正电荷出发到负电荷终止
的曲线,使曲线上每一点的切线方向都跟该点的场强方向一
致,这些曲线就叫做\textbf{电力线}.图6.7是一条电力线,它上面的
$A$、$B$点的场强$E_A$、$E_B$在各该点的切线上,方向如图中箭头
所示.
\begin{figure}[htp]\centering
	\includegraphics[scale=.3]{fig/6-7.pdf}
	\caption{}
	\end{figure}

电力线的形状可以用实验来观察.把奎宁的针状结晶、
木屑或头发屑悬浮在蓖麻油里,再放入电场中,就可以看到微
屑按照场强的方向排列起来(图6.8),显示出电力线的形状.
应该注意,虽然我们可以用实验来显示电力线的形状,但电力
线并不是电场里实际存在的线,而是人们为了使电场形象化
而假想的线.
\begin{figure}[htp]\centering
	\includegraphics[scale=.6]{fig/6-8.png}
	\caption{}
	\end{figure}

\begin{figure}[htp]
	\centering
	\begin{minipage}[t]{0.48\textwidth}
	\centering
	\includegraphics[scale=.5]{fig/6-9-1.pdf}
	\caption*{甲:正电荷}
	\end{minipage}
	\begin{minipage}[t]{0.48\textwidth}
	\centering
	\includegraphics[scale=.5]{fig/6-9-2.pdf}
	\caption*{乙:负电荷}
	\end{minipage}
	\caption{点电荷的电力线}
	\end{figure}
	
图6.9是点电荷的电力线,图6.10是两个等量的电荷的
电力线.从图中可以看出,在离形成电场的电荷越近的地方,
也就是场强越大的地方,电力线越密.所以,用电力线不但可
以形象地表示电场强度的方向,还可以表示电场强度的大小:
场强越大的地方电力线越密,场强越小的地方电力线越稀.
\begin{figure}[htp]
	\centering
	\begin{minipage}[t]{0.48\textwidth}
	\centering
	\includegraphics[scale=.6]{fig/6-10-1.pdf}
	\caption*{甲:等量异种电荷}
	\end{minipage}
	\begin{minipage}[t]{0.48\textwidth}
	\centering
	\includegraphics[scale=.6]{fig/6-10-2.pdf}
	\caption*{乙:等量同种电荷}
	\end{minipage}
	\caption{两个等量电荷的电力线}
	\end{figure}
	


\subsection{匀强电场} 

在电场的某一区域里,如果各点的场强的大
小和方向都相同,这个区域的电场就叫做\textbf{匀强电场}.匀强电
场是最简单的同时也是很重要的电场,在实验研究中常常要
用到它.

在匀强电场里,既然各点的场强的方向都相同,电力线就
一定是互相平行的直线;既然各点的场强的大小都相同,电力
线的疏密程度也一定处处相等.因此,匀强电场中的电力线
是距离相等的互相平行的直线.

两块靠近的平行金属板,
它们的大小相等并且互相正
对,在分别带等量的正电和负
电的时候,它们之间的电场,除
边缘附近外,就是匀强电场(图
6.11).

\begin{figure}[htp]\centering
\includegraphics{fig/6-11.pdf}
\caption{匀强电场}
\end{figure}

\section*{阅读材料:法拉第和场的概念}
相隔一定距离的电荷或磁体间的相互作用是怎样发生
的,这个问题在历史上有过长期的争论.十九世纪前期,大部
分物理学家认为电荷或磁体间的相互作用是超距作用.所谓
超距作用是指这种作用不需要任何媒质传递,就能够由一个
物体立即作用到另一个物体.

然而法拉第通过实验发现,电作用或磁作用跟电荷之间
或磁体之间的媒质有关.他在不同的媒质中进行同样的实验,
其作用效果不同.这引起他对电磁作用本质的深思.法拉第
认为,电磁力不可能是超越空间并与空间中媒质无关的超距
作用.法拉第提出了关于电磁作用的新看法:电荷或磁体在
周围空间产生电场或磁场,正是通过场,才把电作用或磁作用
传递到别的电荷或磁体.

经典力学是以超距作用为基础的,空间中除了粒子以外
什么也没有,没有粒子的地方是一无所有的真空;粒子间的相
互作用是超距作用,不需要通过媒质传递.法拉第提出的场
的模型从基本概念上突破了经典力学的框架,为建立近代物
理开创了新的起点.

法拉第凭着敏锐的直觉不仅提出了场的概念,而且描绘
出一幅清晰的场的图象.他用电力线或磁力线形象地表示电
场和磁场,力线密的地方场就强,力线疏的地方场就弱.力线
上每一点的切线方向表示场强的方向,法拉第用这幅图象解
释了用经典力学无法解释的现象,例如,1831年他发现了电磁
感应现象,他借助于磁力线对这一现象很快地作出了解释:只
要通过闭合电路的磁力线数目发生变化,电路里就会产生
电流.

法拉第提出的场的概念还处于萌芽状态.后来麦克斯韦
用数学方程定量地描述了电磁场的定律,预言了电磁波的存
在,并且把光现象和电磁现象联系起来,得出光波是一种电磁
波的结论.场的概念取得了很大成功,并逐渐在物理学中取
得了主要地位,是基本的物理概念之一.

\subsection*{练习四}

\begin{enumerate}
	\item 有人说电力线就是带电粒子在电场中运动的轨迹.这种说法对吗?为什么?
	\item 在图6.9到图6.11中,所有的电力线都不相交,我们能否断言,电场中任何两条电力线都不相交,为什么?
\end{enumerate}

\section{电场中的导体}
我们在初中学过,导体的特征是它的内部有大量的可以
移动的自由电荷.对于金属导体来说,这种自由电荷就是自
由电子.金属原子的最外层电子跟原子核的联系很弱,在其
余原子的作用下会脱离原来的原子而在整块金属中自由“游
荡”,成为自由电子.失去了外层电子的原子变成带正电的离
子,在平衡位置附近做热振动.所以,整块金属就是由做热振
动的正离子和在它们之间做无规则的热运动的自由电子组
成的.

\subsection{静电平衡状态} 
\begin{figure}[htp]\centering
	\includegraphics[scale=1]{fig/6-12.pdf}
	\caption{}
	\end{figure}

把一个不带电的金属导体$ABCD$放到场
强为$E_0$的电场中,导体内部的自由电子受到电场力的作用,
将向电场的反方向做定向移动(图6.12甲).这样,在金属的
$AB$面上将出现负电荷,在$CD$面上将出现正电荷,这种导体
里的自由电荷由于受到外电场的作用而重新分布的现象,就
是我们前面讲过的静电感应.导体两端出现的正负电荷在导
体内部形成反方向的电场$E'$,它的电力线用虚线表示(图6.
12乙).这个电场与外电场叠加,使导体内部的场强减小.但
是,只要导体内部的场强不等于零,自由电子就继续移动,两
端的正负电荷就继续增加,导体内部的电场就进一步削弱,直
到导体内部各点的场强都等于零时为止.这时自由电子的定
向移动停止(图6.12丙).

导体中(包括表面)没有电荷定向移动的状态叫做\textbf{静电平
衡状态}.\textit{处于静电平衡状态的导体,内部的场强处处为零}.

导体处静电平衡状态时,它表面的场强方向一定跟它
的表面垂直(参看图6.12丙).假如不是这样,场强就有一个
沿导体表面的分量,导体上的自由电子就会发生定向移动,这
就不是平衡状态了,所以,\textit{处于静电平衡状态的导体,表面上
任何一点的场强方向跟该点的表面垂直}.

带电导体可以认为它处于本身所带电荷形成的电场中,
它在静电平衡状态时内部的场强也一定处处为零,假如不是
这样,导体内部的自由电子就会发生定向移动,既然导体内
部的场强处处为零,导体内部就不可能有未被抵消的电荷.这
是因为,假如在导体内部某处有电荷,在它的附近的场强就不
可能为零,所以,\textit{处于静电平衡状态的带电导体,电荷只能分
布在导体的外表面上}.

静电平衡时电荷只分布在导体的外表面上,可以用下述
的法拉第圆筒实验来验证.如图6.13所示,取两个验电器$A$
和$B$,在$B$上装一个几乎封闭的空心金属圆筒$C$(叫做法拉第
圆筒).使$B$和$C$带电,$B$的箔片张开,用有绝缘柄的金属小
球$e$先跟$C$的外部接触,再把$e$移到$A$并跟$A$的金属球接触
(图6.13).经过若干次以后,可以看到$A$的箔片张开,同时
$B$的箔片张开的角度减小.这表明$e$把$C$的一部分电荷搬运
给了$A$.可见法拉第圆筒的外表面是带有电荷的.如果让$e$
不接触$C$的外部,而接触$C$的内部,重做上述实验(图6.14),
不论重复多少次,$A$的箔片都不张开,$B$的箔片张开的角度也
不减小,这表明$e$并没有把$C$的电荷搬运给$A$,可见拉法第
圆筒的内部不带电.

\begin{figure}[htp]
	\centering
	\begin{minipage}[t]{0.48\textwidth}
	\centering
	\includegraphics[scale=.6]{fig/6-13.png}
	\caption{}
	\end{minipage}
	\begin{minipage}[t]{0.48\textwidth}
	\centering
	\includegraphics[scale=.6]{fig/6-14.png}
	\caption{}
	\end{minipage}
	\end{figure}

\subsection{静电屏蔽}

静电平衡时导体内部的场强为零这一现象,
在技术上用来实现静电屏蔽.


如图6.15甲所示,使带正电的金属球靠近验电器,由于
静电感应,验电器的箔片张开,这表示验电器受到了附近的带
电体的影响.如果事先用一个金属网罩把验电器罩住(图6.15
乙),再让带电金属球靠近,验电器的箔片就不张开了.即使
用导线把验电器和金属网罩连接上,箔片也不张开.可见,金
属网罩(或金属包皮)能把外电场遮住,使内部不受外电场的
影响,这就是静电屏蔽.有的电学仪器和电子设备的外面套
有金属罩,通讯电缆的外面包一层铅皮,都是用来防止外界
电场的干扰起屏蔽作用.

\begin{figure}[htp]\centering
	\includegraphics[scale=.6]{fig/6-15.png}
	\caption{静电屏蔽}
	\end{figure}


\section{电势能}
前面我们从电荷在电场中受到力的作用出发,研究了电
场的性质,下面我们从能量的角度来研究电场的性质.

我们知道,物体在重力场中具有重力势能,重力势能是
与重力做功密切相关的,同样,电荷在电场中也具有势能,叫
做电势能.电势能是与电场力做功密切相关的.

物体在地面附近下落时,重力对物体做正功,重力势能减
少;物体上升时,重力对物体做负功,重力势能增加,重力势
能的变化总等于重力对物体所做的功.与此相似,在电场中
移动电荷时,如果电场力对电荷做正功,电势能就减少;如果
电场力对电荷做负功,电势能就增加.电势能的变化总等于
电场力对电荷所做的功.

就功和能之间的关系来讲,场中的情形跟重力场中的
情形完全相似.但由于存在两种电荷,电场力既可以是引力,
也可以是斥力,因此电场力做功的问题要复杂一些.

图6.16表示正电荷$Q$的电场,在电场中把正电荷$q$从$A$
点移到$B$点,电场力的方向与电荷移动的方向相同,电场力对
电荷$q$做正功,电势能减少.可见,正电荷$q$在$A$点的电势能
大于它在$B$点的电势能.在正电荷$Q$的电场中,正电荷$q$离
$Q$越近,电势能越大.
\begin{figure}[htp]
	\centering
	\begin{minipage}[t]{0.48\textwidth}
	\centering
			\begin{tikzpicture}[>=stealth, thick]
	\foreach \x in {0,45,90,...,315}
	{
		\draw [->](0:0)--(\x: 1.5);
		\draw (\x:1.5)--(\x: 2);
	}
	\draw (0,0) [fill=white] circle (.4);
	\draw (1.2,0) [fill=black] circle (1.5pt)node[below]{$B$};\draw (.7,0) [fill=black] circle (1.5pt)node[below]{$A$};
\node at (0,0){$+Q$};
		\end{tikzpicture}
	\caption{正电荷$Q$的电场}
	\end{minipage}
	\begin{minipage}[t]{0.48\textwidth}
	\centering
		\begin{tikzpicture}[>=stealth, thick]
	\foreach \x in {0,45,90,...,315}
	{
		\draw (0:0)--(\x: 1.5);
		\draw [<-](\x:1.5)--(\x: 2);
	}
	\draw (0,0) [fill=white] circle (.4);
	\draw (1.2,0) [fill=black] circle (1.5pt)node[below]{$D$};\draw (.7,0) [fill=black] circle (1.5pt)node[below]{$C$};
\node at (0,0){$-Q$};
		\end{tikzpicture}
	\caption{负电荷$Q$的电场}
	\end{minipage}
	\end{figure}

图6.17表示负电荷$Q$的电场,在电场中把正电荷$q$从$C$
点移到$D$点,电场力的方向与电荷移动的方向相反,电场力对
电荷$q$做负功,电势能增加.可见,正电荷$q$在$C$点的电势能
小于它在$D$点的电势能.在负电荷$Q$的电场中,正电荷$q$离
$Q$越近,电势能越小.

我们在讨论重力势能的时候,要先规定物体在某一位置
的重力势能为零,然后才能确定物体在其他位置的重力势能.
物体在某一位置的重力势能在数值上等于物体从这一位置移
到重力势能为零处重力所做的功.同样,我们在讨论电势能
的时候,也要先规定电荷在某一位置的电势能为零,然后才能
确定电荷在其他位置的电势能.\textit{电荷在电场中某点的电势能
在数值上等于把电荷从这点移到电势能为零处电场力所做的
功}.

在理论研究中,通常取电荷$q$在无限远处的电势能为零.
这样,在图6.16所示正电荷的电场中,因为正电荷$q$在离开
$Q$越远的地方电势能越小,而它在无限远处的电势能为零,所
以正电荷$q$在正电荷$Q$的电场中的电势能都是正值.在图
6.17所示的负电荷的电场中,因为正电荷$q$在离开电荷$Q$越
远的地方电势能越大,而它在无限远处的电势能为零,所以正
电荷$q$在负电荷$Q$的电场中电势能都是负值.


\subsection*{练习五}

\begin{enumerate}
	\item 把两个异种电荷的距离增大一些,电场力做正功还
是做负功?电势能是增加还是减小?把两个同种电荷的距离增
大一些,情况又怎样?
\item 在图6.16中,把负电荷$-q$放在$A$、$B$点,它在哪一点
的电势能较大?无限远处的电势能为零,负电荷$-q$在这个
电场中的电势能是正值还是负值?
\item 在图6.17中,把负电荷$-q$放在$C$、$D$点,它在哪一点的电势能较大?取无限远处的电势能为零,负电荷$-q$在这
个电场中的电势能是正值还是负值?
\item 在图6.18所示的电场中,如果把正电荷$q$由$N$点移到$M$点,$q$的电势能增加还
是减小?如果移动的是负电荷$-q$,电势能又怎样变化?
\begin{figure}[htp]\centering
    \begin{tikzpicture}[>=latex]
\foreach \x in {1,2,3}
{
    \draw[->] (0, \x)--(4,\x);
\node at (1, 2.4){$M$};  \node at (3, 2.4){$N$};
\fill (1,2.1) circle (1pt); \fill (3,2.1) circle (1pt);
}
        
    \end{tikzpicture}

    \caption{}
\end{figure}	

\item  电子在原子核附近运动时,电子的电势能是正值还
是负值?取无限远处的电势能为零,把这个电子由原子核附
近移到无限远处,电子的电势能是增加还是减小?
\end{enumerate}


\section{电势}
电荷在电场中某点具有的电势能跟电荷所带的电量有关
系.设在图6.16所示的正电荷的电场中,正电荷$q$在$A$点的
电势能为$\mathcal{E}_A$,它在数上等于把电荷$q$从$A$点移到无限远处
电场力所做的功,如果把放在$A$点的电荷增加为原来的$n$倍,
那么,在把它移到无限远处的过程中,所受的电场力处处为原
来的$n$倍,电场力所做的功也为原来的$n$倍,因而电势能为原
来的$n$倍.这就是说,电荷在电场中某点具有的电势能$\mathcal{E}_A$跟
电荷所带的电量$q$成正比,不论电量$q$是多少,比值$\mathcal{E}_A/q$都相
同,是跟电量$q$无关的一个恒量.

把正电荷$q$放在图6.16中的$B$点,设电荷的电势能为$\mathcal{E}_B$
.根据同样的分析知道,比值$\mathcal{E}_B/q$也是跟电量无关的
恒量.因为$\mathcal{E}_B$
跟$\mathcal{E}_A$一般并不相同,所以比值$\mathcal{E}_B/q$跟$\mathcal{E}_A/q$
一般也不相同.

上面是就正电荷$Q$产生的电场来分析的,实际上,上述分
析对任何电场都是适用的.

既然在电场中某点比值$\mathcal{E}/q$是跟电量$q$无关的恒量,而
且对电场中不同的点来说这个恒量一般并不相同,可见,这个
恒量是由电场本身决定的,它反映电场本身的一种性质.

\textit{电场中某点的电荷的电势能跟它的电量的比值,叫做这
一点的电势},如果用$U$表示电势,用表示电荷$q$的电势能,
那么
\[U=\frac{\mathcal{E}}{q} \]

如果取$q$为单位正电荷,那么$U$在数值上等于$\mathcal{E}$.可见,
电场中某点的电势在数值上等于单位正电荷在那一点所具有
的电势能.

在国际单位制中,电势的单位是伏特,简称伏,国际符号
是V.电场中的某一点,如果电量是1库的电荷在该点的电
势能是1焦,这一点的电势就是1伏.
\[ 1{\rm V}=1{\rm J}/{\rm C} \]

电势只有大小,没有方向,因此是标量.

电势跟电势能一样,并没有绝对的意义.只有先规定了某
处的电势为零以后,才能确定电场中其他各点的电势的值.电
场中电势为零的位置也就是电荷在该点的电势能为零的位
置.在理论研究中,通常也就取无限远处的电势为零.在实
际应用中,通常取大地的电势为零.

在规定了零电势后,电场中各点的电势可以是正值,也可
以是负值.例如,规定无限远处的电势为零,在图6.16所示
的正电荷$Q$的电场中,因为正电荷$q$的电势能都是正值,所以
电场中各点的电势都是正值,而且离正电荷$Q$越远,电势越
低;在图6.17所示的负电荷$Q$的电场中,因为正电荷$q$的电
势能都是负值,所以电场中各点的电势都是负值,而且离负电
荷$Q$越远,电势越高.

在电场中,我们可以根据电力线的方向判断电场中各点
电势的高低,因为顺着电力线的方向移动正电荷,电场力做
正功,正电荷的电势能减小,所以\textit{顺着电力的方向电势越来
越低}.这个结果不但对正电荷或负电荷的电场是适用的,对
任何电场都适用.

现在我们已经认识了反映电场性质的两个物理量:电场
强度和电势.\textit{电场强度是反映电场的力的性质的物理量}.知
道了电场强度$E$,就可以知道电荷$q$在电场中所受的力$F=
qE$,\textit{电势是反映电场的能的性质的物理量}.知道了电势$U$,
就可以知道电荷$q$在电场中的电势能$\mathcal{E}-qU$.在电势为正值
的地方,正电荷的电势能是正值,负电荷的电势能是负值.在
电势为负值的地方,正电荷的电势能是负值,负电荷的电势能
是正值.

\section{等势面}

我们知道,用电力线能够把电场中各点场强的大小和方
向形象地表示出来.电场中各点电势的大小,是否也可以用
图形来表示呢?同样可以,一般说来,电场中各点的电势不
同,但电场中有许多点的电势相等.我们把电场中电势相等
的点构成的面叫做\textbf{等势面}.在电场中可以用等势面来表示电
势的高低,这跟在地图上用等高线来表示地形的高低是类似
的.

在同一等势面上的任何两点间移动电荷,电场力不做功.
这是因为,假如电场力做了功,这两点的电势就不相等,它们
就不在一个等势面上了.这种情形,跟在同一水平面上的两
点间移动物体时,重力不做功的道理是一样的.

等势面一定跟电力线垂直,即跟场强的方向垂直.假如
不是这样,场强就有一个沿着等势面的分量,这样在等势面上
移动电荷时电场力就要做功,但这是不可能的,因为在等势
面上各点电势相等,沿等势面移动电荷时电场力是不做功的.
所以场强一定跟等势面垂直.

前面已经指出,沿着电力线方向电势越来越低.可见,电
力线不但跟等势面垂直,而且是由电势较高的等势面指向电
势较低的等势面.

\begin{figure}[htp]
	\centering
	\begin{minipage}[t]{0.48\textwidth}
	\centering
	\begin{tikzpicture}[>=latex, scale=.6]
		\foreach \x in {1,2,3}
		{
			\draw (0,\x)--(7,\x);
		}
		\foreach \x in {0,4}
		{
			\draw[ultra thick] (0,\x)--(7,\x);
		}
		\node at (7.5,0){$-$};
		\node at (7.5,4){$+$};
		
		\foreach \y in {1,2,...,6}
		{
			\draw[->,dashed] (\y, 4)--(\y,0);
		}
			\end{tikzpicture}
	\caption{}
	\end{minipage}
	\begin{minipage}[t]{0.48\textwidth}
	\centering
	\begin{tikzpicture}[>=latex, scale=.6]
		\foreach \x in {1,2,3}
		{
			\draw (0,0) circle (\x);
		}
		\foreach \x in {0, 22.5, 45, ..., 337.5}
		{
			\draw[->, dashed](0,0)--(\x:4);
		}   
			\end{tikzpicture}	\caption{}
	\end{minipage}
	\end{figure}


图6.19是匀强电场中的等势面,它们是垂直于电力线的
一族平面.图6.20是点电荷电场中的等势面,它们是以点电
荷为球心的一族球面.图6.21是等量异种的两个点电荷电
场中的等势面.

\begin{figure}[htp]\centering
\begin{minipage}[t]{0.48\textwidth}
\centering
	\includegraphics[scale=.8]{fig/6-21.pdf}
	\caption{}
\end{minipage}
\begin{minipage}[t]{0.48\textwidth}
\centering
		\includegraphics[scale=1.1]{fig/6-22.pdf}
		\caption{带电导体周围的等势面和电力线}
\end{minipage}
		\end{figure}

导体在静电平衡状态时内部场强处处为零,在导体的任
意两点间移动电荷时电场力所做的功等于零,因此导体内各
点的电势相等.\textit{处于静电平衡状态的导体是一个等势体,
它
的表面是一个等势面}.图6.22
是不规则形状的带电导体周
围的电力线和等势面的分布情
况.导体的表面是个等势面;
离导体表面越近,等势面的形
状与导体表面的形状越相似.

实际测量电势比测量场强容易,所以常常用等势面来研
究电场,先测绘出等势面的形状和分布,再根据电力线和等势面
处处垂直这一特性,绘出电力线的形状和分布,就可以知
道整个电场的分布.设计许多电子仪器(如电子显微镜、示波
管等)中的电极的形状、大小及相互位置时,都需事先经过实
验,测绘出等势面的形状和分布,推知带电电极所产生的电场
的分布,以便找出符合实际要求的设计方案.


\subsection*{练习六}

\begin{enumerate}
	\item 电场中A点的电势是3伏,求;
	\begin{enumerate}
		\item 电量为5库的电荷在$A$点的电势能;
		\item 电量为10库的电荷在$A$点的电势能;
		\item 电量为$-5$库的电荷在$A$点的电势能;
		\item 电量为$-10$库的电荷在$A$点的电势能.
	\end{enumerate}
	\item 在图6.16中$A$、$B$两点哪一点电势高?在图6.17中
$C$、$D$两点哪一点的电势高?说明理由.

\item 在图6.23所示的匀强
电场中,如果$A$板是接地的,
$M$、$N$两点哪点电势高?电势是
正值还是负值?如果$B$板是接
地的,结果又怎样?取大地的电
势为零.

\begin{figure}[htp]\centering
    \begin{tikzpicture}[>=latex]
\foreach \x in {0,2}
{
    \draw[very thick] (0,\x) --(5,\x);
}

\node at (5.25,0){$B$};
\node at (5.25,2){$A$};

\foreach \x in {1, 2,3,4}
{
    \draw[->, dashed](\x,2)--(\x,0);
}   
\node at (2.5,1.5){$M$};    \node at (2.5,.5){$N$};
\fill (2.2,1.5) circle (1.5pt);\fill (2.2,.5) circle (1.5pt);
    \end{tikzpicture}
    \caption{}
\end{figure}	

\item 一个初速度为零的正电荷放在电场中,只在电场力
作用下,它向电势高的地方跑还是向电势低的地方跑?一个初
速度为零的电子放在电场中,它向电势高的地方跑还是向电
势低的地方跑?说明理由.
\item 一个初速度为零的电荷放在电场中,不论是正电荷
还是负电荷,都向着电势能低的地方跑,试说明理由.
\item 电场中某点的电势是否跟检验电荷的正负有关?讨
论一下这个问题.
\item 电场中两个电势不同的等势面能不能相交?为什么?
\end{enumerate}

\section{电势差}
用不同的位置作为测量高度的起点,同一地方的高度的
数值就不相同,但两个地方的高度差保持不变,同样的道理,
选择不同的位置作零电势,电场中某点的电势的数值也会改
变,但电场中任意两点间的电势的差值保持不变,正是因为这
个缘故,在物理学中电势的差值用得比电势更为普遍.

\subsection{电势差}
\textit{电场中两点间的电势的差值叫做电势差,有时
又叫做电压},设电场中$A$点的电势为$U_A$,$B$点的电势为$U_B$,
如果$U_A>U_B$,这两点间的电势差就是
\[U_{AB}=U_A-U_B \]
如果$U_B>U_A$,这两点间的电势差就是
\[U_{BA}=U_B-U_A \]

知道了电场中两点间的电势差,可以很方便地计算出在
这两点间移动电荷时电场力做的功.

例如,在图6.16的电场中,正电荷$q$在$A$点的电势能是
$qU_A$,在$B$点的电势能是$qU_B$,由于$U_A>U_B$,把正电荷$q$从$A$
点移到$B$点时,$q$的电势能的减少就是$qU_A-qU_B$.而电势能
的减少等于电场力做的正功,所以正电荷从$A$点移到$B$点时
电场力做的正功
\[W=q(U_A-U_B)=qU_{AB}\]

经过类似的讨论可以知道,如果是把正电荷$q$从$B$点移
到$A$点,电场力要做负功,功的大小仍然等于$qU_{AB}$;如果是把
负电荷$-q$从$A$点移到$B$点,电场力做的也是负功,功的大小
仍然等于$qU_{AB}$;如果是把负电荷$-q$从$B$点移到$A$点,电场
力要做正功,功的大小仍然是$qU_{AB}$.

所以,在电场中$A$、$B$两点间移动电荷时,电场力做的功
$W$等于电量$q$和这两点间的电势差$U$的乘积,即
\[W=qU\]
式中$q$用库作单位,$U$用伏作单位,$W$用焦作单位,利用这个
公式时,$q$、$U$都取绝对值,算出的功$W$也是绝对值,至于功的
正负可以从电荷的正负和移动方向来判断.

\subsection{电子伏特}


人们在研究原子、原子核、基本粒子等微观世
界的时候,常用\textbf{电子伏特}作为能量的单位.\textit{1电子伏特,就是
在电压为1伏特的两点间移动电子时电场力所做的功}.
电子
伏特简称电子伏,国际符号是eV.我们很容易算出电子伏特
跟焦耳的关系.

\[\begin{split}
	1{\rm eV}&=1{\rm e}\x 1{\rm V}\\
	&=1.60\x 10^{-19}{\rm C}\x 1{\rm V}\\
&=1.60\x 10^{-19}{\rm J}	
\end{split}	\]


\begin{example}
	设电场中$A$、$B$两点的电势差$U=2.0\x10^2$伏,
	带电粒子的电量$q=1.2\x10^{-8}$库,把$q$从$A$点移到$B$点,电
	场力做了多少功,是正功还是负功?设$U_A<U_B$.
\end{example}

\begin{solution}
\[\begin{split}
	W&=qU=1.2\x10^{-8}\x2.0\x10^2{\rm J}\\
&=2.4\x10^{-6}{\rm J}
\end{split} \]

正电荷由电势低的位置移到电势高的位置,电势能增加,因此电场力做负功.
\end{solution}

\subsection*{练习七}

\begin{enumerate}
	\item 把带电体从电势为300伏的$A$点移到电势为100伏
的$B$点,电场力做了$3.0\x10^{-8}$焦的负功,带电体带哪种电
荷?电量是多少?
\item 电场中$M$、$N$两点的电势$U_M=800$伏、$U_N=-200$
伏,把电量是$1.5\x10^{-8}$库的负电荷从$M$点移到$N$点,电场力
做了多少功?做正功还是负功?
\item 在电场中把电量为$2.0\x10^{-8}$库的正电荷从$A$点移
到$B$点,电场力做了$1.5\x10^{-7}$焦的正功,再把这个正电荷从
$B$点移到$C$点,电场力做了$4.0\x10^{-7}$焦的负功、$A$、$B$、$C$三点
中哪点的电势最高,哪点的电势最低?$A$、$B$间,$B$、$C$间和$A$、$C$
间的电势差各是多大?
\item 一个原来静止的电子,从电场中的$A$点被加速移到
$B$点.$A$、$B$两点间的电势差是2000伏,电场力所做的功是多
少电子伏?电势能的变化是多少电子伏?设电子是在真空中
移动的,电子在$B$点获得的动能是多少电子伏?
\end{enumerate}


\section{电势差跟电场强度的关系}
场强是跟电场对电荷的作用力相联系的,电势差是跟电
场力移动电荷做功相联系的.正象力和功有联系一样,场强
和电势差也是有联系的,我们以匀强电场为例来研究它们的
关系.
\begin{figure}[htp]\centering
    \begin{tikzpicture}[>=latex]
\foreach \x in {1,2,3,4}
{
	\draw[dashed] (\x,.5)--(\x, 4.5);
	\draw[->](0,\x)--(5,\x);
}
   \node at (5,2.5){$E$};

\draw (.5,2.5)node [left]{$A$}--node[above]{$d$}(3,2.5)node [right]{$B$};
\draw (.5,2.5)--(3,3.5)node [right]{$D$};
\draw (.5,2.5)--(3,1.5)node [right]{$C$};

    \end{tikzpicture}
    \caption{}
\end{figure}

前面讲过,沿着电力线的方向,也就是沿着场强的方向,
电势越来越低.从图6.24中看到,除沿场强方向$AB$外,沿
其他方向$AC$、$AD$,电势也都降低.那么,场强的方向又有什
么特殊性呢?从图中可以看出,虽然电势沿$AB$、$AC$、$AD$的方
向都要降低,但是沿$AB$方向降低得最快,可见\textit{场强的方向是
指向电势降低最快的方向}.

我们再来研究场强和电势
差的数量关系.设图6.24中
$A$、$B$间的距离为$d$,电势差为
$U$,场强为$E$.把正电荷$q$从$A$
移到$B$时,电场力$qE$所做的功
$W=qEd$.利用电势差和功的
关系,这个功又可求得为$W=qU$.比较这两个式子,即可得
到
\[U=Ed\]
这就是说,\textit{在匀强电场中,沿场强方向的两点间的电势差等于
场强和这两点间距离的乘积}.

把上式改写成
\[E=\frac{U}{d} \]
这个等式说明,\textit{在匀强电场中,场强在数值上等于沿场强方向
每单位距离上降低的电势}.

由上式可以得到场强的另一个单位:伏特/米.由于
\[1\frac{{\rm V}}{\rm m}=1\frac{{\rm J}/{\rm C}}{\rm m}=1\frac{{\rm N}\cdot {\rm m}}{{\rm C}\cdot {\rm m}}=1\frac{{\rm N}}{\rm C}  \]
所以场强的两个单位伏/米和牛/库是相等的.

\begin{example}
	图6.25中,金属圆板$A$、$B$相距3厘米.用电
压为60伏的电池组使它们带电,它们间的匀强电场的场强是
多大,方向如何?
\end{example}
\begin{figure}[htp]\centering
    \begin{circuitikz}[>=latex]
\draw(0,0)--(3,0);
\draw (3,0) to[C] (3, 2.5);
\draw (0,2.5) to [battery] (0,0);
\draw(0,2.5)--(3,2.5);        
\node at (3.75, 1){$B$};
\node at (3.75, 1.5){$A$};
\node at (2.75, .9){$-Q$};
\node at (2.75, 1.6){$+Q$};

    \end{circuitikz}
    \caption{}
\end{figure}	

\begin{solution}
	金属板间的电势差就是电池组的电压.知道这个电势差
$U$后,可以用公式$E=U/d$计算出场强$E$:
\[E=\frac{U}{d}=\frac{60}{3\x 10^{-2}}{\rm V}/{\rm m}=2\x 10^8 {\rm V}/{\rm m}\]

$A$板带正电,$B$板带负电,所以场强方向是由$A$板指向$B$
板.
\end{solution}

\subsection*{练习八}

\begin{enumerate}
	\item 两块相距0.05米的带电平行板之间的电场是匀强
	电场,两板的电势差为$10^4$伏.求作用在两板之间的一个电
	子上的电场力.
	\item 平行的带电金属板$A$、$B$间是匀强电场(图6.26),场
	强为$1.2\x10^8\NC$.两板间的距离为5厘米,两板间的电
	势差有多大?电场中有两点$P_1$和$P_2$,$P_1$点离$A$板的距离是
	0.5厘米,$P_2$点离$B$板的距离也是0.5厘米.$P_1$和$P_2$两点间
	的电势差有多大?
\end{enumerate}

\begin{figure}[htp]\centering
    \begin{tikzpicture}[>=latex]
\draw (0,0)--(6,0)node [right]{$B$};
\draw (0,3)--(6,3)node [right]{$A$};        

\foreach \x in {.5,1.5,...,5.5}
{
	\node at (\x, 0.25){$+$};
	\node at (\x, 2.75){$-$};
}

\fill (2,2.7) circle (2pt);
\fill (5,.3) circle (2pt);
\node at (2,2+.4){$P_1$};
\node at (5,1-.4){$P_2$};

    \end{tikzpicture}
    \caption{}
\end{figure}	

\section{带电粒子在电场中的运动}

带电粒子在电场中受到电场力的作用,产生加速度,速度
的大小和方向都可以发生变化.在现代科学实验和技术设备
中,常常根据这个道理,利用电场来改变或控制带电粒子的运
动.这种应用大致可以分成两种情况:一是利用电场来使带
电粒子加速,一是利用电场来使带电粒子偏转.

\subsection{带电粒子的加速}
\begin{figure}[htp]\centering
	\includegraphics[scale=.6]{fig/6-27.png}
	\caption{}
	\end{figure}

如图6.27所示,在真空中有一对平行
金属板,接上电压为$U$的电池
组,在它们之间建立匀强电场.
设有一个正电荷$q$穿过正极板
上的小孔进入电场,在电场中
被加速,到达负极板时从负极
板上正对的小孔穿出,正电荷
穿出时的速度$v$是多大呢?正电荷$q$从正极板移到负极板,电
场力做的功$W=qU$.设$q$是在正极板处由静止开始运动,
到达负极板时它的动能为$\dfrac{1}{2}mv^2$.
根据动能定理得到$qU=\dfrac{1}{2}mv^2$.
由此就可求出电荷$q$到达负极板的速度$v=\sqrt{2gU/m}$.

带电粒子在匀强电场中的上述加速运动,跟物体在重力
场中的自由落体运动相似.不过,物体在重力场中受到的力
跟质量成正比,因此不同质量的物体具有相同的加速度;而带
电粒子在电场中受到的力跟电量成正比,质量相同的粒子可
以带有不同的电量,因而它们在电场中的加速度可以互不相
同.

上述用电压$U$来表达的计算速度的公式$v=\sqrt{2gU/m}$对
非匀强电场也适用.这是因为,不论在什么电场中,电荷$q$通
过电压$U$时,电场力对它做的功总等于$qU$,而对初速度为零
的带电粒子总是有$qU=\dfrac{1}{2}mv^2$
的关系.

\subsection{带电粒子的偏转}


要使以一定速度运动的带电粒子偏
转,可以有两个办法:一是利用磁场,这将在高中三年级再讨
论,一是利用电场,利用电场使带电粒子偏转,人们通常用跟
带电粒子初速度方向垂直的勾强场,这时带电粒子受到一
个跟原来运动方向垂直的电场力,因而发生偏转.
\begin{figure}[htp]\centering
	\includegraphics[scale=.6]{fig/6-28.png}
	\caption{}
	\end{figure}

如图6.28所示,真空中有一对平行金属板,接上电压为
$U$的电池组,在它们之间建立匀强电场,场强为$E=U/d$
其中$d$为两板的距离.设有一些带正电荷$q$的粒子以初速度$v_0$进
入电场,$v_0$的方向跟$E$的方向垂直.带电粒子受到垂直于$v_0$
的侧向电场力$F=qE=qU/d$
的作用,它们在电场内的运动跟
物体在重力场中的平抛运动相似.现在我们来计算带电粒子
在电场中侧向移动的距离$y$.带电粒子在侧向电场力$F$作用
下,沿侧向做初速度为零的匀变速运动,所以$y=\dfrac{1}{2}at^2$.
由牛顿第二定律知道,
\[a=\frac{F}{m}=\frac{qU}{md} \]
带电粒子在电场内运动的时
间$t=\ell/v_0$.由此可得
\[y=\frac{qU}{2v^2_0 md}\ell^2 \]

带电粒子离开电场后,将在偏离原来运动方向某一角度
$\phi$的方向上做匀速直线运动,研究带电粒子的偏转,这个偏
角$\phi$是特别重要的.现在来讨论$\phi$跟哪些因素有关,带电粒
子离开电场时得到一个垂直于初速度的侧向速度$v_{\bot}=at$.而
\[a=\frac{qU}{md},\qquad t=\frac{\ell}{v_0} \]
由此可得
\[v_{\bot}=at=\frac{qU\ell}{mdv_0} \]
带电粒子离开电场时的偏角$\phi$由下式确定:
\[\tan\phi=\frac{v_{\bot}}{v_0}=\frac{qU\ell}{mdv^2_0} \]
对于一定的带电粒子束,$m$、$q$、$v_0$都是确定了的,适当选择$U$、$d$、$\ell$,就可以使
$\phi$符合预定的要求.

\begin{example}
	\begin{figure}[htp]\centering
		\includegraphics[scale=.6]{fig/6-29.png}
		\caption{}
		\end{figure}

	实验表明,赤热的金属丝可以发射电子.在图
6.29中,从赤热金属丝射出的电子流,经电场加速后进入偏
转电场,已知加速电极间的电压是2500伏,偏转电极间的电
压是2.0伏,偏转电极长6.0厘米,相距0.2厘米.电子的质
量是$0.91\x10^{-30}$千克.求:
\begin{enumerate}
	\item 电子离开加速电场时的速度;
	\item 电子离开偏转电场时的侧向速度.
\end{enumerate}
\end{example}

\begin{solution}
1. 经过2500伏的加速电场后,电子获得的动能
\[E_K=2500{\rm eV}=2500\x1.6\x10^{-19}{\rm J}=4.0\x10^{-16}{\rm J} \]
而$E_K=\dfrac{1}{2}mv^2$,所以
\[\begin{split}
	v=\sqrt{\frac{2E_K}{m}}&=\sqrt{\frac{2\x 4.0\x 10^{-18}}{0.91\x 10^{-30}}}\ms\\
	&=3.0\x 10^7\ms
\end{split} \]	

2. 电子离开偏转电场时的侧向速度是
\[\begin{split}
	v_{\bot}&=\frac{qU\ell}{mdv}\\
	&=\frac{1.6\x10^{-19}\x2.0\x6.0\x10^{-2}}{0.91\x10^{-30}\x0.2\x10^{-2}\x3.0\x10^{7}} \ms\\
	&=3.5\x 10^5\ms
\end{split} \]	
\end{solution}

\subsection*{练习九}
\begin{enumerate}
	\item 在真空中有一对平行金属板,相距6.2厘米,加上90
伏的电压,两价的氧离子从静止出发被加速,从一板到达另
一板时,它的动能是多大?

这道题有几种解法?哪种解法比较简便?

\item 两价离子在90伏的电压下从静止加速后,测出它的
动量是$1.24\x10^{-21}{\rm kg}\cdot \ms$,这种离子的质量是多大?

\item 经1000伏加速电压加速后的电子,沿着与电场垂直
的方向进入匀强偏转电场,场强为5000$\NC$.已知偏转电
极长为6厘米,求电子离开偏转电场时的速度.
\item 计算一下本节例题中的电子离开偏转电场时侧向移
动的距离.
\item 图6.30所示的实验装置可以用来验证电场对带电
粒子的加速作用只跟电压有关.左边的非匀强电场使电子加
速,右边的匀强电场使电子减
速,设非匀强电场的电压为$U$,
匀强电场的电压为$U'$.实验结
果是:只要$U'<U$,电流计的指
针就偏转;只要$U'>U$,电流计
的指针就不偏转.你从这个实
验结果可以得出什么结论?
\begin{figure}[htp]\centering
	\includegraphics[scale=.6]{fig/6-30.png}
	\caption{}
	\end{figure}
\end{enumerate}

\section{基本电荷的测定:密立根实验}

电子和质子带有等量异种电荷,实验指出,它们所带的
电量都是$e=1.60\x10^{-18}$库.实验还指出,所有电量或者等
于电子或质子的电量,或者是它们的电量的整数倍,因此,人
们把$1.60\x10^{-18}$库的电量叫做基本电荷.

历史上对电子电荷的测定进行了一系列实验,电子电荷
的精确数值最早是美国科学家密立根(1868—1953)于1917
年用实验测得的.密立根实验是物理学的经典实验之一,下
面从原理上介绍一下这个实验的最简单的方法.

\begin{figure}[htp]\centering
\includegraphics[scale=.7]{fig/6-31.PDF}
\caption{}
\end{figure}

密立根实验仪器示意图如6.31所示.$A$、$B$是两块平
行放置的水平金属板,把它们与电源相接,使$A$板带正电,$B$
板带负电.油滴从喷雾器喷出,经过上板当中的小孔,落到
$A$、$B$之间的匀强电场中,油滴出来时由于摩擦而带电,假如
油滴带负电,它要受到方向向上的电场力$F$作用.油滴还受到
重力$mg$的作用.调节两个板间的电势差,可使带有电量为
的某个油滴所受的电场力$Eq$恰好和重力$mg$平衡,于是油滴悬浮在电场中保持不动.在这种情况下,$Eq=
mg$.根据这个式子可以求出电量$q$:
\[q=\frac{mg}{E}=\frac{mgd}{U}  \]
上式中$U$是两板间的电势差,$d$是两板的距离,它们都可以直
接测得.但是油滴太小,$m$很难直接测量.密立根设法用实验
测出油滴的半径$r$,然后用体积公式$V=\dfrac{4}{3}\pi r^3$
算出油滴的体积,再用油滴的体积乘以油滴的密度算出油滴的质量$m$.这
样,用上式即可得出油滴所带电量$q$.

密立根测定了数千个带电油滴的电量,他对测得的数据
进行分析研究,发现这些电量都等于某个最小电荷的整数倍,
这个最小电荷就是电子或质子所带的电量$e$.在密立根实验
之后,人们还做了许多其他实验,进一步精确地测定电子的电
量,现在测得的基本电荷的精确值是
\[e=1.6021892\x10^{-18}{\rm C}\]
通常可取作
\[e=1.60\x10^{-18}{\rm C}\]

基本电荷是物理基本常数之一,测定它在理论上和实际
上都有重大意义.

密立根实验进一步证实了电子的存在,揭示了电荷的非
连续性,即自然界中的任何电量都只能是某一基本单位的整
数倍,而不能连续变化\footnote{近年来在高能物理的研究中提出了一个设想,认为质子、中子
等粒子是由更基本的层子(又叫夸克)组成的,层子所带电量是基本电
荷的$1/3$或$2/3$.但是,人们一直还没有在实验中观察到层子.}.

\section{电容器~~电容}
\subsection{电容器}
任何两个彼此绝缘而又互相靠近的导体,都可
以看成是一个电容器.这两个导体就是电容器的两个极,两
块正对的平行金属板,它们相隔很近而且彼此绝缘,就组成一
个最简单的电容器,叫做平行板电容器.

使电容器带电叫做充电.充电时总是使电容器的一个导
体带正电,另一个导体带等量的负电,每个导体所带电量的
绝对值,叫做电容器所带的电量.把平行板电容器的一个极
板接电池组的正极,另一个极板接电池组的负极,两个极板就
分别带上等量的异种电荷.

使充电后的电容器失去电荷叫做放电,用一根导线把电
容器的两极接通,两极上的电荷互相中和,电容器就不带电
了.

电容器是电气设备中的重要元件之一,在电子技术和电
工技术中有很重要的应用.

\subsection{电容}
电容器带电的时候,它的两极之间产生电势差.实
验证明,对任何一个电容器来说,两极间的电势差都随所带电
量的增加而增加,而且电量跟电势差成正比,它们的比值是一
个恒量,不同的电容器,这个比值一般是不同的.可见,这个
比值表征了电容器的特性.\textit{电容器所带的电量跟它的两极间
的电势差的比值,叫做电容器的电容}.如果用$Q$表示电容器
所带的电量,用$U$表示它的两极间的电势差,用$C$表示它的电
容,那么,
\[C=\frac{Q}{U} \]

电容器带电的情形可以用直筒容器装水的情形来比喻.
直筒容器装水后水的深度总跟装的水量成正比,水量和水的
深度的比值是一个恒量.不同的直筒容器,这个比值一般是
不相同的,这个比值越大,即水面升高单位高度所需的水量
越大,表示容器的容量越大.

在国际单位制里,电容的单位是法拉,简称法,国际符号
是F.一个电容器,如果带1库的电量时两极间的电势差是
1伏,这个电容器的电容就是1法.
\[1{\rm F}=1{\rm C}/{\rm V}\]

法拉这个单位太大,实际上常用较小的单位:微法($\mu$F)
和皮法(pF).它们间的换算关系是:
\[1{\rm F}=10^6\mu{\rm F}=10^{12}{\rm pF}\]

无线电收音机里常用的电容器,电容从几个皮法到几十
个微法的都有.

\subsection{平行板电容器的电容}


现在我们来研究平行板电容器的
电容跟哪些因素有关.

如图6.32所示,让平行板电容器带电后,用静电计\footnote{静电计是在验电器的基础上制成的,用来测量导体间的电势
差.使用时把它的金属球跟一个导体连接,把它的金属外壳跟另一个
导体连接或同时接地,从指针的偏转角度就可以知道两个导体间的电
势差.}来
测量两极板$A$、$B$间的电势差,不改变$A$、$B$两极板所带的电
量,只改变两极板间的距离,可以看到,距离越大,静电计指出
的电势差越大.这表示平行板电容器的电容随两板距离的增
大而减小.

\begin{figure}[htp]
	\centering
	\begin{minipage}[t]{0.48\textwidth}
	\centering
	\includegraphics[scale=.27]{fig/6-32.pdf}
	\caption{}
	\end{minipage}
	\begin{minipage}[t]{0.48\textwidth}
	\centering
	\includegraphics[scale=.27]{fig/6-33.pdf}
	\caption{}
	\end{minipage}
	\end{figure}

	如图6.33所示,不改变两极板所带电量和它们的距离,
	只改变两极板的正对面积,可以看到,正对面积越小,静电计
	指出的电势差越大.这表示平行板电容器的电容随两极板的
	正对面积的减小而减小.
	\begin{figure}[htp]\centering
		\includegraphics[scale=.27]{fig/6-34.pdf}
		\caption{}
		\end{figure}

	如图6.34所示,保持两极板所带电量、它们的距离、它们
	的正对面积都不改变,而在极板间插入电介质,可以看到,静
	电计指出的电势差减小.这表示平行板电容器的电容由于插
	入电介质而增大.

	由于我们的知识不足,现在还不能从理论上进一步讨论
	上面的实验结果,可以指出:对于一个平行板电容器,如果两
	板的正对面积为$S$,两板的距离为$d$,两板间充满介电常数为
	$\varepsilon$的电介质,那么,它的电容可以用下式来表示.
\[C=\frac{\varepsilon S}{4\pi kd} \]
式中$S$用${\rm m^2}$作单位,$d$用${\rm m}$作单位,静电力恒量$k=9\x 10^9{\rm N}\cdot {\rm m^2}/{\rm C^2}$
,算出的$C$以法为单位.可以看出,\textit{平行板电容器
的电容,跟介电常数成正比,跟正对面积成正比,跟极板的距
离成反比}.这跟上面的实验结果是一致的.

一般说来,电容器的电容是由两个导体的大小和形状、两
个导体的相对位置以及它们间的电介质定的.

\subsection{常用电容器}

懂得了决定电容大小的因素,就可以利用
这些知识来改变电容器的电容.实际上,人们正是这样制成
各种电容器,来满足不同需要的.从构造上看,常用的电容器
可分为固定电容器和可变电容器两类.

固定电容器的电容是固定不变的,由于所用的电介质不
同,又可分为纸介电容器、云母电容器、瓷介电容器、电解电容
器等,下面我们说明一下纸介电容器和电解电容器.

纸介电容器是在两层锡箔或铝箔中间夹以在石蜡中浸过
的纸,一起卷成圆柱体而制成的(图6.35).纸浸过石蜡后,
可避免潮气侵入,使绝缘能力大大增强.改变锡箔或铝箔
的面积,可以制成电容大小不同的纸介电容器.这种电容器
的特点是容易制造出电容较大的电容器,而且价格较低.
\begin{figure}[htp]
	\centering
	\begin{minipage}[t]{0.48\textwidth}
	\centering
	\includegraphics[scale=.5]{fig/6-35.png}
	\caption{纸介电容器}
	\end{minipage}
	\begin{minipage}[t]{0.48\textwidth}
	\centering
	\includegraphics{fig/6-36.png}
	\caption{电解电容器}
	\end{minipage}
	\end{figure}

电解电容器外形如图6.36所示,这种电容器的极性是
固定的,使用时正负极不能接反,并且不能接在交流电路中,
否则它将不能工作,这是它跟其他电容器不同的地方,电解
电容器是利用电解现象制成的,它的原理将在后面第八章中
给予说明.
\begin{figure}[htp]\centering
	\includegraphics{fig/6-37.png}
	\caption{可变电容器}
	\end{figure}

可变电容器的电容是可以改变的,它由两组铝片组成(图
6.37),固定不动的一组铝片叫定片,可以转动的一组铝片叫
动片,定片和动片之间的电介质,通常就用空气,转动动片,
两组铝片的正对面积发生变化,电容就随着改变.

此外,还有半可变电容器,能微小地调整两极片间的距离
或改变它们的正对面积,使电容发生微小改变.图6-38是电
路图中常用的几种电容器的符号.

\begin{figure}[htp]
	\centering
	\begin{circuitikz}[>=latex, ]
\draw (0,3) to [C]  (0,1);
\draw (1.5,3) [eC] to (1.5,1);
\draw (3,1) [C] to (3,3);
\draw [->, thick](2.5,1.5)--(3.5,2.5);
\draw (4.5,1) [C] to (4.5,3);
\draw [-|, thick](4,1.5)--(5,2.5);

\node at (0,0.5){甲};
\node at (1.5,0.5){乙};
\node at (3,0.5){丙};
\node at (4.5,0.5){丁};

	\end{circuitikz}
	\caption{甲:固定电容器,乙:电解电容器,丙:可
	变电容器,丁:半可变电容器.}
\end{figure}

加在电容器两极上的电压不能超过某一限度.超过这个
限度,电介质将被击穿,电容器于是损坏,这个极限电压叫做
击穿电压,电容器工作时的电压应低于击穿电压,电容器上
一般都标明了电容和额定电压的数值.电容器的额定电压是
指电容器长期工作所能承受的电压,它比击穿电压要低.

\subsection*{练习十}
\begin{enumerate}
	\item 电容器带电后电势差增大的情形,跟物体吸收热量后温度升高的情形也很相似,试对这两种现象作一比较.
	\item 一个由圆板制成的平行板电容器,圆板的半径为3.0
厘米,两板的距离为2.0毫米,中间充满介电常数为6.0的电
介质,这个电容器的电容是多少?
\item 一个电容器的电容是$1.5\x10^{-2}$微法,把它的两个极
板接在90伏的电源上,求每个极板上的电量.
\item 有一个电容器,在带了电量$Q$以后,两导体间的电势
差是$U$,如果使它带的电量增加$4.0\x10^{-8}$库,两导体间的电
势差就增大20伏,这个电容器的电容是多少微法?

\begin{figure}[htp]\centering
    \begin{circuitikz}[european, scale=.8]
        \draw (0,0)--(0,3);
        \draw (0,3)to [capacitor=$C$](6,3);
        \draw (6,0)--(6,3);    
        \draw (0,0) to [battery2] (3,0) to [cute open switch=$K$] (5,0)--(6,0)   ;
    \end{circuitikz}
    \caption{}
\end{figure}

\item 如图6.39所示,闭合
电键$K$使平行板电容器$C$充
电,然后断开电键,当增大电容
器两板间的距离时,下述各量
是否改变,怎样改变?
\begin{enumerate}
	\item 电容器所带电量;
	\item 电容器的电容;
	\item 电容器两板间的电势差.
\end{enumerate}
\item 在上题中,充电后如果保持电键$K$闭合,那么,增大
电容器两板间的距离时,下述各量是否改变,怎样改变?
\begin{enumerate}
	\item 电容器两板间的电势差;
	\item 电容器的电容;
	\item 电容器所带的电量.
\end{enumerate}
\end{enumerate}

\section{电容器的连接}
实际使用电容器时,有时会遇到电容器的电容不够或耐
压能力不够,这就需要把几个电容器连接起来使用,连接的基
本方法有串联和并联两种.
\subsection{电容器的串联}

把几个电容器的极板首尾相接,连成一
串,这就是电容器的串联,图6.40是三个电容器的串联,接
上电压为$U$的电源后,两端极分别带电$+Q$和$-Q$.由于静
电感应,中间各极所带的电量也等干$+Q$或$-Q$,所以串联时
每个电容器带的电量都是$Q$.如果各个电容器的电容分别为
$C_1$、$C_2$、$C_3$,电压分别为$U_1$、$U_2$、$U_3$,那么,
\[ U_1=\frac{Q}{C_1},\qquad U_2=\frac{Q}{C_2},\qquad U_3=\frac{Q}{C_3}\]
总电压$U$等于各个电容器上的电压之和,所以,
\[U=U_1+U_2+U_3=Q\qty(\frac{1}{C_1}+\frac{1}{C_2}+\frac{1}{C_3})\]
设串联电容器的总电容为$C$,则$U=Q/C$,所以
\[\frac{1}{C}=\frac{1}{C_1}+\frac{1}{C_2}+\frac{1}{C_3}\]
这就是说,\textit{串联电容器的总电容的倒数等于各个电容器的电
容的倒数之和}.电容器串联之后,相当于增大了两极的距离,
因此总电容小于每个电容器的电容.

\begin{figure}[htp]\centering
    \begin{circuitikz}[european]
        \draw (0,0)--(0,2);
        \draw (0,2)to [capacitor=$C_1$](2,2)to [capacitor=$C_2$](4,2)to [capacitor=$C_3$](6,2);
        \draw (6,0)--(6,2);    
        \draw (0,0) to [battery] (6,0)   ;
        \node at (3,-.75){$U$};

        \foreach \x in {1,3,5}
        {
            \node at (\x,1.5)[left]{$+Q$};
            \node at (\x,1.5)[right]{$-Q$};

        }
        
    \end{circuitikz}

    \caption{电容器的串联}
\end{figure}



\subsection{电容器的并联} 
\begin{figure}[htp]\centering
    \begin{circuitikz}[european]
\draw (0,0)--(6,0);
\draw (0,3)--(6,3);
\draw (0,3) to [C=$C_1$, *-*](0,0);
\draw (2,3) to [C=$C_2$, *-*](2,0);
\draw (4,3) to [C=$C_3$, *-*](4,0);
\foreach \x in {1,2,3}
{
    \node at (2*\x-2-.5,1.7)[above]{$+Q_{\x}$};
    \node at (2*\x-2-.5,1.3)[below]{$-Q_{\x}$};
}
\draw (6,3) to [battery] (6,0)   ;     
\node at (6.7,1.5){$U$};
    \end{circuitikz}

    \caption{电容器的并联}
\end{figure}	

把几个电容器的正极连在一起,负极也
连在一起,这就是电容器的并联.图6.41是三个电容器的并
联.接上电压为$U$的电源后,每个电容器的电压都是$U$.如
果各个电容器的电容分别为$C_1$、$C_2$、$C_3$,所带电量分别为$Q_1$、$Q_2$、$Q_3$,那么,
\[Q_1=C_1U,\qquad Q_2=C_2U,\qquad Q_3=C_3U\]
电器组贮存的总电量$Q$等于各个电容器所带电量之和,
所以,
\[Q=Q_1+Q_2+Q_3=(C_1+C_2+C_3)U\]
设并联电容器的总电容为$C$,则$Q=CU$,所以,
\[C=C_1+C_2+C_3\]
这就是说,\textit{并联电容器的总电容等于各个电容器的电容之和}.
电容器并联之后,相当于增大了两极的面积,因此总电容大于
每个电容器的电容.

电容器串联后,电容减小了,但耐压能力提高了,所以要
承受较高的电压,可以把电容器串联起来;电容器并联后,电
容增大了,耐压能力没有提高,所以在需要大电容时,可把
电容器并联起来.

\begin{example}
电容器$A$的电容为10微法,充电后电压为30
伏,电容器$B$的电容为20微法,充电后电压为15伏,把它
们的正极连在一起,负极连在一起后,它们的电压是多少?
\end{example}


\begin{solution}
连接前电容器$A$的电量为
\[Q_1=C_1U_1=10\x10^{-8}\x30=3.0\x10^{-4}{\rm C}\]
连接前电容器$B$的电量为
\[Q_2=C_2U_2=20\x10^{-8}\x 15=3.0\x10^{-4}{\rm C}\]
它们的总电量为
\[Q=Q_1+Q_2=6.0\x10^{-4}{\rm C}\]
这个总电量并不会因为连接而改变.
连接后的总电容为
\[C=C_1+C_2=3.0\x10^{-5}{\rm F}\]
连接后的共同电压为
\[U=\frac{Q}{C}=\frac{6.0\x10^{-4}}{3.0\x10^{-5}}=20{\rm V} \]
\end{solution}

有兴趣的同学还可以进一步计算连接后每个电容器的电
量,看看电荷是从哪个电容器流到另一个电容器的.

\subsection*{练习十一}

\begin{enumerate}
\item 两个相同的电容器,标有“100皮法、600伏”,串联后
接到900伏的电路上,每个电容器带多少电?加在每个电容器
上的电压是多少?电容器是否会击穿?
\item 把“100皮法、600伏”和“300皮法、300伏”的电容器
串联后接到900伏的电路上,这样连接是否合适?为什么?
\item 平行板电容器的正对面积为$S$,
两板距离为$\ell$,电介质是真空.如果在两
板之间插入一厚度为$d$的金属板(图6.42),试证明它的电容为
\[C=\frac{S}{4\pi k(\ell-d)}\]
\begin{figure}[htp]\centering
    \begin{tikzpicture}[>=latex]
\draw[ultra thick] (-0.5,0)--(0,0);
\draw[ultra thick] (1,0)--(1.5,0);
\draw [ultra thick] (0,-1.8)--(0,1.8);
\draw[ultra thick]  (1,-1.8)--(1,1.8);
\fill [pattern=north east lines, draw ](.4,-1.8) rectangle (.6,1.8);
\draw (0,-2)--(0,-2.5);\draw (1,-2)--(1,-2.5);
\draw [<->](0,-2.25)--node[fill=white]{$\ell$}(1,-2.25);
\draw (.4,2)--(.4,2.5);\draw (.6,2)--(.6,2.5);
\draw [->](0,2.25)--(.4,2.25);
\draw [->](1,2.25)--(.6,2.25);
\node at (.5,2.5){$d$};
    \end{tikzpicture}

    \caption{}
\end{figure}
\item 电容分别为20微法和50微法的
两个电容器并联后,接在电压为100伏的
电路上,它们共带多少电?
\item 电容为 3000 皮法的电容器带电
$1.8\x10^{-6}$库后,撤去电源,再把它跟电容为1500皮法的电容
器并联,求每个电容器所带电量.
\end{enumerate}

\section{静电的防止和应用}
静电现象是一种常见的自然现象,用塑料梳子梳头发,
梳子会吸引头发,有时还会听到响声,脱下尼龙衣服时,有时
也会听到响声,在黑暗中还能看到火花,静电一般是山摩擦
产生的.当两个物体相互摩擦时,它们带上了异种电荷,它们
之间就产生了电势差,电荷积累到一定数量,电势差达到一
定数值时,就如同电容器被击穿一样,带电物体之间发生放电
现象(见第八章第六节),我们就可能看到火花,听到响声.

静电会给人们带来麻烦和危害,在印刷工业中,纸张上
带的静电吸引空气中的尘埃或带颜色的微粒,会使印刷质量
下降,在合成纤维的生产中,由于静电吸引空气中的尘埃,会
使产品质量下降,在制药生产中,由于静电吸引尘埃,会使药
品达不到标准的纯度,静电对现代高精密度、高灵敏度的电
子设备颇有影响.带静电很多的人甚至可以使那些灵敏、脆
弱、小巧玲珑的电子器件被击穿,因而毁坏一部电子仪器,在
家庭中,带静电很多的人从电视机旁走过,会给电视的图像和
声音带来干扰.

静电的最大危害是有可能因静电火花点燃某些易燃物
质,而引起爆炸.石油在管道中流动时,油类从管道口或管道
裂缝高速喷出时,含有煤尘的空气在风管中流动时,向油槽内
灌油时,都会因摩擦而产生静电,一旦电荷积累相当多,达到
和当高的电压(可以达到上千伏甚至上万伏),就会发生放电
而引起爆炸,静电现象是很普遍的.在外科手术台上也出现
过静电火花使乙醚爆炸的事件.

怎样防止静电的危害呢?最简单而又最可靠的办法是用
导线把设备接地,这样可以把电荷引入大地,避免静电积累.
大型油罐要有很多条接地线,油罐车的尾部拖一条铁链,它就
是车的接地线,调节空气的湿度也是防止静电危害的有效方
法.湿度增大时,电荷随时放出,无法积累,静电危险也就消
除了,潮湿的天气里不容易做好静电实验,也是这个道理.

跟其他物理现象一样,静电现象也可以被人们利用,随
着科学技术的发展,静电技术已经有了长足的进步.

在电场中,带电粒子受到电力的作用,向着电极运动,最
后会被吸在电极上,利用这个道理可以除去或收集空气中的
尘粒,设法使空气中的尘粒带电,在电力作用下尘粒到达电
极而被收集起来,这就是静电除尘,制造各种精密的元件或
仪器,对空气的净化程度要求很高,空气中的微粒多,会严重
影响产品的质量,利静电除尘可以有效地提高空气的净化
程度.静电除尘还可用于粉尘较多的场所,除去空气中对人
们有害的微粒,静电除尘也可用来回收物资,如回收水泥
粉尘.

依据相同的道理,可以利用静电方法在物体上喷涂液体
或固体涂料,达就是静电喷涂,例如设法使油漆的微粒带电,
在电力作用下,油漆微粒飞向作为电极的工件,并沉积在工
件表面上,完成油漆工件的任务,使塑料粉末带电,用静喷
涂可以在工件的表面涂上一层均匀光滑且具有高绝缘性能的
塑料涂层,静电喷涂已广泛地用于对自行车、汽车、电气产品
以及家具进行喷涂,利用类似的方法,使绒毛带电,可以把绒
毛植在涂有粘着剂的纺织物上,形成象刺绣似的纺织品,这就
是静电植绒.在喷雾器和植物(例如苹果树)之间建立起静电
场,可以使农药的液滴或粉粒准确地飞向目标,有效地喷撒
农药.

不同种类的粒子,它们在静电场中表现的行为不同,利用
这个道理我们可以把它们区分开来,这就是静电分选,小麦
等细长形状的种子,在强电场中,它们自身的方向会按照电力
线的方向排列起来,不同的种子使它们按电力线方向排列起
来的场强不同,这样就可以对利子进行分选,静电分选有不
同的分选方式,利用不同的分选方式,可以除去米糠,可以分
选茶叶,可以分选矿石和煤炭,可以分选塑料被覆线的铜芯和
塑料包皮,还可以对城市垃圾进行处理回收.

静电技术还用于摄影、复印等方面,静电复印现在已经
有了广泛的应用,它可以把报纸、书刊、文件等印刷品以及工
程蓝图、手写稿等,直接复印在白纸上,大大提高了复制的
速度.

静电技术应用很广,发展很快,有许多课题在等待着人们
去探索,去研究.

\section*{复习题}
\begin{enumerate}
\item 电荷守恒定律的内容是什么?
\item 库仑定律的内容是什么?写出在真空中和电介质中
库仑定律的公式,静电力恒量的数值是多大?
\item 什么是电场强度?电场中某点的场强在数值上等于
什么?方向是怎样规定的?写出场强的定义式和点电荷在真空
中和在充满电介质空间里各点场强的计算式.
\item 电荷在电场中某点的电势能在数值上等于什么?什
么是电势?电场中某点的电势在数值上等于什么?
\item 什么是电势差?知道了电场中两点的电势差,怎样计
算在这两点间移动电荷时电场力所做的功?在匀强电场中电
势差跟电场强度有什么关系?
\item 用电力线和等勢面可以形象地表示电场,知道了某
一电场的电力线或等势面的分布情况,能不能形象地表示场
强的大小和方向?能不能形象地表示电势的高低?是怎样表
示的?
\item 处于静电平衡状态的导体,内部的场强是怎样的?表
面上任一点场强的方向是怎样的?各点的电势是怎样的?电
荷是怎样分布的?
\item 说明用电场使带电粒子加速和偏转的原理.
\item 说明密立根实验的原理,基本电荷的电量是多大?
\item 什么是电容器的电容?平行板电容器的电容跟哪些
因素有关?各是什么关系?写出平行板电容器电容的公式.
\item 串联电容器的总电容等于么?并联电容器的总电
容等于什么?在什么情况下,需要把电容器串联起来?在什么
情况下,需要把电容器并联起来?
\end{enumerate}

\section*{习题}
\begin{enumerate}
\item 有一个绝缘的金属筒,上面开个小孔,通过小孔放入
一个悬挂在丝线上的带正电的小球,在下列各种情况里,金
属筒外壁各带什么电荷?
\begin{enumerate}
\item 小球跟筒的内壁不接触;
\item 小球跟筒的内壁接触;
\item 小球跟筒的内壁不接触,但用手指接
触一下金属筒,然后移开手指,再把小球移出筒外.
\end{enumerate}

\item 有两个带电小球,电量分别为$+Q$和$+9Q$,在真空
中相距0.4米.如果引进第三个带电小球,正好使三个小球都
处于平衡状态,第三个小球带的是哪种电荷?应放在什么地
方?电量是$Q$的几倍.
\item 如图6.43所示,有两
个挂在丝线上的小球,带有等
量的同种电荷,由于电荷彼此
推斥,丝线都偏离竖直线$\theta$角,
已知两小球的质量都为$m$,两丝线长都为$\ell$,求每个小球上所带的电量.

\begin{figure}[htp]
\centering
\begin{minipage}[t]{0.48\textwidth}
\centering
\begin{tikzpicture}[scale=.9]
	\draw (0,0)--(2,-5);
	\draw (0,0)--(-2,-5);	
	\draw [dashed](0,0)--(0,-6);
	\draw  (2,-5)[fill=gray] circle (5pt);
	\draw  (-2,-5)[fill=gray] circle (5pt);
	\node at (-1.5, -2.5){$\ell$};	\node at (1.5, -2.5){$\ell$};
	\node at (-2.5, -5){$m$};	\node at (2.5, -5){$m$};
	\draw (0,-1) arc (-90:-112:1) node [below]{$\theta$};
	\draw (0,-.9) arc (-90:-68:.9) node [below]{$\theta$};
\end{tikzpicture}
\caption{}
\end{minipage}
\begin{minipage}[t]{0.48\textwidth}
\centering
    \begin{tikzpicture}[>=latex]
\draw [|<->|](0,0)--node [fill=white]{2.0m}(0,5);
\draw [|-|, dashed](-1,0)--(-1,5);        
\draw [thick](-1.4,0)--(-1,5); 
\draw[fill=gray] (-1.4,0) circle (3pt);

\draw[very thick] (-2, -1)--(-2, 1)node [left]{$+$}; \draw[very thick] (-.5, -1)--(-.5, 1)node [right]{$-$};
\draw [|<->|] (-1.4,-.25)--node [below]{2.0cm}(-1,-.25);


    \end{tikzpicture}
\caption{}
\end{minipage}
\end{figure}



\item 在氢原子中,可以认为核外电子沿圆形轨道绕原子
核(质子)旋转,轨道半径为$5.3\x10^{-11}$米,电子沿轨道运动
的动能是多大?
\item 下面一些说法哪个正确,哪个错误?说明理由.
\begin{enumerate}
\item 在匀强电场中电势处处相等.
\item 沿着电力线的方向场强越来越小.
\item 正电荷在电场中只能由电势高的地方向电势低的地
方跑.
\item 电荷在电场中只能向着电势能低的地方跑.
\item 初速度为零的电荷在电场中一定向着电势能低的地
方跑.
\end{enumerate}
\item 两块靠近的平行金属板,在两板之间为真空时,使它
们分别带上等量的异种电荷,保持两板带的电量不变,如果
将两板间的距离减小为原来的1/3,两板间的电势差是原来
的多少倍?两板间匀强电场的场强是原来的多少倍?
\item 上题中,保持两板带的电量不变,而在两板间充满介
电常数为8的电介质,两板间的电势差和匀强电场的场强将
如何改变?
\item 有一个电容器,电容是$1.5\x10^{-4}$微法,把它的两板
分别跟直流电源的正负极相连,使两板分别带电$6\x10^{-8}$库
和$-6\x10^{-8}$库,如果两板的距离为1毫米,电容器两板间的
电场强度是多大?
\item 两个相当大的平行金属板相
距10厘米,两板分别跟电池组的正
负极连接,两板间的一个小电荷受到
的电场力为$3\x10^{-4}$牛,现在把两板
的距离增加到15厘米,如果连接的电
池组不变,小电荷受到的力变为多大?
如果在增大两板距离时把所连电池组
换成3倍电压的电池组,小电荷受到
的力又将变为多大?



\item 如图 6.44,质量为$4.5\x
10^{-3}$千克的带电小球用2.0米长的线
悬挂在带等量异种电荷的平行板之
间,平衡时小球偏离竖直位置 2.0厘
米.
\begin{enumerate}
	\item 小球受到的电场力是多大?
	\item 如果两板间的电压是$1.5\x10^4$伏,两板的距离是10厘米,
那么,小球带有多少个多余的电子?
\end{enumerate}
	

\item 在图6.29中,先让一束电子,后让一束氢核通过偏
转电场,设电子和氢核的初速度相同,电子和氢核原来
的动能相同,试分别求出两种情况下电子的偏角$\phi_e$和氢核
的偏角$\phi_H$的正切之比.

\begin{figure}[htp]\centering
	\includegraphics[scale=.5]{fig/6-45.png}
	\caption{}
	\end{figure}
	
\item 图6.45是用来使带正电的粒子加速和偏转的装
置,如果让氢和氦进入并电离,我们将得到一价氢离子,一价
氦离子和二价氦离子的混合物.它们经过同一电场加速后,
在同一电场里偏转,它们是否会分为三股,从而到达荧光屏后
产生三个亮点?回答中要说明理由.

\end{enumerate}












































\chapter{稳恒电流}
电在生产和生活中的应用越来越广泛,其中许多应用都跟电流有关.电流是在电路中流动的,为了有效地利用和控制电流,需要研究电路的规律,这一章我们就来学习电路的基本规律和它们的应用,欧姆定律是分析电路的基础,一定要掌握好,要注意学会用它来分析解决一些简单的电路问题.

\section{电流}
接通电源后,电灯就发光,电炉就发热,电动机就转动,这是因为它们中有了电流,我们在初中学过,\textit{电荷的定向移动形成电流}.金属导体中的电流是自由电子定向移动形成的.电解液中的电流是正、负离子向相反方向移动形成的.

要形成电流,首先要有能够自由移动的电荷——自由电荷,金属中的自由电子,电解液中的正、负离子,都是自由电荷,但是只有自由电荷还不能形成电流.导体中有大量的自由电荷,它们不断地做无规则的热运动,朝任何方向运动的机会都一样.在通常情况下,对导体的任何一个截面来说,在任何一段时间内从截面两侧穿过截面的自由电荷数都相等(图7.1),从宏观上看,没有电荷的定向移动,因而也没有电流.
\begin{figure}[htp]\centering
    \includegraphics[scale=.6]{fig/7-1.png}
    \caption{} 
    \end{figure}

如果把金属导体的一端接在带正电的物体$A$上,另一端接在带负电的物体$B$上(图7.2),那么,由于$A$的电势高,$B$的电势低,导体两端有了电势差,导体内就建立了电场.这时,自由电子除了做无规则的热运动外,还要在电场力的作用下做定向移动,即沿着导体从电势低的一端向电势高的一端移动,于是导体中有了电流.但在图7.2所示的情形中不能得到持续的电流.这是因为:随着电子的移动,$A$、$B$上的正、负电荷将逐渐减少,导体两端的电势差也随着减小;当达到静电平衡状态时,导体两端的电势差变为零,导体内的电场强度也变为零,电子不再做定向移动,电流也就消失了.可见,为了使导体中有持续电流,必须设法保持导体两端的电势差,\textit{导体中存在持续电流的条件,是保持导体两端的电势差}.电源的作用就是保持电路两端的电势差,使电路中有持续的电流.
\begin{figure}[htp]\centering
\includegraphics{fig/7-2.pdf}
\caption{}
\end{figure}


电流有强弱的不同,电流的强弱用电流强度来表示.\textit{通过导体横截面的电量跟通过这些电量所用的时间的比值,叫做电流强度}.如果时间$t$内通过导体横截面的电量为$q$,那么电流强度
\[I=\frac{q}{t} \]
在国际单位制中,电流强度的单位是\textbf{安培},简称安,国际
符号是A,如果在1秒内通过导体横截面的电量是1库仑,导体中的电流强度就是1安培,常用的电流强度的单位还有毫安(mA)、微安($\mu$A).
\[\begin{split}
    1{\rm mA}&= 10^{-3}{\rm A}\\
    1{\rm \mu A}&=10^{-6}{\rm A}
\end{split}\]

导体中的电流既可以是正电荷的移动,也可以是负电荷的移动,还可以是正、负电荷沿相反方向的移动.因为负电荷的移动可看作正电荷沿相反方向的移动,所以为了便于分析问题,习惯上规定正电荷的移动方向为电流的方向,这样,在金属导体中电流的方向就与自由电子移动的方向相反.在电解液中,电流的方向与正离子移动的方向相同,与负离子移动的方向相反.我们知道,正电荷在电场力的作用下是从电势高处向电势低处移动,所以导体中电流的方向是从电势高的一端流向电势低的一端,电源上电势高的电极叫正极,电势
低的电极叫负极,所以在电源外部的电路中,电流的方向是从电源的正极流向负极.

电路中,如果电流的方向不随时间而改变,这样的电流叫做\textbf{直流电};如果电流的方向和大小都不随时间而改变,这样的电流叫做\textbf{稳恒电流},这一章我们研究稳恒电流.

\section{欧姆定律}
在导体两端加上电压,导体中就有电流,导体中电流的强弱跟加在导体两端的电压有什么关系呢?德国物理学家欧姆(1787—1854)通过实验研究,对导体中电流与电压的关系
得出了如下的结论:\textit{通过导体的电流跟加在导体两端的电压成正比},即$I\propto U$.通常把这个关系写作
\[\frac{U}{I}=R\]

式中$R$是电压与电流的比值,实验表明,对同一根导线来说,不管电压和电流的大小念样变化,比值$R$都是相同的,对于不同的导线,$R$的数值一般是不同的.这表明,$R$是一个跟导体本身有关的量,导线的$R$越大,在同一电压下,通过它的电流就越小.可见,比值$R$反映出导线对电流的阻碍作用,我们把它叫做导体的\textbf{电阻}.

上面的公式可写成
\[I=\frac{U}{R}\]
这个公式表示\textbf{导体中的电流强度跟导体两端的电压成正比,跟导体的电阻成反比}.这就是我们在初中学过的\textbf{欧姆定律}.

根据欧姆定律可以规定电阻的单位,电阻的单位是\textbf{欧姆},简称欧,国际符号是$\Omega$. 它是这样规定的:如果在某段导线两端加上1伏特电压,通过它的电流强度是1安培,这段导线的电阻就是1欧姆.
\[1\Omega=\frac{1{\rm V}}{1{\rm A}} \]

1常用的电阻单位还有千欧(k$\Omega$)和兆欧(M$\Omega$).
\[\begin{split}
    1{\rm k}\Omega&= 10^{3}\Omega\\
    1{\rm M}\Omega&=10^{6}\Omega
\end{split}\]

应该注意的是,欧姆定律是在金属导电的基础上总结出来的,对于其他导体是否适用,还要经过实验的检验.实验结果是,除金属外,欧姆定律对电解液导电也适用,但对气体导电就不适用了.

导体中电流跟电压的关系还可以用图线来表示.用横轴表示电压,纵轴表示电流强度,画出的电压-电流图线,叫做导体的伏安特性曲线.在金属导体中,电流强度跟电压成正比,所以伏安特性曲线是通过坐标原点的直线(图7.3),直线的斜率等于导体电阻的倒数.导体的电阻越大,伏安特性曲线的斜率越小.同学们可以自己研究一下,图7.3所示的两条伏安曲线,哪一条所代表的电阻大.
\begin{figure}[htp]
    \centering
\begin{tikzpicture}[>=latex, thick]
    \draw[<->] (0,3)node [left]{$I$}--(0,0)node [below]{$O$}--(4,0)node [above]{$U$};
    \draw (0,0)--(2,2.7)node [above]{1};
    \draw (0,0)--(3,2.6) node [above]{2};
\end{tikzpicture}
    \caption{金属导体的伏安特性曲线}
\end{figure}


对欧姆定律不适用的导体,它们中的电流与电压不是正
比关系,伏安特性曲线不再是直线.

欧姆定律是电流的基本定律,在研究电路时有很重要的应用,电流、电压和电阻是电路中的三个基本物理量,根据欧姆定律,知道了其中任意两个量,就可以求出另外一个量,例如,在保持电压一定的情况下,可以通改变电阻的办法来控制电路中的电流强度,这在实际中是经常遇到的.


\subsection*{练习一}

\begin{enumerate}
    \item 导线中的电流强度为10安,20秒钟内有多少电子通过导线的横截面?
    \item 手电简小灯泡上的电压是3伏时,电阻为8.5欧,求通过小灯泡的电流强度.
    \item 人体通过50毫安的电流时,就会引起呼吸器官麻痹,如果人体的最小电阻为800欧,求人体的安全工作电压.
    \item 根据上题中所给的数字说明:为什么人体触到220伏的电线时会发生危险,而接触干电池的两极(电压为1.5伏)时却没有感觉?
    \item 电路中有一电阻,测得通过它的电流强度是2毫安时,电阻两端的电压是50毫伏,在通过它的电流强度为15毫安时,它两端的电压是多大?
    \item 画出电限为5欧的导体的伏安特性曲线,当导体的电阻增大为10欧时,图线持怎样变化?电阻减小为2.5欧时呢?
\end{enumerate}


\section{电阻定律~~电阻率}
导体的电阻是由导体本身决定的,那么,决定导体电阻大小的因素究竞有哪些呢?

实验表明,用同一种材料制成的横截面积相等而长度不相等的导线,其电阻跟导线的长度成正比;长度相等面横截面积不相等的导线,其电阻跟导线的横截面积成反比,\textit{导线的电阻跟它的长度成正比,跟它的横截面积成反比}.这就是\textbf{电阻定律},用公式来表示可以写作
\[R=\rho\frac{\ell}{S}\]

式中的比例系数$\rho$跟导体的材料有关系.在一定的温度
下,对同一种材料$\rho$是一个常数,对不同的材料$\rho$的数值不同,横截面积和长度都相等的不同材料的导线.$\rho$越大的电阻越大,$\rho$越小的电阻越小.可见,$\rho$是一个反映材料导电性好坏的物理量,叫做材料的\textbf{电阻率}.

把上面的公式改写作
\[\rho=R\frac{S}{\ell}\]
式中$\ell=1{\rm m}$,$S=1{\rm m^2}$时,$\rho$的数值等于$R$.可见,材料的电阻率在数值上等于这种材料制成的长1${\rm m}$、横截面积1${\rm m}^2$的导体的电阻.

根据上式,可以确定电阻率$\rho$的单位,$R$的单位是$\Omega$,$S$的单位是${\rm m^2}$,$\ell$的单位是m.所以$\rho$的单位是欧姆·米,简称欧·米,国际符号是$\Omega\cdot {\rm m}$.

下表列出了几种材料20$^\circ$C时的电阻率.

\begin{center}
    \begin{tabular}{cc}
\hline
        材料  & $\rho$($\Omega\cdot {\rm m}$)\\
    \hline
银   & $ 1.6\x 10^{-8}$   \\
铜  & $ 1.7\x 10^{-8}$   \\
铝  & $ 2.9\x 10^{-8}$   \\
钨  & $ 5.3\x 10^{-8}$   \\
铁  & $ 1.0\x 10^{-7}$   \\
锰铜(85\%铜+3\%镍+12\%锰)  & $ 4.4\x 10^{-7}$   \\
康铜(54\%铜+46\%镍) & $ 5.0\x 10^{-7}$   \\
镍铬合金(67.5\%镍+15\%铬+16\%铁+1.5\%锰) & $1.0 \x 10^{-6}$   \\
    \hline
    \end{tabular}
\end{center}

从上表可以看出,纯金属的电阻率小,合金的电阻率较大,金属中银的电阻率最小,但银的价格昂贵,通常很少用银做导线,只在特殊需要时使用.导线一般都用电阻率较小的铜或铝来制作,铝比铜便宜,因此铝导线用得很多.电炉、电阻器的电阻丝一般都用电阻率较大的合金来制作.

各种材料的电阻率都随温度而变化,金属的电阻率随温度的升高面增大.因此金属导体的电阻也随温度的升高而增大,利用金属电阻的这种性质可以制作电阻温度计.如果已知导体电阻随温度的变化情况,那么,测出导体的电阻,反过来就可以知道温度.常用的电阻温度计是用铂丝或铜丝制作的,铂在温度变化时性质稳定,测温范围宽,可靠性好,但是价格昂贵,铜电阻温度计装置简单,灵敏,在某些特殊条件下,有独特的优点.有些合金,例如康铜和锰铜的电阻率随温度的变化特别小,用这些合金制作的电阻受温度的影响很小,因此常用来作标准电阻.

当温度降低到绝对零度附近时,某些金属、合金和化合物的电阻率会突然减小为零,这种现象叫做超导电现象,处于这种状态的导体叫做\textbf{超导体}.超导体的电阻为零,它还具有一系列其他独特的物理性质,有很重要的实用价值.目前,超导体需要的温度很低,使它的应用受到限制,我国和其他各国现在都在积极进行研究,寻找较高温度下的超导体,探索把超导体应用到实际中去的可能性.

\subsection*{练习二}

\begin{enumerate}
    \item 图7.4是滑动变阻器的结构图,涂有绝缘漆的电阻丝密绕在瓷管上,$A$、$B$是它的两个端点,滑动端$P$可在金属杆上移动,它通过金属片与电阻丝接触,把电阻丝和金属杆连接起来.如果把固定端$A$和接线柱$C$接入电路中,当滑动端从$B$向$A$移动时,电路中的电阻就随着变小,说明其道理.
    \begin{figure}[htp]\centering
        \includegraphics[scale=1.5]{fig/7-4.PDF}
        \caption{滑动变阻器}
        \end{figure}
    \item 一卷铝导线长100米,横截面积为1${\rm mm}^2$,这卷导线的电阻是多大?
    \item 有一段导线,电阻是4欧,把它对折起来作为一条导线用,电阻是多大?如果把它均匀拉长到原来的两倍,电阻又是多大?
    \item 一根做电学实验用的铜导线,长度是60厘米,横截面积是0.5${\rm mm}^2$,它的电阻是多少欧?一根输电用的铜导线,长度是10千米,横截面积是1${\rm cm}^2$,它的电阻是多少欧?为什么做电学实验时可以不考虑连接用的铜导线的电阻,而对输电线路的导线的电阻则需要考虑?
    \item 一根电阻丝长10米,横截面积是0.2${\rm mm}^2$,两端加上10伏电压时,通过的电流强度是0.2安.这根电阻丝的电阻率是多大?它是用什么材料制作的?
\end{enumerate}

\section{电功和电功率}
我们在初中已经学过电功和电功率,利用上一章所学的知识,我们可以更好地理解这两个重要概念.

在导体两端加上电压,导体内就建立了电场.电场力在推动自由电子定向移动中要做功,如果导体两端的电压为$U$,通过导体横截面的电量为$q$,那么,从上一章讲的可知,电场力所做的功$W=qU$.由于$q=It$,所以,
\[W=UIt\]
上式中$W$、$U$、$I$、$t$的单位应分别用焦耳、伏特、安培、秒.

电场力做的功常常说成是电流做的功,简称电功.所以,\textit{电流在一段电路上所做的功,跟这段电路两端的电压、电路中的电流强度和通电时间成正比}.

电场力做功时,正电荷从导体电势高的一端移向电势低的一端,电势能减少.这时减少的电能转化为其他形式的能.可见,电流通过用电器做功的过程,实际上是电能转化为其他形式的能的过程.例如,电流通过电炉做功,电能转化为内能;电流通过电动机做功,电能转化为机械能;电流通过电解槽做功,电能转化为化学能.电流做了多少功,就有多少电能转化为其他形式的能.

电流所做的功跟完成这些功所用的时间的比值叫做电功率,用P表示电功率,那么
\[P=\frac{W}{t}=UI\]
上式中$P$、$U$、$I$的单位应分别用瓦特、伏特、安培.

可见,\textit{一段电路上的电功率,跟这段电路两端的电压和电路中的电流强度成正比}.

为了使用电器安全正常地工作,制造厂对用电器的电功率和工作电压都有规定的数值,并且标明在用电器上,叫做用
电器的额定功率和额定电压.给用电器加上额定电压,用电器正常工作时的功率就是额定功率,例如,标有“220V,40W”的灯泡,接在220伏的线路中,灯泡正常发光,它的功率为40瓦,这时通过灯泡的额定电流为$40/220=0.18$安,如果接在高于220伏的线路中,通过灯泡的电流增大,它消耗的实际功率也增大,灯泡有烧坏的危险;如果接在低于220伏的线路中,通过灯泡的电流减小,它消耗的实际功率也减小,灯泡将变得昏暗不亮.可见,加在用电器上的电压改变时,通过它的电流也改变,它的实际功率也随着改变.

\section{焦耳定律}
\subsection{焦耳定律}


电流通过导体时要产生热,使导体的内能增加,温度升高,这就是电流的热效应,英国物理学家焦耳用实验研究了这个问题,指出:
\textit{电流通过导体产生的热量,跟电流强度的平方、导体的电阻和通电时间成正比},这就是我们在初中学过的\textbf{焦耳定律}.如果热量$Q$的单位用焦耳,电流强度$I$的单位用安培,电阻$R$的单位用欧姆,时间$t$的单位用秒,焦耳定律可以写成如下的公式
\[Q=I^2Rt\]

电流的热效应在生产和生活中有许多实际应用,电灯、电炉、电烙铁、电烘箱等都是利用电流的热效应制作的.但是,电流的热效应在有些地方是有害的、例如,电流通过输电导线、电动机的线圈、电视机中的零件时都要产生热,这不仅
白白消耗电能,而且如果产生的热量使温度升高过多,还会使它们损坏,因此实际中要注意通风散热.

\subsection{电功和电热的关系} 

电流通过电路时要做功,同时,一般电路都是有电阻的,因此电流通过电路时也要产生热.那么,电流做的功跟它产生的热之间又有什么关系呢?

如果电路中只含有电灯、电炉等纯电阻性元件,即所谓纯电阻电路,由于这时电路两端的电压$U=IR$,因此$UIt=I^2Rt$.这就是说,电流所做的功$UIt$跟产生的热量$I^2Rt$是相等的.在这种情况下,电能完全转化为内能.这时电功的公式也可以写成
\[W=I^2Rt=\frac{U^2}{R}t\]

如果电路中还包含有电动机、电解槽等用电器,即电路不
是纯电阻性的,那么,电能除部分转化为内能外,还要转化为机械能、化学能等,这时电功仍然等于$UIt$,产生的热量仍然等于$I^2Rt$,但电流所做的功已不再等于产生的热量,而是大于这个热量;加在电路两端的电压$U$也不再等于$IR$,而是大于$IR$了.在这种情况下,就不能再用$I^2Rt$或$\dfrac{U^2}{R}t$来计算电功.

例如,一台电动机,额定电压是110伏,正常工作时通过的电流是50安.每秒钟内电流做的功是$UIt=5.5\x10^3$焦.电动机线圈的电阻只有0.40欧,每秒钟产生的热量是$I^2Rt=1.0\x10^3$焦.电功比电热大很多,大部分电能变为机械能了.

总之,只有在纯电阻电路里,电功才等于电热;在非纯电阻电路里,要注意电功和电热的区别.

\subsection*{练习三}
\begin{enumerate}
    \item 额定电压相同、额定功率不同的两只灯泡,哪个的额定电流大?哪个的电阻大?
    \item 用户保险盒中安装的保险续允许通过的最大电流一般都不大(几个安培),如果在电路中接入功率在1000瓦以上的用电器,如电炉等,就会把保险丝烧断,这是为什么?
    \item 日常使用的电功单位是“度”,等于功率为1千瓦的电流在1小时内做的功,又叫千瓦时,1度等于多少焦?
    \item 有一个1千瓦、220伏的电炉,正常工作时电流是多少?如果不考虑温度对电阻的影响,把它接在110伏的电压上,它的功率将是多少?
    \item 输电线的电阻共计1.0欧,输送的电功率是100千瓦,用400伏的低压送电,输电线上发热损失的功率是多少千瓦?改用1万伏的高压送电呢?
    \item 用功率为2千瓦的电炉把2千克的水从20$^\circ$C加热到100$^\circ$C,如果电炉的效率为30\%,需要多少时间?水的比热为$4.2\x10^3{\rm J}/({\rm kg}\cdot {}^\circ {\rm C})$.
\end{enumerate}

\section{串联电路}
把导体一个接一个地依次连接起来,就组成串联电路.
图7.5是由三个电阻$R_1$、$R_2$、$R_3$组成的串联电路.在串联电路中,电流沿着一条通路依次流过各个电阻,没有分岔,因此\textit{流过串联电路各电阻的电流强度相等}.
\begin{figure}[htp]\centering
    \begin{circuitikz}[>=latex, european]
\draw (0,0) to [R=$R_1$] (2,0) to [R=$R_2$] (4,0) to [R=$R_3$] (6,0);
\draw [->](-.5,0)--(.25,0)node[above]{$I$} ;  \draw [->](2,0)--(2.25,0)node[above]{$I$};     \draw [->](4,0)--(4.25,0)node[above]{$I$};
\draw [->](5.75,0)--(6.5,0)node[above]{$I$};

\draw[<->] (.25,-.75)--node[fill=white]{$U_1$} (2.25,-.75);
\draw[<->] (2.25,-.75)--node[fill=white]{$U_2$}(4.25,-.75);
\draw[<->] (4.25,-.75)--node[fill=white]{$U_3$}(6.5,-.75);

\foreach \x in {.25,  6.5}
{
    \draw (\x,-.2)--(\x, -1.8);
}
\draw[<->] (.25,-1.5)--node[fill=white]{$U$}(6.5,-1.5);

\foreach \x in { 2.25, 4.25}
{
    \draw (\x,-.2)--(\x, -1.2);
}


    \end{circuitikz}
    \caption{串联电路}
\end{figure}	

电流通过串联电路各电阻时,沿电流方向每通过一个电阻,电势要降低一定的数值,因此电阻两端的电压又叫做电势降落.电流在各电阻上的电势降落之和就是串联电路两端的
电势降落,即总电压(图7.6).\textit{串联电路两端的总电压等于各部分电路两端的电压之和}.在图7.5中就是
\[U=U_1+U_2+U_3\]
\begin{figure}[htp]\centering
    \begin{circuitikz}[>=latex, european]
\draw (0,0) to [R=$R_1$] (3,0)to [R=$R_2$] (5,0)to [R=$R_3$] (8,0);
\draw [fill=white](0,0) circle (2pt);\draw [fill=white](8,0) circle (2pt);
\draw [<->](0,-.2)--(0,-5)--(8,-5);
\draw [very thick] (0,-.8)node[left]{$U$}--(1,-.8)--(2,-1.5)--(3.5,-1.5)--(4.5, -3)--(6,-3)--(7,-5);

\foreach \x in {1,2,3.5,4.5,6,7}
{
    \draw [dashed](\x, -.25)--(\x, -5);
}

\draw [<->|](1,-.8)--node [left]{$U_1$}(1,-1.5);
\draw [<->|](3.5,-1.5)--node [fill=white]{$U_2$}(3.5,-3);
\draw [<->|](6,-3)--node [fill=white]{$U_3$}(6,-5);
\draw [dashed](1, -1.5)--(2, -1.5);
\draw [dashed](3.5, -3)--(4.5, -3);

    \end{circuitikz}
    \caption{}
\end{figure}	


上面讲的是串联电路中电流、电压的基本特点,利用欧姆定律和这些基本特点来研究串联电路的总电阻、电压分配、功率分配等,还可以得出一些特殊的关系和结论,这些结论对于分析、计算电路很有用处.下面我们就分别来进行研究.

(1) \textbf{串联电路的总电阻 } 串联电路中的几个电阻可以用一个电阻来代替,把这个电阻接入串联电路两端时,在相同的
总电压下,通过电路中的电流强度跟原来的相等,也就是说,这个电阻在电路中的作用效果跟原来的几个串联电阻一样.这样的电阻叫做串联电路的等效电阻,也叫串联电路的总电阻.如果用R表示总电阳,那么,根据欧姆定律,在图7.5中,
\[U=IR,\qquad  U_1=IR_1,\qquad U_2=IR_2,\qquad U_3=IR_3\]
把上面的公式代入$U=U_1+U_2+U_3$中,整理后可得
\[R=R_1+R_2+R_3\]

如果串联电路中有$n$个电阻,同理可以推出
$R=R_1+R_2+R_3+\cdots+R_n$

这就是说,\textit{串联电路的总电阻,等于各导体的电阻之和}.

利用中联电路电阻的这个规律,在需要增大电路的电阻时,我们就可以在电路中串联上一个或几个电阻.

(2) \textbf{串联电路的电压分配 } 串联电路两端的总电压等于
各串联电阻上的电压之和,那么各电阻上的电压跟电阻有什么关系呢?在串联电路中,由于
\[U=IR,\;  U_1=IR_1,\; U_2=IR_2,\; U_3=IR_3,\ldots, U_n=IR_n\]
所以,
\[\frac{U_1}{R_1}=\frac{U_2}{R_2}=\frac{U_3}{R_3}=\cdots=\frac{U_n}{R_n}=I \]
这表明,\textit{串联电路中各个电阻两端的电压跟它的阻值成正比}.阻值越大的电阻,两端的电压也越大.

串联电路的每个电阻都分担了一部分电压.在电路中的电压超过用电器额定电压的情况下,可以在电路中串联上电阻,以分去一部分电压,使用电器得到所需的电压,串联电阻的这种作用叫做分压用,起分压作用的电阻叫做分压电阻.

在电学实验中,常用滑动变阻器接成分压器电路来调节
用电器或工作电路所需的电压的大小.滑动变阻器用作分压器时的电路如图7.7所示,变阻器的两个固定端连接在电源$ab$上,其中一个固定端和滑动端$P$跟用电器或工作电路的两端相连.改变滑动端在两个固定端间的位置,输出电压$U_{cd}$就可以在0—$U$之间变化.

(3) \textbf{串联电路的功率分配 } 串联电路中某个电阻$R_k$消耗的功率$P_k=U_kI$,而$U_k=IR_k$,所以$P_k=I^2R_k$.因此各个电阻消耗的功率分别是
\[P_1=I^2R_1,\; P_2=I^2R_2,\; P_3=I^2R_3,\ldots, P_n=I^2R_n\]
所以
\[\frac{P_1}{R_1}=\frac{P_2}{R_2}=\frac{P_3}{R_3}=\cdots\frac{P_n}{R_n}=I^2 \]

这就是说,\textit{串联电路中各个电阻消耗的功率跟它的阻值成正比}.在串联电路中,阻值越大的电阻,消耗的功率越大.

整个串联电路消耗的总功率
\[P=UI=I^2R_1 +I^2R_2+I^2R_3+\cdots+I^2R_n
\]
所以
\[P=P_1+P_2+P_3+\cdots +P_n\]

可见,串联电路中消耗的总功率等于各部分电路消耗的功率之和.

\begin{figure}[htp]
    \centering
    \begin{minipage}[t]{0.48\textwidth}
    \centering
    \begin{circuitikz}[european, >=latex]
        \draw (0,0)node [left]{$b$}--(3,0)node [right]{$d$};
        \draw (0,5)node [left]{$a$}--(1.5,5) to [R, *-*] (1.5, 0);
        \node at (1.9,2.7){$P$};
        \draw [<-](1.7,2.5)--(3,2.5);
        \draw [<->](3,0.1)--node [fill=white] {$U_{cd}$} (3,2.4);
        \draw [<->](0,0+.1)--node [fill=white] {$U$} (0,5-.1);
        \draw [fill=white](0,0) circle (2pt);
        \draw [fill=white](0,5) circle (2pt);
        \draw [fill=white](3,0) circle (2pt);
        \draw [fill=white](3,2.5) circle (2pt)node [right]{$c$};    
            \end{circuitikz}
                \caption{分压器电路}
    \end{minipage}
    \begin{minipage}[t]{0.48\textwidth}
    \centering
    \begin{circuitikz}[european, >=latex]
        \draw (0,0)--(3,0);
        \draw (0,5)--(1.5,5) to [R=$R_2$] (1.5,2.5) to [R=$R_1$, *-*] (1.5, 0);
        \draw (1.5,2.5)--(3,2.5);
        \draw [<->](3,0.1)--node [fill=white] {$U_1$} (3,2.4);
        \draw [<->](0,0+.1)--node [fill=white] {$U$} (0,5-.1);
        \draw [fill=white](0,0) circle (2pt);
        \draw [fill=white](0,5) circle (2pt);
        \draw [fill=white](3,0) circle (2pt);
        \draw [fill=white](3,2.5) circle (2pt);    
            \end{circuitikz}
    \caption{}
    \end{minipage}
    \end{figure}

\begin{example}
    图7.8是实际中常用的一种分压电路,用分压电阻$R_2$分去一部分电压,使$R_1$上得到所需的电压$U_1$.如果加在该电路上的电压$U=50$伏,电阻$R_1=1$千欧,要使$U_1=
    5$伏,分压电阻$R_2$的阻值应为多大?
\end{example}


\begin{solution}
    设$R_2$上的电压为$U_2$,电路中的电流强度为$I$.根据欧姆定律,$U_2=IR_2$,可以求得
    \[R_2=\frac{U_2}{I}\]

因为串联电路两端的电压等于各部分电路电压之和,所
以在图7.8中$U=U_1+U_2$.由此可求出$R_2$上的电压
\[U_2=U-U_1=(50-5)=45{\rm V}\]

从$U_1=IR_1$,可得
\[ I=\frac{U_1}{R_1}=\frac{5}{1000}=5\x10^{-3}{\rm A}\]

将求得的$I$、$U_2$代入$R_2$中,即得
\[R_2=\frac{U_2}{I}=\frac{45}{5\x10^{-3}}\Omega=9{\rm k}\Omega\]

$R_2$也可以利用串联电路的电压分配公式求得,因为
\[\frac{U_1}{R_1}=\frac{U_2}{R_2}\]
由此可得
\[R_2=\frac{U_2R_1}{U_1}=\frac{45/1000}{5}\Omega=9{\rm k}\Omega \]
\end{solution}

显然,后一种解法简便得多,可见,掌握了串联电路的特殊规律,可以使电路的分析、计算得到简化.

\subsection*{练习四}
\begin{enumerate}
    \item 电炉和导线是串联的,把它们接入电源后,导线和电炉丝中通过的电流强度是否一样?为什么这时电炉丝热得发红,导线并不热?
    \item 某同学要为游艺晚会准备一棵装有彩色电灯的小松树,如果所用的每只灯泡的额定电压是8伏,用220伏的市电做电源,那么需要将多少只灯泡串联在一起才能接在电源上?
    \item 一个量程为150伏的电压表,内阻为20千欧,把它与一高电阻串联后接在110伏的电路上,电压表的读数是5伏.求高电阻的阻值是多少?(这是测量高电阻的一种方法)
    \item 直流电动机线圈的电阻很小,所以起动时的电流很大,这对电动机本身和接在同一电源上的其他用电器都产生不良的后果.为了减小电动机起动时的电流,需要给电动机串联一个起动电阻$R$,如图7.9所示,电动机起动后再将$R$逐渐减小,如果电源电压$U=220$伏,电动机的线圈电阻$r_0=2$欧,那么,
    
    \begin{enumerate}
        \item 不串联电阻$R$时的起动电流是多大?
        \item 为了使起动电流减小为20安,起动电阻应为多大?
    \end{enumerate}
    \item 图7.10是一个变阻器分压电路,如果电压$U=12$伏,$R_1=350$欧,$R_2=270$欧,$R_3=550$欧,那么,滑动端$P$从$R_2$下端向上移动时,$a$、$b$间的电压将怎样变化?当$P$在$R_2$最下端和最上端时,$a$、$b$间的电压各是多少?	
\end{enumerate}

\begin{figure}[htp]
\centering
\begin{minipage}[t]{0.48\textwidth}
\centering
        \begin{circuitikz}[european,>=stealth]
            \draw(0,0)--(5,0) to  (5,3);
            \draw (0,3) to [vR=$R$] (5,3);
            
            \draw[<->](0,.1) to node [fill=white]{$U$} (0,3-.1);
            \draw (5,1.5) node[elmech]{};
            \draw [fill=white](0,0) circle (2pt);
            \draw [fill=white](0,3) circle (2pt);  
            \node at (5.7, 1.5){$r_0$};
        \end{circuitikz}
    
        \caption{}
        \end{minipage}
\begin{minipage}[t]{0.48\textwidth}
\centering
\begin{circuitikz}[european,>=latex]
    \draw (0,0)--(2,0) to [R=$R_3$] (2,2) to [R=$R_2$] (2,4) to [R=$R_1$] (2,6)--(0,6);
    \draw (2,0)--(4,0)node[right]{$b$};
    \draw [<-](2.2,3)--(4,3)node[right]{$a$};
    \draw [fill=white](4,3) circle (2pt);
    \draw [fill=white](4,0) circle (2pt);      
    \draw [fill=white](0,0) circle (2pt);
    \draw [fill=white](0,6) circle (2pt);   
    \draw[<->](0,.1) to node [fill=white]{$U$} (0,6-.1);
    \node at (2.4,2.7){$P$}; 
        \end{circuitikz}
\caption{}
\end{minipage}
\end{figure}

\section{并联电路}

\subsection{并联电路}


把几个导体并列地连接起来,就组成了并联电路.同一电路上的各个用电器,通常都是采用并联接法.图7.11是三个电阻$R_1$、$R_2$、$R_3$组成的并联电路.

从图7.11中可以看出,三个并联电阻的首端都连接在一点$A$上,尾端都连接在一点$B$上,所以每个电阻两端的电压都等于$A$、$B$两点间的电压.由此可知,\textit{并联电路中各支路两端的电压相等}.

电流通过并联电路时,总电流分成几条支路.在初中我们通过实验已经知道,并联电路中各支路的电流强度之和等于总电流强度,在图7.11中,流入$A$点的电流$I$等于从该点流出的电流$I_1$、$I_2$、$I_3$之和,即
\[I=I_1+I_2+I_3\]
所以,\textit{并联电路中的总电流强度等于各支路电流强度之和}.
\begin{figure}[htp]\centering
    \begin{circuitikz}[european,>=latex]
        \draw(0,0)to [R=$R_2$] (6,0);  \draw [->](0,0)--(.5,0)node [above]{$I$};
        \draw [->](1,0)--(1.75,0)node [above]{$I_2$};
        \draw [->](1,-1.5)--(1.75,-1.5)node [above]{$I_3$};
        \draw [->](1,1.5)--(1.75,1.5)node [above]{$I_1$};
        \draw (1,0)--(1,-1.5) to [R=$R_3$] (5,-1.5)--(5,0);
        \draw (1,0)--(1,1.5) to [R=$R_1$] (5,1.5)--(5,0);
        \draw [->](5,0)--(5.5,0)node [above]{$I$};
        \draw [fill=black](1,0) circle (2pt); \node at (1,-.25)[left]{$A$};
        \draw [fill=black](5,0) circle (2pt); \node at (5,-.25)[right]{$B$};
        \draw (1,-1.6)--(1,-2.6);\draw (5,-1.6)--(5,-2.6);
        \draw[<->](1,-2.1) to node [fill=white]{$U$} (5,-2.1);

    \end{circuitikz}

    \caption{并联电路}
\end{figure}	

以上是并联电路中电压、电流的基本特点,应用欧姆定律和这些基本特点来研究并联电路的总电阻、电流分配和功率分配,也可以得出一些有用的关系式和结论.

(1) \textbf{并联电路的总电阻 } 并联电路的几个电阻也可以用
一个电阻来代替,把这个电阻接在并联电路的两端时,在相同
的电压下,电路中的总电流保持变,这样的电阻叫做并联电路的等效电阻,也叫做并联电路的总电阻.用$R$代表并联电路的总电阻,根据欧姆定律,在图7.11中,
\[I=\frac{U}{R};\quad I_1=\frac{U_1}{R_1};\quad I_2=\frac{U_2}{R_2};\quad I_3=\frac{U_3}{R_3}\]
将上面各式代入$I=I_1+I_2+I_3$中,整理后可得
\[\frac{1}{R}=\frac{1}{R_1}+\frac{1}{R_2}+\frac{1}{R_3} \]
如果电路中有$n$个导体并联,同理可以推出
\[\frac{1}{R}=\frac{1}{R_1}+\frac{1}{R_2}+\frac{1}{R_3}+\cdots+\frac{1}{R_n} \]
这就是说,\textit{并联电路总电阻的倒数,等于各个导体的电阻倒数之和}.所以,并联电路的总电阻比每一个电阻都小,利用这个规律,在需要减小某一部分电路的电阻时,只要在这部分电路中并联上一个适当的电阻就行了.

(2) \textbf{并联电路的电流分配 } 并联电路中各支路两端的电压相等,根据欧姆定律,在并联电路中,
\[I_1R_1=I_2R_2=I_3R_3=\cdots=I_nR_n=U \]

这就是说,\textit{并联电路中通过各导体的电流强度跟它的电阻成反比}.电阻越小的导体,通过的电流强度越大.

在电路中并联一个电阻$R$,电流就多了一条通路,可以分去电路中的一部分电流(图7.12).在电路中电流强度超过某个元件所能允许的电流强度的情况下,给它并联上一个适当的电阻,就可以使通过元件的电流减小到允许的数值.并联电
阻的这种作用叫做分流作用,起分流作用的电阻叫做分流电阻.
\begin{figure}[htp]\centering
    \begin{circuitikz}[european,>=latex]
\draw(0,2)to [R] (6,2);  \draw [->](0,2)--(2,2)node [above]{$I$};
\draw(0,0)to [R] (6,0);  \draw [->](0,0)--(.5,0)node [above]{$I$};
\draw [->](1,0)--(1.75,0)node [above]{$I_1$};
\draw [->](1,-1)--(1.75,-1)node [above]{$I_2$};
\draw (1,0)--(1,-1) to [R=$R$] (5,-1)--(5,0);
\draw [fill=black](1,0) circle (2pt);
\draw [fill=black](5,0) circle (2pt);

    \end{circuitikz}

    \caption{并联电阻$R$的分流作用}
\end{figure}	

(3) \textbf{并联电路的功率分配} 并联电路中某个电阻$R_k$消耗的功率$P_k=UI_k$,而$I_k=U/R_k$,所以$P_k=U^2/R_k$.因此各个电阻消耗的功率分别是
\[ P_1=\frac{U^2}{R_1},\;  P_2=\frac{U^2}{R_2},\; \cdots  P_n=\frac{U^2}{R_n} \]
所以,
\[P_1R_1=P_2R_2=\cdots=P_nR_n=U^2 \]
这就是说,\textit{并联电路中各个电阻消耗的功率跟它的阻值成反比}.在并联电路中,阻值越大的电阻,消耗的功率越少.

整个并联电路消耗的总功率
\[P=UI=U (I_1+I_2+I_3+\cdots+I_n) \]
所以
\[P=P_1+P_2+P_3+\cdots+P_n\]
即消耗的总功率等于各支路上消耗的功率之和.

\subsection*{练习五}
\begin{enumerate}
    \item 电路中需要一个阻值为15千欧的电阻,现在手边只有几只10千欧的电阻,怎样才能组成一个15千欧的电阻?
    \begin{figure}[htp]\centering
        \begin{circuitikz}[european, >=latex]
            \draw (0,0) -- (5,0) to [R=$R_2$] (5,2.5) to [cute open switch] (5,4)-- (0,4);
            \draw (3,0) to [R=$R_1$, *-*] (3,4);
            \draw [fill=white](0,4) circle (2pt);
            \draw [fill=white](0,0) circle (2pt);
    \draw[<->](0,.1)--node [fill=white]{$U$} (0,4-.1);
            \node at (5.3, 3.5){$K$};
        \end{circuitikz}
    
        \caption{}
    \end{figure}	
    \item 在图7.13所示的电路中,$R_1=10$欧,$R_2=30$欧,$U=6$伏,电键$K$合上前后,
    \begin{enumerate}
        \item 电路中的总电阻各是多少?
        \item 通过$R_1$、$R_2$的电流强度各是多少?
        \item $R_1$、$R_2$上消耗的功率各是多少?
    \end{enumerate}
    \item 在图7.14所示的电路中,要使通过$R_1$的电流强度不超过5毫安,分流电阻$R_2$应为多大?
    \item 在图7.15所示的电路中,$L$是跟$R_2$并联的一条导线,下列说法哪些是正确的?
    \begin{enumerate}
        \item 通过$R_1$、$R_2$的电流强度$I$相等:
        \[I=\frac{U}{R_1+R_2} \]
        \item $R_1$上的电压$U_1=R_1I_1$,$R_2$上的电压$U_2=R_2I$,导线$L$中的电流为零.
        \item $R_1$上的电压$U_1=U$,$R_2$上的电压为零.
        \item $R_1$中的电流强度$I_1=U/R_1$,导线$L$中的电流强度等于$I_1$;$R_2$中的电流强度为零.
\item $R_2$去掉后,电路中的电阻和电流强度不发生变化.
    \end{enumerate}
\end{enumerate}

\begin{figure}[htp]
\centering
\begin{minipage}[t]{0.48\textwidth}
\centering
        \begin{circuitikz}[european, >=latex, scale=.8]
    \draw (0,0) -- (5,0) to [R=$R_1$] (5,4)--(0,4);
    \draw (3,0) to [R=$R_2$, *-*] (3,4);
    \draw [->] (.5,4)--(1.5,4)node [above]{3A};
    \node at (6.2, 2.4) {5mA};
    \node at (6.2, 1.6) {100$\Omega$};
    \draw [fill=white](0,4) circle (2pt);
    \draw [fill=white](0,0) circle (2pt);
        \end{circuitikz}
    
        \caption{}
        \end{minipage}
\begin{minipage}[t]{0.48\textwidth}
\centering
    \begin{circuitikz}[european, >=latex, scale=.8]
        \draw (0,0) -- (5,0) to [R=$R_2$] (5,4) to  [R=$R_1$]  (0,4);
\draw (5, 1)--(6,1)--(6,3)--(5,3);
\draw[<->](0,.1)--node [fill=white]{$U$} (0,4-.1);
        \node at (6.3, 2){$L$};
        \draw [fill=white](0,4) circle (2pt);
        \draw [fill=white](0,0) circle (2pt);
        \draw [fill=black](5,1) circle (2pt);
        \draw [fill=black](5,3) circle (2pt);
    \end{circuitikz}
    \caption{}
\end{minipage}
    \end{figure}	

\section{分压和分流在伏特表和安培表中的应用}
作为应用串联电阻进行分压和并联电阻进行分流的实例,我们来研究一下它们在伏特表和安培表中的应用.

常用的安培表和伏特表都是由电流表改装的,关于电流表的构造和工作原理我们将在以后学习,这里先简单说明一下.在电流表里有一个线圈,当线圈中有电流通过时,在磁场力的作用下线圈就带着指针一起偏转,通过线圈的电流越大,指针的偏角就越大,因此,根据指针的偏角就可以知道电流的大小,这样,如果在刻度盘上标出电流值就可以测定电流了.我们知道,通过电流表的电流跟加在电流表两端的电压成正比,因此,指针的偏角越大,表示加在电流表两端的电压越大,这样,如果在刻度盘上直接标出电压值,就可以测定电压了.

电流表的线圈是用很细的铜丝绕成的,允许通过的最大电流很小,一般不超过几十微安到几毫安,这个电流用$I_g$表示,线圈的电阻一般为几百到几千欧,叫做电流表的内阻,用$R_g$表示.每个电流表都有它的$R_g$值和$I_g$值.当通过它的电流为$I_g$时,它的指针偏转到最大刻度处,所以$I_g$叫满度电流.如果电流超过满度电流,不但指针指示不出数值,电流表还可能烧毁.

\subsection{伏特表}


电流表虽然能够用来测电压,但是由于电流表能够承担的电压$I_gR_g$很小,所以不能直接用电流表来测较大的电压.如果被测电压$U$大于$I_gR_g$,通过电流表的电流将超过$I_g$可能把电流表烧毁.如果给电流表串联一个电阻,分担一部分电压,就可以用来测较大的电压了.加了串联电阻并
在刻度盘上标出伏特值,就把电流表改装成了伏特表(图7.16).伏特表刻度盘上标出的伏特值,不是表示加在电流表上的电压,而是直接表示加在伏特表上的电压.
\begin{figure}[htp]\centering
    \begin{circuitikz}[european,>=latex]
        \draw (-4,0)--(-2,0) to [rmeter, t=G] (0,0) to [R=$R$] (3,0)--(4,0);
        \node at (-4,0) [below]{$+$};
        \node at (4,0) [below]{$-$};

        \draw [->] (-4,0)--(-3,0) node [below]{$I_g$};
\draw [dashed](-2,-1) rectangle (2.5,1);
        \node at (-.25,.3){$R_g$};

        \draw [fill=white](-4,0) circle (2pt);
        \draw [fill=white](4,0) circle (2pt);
    \end{circuitikz}

    \caption{}
\end{figure}	

让我们用一个具体例子来说明如何改装,假设有一个电流表,内阻$R_g$是1000欧,满度电流$I_g$是100微安,要把它改装成量程是3伏的伏特表,应该串联多大的电阻呢?

电流表指针偏转到满刻度时,它两端的电压$U_g=I_gR_g=0.1$伏,这是它能承担的最大电压,现在要让它测量最大为
3伏的电压,即指针偏转到满刻度时伏特表两端的电压为3伏,分压电阻$R$就必须分担2.9伏的电压.串联电路中电压跟电阻成正比,
\[\frac{U_g}{R_g}=\frac{U}{R}\]
由此可以求出
\[R=\frac{U_R}{U_g}R_g=\frac{2.9}{0.1}\x 1000\Omega=29{\rm k}\Omega\]
可见,串联29千欧的分压电阻后,就可以把这个电流表改装成量程为3伏的伏特表.

从上面的计算可以看出,把电流表改装成伏特表,需要给电流表串联一个阻值大的电阻.改装后的伏特表量程越大,需要分去的电压也越大,串联的分压电阻就要越大.

\subsection{安培表}


电流表能够测量的电流不超过毫安级.为了测量几个安培甚至更大的电流,可以给它并联一个分流电阻,分掉一部分电流.这样,在测量大电流时通过电流表的电流也不致超过满度电流$I_g$.并联了分流电阻并在刻度盘上标出安培值,电流表就改装成了安培表(图7.17).安培表刻度盘上标出的安培值,不是表示通过电流表的电流,而是直接表示通过安培表的电流.
\begin{figure}[htp]\centering
    \begin{circuitikz}[european,>=latex]
\draw (-4,0)--(-2,0) to [R=$R$] (2,0)--(4,0);
\draw (-2,0)--(-2,1.5) to [rmeter, t=G] (2,1.5)--(2,0);

\node at (1,1.2){$R_g$};
\draw [->] (-4,0)--(-3,0) node [below]{$I$};
\draw [->] (-2,0)--(-1,0) node [below]{$I_R$};
\draw [->] (-2,0)--(-2,1) node [left]{$I_g$};

\draw [fill=white](-4,0) circle (2pt);
\draw [fill=white](4,0) circle (2pt);
\draw [fill=black](-2,0) circle (2pt);
\draw [fill=black](2,0) circle (2pt);

\node at (-4,0) [below]{$+$};
\node at (4,0) [below]{$-$};

\draw [dashed](-2.75, -.75) rectangle (2.75, 2.25);

    \end{circuitikz}

    \caption{}
\end{figure}	

例如,电阻$R_g$是1000欧、满度电流$I_g$是100微安的电流表,要改装成量程为1安的安培表,我们很容易计算出应该并联多大的分流电阻.

电流表允许通过的最大电流是100微安$=0.0001$安,在测量1安的电流时,分流电阻R上通过的电流应该是$I_R=0.9999$安,由于并联电路中电流强度跟电阻成反比,$I_gR_g=I_R R$.所以,
\[R=\frac{I_g}{I_R}R_g=\frac{0.0001}{0.9999}\x 1000\Omega=0.1\Omega\]
可见,并联0.1欧的分流电阻后,就可以把这个电流表改装成量程为1安的安培表,把电流表改装成安培表,需要给它并联一个阻值小的电阻,改装后的安培表量程越大,需要分去的电流也越大,并联的分流电阻就要越小.


\subsection*{练习六}

\begin{enumerate}
    \item 有一个电流表,内阻为100欧,满度电流为3毫安,要把它改装成量程为3安的安培表,需并联多大的分流电阻?要把它改装为6伏的伏特表,需串联多大的电阻?
    \item 电流表的内阻为$R_g$,满度电流为$I_g$.试证明:
     \begin{enumerate}
         \item 要把它改装成量程为$U=nI_gR_g$的伏特表,串联电阻的阻值应为$R_{\text{串}}=(n-1)R_g$;
         \item 要把它改装成量程为$I=nI_g$的安培表,并联电阻的阻值应为$R_{\text{并}}=R_g/(n-1)$.
     \end{enumerate}
     \item 在第(1)题中,改装后的安培表的量程比电流表原来的电流量程扩大的倍数$n$是多少?改装后的伏特表比电流表原来的电压量程扩大的倍数$n$又是多少?利用第(2)题中推出的两个公式,重新计算第(1)题中的并联电阻和串联电阻,比较两次计算的结果是否相同.
     \item 有一安培表,内阻为0.03欧,量程为3安,测量电阻$R$中的电流强度时,本应与$R$串联,如果不注意,错把安培表与$R$并联了(图7.18),将会产生什么后果?假设$R$两端的电压为3伏.
\end{enumerate}
\begin{figure}[htp]\centering
    \begin{circuitikz}[european]
\draw (0,0)node[left]{$-$}--(4,0) to [R=$R$] (4,4)--(0,4) node[left]{$+$};
\draw  (2.5,0) to [rmeter, t=A, *-*] (2.5, 4);
\draw [fill=white](0,0) circle (2pt);
\draw [fill=white](0,4) circle (2pt);
    \end{circuitikz}

    \caption{}
\end{figure}	

\section{电路的分析和计算}

实际应用的电路大多是既包含并联电路又包含中联电路的混联电路,运用前面讲过的串联电路和并联电路的知识就可以对混联电路进行分析和计算,下面举两个例子来说明混联电路的实际意义和它的分析计算方法.

\begin{example}
    线路电压为220伏,每根输电导线的电阻$r=1$欧,电路中并联了100盏“220V,40W”的电灯.求:
    \begin{enumerate}
        \item 只打开其中的10盏灯时每盏灯的电压和功率;
        \item 100盏灯全部打开时每盏灯的电压和功率.
    \end{enumerate}
\end{example}

\begin{solution}
    根据题意,100盏电灯是并联的,电灯与输电导线足串联的,电路如图7.19所示,其中$r$是每根输电导线的电阻.从图上可以看出,电灯的电压等于线路电压减去输电导线上的电压.求出并联电灯的电阻$R_{\text{并}}$、电路的总电阻$R_{\text{总}}$,算出电路中的总电流强度,就可以求出输电导线上的电压,从而可以求得电灯的电压和功率.
    \begin{figure}[htp]\centering
        \begin{circuitikz}[european,>=latex]
    \draw (0,0) to [R=$r$] (3,0)--(8,0) to [lamp] (8,3)--(3,3) to [R=$r$] (0,3);
    
    \foreach \x in {5, 3.5}
    {
        \draw (\x,0) to [lamp, *-*] (\x, 3);
    }
    \draw [dashed] (6.5,0) to [lamp, *-*] (6.5, 3);
    
    
    \draw [fill=white](0,0) circle (2pt);
    \draw [fill=white](0,3) circle (2pt);
    \draw [<->](0,0+.1)--node [fill=white] {220V} (0,3-.1);
        \end{circuitikz}
    
        \caption{}
    \end{figure}	

(1) 只打开10盏电灯的时候:

每盏电灯的电阻
\[R=\frac{U^2}{P}=\frac{220^2}{40}\Omega=1210\Omega \]
10盏电灯并联的电阻
\[R_{\text{并}}=\frac{R}{10}=121\Omega\]
电路中的总电阻
\[R_{\text{总}}=R_{\text{并}}+2r=121+2=123\Omega \]
电路中的总电流强度
\[I=\frac{U}{R_{\text{总}}}=\frac{220}{123} {\rm A}=1.8{\rm A}\]
两根输电导线上的电压
\[2U_r=2Ir =2\x1.8\x 1=3.6{\rm V}\]
电灯的电压
\[U_L=U-2U_r=220-3.6=216.4{\rm V}\]
每盏电灯的功率
\[P=\frac{U^2_L}{R}=\frac{216^2}{1210}=39{\rm W}\]

(2)100盏电灯全部打开的时候:100盏电灯并联的电阻
\[R_{\text{并}}=\frac{R}{100}=12.1\Omega\]
电路中的总电阻
\[R_{\text{总}}=R_{\text{并}}+2r=12.1+2=14.1\Omega \]
电路中的总电流强度
\[I=\frac{U}{R_{\text{总}}}=\frac{220}{14.1} {\rm A}=16{\rm A}\]
两根输电导线上的电压
\[2U_r=2Ir =2\x16\x 1=32{\rm V}\]
电灯的电压
\[U_L=U-2U_r=220-32=188{\rm V}\]
每盏电灯的功率
\[P=\frac{U^2_L}{R}=\frac{188^2}{1210}=29{\rm W}\]
\end{solution}

从这道例题可以看出,100盏电灯全部打开时比只打开10盏时加在电灯上的电压减小了,每盏灯上消耗的功率也减小了.一般说来,电路里并联的用电器越多,并联部分的电阻就越小,在总电压不变的条件下,电路里的总电流就越大,因此输电线上的电压就越大.这样,加在用电器上的电压就越小,每个用电器消耗的功率也越小.我们在晚上七、八点钟开灯,那时大家都用电灯照明,电灯比深夜时暗些,就是这个缘故.

\begin{example}
图7.20甲是一个分压器电路,由两个10千欧的电阻串联而成.线路两端电压为50伏时,$a$、$b$间的电压应为25伏.用内阻为25千欧的伏特表$V_1$测量$a$、$b$间的电压时,伏特表的读数不到21伏;改用内阻为500千欧的伏特表$V_2$测量时,伏特表的读数接近25伏.试分析其原因.
\end{example}
\begin{figure}[htp]\centering
    \begin{circuitikz}[european, >=latex]
\draw (0,0)--(3,0) node [right]{$b$};
\draw (0,5)--(1.5,5) to [R=$10{\rm k}\Omega$] (1.5,2.5) to [R=$10{\rm k}\Omega$, *-*] (1.5, 0);
\draw (1.5,2.5)--(3,2.5) node [right]{$a$};
\draw [<->](0,0+.1)--node [fill=white] {50V} (0,5-.1);


\draw [fill=white](0,0) circle (2pt);
\draw [fill=white](0,5) circle (2pt);
\draw [fill=white](3,0) circle (2pt);
\draw [fill=white](3,2.5) circle (2pt);
\node at (1.5,-1){甲};

    \end{circuitikz}
    \begin{circuitikz}[european, >=latex]
        \draw (0,0)--(3,0) node [right]{$b$};
        \draw (0,5)--(1.5,5) to [R=$10{\rm k}\Omega$] (1.5,2.5) to [R=$10{\rm k}\Omega$, *-*] (1.5, 0);
        \draw (1.5,2.5)--(3,2.5) node [right]{$a$};
        \draw [<->](0,0+.1)--node [fill=white] {50V} (0,5-.1);
        \draw (3,0) to [rmeter, t=$V_1$] (3,2.5);
        
        \draw [fill=white](0,0) circle (2pt);
        \draw [fill=white](0,5) circle (2pt);

        \node at (1.5,-1){乙};
        
            \end{circuitikz}
            \begin{circuitikz}[european, >=latex]
                \draw (0,0)--(3,0) node [right]{$b$};
                \draw (0,5)--(1.5,5) to [R=$10{\rm k}\Omega$] (1.5,2.5) to [R=$10{\rm k}\Omega$, *-*] (1.5, 0);
                \draw (1.5,2.5)--(3,2.5) node [right]{$a$};
                \draw [<->](0,0+.1)--node [fill=white] {50V} (0,5-.1);
                \draw (3,0) to [rmeter, t=$V_2$] (3,2.5);

                
                \draw [fill=white](0,0) circle (2pt);
                \draw [fill=white](0,5) circle (2pt);

                \node at (1.5,-1){丙};
                
                    \end{circuitikz}
    \caption{}
\end{figure}	

\begin{solution}
    伏特表未接入电路时,$a$、$b$间的电压$U_{ab}=25$伏(图7.20甲),伏特表接入电路后,伏特表与下面的10千欧电阻形成并联(图7.20乙和丙),整个电路变为混联电路.由于$a$、$b$间的电阻发生变化,所以电压也发生变化.这就是为什么伏特表测得的结果不是25伏的原因.

(1) 用内阻为25千欧的伏特表$V_1$测量时(图7.20乙),$a$、$b$间的并联电阻
\[R_{\text{并}}=\frac{25\x 10}{25+10}=7.1{\rm k}\Omega\]
电路中的总电阻
\[R_{\text{总}}=10+7.1=17.1{\rm k}\Omega \]
电路中的电流强度
\[I=\frac{U}{R_{\text{总}}}=\frac{50}{17.1\x 10^4} {\rm A}=2.9\x 10^{-3}{\rm A}\]
$a$、$b$间的电压,即伏特表$V_1$测得的电压
\[U_{ab}=IR_{\text{并}}=2.9\x10^{-3}\x7.1\x10^3=20.6{\rm V}\]

(2)用内阻为500千欧的伏特表$V_2$测量时(图7.20丙),$a$、$b$间的并联电阻
\[R_{\text{并}}=\frac{500\x 10}{500+10}=9.8{\rm k}\Omega\]
电路中的总电阻
\[R_{\text{总}}=10+9.8=19.8{\rm k}\Omega \]
电路中的电流强度
\[I=\frac{U}{R_{\text{总}}}=\frac{50}{19.8\x 10^3} {\rm A}=2.5\x 10^{-3}{\rm A}\]
$a$、$b$间的电压,即伏特表$V_2$测得的电压
\[U_{ab}=IR_{\text{并}}=2.5\x10^{-3}\x 9.8\x10^3=24.5{\rm V}\]
\end{solution}

从这道例题可以看出,伏特表接入电路时,由于其内阻与被测电路并联引起被测电路两端的电压变化,因而使测量产生误差.伏特表的内阻比被测电路的电阻大得越多,对被测电路的影响就越小,测量的误差也越小,因此,用内阻较大的伏特表进行测量比较准确.特别是在测量电阻较大的电路时,一定要选用内阻大的伏特表,否则将会产生较大的误差.

从上面两道例题可以看出,分析、计算电路时,首先要弄清电路各部分间的串、并联关系,然后再运用串、并联电路的电流、电压、电阻和功率关系进行计算.


\subsection*{练习七}
\begin{enumerate}
    \item 图7.21是某电路中的一部分,$R_2$、$R_3$、$R_4$是三个电子管的灯丝电阻,阻值分别为$R_2=R_3=84$欧,$R_4=21$欧,分压电阻$R_1=31$欧,$AB$间的电压是28伏,求通过各电子管灯丝的电流强度.
    \begin{figure}[htp]\centering
        \begin{circuitikz}[european, thick]
    \draw (-2,0) node[above]{$A$} to [R=$R_1$] (2,0) to [bulb=$R_2$] (4,0) to [bulb=$R_4$] (7,0) -- (8,0) node[above]{$B$};
    
    \draw (1.5,0)--(1.5,-2) to [bulb=$R_3$] (4.5,-2)--(4.5,0);
    \draw [fill=white](-2,0) circle (2pt);
    \draw [fill=white](8,0) circle (2pt);
    \draw [fill=black](1.5,0) circle (2pt);
    \draw [fill=black](4.5,0) circle (2pt);
    \node at (-2-.3,0){$+$};
    \node at (8+.3,0){$-$};
        \end{circuitikz}
        \caption{}
    \end{figure}	

    \item 图7.22中,电源电压为220伏,各段输电导线上的电阻$r$都为2欧,用电器$R_1$的电阻为200欧,$R_2$的电阻为196欧,$R_1$、$R_2$两端的电压各是多少?
    \begin{figure}[htp]\centering
        \begin{circuitikz}[european, >=latex]
    \draw (0,0)--node [below]{$r$} (3,0)--node[below]{$r$}(5,0) to [R=$R_2$] (5,3)
    --node[above]{$r$}(3,3) --node[above]{$r$}(0,3);
            \draw (3,0) to [R=$R_1$, *-*] (3,3);
    
    
    \draw [<->](0,0.1)--node [fill=white]{220V} (0,3-.1);
    \draw [fill=white](0,0) circle (2pt);
    \draw [fill=white](0,3) circle (2pt);
    
        \end{circuitikz}
    
        \caption{}
    \end{figure}

    \item 一个600欧的电阻和一个400欧的电阻串联后接在电压为90伏的电源上,用伏特表测得600欧电阻上的电压为45伏.
    \begin{enumerate}
        \item 伏特表接入电路后,600欧电阻上的电压改变了多少?
        \item 这个伏特表的内阻是多少?
    \end{enumerate}
    \item 在图7.23中滑动变阻器作分压器使用,负载电阻$R$一端接在变阻器的固定端$A$上,另一端接在滑动端$P$上,滑动端$P$在$AB$间移动时,$R$上就能得到不同的电压.
    \begin{figure}[htp]\centering
        \begin{circuitikz}[european,>=latex]
    
     \draw (0,0) to [R] (5,0) to [cute open switch] (6,0)--(7,0) ;      
     \draw (0,0)--(0,-2);
     \draw (7,0)--(7,-2);
    
     \draw [fill=white](0,-2) circle (2pt);
     \draw [fill=white](7,-2) circle (2pt);
     \draw [fill=black](0,0) circle (2pt);
    
    \draw (0,0) -- (0,1) to [R=$R$] (2.5,1); 
    \draw [->](2.5,1)--node [right]{$P$} (2.5,.2);
    \node at (2-.1,-.4){$A$};
    \node at (3+.1,-.4){$B$};
    \node at (5.5,-.4){$K$};
    
    
    
    \draw [<->](0+.1,-2)--node [fill=white]{$U$} (7-.1,-2);
    
    
       \end{circuitikz}
    
        \caption{}
    \end{figure}

    \begin{enumerate}
        \item 当滑动端从$A$向$B$移动时,$R$上的电压怎样变化?
        \item 如果电压$U=6$伏,变阻器的电阻$R_{AB}=50$欧,负载电阻$R=100$欧,当滑动端$P$在$A$点、$R$点时,$R$上的电压各是多少?
        \item 当$P$在$AB$中点时,$R$上的电压是否为3伏?通过变阻器各部分的电流是否相等?
    \end{enumerate}
\end{enumerate}

\section{电动势~~闭合电路的欧姆定律}

要使电路中有电流,必须把电路与电源相连,组成闭合电
路.闭合电路中的电流强度跟什么有关系呢?为了研究这个问题,需要先介绍一个表征电源特性的物理量——电动势.

\subsection{电动势}
\begin{figure}[htp]\centering
    \includegraphics[scale=.6]{fig/7-24.png}
    \caption{}
    \end{figure}

电源有两个极,正极的电势比负极的高,两极间有一定的电压.电源的作用就是保持两极间有一定的电压.把伏特表接在于电池的正、负极上(图7.24),可以测得干电池两极间的电压是1.5伏.用伏特表对不同型号的干电池进行测量,结果表明,它们两极间的电压都是1.5伏.用伏特表测量蓄电池两极间的电压,结果表明,只要蓄电池是同种的,它们两极间的电压都相同,例如常用的铅蓄电池两极间的电压是2伏.可见,同种电源两极间的电压相同,不同种类的电源,一般说来,两极间的电压不相同.为了表征电源的这种特性,物理学中引入了电动势这个物理量.电源的电动势在数
值上等于电源没有接入外电路时两极间的电压,上面把伏特表直接接在电源的两极上测出的就是电源的电动势.

把电源跟外电路接通后,再用伏特表测量两极间的电压
(图7.25),可以看到,这时伏特表的读数小于电源的电动势.这是什么原因呢?原来,在电源跟外电路接通后,不仅外电路中有电流通过,电源内部的路中也有电流通过.内电路也有电阻,所以电流通过内电路时也产生电势降落.电源的电动势是一定的,要不是内电路中也有电势降落,接通电路后外电路两端的电压应该等于电源的电动势.考虑到内电路上的电势降落,我们可以设想内、外电路上的电势降落之和应该等于电源的电动势.
\begin{figure}[htp]\centering
\begin{minipage}[t]{0.48\textwidth}
\centering
    \begin{circuitikz}[european]
\draw (0,4) to [battery2] (0,0)--(3,0) to [cute open switch](3,2) to[R=$R$] (3,4) -- (0,4);
 \draw (1.5,0) to [rmeter, t=V, *-*] (1.5, 4);       
    \end{circuitikz}

    \caption{}
\end{minipage}
\begin{minipage}[t]{0.48\textwidth}\centering
    \includegraphics[scale=.6]{fig/7-26.png}
    \caption{内、外电路上的电压之和等于电动势}
    \end{minipage}
    \end{figure}

实际情况是否如此呢?这需要用实验来验证,在图7.26中,$E$是实验用的电池,滑动变阻器作外电路;伏特表$V$连接在电池的两极上,用来测量外电路上的电压$U$;伏特表$V'$连接在插在电极附近的探针$A$和$B$上,用来测量内电路上的电压$U'$.先断开外电路,用伏特表$V$测出电源的电动势$\mathcal{E}$.然后接通外电路,测量$U$和$U'$.移动变阻器的滑动头,改变外电路的电阻,从而改变电路中的电流强度,可以看到,内、外电路上的电压$U$和$U'$也随着变化,但是$U$和$U'$之和却保持不变.总等于电源的电动势$\mathcal{E}$,即
\[\mathcal{E}=U+U'\]
这表明,在外电路接通时,电源的电动势等于内、外电路上的电压之和.

利用这个结果,我们可以从电路中能量转化的角度来理
解电动势的物理意义.在公式$\mathcal{E}=U+U'$中,右端$U+U'$在数值上等于1库仑电量通过外电路和内电路时消耗的总电能.电路中消耗的电能是由电源提供的,公式左端的$\mathcal{E}$在数值上等于电路中通过1库仑电量时电源提供的能量.因此,\textit{电动势反映了电源的一种特性,它在数值上等于电路中通过1库仑电量时电源所提供的电能}.电源提供的电能是由其形式的能转化来的.例如,在化学电池中电能是由化学能转化来的,在发电机中电能是由机械能转化来的.从本质上来说,各种电源都是把其他形式的能转化为电能的装置.电动势越大,表明电源把其他形式的能转化为电能的本领越大.

\subsection{闭合电路的欧姆定律}

利用前面得出的公式$\mathcal{E}=U+U'$和欧姆定律,就可以求出闭合电路中的电流强度跟什么有关
系.如果用$R$表示外电路的电阻,$r$表示内电路的电阻,$I$表示电路中的电流强度,那么,根据欧姆定律可以写出:$U=IR$,$U'=Ir$,代入$\mathcal{E}=U+U'$中,可得
\[\mathcal{E}=IR+Ir \]
由此可得
\[I=\frac{\mathcal{E}}{R+r} \]

这就是\textbf{闭合电路的欧姆定律}.它表示:\textbf{闭合电路中的电流强度跟电源的电动势成正比,跟内、外电路中的电阻之和成反比}.

这是关于电路的一条重要定律,以后经常要用到.

\begin{example}
在图7.27中,$R_1=14.0$欧,$R_2=9.0$欧,当单刀双掷开关$K$扳到位置1时,测得电流强度$I_1=0.20$安;当$K$扳到位置2时,测得电流强度$I_2=0.30$安.求电源的电动势和内电阻.
\end{example}
\begin{figure}[htp]\centering
    \begin{circuitikz}[european]
\draw(0,0) to [battery2] (5,0) to [rmeter, t=A, -*](5,2) -- (5,3);
\node at (2.5,-.5)[below]{$\mathcal{E}\;  r$};
\draw (5,2) to  [R=$R_2$] (1.5,2);
\draw (5,3) to  [R=$R_1$] (1.5,3);
\draw [fill=white](1.5,2) circle (2pt) node [above]{2};
\draw [fill=white](1.5,3) circle (2pt)node [above]{1};
\draw [very thick](.9,2.5)--(1.5,2.9);
\draw (0,0)--(0,2.5)--(.9,2.5)node [below]{$K$};
\draw [fill=white](.9,2.5) circle (2pt);
    \end{circuitikz}
    \caption{}
\end{figure}	

\begin{solution}
    根据闭合电路的欧姆定律,可列出方程组:
    \[\begin{split}
        \mathcal{E}&=I_1R_1+I_1r\\
        \mathcal{E}&=I_2R_2+I_2r\\
    \end{split}\]
    消去$\mathcal{E}$,可得
    \[I_1 R_1+I_1r=I_2R_2+I_2r\]
    所以,电源的内电阻
\[\begin{split}
    r&=\frac{I_1R_1-I_2R_2}{I_2-I_1}\\
    &=\frac{0.20\x 14.0-0.30\x 9.0}{0.30-0.20}=1.0\Omega
\end{split}\]
    把$r$值代入$\mathcal{E}=I_1R_1+I_1r$中,可得电源的电动势
    \[\mathcal{E}=0.20\x 14.0+0.20\x 1.0=3.0{\rm V} \]
\end{solution}

这道例题介绍了一种测量电源的电动势和内电阻的方法.


\section*{阅读材料:欧姆定律的建立}

欧姆是德国物理学家,当过多年的中学数学和物理教师,对研究工作很有雄心抱负.他在缺少时间和书籍以及适当的仪器的情况下,自己制作了许多仪器,独自坚持研究工作.经过多年努力,终于建立了欧姆定律.

在欧姆进行研究时,科学上还没有建立起电动势、电流强度、电阻等明确的概念,更没有准确测量这些量的仪器.欧姆进行实验时所用的电源是温差电偶,他用验电器测量电源两端的电势差.欧姆用许多粗细相同、长度不同的铜导线作为电阻进行实验,他根据电流使悬挂的磁针偏转的角度来测量电流的强弱.1826年欧姆根据测得的数据得出了下面的公式
\[\chi=\frac{a}{b+x}\]

其中$\chi$代表电流磁效应的强弱,相当于电流强度;$x$代表铜导线的长度,相当于电阻;$a$代表电源的“激活力”,也就是电动势;$b$由电路其他部分决定,如果不考虑连接导线的影响,则相当于电源的内电阻.上式相当于课文中讲的全电路欧姆定律的公式.

在实验研究的基础上,欧姆把电流跟热流、水流等现象进行对比,从中得到启发,认为电流中的电势差起着跟热流中的
温度差、水流中的高度差相似的作用.通过对比,他引入了电流强度、电动势、电阻等概念,并确定了它们之间的关系.


\subsection*{练习八}
\begin{enumerate}
    \item 电源的电动势为1.5伏,内电阻为0.12欧,外电路的电阻为1.28欧,求电路宁的电流强度.
    \item 把一个定值电阻和电源连成图7.28所示的电路,可以测得电源的内电阻,定值电阻$R$为10欧,合上开关$K$时,伏特表的读数为5.46伏,打开$K$时,伏特表的读数为6.0伏.求电源的内电阻为多少.
    \item 在图7.29所示的电路中$\mathcal{E}=9.0$伏,$r=3.0$欧,$R=15$欧,当$K$闭合时,$U_{AB}$是多少?当$K$打开时,$U_{AB}$又为多少?
    \item 利用图7.30所示的电路可以测出电源的电动势和内电阻.当变阻器的滑动端在某一位置时,安培表和伏特表的读数分别是0.20安和1.98伏,改变滑动端的位置后,两表的读数分别是0.40安和1.96伏,求电池的电动势和内电阻.

\end{enumerate}

    \begin{figure}[htp]\centering
    \begin{minipage}[t]{0.48\textwidth}
\centering
                \begin{circuitikz}[european]
    \draw(0,0)--(5,0) to [R=$R$](5,3) to  [cute open switch] (2,3)--(0,3) to[battery2] (0,0);
    \draw (2,0) to [rmeter, t=V, *-*] (2,3);    
    \node at (7/2,3)[above] {$K$}    ;
        \end{circuitikz}
    
        \caption{}
    \end{minipage}
\begin{minipage}[t]{0.48\textwidth}
\centering
        \begin{circuitikz}[european, xscale=.8]
    \draw(0,0) to [battery2] (6,0) node [below]{$B$} to[cute open switch] (6,2) to[R=$R$] (0,2)node [above]{$A$}--(0,0) ;
    
    \node at (6,1)[right]{$K$};
    
    \node at (3,-.5)[below]{$\mathcal{E}\;  r$};
        \end{circuitikz}
    
        \caption{}\end{minipage}
    \end{figure}    


        \begin{figure}[htp]\centering
    \begin{circuitikz}[european,>=latex, yscale=.8]
    \draw (0,0)to [battery2] (6,0)--(6,2)to [rmeter, t=V, *-*] (0,2) to [rmeter, t=A] (0,0);
    \draw (0,2)--(0,4)--(2,4) to [R] (4,4)--(4.5,4)to [cute open switch] (6,4)--(6,2);
            \draw (1.5,4) to (1.5,4.7)--(3,4.7);
            \draw [->](3,4.7)--(3,4.2);
        \end{circuitikz}
        \caption{}
    \end{figure}    

\section{路端电压}
从图7.26所示的实验中可以看出,外电路中的电阻发生变化时,内、外电路上的电压$U'$和$U$都随着变化.因为用电器都是接在外电路中的,电源的“有效”电压是外电路上的电压,因此,研究外电路上的电压的变化规律是很重要的,通常把外电路两端的电压叫做\textbf{路端电压}.下面我们用闭合电路的欧姆定律来分析路端电压的变化规律.

公式$\mathcal{E}=U+U'$可以改写作$U=\mathcal{E}-U'$,这表示路端电压
等于电源的电动势减去电源内部的电压.因为$U'=Ir$,代入式中可得
\[U=\mathcal{E}-Ir\]

就某个电源来说,电动势$\mathcal{E}$和内电阻$r$都是一定的,从上式可以看出路端电压$U$跟电路中的电流强度有关系.电流强度$I$增大时,电源内部的电压$Ir$增大,路端电压就减小;电流强度减小时,电源内部的电压$Ir$减小,路端电压就增大.路端电压所以随电流而变化,根本原因是电源有内电阻,如果没有内电阻,不论电流怎样变化,路端电压也不会变化,总等于电源的电动势.

\begin{figure}[htp]\centering
    \begin{tikzpicture}[>=latex, scale=1.2]
\draw [<->](0,2.5)node[above]{$U$}--(0,0)node[below]{$O$}--(3.5,0)node[above]{$I$};
    \draw [dashed] (0,2)node[left]{$\mathcal{E}$}--(3,2); 
    \draw (0,2)--(3,1.5);   

    \end{tikzpicture}
\caption{$U$-$I$关系曲线}
\end{figure}

路端电压跟电流强度的关系常用图线来表示,用横轴表示电流强度$I$,纵轴表示路端电压$U$,利用上面的关系式可以画出$U$-$I$关系曲线(图7.31).从图上可以看出,此关系曲线
是一条向下倾斜的直线.当$I=0$时,$U=\mathcal{E}$;随着$I$的增大,$U$逐渐减小,直线倾斜的程度跟内电阻有关系,内电阻越大,倾斜得越厉害;内电阻越小,这条直线越平;内电阻趋于零时,这条直线趋近于跟横轴平行.


根据闭合电路的欧姆定律,电流强度$I=\mathcal{E}/(R+r)$.$\mathcal{E}$和$r$都是不变的,电流强度$I$是随外电阻$R$而变化的,因此,路端电压也随外电阻$R$而变化.$R$增大时,$I$减小,路端电压增大;$R$减小时,$I$增大,路端电压减小.

下面来讨论两种特殊情况.

(1)当外电路断开,即断路时,$R$变成无限大,$I$变为零,$Ir$也变为零,$U$等于$\mathcal{E}$,即断路时的路端电压等于电源的电动势.

上一节我们讲过,用伏特表测出的断路时的路端电压就
是电源的电动势.可是这时伏特表本身成了外电路,因此测出的路端电压并不准确地等于电动势,不过由于伏特表的电阻很大,电路中的电流很小,
$Ir$也很小,因此$U$和$\mathcal{E}$相差很小.只要不要求特别准确,用这个办法测电动势很方便.

(2)当$R$趋近于零,即短路时,路端电压$U$也趋近于零,这时电流强度就趋近于$\mathcal{E}/r$.

发生短路时,电流强度取决于电源的电动势和内电阻.电源的内电阻一般都很小,例如铅蓄电池的内电阻只有0.005—0.1欧姆,所以短路时电流很大.这时电源提供的全部能量都消耗在内电路上,短时间内将产生很大的热量,会烧毁电源.因此在实际中要切实注意防止短路.

\subsection*{练习九}

\begin{enumerate}
    \item 试分析说明:外电路中的电阻发生变化时为什么会影响路端电压的变化.
    \item 在两个电路中,电源的电动势相同,但内电阻不同,当它们的外电路中流过的电流相同时,哪个电路的路端电压大?
    \item 在图7.32所示的电路中,当$P$由左向右滑动时,安培表和伏特表的读数怎样变化?$P$的位置在何处,伏特表的
指示更接近于电源的电动势?

\begin{figure}[htp]\centering
    \begin{circuitikz}[>=latex, european]
\draw (0,0) to [battery2] (6,0)--(6,1) to [rmeter, t=V, *-*] (0,1)--(0,0);
\draw (0,1)--(0,2) to [rmeter, t=A] (2,2) to[R] (6,2)--(6,1);
\draw (6,2)--(6,3)--(4,3);
\draw [->](4,3)--node [left]{$P$}(4,2.2);
\fill [white] (6-.02,2.2) rectangle (4.58,1.8);
    \end{circuitikz}
    \caption{}
\end{figure}

\item 发电机的电动势为240伏,内电阻为0.40欧,给200盏电阻均为1210欧的电灯供电,电灯上的电压是多大?如果
再接入100盏同样的电灯,电灯上的电压又是多大?利用所得的结果说明:电路中的用电器增多时,加在用电器上的电压将
怎样变化?
\end{enumerate}


\section{电池组}
我们知道,用电器要在额定电压和额定电流下才能正常工作.任何一个电池都有一定的电动势和允许通过的最大电流.如果用电器的额定电压低于电池的电动势,额定电流也小于电池允许通过的最大电流,我们可以用单个电池来给电路供电.实际上,用电器的额定电压常常高于电池的电动势,
额定电流也常常大于电池允许通过的最大电流,在这种情况下,需要把几个电池连成电池组,以便提高供电的电压或者增大输出的电流.晶体管收音机的直流电源,火车上照明用的电源,汽车发动机启动和照明用的电源,都是用电池组.电池组一般都是用相同的电池组成的.

\subsection{串联电池组}

把第一个电池的负极和第二个电池的正极
相连接,再把第二个电池的负极和第三个电池的正极相连接,象这样依次连接起来,就组成了串联电池组(图7.33).第一个电池的正极就是电池组的正极,最后一个电池的负极就是电池组的负极.
\begin{figure}[htp]\centering
    \begin{circuitikz}[european]
\draw (1,0)node [above]{$+$}--(0,0) to [battery2] (-1,0) to [battery2] (-2,0)to [battery2] (-3,0)to [battery2] (-4,0)--(-5,0) node [above]{$-$};
    \end{circuitikz}
    \caption{}
\end{figure}	

设串联电池组是由$n$个电动势都是$\mathcal{E}$、内电阻都是$r$的电池组成.由于开路时路端电压等于电源的电动势,每一个电池正极的电势比它负极的电势高$\mathcal{E}$,而前一个电池的负极和后一个电池的正极电势相同.因此,串联电池组正极的电势比
它负极的电势高$n\mathcal{E}$,整个电池组的电动势
\[\mathcal{E}_{\text{串}}=n\mathcal{E}\]
电池是串联的,电池的内电阻也是串联的,中联电池组的内电阻
\[r_{\text{串}}=nr\]

\textit{串联电池组的电动势等于各个电池电动势之和,串联电池组的内电阻等于各个电池内电阻之和.}

串联电池组的电动势比单个电池的高,当用电器的额定
电压高于单个电池的电动势时,可以用串联电池组供电.这时全部电流要通过每个电池,用电器的额定电流必须小于单
个电池允许通过的最大电流.

把几个电池组成串联电池组时,注意不要把某些电池接反.例如,用两个1.5伏特电池组成串联电池组,如果连接正确,可以得到3伏特的电动势,使小灯泡发光(图7.34甲);如果接反了,则电池组的电动势为零,小灯泡不发光(图7.34乙).

\begin{figure}[htp]
\centering
        \begin{tikzpicture}[>=stealth, thick]
    \draw(-1.5,-2)to[battery2] (1.5-1.5,0-2)--(2-1.5,0-2)to[battery2] (3-1.5,0-2);
            \draw(0-1.5,0-2)--(0-1.5,2-2)to [lamp] (3-1.5,2-2)--(3-1.5,0-2);

\foreach \x in {30,60,...,360}
{
    \draw (\x:.5)--(\x:.75);
}

        \end{tikzpicture}\qquad
        \begin{tikzpicture}[>=stealth, thick]
            \draw(0,0)to[battery2] (1.7,0)--(2,0);
            \draw (3,0)to[battery2] (2,0);
            \draw(0,0)--(0,2)to [lamp] (3,2)--(3,0);            
        \end{tikzpicture}
\caption{}
\end{figure}\qquad\qquad
\begin{figure}[htp]
\centering
\begin{circuitikz}[european, scale=1.2, >=stealth]
\draw(0,0)node[left]{$+$}--(1-.2,0)--(1-.2,1.5) to [battery2] (2+.2,1.5)--(2+.2,0)--(3,0)node[right]{$-$};
\draw (1-.2,0)--(1-.2,0.5) to [battery2,*-*] (2+.2,.5)--(2+.2,0);
\draw (1-.2,0)--(1-.2,-0.5) to [battery2,*-*] (2+.2,-.5)--(2+.2,0);
\draw (1-.2,0)--(1-.2,-1.5) to [battery2] (2+.2,-1.5)--(2+.2,0);
\draw (1-.2,0) [fill=black] circle (1.2pt);
\draw (2+.2,0) [fill=black] circle (1.2pt);
    \end{circuitikz}    
\caption{}
\end{figure}


\subsection{并联电池组}


把电动势相同的电池,正极和正极相连接,负极和负极相连接,就组成并联电池组(图7.35).连在一起的正极是电池组的正极,连在一起的负极是电池组的负极.

设并联电池组是由$n$个电动势都是$\mathcal{E}$、内电阻都是$r$的电池组成的.用导线连接起来的所有极板的电势都相等,所以并联电池组正负极间的电势差等于每个电池正负极间的电势差.而开路时正负极间的电势差等于电动势,所以并联电池组的电动势
\[\mathcal{E}_{\text{并}}=\mathcal{E} \]
电池是并联的,电池的内电阻也是并联的,并联电池组的内电阻
\[r_{\text{并}}=\frac{r}{n} \]

\textit{由$n$个电动势和内电阻都相同的电池连成的并联电池组,它的电动势等于一个电池的电动势,它的内电阻等于一个电池的内电阻的$n$分之一}.

并联电池组的电动势虽然不高于单个电池的电动势,但每个电池中通过的电流只是金部电流的一部分,整个电池组允许通过较强的电流.因此,用电器的额定电流比单个电池允许通过的最大电流大时,可以采用并联电池组供电.

当电池的电动势和允许通过的最大电流都小于用电器的额定电压和额定电流时,可以先组成几个串联电池组,再把几个串联电池组并联起来,使用电器得到所需的电压,而且每个电池实际通过的电流小于允许通过的最大电流.象这样把几个串联电池组再并联起来组成的电池组,叫做混联电池组.


\subsection*{练习十}
\begin{enumerate}
    \item 每个铅蓄电池的电动势为2.0伏,想用它们给一个额定电压为6伏的用电器供电,应该怎么办?
    \item 每节干电池的电动势为1.5伏,允许通过的最大电流为0.05安.现在需要一个电动势为6伏,最大电流为0.1安的电源,应该怎么办?
    \item 有10个相同的蓄电池,每个蓄电池的电动势为2.0伏,内电阻为0.04欧,把这些蓄电池接成串联电池组,外接它阻为3.6欧.求电路中的电流强度和电池组两端的电压.
    \item 找一个半导体收音机,打开看看里面有几节干电池,是怎样连接的,算一算这个收音机的电源电压是多少.
    \item 把两节干电池串联起来组成电池组,用伏特表量出电池组的电动势.再把三个小灯泡照图7.36依次连入电路中,注意每增加一个小灯泡时伏特表读数的变化,说明伏特表的读数为什么会发生变化.这个变化表明了什么?
\end{enumerate}

\begin{figure}[htp]\centering
    \begin{circuitikz}[european]
%\draw (0,3)--(6,3) to[lamp] (6,0)--(0,0) to [battery](0,3);
\draw (0,0)--(6,0) to [lamp] (6,3)--(0,3) to [battery](0,0);

\draw (1.5,0) to [rmeter,t=V, *-*] (1.5,3);
\draw (4,0) to [lamp, *-*] (4,3);
\draw (5,0) to [lamp, *-*] (5,3);

    \end{circuitikz}

    \caption{}
\end{figure}	


\section{电阻的测量}
我们在实际工作中经常需要测量电阻.测量电阻的方法很多,我们先讨论原理最简单的伏安法,然后介绍实际测量中常用的欧姆表和惠斯通电桥.

\subsection{伏安法}

根据欧姆定律$U=IR$,用伏特表测出电阻两端的电压,用安培表测出通过电阻的电流,就可以求出电阻值,这就是测量电阻的伏安法.

伏安法测量电阻在原理上是非常简单的,但由于伏特表
和安培表都有内阻,把它们连入电路中不可避免地要改变电路本身,这就给测量结果带来了误差.
\begin{figure}[htp]\centering
    \begin{circuitikz}[european, scale=.8]
\draw (0,0) to [rmeter, t=A] (2,0) to (2.5,0) to [R=$R$, *-*] (5,0)--(6,0);
\draw (2.5,0)--(2.5,1.5) to [rmeter, t=V] (5,1.5)--(5,0);
\node at (3,-1){甲};
    \end{circuitikz}\qquad 
    \begin{circuitikz}[european, scale=.8]
        \draw (0,0) to (1,0) to [rmeter, t=A, *-] (3,0) to [R=$R$, -*] (5,0) to (6,0);
        
\draw(1,0)--(1,1.5) to [rmeter, t=V] (5,1.5)--(5,0);
\node at (3,-1){乙};

            \end{circuitikz}
    \caption{}
\end{figure}

用伏安法来测电阻,可以有两种方法把伏特表和安培表连入电路,如图7.37的甲、乙所示.采用图甲的接法时,由于
伏特表的分流,安培表测出的电流强度比通过电阻的电流强度要大些,这样计算出的电阻值就要比真实值小些,采用图乙的接法时,由于安培表的分压,伏特表测出的电压比电阻两端的电压大些,这样计算出的电阻值就要比真实值大些.

待测电阻的阻值比伏特表的电阻值小得越多,采用图甲的接法时由于伏特表的分流而引起的误差越小.因此,测量小电阻时应采用这种接法.

待测电阻的阻值比安培表的电阻值大得越多,采用图乙的接法时由于安培表的分压而引起的误差越小,因此,测量大电阻时应采用这种接法.

\subsection{欧姆表}


伏安法测电阻比较麻烦,实际中常用能直接读出电阻值的欧姆表来测电阻.

欧姆表是根据闭合电路的欧姆定律制成的,它的原理如图7.38所示.$G$是内阻为$R_g$、满度电流为$I_g$的电流表.$R$是可变电阻,也叫调零电阻.电池的电动势是$\mathcal{E}$、内电阻是$r$.

当红、黑表笔相接时(图7.38甲),调节$R$的阻值,使
\[\frac{\mathcal{E}}{R_g+r+R}=I_g\]
则指针指到满刻度,表明红、黑表笔间的电阻
为零.当红、黑表笔不接触时(图7.38乙),电路中没有电流,指针不偏转,即指着电流表的零点,表明表笔间的电阻是无限大.当红、黑表笔间接入某一电阻$R_x$时(图7.38丙),则通
过电流表的电流强度
\[I=\frac{\mathcal{E}}{R_g+r+R+R_x}\]
$R_x$改变,$I$随着改
变,每一个$R_x$值都有一个对应的电流强度值$I$.如果在刻度盘上直接标出与$I$对应的电阻值$R_x$,用红、黑表笔分别接触待测电阻的两端,就可以从表盘上直接读出它的阻值.

用欧姆表测电阻是很方便的,但是电池用久了,它的电动
势和内电阻都要变化,那时欧姆表指示的电阻值,误差就相当大了.所以欧姆表只能用来粗略地测量电阻.

\begin{figure}[htp]\centering
\includegraphics[scale=.6]{fig/7-38.png}
\caption{}
\end{figure}

\section{惠斯通电桥}
用欧姆表测量电阻虽然很方便,但不够准确.在实验室里要比较准确地测量电阻,常用惠斯通电桥.

\begin{figure}[htp]
	\centering
	\begin{circuitikz}[european, scale=1.5, >=stealth]
\draw (-1.5,0)node[left]{$A$} to[R=$R_3$, *-*](0,1.5)node[above]{$B$};
\draw (-1.5,0) to[R=$R_1$, *-*](0,-1.5)node[below]{$D$};
\draw (0,-1.5) to[R=$R_2$, *-*](1.5,0);
\draw (1.5,0)node[right]{$C$} to[R=$R_x$, *-*](0,1.5);
\draw (0,1.5) to [rmeter, t=G] (0,-1.5);
\draw (-1.5,0)--(-1.5,-2.5)to [battery2](1.5,-2.5)--(1.5,0);
\draw[->](-.4,1.1)node[above]{$I_3$}--(-.1,1.5-.1);
\draw[->](0,1.5)--(.3,1.2)node[above]{$I_4$};

\draw[->](-.4,-1.1)node[below]{$I_1$}--(-.1,-1.5+.1);
\draw[->](0,-1.5)--(.3,-1.2)node[below]{$I_2$};
\draw[->](0,1.4)--(0,.7)node[right]{$I_g$};

	\end{circuitikz}	
	\caption{}
\end{figure}

图7.39是惠斯通电桥的
电路图.四个电阻$R_1$、$R_2$、$R_3$、$R_x$连成四边形,每一边叫做电桥的一个臂.$R_x$是待测电阻,其余三个是可调的已知电阻.电源接在$A$、$C$两点之间,灵
敏电流表$G$接在$B$、$D$两点之间,用来比较这两点的电势是否相等.所谓“桥”指的就是$B$、$D$两点间的连线.当$B$、$D$两点的电势相等时,叫做\textbf{电桥平衡},这时通过电流表$G$的电流强度$I_g=0$.当$B$、$D$两点的电势不相等时,叫做电桥不平衡,这时有电流通过电流表.

惠斯通电桥是利用电桥平衡条件来测量电阻的.测量
时,调节电阻$R_1$、$R_2$、$R_3$的阻值,使通过电流表的电流强度$I_g=0$.这时电桥平衡,表明$B$、$D$两点的电势相等.所以
\[\begin{split}
    U_{AB}&=U_{AD}\\
    U_{BC}&=U_{DC}
\end{split}\]
$U_{AB}=I_3R_3$,$U_{AD}=I_1R_1$,$U_{BC}=I_4R_x$, $U_{DC}=I_2R_2$.代入上面二式中,可得
\[\begin{split}
    I_3R_3&=I_1R_1\\
I_4R_x&=I_2R_2
\end{split}\]

因为这时通过电流表的电流强度$I_g=0$,所以通过$AB$和$BC$ 两臂的电流强度相等,即$I_3=I_4$,通过$AD$和$DC$两臂的电流强度也相等,即$I_1=I_2$.代入上面二式中,并把二式相除,最后得到
\[\frac{R_3}{R_x}=\frac{R_1}{R_2} \]
或\[R_x=\frac{R_2R_3}{R_1} \]

把已知的$R_1$、$R_2$、$R_3$的阻值代入,即可求出待测电阻$R_x$的值.

用惠斯通电桥测量电阻,精确度跟已知电阻的准确程度
和电流表的灵敏度有关.因为在测量时我们是根据电流表指针是否偏转来判断电桥是否平衡的,电流表不偏转,并不说明通过电流表的电流强度$I_g$绝对为零,只是说明$I_g$小到电流表检测不出来的程度.所以,已知电阻的准确程度越高,电流表的灵敏度越高,我们的测量结果就越精确.

惠斯通电桥有多种形式,学校里常用的是滑线式电桥,如图7.40所示.电桥的主要部分是一条1米长的均匀的电阻线$AC$.待测电阻$R_x$接在$B$、$C$间,作已知电阻用的电阻箱$R$接在$A$、$B$间.$D$是滑动触头,可沿$AC$线移动,平时不跟$AC$线接触,按下后接通,松手后又断开.

\begin{figure}[htp]
	\centering
	\begin{circuitikz}[european, scale=1.2, >=stealth]
\draw (-3,0)node[left]{$A$}--(3,0)node[right]{$C$} to [R=$R_x$,*-*] (0,2)node[above]{$B$} to [R=$R$,*-*] (-3,0);
\draw (-3,0)--(-3,-2) to [cute open switch] (-1,-2) to [battery] (1,-2)	to [R] (3,-2)--(3,0)	;
\draw[->](1,-2)--(1,-1.3)--(2,-1.3)--(2,-1.8);
\draw (1,-2) [fill=black] circle (1.2pt);
\draw (0,2) to [rmeter, t=G] (0,0.4);
\draw [->](0,0.4)--(-.5,.4)--(-.5,0)node[below]{$D$};
\node at (-1.75,0) [below]{$\ell_1$};
\node at (1.25,0) [below]{$\ell_2$};
	\end{circuitikz}	
	\caption{}
\end{figure}

电阻线$AC$是均匀的,$AD$段的电阻跟$DC$段的电阻之比等于它们的长度比$\ell_1/\ell_2$.接通电路,按下触头后,如果电流表中没有电流通过,我们可以用
\[R_x=\frac{\ell_2}{\ell_1}R\]
算出$R_x$的阻值.

\subsection*{练习十一}
\begin{enumerate}
    \item 在图7.37甲中,如果安培表的读数是0.2安,伏特表的读数是30伏,根据这些数据算出的$R$的阻值是多大?如果已知伏特表的内阻是3千欧,那么,$R$的真实值是多大?采用这种接法时,算出的$R$值比真实值大还是小?
    \item 在图7.37乙中,如果伏特表的读数是5伏,安培表的读数是0.5安,根据这些数据算出的$R$的阻值是多大?如果已知安培表的内阻是0.2欧,那么,$R$的真实值是多大?采用这种接法时,算出的$R$值比真实值大还是小?
    \item 已知伏特表的内阻为5千欧,安培表的内阻为0.2欧,如果用它们来测量一个线圈的电阻,估计这个线图的电阻大约为几个欧姆,那么,怎样连接电路测得的结果误差较小?画出电路图.
    \item 在图7.40中,$R$为15欧,电桥平衡时,$\ell_1$为0.45米,$\ell_2$为0.55米,求待测电阻$R_x$.
    \item 在图7.40中,如果$AB$支路发生断路,当滑动触头
    $D$从$A$移向$C$时,电流表$G$中的电流如何变化?如果$BC$支路发生断路,当$D$从$A$移向$C$时,电流表$G$中的电流又如何变化?
\end{enumerate}

\section*{复习题}
\begin{enumerate}
    \item 什么是电流?什么是电流强度?存在持续电流的条件是什么?金属导体中的电流是怎样形成的?
    \item 欧姆定律的内容和公式是什么?
    \item 电阻定律的内容和公式是什么?什么是材料的电阻率?
    \item 写出电功和电功率的公式,在电路中,电流做功消耗了什么能?消耗的能量是由什么供给的?
    \item 什么是用电器的额定电压和额定功率?当加在用电
    器上的电压低于额定电压时,用电器的实际功率还等于额定功率吗?
    \item 串联电路中电流、电压的基本特点是什么?串联电路的总电阻、电压分配和功率分配是怎样的?为什么串联电阻有分压作用?
    \item 并联电路中电流、电压的基本特点是什么?并联电路的总电阻、电流分配和功率分配是怎样的?为什么并联电阻有分流作用?
    \item 把电流表改装为安培表应该串联还是并联一个电阻?为什么?改装为伏特表呢?
    \item 电源的电动势等于什么?闭合电路的欧姆定律的内容和公式是什么?
\item 什么是路端电压?外电路上的电阻增大或减小时,路端电圧怎样变化?为什么这样变化?
\item 串联电池组的电动势和内电阻各等于什么?并联电池组呢?在什么情况下应使用串联电池组?在什么请况下应使用并联电池组?
\item 伏安法测电阻的原理是什么?试分析说明伏安法测电阻产生误差的原因.
\item 说明欧姆表的原理.
\item 惠斯通电桥测量电阻的原理是什么?
\end{enumerate}


\section*{习题}

        \begin{figure}[htp]
            \centering
            \begin{minipage}[t]{0.48\textwidth}
            \centering
            \begin{circuitikz}[european, scale=.7, >=stealth]
        
                \draw (0,0) --(4,0) to [R=$R$] (4,5)--(3,5); 
            \draw(1,5)--(0,5)to [battery2] (0,0);
        \draw (2,0) to [C=$C$, *-] (2,4);
        \node at (2,-.5){$b$};
        \draw [ultra thick ] (2,5.25)--(2,4);
        \draw [->] (2,5.25) arc (90:60:1.5);
        \draw [->] (2,5.25) arc (90:120:1.5);
        
        
        \draw (2,4) [fill=white] circle (1.5pt)node[left]{$a$};
        \draw (1,5) [fill=white] circle (1.5pt)node[above]{1};
        \draw (3,5) [fill=white] circle (1.5pt)node[above]{2};\draw (2,5.25) [fill=white] circle (1.5pt);
        
                \end{circuitikz}
            \caption{}
            \end{minipage}
            \begin{minipage}[t]{0.48\textwidth}
            \centering
            \begin{circuitikz}[european, scale=1.5]
    
                \draw (0,0) --(1,0) to [rmeter, t=mA] (2,0); 
                \draw [dashed](2,0)--(3,0)--node[right]{$B$}(3,2)--(2,2);
                \draw (0.8,0) to [rmeter, t=V, *-*] (.8,2); 
            \draw(2,0)--node[right]{$E$}(2,2)--(0,2)to [battery] (0,0);
            \node at (-.5,1){$A$};
                \end{circuitikz}
            \caption{}
            \end{minipage}
            \end{figure}



\begin{enumerate}
    \item 在图7.41中,电键可以向左扳将$a$与1接通,也可以向右扳将$a$与2接通,电键接通1的瞬间,电流的方向怎样?
接通1后再向右扳电键,将$a$与2接通,接通2的瞬间,电流的方向又怎样?
\item $A$和$B$两地相距40千米,从$A$到$B$的两条输电线的总电阻为800欧,如果在$A$、$B$之间的某处$E$两条电线发生短路(图7.42),可用伏特表、毫安表和电池组检查出发生短路的地点.如果在$A$处测得伏特表的读数是10伏,毫安表的读数是40毫安,求短路处$E$到$A$的距离.
\item 有两个灯泡,一个是110伏、100瓦,一个是110伏、
40瓦,把它们串联后接入220伏的电路中使用行不行?为什么?有一个变阻器,把它怎样连入电路中可以使两灯泡正常发光?这时变阻器的阻值应调至多大?
\item 有一用电器$W$,额定电压为100伏,额定功率为150瓦,用120伏的电源供电.为了使用电器能正常工作,用一
电阻为210欧的变阻器进行分压(图7.43).$R_1$、$R_2$为多大时,用电器才能正常工作?
\item 图7.44所示的是有两个量程的安培表,当使用$a$、$b$两端点时,量程为1安,当使用$a$、$c$两端点时,量程为0.1安.已知电流表的内阻$R_g$为200欧,满度电流$I_g$为2毫安,求电阻$R_1$和$R_2$.
\begin{figure}[htp]
	\centering
	\begin{minipage}[t]{0.48\textwidth}
	\centering
	\begin{circuitikz}[european, scale=1.3, >=stealth]
		\draw [->](0,0)--(2,0)--(3,0)to [R=$W$](3,2)--(2.15,2);
		\draw (2,0)--(2,1) to [R] (2,3)--(0,3);
		
		\draw[<->] (0,.05)--node[fill=white]{$U$}(0,3-.05);
		\draw (2,0) [fill=black] circle (1.2pt);
		\draw (0,0) [fill=white] circle (1.2pt);
		\draw (0,3) [fill=white] circle (1.2pt);
		\node at (1.5,1.5){$R_2$};\node at (1.5,2.5) {$R_1$};
			\end{circuitikz}	\caption{}
	\end{minipage}
	\begin{minipage}[t]{0.48\textwidth}
	\centering
	\begin{circuitikz}[european, scale=1, >=stealth]
		\draw (-2,0) to [R=$R_1$, *-*] (0,0) to  [R=$R_2$, *-*] (2,0);
		\draw (-2,-1.5)--(-2,1.5) to [rmeter, t=G] (2,1.5)--(2,-1.5);
		\draw (0,0)--(0,-1.5);
		\draw (0,-1.5) [fill=white] circle (1.5pt) node[right]{$b$};
		\draw (-2,-1.5) [fill=white] circle (1.5pt)node[right]{$a$};
		\draw (2,-1.5) [fill=white] circle (1.5pt)node[right]{$c$};
		\node at (0,-1.5)[below]{1A}; \node at (0,1.1)[below]{$R_3$};
		\node at (2,-1.5)[below]{0.1A};
		\draw [dashed](2.5,2.2) rectangle (-2.5, -1);
		
			\end{circuitikz}
	\caption{}
	\end{minipage}
	\end{figure}
\item 在图7.27中,不用安培表,改用伏特表能不能测出电源的电动势和内电阻?画出伏特表应怎祥接入电路,说明要取得哪些数据,写出计算电动势和内电阻的公式.
\item 在图7.45中,用一台直流发电机$F$给一台电动机$M$和一些电灯供电.已知发电机的电动势$\mathcal{E}=240$伏,内电阻$r=1$欧,输电线的总电阻$R=3$欧,电动机的工作电流为3安,供给电灯的总电流为9安,求电灯和电动机两端的电压.
\begin{figure}[htp]\centering
    \begin{circuitikz}[european]
\draw (0,0)--(8,0) to [lamp] (8,3)--(3,3) to [R=$R$] (0,3) to (0,0);
\draw (3,0)--(3,3);
\draw (5,0)to [lamp] (5,3);
\draw [dashed](6.5,0)to [lamp] (6.5,3);        
\draw (0,1.5) node[elmech]{$F$};
\draw (3,1.5) node[elmech]{$M$};
\node at (1,1.5){$\mathcal{E},\; r$};
\draw [fill=black](3,3) circle (2pt);
\draw [fill=black](5,3) circle (2pt);
\draw [fill=black](6.5,3) circle (2pt);
\draw [fill=black](3,0) circle (2pt);
\draw [fill=black](6.5,0) circle (2pt);
\draw [fill=black](5,0) circle (2pt);
    \end{circuitikz}

    \caption{}
\end{figure}

\item 现有电动势为1.5伏,内电阻为1欧的电池若干,每个电池允许输出的电流为0.05安,又有不同阻值的电阻可作为分压电阻,试设计一种电路,使额定电压为6伏、额定电流为0.1安的用电器正常工作,画出电路图,并标明分压电阻的值.
\item 用伏安法测电阻,如果所用的安培表的内阻$R_A=0.1$
欧,伏特表的内阻$R_V=1000$欧,那么,用图7.37所示的两种不同接法测量$R=1$欧的电阻时,哪种方法产生的误差较小?测量$R=500$欧的电阻时,哪种方法产生的误差较小?测量较小的电阻和较大的电阻,各应采用什么方法?
\item 在图7.40中,滑动触头$D$从$A$向$C$移动时,电流
表中都有电流通过,但电流强度逐渐减小,这时可能是什么地方发生了断路?如果电流表指示的电流强度逐渐增大,又可能是什么地方发生了断路?
\item 在图7.40中,无论在从$A$经电源到$C$的电路中发
生断路,还是从$B$经电流表到$D$的电路中发生断路,在按下滑动触头时都没有电流通过电流表.用什么办法可以把这两种情况区别开来?
\end{enumerate}







\chapter{物质的导电性}

我们知道,容易导电的物质叫做导体,不容易导电的物质叫做绝缘体.金属是导体,能够导电,这是大家都熟悉的.除了金属之外,有的液体和气体也能够导电,但是它们跟金属导电情况并不完全相同,还有一类物质,介于导体和绝缘体之间,叫做半导体,又有自己的导电特点.这一章,我们将根据物质的微观结构来讨论各种物质的导电性.

\section{金属的导电性}

在常温下,金属一般都是晶体.金属晶体的空间点阵是由金属原子释放出价电子后形成的金属离子组成的,释放出的价电子可以在整个金属晶体里自由运动,叫做自由电子,金属导体就是靠自由电子来导电的,这种导电现象叫做\textbf{电子导电}.


通常情况下,金属中的自由电子不断地做无规则的热运动,它们朝任何方向运动的机会都一样,从宏观上看,没有电荷(自由电子)的定向移动,因而也没有电流(参看图7.1),如果导体的两端有电势差,在导体内部就建立了电场,导体中的自由电子就要受到电场力的作用,这样,自由电子在导体中除了做无规则的热运动外,还要在电场力的作用下定向移动,
形成电流.

金属导体中的电流强度跟自由电子的定向移动速率有关系,它们之间的关系可以用下述的方法简单计算出来.
\begin{figure}[htp]\centering
\begin{tikzpicture}[>=latex]

\draw(0,.6)--(5,.6);
\draw(0,-.6)--(5,-.6);
\draw (5,0) ellipse[x radius=.3, y radius=.6]; 
\foreach \x in {0,1.5,3.5}
{
	\draw[dashed](\x,0)ellipse[x radius=.3, y radius=.6]; 
	\draw(\x,.6) arc [start angle = 90, end angle=270, x radius=.3, y radius=.6];
}
\node at (3.5,0){$S$};
\draw(1.5,.7)--(1.5,1.5);
\draw(3.5,.7)--(3.5,1.5);
\draw[<->](3.5,1.2)--node[fill=white]{$vt$}(1.5,1.2);

	\tkzDefPoints{1.5/0/A, 2.5/0/B, 1.7/-.4/C, 2.7/-.4/D, 1.7/.4/E, 2.7/.4/F}
\foreach \x in {A,B,C,D,E,F}
{
	\draw[->](\x)--+(.5,0);
	\draw[fill=white](\x)node{\tiny $-$}circle (3.5pt);
}


\end{tikzpicture}	
    \caption{}
    \end{figure}

如图8.1所示,设金属导体的横截面积为$S$,自由电子密度(单位体积内的自由电子数)为$n$,自由电子平均定向移动速率为$v$,那么时间$t$内通过某一横截面的自由电子数为$nSvt$.如果电子的电量为$e$,那么时间$t$内通过横截面的电量$q=neSvt$.根据电流强度的公式$I=q/t$,就可以得到电流强度和自由电子平均定向移动速率的关系式:
\[I=neSv\]

设横截面积为1.0${\rm mm}^2$的铜导线中通过1.0安的电流,铜的自由电子密度为$8.5\x10^{28}{\rm m}^3$,电子的电量为$1.6\x10^{-19}$库,由上式可以算出这时自由电子平均定向移动速率是$7. 4\x10^{-5}\ms$.而常温下金属中自由电子热运动平均速率约为$10^5\ms$.可见,在金属导电的时候,自由电子只不过在速率巨大的无规则热运动上附加了一个速率很小的定向移动就是了.

既然自由电子定向移动的速率很小,为什么我们合上开关,电流会立即传到远处,使那里的用电器工作呢?原来,我们觉察到的“电的传播速率”不是自由电子的定向移动速率,而是电场的传播速率,电场的传播速率是很大的,等于光速($3\x10^8\ms$),在电路未接通前,电路里没有电场,自由电子没有定向移动.电路一且接通,电路里便以$3\x10^8\ms$
的速率在各处极迅速地建立起电场,在这个电场作用下,电路里各处的自由电子几乎同时立即开始做定向移动,整个电路中几乎同时形成了电流.

金属导体中的电流是自由电子的定向移动形成的,从这种电子论的观点可以对欧姆定律作出微观解释.
\begin{figure}[htp]\centering
\begin{tikzpicture}[>=latex]
\draw(0,.6)--(4,.6);
\draw(0,-.6)--(4,-.6);
\draw[dashed](0,0)ellipse[x radius=.3, y radius=.6]; 
\draw (4,0)node[]{$S$} ellipse[x radius=.3, y radius=.6]; 
\draw(0,.7)--(0,1.5);
\draw(4,.7)--(4,1.5);
\draw[<->](4,1.2)--node[fill=white]{$\ell$}(0,1.2);
\draw(0,.6) arc [start angle = 90, end angle=270, x radius=.3, y radius=.6];

\end{tikzpicture}
    \caption{}
    \end{figure}

如图8.2所示,设有一段金属导体,横截面积为$S$,长度为$\ell$. 如果导体的两端有电压$U$,导体中的电场强度就是$E=U/\ell$.这时,作用在自由电子上的电场力$F=eE$. 自由电子正是在这个力的作用下做定向移动的.设电子的质量为$m$,根
据牛顿第二定律可以求出自由电子定向移动的加速度
\[a=\frac{F}{m}=\frac{eE}{m}=\frac{e}{m\ell}U\]

运动的自由电子要频繁地跟金属正离子碰撞,结果使自由电子的定向移动很快受到破坏,限制了定向移动速率的继续增加,自由电子在碰撞后向各个方向弹射的机会相等,因而失去了碰撞前具有的定向移动的特性,又要从新开始做初速为零的定向加速运动.

自由电子相继两次碰撞的时间间隔有长有短,设平均时间为$\tau$,那么,自由电子在下次碰撞前的定向移动速率$v_{\tau}=a\tau$,在时间$\tau$内的平均定向移动速率
\[v=\frac{1}{2}v_{\tau}=\frac{1}{2}a\tau\]
这样,对大
量电子来说,可以认为每个电子都以这个平均定向移动速率
做定向移动.上面已经得出
\[a=\frac{e}{m\ell}U\]
因此,自由电子平均定向移动速率
\[v=\frac{e\tau}{2m\ell}U\]
把此式代入前面推出的公式$I=neSv$中得
\[I=\frac{e^2 nS\tau}{2m\ell}U\]

对于一定的金属材料,在一定的温度下,$\tau$是个确定的数值($10^{-14}$—$10^{-12}$秒).这就是说,对于一段金属导体,$\dfrac{e^2 nS\tau}{2m\ell}$是
一个常量.因此,我们得到,导体中的电流强度$I$跟这段导体
两端的电压$U$成正比,这正是欧姆定律告诉我们的.

这种以经典物理为基础的电子论(经典电子论)虽然能对金属导电规律作出解释,但它只能给出金属导体的电子导电的概况,而不能获得满意的定量的结果.例如,对于金属导体的电阻跟温度的关系,从电子论得到的是电阻跟温度的平方根成正比,而实验结果却是电阻跟温度成正比.这说明经典电子论本身存在着缺陷.近代物理指出,只有用量子理论来研究金属导电,才能得到与实验相符合的结果.

\subsection*{练习一}
\begin{enumerate}
    \item 在金属导体中,自由电子的热透动速率和定向移动逮率之间有什么区别?这两种速率哪个大?
    \item 电路接通后,为什么整个电路中几乎同时形成电流?
    \item 利用公式$I=\dfrac{e^2 nS\tau}{2m\ell}U$,求出这段导体的电阻$R$和制
成这段导体的材料的电阻率$\rho$.
\end{enumerate}


\section{液体的导电性}
我们在化学中学过,酸、碱、盐是电解质,它们的水溶液或者它们熔解成的液体能够导电,这就是液体导电,液态金属也能够导电,也属于液体导电,液态金属的导电情况跟固态金属相同,都是电子导电,这里不再讨论,这一节,我们讨论电解质导电.

电解质在水溶液中或者熔解成液体时都要发生电离,它们的分子电离成正离子和负离子,这些离子不断地做无规则的热运动,没有电场存在时,从宏观上看没有电荷(离子)的定向移动,不显示出电流.

在电解质的液体里插入两个电极,把它们分别接在电源的正负极上,跟电源正极相连的叫阳极,跟电源负极相连的叫阴极.于是液体中出现电场,电解质的正负离子除了做热运动外,还要在电场力的作用下散定向移动,正离子向阴极移动,负离子向阳极移动.电解质就是靠正负离子的定向移动形成电流的,这种导电现象叫做\textbf{离子导电}.显然,这跟金属中的电子导电是不同的.

在电解质导电过程中,同时发生电解现象.电解时,正离子在阴极板上得到电子发生还原反应,负离子在阳极板上失去电子发生氧化反应,电解质的导电过程要发生化学变化,这也是跟金属导电不同的.金属导电时,金属本身并不发生化学变化.例如,在氯化铜溶液中氯化铜分子电离成铜离子${\rm Cu}^{2+}$和氯离子${\rm Cl}^-$.导电时,带正电的铜离子向阴极移动,带负电的氯离子向阳极移动(图8.3).在阳极,氯离子失去电子而氧化成中性氯原子,两个氯原子又结合成氯分子,从阳极放出.在阴极,铜离子获得电子而还原成中性铜原子,就覆盖在阴极上.
\begin{figure}[htp]\centering
    \includegraphics[scale=.6]{fig/8-3.png}
    \caption{}
    \end{figure}

电解在工业上有着广泛的应用,如电镀、电冶、电解精炼等,这些都已在化学课学习过,这里就不再重复了.利用电解还可以使金属表面氧化,例如,把铝板放在硼酸或硼砂的水溶液中作为阳极,利用电解现象,可以在铝的表面得到非常薄的氧化膜,这个氧化膜有良好的绝缘性.因此,把铝作阳极,把电解液作阴极,可以做成电容量很大的电容器,这就是电解电容器.由于电解电容器是利用电解产生的绝缘膜作电介质的,因此把它连在电路中时,它的正负极不能接反,并且不能接在交流电路中.

\section{法拉第电解定律}
电解质导电时,要发生化学变化,有物质在极板处析出.那么,析出的物质的质量跟通电的电流强度和时间有什么关系呢?

这个关系可以用实验来确定.给某种电解质溶液通电,先使各次的通电时间相同,而电流强度不相同,再使各次通电
的电流强度相同,而时间互不相同,每次都精密地测出极板(阳极或阴极)处析出的物质的质量.实验结果表明:在通电时间相同的情况下,析出物质的质量跟电流强度成正比;在电流强度相同的请况下,析出物质的质量跟通电时间成正比.例如,氯化铜溶液电解时,从阳极放出的氯气的质量或在阴极上得到的铜的质量都跟通电的时间和电流强度成正比.

\textit{电解质导电时,在极板处析出的物质的质量$m$跟通电时间$t$和电流强度$I$成正比,或者说跟通过电解液的电量$q$成正比}.这就是\textbf{法拉第电解第一定律},写成公式就是
\begin{equation}
    m=kIt=kg
\end{equation}
式中的$k$是比例恒量,叫做\textbf{电化当量}.从实验知道,电化当量的数值随着被析出的物质种类而不同,某种物质的电化当量在数值上等于通过1库仑电量时析出的这种物质的质量.电化当量$k$的单位是
${\rm kg}/{\rm C}$.

\begin{center}
    \begin{tabular}{ccc}
        \hline
        物质  &  电化当量$k$  &  化学当量$M/n$\\
        & ($10^{-8}{\rm kg}/{\rm C}$)& ($10^{-8}{\rm kg}/{\rm C}$)\\
        \hline
H$^+$       &   0.01044    &  1.0079     \\
Al$^{3+}$       &  0.00317     &    8.9938   \\
Fe$^{3+}$       &  0.1930     &  18.6157     \\
Cu$^{2+}$       &  0.3294     &   31.772    \\
Ag$^{+}$       &   1.118    &   107.868    \\
O$^{2-}$       &  0.0829     &   7.9997    \\
Cl$^{-}$       &  0.3672     &   35.453    \\
OH$^{-}$       &  0.1762     &  17.007     \\
SO${}_4^{2-}$       &   0.4975    &  48.029     \\
NO${}_3^{-}$       &  0.642     &  62.005     \\
        \hline
    \end{tabular}
\end{center}


电化当量$k$可以从实验得出,通过电解质的电流强度$I$、
通电时间$t$和析出物质的质量$m$都可以由实验测定,根据法
拉第电解第一定律就可以求出$k=m/(It)$.
上表的第二列给出了
几种物质的电化当量.

知道了物质的电化当量,就可以由通电时间和电流强度
求出析出的物质的质量.

物质的电化当量又跟物质的什么性质有关呢?法拉第在大量实验事实的基础上指出:\textit{物质的电化当量跟它的化学当量成正比},这就是\textbf{法拉第电解第二定律}.

某物质的化学当量是该物质的摩尔质量$M$跟它的化合价$n$的比值,化学当量的单位是${\rm kg}/{\rm mol}$.例如,铜的摩尔质量
是0.063546${\rm kg}/{\rm mol}$,二价铜的化合价$n=2$,所以它的化学当量$M/n$等于摩尔质量的一半,即0.031772${\rm kg}/{\rm mol}$,上表第三列给出了几种物质的化学当量.

通常,电化当量$k$跟化学当量$M/n$之间的比例常数用$1/F$来表示,于是法拉第电解定律可用公式表示为
\begin{equation}
    k=\frac{M}{Fn}
\end{equation}

式中的$F$叫做\textbf{法拉第恒量},对于任何物质都是相同的,实验测量结果表明,$F=9.65\x10^4{\rm C}/{\rm mol}$.同学们根据上面表中的两列数据,算一算各种物质的电化当量$k$跟化学当量$M/n$的比值,就会得到它们的比值是相同的.

法拉第恒量的意义是什么呢?把电解第二定律的公式(8.2)代入电解第一定律的公式(8.1)中,得到
\begin{equation}
    m=\frac{M}{Fn}q
\end{equation}

上式中如果$m$的数值等于$M/n$的数值,那么$F$的数值等于$q$.可见,\textit{法拉第恒量在数值上等于析出数值等于化学当量的物质时,需要通过电解质的电量}.例如一价银的化学当量是0.107868${\rm kg}/{\rm mol}$,析出0.107868kg的银,需要$9.65\x10^4$C的电量.二价铜的化学当量是0.031772${\rm kg}/{\rm mol}$,析出0.031772kg的铜,也需要$9.65\x10^4$C的电量.

\section{电子电量的确定}
法拉第电解定律是实验定律.法拉第发现电解定律时,人们还不清楚物质的分子、原子结构,也不知道存在着电子.然而,法拉第电解定律的发现,给予人们有益的启示,为人们认识基本电荷起了重要作用.

我们知道,电解质导电是离子导电,而每种离子都带有一定的电荷.因此,在电极处析出一定质量的某物质,就表明有一定数量的这种物质的离子到达电极.到达电极的离子数越
多,通过电解质的电量就越多,析出的物质也就越多.这就是电解第一定律所表达的内容.

下面我们进一步指出,根据电解第二定律所确定的法拉第恒量,我们可以算出离子所带的电量,而一价离子所带的电量就等于电子的电量,即基本电荷.

我们知,在数值等于化学当量$M/n$的物质中,含有的
离子数是$N/n$,其中$N$是阿伏伽德罗常数,等于$6.02\x10^{23}{\rm mol^{-1}}$.有$N/n$个离子到达电极,通过电解质的电量的数值等于$F_1$为$9.65\x10^4$.因此,$n$价离子所带的电量$q_n$是:
\[q_n=\frac{F}{N/n}=n\frac{F}{N} \]

对于一价离子,$n=1$,$q_n$就等于电子的电量,即
\[e=\frac{F}{N}\]

可见,知道了法拉第恒量$F$和阿伏伽德罗常数$N$,就可以求出电子的电量 
\[e=\frac{F}{N}=\frac{9.65\x10^4}{6.02\x10^{23}}=1.60\x 10^{-19}{\rm C}\]
这个结果跟用其他方法(例如第六章讲的密立根实验)求得的电子电量的数值是一致的.

实际上,公式$e=F/N$把法拉第恒量、基本电荷、阿伏伽德罗常数这三者联系起来了,测出法拉第恒量和阿伏伽德罗常数,可以求出基本电荷;测出法拉第恒量和基本电荷,也可以求出阿伏伽德罗常数.

\section*{阅读材料:$e$和$N$的测定}
电子电量$e$和阿伏伽德罗常数$N$都是物理学的基本常数,从公式$e=F/N$可以看出,在$e$和$N$这两个常数的测定中,如果有一个常数有了新的测定数值,另一个的数值就要随着变化,二者的数值测定明显地有着互相推动的作用.

阿伏伽德罗常数最早是洛喜密脱于1865年左右根据分子运动论所确定的关系来测定的,二十世纪初又有人利用研究布朗运动所得到的关系以及利用其他方法来测定它的数值.但是,这些方法本身都有缺点,所得的结果都不很准确.

直到1917年,美国物理学家密立根用油滴法测出了电子电量,并利用已知的法拉第常数,由公式$e=F/N$求出阿伏伽德罗常数,这才有了在当时看来准确度较高的结果.

1928年,瑞典物理学家贝克林利用X射线的方法测得了
阿伏伽德罗常数,反过来又求出电子电量,他得到的$N$和$e$的数值,跟密立根的结果相比有较大的差别.尽管他的方法是合理的,然而当时普遍认为密立根的方法是无可怀疑的,因此两个结果到底哪一个较为准确,一时难于确定.

1932年,日本的芝龟吉指出了密立根的结果出现误差的原因.考虑到这个误差,1936年贝克林和弗仑堡重新用油滴法测定电子电量,结果得到了跟X射线法十分接近的$e$和$N$的数值.以后,又有人多次用两种方法测定$e$和$N$,得到了更加一致的结果,这样,原来两种方法得到的$e$和$N$的数值的差别,在实验的基础上基本得到统一.$e$和$N$的数值测定在物理学中具有重要意义,随着科学技术的发展,人们在不断研究精确测定它们的方法.

\subsection*{练习二}
\begin{enumerate}
    \item 通过硫酸铜溶液的电量是$2.0\x10^4$库,在阴极上能析出多少克铜?
    \item 如果要在表面积是5${\rm cm^2}$的器件上,镀上一层20微米厚的银层,需要通过多少库仑的电量?
    \item 有一个学生电解硫酸铜溶液来测定铜的电化当量.他在通电以前称一次阴极板,通电25分钟以后再称一次,从而知道析出的铜的质量是0.29克,还知道通过溶液的电流强度是0.6安,从这些数据算出铜的电化当量是多少?
    \item 锌的摩尔质量是0.06538kg/mol,它的化合价是2,求锌的电化当量.
    \item 金的摩尔质量是0.1972kg/mol,它的化合价是3,要想使金电解池的阴极上析出1克金,需要通过多少库的电量?
\end{enumerate}


\section{气体的导电性}
在通常情况下,气体是不导电的.例如,采用裸导线的输电线路,不会由于空气的存在而把电路连通,造成短路.但
是,在某些外界条件作用下,气体是可以导电的.象图8.4所示的那样,取两个验电器,让一个带正电,另一个带负电,并使它们上端的金属球互相靠近.如果空气是干燥的,验电器的带电状态将保持不变.把酒精灯火焰置于两个金属球之间,可以看到两个验电器很快就不带电了,说明这时空气变成了导体,把两个验电器导通了.
\begin{figure}[htp]\centering
    \includegraphics[scale=.6]{fig/8-4.png}
    \caption{}
    \end{figure}

空气为什么会变成导体呢?通常情况下,气体几乎完全是由中性的原子或分子组成的,因此它是不导电的.而空气在火焰的作用下能够电离,部分中性分子变成带电的离子,结果气体变成导体.

不仅火焰能使气体电离,利用其他方法,例如用紫外线、
X射线或放射性元素发出的放射线照射,也能使气体电离.能使气体电离的物质叫做\textbf{电离剂}.

气体电离后发生的导电现象叫做\textbf{气体放电}.

气体导电跟电解质导电相似,它们都是由于发生了电离而具有导电性的,但是也有区别,电解质电离时,分解成正离子和负离子;气体电离时,分解成正离子和电子,在气体中也可以形成负离子,这是由于电子附在中性分子上的结果,可见,气体中既有象金属中那样的电子导电,也有象电解质中那样的离子导电.

现在我们利用如图8.5的实验装置对气体放电现象作进
一步的讨论,放电管是装有两个电极的玻璃管,里面充有一
定压强的气体,给两电极加上一定的电压,在电离剂的作用下,放电管里面的气体发生放电.拿走电离剂,气体不再发生电离,放电现象也就停止.但是,如果增大两电极间的电压,即使没有电离剂的作用,也能发生放电现象.
\begin{figure}[htp]\centering
    \includegraphics[scale=.6]{fig/8-5.png}
    \caption{}
    \end{figure}

在电离剂的作用下才能发生的气体放电现象叫做\textbf{被激放电}.没有电离剂的作用而发生的气体放电现象叫做\textbf{自激放电}.

放电管里的气体,为什么在高电压下会发生自激放电呢?

通常,气体中总会有少量电子和离子.电子在电场的作用下向阳极运动.在前进的道路上,电子会碰到离子和中性原子.在两次碰撞之间,由于电场力做功,电子的动能增加.两电极间的电压越大,电场越强,电子的动能增加得就越大,电压足够高,电场足够强,电子的动能增大到一定程度时,电子跟中性原子碰撞,就会从原子中打出电子,发生气体电离.这种电子碰撞电离\footnote{实验结果表明,正离子在电场作用下跟中性原子碰撞时,很少使原子发生电离,因此,这里只讨论电子碰撞电离.}的过程,可以用图8.6来说明,图甲表示碰撞前的情形,图乙表示碰撞后的情形,碰撞后有两个电子,一个是跟原子相碰的电子,一个是从原子中打出来的电子,它们也要在电场中获得动能,并使被到的原子电离.
\begin{figure}[htp]\centering
    \includegraphics[scale=.6]{fig/8-6.png}
    \caption{}
    \end{figure}

但是,单靠电子碰撞使气体电离并不能维持自激放电.因为碰撞产生的所有电子都要向阳极方向运动,到达阳极后就停止了,为了维持放电,必须使阴极能够源源不断地提供电子,通常把阴极提供电子的过程叫做\textbf{电子发射}.

气体电离时产生的正离子,在向阴极运动时,也具有动能.如果这个动能足够大,正离子跟阴极碰撞时,就会从阴极表面打出电子,这些电子又会使气体发生电子碰撞电离,产生新的电子和正离子,这就维持了气体电离,形成自激放电.

从上面的讨论可以得出,产生自激放电的条件是气体电离和阴极发射电子.

象上面讲的那样,正离子跟阴极碰撞使阴极发射电子,叫做正离子轰击发射.然而相当多的情况是给阴极加热,使阴
极在高温下发射电子,这叫做热电子发射,如果你想对电子发射有进一步的了解,可以看一看下面的阅读材料:电子发射.


\section*{阅读材料:电子发射}
维持自激放电,阴极要能够提供电子.后面就要讲到的阴极射线管、示波管以及其他电子器件,它们的阴极都是能提供电子的源泉,阴极的这种提供电子的过程叫做电子发射.

阴极为什么能够发射电子呢?

我们知道,金属制成的阴极里有许多自由电子,这些自由电子在正离子间不停地做无规则的热运动.当电子趋近金属表面时,它们所受的正离子向内的拉力将急剧增加,这种情况好象是金属表面对电子形成了一道壁垒.通常条件下,自由电子的能量太小,几乎无法越出这个壁垒,只能局限在金属内部运动.然而,可以设法增加自由电子的能量,或者设法减弱金属表面对自由电子逸出金属的阻力.这样,自由电子就可以冲破金属表面壁垒的封锁,从金属里释放出来,这就是电子发射现象.

电子发射可以分为下列几种形式:

(1)热电子发射.给金属加热,做热运动的自由电子的能量随着温度的升高而增大,其中一部分电子可以获得足够的能量,克服阻止它们逸出的阻力而逸出金属.

(2)正离子轰击发射.气体电离产生的正离子,在电场力作用下高速运动而具有很大的动能,当它们跟阴极碰撞时,
可以使一部分自由电子获得足够的能量,从阴极释放出来.

(3)光电子发射.某些金属受到光的照射时,其中的一部分自由电子能够吸收足够的能量,而产生电子发射,这种电子发射也叫做光电效应.

(4)场致电子发射,当金属表面存在很强的电场时,可以减弱金属表面对自由电子逸出的阻力,而使能量较大的电子从金属里释放出来.这种电子发射又叫冷发射.

除了以上几种电子发射外,还有二次电子发射等其他一些形式,这里不再介绍.

热电子发射有着极为广泛的应用,下面简单介绍这种形式发射的阴极——热阴极.

常见的热阴极有直热式和间热式.图8.7所示的是一种直热式钨阴极.将钨丝弯成V字形,用金属支柱固定,直接给钨丝通电加热,电子便从它的顶端发射出来,形成一个点状的电子源.
\begin{figure}[htp]
    \centering
    \begin{minipage}[t]{0.48\textwidth}
    \centering
    \includegraphics[scale=.6]{fig/8-7.png}
    \caption{}
    \end{minipage}
    \begin{minipage}[t]{0.48\textwidth}
    \centering
    \includegraphics[scale=.6]{fig/8-8.png}
    \caption{}
    \end{minipage}
    \end{figure}

钨的熔点很高,钨阴极的工作温度可达2000K以上,它有足够的机械强度和良好的稳定性.有的电子器件(如电子显微镜),工作时需要连续抽气,有时又需要暴露在大气中,它们的阴极大都是直热式钨阴极.

图8.8所示的是示波管和显像管的阴极,它是一种间热
式氧化物阴极.镍制套管的顶部是镶合金制成的阴极帽,上面涂有碱土金属(钙、锶、钡)的氧化物层,套管里有一根绕成螺线状的钨丝,叫做热丝.热丝的表面有绝缘物质氧化铝层跟套管绝缘,给热丝通电,阴极被间接地加热,使氧化物表面产生电子发射.间热式阴极是间接加热的,因此,具有这种阴极的电子器件工作时需要预热.

氧化物阴极的工作温度较低(1000K左右),但发射电子的能力强,能够产生很强的电子流.接收、放大电子管和中小型振荡电子管几乎全部采用氧化物阴极,电子束管(如示波管,显像管等)的阴极也大都是氧化物阴极.

\section{几种自激放电现象}
\subsection{辉光放电}
\begin{figure}[htp]\centering
    \includegraphics[scale=.6]{fig/8-9.png}
    \caption{}
    \end{figure}

如图8.9所示,一支两端装有电极的玻璃管,当里面的气压比较高时(例如与大气相通),给两极加上几百伏的电压,管里的空气并不发生放电现象.现在用抽气抽出管里的空气,当管里的空气十分稀薄,气压降到几千帕斯卡以下时,管里的空气就会放
电,出现彩色光柱,这就是辉光放电,光柱的颜色和形状随管内气体的压强不同而改变.管内气体不同,光柱的颜色也不同.

稀薄气体的辉光放电是怎样形成的呢?

气体稀薄时,气体分子间的距离增大,电子连续两次碰撞所经过的路程加大,因而在电场力作用下获得较大的动能,它们跟中性分子碰撞,会发生电子碰撞电离.同样的道理,正离子也可以获得较大的动能,跟阴极碰撞时,会从阴极表面打出电子.于是,气体发生了自激放电.气体的压强越低,发生自激放电需要的电压也越低.

\subsection{弧光放电} 

把两个碳棒分别跟电源的两极相连,让它们接触后再稍稍分离,在两碳棒间就会出现明亮的放电现象,这就是弧光放电.在大气中产生弧光放电时,两碳棒间的电压并不高,通常只有几十伏,而电流可达几安甚至几十安.这说
明这时气体有很好的导电性,电阻很小.

弧光放电是怎样形成的呢?当两个碳棒相接触时,由于接触处的电阻很大,电流通过时要放出大量的热量,从而使这里的温度升得很高.当两个碳棒分离后,这样高的温度足以使得气体发生电离和阴极发射电子,于是开始放电.这就是说,开始产生弧光放电的原因是由于气体和阴极有很高的温度.使弧光放电维持下来的原因也是由于高温,被电场加速了的电子和离子跟气体的中性分子碰撞,加剧了气体分子的热运动,使气体温度升得很高,继续发生电离.被电场加速了的正离子轰击阴极,使阴极保持很高的温度,继续发射电子.在大气压下的弧光放电,气体的很度高达2000K以上.

弧光放电不仅可以在两个碳棒电极之间发生,也可以在两个金属电极之间发生.

在辉光放电中,如果减少外电路的电阻,使放电电流增大,那么,当电流很强的时候,可以使气体和阴极的温度升得
很高,因而辉光放电变成弧光放电.

弧光放电在实际中有着广泛的应用.弧光放电时发出强光,可以用来制造各种气体放电光源.弧光放电时放出大量的热量,可以被用来作为热源,如用于电炉炼钢(图8.10)和电焊(图8.11).
\begin{figure}[htp]
\centering
\begin{minipage}[t]{0.48\textwidth}
\centering
\includegraphics[scale=.6]{fig/8-10.png}
\caption{电炉炼钢}
\end{minipage}
\begin{minipage}[t]{0.48\textwidth}
\centering
\includegraphics[scale=.6]{fig/8-11.png}
\caption{电焊}
\end{minipage}
\end{figure}

\subsection{火花放电}


把起电机两电极的放电球互相靠近到几厘米,让起电机起电,随着放电球上电荷的增加,两极间的电势差和电场强度都要增大.当电场强度增大到一定程度时(约为$3\x10^6{\rm V}/{\rm m}$),在两极间就出现一束明亮的、曲折而又分叉的细丝.这些明亮的细丝很快地穿过两极间的气体,一个接着一个地出现,并且伴有爆炸声,这就是火花放电.

产生火花放电的电势差的大小跟气体的性质和压强、电极的大小和形状以及两极间的距离等因素有关系.气体的压强越低,两电极间的距离越小,产生火花放电的电势差就越小,尖端状电极容易发生放电.例如在大气压强下,电势差
为$2\x10^4$伏,直径为5厘米的球状电极相距5.8毫米时产生火花放电,而尖端状电极相距15.5毫米即可产生火花放电.

当电源的功率不是很大时,火花放电的放电过程是间断
发生的.这是因为放电时气体被火花击穿,电阻变小,这样在电路里就会有很大的电流,因而引起电势差的从新分布,结果两极间的电势差降到很小,于是停止放电,接着,当两极间的电势差回升到原来的数值时,又重新开始放电.当电源的功率足够大时,放电过程可以连续地维持下去,这时火花放电变成弧光放电.

火花放电中使气体电离的原因,是强电场作用下的电子碰撞.此外,火花本身的辐射也会使气体电离.

闪电是自然界发生的火花放电,带有相反电荷的两块云,当它们带的电量足够多时,在它们之间就会形成强大的电场,于是发生火花放电、云层跟大地之间也能够发生闪电,这就是雷击.根据计算,发生闪电时云层之间或云层和大地之间的电势差可高达几十亿伏,电流强度可达几十万安,而放电时间很短,只有百万分之一秒到百分之一秒.火花放电路径里的温度很高,常常可以达到10000K.一次闪电放出的能量很大,可达$10^{14}$焦,因此,雷击是一种非常危险的现象,会给人们带来危害.为了防止雷击,雷雨天不要在高大的树下避雨,电视机、收音机装有室外天线时,要有妥善的接地装置,高大的建筑物安装避雷针,也是为了避免雷击.

\subsection{电晕放电}


在大气中,当导体的电压很高时,靠近它的地方有很强的电场,能够使空气电离而产生放电,这种放电只发生在靠近导体的薄的一层空气里,在暗处会看到发出微弱的光,这种放电现象叫做电晕放电.

带电导体的尖端附近电场最强,因而最容易放电,这种尖端放电就是电晕放电.如果起电机的放电球上带有尖
端,在暗处起电时,就会看到尖端处出现淡紫色光点.一些高压设备的电极常做成光滑的球面,就是为了避免尖端放电.高压输电线也会发生电晕放电,这是一种电能损失.因此高压输电线的表面要做得很光滑,不能有凸起或尖端.

尖端放电也可以被人们所利用,前面提到的避雷针就是一个典型的例子.避雷针是一个金属的尖端导体,把它安装
在建筑物顶端,并用粗铜缆把它与大地接通.铜缆要接到埋在地下的金属板上,以保持避雷针与大地接触良好.当带电的云层与建筑物接近时,放电通过避雷针和接地导体这条通路不断进行,避免电荷积累而发生雷击.

\section{气体电光源}

气体放电时能发光,这可以用来制成气体电光源.气体电光源的种类很多,这里介绍常见的几种.

\subsection{霓虹灯}


霓虹灯是一种稀薄气体放电灯.把细长的玻璃管弯成所需要的形状,两端装上电极,玻璃管抽成真空后充入压强几百帕斯卡的稀薄气体,通电时,管里的气体发生辉光放电,发出均匀柔和的光.把不需要发光的部位的管壁涂黑,就可以显出字形或图案.霓虹灯需要的电压较高,灯管越细、越长,需要的电压越高.现在,霓虹灯大都按一定的规格制作,起动时需要的电压约为15000伏,维持正常工作的电压约
为7500伏.但通过的电流很小,通常不超过25毫安,管里充入的气体不同,霓虹灯的颜色也不一样.例如,氖气发桔红色光,氦气发金色光,氩气发淡蓝色光等.如果在管的内壁涂不
同的荧光粉,还可以产生更多的颜色.

霓虹灯的色彩鲜艳夺目,能引人注意,因此在商店、车站、码头等场所常常用它来作招牌、广告和装饰用的灯.在机场还可以用它来作信号灯.

\subsection{日光灯}
日光灯管的结构如图8.12所示.管的两端各
有一条灯丝,管里抽成真空后装人少量的水银,并充有压强约为400帕斯卡的氩气,管的内壁涂有荧光粉,通电后,水银变成蒸气,发生气体放电.水银蒸气放电发出的光只有小部分是可见光,大部分是看不见的紫外线.紫外线射到荧光粉上,使荧光粉发出可见光.选用适当的荧光粉,可以使灯管发出的光的颜色接近日光,日光灯的名子就是由此而来的.
\begin{figure}[htp]\centering
    \includegraphics[scale=.6]{fig/8-12.png}
    \caption{}
    \end{figure}

日光灯发出的光比较柔和,并且发光效率高,比白炽灯高4—5倍,因此,日光灯广泛地用作室内照明,但是,日光灯的体积大,还需要镇流器等辅助器件,而且功率也不能做得太大,这使它的应用受到了一定的限制.

\subsection{高压水银灯}

照明用的高压水银灯如图8.13所示,外壳呈椭圆体形,是硬玻璃制成的,起保温作用,防止周围环境对灯的影响,外壳的内壁涂荧光粉.中央是耐高温的石英玻璃管,内充一定量的水银和少量氩气,两端装有主电极,其中一
个主电极附近有一个辅助电极.辅助电极跟相邻的主电极之间的距离很近,通常只有2—3毫米,因此,给灯通电后,这两个电极之间的气体首先发生电离,形成辉光放电,放电时产生大量的电子和离子,于是引起两主电极的辉光放电,并逐步形成弧光放电.放电时放出的热量使水银大量蒸发,当水银全部蒸发后,灯管达到稳定工作状态.从启动到稳定工作状态通常需要4—10分钟.气体放电时发出可见光和紫外线,紫外线又使外壳内壁的荧光粉发出可见光,从而得到很强的可见光,这种灯工作时的温度很高,电弧轴心的温度可达5500K左右,管壁的温度也有600—700K.在这样高的温度下,管内水银蒸气的压强也很高,约为$2\x10^5$—$6\x10^5$帕斯卡,高压水银灯的名字就是由此而来的.相比之下,日光灯的工作温度约为300K多些,这时水银蒸气压只有0.8帕斯卡.因此,日光灯又叫做低压水银灯.
\begin{figure}[htp]\centering
    \includegraphics[scale=.6]{fig/8-13.png}
    \caption{}
    \end{figure}

高压水银灯的体积小、亮度大,因而通常用在街道、广场、仓库等地的照明,高压水银灯也有它的不足之处.如上面提到的,启动时间较长,需要4—10分钟,同时不能连续启动,熄灭后要停5—10分钟,等灯的温度下降,管内的压强降低后,才能重新启动,跟日光灯一样,也需要有镇流器等辅助器件.

除了以上几种气体电光源外,还有超高压水银灯、钠灯和氙灯等.超高压水银灯的水银蒸气的压强在$10^6$帕斯卡以上,亮度比高压水银灯更大.高压钠灯的发光效率最高,它发出的光呈金黄色,现在大中城市的主要街道上已用它来照明.氙灯的亮度最大,功率可达几万瓦,常用在大的车站、码头、机场和建筑工地等需要大面积照明的地方.

\section{真空中的电流}

在稀薄气体辉光放电的实验中,如果继续抽出管内的空气,当管内的压强降到约0. 1帕斯卡的时候,也就是通常所说的抽到真空的时候,就不再发生辉光放电,这是因为管内的空气相当稀薄,电子在管中运动几乎碰不到气体分子,所以不能使气体电离发光.这时对着阴极的玻璃管壁却发出荧光;如果在管中放一个十字形的金属片,荧光就出现十字形阴影.这表明玻璃壁发出的荧光是从阴极发出的一种射线引起的,这种射线叫做阴极射线.产生阴极射线的管子叫做阴极射线管.

为了查明阴极射线的性质,人们做了大量的实验.在阴极射线管内装一个可以在支架上滚动的叶轮,阴极射线射到叶轮上会使叶轮滚向阳极.这表明阴极射线是高速运动的粒子流,有很大的动能.

在阴极射线管内装一个涂有荧光物质的板,用来显示阴极射线的轨迹;在阴极射线通路的两侧各装一个极板,把它们接到直流电源的两极上,会看到阴极射线向
正极板偏转.这表明阴极射线是带负电的粒子流.

1897年,英国科学家汤姆生测定了这种粒子的电荷跟质量的比,发现它是带有最小负电荷的粒子,也就是电子,阴极射线是从阴极发出的电子流.因此,阴极射线又叫做电子射线,阴极射线管又叫做电子射线管.

前面已说明,这里讲的电子射线管内的压强约为0.1帕斯卡,这种情况下,电子射线的产生,是属子前面讲的正离子轰击发射,也就是说,正离子在电场的作用下获得很大的动能,它跟阴极碰撞,使阴极发射电子.

\section{示波管}
示波管是一种电子射线管,它的构造如图8.14所示.玻璃管内的真空度很高,可达$10^{-4}$帕斯卡.在这样高度真空下,它是采用热电子发射的方式来产生电子射线的.
\begin{figure}[htp]\centering
    \includegraphics[scale=.6]{fig/8-14.png}
    \caption{}
    \end{figure}

灯丝通电后温度升高,给阴极加热,使阴极发射电子.电子经过圆筒形的控制极和阳极的小孔,形成一束很细的加速的电子束.用来产生电子束的这一部分装置通常叫做电子枪,从电子枪射出的电子束,打在管底的荧光屏上,形成一个
小的亮斑.亮斑的大小和亮度,由加在控制极和阳极上的电压来控制.

在电子枪的前面是两组偏转极板:水平偏转极板$XX'$和竖直偏转极板$YY'$.如果仅在水平偏转极板上加一个电压,例如极板$X$的电势高,极板$X'$的电势低,那么电子束从两极板间通过时就要向极板$X$一边偏转,荧光屏上的亮斑也要向极板$X$方向水平偏移(图8.15甲).反之,如果极板$X$的电势低,极板$X'$的电势高,亮斑就向极板$X'$方向水平偏移(图8.15乙).$XX'$间的电势差越大,亮斑偏移得越大.采用特定的电子线路,可以使$XX'$间的电势差从某一负值均匀地上
升到某一值,然后迅速地返回原来的负值,再均匀上升,这时荧光屏上的亮斑就从一边匀速地水平移向另一边,然后迅速地返回原处,再匀速地水平移向另一边.这个过程叫做扫描;加在水平偏转极板上能够产生扫描的电压叫做扫描电压.当扫描的频率加快时,由于视觉暂留和荧光物质的残光特性,在荧光屏上移动的亮斑看起来成了一条亮线(图8.16).
\begin{figure}[htp]
    \centering
    \begin{minipage}[t]{0.48\textwidth}
    \centering
    \begin{tikzpicture}

        \draw [dashed] (-3,0)node[left]{$X'$}--(3,0)node[right]{$X$};
        \draw (-1.5,0) circle (1);
        \draw (1.5,0) circle (1);
\draw [dashed](-1.5,1.5)--(-1.5,-1.5)node[below]{甲}; \draw [dashed](1.5,1.5)--(1.5,-1.5)node[below]{乙};
\draw[fill=black] (-.8,0) circle (1.5pt);
\draw[fill=black] (.8,0) circle (1.5pt);
    \end{tikzpicture}
    \caption{}
\end{minipage}
\begin{minipage}[t]{0.48\textwidth}
\centering
    \begin{tikzpicture}
        \draw (0,0) circle (1);
        \draw [dashed](-1.5,0)--(1.5,0);         \draw [dashed](0,-1.5)--(0,1.5); 
        \draw [ultra thick](.55,0)--(-.55,0);
    \end{tikzpicture}
    \caption{}
\end{minipage}
\end{figure}

同样的道理,给竖直偏转极板$YY'$加一个电压时,荧光
屏上的亮斑就要发生竖直偏移,如果$YY'$上的电压随时间变化,那么亮斑将在竖直方向移动,移动的规律跟电压变化的规律相同.

如果在竖直偏转极板上加一个变化的电压,同时又在水
平极板上加一个一定频率的扫描电压,那么荧光屏的亮斑在竖直方向移动的同时,又在水平方向匀速移动.于是,从亮斑的位置变化就可以知道竖直极板上的电压随时间变化的情况.

通常,加在竖直极板上的要研究的电压是周期性变化的.如果扫描电压的周期与竖直极板上电压的周期相同,并且起始时间也相同(这在技术上叫做同步),那么,在荧光屏上就可以看到一条稳定的图线.这时得到的图线表示出竖直极板上的电压在一个周期内随时间而变化的规律.例如,加在竖直极板上的电压是随时间按正弦规律变化的,它的变化周期与扫描电压的周期相同,起始时间也相同,那么在荧光屏上就显示出一条正弦曲线(图8.17).
\begin{figure}[htp]\centering
    \begin{tikzpicture}
        \draw (0,0) circle (1);
        \draw [dashed](-1.5,0)--(1.5,0);         \draw [dashed](0,-1.5)--(0,1.5); 
       \draw [ultra thick] plot[domain=-.7:.7, samples=100] function{-.5*sin(3.1416/0.7*x)}  ;
    \end{tikzpicture}

    \caption{}
\end{figure}

示波管是示波器的核心部件.示波器内部的电子线路比较复杂,我们不作具体的研究,我们将在学生实验里初步学习它的使用方法,由于电子的惯性很小,电子流的偏转能几乎立即跟上电压的变化,因此,亮斑在荧光屏上的运动能准确地反映偏转极板上电压的微小的或速的变化.这对于研究各种信号电压的变化情况是很方便的.此外,如振动、光、温度
等的变化,也可以转化成电压的变化,然后用示波器来研究.示波器已经成了科学研究、检测和修理各种电子仪器不可缺少的工具.


\subsection*{练习三}
\begin{enumerate}
    \item 放电管里气体的自激放电是怎样形成的?
    \item 为什么安装了避雷针能够避免雷击?
    \item 示波管荧光屏上显示出的一条完整的正弦曲线(图8.17),是在什么条件下得到的?这时,如果使扫描电压的周期增大为原来的二倍,那么,荧光屏上将显示出什么样的图线?
\end{enumerate}


\section{半导体的导电性}
有一类物质,它的导电能力介于导体和绝缘体之间,称为半导体,如硅、锗、氧化亚铜、砷化镓等都是.用半导体制成的各种器件有着极其广泛的应用,从日常生活到现代通讯设备、电子计算机、空间技术等都离不开它.半导体对于科学技术的发展已经产生了十分深远的影响,半导体所以有如此重要的作用,在于它有着自己特殊的导电性质,下面我们用半导体硅为例来说明.

硅是四价元素,硅原子最外层有四个价电子,在硅的单晶体中,硅原子的排列是很有规律的,每个原子都以四个价电子与相邻的四个原子联系着,这样,相邻的两个原于就有一对共有电子,形成共价键.图8.18是硅单晶的共价键结构
的平面示意图,处子共价键中的电子是一种束缚电子,不能
自由移动.但是,由于热运动,其中极少数电子有可能获得足够的能量,挣脱束缚成为自由电子.在外电场作用下,这些自由电子就逆着电场方向定向移动形成电流,这是半导体的电子导电.
\begin{figure}[htp]
    \centering
    \begin{minipage}[t]{0.48\textwidth}
    \centering
    \includegraphics[scale=.6]{fig/8-18.png}
    \caption{}
    \end{minipage}
    \begin{minipage}[t]{0.48\textwidth}
    \centering
    \includegraphics[scale=.6]{fig/8-19.png}
    \caption{}
    \end{minipage}
    \end{figure}

当共价键中的一个束缚电子挣脱为自由电子时,在原来共价键中就留下一个空位,叫做\textbf{空穴}(图8.19).我们知道,原子是电中性的,空穴是失去了带负电的电子而形成的,因而可以把空穴看作是带正电的.这个空穴很容易由附近共价键中的束缚电子来填补,于是又出现了一个新的空穴.束缚电子的这种填补运动,从效果上看相当于空穴向着相反方向的运动,为了跟自由电子的移动相区别,把束缚电子的这种填补运动叫做空穴运动,我们可以打个比方,比如大礼堂里坐满了人,第一排走了一人,出现一个空位,第二排的人向前坐了第一排的空位,第二排又出现一个新的空位,后排的人陆续向前填补.这样就出现了人的从后往前的填补运动,从效果上看好象空位向后运动一样.在外电场作用下,空穴顺着电
场方向定向移动形成电流,这叫做半导体的\textbf{空穴导电}.

在纯净的半导体中,自由电子和空穴是成对出现的,叫做电子-空穴对.自由电子和空穴也会重新结合,叫做复合.

可见,半导体导电与金属导电的情况不同,金属只有电子导电,而半导体既有电子导电又有空穴导电.金属虽然只有电子导电,但它的自由电子数量很多,导电性好;纯净的半导体虽然有电子导电和空穴导电,但它的自由电子和空穴的数量很少,导电性差.

然而,在某些情况下,半导体也可以产生较多的自由电子和空穴,从而使它的导电性发生明显的改善.例如,半导体的温度升高或受到光照时,就会使共价键中更多的电子挣脱束缚,形成更多的电子-空穴对,大大增强了它的导电性,这就是半导体的热敏特性和光敏特性,利用半导体的热敏特性和光敏特性可以制成热敏电阻(它的阻值随着温度的高低而明显地变化)和光敏电阻(它的阻值随着光照的强弱而明显地变化).热敏电阻和光敏电阻在测温和自动控制中有着重要的应用.

\section{N型半导体和P型半导体}
在纯净的半导体中掺入微量的杂质,会使它的导电性大大增强.掺入的杂质有两类,因而形成两类半导体.

一类是在纯净的半导体中掺入比它的价电子多的杂质.
例如在硅中掺入微量的五价元素磷P(或砷As、锑Sb),一些硅原子就会被磷原子代替,磷原子有五个价电子,它与周围
硅原子组成共价键时多出一个价电子.这个价电子受磷原子的束缚很弱,很容易成为自由电子(图8.20).磷原子失去一个价电子后成为正离子.在这类半导体中,每个五价原子都能提供一个自由电子,因而自由电子的数目显著增多.当然,由于热运动也会产生电子-空穴对,这类半导体中也有少数的空穴,但自由电子的浓度要比空穴的浓度大得多.这类半导体主要靠电子导电,叫做电子型半导体或\textbf{N型半导体}.
\begin{figure}[htp]
    \centering
    \begin{minipage}[t]{0.48\textwidth}
    \centering
    \includegraphics[scale=.6]{fig/8-20.png}
    \caption{}
    \end{minipage}
    \begin{minipage}[t]{0.48\textwidth}
    \centering
    \includegraphics[scale=.6]{fig/8-21.png}
    \caption{}
    \end{minipage}
    \end{figure}

另一类是在纯净的半导体中掺入比它的价电子少的杂质.例如在硅中掺入微量的三价元素铟In(或铝Al、镓Ga、硼B),一些硅原子就会被铟原子代替.铟原子只有三个价电子,它与周围硅原子组成共价键时缺少一个电子,附近共价键中的电子很容易前来填补,从而形成一个空穴(图8.21).铟原子得到一个电子后成为负离子.在这类半导体中,每个三
价原子都能提供一个空穴,因而空穴的数目显著增多.当然,由于热运动会产生电子-空穴对,这类半导体中也有少数自由电子,但是空穴的浓度要比自由电子的浓度大得多、这
类半导体主要靠空穴导电,叫做空穴型半导体或\textbf{P型半导体}.

\section{PN结~~晶体二极管}
由上节可知,在纯净的半导体中掺入不同的微量杂质,可以使纯净的半导体成为N型半导体或P型半导体.用不同的掺杂工艺,也可以使一块纯净的半导体的一边成为N型半导体,另一边成为P型半导体,这时在N型和P型的交界处形成一个具有特殊性质的区域,叫做PN结.正是由于PN结的形成,才产生了各种类型的半导体器件;PN结是组成晶体二极管、晶体三极管及其他半导体器件的基础.
\begin{figure}[htp]\centering
    \includegraphics[scale=.6]{fig/8-22.png}
    \caption{}
    \end{figure}

现在我们讨论PN结的形成.我们知道,N型半导体中
自由电子的浓度大,P型半导体中空穴的浓度大.把N型和P型半导体结合在一起时,在它们的交界面处,自由电子要从N区向P区扩散并与P区的空穴复合,空穴要从P区向N区扩散并与N区的自由电子复合(图8.22).

扩散之前,N型和P型半导体都是电中性的,在交界面处没有电场,扩散开始后,在靠近交界面处,N区一边由于跑掉
了自由电子而留下了带正电的离子,P区一边由于跑掉了空穴而留下了带负电的离子.这些不能移动的带电离子集中在交界面附近,形成了一个电场,其方向从带正电的N区指向带负电的P区.显然,这个电场的作用是阻止自由电子向P区和空穴向N区进行扩散.因此,N型和P型半导体交界面处附近的带电离子区叫做阻挡层,它形成的电场叫做阻挡层电场(图8.23).随着扩散运动的进行,阻挡层电场增强,扩散运动减弱,最后达到稳定状态,在N型和P型半导体交界面处形成一个很薄的阻挡层,这就是\textbf{PN结}.
\begin{figure}[htp]\centering
    \includegraphics[scale=.6]{fig/8-23.png}
    \caption{}
    \end{figure}
\begin{figure}[htp]\centering
\begin{circuitikz}[european, >=stealth, thick]
\draw (0,0) to [R] (5,0);
\node at (2-.3,0)[above]{$+$};\node at (3+.3,0)[above]{$-$};
\draw (2.5,-.18)--(2.5,.18);
\node at (2.2,0){P};\node at (2.8,0){N};

\draw (0,-1) to [empty diode, fill=black] (5,-1);   
\node at (2,-1)[above]{$+$};\node at (3,-1)[above]{$-$};


        \end{circuitikz}
    \caption{}
    \end{figure}
在PN结的N区和P区各引出一个电极,再装上管壳,就成了一个晶体二极管.跟N区相连的是晶体二极管的负极,跟P区相连的是晶体二极管的正极(图8.24).晶体二极管按结构可分为点接触型和面接触型两类,如图8.25所示.
\begin{figure}[htp]\centering
    \includegraphics[scale=.6]{fig/8-25.png}
    \caption{}
    \end{figure}

\begin{figure}[htp]
        \centering
        \begin{tikzpicture}[>=stealth, thick]
    
            \draw (-3,-1)to [battery] (3-3,0-1);
            \draw (0-3,2-1)to [empty diode, fill=black](3-3,2-1)to [lamp] (3-3,0-1);
            \draw (0-3,2-1)--(-3,-1);
\foreach \x in {30,60,...,360}
{
    \draw (\x:.5)--(\x:.75);
}
\node at (-1.5,-2){甲};
\node at (-1.5,1.75){D};

\draw[->] (-2,-.5) arc (250:-60:.5);

        \end{tikzpicture}\qquad\qquad  
        \begin{tikzpicture}[>=stealth, thick]
            \draw (0,0)--(0,2)to [empty diode, fill=black](3,2)to [lamp] (3,0)  to [battery](0,0);
            \node at (1.5,-1){乙}    ;
            \node at (1.5,2.75){D};
                    
        \end{tikzpicture}
        \caption{}
    \end{figure}

把二极管、小灯泡和电池象图8.26甲那样连成电路,这
时小灯泡发光,可是把电池的正负极互相调换,象图8.26乙那样连成电路,小灯泡就不发光了,可见晶体二极管只允许一个方向的电流通过,也就是说,晶体二极管有单向导电性.

现在我们简单地说明一下晶体二极管单向导电性的原因.象图8.27甲那样,当N区接电源的负极,P区接电源的正极时,电源加在晶体管上的外电场方向跟阻挡层的电场方向相反,削弱了阻挡层电场,阻挡层变薄.这时N区的自由电子和P区的空穴可以顺利地通过PN结,在外电场的作用下形成电流,使小灯泡发光.象图8.27乙那样,当N区接电源的正极,P区接电源的负极时,加在晶体管上的外电场方向跟阻挡层的电场方向相同,增强了阻挡层电场,阻挡层变厚.
这时P区的自由电子和N区的空穴可以通过PN结,在外电场的作用下形成反向电流.但是,P区的自由电子和N区的空穴都很少,所以反向电流很小,粗略地可以认为没有电流通过,不能使小灯泡发光,因此晶体二极管表现为单向导电.
\begin{figure}[htp]\centering
    \includegraphics[scale=.6]{fig/8-27.png}
    \caption{}
    \end{figure}

\section{晶体三极管}

\subsection{晶体三极管的结构}

晶体三极管由两个PN结、三个电极线和管壳构成,分为PNP型和NPN型两类,图8.28是PNP型的结构和符号,图8.29是NPN型的结构和符号.
\begin{figure}[htp]\centering
    \includegraphics[scale=.6]{fig/8-28.png}
    \caption{PNP型晶体三极管}
    \end{figure}

    \begin{figure}[htp]\centering
        \includegraphics[scale=.6]{fig/8-29.png}
        \caption{NPN型晶体三极管}
        \end{figure}

从图可以看出,晶体三极管内分三个区:\textbf{发射区、基区、集电区}.它们各有一条电极引线,分别叫\textbf{发射极$e$}、\textbf{基极$b$}和\textbf{集电极$c$}.符号图中的发射极的箭头方向表示电流方向.发射区与基区之间的PN结叫做\textbf{发射结},集电区与基区之间的PN结叫做\textbf{集电结}.

发射区和集电区是同一类型的半导体,但它们并不完全相同,发射区的杂质浓度较大,跟基区的接触面积较小.集电区的杂质浓度较小,跟基区的接触面积较大.基区很薄,杂质浓度更小.由于在结构上有这些特点,晶体三极管(以下简称三极管)并不等于两个二极管的简单组合,而有许多新的性能,最重要的是具有放大作用.

\subsection{三极管的放大作用}

PNP型和NPN型三极管的工作
原理是相同的,我们以PNP 型为例来研究三极管的放大作用.

把一个PNP 型三极管按照图8.30那样连接在电路中,从串联在电路中的毫安表和微安表可以读出通过发射极的电流$I_e$、通过基极的电流$I_b$和通过集电极的电流$I_c$.电路中的$R$是限流电阻,$W$是电位器,当改变电位器$W$的阻值时,基极电流$I_b$就改变,同时集电极电流$I_c$和发射极电流$I_e$也随着改变.为了研究$I_b$、$I_c$、$I_e$之间的关系,我们可以多次改变$W$的阻值,把每次的$I_b$、$I_c$和$I_e$的值都记录下来,下表是某个三极管的测试记录.
\begin{figure}[htp]\centering
    \includegraphics[scale=.6]{fig/8-30.png}
    \caption{}
    \end{figure}

\begin{center}
    \begin{tabular}{ccccccccccc}
        \hline
测量次数  & 1&2&3&4&5&6&7&8&9&10\\
\hline
$I_b$(毫安)& 0.00 & 0.01 & 0.02 & 0.03 & 0.04 & 0.05 &0.10  & 0.20 & 0.30 & 0.50\\ 
$I_c$(毫安)&  0.10  & 0.50 & 0.95 & 1.42 & 1.91 & 2.41 & 4.91 & 9.51 & 13.7 &21.3\\
$I_e$(毫安)& 0.10 & 0.51 & 0.97 & 1.45 & 1.96 & 2.46 & 5.01 & 9.71 & 14.0 & 21.8 \\  
\hline
    \end{tabular}
\end{center}

从测试记录可以看出,发射极电流总是等于基极电流和集电极电流之和:$I_e=I_b+I_c$,而且基极电流远小于集电极电流:$I_b\ll I_c$.这说明发射极电流流进三极管后分成两部分,小部分由基极流出,大部分由集电极流出.三极管的这种电流分配关系可用图8.31形象地表示出来.
\begin{figure}[htp]\centering
    \includegraphics[scale=.6]{fig/8-31.png}
    \caption{}
    \end{figure}

从测试记录还可以看出,当基极电流稍有变化时,集电极电流就有较大的变化.例如,当基极电流$I_b$从0.01毫安变到0.02毫安时,集电极电流就由0.50毫安变到0.95毫安.基极电流的变化量$\Delta I_b$只有0.01毫安,而集电极电流的变化量$\Delta I_c$却有0.45毫安.集电极电流的变化量是基极电流变化量的45倍,三极管的这种作用叫做它的放大作用.

由于三极管有放大作用,因此获得了广泛地应用.三极管的放大作用也可以根据PN结的性质作出解释,本书就不讲了.


\section*{复习题}

\begin{enumerate}
    \item 金属中的电流是怎样形成的?简述用电子论的观点解释欧姆定律.
    \item 电解质中的电流是怎样形成的?电解质导电跟金属导电有什么不同?
    \item 法拉第电解第一定律的内容是什么?法拉第电解第二定律的内容是什么?法拉第恒量的意义是什么?法拉第恒
    量、基本电荷、阿伏伽德罗常数三者之间的关系是怎样的?
    \item 气体导电跟金属导电和电解质导电有什么相同点和不同点?
    \item 什么是气体放电?什么是被激放电和自激放电?简述自激放电的几种形式.
    \item 为什么说阴极射线是从阴极发出的电子流?
    \item 简述示波管的构造和它的工作原理.
    \item 半导体的导电特点是什么?什么是N型半导体?什么是P型半导体?
    \item 什么是PN结?它是怎样形成的?
    \item 为什么晶体二极管有单向导电性?
    \item 什么是晶体三极管的放大作用?
\end{enumerate}







\chapter{学生实验}

\section{验证玻意耳-马略特定律}

在这个实验中,我们用一个带有刻度的注射器近似地验证玻意耳-马略特定律.

实验研究的对象是封闭在注射器里的空气柱,空气柱的体积可由注射器上的刻度直接读出.空气柱的压强$p=p_0\pm \dfrac{F}{S}$,其中$p_0$为大气压强,$F$为活塞对空气柱的压力或拉力,$S$为活塞的横截面积.(考虑一下,哪种情况取正号,哪种情况取负号.)

实验时,先用弹簧秤称出活塞和框架的重量,用刻度尺测出注射器的全部刻度的长度,用这个长度去除它的容积即得活塞的横截面积$S$.记下气压计(全班共用一个)指示的大气压强$p_0$.

将活塞插入注射器内一部分后,将注射器的小孔堵住,以封入一定质量的空气.

把注射器固定好,在框架的两侧加挂钩码,使空气柱的体积减小(图9.1).改变钩码的个数,再做两次,记下每次加挂的钩码数和相应的空气柱的体积.
\begin{figure}[htp]
    \centering
    \begin{minipage}[t]{0.48\textwidth}
    \centering
    \includegraphics[scale=.6]{fig/9-1.png}
    \caption{}
    \end{minipage}
    \begin{minipage}[t]{0.48\textwidth}
    \centering
    \includegraphics[scale=.6]{fig/9-2.png}
    \caption{}
    \end{minipage}
    \end{figure}

然后,用弹簧秤钩住活塞框架的上边,慢慢竖直提起活
塞,使空气柱的体积增大(图9.2),记下三组每提到一定高度时弹簧秤的读数和相应的空气柱的体积.

把记录的数据填入自己设计的表格里,根据公式$p=p_0\pm \dfrac{F}{S}$算出各个压强值(要注意活塞和框架的重量对压强的影响),求出各个压强$p$跟相应的体积$V$的乘积.比较这些乘积,能得出什么结论?

\section{验证气体状态方程}
这个实验,我们利用实验一的装置来验证气体状态方程.

跟实验一一样,先用弹簧秤称出活塞和框架的重量;测出注射器的全部刻度的长度,求出活塞的横截面积$S$;记下这时的大气压强$p_0$.在注射器内封入一定质量的空气.
\begin{figure}[htp]\centering
    \includegraphics[scale=.6]{fig/9-3.png}
    \caption{}
    \end{figure}

照图9.3那样,固定好注射器和烧杯.在活塞框架的两侧加挂钩码,用公式$p=p_0+ \dfrac{F}{S}$计算出空气柱的压强(注意压力$F$中应包括活塞和框架的重
量).向烧杯里倒入适量的水,使注射器内的空气柱位于水面之下.经过两分钟左右,用温度计测出水的温度$t$,可以认为这个温度就是空气柱的温度,把它换算成热力学温度$T$.记下这时空气柱的体积$V$.

改变加挂的钩码数和烧杯中水的温度,再分别做四次上面的实验.

把上面得到的数据填入自己设计的表格里,并算出每次
实验得到的$pV/T$的值,看看它们是否相等,从实验可以得出什么结论?

\section{测定冰的熔解热}

单位质量的某种物质熔解成同温度的液体时吸收的热量,叫做这种物质的熔解热.在这个实验里我们利用量热器用混合法来测定冰的熔解热.

设有$m_{\text{冰}}$克的冰,温度为0$^\circ$C,与$m_{\text{水}}$克温度为$t_0$的水混合,冰全部熔解跟水混合以后的平衡温度为$t$.$m_{\text{冰}}$克冰熔解成水并升高到温度$t$吸收的热量,等于$m_{\text{水}}$克水以及盛水容器从温度$t_0$下降到温度$t$放出的热量,即
\[m_{\text{冰}}\lambda+m_{\text{冰}}c_{\text{水}}t=(m_{\text{水}}c_{\text{水}}+m_{\text{筒}}c_{\text{筒}})(t_0-t)  \]
其中,$\lambda$为冰的熔解热,$c_{\text{水}}$为水的比热,$c_{\text{筒}}$为容器的比热,$m_{\text{筒}}$为容器的质量,这样,把$c_{\text{水}}$和$c_{\text{筒}}$作为已知量,$m_{\text{冰}}$、$m_{\text{水}}$、$m_{\text{筒}}$和$t_0$、$t$都可以由实验获得,从而利用上式求出冰的熔解热$\lambda$.

实验开始时,先用天平称出量热器小筒的质量$m_{\text{筒}}$(包括搅拌器),然后把比室温高10—15$^\circ$C的温水(150克左右)倒入量热器小筒,再称出水和小筒的质量,算出水的质量$m_{\text{水}}$.把装着水的量热器小筒放在大筒的木架上,用温度计测出水和量热器小筒的初温$t_0$.把准备好的温度为0$^\circ$C的冰块\footnote{实验室里,冰水混合物的温度可以认为是0$^\circ$C.}(20克左右)迅速放入小筒的水中,并盖好量热器盖子,搅动小筒中的水,同时观察插入量热器里的温度计,当温度下降到最低
时,记录下来的温度$t$就是冰、水混合后的平衡温度,最后再称量一下量热器小筒和水的质量(其中包括冰的质量),算出冰的质量$m_{\text{冰}}$.

把由实验得到的数据代入第二段中的公式,求出冰的熔
解热,水的比热$c_{\text{水}}$,可取为$4.2\x10^3{\rm J}/({\rm kg}\cdot {}^\circ{\rm C})$,铝制小筒的比热$c_{\text{筒}}$可取为$8.9\x10^2 {\rm J}/({\rm kg}\cdot {}^\circ{\rm C})$,铜制小筒的比热$c_{\text{筒}}$可取为$3.9\x10^2{\rm J}/({\rm kg}\cdot {}^\circ{\rm C})$.

实验中要注意读准温度计的示数,冰块不宜太大,为什么?在这个实验中,误差的主要来源是什么?

\section{测定空气的相对湿度}
这个实验我们学习测定空气的相对湿度,实验装置如图9.4所示.圆柱形金属盒的一个底面十分光亮,侧面有开口,开口旁边有一小孔,用来插入温度计,环形金属片套在金属盒上,它的一面也是十分光亮,井与金属盒的光亮面在同一平面内,金属盒和环形片用胶木垫圈隔开,防止相互间的热传导,搅拌器插在开口中.
\begin{figure}[htp]\centering
    \includegraphics[scale=.6]{fig/9-4.png}
    \caption{}
    \end{figure}

实验时,先记录下实验室的温度,用柔软的绒布仔细地把金属盒和环形片的光亮面擦得十分干净,在金属盒里注入约
半盒室温的水,再向水里投入适量的碎冰块(注意不要沾污光亮面),装上温度计,并使它的刻度向着金属盒的光亮面,搅动冰块,使水温迅速下降,同时密切注视金属盒和环形片的光亮面,当金属盒的光亮面上刚刚出现细小的露滴时,记录下这一瞬间盒里水的温度,等水的温度又开始上升,金属盒光亮面
上的细小的露滴完全消失时,再记录下这一瞬间的温度,两次温度的平均值就是露点.

从课本里查出温度为测得的露点时水的饱和汽压$p$,这就是测量时空气中水蒸气的压强,即空气的绝对湿度,再查出测量时的室温下水的饱和汽压$P$.此时的相对湿度就是
\[B=\frac{p}{P}\times 100\%\]

\section{电场中等势线的描绘}
在这个实验里,我们学习用描迹法画出电场中平面上的等势线.

如图9.5所示,在平整的木板上铺一张白纸,白纸上依次铺放复写纸和导电纸,导电纸有导电物质的一面向上,白纸、复写纸和导电纸一起用图钉固定在木板上,导电纸上平放着跟它接触良好的两个圆柱形电极,电极A与电源的正极相连作为正电荷,电极$B$与电源的负极相连作为负电荷\footnote{我们在第六章学习的是静电场.直接接绘静电场中的等势线是相当困难的,由于静电场和稳恒电流场遵守的规律相似,这里是用在导电纸上形成的稳恒电流场模拟静电场来做实验.}.两电极之间的距离约为10厘米,电压约为6伏.
\begin{figure}[htp]\centering
    \includegraphics[scale=.6]{fig/9-5.png}
    \caption{}
    \end{figure}

现在,我们来描绘正、负电荷$A$、$B$在纸面上的等势线.从灵敏电流表的两个接线柱引出两个探针,用来探测导电纸上的等势点.先在导电纸平面两电极的连线上,选取间距大致相等的五个点作为基准点,并用探针把它们的位置复印在白纸上,在某一基准点将一个探针跟导电纸相接触,然后在导电纸平面上两电极连线的一侧,距此基准点约1厘米处再选一个点,在此点将另一探针跟导电纸相接触,一般这时会看
到电流表的指针有偏转,左右移动另一探针的位置,直到找到一点,使电流表的指针没有偏转为止,电流表的指针没有偏转,说明这个点跟基准点的电势相等,用探针把这个点的位置复印在白纸上,照上述方法,在这个基准点的两侧,各探测出五个等势点,每个等势点大约相距1厘米.用同样的方法,探测出另外四个基准点的等势点.最后,取出白纸.根据五组等势点画出五条平滑的曲线,它们就是等势线,你能不能根据这些等势线在白纸上画出两个异种电荷的电力线:画一画看.

\section{利用电容器放电测电容}
现在,我们通过实验来学习一种测量电容器电容的简单方法.

我们知道,电容器的电容等于电容器所带电量跟两极之
因此,如果测量出某一电压下间的电势差的比值,即$C=Q/U$.因此,如果测量出某一电压下
电容器所带的电量,就可以求出电容器的电容.怎样才能测量出电容器所带的电量呢?
\begin{figure}[htp]
    \centering
    \begin{circuitikz}[european, scale=1.5, >=stealth]
\draw (0,0) --(2,0)--(3,0)--(5,0)to [rmeter, t=$\mu{\rm A}$] (6,0) to [R=$R$] (6,2)--(5,2) ;
\draw (3,2)--(2.5,2);
\draw[->](2.5,2)--(2.5,1)--(2.15,1);

\draw (2,0) to [R=$R'$, *-](2,2) ;\draw (0,2) to [battery2](0,0);
\draw (3,2) to [cute open switch] (5,2);
\draw(0,2) to  [cute open switch] (2,2);
\node at (4,2)[below]{$K$}; 
\node at (1,2)[below]{$K'$}; 

\draw (4.75,2) to [ecapacitor=$C$, *-*] (4.75,0); 
\draw (3.25,0) to [rmeter, t=V, *-*] (3.25,2); 
    \end{circuitikz}    
    \caption{}
\end{figure}

测量电路如图9.6所示,合上电键$K'$、$K$,对电容器$C$充电.当电容器两端电压$U_c$上升到某一稳定电压$U$时,充电完毕.然后将电键$K$打开,这时容器通过电阻$R$放电,放电电流$i_c$随时间$t$的增加而逐渐减小,放电完毕时$i_c=0$.在电容器放电过程中,如果在某一时刻的放电电流为$i_c$,那么在一小段时间间隔$\Delta t$里,从电容器正极转移到负极上的电量就等于$i_c\Delta t$,将整个放电过程中每小段时间所转移的电量加起来,就得到电容器所带的电量$Q$.

按图9.6接好线路,电源可用学生电源,电容器$C$可选用
470微法的电解电容器,微安表可选用500微安量程的,$R$用27
千欧的定值电阻,接线时要注意电解电容器的极性不要接反.接通电源后,先合上电键$K$,调节变阻器$R'$使伏特表指示到实验需要的电压值12伏,然后合上电键$K$,给电容器充电,充电完毕,记下这时伏特表和微安表的读数.把电键$K$打开,同时开始计时,并且每间隔5秒钟读取一次微安表的电流值,直到电流值减至零为止.

根据记录的数据,在坐标纸上,以时间$t$为横坐标,以电流$i_c$为纵坐标作出$i_c$-$t$图象,然后再根据所画的$i_c$-$t$图象,求出电容器所带电量$Q$(同学们思考一下,怎样利用$i_c$-$t$图象求出电量$Q$),最后计算出电容器的电容.

\section{测定金属的电阻率}
这个实验是测定金属的电阻率.

电阻定律告诉我们,用电阻率为$\rho$制成的长$\ell{\rm m}$、横截面积$S{\rm m^2}$的导线的电阻
\[R=\rho\frac{\ell}{S}\]

因此,测出一段导线的长度和直径(由直径可以算出横截面积)以及这段导线的电阻,就可以求出制成这段导线的材料的电阻率.

现在有一段长约0.5米、直径约0.3毫米,阻值约3欧姆的金属导线,你应当选用哪些实验器材来测定它的电阻率?考虑一下,这个实验应当怎样进行?通过实验,你测得制成这段
导线的材料的电阻率是多少?

需要注意的是,在给导线通电时,电流不宜太大,想想看,这是为什么?

\section{把电流表改装为伏特表}
我们学习了把电流表改装为安培表和伏特表的原理,在这个实验里,我们练习把电流表改装为伏特表.

改装电流表,需要知道它的三个数据:满度电流$I_g$,满度电压$U_g$(电流表的指针偏转到满刻度时加在表头上的电压)和内电阻$r_g$.这三个数据中,知道任何两个,就可以根据欧姆定律算出第三个,电流表的$I_g$可以从刻度盘上直接读出,$r_g$可用实验方法测出,于是就可以算出$U_g$.
\begin{figure}[htp]
    \centering
    \begin{circuitikz}[european, scale=1, >=stealth]
        \draw (4,0) to[battery2] (3,0)--(0,0)--(0,4)to [rmeter,t=G](4,4)to [R](8,4)--(8,0);
        \draw [->](7.25,4)--(7.25,4.75)--(6,4.75)--(6,4.2);
        \draw (4,0) to[cute open switch] (8,0);
        \node at (6,4-.5){$R$};
        \draw (0,1.5)to [cute open switch] (2,1.5) to [vR=$R'$](4.5,1.5)--(4.5,4);
        \draw (0,1.5) [fill=black] circle (1.2pt);\draw (4.5,4) [fill=black] circle (1.2pt);
        \draw (7.25,4) [fill=black] circle (1.2pt);
        \node at (6,0)[below]{$K_1$};
        \node at (1,1.5)[below]{$K_2$};
    \end{circuitikz}    
    \caption{}
\end{figure}

我们利用图9.7所示的电路来测定电流表的内电阻$r_g$.$R$可用470千欧的电位器\footnote{电位器是一种可以连续调节电阻值的变阻器,常用作分压器.},$R'$是电阻箱,合上电键$K$,调整电位器$R$,使电流表指针偏转到满刻度(要注意,不应使通过电流表的电流超过它的满度电流值,以免把表烧坏),然后再合上电键$K_2$,调整电阻箱$R'$的阻值,使电流表指针偏转到正好是满刻度的一半,当$R$比$R'$大很多时,接入$R$后,干路中电流变化不大,因此可以认为$r_g=R'$.


测出$r_g$后,再计算出电流表的满度电压$U_g$,然后算出把它改装为2伏特的伏特表时,应该串联多大的电阻$R_1$.在电阻箱上取好阻值$R_1$,把电流表跟电阻箱串联起来,就是一个量程是2伏特的伏特表了.
\begin{figure}[htp]
    \centering
    \begin{minipage}[t]{0.48\textwidth}
    \centering
    \begin{circuitikz}[european, scale=.9, >=stealth]
        \draw(0,0)--(0,1)--(4,1) to [R=$R_2$] (6,1)--(6,0) to [cute open switch] (3,0) to [battery2] (0,0);
        \draw (0,1)--(0,3) to [rmeter, t=G] (3,3) to [vR=$R_1$] (5.4,3);
        \draw[->]  (5.4,3)-- (5.4,1.2);
        \draw (0,2)to [rmeter, t=V,*-*] (5.4,2);
        \node at (4.5,0)[above]{$K$};

    \end{circuitikz}
    \caption{}
    \end{minipage}
    \begin{minipage}[t]{0.48\textwidth}
    \centering
    \begin{circuitikz}[european, xscale=1.3, >=stealth]
        \draw(0,0)--(0,1.3) to [rmeter, t=V, *-*] (4,1.3)--(4,0) to [cute open switch] (2,0) to [battery2](.5,0) --(0,0);
\draw(0,1.3)--(0,2.5)--(1,2.5) to [R] (2,2.5) to [rmeter, t=A] (4,2.5)--(4,1.3);
\draw [->](.35,2.5)--(.35,3.2)--(1.5,3.2)--(1.5,2.7);
\node at (1.5,2){$R$};
\node at (2.5/2,-.5){$\mathcal{E}, \; r$};
\node at (3,0)[above]{$K$};
    \end{circuitikz}    
    \caption{}
    \end{minipage}
    \end{figure}
    

最后把改装成的伏特表跟标准伏特表核对一遍.实验电路如图9.8所示,$V$是标准伏特表,改变变阻器$R_2$的触点,使$V$的读数分别为0.5伏特、1伏特、1.5伏特、2伏特时,核对一下改装的伏特表的读数是否正确,核对时要注意搞清楚改装后电流表上刻度的每一小格表示多大电压.最后算出改装的伏特表满刻度时的百分误差.例如改装的伏特表在满刻度2伏特时,标准伏特表的读数为2.1伏特,那么满刻度时的百分误差就是
\[\frac{|2.1-2|}{2.1}=4.8\%\]

\section{用安培表和伏特表测定电池的电动势和内电阻}

这个实验是用安培表和伏特表测出电流和路端电压,再用闭合电路的欧姆定律来求出电动势和内电阻,实验电路如图9.9所示.

我们知道,只要改变$R$的阻值,测出两组$I$、$U$的数据.代入方程组
\[\begin{split}
    \mathcal{E}&=U_1+I_1r\\
    \mathcal{E}&=U_2+I_2r\\
\end{split}\]
就可以求出电动势$\mathcal{E}$和内电阻$r$,这样做在原理上虽然很简单,但偶然误差却很大.

为了减小偶然误差,我们可以多测出几组$I$、$U$的数据,求出几组$\mathcal{E}$、$r$值,最后分别算出它们的平均值.此外,物理实验中还经常用作图法,现在我们就来学习作图法.

利用变阻器$R$测出几组$I$、$U$值后,在坐标纸上以$I$为横坐标,$U$为纵坐标,画出$U$-$I$关系图象,根据闭合电路的欧姆定律,$U=\mathcal{E}-Ir$,因此$U$是$I$的一次函数,它们的图象应该是一条直线,你得出的是不是一条直线?把这条直线延长,使
它跟纵轴相交,这个交点有什么物理意义?在图象中内电阻是怎样表示出来的?你怎样利用自己作出的图象来得到电池的电动势和内电阻?

这里还要作一点说明,作图时要适当选取横坐标、纵坐标的比例和坐标的起点,使实验数据大致布满整个图纸,不要集中在一边或一角.这个实验的$U$值不宜过小,因此纵坐标$U$的起点不要从零开始,而横坐标$I$仍要以零为起点.(为什么?)

\section{练习使用万用电表}
万用电表(常简称为万用表)是一种多用仪表,一般可以用来测量电流、电压、电阻等,并且每一种测量项目有几个量
程,由于万用表具有用途多、量程广、使用方便等优点,因此得到了广泛的应用,这个实验我们来学习万用表的使用.

万用表的型号很多,但使用方法基本相同,下面以J0411型万用表为例来说明它的使用方法和注意事项.
\begin{figure}[htp]\centering
    \includegraphics[scale=.6]{fig/9-10.png}
    \caption{}
    \end{figure}

J0411型万用表的外形如图9.10所示.它的上半部是表
头,表盘上有电阻、电流、电压等各种量程的刻度,有的刻度是均匀的,因此合用一个刻度.下半部是选择开关,它的四周刻着各种测量项目和量程.应当注意,电流和电压又分为直流(用符号“$-$”表示)和交流(用符号“$\sim$”表示),要区别开,不要弄错.另外还有欧姆档的调零旋钮和测试笔插孔.

测量前,应先检查表针是否停在左端的“0”位置,否则,要用小螺丝刀轻轻地转动表盘下边中间的调整定位螺丝,使指针指零.万用表有两根测试笔,将红表笔和黑表笔分别插入正($+$)、负($-$)测试笔插孔.

测量时,应把选择开关旋到相应的项目和量程上,读数时,要看跟选择开关的档位相应的刻度.

测量电流时,跟电流表一样,应把万用表串联在被测电路里;对于直流电,还必须使电流从红表笔流进万用表,从黑表笔流出来.

测量电压时,跟电压表一样,应把万用表和被测部分并联;对于直流电,必须用红表笔接电势较高的点,用黑表笔接电势较低的点.

测量电阻时,在选择好选择开关的档位后,要先把两根表笔相接触,调整欧姆档的调零旋钮,使指针指在电阻刻度的零位上(注意,电阻刻度的零位在表盘的右端),然后再把两表笔分别与待测电阻的两端相接,进行测量.换用欧姆档的另一量程时,需要重新调整欧姆档的调零旋钮,才能进行测量.应当注意,测量电阻时待测电阻要跟别的元件和电源断开.(为什么?)

测量时,注意手不要碰到表笔的金属触针,以保证安全和测量的准确;使用后,要把表笔从测试笔插孔拔出,井且不要把选择开关置于欧姆档,以防电池漏电;长期不使用时,应把电池取出.

在了解了你使用的万用表之后,就可以按照老师的要求,
来进行电流、电压和电阻的测量了.

\section{用惠斯通电桥测电阻}

这个实验是用滑线式电桥来测电阻.
\begin{figure}[htp]
    \centering
    \begin{circuitikz}[european, >=stealth]
\draw (-3,0)node[left]{$A$} to [vR=$R_0$,*-*](0,3)node[above]{$B$}to [R=$R_x$,*-*](3,0)node[right]{$C$}--(-3,0);
\draw (3,0)-- (3,-2) to [cute open switch] (1,-2)to [battery2] (-1,-2) to [R=$R$] (-3,-2)--(-3,0);
\draw[->](-.5,-2) --(-.5,-1.5)--(-2,-1.5)--(-2,-2+.2);
\draw (0,3) to [rmeter, t=G] (0,.5);
\draw[->, thick] (0,.5)--(-.7,.5)--(-.7,0)node [below]{$D$};
\node at (-1.85,0)[below]{$\ell_1$};
\node at (1.15,0)[below]{$\ell_2$};
\node at (2,-2)[above]{$K$};
    \end{circuitikz}    
    \caption{}
\end{figure}

实验电路如图9.11所示,其中$R_x$是待测电阻,$R_0$是作已知电阻用的电阻箱,$G$是灵敏电流表,按图接好电路后,先把变阻器$R$调到阻值较大的位置,然后进行实验.

根据误差理论,触头$D$在$AC$中点附近电桥平衡时实验误
差较小(这个道理在这里就不讲了),我们先用万用表测出$R_x$的大约值,在电阻箱上选取跟它接近的某一阻值$R_0$.合上电键$K$,把滑动触头$D$移到电阻线$AC$中点附近業一位置,瞬时按下触头,一般会看到电流表的指针有偏转,稍稍移动触头,再把它瞬时按下,比较电流表指针两次偏转的情况.根据指针偏转的方向是否相同和偏角是增大还是减小,你应该能判断出应向哪个方向移动触头才能使电桥平衡.继续移动触头直到电桥平衡,电流表的指针不再偏转为止.要注意,每次按下触头的时间要尽量短,用力不要过大,更不要在按下触头后又设法移动它.

电桥平衡后,打开电键$K$,读出或量出$AD$的长度$\ell_1$和$DC$的长度$\ell_2$,根据$R_0/R_x=\ell_1/\ell_2$求出$R_x$,这就初步测出了$R_x$的值.

现在来进一步更精确地测定$R_x$.先在电阻箱上取跟初步测出的$R_x$相近的阻值,重新使电桥平衡,然后逐步减少变阻器$R$的阻值,以增大$AC$间的电压,但要注意通过电阻线$AC$
的电流不能超过它的允许值,可以看到,每当$R$的阻值减少后,按下触头$D$时电桥又可能不平衡了,每次都要重新调整触头$D$的位置,才能使电桥恢复平衡.同学们想想看,这是什么道理.要注意这时每次都只能微调触头$D$的位置,以免烧毁电流表,当$R$的阻值减小到一定程度时,使电桥平衡,然后读出或量出$l_1$和$l_2$,利用公式算出$R_x$.为什么现在求出的$R_x$比初测的$R_x$精确?

\section{测定铜的电化当量}
在这个实验里,我们根据法拉第电解第一定律$m=kIt$,测出$m$、$I$和$t$的值,从而确定电化当量$k$.
\begin{figure}[htp]\centering
\begin{circuitikz}[>=latex]
\draw  (3,0) to [cute open switch] (2,0) to [battery2] (0,0) --(0,4)--(3,4)--(3,3);
\draw (6,0)to [rmeter, t=A] (6,3.5)--(1.5,3.5)  ;
\draw (1.5,3.5)--(1.5,3);
\draw (3,3.5)--(3,3);
\draw (4.5,3.5)--(4.5,3);   
\draw (6,0) to [european, R=$R$]  (3,0) ;
\fill [blue!40] (.75, 1.5) rectangle (5.25,2.8);

\draw [<-](4.5,.2)--(4.5,.75)--(5.5,.75)--(5.5, 0);
\draw (1.4,2) rectangle (1.6,3);
\draw (2.9,2) rectangle (3.1,3);
\draw (4.4,2) rectangle (4.6,3);
\draw [very thick](.75,3.2)--(.75, 1.5)--(5.25,1.5)--(5.25,3.2);
\node at (1.4,3)[left]{$+$};
\node at (3.1,3)[right]{$-$};
\node at (4.6,3)[right]{$+$};
\node at (2.5,.25){$K$};


    \end{circuitikz}
    \caption{}
    \end{figure}

准备三块铜片,两块作为阳极,一块作为阴极,并用细砂纸把铜片擦干净,用天平仔细称量作为阴极的铜片的质量.把铜片放入盛有硫酸铜溶液的电解槽内,按照图9.12接好电路(注意电源和安培表的正负端不要接错),合上电键$K$,调节变阻器$R$使安培表的读数为2安培左右,并开始计时.25—30分钟后,打开电键$K$,停止电解.注意要在整个电解过程中,调节变阻器使电流强度保持不变.电解结束后,取出电极,用酒精灯把阴极板烘干,再用天平仔细称量出这时阴极板的质量,比较两次称量的阴极板的质量,就可以得到电解过程中在阴极板上析出的铜的质量$m$.把$m$、$I$和$t$带入法拉第电解第一定律公式,算出铜的电化当量.

你测定的铜的电化当量是多少?跟课本上给出的数值相差多少?考虑一下,实验误差的主要原因是什么?应当怎样改进这个实验?

\section{练习使用示波器}
示波器是一种常用的电子仪器,它的核心部分是一只示波管,利用它能够直接观察电信号随时间而变化的情况,并且可以测量电压和电流.我们现在初步学习一下示波器的使用方法,在后面的实验里还要多次用到它.

示次器的型号很多,使用方法基本相同,下面以J2459型示波器(图9.13)为例来说明.
\begin{figure}[htp]\centering
    \includegraphics[scale=.4]{fig/9-13.png}
    \caption{J2459型示波器的面板}
    \end{figure}

我们先来了解示波器面板上各个旋纽和开关的名称和作用.荧光屏右边最上端的旋钮是辉度调节“$\star$”,用来调节图象的亮度,顺时针旋转时亮度逐渐增大,它下面的旋钮是聚焦调节“$\odot$”和辅助聚焦“$\textcircled{~}$”,这两个旋钮配合着使用,能使电子射线会聚,在荧光屏上产生一个小的亮斑,得到清晰的图象,再下面是电源开关和指示灯,用后盖板上的电源插座接通电源后,把开关板向“开”的位置,指示灯亮,经过一两分钟
的预热,示波器就可以使用了.

荧光屏下边第一行左右两端的旋钮是竖直位移“$\uparrow\downarrow$”和水平位移“$\leftrightarrows$”,分别用来调整图象在竖直方向和水平方向的位置.它们中间的两个旋钮是“Y增益”和“X增益”,分别用来调整图象在竖直方向和水平方向的幅度,顺时针旋转时,幅
度连续增大.

中间一行左边的大旋钮是“衰减”,它有1, 10、100、1000四档,最左边的“1”档不衰减,其余各档分别使输入电压衰减为
原来的1/10、1/100、1/1000,因此图象在竖直方向的幅度都减小为前
一档的十分之一;最右边的正弦符号“~”档不是衰减,而是由
示波器内部自行提供竖直方向的交流试验信号电压,可用来观察正弦波形或检查示波器是否正常工作.右边的大旋钮是“扫描范围”,也有四档,可以改变加在水平方向的扫描电压的频率范围,左边第一档是10—100赫兹,向右旋转每升高一档,扫描频率都增大10倍,最右边的是“外X”档,使用这一档时,机内没有加扫描电压,水平方向的电压可以从外部输入.中间的小旋钮是“扫描微调”,用来调整水平方向的扫描频率,顺时针转动时频率连续增加.

底下一行中间的旋钮“Y输入”、“X输入”和“地”分别是竖直方向、水平方向和公共接地的输入接线柱.左边的“DC、AC”是竖直方向输入信号的直流、交流选择开关,置于“DC”位置时,所加的信号电压是直接输入的;置于“AC”位置时,所加的信号电压是通过一个电容器输入的,它可以让交流信号通过而隔断直流成分,右边的“同步”也是一个选择开关,置于“$+$”位置时,扫描由被测信号正半周起同步,置于“$-$”位置时,扫描由负半周起同步.

现在,我们来练习使用示波器.先把辉度调节旋钮反时针转到底,竖直位移和水平位移旋钮旋转到中间位置,衰减旋钮置于最高档,扫描范围旋钮置于“外X”档,打开电源开关,
指示灯亮,经预热后,顺时针旋转辉度调节旋钮,屏上即出现一个亮斑,亮斑的亮度要适中,注意不应使亮斑过亮,特别
是当亮斑长时间停留在屏上不动时,应把亮度减弱,以免损伤荧光屏,减少示波管的使用寿命,旋转聚焦调节和辅助聚焦旋钮,观察亮斑的变化,使亮斑最圆最小,旋转竖直位移旋钮,观察亮斑的上下移动,旋转水平位移旋钮,观察亮斑的左右移动.

把X增益旋钮顺时针转到三分之一处,扫描微调旋钮反时针转到底,扫描范围旋钮置于最低档.可以看到扫描的情形:亮斑从左向右移动,到右端后又很快回到左端,顺时针旋转扫描微调以增大扫描频率,可以看到亮斑迅速移动成为一条亮线,调整X增益,可以看到亮线长度的改变.

现在给竖直方向加一个直流电压.事先把扫描范围旋钮置于“外X”档,使亮斑位于屏的中心,把“DC、AC”开关置于“DC”位置,照图9.14连接电路,直流电源用一、二节干电池即可,逐步减小衰减档,观察亮斑的向上偏移,再调整Y增益使亮斑偏移一段适当的距离,调整变阻器改变输入电压,可以看到亮斑的偏移随着改变,电压越高,偏移越大,调换电池的正负极,改变输入电压的方向,可以看到亮斑改为向下偏移.
\begin{figure}[htp]\centering
    \begin{circuitikz}[>=latex]
\draw(0,0) to [cute open switch] (2,0) to [battery2](3,0) --(4,0)--(4,4);
\draw(0,0)--(0,3) to [european, R] (3,3)--(3,0);
\draw[<-]  (1.5,3.2)--(1.5,4)--(3.5,4);   
\draw (4,4)[fill=white] circle (1.5pt)node[right]{地};
\draw (3.5,4)[fill=white] circle (1.5pt)node[above]{Y输入};
\draw (3.7,5.5) circle (.3);
\draw (3,3.5) rectangle (4.5, 6);
    \end{circuitikz}
    \caption{}
    \end{figure}

亮斑偏移的距离跟输入的电压成正比,因而利用示波器
能够测量电压,J2459型示波器的竖直位移已经校准,当衰减旋钮处在“1”的位置,Y增益旋钮顺时针转到底时,如果输入电压为50毫伏,则亮斑恰好偏移1格.这样,我们就可以根据亮斑偏移的格数来算出输入的电压值,测量时要注意把Y增益旋钮顺时针转到底;衰减旋钮处在10、100或1000档时,算出的电压值应乘以相应的倍数,现在来测一节干电池的电压,你测出的数值是多少?

利用示波器还可以测量电流.把一个已知阻值的小电阻串联在待测电流的电路里(或利用原电路中的已知电阻),用示波器测量这个电阻两端的电压,利用欧姆定律就可以算出电路中的电流.这种测量我们就不做了.

实验完了,关机前要注意把辉度调节旋钮反时针方向转到底,使亮度减到最小.



\backmatter
\chapter{课外实验活动}
\setcounter{section}{0}
\section{观察扩散现象}

在容器里装一半水,然后用长颈漏斗小心地把硫酸铜溶液倒进容器的底部,硫酸铜溶液比水的密度大,沉在容器下部,可以看到无色的水和蓝色的硫酸铜溶液的界面非常清楚.由于扩散,经过几天以后,界面上方的水变蓝,界面下方的溶液的颜色变浅,界面变得模糊了,经过更长的时间,全部溶液将变成均匀一致的颜色.把装有水和硫酸铜溶液的容器放在不容易碰到的地方,每隔一天,按时观察现象,看看要多长时间全部液体才变成均匀一致的颜色.

\section{自制冰淇淋}
冰和盐的混合物的熔点可以降到零下二十多摄氏度,因
此可以用冰盐混合物进行冷却,现在我们用冰盐混合物致冷来自制冰淇淋.

制冰淇淋需要的原料有牛奶、鸡蛋、玉米粉、糖以及其他香料,把配好的原料\footnote{原料的配法:把生鸡蛋打开放在碗里,加入少许玉米粉搅匀.然后慢慢冲入煮沸的加糖牛奶并且要边冲边搅.最后,放入香料,再用锅稍稍熬一会儿,冷却后即成.}放在小铝锅里.找一个大容器(如一只大锅),里面装冰盐混合物(大致是三份冰,一份盐的比例)或雪盐混合物,把小铝锅放在冰盘混合物中,只有锅盖露在外面.过一些时间打开锅盖,看看是否快要冻结.当快要冻结时要
搅拌,否则结成的冰粒很粗,不好吃;也不要冻结得太厉害,因为冻得太硬,也不好吃.

\section{人造云雾}
用冰盘混合物致冷可以人工造成云雾.找一个小铁罐(如一个罐头盒子),放在冰盐混合物或雪盐混合物中,小铁罐里的空气很快就冷却,对着小铁罐吹口气,水蒸气就被带进铁罐里,由于里面温度很低,水蒸气凝结成小水滴,这样就造出了淡淡的云雾,用手电筒照一下小铁罐里面,就可以看到人造的云雾.

\section{测定水的汽化热}
用下述办法可以粗略地测定水的汽化热,在铝锅里装一些室温($\theta_1{}^\circ{\rm C}$)下的水,将铝锅放在炉子上加热.设经过时间$t_1$水达到沸点$\theta_2{}^\circ{\rm C}$,再经过时间$t_2$水全部汽化完.已知水的比热$c=4.2\x10^3{\rm J}/({\rm kg}\cdot {}^\circ{\rm C})$,由下式即可求出水的汽化热:
\[L=c(\theta_2-\theta_1)\frac{t_2}{t_1} \; {\rm J}/{\rm kg}  \]

上式是这样得到的:设锅里的水每秒吸收的热量为$g$焦,把室温的水加热到沸点所需的热量$Q_1$和水全部汽化所需的热量$Q_2$分别为
\[\begin{split}
    Q_1&=mc(\theta_2-\theta_1)=gt_1\\
Q_2&=mL=gt_2
\end{split}\]
将上面两式相除,得到
\[\frac{c(\theta_2-\theta_1)}{L}=\frac{t_2}{t_1}  \]
由此即可求得$L$.

用这个办法只能粗略地测定水的汽化热.误差的主要来
源是什么?

\section{估计水升高的温度}
有一个直径约为1厘米的小玻璃管,装入10毫升室温的水.点燃一根火柴,放在小玻璃管底部给水加热.待这根火柴燃烧到尽头时,你能估计出小玻璃管里的水上升了多少摄氏度吗?例如是上升百分之几摄氏度,还是十分之几摄氏度、几摄氏度、十几摄氏度、二十几摄氏度…….你估计的根据是什么?做一做看,水的温度实际上升了多少摄氏度?跟你估计的是否一致?不一致的原因在哪里?

\section{用自制的验电器做静电实验}
如图9.15所示,是一个简易验电器,金属丝对折后穿过瓶盖插入透明的玻璃瓶里,取两条约长2厘米、宽4毫米的金属箔,分别挂在金属丝的两端.两金属箔不带电时自由下垂,带电时互相推斥而张开.
\begin{figure}[htp]\centering
    \includegraphics[scale=.6]{fig/9-15.png}
    \caption{}
    \end{figure}

照上面说的那样,自己做
一个验电器,要注意,应把瓶盖
擦干净,不能潮湿,以保证它有良好的绝缘性能.利用你自制的验电器,以及塑料尺子、金属小筒、金属网等器材,做一做摩擦起电、感应起电、导体上的电荷分布、静电屏蔽等实验.

\section{自制电池}
把不同的金属放在电解质溶液中,由于化学作用,产生了电动势,就成为一个化学电源.现在把同样大小的四、五个光洁的锌片和钢片交替相叠,并在锌片和钢片之间夹一片吸满盐水的吸水纸,这样就做成了一个简单的电池,锌片和锌片相连是电池的负极,铜片和铜片相连是电池的正极.

用细漆包线在指南针外壳上绕50圈,当漆包线中有电流通过时,磁针发生偏转,这样用指南针就可以检验有无电流.

把漆包线两端的漆刮掉,露出铜丝,分别接在自制电池组的两极上,检验导线中有无电流通过.

\section{研究电灯泡的电阻}
取一只电灯泡,看一看它上面标明的额定功率和额定电压是多少,并利用额定功率和额定电压的数值算出灯丝的电阻.你计算得出的电阻是多少?

再用万用表直接测出这只电灯泡灯丝的电阻,你测得的电阻又是多少?跟计算得出的数值是否一致?想想看,这是什么道理?




































\begin{appendix}
	\chapter{常用的热学量和电学量的国际单位制单位}

\begin{table}[htp]\centering
    \begin{tabular}{cc|cc|c}
        \hline
\multicolumn{2}{c|}{物理量} & \multicolumn{2}{c|}{单位} &量纲式\\
名称&符号 &中文符号&英文符号\\
\hline
热力学温度      &   $T$               &    开      &    K  &  $[\Theta]$   \\
热量      &   $Q$             &     焦     &    J  &  $[L^2MT^{-2}]$ \\
比热      &   $c$             &  焦/千克$\cdot $开        & J/(kg$\cdot$K)  & $[L^2T^{-2}\Theta^{-1}]$\\  
电流      &    $I$             &   安       &    A & $[I]$ \\
电量      &    $Q$            &    库      &    C & $[TI]$ \\
电场强度      &  $E$              &  伏/米$^2$        &   V/m$^2$  & $[LMT^{-3}I^{-1}]$ \\
电势(差)、电压      & $U(V)$              &    伏      &   V  & $[L^2MT^{-3}I^{-1}]$ \\
电动势      &  $\mathcal{E}$              &    伏      &    V & $[L^2MT^{-3}I^{-1}]$ \\
电容      &    $C$          &    法      &   F  & $[L^{-2}M^{-1}T^{4}I^{2}]$ \\
电阻      &    $R$        &     欧     &    $\Omega$       & $[L^{2}MT^{-3}I^{-2}]$ \\
电阻率      &    $\rho$          &   欧$\cdot$米       & $\Omega\cdot{\rm m}$   &  $[L^{3}MT^{-3}I^{-2}]$ \\
\hline
    \end{tabular}
\end{table}
\end{appendix}









\end{document}